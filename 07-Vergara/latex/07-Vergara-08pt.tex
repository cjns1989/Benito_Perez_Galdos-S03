\PassOptionsToPackage{unicode=true}{hyperref} % options for packages loaded elsewhere
\PassOptionsToPackage{hyphens}{url}
%
\documentclass[oneside,8pt,spanish,]{extbook} % cjns1989 - 27112019 - added the oneside option: so that the text jumps left & right when reading on a tablet/ereader
\usepackage{lmodern}
\usepackage{amssymb,amsmath}
\usepackage{ifxetex,ifluatex}
\usepackage{fixltx2e} % provides \textsubscript
\ifnum 0\ifxetex 1\fi\ifluatex 1\fi=0 % if pdftex
  \usepackage[T1]{fontenc}
  \usepackage[utf8]{inputenc}
  \usepackage{textcomp} % provides euro and other symbols
\else % if luatex or xelatex
  \usepackage{unicode-math}
  \defaultfontfeatures{Ligatures=TeX,Scale=MatchLowercase}
%   \setmainfont[]{EBGaramond-Regular}
    \setmainfont[Numbers={OldStyle,Proportional}]{EBGaramond-Regular}      % cjns1989 - 20191129 - old style numbers 
\fi
% use upquote if available, for straight quotes in verbatim environments
\IfFileExists{upquote.sty}{\usepackage{upquote}}{}
% use microtype if available
\IfFileExists{microtype.sty}{%
\usepackage[]{microtype}
\UseMicrotypeSet[protrusion]{basicmath} % disable protrusion for tt fonts
}{}
\usepackage{hyperref}
\hypersetup{
            pdftitle={VERGARA},
            pdfauthor={Benito Pérez Galdós},
            pdfborder={0 0 0},
            breaklinks=true}
\urlstyle{same}  % don't use monospace font for urls
\usepackage[papersize={4.80 in, 6.40  in},left=.5 in,right=.5 in]{geometry}
\setlength{\emergencystretch}{3em}  % prevent overfull lines
\providecommand{\tightlist}{%
  \setlength{\itemsep}{0pt}\setlength{\parskip}{0pt}}
\setcounter{secnumdepth}{0}

% set default figure placement to htbp
\makeatletter
\def\fps@figure{htbp}
\makeatother

\usepackage{ragged2e}
\usepackage{epigraph}
\renewcommand{\textflush}{flushepinormal}

\usepackage{indentfirst}

\usepackage{fancyhdr}
\pagestyle{fancy}
\fancyhf{}
\fancyhead[R]{\thepage}
\renewcommand{\headrulewidth}{0pt}
\usepackage{quoting}
\usepackage{ragged2e}

\newlength\mylen
\settowidth\mylen{……………}

\usepackage{stackengine}
\usepackage{graphicx}
\def\asterism{\par\vspace{1em}{\centering\scalebox{.9}{%
  \stackon[-0.6pt]{\bfseries*~*}{\bfseries*}}\par}\vspace{.8em}\par}

 \usepackage{titlesec}
 \titleformat{\chapter}[display]
  {\normalfont\bfseries\filcenter}{}{0pt}{\Large}
 \titleformat{\section}[display]
  {\normalfont\bfseries\filcenter}{}{0pt}{\Large}
 \titleformat{\subsection}[display]
  {\normalfont\bfseries\filcenter}{}{0pt}{\Large}

\setcounter{secnumdepth}{1}
\ifnum 0\ifxetex 1\fi\ifluatex 1\fi=0 % if pdftex
  \usepackage[shorthands=off,main=spanish]{babel}
\else
  % load polyglossia as late as possible as it *could* call bidi if RTL lang (e.g. Hebrew or Arabic)
%   \usepackage{polyglossia}
%   \setmainlanguage[]{spanish}
%   \usepackage[french]{babel} % cjns1989 - 1.43 version of polyglossia on this system does not allow disabling the autospacing feature
\fi

\title{VERGARA}
\author{Benito Pérez Galdós}
\date{}

\begin{document}
\maketitle

\hypertarget{i}{%
\chapter{I}\label{i}}

\large
\begin{center}
\textbf{De D. Pedro Hillo a los Sres. de Maltrana.}              \\
\end{center}
\normalsize

\bigskip
\begin{flushright}\small \textit{Miranda de Ebro, Octubre de 1837.}\normalsize\end{flushright}   
\bigskip

Señora y señor de todo mi respeto: Con felicidad, mas no sin estorbos,
por causa del sinnúmero de tropas que nos han acompañado en todo el
camino, marchando en la propia dirección, llegamos a esta noble villa
realenga ayer por la mañana. Soldados a pie y a caballo descendían por
las cañadas, o aparecían por atajos y vericuetos, y engrosando la
multitud guerrera en el llano por donde el Ebro corre, nos vimos al fin
envueltos en el torbellino de un grande ejército, o al menos a mí me lo
parecía, pues nunca vi tanta tropa reunida. Generales y convoyes pasaban
sin cesar a nuestro lado tomándonos la delantera, y ya próximos a
Miranda vimos al propio caudillo, Conde de Luchana, seguido de brillante
escolta, y a otros afamados jefes y oficiales, que al punto conocieron a
Fernando y le saludaron gozosos. Nuestra entrada y acomodamiento en la
antigua \emph{Deóbriga} fue, como pueden ustedes suponer, asaz
dificultosa. Éramos un brazo que se empeñaba en introducirse en una
manga ya ocupada con otro brazo robusto. En ningún albergue público ni
privado de los que en toda población existen para personas y caballerías
hallamos hueco, ni aun pidiéndolo del tamaño preciso para alfileres; y
ya nos resignábamos a la pobreza de acampar en mitad del camino, como
mendigos o gitanos, cuando nos deparó Dios a un sujeto, que no sé si
llamar enemigo o amigo, aunque en tal ocasión y circunstancias bien
merece este último nombre, el cual, con demostraciones oficiosas y todo
lo urbanas que su rudeza le permitía, nos colocó bajo techo, entre cabos
y sargentos de artillería montada, con los correspondientes arreos,
armones, sacos, cajas y regular número de cuadrúpedos.

Era el tal D. Víctor Ibraim capellán castrense, antaño en la Guardia
Real, hogaño en un regimiento de artillería, y tengo que calificarle,
con perdón, como uno de los más soberbios animales que han comido pan en
el mundo, si bien yo creo que a este sujeto todo lo que come le sabe a
cebada y paja, y como tal alimento lo saborea. Cuando yo tenga el gusto
de volver a esa noble casa contaré a ustedes motivos de la santa inquina
que profeso a mi colega, el marcial presbítero, andaluz por más señas, y
tengo por seguro que se han de reír de tan donosa historia. Por hoy
conste que perdono al señor Ibraim sus agravios de otros días, y
reconozco que nos ha dado a Fernando y a mí una prueba de cordialidad,
procurándonos este alojamiento, que si detestable y con enfadosas
apreturas, nos permite comer algo caliente y guardar nuestras personas
al abrigo de la intemperie. Nuestras bestias campesinas han entrado en
gran confianza con los guerreros caballos del regimiento; Sabas y Rufino
hacen buenas migas con la tropa, y nosotros anudamos cada hora nuevas y
más alegres amistades con oficiales muy simpáticos y con capellanes
menos brutos que el desdichado Ibraim. No nos va mal, y Fernando ha
tenido el gusto de encontrar amigos queridísimos entre estos campeones
de Isabel II: D. Juan Zabala, D. Antonio Ros de Olano y otros cuyos
nombres y títulos se me escapan de la memoria.

Antes que se me olvide, señora y caballero: recibí de manos del propio,
en Leciñana del Camino, el mensaje reservado, y puedo asegurarles que el
pobre chico lo hizo con la discreción que le fue que prescrita. No se
enteró Fernando, a quien di la carta de su mamá, dejándole que se
entregara con avidez al gozo de leerla; y en cuanto yo tuve coyuntura de
soledad leí la de ustedes, que me ha causado sorpresa, ira y recelo.
¿Pero qué pretende ese badulaque? ¡Habrá insolencia igual! ¡Atreverse a
medir su barbarie con la finura de Fernando, y brindar a este una
concordia que para nada le hace falta, o amenazarle con una hostilidad
que no puede infundirle ningún temor! En fin, sea lo que quiera, y venga
con estas o las otras intenciones, yo estaré con muchísimo cuidado, a
fin de cortarle el paso si a nuestro caballero quiere aproximarse, o
inutilizar su malicia y audacia, aunque para ello tenga que valerme de
nuestras relaciones en el Cuartel General\ldots{} ¡y qué relaciones,
señora y señor míos!

Ya comprenderás que teniendo Fernando tantos amigos en la liberal
milicia, y gozando como nadie del don de simpatía, en pocas horas se ha
visto obsequiado y traído de una parte a otra. De boca en boca llegó su
nombre a oídos del gran Espartero, el cual anoche le mandó llamar por
uno de sus ayudantes. Allá se fue; departieron un ratito, casi todo
consagrado a comentar el increíble viaje de D. Beltrán al campo del
Maestrazgo, y su prisión y nunca vistas desventuras en aquella tierra
facciosa. Hoy repitió la visita, regresando al poco rato con la embajada
de que fuese yo también a la presencia del de Luchana, pues este deseaba
verme, y tenía que hablarme, ¡ay!, de mi incumbencia
eclesiástico-castrense. Creí que eran bromas del señorito, o que con mi
timidez y cortedad quería divertirse, pues ya sabe él y saben todos que
no soy hombre para codearme con señorones y celebridades de tal fuste;
pero tanto insistió mi discípulo, que allá nos fuimos, después de dar
restregones a mi balandrán para limpiarlo de barros y otras materias, y
tuve la satisfacción de ver de cerca al gran héroe y de platicar mano a
mano con él durante unos diez minutos, que me parecieron diez horas; tan
sofocado y descompuesto estaba yo por el honor inmenso de aquella
entrevista. Díjome que había separado del servicio a tres capellanes,
por sospechas de espionaje, y que celebraba y agradecía que el Vicariato
pusiese mano en purificar el personal, desechando a todos los individuos
del \emph{cuerpo} que por sus antecedentes o su mala conducta no eran
dignos de seguir bajo las banderas gloriosas. Contestele con trémula voz
manifestando un asentimiento incondicional a todo lo que de sus
autorizados labios salía\ldots{} añadí la oferta de mi inutilidad para
mejorar el importantísimo servicio castrense\ldots{} indiqué, divagando,
que en el \emph{cuerpo} hay dignísimos sacerdotes; mas otros, aunque en
el servicio se muestran puntuales, fuera de él, y en los ratos de ocio,
emulan con los oficiales en la desvergüenza de palabras y en la
liviandad de la conducta\ldots{} que se intentaba purgar el
\emph{cuerpo} y limpiarlo de todo maleficio para que respondiese a los
fines del ministerio militar y religioso\ldots{} \emph{etcétera}.
Serenándome al fin, solté cuatro generalidades pomposas, para disimular
mi indiferencia de todo lo que al dichoso \emph{cuerpo} se
refiere\ldots{}

Ya ven los señores que mi conferencia con el insigne caudillo fue
luminosa por todo extremo, inspirada en el \emph{bien público} y en
\emph{el espíritu del siglo}. No me asombrará que de ella den cuenta los
papeles, pues mis palabras fueron gratas al General, que las apoyó con
cabezadas enérgicas. Espero que el día del juicio dará óptimos frutos la
inspección que el Vicariato ha encomendado a mi ardoroso celo castrense,
a mi\ldots{}

Obligado me veo a interrumpir esta, porque del Estado Mayor me llaman
para un asunto muy grave\ldots{} No asustarse, señora y caballero, pues
no es cosa nuestra, ni hay en ello relación derecha o torcida con el
Sr.~D. Fernando. Sólo a un servidor de ustedes afectan las tristezas del
desagradable negocio que me encomienda el Estado Mayor, y en cuanto me
desocupe de esta obligación dolorosa tendrá el gusto de referirla
puntualmente su obligado servidor, amigo y capellán---\emph{Pedro
Hillo}.

\hypertarget{ii}{%
\chapter{II}\label{ii}}

\large
\begin{center}
\textbf{Del mismo a los mismos. Terminada por                        \\
        D. Fernando.}                                                \\
\end{center}
\normalsize

\bigskip
\begin{flushright}\small \textit{Miranda 30 de Octubre.}\normalsize\end{flushright}   
\bigskip

Señores míos muy amados: Si no lo sabían, esta carta les informará de
que soy el hombre más pusilánime y para poco que ha echado Dios al
mundo. ¡Ay de mí! Jamás pensé verme en trance tan aflictivo como el que
hoy ha llenado mi espíritu de turbación y congoja. Ni en pesadilla sentí
jamás angustias como estas: tales fueron, que durante largo rato las
tuve por hechura de mi mente febril. Figúrense mi terror cuando el
brigadier Sr.~Aristizábal me comunica que tengo que auxiliar a no sé
cuántos reos de muerte, por no haber en este ejército suficiente
personal de capellanes para tan triste servicio. Yo que tal oigo, échome
a temblar; los cabellos se me ponen de punta y no me queda gota de
sangre en el mísero cuerpo. Nunca había visto yo la muerte violenta más
que en la Plaza de Toros, donde, por tratarse de animales, rarísima vez
de personas, nuestra emoción no pasa del grado inferior, y va compensada
del entusiasmo y alegría que a los aficionados a este arte nos comunica
el calor del fiero espectáculo. Pero ¡ay, Jesús mío!, en ningún tiempo
vi matar a mis semejantes, y menos con la fría serenidad aterradora de
los actos de justicia. No, no: yo no sirvo para eso, y abomino del
ministerio castrense, que somete al mayor de los suplicios mi alma
generosa y cristiana. «Pero ¿qué reos son esos a quienes tengo yo que
auxiliar?---me decía yo, vagando como un demente de una parte a otra con
las manos en la cabeza.---¿Qué delito han cometido para que se les
sacrifique inhumanamente? Antes que conducirles al matadero, iré a ver a
mi amigo el de Luchana, y de rodillas le pediré la vida de esos
infieles, probablemente condenados por alguna falta de disciplina, la
cual, digan lo que quieran los espadones, no es ley moral ni cosa que lo
valga.»

Y cuando esto decía, me vi cogido del brazo por Fernando, el cual me
hizo notar que toda la tropa se ponía en movimiento hacia el camino de
Vitoria, con vivo estrépito de cajas y clarines. Hermoso era el
espectáculo según él, a mis ojos tristísimo, porque la formación, y los
toques militares, y el paso guerrero, y la vista de los gallardos jefes
a caballo, y todo aquel tumulto de vocerío y colorines, traía con más
vigor a mi mente la idea de la cruel Ordenanza. Llevome consigo Fernando
a los alcances de la tropa, y por el camino me dijo que se preparaba un
acto de reparación con toda la pompa y rimbombancia que la justicia
militar exige. Espartero quería castigar con mano severa los actos
sediciosos de Miranda, Hernani, Vitoria y Pamplona, y a los infames
asesinos de Ceballos Escalera y D. Liborio González, de Sarsfield y
Mendívil, pues si no se contenía la indisciplina, el ejército se
convertiría en horda salvaje; el arma creada por la Nación para su
gloria y defensa sería una herramienta de ignominia\ldots{} y entre
facciosos y jacobinos harían mangas y capirotes de la pobre España,
resultando al fin que las naciones extranjeras vendrían a ponernos
grilletes y bozales. Declaro que Fernando me convencía y no me
convencía; no sé cómo expresarlo. Sus razonamientos eran juiciosos; pero
a mí no me entraba en la cabeza que por achaque de marcial honrilla
tuviese yo que añadir mi autoridad religiosa al acto fúnebre de castigar
a los que por matar \emph{sin reglas} deshonraron su oficio de matar.
Esta idea me volvía loco. En el principio se dijo: «no matarás.» Cristo
Nuestro Señor nos ordenó perdonar las ofensas y hacer bien a nuestros
enemigos. Al que me compagine esto con las guerras y con la Ordenanza
militar, le regalo mi jerarquía vicarial castrense, con el uso de
collarín y botones morados, y de añadidura mi encomienda de Isabel la
Católica, última gracia que merecí de los superiores, sin que sepa nunca
por qué.

De nada me valía mi santa indignación, y allá me fui casi arrastrado por
Fernando, que presenciar quería la hecatombe. Y por Cristo que D.
Baldomero había dispuesto con arte la escena, formando toda su hueste en
un grandísimo cuadro. Detrás de la infantería del Provincial de Segovia,
que era el cuerpo delincuente, vi masas de caballería formidable; a esta
otra parte, la artillería, cargada con metralla, según me dijeron;
enfrente, los \emph{Guías del General}, la tropa de más confianza; en
medio, recorriendo las filas, el de Luchana, en un fogoso caballo que
pintado parecía. El gallardo mover de sus remos, la arrogancia de su
enarcado cuello, como su espumante boca, mostraban el hervor de su
sangre guerrera. Con militar grito, que hacía poner los pelos de punta,
Espartero mandó armar bayoneta. El chirrido que a esta operación
acompaña recorrió las filas de un cabo a otro, produciendo en mi pobre
piel el mismo efecto que si todas las puntas de aquellos hierros
quisieran acariciarla. Siguió un silencio angustioso, en el cual se
precipitó de improviso, como los truenos en el seno de la noche, el
ruido de todos los tambores redoblando juntos. Cuando callaron, el
silencio era más imponente. En mis oídos zumbaba la sangre de mi
cerebro, repitiendo la palpitación de los pulsos de todos los hombres
que estaban allí. Mirando a las caras mas próximas, en ellas veía
reflejada mi pavura.

Mandó Espartero a su escolta y ayudantes que se alejasen, y se quedó
solo en medio del cuadro\ldots{} Accionando con la espada, rompió en
voces que parecían truenos\ldots{} Nunca, ni en el púlpito, ni en los
clubs, ni en las Cortes, oí una voz que más hondo penetrara en el oído
de los que escuchan. Apliqué mi oreja, haciendo con la mano pabellón, y
sin entender bien los conceptos, ello es que me conmovían, no sé por
qué. El tono elocuente me llegaba al alma, y si el sentido se quedaba en
el aire, yo adivinaba en él no sé qué grande, sublime lección. Al
principio apenas cogía palabras sueltas; luego, como si el silencio, a
cada instante más profundo, destacase las ideas, llegué a pescar trozos
oratorios. Oí este: \emph{Sangre preciosa tantas veces prodigada en los
campos de batalla}\ldots{} El orador hizo luego una interrogación, a la
que contestó todo el ejército con un sí, que me sonaba como el silbido
de un huracán.

Después oí algo más, esta frase: \emph{Era la noche\ldots{} un fúnebre
ensueño ocupaba mis sentidos\ldots{} La feroz discordia que peina
serpientes por cabellos}\ldots{} Por Dios que fue de mi agrado la
figura; mas no comprendí a qué venía. Pareciome después que el General
se lanzaba a la \emph{idolopeya}\ldots{} describía la aparición de un
espectro, que no podía ser otro que el de Ceballos Escalera\ldots{}
\emph{Sombra ensangrentada}, \emph{despeluznada}, \emph{yerto el rostro
y despedazado su cuerpo}\ldots{} Pensé yo que en el estilo militar
podían perdonarse tantas asonancias\ldots{} La sombra habla al orador, y
le dice: Mira cómo me dejaste, mira cómo me ves. Repara mi agravio,
salva a la patria\ldots{} En aquel momento, la voz de Espartero no
parecía voz humana. Sin poder fijarme en la retórica, yo lloraba. Quería
ser crítico, y era un pobre ignorante, fascinado por la ocasión, por el
aparato escénico, y, sobre todo, por el acento, por el arranque, por el
gesto del orador. Vuelto hacia el paraje donde yo me agazapaba tras de
la tropa para oírle, señaló con la espada a la villa, y pude oír
claramente estas expresiones: \emph{Allí\ldots{} allí unos cuantos
asesinos, pagados por los agentes de D. Carlos, clavaron el alevoso
puñal en el corazón de un hijo predilecto de la patria\ldots{} Allí el
trono de la inocente Isabel se conmovió al faltarle una de sus más
fuertes columnas\ldots{} allí os arrebataron un amigo, digno de serlo
vuestro, porque lo era mío; allí el príncipe rebelde consiguió una
brillante victoria con la muerte de un poderoso enemigo, y allí, por
último, los manes humeantes de la ilustre víctima claman venganza.}..
Vuelto hacia el otro lado, soltó un hermoso epifonema, después una
vituperación, inmediatamente una histerología o locución prepóstera, y
luego, señalando al Provincial de Segovia, en cuyas filas se ocultaban
los asesinos, gritó: \emph{Que les delaten inmediatamente sus
compañeros, o el regimiento será diezmado en el acto}. La voz y la
espada eran rayos\ldots{} Me retiré con las manos en la cabeza. No podía
oír más. ¡Horrible susto!\ldots{} creí que ya estaban contándolos para
matar uno de cada diez.

Después supe que, aterrados y confusos, algunos delataron a los
culpables. Eran éstos treinta y tantos\ldots{} Yo corrí; pero con mala
suerte, porque me cogió Fernando, señalándome el camino que había de
seguir, el cual a una venta próxima conducía. «¿Y qué tengo yo que hacer
en la venta?» le dije\ldots{} No pude escabullirme, y allá me llevaron,
teniendo la desdicha de encontrar por el camino al maldito Ibraim, que
me daba prisa, como si fuéramos a una fiesta, o a apagar un fuego. La
tropa se puso en marcha\ldots{} Vi a los delincuentes escoltados por los
Guías\ldots{} Metiéronles en la venta\ldots{} Un consejo de guerra, que
actuar y sentenciar debía sumariamente, les aguardaba\ldots{} Cinco
capellanes éramos; pocos a mi entender para tantas víctimas. Luego supe
que los condenados a morir, o sea los más criminales, eran sólo diez.
Los demás irían a presidio. ¡Diez! También me parecía mucho.

No tuvimos que esperar largo tiempo los ministros espirituales, porque
los de la ley humana despacharon en un periquete, dándonos el ejemplo de
la brevedad, tan recomendada en cosas militares. Ibraim me pareció
satisfecho de contribuir con su capacidad eclesiástico-castrense a la
purificación del ejército. Encontraba muy natural la pena, y se condolía
de que hubiera tardado tanto su aplicación. Mejores entrañas revelaban
los otros tres compañeros, y uno de ellos allá se iba conmigo en
aflicción y pusilanimidad. Al entrar y ver el tristísimo grupo de los
diez pobres condenados, no pude contener mis lágrimas, y mentalmente les
dije: «Pero, hijos míos, ¿a qué habéis hecho esa gran tontería de matar
a vuestro General? ¿No sabéis que esas locuras se pagan con la
vida\ldots? ¡Vaya, que si vuestras madres os vieran en este
trance\ldots! ¿Por qué no os acordasteis de ellas antes de hacer fuego
contra el superior\ldots? Sin que me lo digáis, sé yo que todo fue obra
de un arrebato, una funesta obcecación. No fuisteis a él, no, con
intento de matarle; pero la enredó el demonio, y os perdisteis en un
momento. Sin duda habíais bebido más de la cuenta\ldots{} Ya os veo
arrepentidos; lo estabais antes de ser condenados, ¿verdad? No sois
vosotros tan malos como el General os cree. ¡Vaya, que os ha dicho unas
cosas\ldots! Perdonadle también, y preparaos a gozar de Dios, que os
espera\ldots» Casi las mismas expresiones empleé después con los dos que
me tocaron, guapos chicos, ¡ay dolor! Y que estaban de veras
arrepentidos. Mataron como por juego, sin mala idea. La guerra les
enseña a segar vidas, a hendir con la bayoneta vientres y espaldas, a
disparar el fusil contra cráneos y pechos, y acaban por apreciar en poco
las vidas de nuestros semejantes. Cierto que su General era su General.
¡Pues estaría bueno que las honrosas armas empuñadas para defender a la
Reina, contra un corifeo de la misma augusta señora se volviesen! Hay
que matar con reglas, ya que el matar dicen que es necesario. ¡Maldita
guerra, escuela de pecados, salvoconducto de los impíos, precipicio a
que ruedan las almas, simulacro del infierno!

El segundo que confesé era un chiquillo, que para interesarme y
conmoverme más demostraba un valor sereno, enteramente a la romana.
Creía merecer su castigo, lo aceptaba con estoica fiereza y una torva
conformidad con tan cruel justicia. La confesión fue breve y me llenó el
alma de angustia. Con la ternura más viva le prometí el Cielo, le pinté
en breves rasgos las miserias de este mundo, ponderé las delicias de la
bienaventuranza con que galardona Dios los pecadores que llegan a Él
purificados por el martirio, limpia la conciencia de todo mal\ldots{} El
pobrecillo me creía\ldots{} Vi en su rostro un no sé qué de confianza y
placidez\ldots{} Díjome que era vizcaíno, y que por intimar demasiado
con camaradas de mala conducta se veía en aquel trance; que si era
cierto que podía entrar en la Gloria, moriría pensando que Dios le
franqueaba las puertas de ella, y pediría misericordia con toda su alma.
Repetile mis consuelos, las seguridades de que pasaba a un mundo de
perdón y felicidad. Le di un abrazo apretadísimo\ldots{} Habría
prolongado mis exhortaciones, mis cariños; pero no podía ser: ya todos
concluían; las ejecuciones debían seguir al acto religioso con la
prontitud que es norma del procedimiento militar. Breve es la misa,
breve la confesión, todo rápido y a paso de carga, para tener contento
al tiempo, el gran amigo de Marte.

Sacáronles a unas eras cercanas, y les colocaron de rodillas junto a una
tapia, nosotros junto a ellos, hasta que con una seña nos mandaron
retirar. Ibraim daba fuertes voces a los dos que asistía. Yo, a los
míos, no sabía ya qué decirles. Creyérase que me fusilaban también a mí,
según estaba de macilento y lívido. Por fin\ldots{} Yo no había
presenciado nunca cosa tan horrible. Sentí un pánico superior a toda mi
entereza de varón y de sacerdote; quise huir; tropecé\ldots{} recogiome
en sus forzudos brazos el bruto de Ibraim. Por un instante perdí el
conocimiento, y al abrir los ojos vi los diez cuerpos en el suelo entre
charcos de sangre. Sonaban los tambores como mil truenos.

Vi al capitán y a dos capellanes que se inclinaban sobre el fúnebre
montón, reconociendo entre las víctimas a una que se incorporaba,
pataleando. Era el mío, que había quedado vivo, sin ninguna herida
mortal. ¡Jesús, qué susto, qué congoja! Alguien habló de rematarle.
Sintiendo como si un rayo me traspasara, me arrodillé ante el capitán de
Guías y le dije: «Si a este, que se ha salvado milagrosamente, no se le
perdona la vida, que me fusilen también a mí. Así se lo diré a mi amigo
el General en jefe.» En tanto, el pobre chico se ponía en pie,
ensangrentado, más por la sangre de los demás que por la suya. Le cogí
en mis brazos, gritando como un loco: «¡Perdón, perdón!» Los oficiales,
para gloria suya lo digo, se pusieron de mi parte, y el Capitán corrió a
ver a Espartero. Minutos después venía el indulto\ldots{} Dispénsenme
mis buenos amigos: al llegar a este punto me siento tan mal por causa de
la extenuación, de las terribles angustias de este crítico día, que me
veo precisado a suspender la carta. Mi temblor y debilidad exigen que me
recoja. La pluma dejo a Fernando, que rabiando está por quitármela, no
sólo por su afán de que yo descanse, sino por el gustazo de escribir a
ustedes. Él lo hará con menos turbación que este su atribulado amigo y
capellán---\emph{Pedro Hillo}.

\emph{Termina D. Fernando}.---¡Qué pena, amigos de mi alma, ver a
nuestro pobre clérigo en funciones tan impropias de su alma candorosa,
de su condición pacífica y dulce! El pobre ha sufrido lo indecible,
sacando fuerzas de su flaqueza, y alientos de su cristiana ternura. He
quitado de su mano la pluma, pues su estado nervioso y febril me
inspiraba inquietud, y obligándole a tomar algún alimento, le mando a la
cama, entendiendo por esto un abrigado espacio entre albardones, mullido
con buenas mantas.

Leída su relación, la encuentro tan ajustada a la verdad, que en ella no
tengo que añadir ni una tilde. Contará la Historia el terrible
escarmiento tal y como nuestro capellán lo ha referido, con la añadidura
del milagro del pobre chico ileso, que más bien parecía resucitado. Le
corresponde cadena perpetua; pero su juventud puede confiar en los
indultos que traiga la política, o en los sucesivos actos de regia
clemencia. Se llama Buenaventura Iturbide, y es natural de Bilbao. Le
han metido en la cárcel, donde apenas pueden revolverse los infelices
presos por espionaje, deserción y otros delitos. Mis amigos y yo les
hemos socorrido para que no perezcan de hambre. Las tristezas del
desgobierno de la nación, el espectáculo de los infinitos males y
desórdenes que ocasiona la guerra, abruman nuestro espíritu,
incitándonos a buscar en un obscuro retiro el olvido y el aislamiento.
Deseo con toda mi alma salir de este pueblo, reponerme del fúnebre
espectáculo de la justicia militar. Terminada esta carta, escribiré a mi
madre con la extensión que ella desea y que es para mí el más grato
empleo del tiempo; le contaré todo, le daré razón detallada de mis
pensamientos más íntimos, y cumplido este deber, buscaré algún descanso
entre albardas, para continuar nuestro viaje mañana tempranito.

En mi cerebro traje y conservo con amor vuestra casa y vuestras
personas. Vivís todos en mí: la casa con su placidez, con su blancura;
vosotros con la bondad y el cariño que en mí habéis puesto, y a que
correspondo queriéndoos como a hermanos. ¿Qué me dicen mis discípulas,
qué mis queridos chicuelos? Me considero estampado en su memoria, como
ellos están en la mía, donde les veo y les oigo. Nos hemos quedado muy
tristes con esta ausencia, ¿verdad? Yo les juro que de buena gana
picaría espuelas hacia Villarcayo, si no tuviera el compromiso de
acompañar a mi capellán hasta Vitoria. No se conoce la intensidad de los
afectos, y la dureza de sus ligaduras, hasta que nos damos un tirón como
este que me ha separado de vosotros. En fin, no digáis que me pongo
romántico y sentimental. Más sencillo es deciros llanamente que os
quiero con el alma. No os he perdido, no, porque deje de veros. Feliz
como ninguno será el día en que os recobre vuestro
hermano---\emph{Fernando}.

\hypertarget{iii}{%
\chapter{III}\label{iii}}

\large
\begin{center}
\textbf{De Pepe Iturbide a su padre, Casiano Iturbide,               \\
        residente en Bilbao.}                                        \\
\end{center}
\normalsize

\bigskip
\begin{flushright}\small \textit{Miranda de Ebro 1.º de Noviembre.}\normalsize\end{flushright}   
\bigskip

Señor padre: Sabrá que mi querido hermano Ventura es salvo, no por
misericordia del superior, sino por milagro que hizo el Altísimo, no
permitiendo que le dieran muerte las balas disparadas sobre él; con lo
que queda dicho que le fusilaron, sin que pudiéramos mis compañeros y yo
hacer nada para librarle de la pena, por lo que le diré ahora o después;
que tantas cosas desgraciadas nos ocurren, juntamente con la felicidad
de ver vivo a Ventura, que no sé por cuál empezar. Trajéronle a la
cárcel, donde le están curando las heridas, que no son graves; su
condena, por conmutación, es de presidio para toda la vida, y aquí le
tenemos, con lo que dicho queda que en esta malditísima cárcel moramos
todos, el que suscribe, y Zoilo Arratia, y también el amigo Pertusa, a
quien damos la encomienda de escribir por todos, pues ya sabe usted lo
torpes que somos Zoilo y yo para la escritura corrida, y lo bien que
menea la pluma D. Eustaquio.

La parada que por cosas de Zoilo tuvimos que hacer en Villarcayo nos
retrasó, y llegamos aquí más tarde de lo que creíamos. Era mi propósito
entregar al General Van-Halen la carta del Sr.~de Gaminde, y empezar mis
diligencias al objeto de sacar a Ventura del Provincial de Segovia
(señalado por indisciplina para un severo castigo) y pasarle a otro
cuerpo. Pero la mala suerte o nuestra tardanza, ¡ay de mí!, quisieron
que aquellos cálculos tan juiciosos salieran fallidos, pues apenas
entramos en el pueblo, y cuando nos hallábamos reparando el cuerpo con
unas sopas, fuimos detenidos y apaleados, se nos registró de la
coronilla a los calcañales, quitándonos cuanto llevábamos, dinero,
armas, cartas y papeles, y para remate de tanta picardía nos encerraron
a los tres en el más pestilente calabozo de esta cárcel, donde pedimos a
Dios y a la Virgen Santísima que los gruesos muros se vuelvan de cartón
para escaparnos, o que a traernos la preciosa libertad venga una mano
bienhechora.

Pero han pasado dos días, y no viene a salvarnos mano de hombre ni
providencia de Dios, y estamos ya en el colmo de la desesperación,
maldiciendo al cielo y a la tierra. Zoilo es el más inconsolable: se da
golpes en la cabeza, se arrastra por el suelo, echa de su boca horrores,
muerde los barrotes de la reja, como un ratón cogido entre alambres.
Pertusa es el que lleva con más calma nuestra prisión, pues su acendrada
fe le da confianza en Dios misericordioso y en el triunfo de la
inocencia. Mientras Zoilo blasfema y se da golpes, Eustaquio reza; su
religiosidad se me va pegando, aunque no tanto como yo quisiera. Yo
lloro; pienso en mi casa y mi familia, y aguardo el instante de la
libertad preciosa que nos han robado estos cafres. Desde el calabozo,
diré más bien sepulcro, oímos ayer el ruido de la tropa que salió a
formar cuadro hacia la parte del camino de Vitoria. Al estruendo de los
tiros, temblamos de pavor, redoblando cada cual sus demostraciones: yo
mis llantos, Zoilo sus blasfemias, Eustaquio sus Padrenuestros y
Avemarías. A poco de esto vimos por la reja que traían a Ventura vivo,
aunque manchadito de sangre; me puse a chillar con fuertes alaridos, y
los carceleros se apiadaron de mí, permitiéndole entrar en nuestra
mazmorra, para que yo pudiera abrazarle y él contarnos el caso feliz de
su fusilamiento milagroso. Dice Eustaquio que ya en esto se ve
claramente la mano de Dios, la cual no ha de tardar en venir hacia
nosotros, pobrecitos inocentes perseguidos de infame justicia. Luego se
llevaron a mi hermano a la enfermería, para curarle sus leves heridas
con salmuera y vinagre, y no he vuelto a verle, aunque sé por el
calabocero que está bien, comiendo como un descosido y deseando que le
destinen a donde ha de cumplir su condena.

Por la declaración que hoy nos han tomado caigo en la cuenta de que nos
acusan de espías del faccioso, y a mí, por añadidura, de desertor, lo
que si es verdad por un lado, por otro no lo es. Cierto que me escapé
del Provincial de Toro; pero yo y otros doce muchachos bilbaínos que
fuimos agregados al batallón, no servíamos como tales soldados de la
Reina, sino como milicianos auxiliares, y no teníamos obligación de
estar en filas más que dentro del terreno de Vizcaya, conforme a fuero,
y así consta en papeles que firmaron D. José Arana y el General San
Miguel\ldots{} De los trece, cinco abandonamos el batallón en
Guardamino, después de batirnos heroicamente, aunque me esté mal el
decirlo. Bilbaínos somos, y pertenecemos a la sacra Milicia Urbana, que
obligada está, ¡vive Dios!, a defendernos contra esta picardía de meter
en la cárcel a tres hombres de bien, que han derramado sangre preciosa
por la patria, bajo estas o las otras banderas.

Haga por librarnos de tan horrendo suplicio, amado padre, poniendo en
conocimiento del Sr.~Arana, del Sr.~Gaminde y de todos los pudientes de
esa, la desgracia que nos aflige, para que manifiesten al señor
Van-Halen y al invicto General Espartero nuestra honradez y
circunstancias.

Cedo la vez a Zoilo, que ahora sale con la tecla de no querer escribir,
porque su rabia le corta el dictado y no sabe poner sus ideas en orden,
como es conveniente en todo buen discurso. Reniega del género humano, y
hasta de las potencias celestiales, llegando a la gran abominación de
decir de Dios cosas muy feas por haber consentido este vituperio. Tanto
yo como D. Eustaquio, con su bendita mansedumbre, tratamos de traerle a
conformidad, y le hablamos de su cara familia para despertar en él
sentimientos que no sean la ira loca. Pero no cede a nuestras razones
blandas, ¡pobre amigo!, y me temo que su furor de independencia y el ver
su voluntad entre hierros le lleven a convertirse de hombre sesudo en
bestia feroz. ¡Dios tenga piedad de él y de nosotros!, ¡ay! Por su
cuenta notifico que no hemos encontrado a \emph{Churi}, y que en ningún
punto de los recorridos nos han dado razón del desdichado sordo. Dice D.
Eustaquio, y por su cuenta lo pone, que cuando le conoció en el Bocal,
iba pegadito a las faldas de una que llaman \emph{Saloma la baturra}, de
quien estaba locamente enamorado, en tal extremo de pasión, que era un
puro volcán que reventaba con gestos furiosos y expresiones desatinadas.
Le tuvo entonces por hombre perdido, abocado a un fin desastroso, el
cual teme sea ya un hecho, o, lo que es lo mismo, que ya no se encuentre
el pobre \emph{Churi} en el mundo de los vivos. Con todo, si nos
devuelven la libertad y Zoilo recobra su ser, indagaremos hasta
encontrarle, empezando por tomar lenguas de esa señora baturra, que
pertenece a la cuadrilla del llamado \emph{Uva}, cantinero.

Concluyo, mi señor padre, pidiendo a usted la bendición, y mandando los
cariños más acendrados a mi amadísima hermana Mercedes, y a mis
hermanitos Deogracias y Lucas, a quien repartirá usted cuantos besos
sean menester para contentarles a todos, así como buenas memorias a la
Encarnación y a Camilo, y a los demás de casa. ¡Cuánto han de llorar,
señor padre, usted el primero, cuando sepan la infausta prisión mía!
Señor, desde este infierno lanza un ¡ay!, dolorido en demanda de
socorro, y con las alas del corazón hacia Bilbao gimiendo vuela, su
cautivo amante hijo---\emph{José Iturbide}.

\emph{P. D}.---Como hasta hoy martes, día de los Fieles Difuntos, no
puede salir la carta, le añadimos este parrafito para que sepan que
seguimos en la propia miseria y desesperación, ignorantes de cuál será
nuestro fin. A mi hermano le oímos cantar anoche en el calabozo donde
está con sus compañeros, condenados a pasarse la vida en Ceuta. ¿Que
harán con nosotros? Espartero se ha ido; Van-Halen con él, y estos tres
míseros mortales sepultados aquí, esperando que la caridad y la justicia
nos abran la puerta. ¿Por dónde andan estas señoras\ldots? ¿Y entre los
tantísimos Santos de ayer, ¡solemne fiesta!, no hay uno, uno siquiera
que nos salve? ¡Oh injuria del Cielo, oh negación de la
Omnipotencia!\ldots{} Señor, estamos locos. D. Eustaquio le escribe al
Obispo, a la Reina Gobernadora, a D. Pío Pita Pizarro, y creo que
también al Papa. Me ha dicho que esta mañana mientras yo dormía, Zoilo
dictó y firmó una carta para el caballero de Villarcayo. Ha trocado el
furor por la risa, una risa enferma que da escalofríos. A sus carcajadas
acompañan temblores de todo el cuerpo, y cuando D. Eustaquio le habla de
Dios, ríe mordiéndose las manos. ¡Señor, Señor, piedad de estos
pobres!\ldots{}

\hypertarget{iv}{%
\chapter{IV}\label{iv}}

\large
\begin{center}
\textbf{De D. Pedro Hillo a los Sres. de Maltrana.}                  \\
\end{center}
\normalsize

\bigskip
\begin{flushright}\small \textit{La Puebla de Arganzón, Noviembre.}\normalsize\end{flushright}   
\bigskip

Aprovecho, mis caros amigos, un corto descanso en esta villa para darles
referencia de nuestra feliz salida de Miranda, ambos con triste
impresión de la tragedia que muy a pesar nuestro presenciamos. Si
Fernando goza de perfecta salud, no puedo decir lo propio de su
acompañante, el cual, por el camino, ha sentido que le rondan achaques
antiguos. Hállase el tal, es decir, yo, un tanto febril, y no veo las
santas horas de llegar a Vitoria para descansar a mis anchas. Creo que
el susto de Miranda, que considero el más terrorífico de mi vida, me ha
revuelto toda la naturaleza, sacando de los últimos fondos de esta males
viejos, que yo creí dormidos o arrumbados para siempre. No se asusten,
porque ello no será nada, y con reponerme de aquel terror, y con
alimentarme y coger un largo sueño, pienso que he de tornar a mi
habitual temple.

El tal Zoilo Arratia y sus dos compañeros entraron en Miranda, según mis
noticias, el mismo día que nosotros, habiendo hallado alojamiento con
rara prontitud, aunque la vivienda que se les dispuso no fuera muy de su
agrado. A poco de llegar, se abrieron para los tres las puertas de la
cárcel, donde gimen por los graves delitos de deserción y espionaje. De
esto se les acusa; falta que sea verdad su delincuencia, y me guardo muy
mucho de sentenciar a nadie sin conocimiento, que yo también, ¡ay!, he
sido enchiquerado por conspirador, hallándome tan inocente y puro como
los ángeles del cielo. Sean o no criminales los antedichos sujetos,
tienen mi compasión por la pérdida de su libertad, y les deseo un buen
juez, \emph{rara avis}, que les redima o les condene según su merecido.
Creí yo que el bilbaíno, tan oportunamente puesto a la sombra, no nos
molestaría; pero no ha sido así. Poco antes de partirnos de Miranda, y
cuando nuestro caballero se despedía de sus amigos en el parador
cercano, llegó al nuestro una esquela escrita en la prisión por el
Arratia, y a Fernando dirigida, en la cual manifiesta sentimientos
contradictorios, extraña confusión de arrogancia y miedo, de amenaza y
súplica, bien como quien se engendró en una cárcel, donde toda
desesperación y delirio tienen su asiento. No viendo que por ahí nos
pueda venir peligro, y atento a evitar a Fernando hasta el más leve
motivo de disgusto, guardé la carta y nada le dije. Informo a ustedes
del suceso, porque es mi deber procurar que nada ignoren; mas no vean en
él motivo alguno de intranquilidad, pues para mí no lo hay. Sólo me
inquieta mi endeble salud y el deseo de llegar pronto a la gran Vitoria,
donde nos alojara mi amigo el Canónigo patrimonial, D. Vicente de
Socobio y Zuazo, a quien daríamos el gran berrinche si nos fuéramos a la
posada. Cualquiera que sea nuestro albergue, el Sr.~de Socobio recibirá
las cartas que de Villarcayo, de Madrid o de otra parte del globo
terráqueo se nos dirijan\ldots{} Ya viene Fernando; ya nos avisan que
todo está dispuesto. Oigo el piafar de los briosos corceles.
Partamos\ldots{} Dios nos acompañe. Reciban los vivos afectos del
caballero y los dos mozos, así como de este humilde
capellán---\emph{Pedro Hillo}.

\hypertarget{v}{%
\chapter{V}\label{v}}

\large
\begin{center}
\textbf{De D. Fernando Calpena a Pilar de Loaysa.}                  \\
\end{center}
\normalsize

\bigskip
\begin{flushright}\small \textit{Vitoria, Noviembre.}\normalsize\end{flushright}   
\bigskip

Querida madre: Ya no puedo ocultar a usted por más tiempo el verdadero
motivo de nuestra larga detención en esta ciudad. No había querido
hablarle de la penosa dolencia de nuestro buen D. Pedro, esperando a que
su estado me permitiese juntar en una sola noticia la enfermedad y su
alivio. Por desgracia, no puedo hacerlo así, ni sabe ya contenerse mi
aflicción, la cual ha de ser mayor si no la manifiesto a la persona que
más quiero en el mundo. Sí, madre querida; nuestro excelente y leal
amigo, el que a entrambos nos dio consuelo y ayuda en los tristes días
de nuestra separación, se halla gravemente enfermo desde que a Vitoria
llegamos, y hasta hoy vanos han sido los cuidados y la solicitud con que
le asistimos tanto yo como el Sr.~de Socobio y sus angelicales
sobrinitas. El mal que le aqueja es de los peores y más dolorosos: una
antigua afección a la vejiga, exacerbada en este viaje. Gran quebranto
sufrió la flaca naturaleza de nuestro amado presbítero con el espanto de
las terribles escenas de Miranda de Ebro; mas aunque le vi profundamente
afectado, pensé que con la distracción del viaje y mi compañía, para él
siempre la más grata, no quedarían rastros de aquel trastorno. Ello es
que no volví a ver en mi amigo la jovial sonrisa y el temple festivo que
constituyen su personalidad. En La Puebla empezaron a molestarle los
síntomas primeros de su mal: su tristeza en todo el camino me reveló su
padecimiento, aunque se esforzaba en ocultarlo. En cuanto nos apeamos,
fue preciso llamar al médico, y el ataque tomó en los días siguientes
alarmantes proporciones. Mantúvose una semana en situación estacionaria,
sin alivio notorio del sufrimiento ni crisis de mayor gravedad. Pero en
la siguiente, esta se ha manifestado con caracteres inflamatorios que me
hacen temer un desenlace funesto. Nada he de decir a usted de la
conformidad y paciencia con que este santo varón lleva su terrible mal:
ahoga sus quejidos para no causarme pena, y en los trances más dolorosos
intenta enmascarar su inmenso padecer con una sonrisa que me destroza el
alma. Habla de la muerte sin temor y hasta con regocijo; asegura que no
le importa morirse después de ver arreglados nuestros asuntos, y a usted
y a mí en libertad y disposición de amarnos. Esta era su aspiración,
este su anhelo. Viéndolo cumplido, no tiene nada que hacer en el mundo.
¡Cuánta abnegación, qué alma tan hermosa!

La asistencia facultativa es excelente, pues el Sr.~Busturia, hombre de
no común saber, grave y estudioso, pone sus cinco sentidos en mi
enfermo. De mi cuidado y vigilancia, velando a su lado noche y día, nada
tengo que decir a usted, pues ya comprenderá que no haría más por el
hermano más querido\ldots{} Si ocasiono a usted una gran pena contándole
el malestar de nuestro pobre amigo, me consuela el dar a mi madre una
parte de mi tribulación, seguro de que la tomará generosa, por ser mía,
y por ser objeto de ella el hombre nobilísimo y desinteresado que con
tanta lealtad nos ha servido.

En estas ansiedades que sufro, siento a mi madre conmigo; ella me da
aliento; ella redobla mi abnegación; su grande espíritu me conforta.
Quiera Dios que en mi próxima carta pueda enviarle mejores noticias su
amante hijo---\emph{Fernando}.

\hypertarget{vi}{%
\chapter{VI}\label{vi}}

\large
\begin{center}
\textbf{Del mismo a la misma.}                              \\
\end{center}
\normalsize

\bigskip
\begin{flushright}\small \textit{Vitoria, Diciembre.}\normalsize\end{flushright}   
\bigskip

Madre querida: Si en mis tres últimas vengo transmitiendo a usted
esperanzas con gradación muy lenta, en esta, que es la cuarta de
Diciembre, creo poder darlas con menos miedo de equivocarme. Me dice el
médico que cree sorteado el gran peligro, y que el enfermo entra en un
período de reparación, si bien es tal su debilidad, que aquella no puede
ser rápida. Ya su estómago admite alimento, y estas noches últimas ha
dormido con sosiego algunos ratos. El grave riesgo de la reabsorción
parece conjurado totalmente. No obstante, me abstengo de entregarme aún
a la alegría del triunfo, pues este es dudoso. Aprovechando los momentos
en que le tenemos despejado, le he leído algunos trozos de las últimas
cartas de Madrid, y aquel en que me expresaba usted su anhelo de vernos
juntos los tres festejando el restablecimiento de nuestro capellán le
afectó de tal modo, que hube de suspender la lectura porque el llanto le
ahogaba.

Por cierto que no sé cómo hemos de pagar a este Sr.~de Socobio y a su
familia abnegación tan extremada. Llevamos aquí cuarenta días, con las
increíbles molestias que ocasiona un enfermo grave, y ni un instante he
visto desmentida la bondadosa paciencia de estos señores, ni en ninguna
cara muestras de contrariedad o cansancio. ¿Proceden así por efusión
caritativa, o por un exceso de sociabilidad, en la cual prevalece el
culto de los cumplimientos? Creo que de todo hay en un grado superior.

Mucho me complace que ya esté en Villarcayo nuestro ínclito D. Beltrán.
Aguardo impaciente su primera carta. Ojalá sea histórica, y que siga el
hombre con la vena de comunicarte los sucesos políticos y militares con
su gracioso pesimismo. La última que me escribió de Madrid con la reseña
biográfica del nuevo Ministerio es deliciosa. ¡Cuánto más dignas de los
honores de la letra de molde son esas donosas pinturas que las infinitas
insulseces que \emph{fatigan las prensas} uno y otro día, y que sólo
servirán, como dice Bretón, \emph{para envolver los dátiles y el queso!}
Y ya que hablamos de notas biográficas, algo tengo que decir a usted de
las mías, pues mi pobre historia, aunque parece dormida, no lo está, y
cuando menos lo pienso se remueve, causándome tristezas y zozobra.
Cuando esté más tranquilo y vea libre de todo peligro a mi caro
capellán, le contaré a usted\ldots{} Pero no, no: se lo contaré ahora
mismo, para que no caiga en cavilaciones, que la mortificaran más de lo
justo.

Vamos a ello, que tengo toda la noche por mía para darle a la pluma.
Hillo duerme y yo velo, platicando con mi adorada madre, que se me
figura está detrás de mí, mirando por encima de mi hombro lo que
escribo. Esta mañana, hallándose el enfermo muy animado, y, según decía,
con ganas de vivir, hablome así: «Fernando, se librará mi alma de un
gran peso si te revelo un secretico.» Total: que Zoilo Arratia se
presentó en Villarcayo preguntando por mí el día siguiente al de nuestra
salida. No es esto sólo. En Miranda, a donde se cree que fue en mi
seguimiento, acompañado de otros dos individuos que Hillo desconoce, me
libré de tan enojosa visita por la circunstancia de haber sido presos
los tres caminantes a poco de su llegada, ingresando en la cárcel. ¡Qué
raro es todo esto!, ¿verdad, madre mía? Entiende D. Pedro, por algo que
oyó en Miranda, que les detuvieron por espías y desertores. Casi estoy
por salir a la defensa de Zoilo Arratia, no creyéndole capaz de tan feos
delitos, si bien por otras infames violaciones de la ley moral le juzgue
merecedor de condenación eterna. Bueno: sigamos, que aún falta lo mejor
del secretico. En su calabozo escribiome el bilbaíno una carta, que
recibió D. Pedro mientras estaba yo en la calle despidiéndome de mis
amigos. Naturalmente, por no disgustarme, se abstuvo de dármela, y la
guardó en la cartera donde lleva sus testimoniales y otros papeles de
importancia. «Busca, hijo, busca ese documento y descífralo si puedes,
que para mí el que tales desatinos ha escrito, más que en el calabozo de
una cárcel, debiera ser aposentado en la jaula de una casa de locos.» No
tardé en encontrar la carta, y a la vista la tengo. Escrita con
excelente letra española de pendolista, lleva en torcidos garabatos la
firma del esposo de Aura. Extracto en forma breve sus conceptos
delirantes y su nervioso estilo: «Estoy preso. Juro a usted que soy
inocente. Bien puede creerme esto, como creerá que le odio con todo mi
corazón. He venido en busca del señor D. Fernando para que celebremos
pacto de amistad, matándonos como dos hombres bravos\ldots{} Sálveme,
señor\ldots{} Usted me aborrece, yo le aborrezco\ldots{} Decidamos
noblemente cuál debe vivir. Si usted estuviera preso, yo le salvaría. Yo
carezco de libertad: démela usted; sálveme, que bien puede hacerlo con
sus influencias. Seamos uno y otro libres, y al punto se verá cuál de
los dos debe vivir y cuál no\ldots»

Vea usted, señora madre, una verdad romántica, salida de la vida real, y
rectifique lo que no hace mucho me escribía, asegurando que el
romanticismo no tiene existencia mas que en los libros y en el irritado
numen de los poetas. La tranquilidad espiritual que ahora goza usted le
inspira estos juicios. Según vivimos, así pensamos. Las ideas audaces,
las antítesis violentas son el centelleo de las Pasiones que nos agitan.
La sensatez y el razonar frío nacen de la regularidad, de la
satisfacción de los deseos\ldots{} La intensidad dramática de un
conflicto personal, de uno de esos nudos fatales que ofrece la vida,
hacen de cualquier hombre vulgar un personaje de Víctor Hugo o Dumas.
Andan por el mundo más Hernanis y más Antonys de lo que ordinariamente
se cree\ldots{} Sea usted benévola con mi pedantería, y no se inquiete
por el repentino hallazgo de la carta romántica, que a mí no me ha
causado el efecto que su autor, en este caso poeta sin saberlo, ha
querido producir en mí. La guardo y espero. Me va muy bien con este
clasicismo a que hemos llegado, después de tantas turbaciones y
angustias, y no quiero salir de un estado en que gozo la inefable dicha
de vivir en comunidad de ideas y sentimientos con mi querida madre.
Pongo fin a estas cosillas con un aforismo que acabo de descubrir, y del
cual doy a usted traslado para que se ría o nos riamos juntos: \emph{La
felicidad es clásica}.

\emph{Domingo}.---No tuve prisa en terminar mi carta, porque el furioso
temporal de nieve nos priva de correo, según dicen, en dos o tres días.
El de Villarcayo me trajo ayer carta de Valvanera, con la noticia de
estar Pepita afectada de calenturas, aunque leves, alarmantes por la
deplorable propensión de esas criaturas a los males de pecho.
Afortunadamente había remitido la fiebre, y esperaban una pronta
mejoría. Trájome también una donosa epístola de D. Beltrán, de letra de
Nicolasita, pues la menguada vista del ilustre señor difícilmente le
permite ya, ni aun con cristales de gran fuerza, largas tareas de
escritura. Pero su inteligencia y gracia no merman al compás de la
vista. Había de leer usted, para gozar de ella como yo, la pintura de
las fatigas que está pasando el pobre conde de Ofalia en la Presidencia
de Ministros. Según D. Beltrán, las napolitanas han llevado al
Ministerio al noble prócer y diplomático D. Narciso de Heredia, porque
en él ven al único arreglador de la intervención extranjera que nos
libre de la guerra civil. Créese que esta vez, como las anteriores, Luis
Felipe, a pesar de su amistad personal con Ofalia y de lo mucho que le
considera, dirá como el pastor: «Con tu pan hago las migas, que con el
viento no se oye.» En cambio, el bondadoso Conde anda como atontado
entre el barullo de las Cortes, elegidas antes de su nombramiento,
compuestas de oradores fogosos que a todo trance quieren
\emph{ministrar}, aunque sea sólo por un par de semanas, para repartir
docena y media de destinitos entre los hambrones de la familia. Las
disensiones del General en jefe con el Gobierno le traen loco; el
militarismo crece y todo lo avasalla. ¿Dónde está el hombre de Estado
por quien la nación suspira? El festivo historiador Urdaneta cree que el
Mesías político que esperamos no es otro que su nieto el marqués de
Sariñán, hace días electo diputado por Tudela, y ya camino de la Corte,
\emph{apretándole a ello} la falta que hace en España su presencia,
según los agravios que piensa desfacer y tuertos que enderezar. Con
estas burlas de su propia estirpe mezcla D. Beltrán gallardamente
juicios muy acertados sobre las diversas cuestiones pendientes, como esa
zaragata que ahora se traen por restablecer los diezmos en el ser y
estado que tenían antes del corte que les dio Mendizábal. La lucha entre
el \emph{progreso} y el retroceso como ahora dicen, se parece a la
controversia que entablaron los conejos acerca de si era pachón o
podenco el can que les perseguía. Confía D. Beltrán que \emph{Higinio} y
\emph{Alejandro}, los héroes de la Granja, habrían de encontrar
arbitrios de gobierno más eficaces que los de estos señores, si les
pusieran en las poltronas, y les dejaran proceder conforme a su
elemental criterio, sin nada de lo mal aprendido en libros o peor
cursado en las aulas parlamentarias. No le oculto a usted que el donaire
de nuestro anciano me hace dichoso, y que no puedo menos de ver en el
fondo de él una observación sagaz y un sentido justo. Es el siglo
pasado, filósofo y analizador, que se ríe del barullo en que nos hemos
metido los del presente, queriendo cambiar de mogollón ideas, formas y
costumbres. Si digo un disparate, no me haga usted caso.

\emph{Martes}.---Continúa el mal tiempo, y los correos empantanados,
contratiempos que tengo por insignificantes, junto a la felicidad de ver
a mi querido clérigo en franca mejoría. Lo que siento es no poder
transmitir a usted por los aires la expresión de mi gozo. Hoy quería D.
Pedro escribir a usted un parrafito; pero no se lo he permitido, porque
aún está muy débil. Ya lo hará otro día, cuando los buenos caldos de
gallina que le administran estas señoras vayan dando a sus sentidos
corporales la energía de que hoy carecen. Leo al enfermo lo que escribo,
y con esto se entretiene y es feliz. De esta familia de Socobio contaré
a usted muchas cosas: aún no es tiempo. Son todos ellos, varones y
hembras, un poco arrimados al \emph{retroceso}, lo cual no quita un
ápice a la bondad de sus corazones y a la excelencia de su conducta
social. Parientes cercanos tienen en la facción, y alguno va y viene que
les trae noticias frescas de lo que en ella pasa. Me abstengo por
delicadeza de hacer indagaciones sobre estos particulares, y nada les
pregunto. Hoy hablaré a usted con preferencia de un conocimiento que
hice anoche mismo, a poco de cenar, por mediación de nuestro bondadoso
D. Vicente de Socobio. Hablome de un joven que ardientemente deseaba
conocerme, y abriendo yo al instante las puertas de mi confianza al
desconocido sujeto, no tardé en verle llegar a mí. En el comedor
trabamos un largo coloquio, del cual vino algo parecido a la amistad,
con las naturales reservas, pues el individuo de autos me ha parecido
sumamente agudo, de estos que, revelando extenso saber de cosas, aún dan
la impresión de que ocultan mucho más de lo que revelan. Es pájaro de
cuenta, según las primeras sensaciones de mi olfato, y no rehuyo las
nuevas visitas que me anuncia, pues la de hoy, para hacer boca, ha sido
sustanciosa y de gran interés para mí, como verá usted por lo que voy a
contarle.

D. Eustaquio de la Pertusa, que así se llama, o dice llamarse, este
despabilado mozo, empezó revelándose como uno de los tres individuos
presos en la cárcel de Miranda el día mismo de nuestra llegada: sus
compañeros de viaje y de infortunio eran Zoilo Arratia y otro bilbaíno
nombrado Iturbide. ¿Qué tal? Esto no lo esperaba usted, ni tampoco que
mi visitante se declaró autor de la carta de Zoilo en su parte
caligráfica y en algunos toques de su extravagante estilo. Vamos de
sorpresa en sorpresa, mi querida madre, y no es la menor que el señor de
la Pertusa está libre, como atestigua su presencia corporal, y los otros
infelices continúan presos. ¿Por qué esta diferencia de suerte? ¿Será
porque se ha demostrado que Iturbide y Arratia son criminales, y D.
Eustaquio inocente? No, señora; precisamente ocurre todo lo contrario, y
vea usted el giro paradójico de este singular caso, que entra de lleno
en la esfera de las creaciones románticas. En un arranque de sinceridad
y de confianza, que no sé si me asombra o me asusta, el Sr.~de la
Pertusa me ha revelado que sus compañeros se hallan tan limpios del
crimen que se les imputa como los ángeles del cielo; él, mi romántico
personaje, no podía decir lo mismo. Sin dar tiempo a que yo expresara
las observaciones que sobre tan extraña confesión se me hacían, me
agració con preciosos datos de su historia. En su agitada vida militar y
política había desertado dos veces: la primera, de las filas de los
urbanos de Huesca, donde defendió la causa de Isabel; la segunda, de las
filas de Cabrera (división de Forcadell), donde combatió por la Causa de
D. Carlos. La realidad y la experiencia persuadiéronle de que ambos
ejércitos eran cuadrillas de locos, igualmente ominosas ambas banderas,
funestos sus caudillos, infernales sus armas; y por estas y otras
razones que no podía revelar, hase afiliado en las banderas de la paz, o
sea en el salvador, en el honrado y noble partido que trabaja por la
terminación de la guerra, no con pólvora y balas, sino con perdones y
abrazos. Siguió a esto un ardiente encomio de los elementos de
inteligencia y fuerza que constituyen el tal partido, al cual pintó como
un gran cuerpo invisible dentro y debajo de las multitudes combatientes
y en toda la extensión de la masa social española. Clero y milicia,
nobleza y estado llano, forman la inmensa hueste de la concordia, y ha
de alcanzar esta provocando lo contrario, o sea la discordia, en el seno
de cada uno de los partidos guerreros. No me parecía mal este plan de
campaña de los pacíficos, y al punto lo relacioné con los últimos
disturbios en el ejército de la Reina y los síntomas de indisciplina en
el de Don Carlos. En buen hora viniese la descomposición si con ella
venía la paz; pero esta no me parecía, y así se lo dije, muy firme y
sólida, fundada sobre el cimiento de las energías corruptas.

Oyendo al exaltado joven, que se me iba representando como un pez muy
largo y de muchísima trastienda, me asaltó una idea, después
otra\ldots{} Pensé primero en la monstruosidad inconcebible de que
siendo culpable D. Eustaquio e inocentes sus compañeros, hubiera
recobrado el malo la libertad y los buenos no. Interrogado por mí con
vehemencia acerca de este punto, díjome calmoso, clavando en mí sus ojos
penetrantes: «Ellos están presos porque no tienen quien les ampare. Yo
estoy libre porque cuento con relaciones, y por muy hondo que caiga, no
me falta nunca un clavo sólido a que agarrarme. Escribí a un amigo, este
habló con un personaje que no puedo nombrar, y héteme en la calle, sin
que se nos dijera por qué salía yo y mis compañeros se quedaban.» Tanta
iniquidad, injusticia tan cínica y desvergonzada, me sublevaron. ¿Pero
España es así y ha de ser siempre así? ¿Es en ella mentira la verdad,
farsa la justicia, y únicos resortes el favor o el cohecho? ¿Y sobre ese
terreno, más bien charca cenagosa, se quiere fundar cosa tan grande como
la paz?

Voy a la otra idea, que sin atormentarme como esta, también embargaba mi
espíritu. «¿Por qué viene D. Eustaquio a contarme a mí todas estas
cosas?---me decía yo, observándole sin dejar de oírle.---¿Qué ha visto
en mí que pueda inducirle a tales confidencias? ¿Es un conspirador, un
temible espía, o un farsante insustancial? Si su oficio es el espionaje,
¿por cuenta de quién lo practica?» De pronto surgió rápidamente de estas
ideas otra, y sin preparación alguna se la solté en esta forma ruda:
«Sr.~de la Pertusa, usted es agente de D. Eugenio Aviraneta. No le
pregunto por qué o por quién conspira, ni me importa saberlo. Sólo le
digo que pierde usted el tiempo si ha intentado tantearme para que le
ayude en sus maquinaciones.» Y él replicó al instante, gozoso,
estrechándome la mano: «Sr.~D. Fernando, no puedo revelar a usted quién
es mi jefe inmediato. Sólo le digo que soy soldado de la paz, algo más
que soldado, aunque no es bien que declare por ahora mi graduación. Por
la paz trabajo, por la paz sufro persecuciones. He querido conocer y
tratar a usted, porque el señor Socobio, a quien reverencio como a uno
de los más calificados de la \emph{Causa pacífica}, le designó entre los
que creen que para terminar la guerra debemos meter cizaña en ambos
ejércitos, desacreditar a sus caudillos, fomentar el cansancio de la
tropa, el hastío de los pueblos.» Yo no había sostenido que esto se
hiciera y trabajara como se amasa y cuece un pan, sino que era un hecho,
un caso real, engendrado por hechos y casos precedentes. Pertusa, que,
como todos los conspiradores, declaraba obra suya los fenómenos
históricos, producto de la vida colectiva, afirmó que lo que yo llamaba
hechos era resultado de la campaña de los \emph{pacíficos}. Despedile al
fin, fatigado de tan larga conferencia; pero él me anunció nueva
monserga para el siguiente día, ansioso de comunicarme cosas que a su
parecer me interesaban, y a cambio de este servicio me pediría mi
cooperación en una forma que no había de comprometerme. Más que mi
recelo ha podido mi curiosidad, y aquí me tiene usted con más deseo que
temor de que vuelva.

He vacilado, querida madre, en expresar aquí una idea que me asaltó;
pero dejando pasar la noche sobre ella, mi voluntad se ha decidido a
manifestar a usted todo lo que pienso. He dormido mal, atormentado por
esta idea, más bien propósito, que va usted a conocer ahora mismo. La
injusticia me irrita, me subleva. No sea el favor instrumento del mal;
séalo alguna vez del bien. Tengo amistades valiosas; dispongo de algún
favor. No soy digno de mí si no voy a Miranda y pongo en libertad a los
dos inocentes Zoilo Arratia y José Iturbide.

\hypertarget{vii}{%
\chapter{VII}\label{vii}}

\large
\begin{center}
\textbf{Del mismo a la misma.}                              \\
\end{center}
\normalsize

\bigskip
\begin{flushright}\small \textit{Vitoria, Diciembre.}\normalsize\end{flushright}   
\bigskip

Madre amadísima: Doy y usted me da los parabienes por la mejoría de
nuestro capellán, ya bien manifiesta, y la informo de la segunda
aparición del tal Pertusa, en el cual veo ya claramente un pájaro muy
sutil. Añado que es agradable, de rostro moreno, con vivísimos ojos de
ratón, sonrisa de pícaro redomado, mediano de cuerpo, de palabra fácil y
graciosa. Un detallito para concluir de pintarle: estudió para cura;
hasta recibir las primeras órdenes. Dejando la Iglesia por las armas,
recibió en las filas de los urbanos primero, en las de Cabrera después,
la última mano de la educación social con borla de doctor en toda humana
picardía. En filas le dieron el mote de \emph{El Epístola}, que ostenta
como recuerdo glorioso de sus campañas.

Voy a mi asunto. En la de hoy interesante visita (trasposición tenemos),
empezó por suplicarme el suministro de cuatro onzas para proseguir su
viaje, de que han de resultar notorios beneficios a la \emph{Causa
pacífica}, y antes de saber mi conformidad con este audaz expolio, me
doró la píldora, notificándome que en Vitoria se hallaba la cuadrilla de
\emph{Uva}, en la cual hay personas que podrán darme informes preciosos
de lo que más vivamente me interesa. ¿He dicho algo a usted de la
cuadrilla de \emph{Uva}? Creo que sí. En efecto, la banda de cantineras
ha entrado en Vitoria con la división de Buerens. Y puedo decirlo por
propio conocimiento, pues cuando escribo esta ya estoy de vuelta de la
posada de San Blas, donde, guiado por el amigo Pertusa, he podido
ponerme al habla con los apreciables vagabundos que surten de
aguardiente a nuestros soldados. El primero que me saltó a la vista, por
conocerle de antiguo, fue \emph{Churi}, el endiablado sordo, que se
manifestó descontento de verme, y no empleaba, como otras veces, el
lenguaje de sus garatusas expresivas. Su estado de ropa y carnes es
lastimoso. Me causó mucha pena; díjele como pude que a Bilbao volviese
con su familia, y el Sr.~\emph{Uva}, un sujeto que afecta gravedad
impropia de su condición y oficio, respondiome por él que eso mismo le
recomendaba la cuadrilla toda, sin conseguir quitársele de encima. Una
mujer a quien llaman \emph{Seda}, huesuda, larguirucha y muy charlatana,
pegó la hebra; y como notase en mí no poco agrado de oírla, me llevó
aparte, y entre sacos de paja y dornajos, me largó esta página
biográfica, que extracto para no cansar a usted.

El tal \emph{Churi}, que padece la enfermedad o monomanía del amor, con
la contrariedad de que su sordera le imposibilita para satisfacer su
espiritual anhelo, se prendó locamente de una hermosa mujer llamada
\emph{Saloma la navarra}; rechazado por esta, y brutalmente apaleado por
un tal Galán, al parecer marido, recayó el infeliz en su dolencia,
eligiendo para dama de sus pensamientos a otra graciosa mujer, también
llamada Saloma, con el aditamento diferencial de \emph{la Baturra}, y
tanto la persiguió el pobre bilbaíno con sus galantes obsequios, tales
muestras le dio de la fineza de su inclinación, que hubo la moza de
sentir, si no amor, compasión, accediendo a concederle su cariño. Si
este satisfizo en los primeros días al desgraciado joven, pronto hubo de
encontrar que el forzado afecto de \emph{la baturra} no colmaba la
ilusión de su alma enamorada, ávida de inefables consuelos. Se advierte
que las aspiraciones amorosas de \emph{Churi} son elevadísimas, no
contentándose con la fácil conquista de la mujer, sino pretendiendo la
suprema comunión, el himeneo ideal\ldots{}

Ya comprenderá usted, querida madre, que con los datos que me da la
señora \emph{Seda}, en su rudo y deslavazado estilo, compongo yo mi
historia, procurando la mayor fidelidad en lo substancial. Sigo, con el
recelo de que usted verá en lo que escribo antes la novela que la
historia. Lo mismo da: adelante\ldots{} Pues a las dos semanas, Saloma
no podía resistir ni la persona ni las extremadas demostraciones
patéticas del pobre \emph{Churi}. No pocos anduvieron en compañía de dos
individuos de la cuadrilla de \emph{Galvana}, trayendo y llevando
recados a una señora que se apareció medio loca en Orduña, y anduvo
desatinada por los caminos, hasta que su familia la recogió en Salinas
de Oñoro. Con los enredos que de dicha señora se traían, fueron Saloma,
\emph{Churi} y sus dos compañeros a La Guardia; siguieron hacia la
Bastida, y como \emph{la baturra} no se recatase en manifestar su
preferencia por uno de los de \emph{Galvana}, guapo mozo, cabal en todos
sus sentidos, trabáronse el tal y \emph{Churi} en grande pelea, primero
a puño limpio, luego con navajas, de la cual porfía resultó la dama más
estropeada que los galanes; volvió el sordo lleno de achuchones y
puntazos al corral pacífico de \emph{Uva}, y de \emph{Saloma} no se supo
más sino que en Miranda terminó su turbada existencia, recibiendo
cristiana sepultura en el camposanto de aquella villa.

Madre mía, oigo a usted exclamar: «novela, novela,» y yo digo:
«historia, historia.» Pulimentando la forma del texto, por el maldito
vicio de corrección a que nos induce la llamada cultura, sé que echo a
perder el pintoresco relato de la señora \emph{Seda}. Pero ya no tiene
remedio. ¿Cuándo inventarán un daguerrotipo de los sonidos que nos
permita sorprender la palabra humana en toda su espontánea
belleza\ldots? Pues sigo\ldots{}

No, no sigo, que estoy cansado. Hasta mañana.

\emph{Viernes}.---¿Se fijó usted en la muerte de \emph{la baturra}? He
aquí un enigma descifrado. Yo mismo empiezo a dudar, y digo con usted:
«¿novela\ldots?» Adelante. Agregado \emph{Churi} otra vez a esta
cuadrilla, no pasó mucho tiempo sin que aparecieran nuevas erupciones
del volcán de su pecho. No habiendo por allí hembras del buen ver de las
dos Salomas, \emph{navarra} y \emph{baturra}, ofreció su alma a una
viuda que vendía tabaco, la cual le doblaba la edad, conservando restos
apenas perceptibles de una destruida hermosura, contemporánea de
Talavera y Arapiles. Díjome \emph{Seda} con discreción que si no había
logrado el sordo poner digno remate a su conquista, no debía de andar
muy lejos de ello, a juzgar por ciertas blanduras que notaba en el
arisco carácter de la \emph{Pringosa}, que así llamaban al nuevo ídolo.
Lleváronme a verles en un corral donde el galán y la dama, con otros de
la partida, se ocupaban en los poéticos menesteres de limpiar él los
borricos, y ella de remendar los aparejos. Hallé en la dama notoria
semejanza con una característica que hemos visto en Madrid mil veces
haciendo papeles de patrona o de Celestina en piececillas y sainetes;
pero no puedo recordar cómo se llama. Traté de interrogar a \emph{Churi}
para que me aclarase el punto (convengamos en que la verdad se tuerce y
descompone en mis pobres manos, convirtiéndose en novela), el punto
obscuro, digo, de la señora trastornada, de la señora que vagaba por la
Peña de Orduña, de la señora\ldots{} en suma, de la que habría tenido un
dramático fin, si no la recogiera su familia en Salinas de Oñoro; mas
nada pude obtener del desgraciado mozo, que parece ya tan corto de
inteligencia como de oído, y es un arca cerrada con las llaves de la
imbecilidad. Sus ojos, antes tan vivaces, ya se cuajan atónitos y
mortecinos; su boca ha perdido los mohines que sustituían la palabra; su
cuerpo languidece. No hay manera de entenderse con él ni de que
pronuncie dos conceptos acordes. Parece que sólo le entiende la
\emph{Pringosa}, y que su alma, aislada de todo el Universo, sólo para
ella tiene lenguaje y expresión de alma humana. Dejele al fin, cansado
de sacudir golpes en aquella puerta para que se abriese. Está
enmohecida, y las ideas que guarda también son roña y podredumbre.
¡Infeliz \emph{Churi}!

Antes que se me olvide: el gran presbítero entra en convalecencia
franca. Come y bebe con mediano apetito. Le permito el uso de lápiz y
papel para que satisfaga el deseo de escribir a usted participándole su
resurrección. Pues sigo: me ha parecido que el servicio del
\emph{Epístola}, dándome a conocer la sociedad de los aguardenteros, a
quienes debo tan útiles informes, bien merece una recompensa. He puesto
en su mano tres onzas, asegurándole que disfrutará de otras tres si
cuando regrese de Vizcaya, para donde parte sin dilación, me trae
noticias auténticas de todos los individuos de la familia de Arratia.

\emph{Sábado}.---Me ha turbado toda la noche, quitándome el sueño, el
recelo de que usted no apruebe el encargo que di al condenado
\emph{Epístola}. Lo primero que hoy hice, al levantarme, fue mandarle
venir a mi presencia para retirar mis órdenes y deseos de nuevas
noticias. Con otra pelucona completo lo que me pidió, y le advierto que
no quiero saber nada, que no se acuerde más del santo de mi nombre. Pero
mientras corto comunicación con un pasado triste, veo que se adhiere más
y más a mi espíritu la idea que ya manifesté. Quiero libertar a Zoilo
Arratia, quiero emplear en aquel desgraciado enemigo mío los
sentimientos de justicia que llenan mi corazón. Nada haré sin el
consentimiento de usted. ¿Cree que me conviene guardar para otra ocasión
mi sed de justicia, y que mi cristiana idea no debe tener aplicación por
ahora? Dígamelo: que no hay para mí mayor gozo que someter mi criterio
al de mi buena madre, y expresar con mi subordinación mi grande amor.
¡Oh, que no fuera mañana mismo el venturoso suceso que usted me anuncia,
reunirnos en una casa que comprará en Burgos, Briviesca, o Medina de
Pomar! ¿Dónde? Si usted no me lo dice, me encariñaré con el sitio antes
de conocerlo. Puesto que usted aguarda sólo a que calmen los fríos para
venir cerca de mí, a mi lado quizás, yo al lado suyo, contaré los días
que restan de Diciembre, los del próximo Enero, calculando que al
término de ellos comenzará la mayor dicha de mi vida. Y cierro esta: ya
es bastante. El tiempo mejora; la nieve se derrite; el frío es
tolerable. Que pase, que pase pronto. Días asoleados y placenteros,
venid, venid. Abrazos mil de su amante hijo---\emph{Fernando}.

\hypertarget{viii}{%
\chapter{VIII}\label{viii}}

\large
\begin{center}
\textbf{De Pilar de Loaysa a D. Fernando.}                              \\
\end{center}
\normalsize

\bigskip
\begin{flushright}\small \textit{Madrid, Enero de 1838.}\normalsize\end{flushright}   
\bigskip

Hijo mío, niño, sí, sí, cuando pasen los fríos\ldots{} Pero estos fríos,
¿qué hacen que no pasan? Por mí no los temo, a pesar de mi delicada
salud; pero me han fijado ese plazo, y es forzoso que yo me someta a la
voluntad de quien puede y debe dirigirme\ldots{} Ya han pasado los
Santos Reyes, tan guapos con sus trajes de púrpura, su lucido séquito,
sus camellos arroganes\ldots{} Ahora estoy esperando al venerable San
Antón, con la barba hasta la cintura, su tosco sayal, y el cerdito tan
mono; le oigo ya los pasos\ldots{} Tras él, muy cerquita, viene San
Sebastián, y poco falta ya para estar a las puertas de febrerillo loco.
Pronto, niño mío, sí, prontito\ldots{} ¡qué gusto!

¡Ay, ay, cuánto he llorado con tu última carta! Tu anhelo de justicia,
tu sublime rasgo de caridad, salvando al enemigo injustamente condenado,
te enaltece a mis ojos; me siento orgullosa de ti. Ríanse otros de la
caballería, de ese ideal del bien y la justicia tan arraigado en almas
españolas; yo no me río, no puedo reírme de eso. Lo llevo en la masa de
la sangre. Caballeros mil tengo entre mis antepasados. En ti se
reproduce mi raza generosa, cristiana, grande por el valor, por la
abnegación y el heroísmo. Tienes a quién salir.

Te diré con entera franqueza lo que pienso sobre el particular. La
catástrofe de tus amores en Bilbao me obligó a imponerte una sumisión
absoluta, y con ella te salvé de mayores desastres; pero no he querido,
no, decapitar tu voluntad ni matar tu iniciativa. No puedo menos de
considerar, al propio tiempo, que al revelarme a ti y descorrer el velo
de tu origen, si te he dado el consuelo dulcísimo de poseer una madre,
he quitado a tu personalidad en el mundo aquel brillo, aquella dignidad
¿por qué no decirlo?, que ostentan personas nacidas de padres menos
ilustres, pero en condiciones normales y regulares. Esto es tan delicado
que no sé cómo decirlo. Pero tú lo entiendes, mi bien, y me basta.
Bueno: pues el conocimiento de tu origen nos trajo, creo yo, la
abdicación de tu voluntad. Mi amado hijo me resulta un muñequito, ¡ay,
sí!, un lindo juguete sin vida para recrear la mía. No, no: esta
condición muñequil no puede satisfacerte, ni a mí tampoco me satisface.
El vacío de que antes hablé, producido por la irregularidad del origen,
no se llena sino con la rehabilitación de la voluntad, para que con ella
emprendas altas y nobles acciones. Lo que te falta, aprecio de ti mismo,
conciencia robusta de tu valer, créalo tú con potente audacia, fundando
un hombre nuevo sobre las ruinas del pobrecito chasqueado en la Villa
heroica; lo que de menos tienes en dignidad por tu origen, búscalo ahora
y agrégatelo y complétate\ldots{} ¿Me entiendes? Creo que sí\ldots{}
Pues bien: tus impulsos de caballería me saben a gloria\ldots{} Soy muy
caballeresca. Te reconozco. Apruebo plenamente que quieras ganar lo
perdido. Tus ideas cristianas de suprema hidalguía y virtud son la
grandeza que yo quiero para mi hijo. Sí, da libertad a ese hombre.

Pero ¡ay!\ldots{} aguarda\ldots{} no\ldots{} Me dejo arrastrar de mi
imaginación\ldots{} ¿Y si te pasa algo? Ya sale aquí la madre. ¡Oh, sí!,
la madre tiene que mirar por tu vida, por tu felicidad. ¿Y si todas esas
grandezas morales y caballerescas me privan de tu felicidad, de tu
vida\ldots? No, Fernando, no hagas caso de ajenas desdichas. Deja a ese
hombre que se arregle como pueda\ldots{} Retiro lo que habrás leído.
Habló antes la ricahembra; ahora habla la madre. Súbitamente me vuelvo
muy ñoña. No me resigno a que el amor de mi vida afronte los peligros de
la ingratitud, de la brutalidad de un hombre que es quizás un
malvado\ldots{} No, no: consérvateme muñequito; desechemos las
aventuras, el quijotismo, las sublimidades peligrosas\ldots{} Ya soy
vieja, y quiero mi paz, tu felicidad. Seamos clásicos, muy
clásicos\ldots{}

Permíteme que suspenda esto y que aguarde algunas horas para pensarlo
mejor\ldots{}

He pensado, y me decido al fin por que no tomes ninguna resolución, al
menos hasta que yo vaya y hablemos. El otro podrá aguardar en la cárcel.
¿Qué le importa un mes más o menos? Seamos egoístas\ldots{} digo,
clásicos.

No estoy conforme, no. Me tomaré un plazo más largo, toda esta tarde y
toda la noche. Mañana, con mi cabeza despejadita y fresca, pronunciaré
sentencia definitiva. En tanto, no habiendo para mí otra alegría que
escribirte (pues mientras vacío en el papel mis pensamientos, me figuro,
como tú, que por encima de mi hombro miras lo que escribo), déjame que
garabatee un poco más, hablándote de otros asuntos. Pues sí: le cuento
los pasos al buen San Antón, y preparo mis bártulos minuciosamente,
apuntando todo lo que he de llevar para que no se me olvide nada. A mi
muñequito le llevo mil juguetes. Otros muñequitos como él, que se llaman
Víctor Hugo, Dumas, Byron, Walter Scott, a los que he provisto de
elegantísima ropa, encuadernación lujosa, con cantos dorados. Esto de
los cantos dorados es objeto de mis mayores ansias, y a propósito del
brillo y pureza del oro, he tenido terribles agarradas con el sastre de
libros, \emph{vulgo} encuadernador. Para tu romántica persona llevo
también tapas lujosas, abrigos de pieles, pues me temo que aun después
de mi llegada persistan los fríos enojosos. Y para nuestro buen Capellán
no faltará provisión de magnífica ropa de invierno. Vigilo el arreglo de
mi silla de postas y la proveo de todas las comodidades. Y no quiero
ocultarte que iré bien preparada también de recursos morales, de hábiles
defensas contra las intrigas de Juana Teresa. Por Valvanera he sabido
que fue a La Guardia con el único objeto de denigrarme, revelando a los
Navarridas secretos que descubrió revolviendo los papeles de D. Beltrán.
La impresión producida en aquella gente sencilla y timorata ha sido de
recelo y disgusto, pues \emph{Doña Urraca} supo presentar las cosas por
el lado que le favorecía, y llenar de escrúpulos el cerebro de las
muchachas y de sus apreciables tíos. La situación, hoy por hoy, es la
que a renglón seguido te expreso: Doña María Tirgo, resueltamente en
contra nuestra, con terquedad irreductible; D. José María, vacilante,
sufre grandes angustias y bascas, pues queriéndote de veras y
admirándote, se siente bajo la presión y horrible dominio de los de
Cintruénigo. Su mansedumbre y debilidad son un gran peligro, pues me
temo que al fin su hermana le arrastre, y le veamos en una actitud
marcadamente hostil. Fíjate bien en que D. José María es tutor de las
niñas, y que Demetria se halla bajo la autoridad tutelar hasta los
veintitrés años, que cumplirá en Mayo del 39. ¿Te vas enterando?
Demetria no podrá contraer matrimonio sin licencia de su tutor, y este,
según la ley, no está obligado a dar ninguna explicación de su negativa.
Por todo lo expuesto, mi querido hijo, en conciencia debo aconsejarte
que suspendas por ahora tu viaje a La Guardia. Conviene que nos demos un
poquito de tono. Nuestra dignidad nos exige no mostrar un interés
excesivo, ni las prisas del solicitante importuno. Ello ha de venir por
su propia madurez: no nos precipitemos. ¿Estás conforme? Aseguro que sí.

Y va de noticias. Ha llegado a Madrid mi excelso sobrino el Marqués de
Sariñán, con la investidura de diputado por Tudela. Pásmate: no ha ido a
buscar alojamiento apropiado a su categoría en Genieys ni en las otras
dos medianas fondas que aquí tenemos, y se ha metido en casa del amigo
Mendizábal, sujetándose a un modesto pupilaje. Viste regularmente; pero
sus camisas, obra de la tijera y aguja de \emph{Doña Urraca}, ofrecen un
corte de cuellos de extraordinaria novedad. A poco de jurar su cargo, se
ha lanzado a la oratoria, haciendo su estreno en la marimorena de los
diezmos con un discursito pálido, aprendido de memoria, que ha pasado
como un rumor, sin dejar eco más que en el \emph{Diario de las
Sesiones}. Forma en las filas del más furioso \emph{retroceso}, con
Alejandro Mon, y Castro y Orozco. Dícenme que gestiona la compra de
bienes monacales a bajo precio, entendiéndose con los que liquidan y
tasan. De esto no respondo. Lo verosímil no siempre es verdadero.

\emph{Domingo}.---He pensado, he meditado anoche\ldots{} Vuelvo de misa:
en mi espíritu se confirma esta resolución, que sin duda me inspira
Dios. Hijo mío, haz lo que te dicte tu gran corazón. No me determino a
limitar tu libertad, la preciosa iniciativa de quien lleva en sus venas
sangre de tantos héroes antiguos y modernos. Sé lo que digo, y lo
escrito, escrito está. Llena mi alma la convicción de que Dios ha de
protegerte, y a mí no me negará el consuelo de verte triunfante. Ansío
que tu alma se fortalezca de dignidad, que tu conciencia se recree
contemplando la nobleza de tus acciones. Dios está contigo. ¿Cómo no, si
yo soy buena, si te idolatro, si eres mi vida? No temo nada. Que a ti y
a mí nos gobierne tu magnánimo corazón. Mil besos de tu madre
amorosa---\emph{Pilar}.

\hypertarget{ix}{%
\chapter{IX}\label{ix}}

\large
\begin{center}
\textbf{De D. Beltrán de Urdaneta a Fernando Calpena.}                   \\
\end{center}
\normalsize

\bigskip
\begin{flushright}\small \textit{Villarcayo, Enero.}\normalsize\end{flushright}   
\bigskip

Joven ilustre: En estos regalados ocios, mi ancianidad se repara de sus
quebrantos, y heme aquí menos vejestorio, no te rías, de lo que a
primera vista represento. Hasta la facultad de ver, que era entre todas
las mías la más averiada, parece recobrarse, y aquí me tienes
escribiéndote sin auxilio de Nicolasita. Esta y su hermana me encargan
que no deje para lo último el ponerte sus memorias; insisten en que las
eche por delante, en los comienzos de la carta. Así lo hago, y relámete,
ingratuelo, con los dulces afectos que te envían mis nietas. Toda la
descendencia de mis queridos hijos está vendiendo vidas, lo que me
regocija en extremo, porque dice Valvanera que yo he traído la salud a
su casa. ¡Qué orgullo para mí\ldots! Entre paréntesis, me hiciste mucha
falta para las magnas obras del nacimiento que armé a los chiquillos, y
para la venida de los Reyes, que representamos en el salón con desusada
solemnidad, sin que faltaran camellos corpóreos, negros \emph{de carne},
y la estrella refulgente. ¡Y tú en Vitoria, detenido por la enfermedad
del eximio capellán! Gracias sean dadas a Dios por la mejoría de tu
amigo. Sólo falta que decrete pronto el restablecimiento y os traiga a
los dos para acá.

Ya sé que presenciaste en Miranda un suceso histórico. Fea y
horripilante página te tocó, joven ilustre. Pero así se aprende. En mi
campaña del Maestrazgo hube de familiarizarme de tal modo con los
fusilamientos y el continuo sacrificio de seres humanos, que ya ni un
ligero temblor me producían espectáculos tan terribles. ¡Bonita Historia
de España están escribiendo unos y otros, mi querido Fernando! En
parangón con esos trágicos anales, debemos presentar nosotros los del
género festivo, de que te mandé algunos capítulos matritenses, que
guardarás como oro en paño. La Providencia se encarga de encariñarme con
esta para mi fácil tarea, proporcionándome activos corresponsales, que
me envían, sin yo pedirlos, preciosos datos. Dime tú: ¿tienes noticia de
la toma de Morella por los carlistas? ¿Sabes cómo fue? ¿A que no? Pues
yo he recibido hoy mismo carta de un amigo que dejé por allá, Nicasio
Pulpis, el cual, como autor principalísimo en aquel lance, me lo
describe puntualmente. Antes de referírtelo, déjame filosofar un poco,
déjame que sea también algo profeta, que el profetizar es propio de
ancianos alumbrados por la experiencia.

Pues digo que ahora, con la posesión de aquella plaza en el riñón del
Maestrazgo, centro de una imponente masa de baluartes construidos por la
Naturaleza, Cabrera, cuyo militar instinto y ciega bravura conozco
\emph{de visu}, será dueño de toda la región española que derrama sus
aguas en el Mediterráneo. Pronto le verás dominando la plaza de
Castellón. Ambas riberas del Ebro, desde Caspe a los Alfaques, serán
suyas, y, por fin, Valencia prolífica, con sus codiciados frutos y sus
lindas muchachas, caerán en la garra del fiero leopardo. Este se ha de
crecer, no sólo por la importancia colosal de las posiciones que posee,
sino porque su ejército y territorio se mantienen libres de la discordia
y corrupción que reinan en el Norte. Lo que creó Zumalacárregui en
Navarra y Guipúzcoa se desmorona por la imbecilidad del partido
eclesiástico; en cambio, lo creado por Cabrera en Oriente adquiere cada
día más vigor, porque allí no hay partidos, allí no hay más que la
voluntad férrea de un gran soldado. El dualismo destruye la facción en
el Norte; la unidad la fortifica en el Este. Verás muy pronto a Cabrera
emancipándose de la autoridad de su menguado Rey, y combatiendo por un
absolutismo acéfalo, que llamaremos protectorado, dictadura. He aquí,
Fernandito, que lo que no han podido las realezas con el apoyo clerical
y las defecciones del ejército, lo puede un pelanduscas con algunos
puñados de barro popular. Apunta todo esto que te digo, para que si
cierro el ojo antes de lo que deseo, veas confirmada en los hechos la
profecía del humorístico D. Beltrán. Cuando la realeza falla, cuando la
milicia es impotente, inepto el cleriguicio, incapaz la aristocracia,
veamos, hombre, veamos si aparece algo grande y fuerte en medio del
surco abierto en la tierra, allí por donde anda la reja del arado. ¿En
dónde crees tú que está la energía? ¿En los señoritos, en la nube de
palaciegos y empleados, en los de pluma en la oreja, en los de espada al
cinto, en los asentistas y contratantes, en los que comen de fonda, en
los que andan muy huecos porque han bebido algunas gotas de lo que
llaman \emph{el espíritu del siglo}? No sabes contestarme. Miras en
derredor tuyo, y no ves la energía. Yo tampoco la veo; pero sé dónde
está y me lo callo, porque no crean que chocheo, que desvarío. Y como te
veo arrugar el ceño, corto aquí mi vena profética y te contaré cómo
ganaron los carlistas la plaza de Morella, y el ingente castillo
enclavado en risco inexpugnable. Pues salió de la plaza un aprovechado
artillero cristino, más traidor que Judas, y propuso a Cabrera construir
una escalerita, cuyas medidas bien tomadas dio, con la cual podían subir
al castillo veinte hombres, favorecidos de la obscura y tempestuosa
noche. Ello fue un asalto de teatro; vieras allí trepar a los baluartes,
franqueando ásperas rocas talladas a pico, a la vil comparsa con el
traidor a la cabeza. Sorprenden al centinela y le dejan seco. Apodéranse
del depósito de granadas de mano, y la emprenden contra la guarnición,
que acude a una defensa tardía. El Gobernador trata de forzar la puerta
del castillo, ya en poder del audaz asaltante, y resbala y cae, y se
disloca ambos tobillos. La guarnición desmaya, recoge del suelo a su
jefe, y adiós Morella. Se largan de la plaza, viendo la imposibilidad de
defenderla, una vez perdida la cúspide del fortísimo mogote, que es como
un gigante con cabeza de hierro, manos de fuego y patas de granito.

¿Qué te parece de este hecho de armas? Dirás que es vulgar, villano. No,
hijo: es la guerra elemental y primitiva. Ahí tienes cómo sin paralelas,
ni planos, ni artillería, ni minas, ni nada de ciencia militar, se toma
una formidable plaza. ¿Pero qué digo? Fundamento de la militar ciencia
es la astucia. Añádele el arrojo, y tienes el perfecto soldado. Ahora
irán los sabios a recobrar a Morella, y verás lo que sacan\ldots{} Te lo
repito, sé dónde está la energía; pero me lo callo. Quiero llevarme a la
tumba ese supremo conocimiento.

Y hablemos de otra cosa, ea. Al pobre Don José M. de Navarridas le
tenemos loco, de la grande perplejidad en que le ha puesto \emph{Doña
Urraca}, pintándote como un monstruo de vilipendio. ¡Horror de los
horrores! ¡Vaya, que tú monstruo! ¿Y yo, qué seré\ldots? Lo menos el
Anticristo. Nuestra generala Pilar, que ya se dispone a venir a
regocijarnos con su presencia divina, nos manda suspender las
hostilidades, y a mí me recomienda la prudencia, pues opina, con muy
buen juicio, que si tomo partido por vosotros con demasiado coraje, el
furor de la \emph{hidra de Cintruénigo} puede precipitar las cosas de un
modo desfavorable para ti. No hay duda que el benditísimo Navarridas, a
quien tiene trincado por los cabezones la implacable Tirgo, negaría el
consentimiento si fuésemos tan simples que pidiéramos a deshora la mano
de la niña. No haremos tal. Nos consta que las últimas embestidas para
que apechugue con Rodriguito han sido tan infructuosas como las de
marras. Se mantiene en sus trece, ¡vaya una hembra!, guardando en su
alma, con piadoso recogimiento, la devoción del monstruo.

Adiós, hijo mío. Recibe los dulces afectos de esta familia y la
bendición de tu anciano amigo y maestro---\emph{D. Beltrán}.

\hypertarget{x}{%
\chapter{X}\label{x}}

\large
\begin{center}
\textbf{Del mismo al mismo.}                                         \\
\end{center}
\normalsize

\bigskip
\begin{flushright}\small \textit{La Nestosa, Febrero.}\normalsize\end{flushright}   
\bigskip

\emph{Chiquío}: Allá te va más historia, y de la palpitante, de la que
duele. Henos aquí refugiados en la villa de La Nestosa, donde hemos
tenido que replegarnos todos con la familia menuda, batería de cocina y
regular impedimenta de provisiones, huyendo del dios Marte, que se metió
inopinadamente en nuestro valle de Mena, mandando primero por delante
gavillas de facciosos, trayéndonos después dos divisiones del ejército
del Norte, que iban al socorro de Balmaseda. Tan feo mohín vimos en la
cara y entrecejo del citado dios de la guerra, que acordamos retirarnos
por el foro, trasladándonos a la casa de Juan Antonio en La Nestosa,
donde hemos esperado el resultado de los brillantes hechos de armas que
han despejado aquel territorio, arrancando a Balmaseda \emph{de las
garras del retroceso} (así dice el alcalde de esta villa, el cual goza
de merecida fama por la finura de su estilo).

A la salida de Villarcayo me encontré a Baldomero, con quien charlé como
una media hora, de la cual consagramos algunos minutos a tu persona,
pues él me preguntó por ti, y yo le informé de tu feliz situación
presente, agregando los vituperios que me parecieron del caso. También
vi al General Fermín Iriarte, a Latre y a Castañeda. Conociendo mi
repugnancia de referir hechos militares, que comúnmente son cortados por
un patrón casi invariable, no me exigirás puntual noticia de los
achuchones que en aquellos riscos y barranqueras se dieron unos y otros.
Ello es que el caudillo faccioso Cástor Andéchaga recibió un tremendo
palizón, y que serán inscritos en el libro de la Historia los nombres de
Biérgol, Orrantía y Gordejuela, donde corrieron torrentes de sangre,
según dicen, que yo no lo he visto. Uno y otro día, desde el 29 de
Enero, escaramuzas y combates se sucedían, llevando la mejor parte los
de acá. Pero tanta y tanta fuerza acumularon esos indinos en los montes
circundantes de Balmaseda, que el de Luchana tuvo que echar el resto,
embistiendo con el brío que suele gastar, y al fin las \emph{huestes del
progreso} (sigue hablando mi alcalde) forzaron el paso de Orrantía, con
lo que quedó sellada la victoria, y el \emph{servilismo} en desordenada
fuga. Veremos lo que duran estas ventajas, pues, según observo, en la
presente guerra no hay mas que un tejer y destejer continuo, y un tomar
y dejar territorios. Cruel sangría derrama la vida de la patria en el
suelo de esta, y si no se la cierra pronto, las venas no contendrán más
miseria y podredumbre. Ya me parece un bromazo demasiado cruel la
contienda entre el \emph{D. Isidro} y la \emph{angélica}, y hay que
pedir a Dios y al Rey de Francia otros cien mil tataranietos de San
Luis, o de San Felipe, que vengan a poner orden y concierto en esta casa
de orates, donde no hay ningún loquero que sepa su obligación.

En fin, hijo mío, que tú has de ver muchas cosas que ojalá no sean tan
tristes como las presentes. Aunque todo ha terminado, y Balmaseda y su
comarca son de Isabel, y ningún riesgo correríamos en Villarcayo,
seguiremos disfrutando del buen tiempo y del sosiego de este lindo
valle, y aquí estaremos hasta que recale tu madre en Medina,
acontecimiento dichoso que nos anuncia para el próximo marzo. Valvanera
y Juan Antonio te escribirán. Hoy me toca a mí, con el auxilio de
Nicolasa (pues la condenada vista se me ha resentido de la jarana de
estos días), ponerte al corriente de nuestra fuga, sin que grandes ni
chicos hayan sufrido la menor alteración en su salud. Ni una tos
infantil hemos oído en el tiempo que aquí llevamos, y fuera de ansiedad
por lo que pudiera ocurrir en la casa de Mena, todo ha sido
bienandanzas. Que te veamos pronto, niño, y que tu Capellán se recobre,
y que tu mamá nos visite, y que nos reunamos todos para general
satisfacción, presididos por la venerable persona del
viejo---\emph{Urdaneta}.

\hypertarget{xi}{%
\chapter{XI}\label{xi}}

Agotada la preciosa colección de cartas que un Hado feliz puso en manos
del narrador de estas historias (lo que no ha sido flojo alivio de tan
rudo trabajo), su afán de proseguirlas, revistiendo de verdad la
invención y engalanando lo verdadero, oblígale a lanzarse otra vez por
valles y montes, ojeando los acontecimientos y las personas, que de unas
y otros da pingüe cosecha la España de aquellos días. Favorecido de otro
Hado benéfico, de los muchos que andan entre gente de pluma, tuvo la
suerte de adquirir en su primera salida conocimientos muy útiles, y allá
van del magín al papel, comenzando por la noticia bien comprobada de que
hasta principios de Marzo no pudo abandonar Calpena la hospitalaria
esclavitud de los señores de Socobio en Vitoria, por no permitir salida
más temprana la convalecencia del capellán, que sólo en aquella fecha se
presentó segura. En un buen coche, con escolta de los dos criados,
bajaron a Miranda, donde sólo se detuvieron algunas horas. Después de
celebrar breve plática con D. Leopoldo O'Donnell, que mandaba la fuerza;
de repararse de alimentos y dejar en la cárcel un recado verbal, por
mediación del presbítero Bonifacio Cebrián, primo de Sabas, partieron
para Briviesca, donde estaba concertado el encuentro con la señora
condesa de Arista, que venía de Madrid. No consta la fecha exacta de la
extremada felicidad de la madre y el hijo al verse juntos de hecho,
aunque ya por el pensamiento y el amor lo estaban muy estrechamente;
pero ello fue algunos días antes de la festividad del glorioso Patriarca
San José. Y como el más lerdo puede imaginar, cual si las viera, las
ternuras, la hermosa efusión del encuentro de aquellas almas, se omite
la descripción prolija del suceso. Fernando reconoció en su madre la
dama ilustre, amorosa, inteligente, tal como su viva imaginación la
construyera; Pilar le había visto como al escondite, en teatros y sitios
públicos, el año de Mendizábal; mas viéndole ya sin miedo, y teniéndole
tan seguro en sus brazos, por larguísimo rato le apretó en ellos con
rígida fuerza, como si temiera que se le quitaran. En el agraciado
rostro de Pilar de Loaysa, la huella de las penas y ansiedades largo
tiempo sufridas concordaba las facciones con la edad; pero en el cuerpo
y talle salían burlados los años, pues por mucho que se quisiera
estirar, los cálculos no podían pasar de los treinta. De la dignidad,
nobleza y elegancia de su porte, cuanto se diga sería pálido. Voz y
modales declaraban la mujer de alto nacimiento. «¿Recuerdas haberme
visto alguna vez?»---preguntó a Fernando.

---Sí: una vez, una noche, en el teatro del Príncipe.

---Es verdad. Hacían los \emph{Hijos de Eduardo}. ¿Y tú\ldots?

---No sospeché, no\ldots{} Recuerdo haber dicho: «¡Qué elegante
señora!\ldots» Usted me miró un momento con los gemelos, nada más que un
momento\ldots{} Yo la miré con los míos largo rato. Entró en el palco mi
entonces jefe, el gran D. Juan Álvarez\ldots{}

---¿Por qué no me tuteas?

---Porque, con su permiso, el tutear a las personas mayores me parece
irrespetuoso. No todas las modas novísimas me convencen.

Este breve diálogo y el decir D. Pedro, elevando al cielo las palmas de
las manos, que aquel era el día más feliz de su vida, fue una suave
transición desde la escena de ternura a la espléndida comida que se les
sirvió en el parador de Briviesca. Traía la Condesa cuatro individuos de
servidumbre, de los cuales tres pertenecían al sexo fuerte, y un mediano
cargamento de baúles y cajas. En lo restante de aquel día y parte de la
noche, no dieron D. Fernando y Pilar paz a las lenguas, ávidos de la
comunicación verbal, que por primera vez gustaban, y que les resarcía de
las reservas y discreciones que impone la escrita. El gesto, el signo,
la sonrisa, la expresión de ojos y boca, eran para entrambos nuevo
lenguaje que estrenaban con delicia. No se saciaban, no veían el fin de
su charla seria, festiva, grave, infantil. Durmieron tranquilamente, y
al siguiente día tempranito partieron, por Oña, a Medina de Pomar, con
la buena compañía de un tiempo primaveral que estimulaba el regocijo de
sus corazones. Entraron en la ilustre villa al caer de la tarde,
ocupando una de las mejores casas del Condestable, Duque de Frías,
arrendada por Pilar desde principio de año, y ya con todo esmero
provista de cómodos muebles y de cuanto han menester las personas hechas
a la vida regalada. Con los criados que desde Febrero estaban allí y los
que acompañaron a la Condesa, el caserón tomó prontamente aspecto de
señoril morada, sin que nada faltase en ella. Las primeras visitas
fueron las de Maltrana y D. Beltrán, que no cabía en su pellejo de
alborozado y vanaglorioso. Poco tardó en presentarse Valvanera con sus
niñas, y no hay para qué decir que el besuqueo y las ternezas no tenían
fin. Quince o más días duraron aquellas satisfacciones, y tan del gusto
de Pilar era la compañía del viejo Urdaneta, que al despedirse los
Maltranas, le retuvo en su palaciote, con mucho gusto de él y de D.
Fernando. Forzoso era que este partiese al cumplimiento de obligaciones
que se había impuesto, y en las cuales hubo de confirmarse, previo el
asentimiento de su buena madre, que una y otra vez le repitió estas
memorables expresiones: «Hijo mío, yo te privé de la voluntad en una
época de revolución; pero te la he devuelto. En ti resigno toda
autoridad; tu corazón grande a ti y a mí nos gobierna. Confío en Dios,
que apartará de tu cabeza todo mal.»

Convinieron en que D. Pedro no le acompañaría, por el quebranto, no bien
reparado aún, de su salud endeble, y se agregó a la servidumbre de D.
Fernando un criado antiguo de la casa de Cardeña, al cual Pilar trajo
consigo; hombre muy para el caso, honrado y valiente como buen
guipuzcoano, del propio Eibar, fuerte como un oso, leal como un perro,
muy corriente en lengua éuskara, y conocedor de la topografía del país,
así como de toda Navarra y alta Rioja. Llamábase Juan Urrea, que quiere
decir el oro, y había servido en los estados aragoneses de Arista y
Javierre antes de pasar a la guardería de la Encomienda, famoso coto de
la casa ducal cerca de Madrid. Pilar fiaba en sus cualidades, que
realmente eran oro puro, y en su poder muscular, semejante a la virtud
del acero. Retirose a Villarcayo el criado de Maltrana, y D. Fernando
salió con Urrea y Sabas, dejando en Medina el coche, que más bien les
servía de estorbo en los caminos que habían de emprender. Triste se
quedó la de Arista en su caserón; pero confiada en la buena estrella de
su amado hijo, sobre cuya cabeza veía y sentía la bendición del cielo,
juntándose para fortificar esta confianza el amor y la fe. D. Beltrán y
D. Pedro extremaban los recursos sociales para distraerla, y a los pocos
días le mandó Valvanera, en compañía del mayordomo de la casa y del cura
de Medina, a su hija Nicolasita, para mejor asistencia en la soledad de
la noble señora.

Llegado que hubo el caballero a Miranda, se personó en el alojamiento de
O'Donnell y allí se estuvo dos largas horas; salieron juntos, regresaron
con otro señor que parecía como anfibio, entre paisano y militar; la
siguiente mañana se la pasó D. Fernando midiendo repetidas veces con sus
pasos la distancia entre la cárcel y el Ayuntamiento, y entre este y la
Comandancia militar, acompañado en estas correrías por el diligente
padrito Cebrián, pariente de Sabas. Durillo estaba el empeño en que puso
toda su energía el Sr. de Calpena; mas tanto pudo al fin su constancia,
su abnegación, y en algunos puntos del via crucis su largueza, que al
fin, a las seis de la tarde del 4 de Abril entró en la cárcel de
Miranda, con la orden a raja tabla para que el alcaide pusiera en
libertad a los presos Zoilo Arratia y José Iturbide. Era un caso, no
nuevo, de las corruptelas de la justicia en tiempo y país de guerra; mas
el caso suele acontecer aquí en tiempos y territorios de paz. Achaque es
este del favor, forma del \emph{milagro administrativo}, sustituto de la
razón así para el mal como para el bien.

La entrada de D. Fernando en el calabozo donde materialmente se pudrían
en mísera inanición dos seres humanos, fue por demás patética.
«¡Eh!\ldots{} Iturbide, Arratia---dijo al franquear la puerta, seguido
del calabocero y del curita,---están ustedes libres. ¡Al fin!\ldots{}
Más vale tarde que nunca.»

Iturbide saltó del suelo, en que yacía como un ovillo, y exclamó
abriendo los brazos: «¡Jesús, Jesús mío!» Zoilo, tumbado como un tigre
moribundo, rugió palabras ininteligibles. No se enteró de lo que oía: su
actitud era de estupor soñoliento, casi de idiotismo. Por la reja
entraba bastante luz solar para que Calpena pudiera ver la frente y
mejillas del bilbaíno despellejadas por sus propias uñas, el desvarío de
su mirada, la demacración de sus facciones. Hubo de atender a Iturbide,
que atacado de loca alegría se hincó a sus pies besándole las manos.

«¿Es usted\ldots{} ese D. Fernando? Le esperábamos\ldots{} Nos dijo el
padrico que usted nos sacaría\ldots{} Zoilo juraba que no\ldots{} Yo
confiaba en Dios\ldots{} y en usted, D. Fernando de mi alma.»

---Fuerte bromazo, ¿verdad? ¡Cinco meses!

---¡Cinco siglos, señor!\ldots{}

---¿Y qué ha dicho la ley?

---¡La ley\ldots! Esa puerca indecente, ¿qué ha de decir? Aquí han
entrado los ministriles a preguntarnos cosas que no sabíamos, y a
enredarnos en mil trampantojos\ldots{} Tan pronto éramos desertores como
ladrones en cuadrilla. Y papeles van, papeles vienen. Preguntar a
Bilbao, preguntar a Burgos\ldots{} Ya ni sabíamos qué declarar; y si
mentíamos, malo; si decíamos la verdad, peor. Hemos estado en el
infierno antes de morirnos, y bendito sea el ángel de Dios que nos ha
sacado, bendito mil veces.

---Díganme\ldots{} ¿qué ángel sacó al compañero de ustedes, \emph{el
Epístola}?

---Un señor militar que no conocemos. Entró y dijo: «Pertusa, ven,» y
nada más. Nos quedamos solos Arratia y yo.

---¿Y nadie ha mirado por estos dos pobres mártires?

---Por estar padre baldadito, vino un amigo de casa; pero nada pudo
conseguir. Llegó luego D. Sabino, el padre de Zoilo, con un rimero de
cartas para generales, clerigones de acá y de allá, y después de andar
de Herodes a Pilatos, como un loco, se fue en busca de Van-Halen, que
está no sé dónde, y de D. Santos San Miguel, a quien se habrá tragado la
tierra. Un mes hace que D. Sabino se despidió de nosotros, hecho un mar
de lágrimas, diciendo: «volveré pronto,» y esta es la hora que no le
hemos visto. Si usted no nos salva, creo yo que aquí nos habríamos
muerto de rabia y miseria.

Zoilo, en esto, se había puesto en pie con no poca dificultad,
arrimándose a la pared y miraba con espantados ojos a los tres sujetos
allí presentes. No creyó D. Fernando que era ocasión de mayores
explicaciones dentro de aquel insalubre, odioso recinto, y cogiendo a
Zoilo por un brazo, dijo: «Aquí no hacemos nada. Vámonos fuera.» Dejose
llevar el bilbaíno sin proferir palabra. La impresión del aire, la viva
luz de la calle, abatiéronle de tal modo, que no pudo tenerse en pie y
cayó como cuerpo muerto. Urrea y Sabas, que en la puerta aguardaban,
cogiéronle en brazos y le llevaron al alojamiento de su señor, en una de
las mejores casas de la calle principal. Iturbide, ansioso de vivir,
animalizado por el hambre, devoró los primeros alimentos que se le
presentaron. Zoilo fue colocado en el propio lecho de Calpena, donde no
hacía más que dar vueltas, morderse los puños y proferir expresiones
obscuras, que ya parecían rencorosas, ya de piedad o desconsuelo. Gran
parte de la noche, su aspecto y actitud fueron de un animal herido. Cayó
por fin en profundo sopor. Durmiose D. Fernando en la propia estancia,
sobre un duro canapé, y a la madrugada, despertado súbitamente por la
torcedura de cuello y los dolores que su angosto lecho le producía,
sintió rebullir a Zoilo y creyó que lloraba.

Así era, en efecto. Le observó, acercando a su rostro el candil que
había quedado encendido, y en tono campechano, de amistosa reprensión,
le dijo: «Sr. Arratia, paréceme que las tres de la madrugada no es la
hora más propia para llorar. Más cuenta le tendría comer algo, pues
desde que salió de la cárcel no ha entrado en su cuerpo ni un buche de
agua\ldots{} Qué, ¿no me contesta\ldots? Bueno: pues yo me voy a dormir
a otro cuarto, y llore usted todo lo que quiera\ldots{} Mire: sobre
aquella mesa hay un buen trozo de cordero asado que, aunque frío, está
muy sabroso, y pan y vino superior. Elija entre vaciar de lágrimas el
cuerpo, o echarle el sustento que ha menester. Yo no he de ponerme más
gordo ni más flaco por lo que usted coma\ldots{} Qué, ¿no contesta y
vuelve la cara?\ldots{} Pues le aseguro que no tengo ningún interés en
que usted viva\ldots{} Cada uno hace de su vida lo que le place\ldots{}
Bueno: ahí se queda. Yo me voy\ldots»

Ya salía, cuando Zoilo le cogió por el faldón, deteniéndole suavemente,
sin mirarle. De pronto se incorporó, diciendo con voz opaca: «Señor, yo
lloro de rabia\ldots{} de rabia contra mí mismo\ldots{} Sepa usted que
soy hombre de un querer muy fuerte, y cuando quiero una cosa, la quiero
tanto\ldots{} que por la fuerza de mi querer, sucede. ¿Me entiende?»

---Explíquese mejor, amigo.

---Pues libre estoy rabioso, como rabioso estuve preso, porque no me ha
salido la cuenta. Yo quería la libertad; pero quería que me la diese
otro, no usted\ldots{} Y quería que no hiciera caso de la carta que le
escribí\ldots{} Este era mi querer fuerte, fuerte, como todo querer
mío\ldots{} Y luego resultó lo contrario: que no me sacó otro, que me
sacó usted, que hizo caso de mi carta, que se olvidó de nuestras
ofensas\ldots{} y por eso estoy furioso, señor, porque no me gusta
equivocarme, porque no me he equivocado nunca\ldots{} y porque ahora me
encuentro que, siendo usted mi salvador, tengo que quererle, y no
quiero, no quiero\ldots{}

---¡Oh!, eso es mortificarse vanamente, pues a mí me importa poco que
usted me quiera o no. Si le agrada el tenerme rencor, porque así lo
siente, téngalo en buen hora; si piensa que busco el agradecimiento, se
equivoca. A nada está usted obligado conmigo. Y libre queda el hombre
para querer quererme, o para querer lo que más le acomode. Ea, que yo
necesito descansar. Ahí se queda usted con sus quereres y sus rabias, y
puede elegir, a su libérrimo querer, entre la comida que allí tiene y el
comerse sus propios puños. Abur, amigo, y hasta mañana.

Sin añadir una palabra ni esperar respuesta, se retiró D. Fernando a
otra estancia, donde pudo dar algún descanso a sus molidos huesos.

\hypertarget{xii}{%
\chapter{XII}\label{xii}}

Trajo el siguiente día la novedad de que la expedición del Conde de
Negri había entrado en tierra de Burgos, lo que puso en inquietud a
Calpena, por si la guerra turbaba el sosiego de su madre en el apacible
retiro de Medina.

Mas O'Donnell le tranquilizó, asegurándole que las operaciones contra
Negri eran hacia la parte de Belorado y límite de Soria. Desayunándose
con su gente en una estancia baja, que sólo porque comían en ella tenía
derecho al nombre de comedor, le dijo Iturbide: «A ese bruto de Zoilo
hay que dejarle con sus manías, y no pretender meter una razón dentro de
aquella cabeza, que es un sillar redondo, señor, un verdadero sillar que
no tendría precio para rueda de molino\ldots{} Ahora está con la tema de
que el agradecer es carga muy pesada. Para mí no es carga, señor, sino
más bien alas con que uno vuela.

---¿Y qué tal? ¿Ha comido?

---Todo el cordero que allí había, y otro tanto que le llevé yo después.
Come que come, pues una vez en ello no sabe acabar, me decía: «Veré si
con el alimento voy entrando en caja y me sale la gratitud. Es un
compromiso, Pepe, deberle uno la libertad a ese Don Fernando\ldots{}
Nunca creí que yo pudiera ser esclavo de nadie, y ahora lo soy, pues
para mayor pena, hasta nos da de comer. Tengo que ser su amigo, y él
podrá despreciarme si quiere, y hacerme más infeliz de lo que soy.»

Creyendo ver Fernando en la franqueza de Iturbide buena ocasión para
adquirir los anhelados informes de la familia de Arratia, se le llevó de
paseo, y no fue necesario ningún estímulo para que el bilbaíno siempre
locuaz, en aquel caso agradecido, desembuchase cuanto sabía.

«Puedo asegurarle, señor, que Zoilo casó el mismo día o noche de
Luchana, y que sin esperar a la entrada de Espartero se largó a Bermeo
toda la familia con los recién casados\ldots{} ¿Qué dice? ¿Que ya esto
lo sabe? ¿Sabe también que Aura, por soplos de gentuza, se enteró de que
usted vivía y de que fue a Bilbao, trastornándose con la noticia y
poniéndose tan perdida de la cabeza que se escapó, y que más de un mes
estuvieron sin poder encontrarla, y la dieron por muerta, y hasta le
cantaron el funeral?»

---Lo del funeral no lo sabía. Sigue.

---¿Sabe que una vez encontrada, y conducida en coche a Bilbao, ha
sufrido unos rarísimos cambios de humor, un quita y pon de razón y
locura, pues semanas tenía de querer a su marido y hacerle fiestas,
semanas de odiarle y recibirle con las uñas cuando a ella se acercaba?

---De ese tejemaneje de sinrazón y cordura no tenía noticia. Adelante.

---Todas las mujeres son de muy extraña condición; pero esa más que
ninguna. ¿Sabe usted que Zoilo estaba dado a los demonios y no vivía y
se tiraba de los pelos, y que no quedó médico en Bilbao que a la niña no
visitara? ¿Sabe que Zoilo encontró una carta escrita por usted a Doña
Aura, y llevada por \emph{Churi}\ldots{} y que cuando la leyó se puso
más loco que su mujer, y quiso pegar a su padre y a su tío y a todo el
género humano? Pues fue un paso terrible, del cual se enteró todo
Bilbao. El motivo de venir \emph{Luchu} a estas tierras fue como le voy
a contar. Quería buscarle a usted y proponerle, por buena composición,
que se hiciera otra vez el muerto, para que, con el convencimiento de
que el D. Fernando no existía, entrase en razón Doña Aura y pudiese el
matrimonio vivir en paz. Si usted a esta figuración de muerte se
prestaba, de acuerdo con la familia, serían los dos amigos, Arratia y D.
Fernando; si a la farsa saludable no se avenía, no quedaba más remedio
que quitarse de en medio uno de los dos, desafiándose a muerte. Esta era
su idea; pero la familia no quería verle en tales trapisondas y le
estorbaba la salida. Muy terco es él, como usted sabe, y cuando se le
mete una idea en la cabeza, antes muere que dejársela quitar. Su tío
Valentín era el único en la familia que apoyaba el viaje de Zoilo a
Castilla, para que recogiese a \emph{Churi} y le llevase atado codo con
codo. Esto y el aquel de acompañarme a mí, cuando mi padre me mandó a
sacar a mi hermano del Provincial de Segovia, sirvieron de pretexto al
amigo Arratia para ponerse en camino\ldots{} Y sólo me falta decirle que
más allá de Balmaseda nos encontramos a Eustaquio de la Pertusa, con
quien habíamos hecho amistad en Bilbao, estimándole por su agudeza y
buena conformidad. Juntos los tres, \emph{el Epístola} nos sirvió de
mucho para franquear los pasos ocupados por facciosos, pues con ellos
hace buenas migas. Entre paréntesis, diré a usted que Pertusa reparte
papeles impresos con la cantinela de \emph{Paz y fueros netos,} que es
la bandera que sacan ahora los que ya están hartos de guerra y de
Pretendiente absoluto\ldots{} Pues sigo: andando los tres, cada cual con
su objeto, llegamos a Miranda, donde nos pasó lo que usted sabe; que, a
mi cuenta, nuestra prisión y desgracia no tuvieron otro motivo que el
haber venido con Pertusa, hombre muy travieso y fino, que se mete por el
ojo de una aguja, por lo que le anda siempre buscando las vueltas la
policía del General Espartero\ldots{} Ya conoce el señor el milagro a
que debió mi hermanillo la vida en el fusilamiento del 30 de Octubre, y
la conmutación de su pena\ldots{} De los cinco meses de martirio en la
cárcel, nada tengo que decirle, pues anoche le conté cuánto padecimos
hasta que se nos apareció el ángel en forma de D. Fernando, que nos dio
la libertad y la vida. Bendito sea mil veces, y Dios le prospere y haga
dichoso en premio de su grande caridad.

---Ignoraba yo---le dijo Calpena gozoso,---mucho de lo que me has
contado, y con ello se disipan las dudas que me atormentaban. Ya empiezo
a cobrar tu parte de deuda conmigo por la libertad que te di. Si quieres
completar el pago, habla con ese bruto, persuádele a que sea explícito y
franco conmigo, declarándome sin ningún rebozo todo lo que piense y
cuantos propósitos respecto a mí le inspire su terquedad. Los tercos en
ese grado me hacen gracia; digo mal, me cautivan, me entusiasman; creo
que de los tercos indómitos es el reino de la tierra.

Toda aquella tarde estuvo Iturbide trasteando a su amigo y amansándole
el genio, para lo cual, en vista del reparador apetito que se le había
despertado, empleó argumentos de comida exquisita y de vinos superiores,
y la cabeza de luchu recobraba lentamente su facultad pensante, sin
perder nada de su dureza de pedernal. Toda la mañana siguiente estuvo
Calpena en la Comandancia recogiendo noticias de la guerra, sin desechar
las que de política corrían, las unas verosímiles, absurdas las otras.
Véase la muestra: se había descubierto una conspiración civil y militar
para quitar la Regencia a Doña María Cristina y darla\ldots{} ¿a quién,
Señor?, al Infante D. Francisco de Paula. Por lo disparatado y
extravagante, encontró este notición fácil acceso en la mayoría de las
cabezas. Ello debía de ser, en opinión de muchos, un nuevo delirio
masónico. Por otra parte, el moderantismo triunfante, o retroceso,
desataba vientos de discordia. En casi toda la Península se había
declarado el estado de sitio, sin más objeto que perseguir y encarcelar
a los \emph{libres}; la imprenta era toda mordazas; el Ministerio
marchaba francamente por la senda del absolutismo, emulando al Príncipe
rebelde en la estolidez de sus disposiciones tiránicas, y para colmo de
locura, se arrastraba a los pies de Luis Felipe, pidiéndole una
intervención humillante para terminar la guerra, sin obtener más que los
\emph{desdenes de las Tullerías} (así hablaban los que querían
distinguirse por un fino lenguaje). Y en tanto, las dos hermanitas
napolitanas habían reñido, y la Gobernadora, que hasta entonces fiara en
la espada de Espartero como garantía de su causa, comenzaba a recelar
del de Luchana, volviendo sus ojos a Ramón Narváez, como amparador más
seguro y arriscado. Para darle la fuerza material de que carecía, se le
mandó organizar un ejército llamado de reserva, con cifra de cuarenta
mil hombres, y el aparente objeto de perseguir bandidos y facciosos en
las provincias manchegas y andaluzas. De todo esto, que a Miranda
llegaba desfigurado y con más bulto del que realmente tenía, sacaban los
oficiales comidilla y distracción en la tediosa vida del campamento.

De vuelta Fernando en la casona que habitaba, hallose a Iturbide de gran
parola con Arratia en el comedor, frente a un jarro de vino, y con el
pasatiempo de una barajilla sebosa. Soltó Zoilo con desdén las cartas al
ver a su libertador, y brindándole el asiento más próximo, se arrancó al
instante con lo que tenía que decirle, ya muy pensado y medido desde por
la mañana: «Señor, dice Pepe que sea yo franco con usted, y yo digo a
Pepe que más claro he de ser que el agua, pues la claridad está en mi
natural. Con lo que he comido se me ha vuelto a meter la razón en esta
parte de la cabeza donde tiene su hueco, y con la razón y la claridad en
mí, por muy bruto que yo sea, no puedo desconocer que al señor le debo
la libertad y la vida, contra lo que yo deseaba. Pero ante lo que es, no
valen suposiciones ni falsos quereres\ldots{} Hasta hace poco tiempo era
mi voluntad que usted se muriera, y créame que la noticia de su verídica
muerte habría sido mi mayor alegría. Hoy, ya que no puedo desearle la
muerte de verdad, sí quiero que lo sea de figuración, para que mi esposa
se cure de su mal de recuerdo, y perdida la esperanza, se acaben en ella
los arrechuchos lunáticos que son mi desesperación, mi rabia y la mayor
desdicha que puede padecer un marido enamorado.»

---Pero, hombre---le dijo Calpena con jovialidad,---¿cómo quieres que yo
me haga el muerto? Dile a tu mujer que no existo, a ver si te cree.
Corres el peligro de que habiéndola engañado la primera vez, no te crea
en la segunda\ldots{} Pero, en fin, ¿cómo hemos de componer esa falsa
opinión de mi muerte? Explícalo tú.

---Pues, señor\ldots{} o muriéndose de verdad\ldots{} o fingiéndolo,
como en una comedia que vi yo en Bilbao, en la cual uno, que no me
acuerdo cómo se llamaba, salía en el ataúd, y en el propio panteón le
metían, resultando que no estaba sino dormido por la virtud de un
brebaje\ldots{}

---¿Y esas paparruchas de comedia quieres tú que las llevemos a la vida
real? La curación de tu mujer podría costarme cara, y no estoy yo en
disposición de prestarme a esos fingimientos ridículos y peligrosos,
después de lo que padecí con su deslealtad y tu atrevimiento, pues tú no
ignorabas que Aura era mía, y con tu obstinación, ayudada de malas
artes, la engañaste y la hiciste tuya. Ya no te la disputo: puedes estar
tranquilo; pero no he de ayudarte a devolverle la razón, pues no fui yo
quien se la quitó, sino tú.

---Señor---dijo Zoilo levantándose con movimientos difíciles, como quien
sufre desazón y mal gobierno de todos los músculos de un lado,---si me
riñe lo aguanto, porque es mi deber aguantarlo\ldots{} Pero yo no callo
nada de lo que siento, y con toda la verdad de mi corazón declaro que no
hay más que dos caminos para mí: o que usted se muera o que yo me mate,
pues así, créamelo, Zoilo Arratia no puede vivir.

---Yo he cumplido contigo un deber de conciencia, y nada más tengo que
hacer. No quiero yo la vida para jugar con ella imitando lances de
teatro, y mientras estés en mi compañía no he de consentir que te mates.

---Señor, si mi mujer no cura, yo no vivo.

---Tu mujer curará.

---¿Cuánto? Veinte médicos han dicho que no curará mientras sepa que
vive el que me escucha.

---Pues hay otro médico que dirá lo contrario, si le consultas.

---¿Cuál? ¿Dónde está?

---Es el tiempo, bruto.

---¡El tiempo\ldots! Eso dice mi padre. Claro, si viviéramos quinientos
años, puede que para entonces\ldots{}

---El tiempo corre y pasa, y, por tanto, cura, más pronto de lo que tú
crees\ldots{} ¿Qué dices, qué piensas?

---Señor---replicó Zoilo tras larga pausa, en la cual parecía querer
horadar su frente con el dedo índice,---estoy pensando una cosa\ldots{}
Se me ha ocurrido una idea, una gran idea\ldots{} ¿Quiere que se la
diga? Pues pienso que para el caso nuestro, ya que usted no se muera, al
menos, al menos\ldots{} debía casarse. Todo es matar la esperanza.

---¡Casarme! ¿Y es esa la defunción fingida que me propones?\ldots{} No
te digo que no me case algún día\ldots{} ¿Qué estás remusgando ahí? ¿Que
ha de ser pronto? ¡Pues, hombre, no pretendes poco!\ldots{} Todo se ha
de arreglar a tu satisfacción.

---Siempre quiero las cosas con fuerza, con toda mi alma, y por eso lo
que yo quiero es.

---También yo he querido con fuerza, y\ldots{} nada.

---Porque no quiere como es debido\ldots{} Porque usted duda, y sabe
cosas que le hacen dudar más; porque usted no es un bruto del querer.

---Pues ahora quiero una cosa\ldots{} Verdad que es fácil. Pero aunque
fuera difícil se haría. Mañana nos vamos. ¡Oído! Que todo el mundo se
prepare. Os llevaré a Vitoria, donde me has dicho que está tu padre.

Aseguró Iturbide que, por unos alaveses llegados aquella mañana, se
sabía que el señor D. Sabino había salido de Vitoria en busca de su
grande amigo el general carlista Guergué. Mandó D. Fernando a Sabas a la
Comandancia para que se informase del paradero del tal cabecilla, pues
el bien montado espionaje daba diariamente noticia de los movimientos
del enemigo, y la respuesta no tardó en venir: Guergué estaba en
Peñacerrada. Al pronto no se hizo cargo D. Fernando de la situación de
esta villa, cuyo nombre hirió sus oídos como lugar conocido; pero Sabas
le sacó de dudas diciendo: «Está entre La Guardia y el condado de
Treviño.»

---Pues por esa parte---dijo D. Fernando con nervioso susto, más bien
desgana, que no pudo disimular---irán ustedes, yo no.

---¿Lo ve, lo ve?---ritó prontamente Zoilo gesticulando con ardor.---No
sabe querer\ldots{} ¡A La Guardia, señor!\ldots{} Lo quiero con toda mi
alma. Lo quiero, lo quiero, y como no vayamos todos allí, me estrello la
cabeza contra la pared.

---Eres un bárbaro\ldots{} ¿Y qué fundamento, dímelo, qué razón tienes
para ese querer tan vivo?\ldots{}

---¡A Peñacerrada y La Guardia!

---¿Crees que encontrarás a tu padre?\ldots{} ¿Y si antes de dar con él
dan con nosotros los carlistas, y nos prenden o nos matan?

---Usted teme, usted no sabe querer.

---Hombre, es que\ldots{}

---El que quiere con fuerza no teme.

---Está bien. Pero supongamos\ldots{}

---El que quiere con fuerza no supone nada: va derecho a su fin\ldots{}
A La Guardia, señor\ldots{}

---¿Por qué ese empeño en que vayamos a La Guardia?

---Señor, porque allí está su novia.

\hypertarget{xiii}{%
\chapter{XIII}\label{xiii}}

Festivo y locuaz estuvo Calpena el resto de la tarde, tirando de la
lengua al bruto de Zoilo para gozar con sus extravagantes teorías del
querer fuerte, y reunidos en el llamado comedor, bebieron y jugaron con
discreta fraternidad amo y criados y amigos, guardando cada cual su
puesto en las alegrías de aquella igualdad temporal. Como llegaran
nuevas referencias del paradero de Guergué, dándole por internado en el
Condado de Treviño, resurgieron las dudas acerca del punto adonde se
dirigirían. Iturbide se mostraba temeroso, Zoilo aferrado a su violento
querer, y al fin propuso Fernando que decidiera la suerte,
comprometiéndose todos a la obediencia de lo que el misterio de la
fatalidad les señalara. El arduo caso fue sometido al fallo de
\emph{cara o cruz}, encargándose Zoilo, como el más inocente de la
cuadrilla, de arrojar al aire la moneda, previa designación de La
Guardia por la figura y Treviño por la cruz. Salió esta, y nadie se
atrevió a manifestar oposición a tan grave sentencia. Los medrosos y los
arrojados ocupáronse con igual ardor en los preparativos para la
caminata del siguiente día, que emprendida fue sin tropiezo al despuntar
de la aurora, por el camino real de la Puebla.

Buenos caballos adquirió Fernando para los dos bilbaínos; pero Iturbide,
que se había pasado la vida, primero en su oficio de fabricar poleas,
después en el servicio militar de infantería, no era un prodigio en la
equitación, y su impericia daba lugar a cada instante a lances muy
graciosos. A Zoilo, regular jinete, no le permitía su debilidad
mantenerse en la silla con todo el garbo que él deseara. No habían
andado dos leguas, cuando encontraron un destacamento de tropas que
salió de Miranda la noche anterior. El capitán que lo mandaba les dijo:
«¿Pero están ustedes locos? ¿A dónde demonios van?» De los informes
resultó que todo el Condado hervía de facciosos, que las comunicaciones
con Vitoria estaban interrumpidas, que en Peñacerrada habían acumulado
mucha fuerza, fortificando todas las alturas. Lo mejor que podían hacer
los caminantes era volverse a Miranda, o tirar para Salinas, aunque por
este punto también había peligro.

Pasados los primeros minutos de perplejidad, manifestáronse dos
opiniones: en la boca de D. Fernando, valeroso y prudente, la de seguir
el juicioso consejo del Capitán; en la de Zoilo, que era la temeridad
irreflexiva, la de marchar hacia adelante, obedientes al oráculo de la
moneda arrojada al aire. Seguramente prevalecería la voluntad del que
era señor y amparo de todos, en quien el sentimiento del deber y la
responsabilidad de las ajenas vidas se aunaban. Apartándose del camino,
echaron pie a tierra para descansar y tomar alimento, al pie de unos
álamos que ya se vestían de su hoja nueva, y eran como apacible tienda
de sombra y frescura. Allí se repusieron, y no habían concluido de matar
el hambre, cuando vieron venir una partida de aldeanos de ambos sexos,
en borricos y a pie, como gente presurosa o fugitiva.

---Paisanos, ¿qué ocurre\ldots?---les preguntó Sabas saliéndoles al
encuentro.---¿Hay olor de facciosos por esta parte?

---Olor no, sino peste de ellos---replicó un viejo ladino que montaba el
burro delantero.---Somos de Berganzo, y de allí nos ha echado el
\emph{asoluto}, después de quemarnos el pueblo. \emph{Asolación} mayor
no se ha visto.

---¿Hacia la parte de Samaniego, ocurre algo?

---En Samaniego---chilló una mujer, que con dos niños en brazos montaba
el segundo borrico,---no han dejado esos perros ni cántara de vino, ni
doncella, ni nada.

---¿Qué sabéis de La Guardia?

---Que anoche, \emph{dende} Toloño, se veían las llamas de la villa,
ardiendo por los cuatro costados\ldots{} En Peñacerrada han metido los
\emph{carlinos} sin fin de tropa, y han puesto cañones en el castillo,
cañones en Larrea\ldots{} No es mal hueso el que arman allí. Díganme,
señores: ¿vendrá D. Espartero a roerlo? Porque si no viene, y pronto,
¡pobre Rioja alavesa!\ldots{} Dios nos tenga de su mano. Ea, caballeros,
que tenemos prisa para llegar a Miranda, pues de atrás no vendrá cosa
buena. Hace un cuarto de hora, al rebasar de Berantevilla, oímos ruido
de zalagarda\ldots{} ¡Hala, que es tarde!\ldots{} abran calle\ldots{}
Agur, y viva \emph{la Isabel}\ldots{}

Apenas se alejó, buscando el camino real, la medrosa caravana, miraron
todos el rostro de D. Fernando, que, poniendo corto espacio entre la
duda y la afirmación, resolvió de plano con firmeza y aplomo.
«Amigos---dijo,---avancemos por el rastro de esa pobre gente, y tal vez
hallaremos otros fugitivos a quienes podamos prestar socorro.»

Con gallarda confianza respondieron los cuatro a tan airosa
determinación, y Zoilo se lanzó delante, gritando: «¿Ve usted, señor,
cómo sale lo que yo quería? Mi querer fuerte apuntó para La Guardia, y a
La Guardia vamos. ¡Marchen! No puede pasarnos cosa mala.» Media legua
más allá encontraron nuevos grupos que confirmaban las alarmantes
noticias del primero, con alguna variación, pues el pueblo que desde
Toloño se había visto arder no era La Guardia, sino Páganos. Cada cual
agregaba nuevos horrores dictados por el miedo. Halló Sabas gente
conocida; le daba en la nariz el tufo de su tierra, oliendo a quemado, y
el hombre no vivía; habría querido ir de un vuelo, y ver y apreciar la
extensión del desastre. Las últimas noticias recogidas a media tarde
eran que los \emph{absolutos} habían pasado la sierra de Toloño; que
casi todos los habitantes de La Guardia habían huido, pasando el Ebro
por el vado de Cenicero, no sin peligro, pues también rondaban partidas
por aquella parte; que Peñacerrada era un infierno de fortificaciones;
que\ldots{} en fin, que se acababa el mundo, y que nos encontraríamos
todos en el valle de Josafat.

Sin perder sus bríos ante tales demostraciones de pánico, siguieron su
marcha, y a la caída de la tarde, Sabas descubrió dos aldeanos de
Samaniego, el uno pariente suyo, por quien tuvieron más claros informes
de lo que vivamente les interesaba. Aterradas por el incendio de
Páganos, escaparon de La Guardia todas las familias pudientes que no
pertenecían a la opinión \emph{servil}. Las niñas de Castro y Doña María
Tirgo, formando caravana con las de Álava, no fueron de las últimas en
la escapatoria; mas ignoraba el informante si corrían hacia el Ebro,
pues algunos que tomaron aquella dirección habían regresado desde El
Ciego, huyendo de una partida. Era lo más probable que hubieran tratado
de escabullirse hacia San Vicente de la Sonsierra, para buscar el vado y
pasar a Briones\ldots{} Mientras más embarulladas y contradictorias eran
las noticias que recibían, más se confirmaban los cinco expedicionarios
en la resolución de ir adelante, movidos simultáneamente de un generoso
impulso que no sabían definir. Era la voz del destino que aquella
dirección les marcaba, impeliéndoles hacia un fin favorable o adverso,
hacia el cual corrían como las mariposas hacia la luz.

Anduvieron hasta el anochecer en medio de una gran desolación. La tarde
estaba serena, el cielo transparente y limpio, como un rostro que
quisiera expresar la absoluta indiferencia de toda cosa humana\ldots{}
Hablaban poco; tan pronto iba Zoilo delante, tan pronto a retaguardia,
canturriando entre dientes, erguido sobre el caballo, y olfateaba el
horizonte, curado ya como por ensalmo de aquel torcedor doloroso de su
cuerpo. A sus espaldas se puso el sol, y ellos, picando siempre hacia
Levante, que con los reflejos del sol poniente se tiñó de resplandores
opalinos, luego de un gris violáceo muy puro y uniforme en suave
gradación. Sobre esta densa cortina se fue destacando un astro rojo:
Marte. La noche entró tenebrosa, sin otra claridad que la de las
estrellas. Víspera de luna nueva, el disco de la luna había precedido al
sol en el ocaso. De pronto, al descender de una loma, vieron los jinetes
frente a sí siniestra claridad rojiza que se difundía en el morado
intenso del cielo. Era la cabellera de un incendio. Detenidos por un
solo impulso, los cinco dijeron a una voz: «Un pueblo que arde.»

Conocedor del terreno, Sabas examinó con experta vista el horizonte. «No
puedo calcular la distancia del fuego---dijo;---pero si está a dos
leguas, no puede ser más que Berganzo; si está más lejos, será
Peñacerrada.»

Y D. Fernando: «Sea lo que fuere, adelante. El que tenga miedo, que se
vuelva.»

Nadie pronunció palabra, y Zoilo se puso nuevamente a vanguardia,
alejándose buen trecho del grupo principal. El fuego parecía crecer:
ráfagas de viento Sur desmelenaban el resplandor hacia el Norte. De
pronto vieron los caminantes que Zoilo se detenía: picando para llegar
pronto a donde él estaba, oyéronle decir: «Viene gente armada.» Aguzaron
todos el oído, imponiendo silencio; pero no percibieron ningún rumor;
mas Zoilo insistía en que había sentido algazara de tropa. Afirmó que
nadie le ganaba en fineza de tímpano, así como en alcance de vista,
teniendo además la cualidad de ver en las tinieblas, como los gatos.
Adelantose otra vez, y volvió asegurando que estaban próximos a un
pueblo, que él veía paredes negras y una torre, y que oía run-run de
gente. No supo Sabas determinar qué aldea o villorrio caía por aquellas
soledades, y habló de una gran casa de labor o alquería del marquesado
de Zambrana. Fuera lo que fuese, a los pocos pasos confirmaron todos lo
anunciado por Arratia, pues ya se hallaban a medio tiro de fusil de unas
tapias altísimas, y no tardaron en oír claramente voces humanas.

«La Santísima Virgen nos ampare---murmuró Iturbide.---Como esta es
noche, hemos caído en una trampa facciosa.»

Detuviéronse los cinco por cesación súbita, pavorosa, del impulso
interno que hasta allí les había llevado. Transcurridos algunos
segundos, que horas parecieron, dijo D. Fernando: «Si estamos cogidos,
sepamos por quien; que no hay suplicio como la incertidumbre.» Y aún no
había concluido de decirlo, cuando una robusta voz estalló en la
obscuridad, gritando: «¿Quién vive?» Y en el mismo instante se oyeron
las voces: «¡Alto, alto!» A la repetición estentórea del \emph{¿quién
vive?} respondió D. Fernando con toda la fuerza de sus pulmones:
«¡España!» De las tinieblas surgieron varios hombres con los fusiles
preparados. Su aspecto no era de tropas regulares, pues vestían con
desiguales prendas y arreos, y llevaban gorra de piel los unos, los
otros boina blanca o roja. Adelantose uno diciendo: «Alto, y se les
reconocerá. ¡Viva Isabel II!» A este grito, que ponía fin a la ansiedad
de aquel encuentro, los caminantes, gozosos, libres ya de su mortal
sobresalto, respondieron con otro ¡viva! en que echaron toda el
alma\ldots{} Breve y satisfactorio fue el primer reconocimiento; pero
les mandaron no dar un paso más hasta que llegase el capitán. Salió por
fin este, repitiendo las preguntas de ordenanza; cumplidamente las
satisfizo Calpena, que a su vez se permitió interrogar: «¿Qué fuerza es
esta, mi capitán?

---Es la columna que mando yo, Santiago Ibero. Pertenecemos a la
división de D. Martín Zurbano.

Y cuando esto decía, fue reconocido por Sabas, que prorrumpió en
exclamaciones de gozo: «¡D. Santiago\ldots{} Santiago Ibero!

---¿Eres de La Guardia?

---De Páganos, para servirle, y usted también. ¿Pero no conoce a Sabas
de Pedro?

---¡Otra! ¿Eres tú\ldots? Adelante, señores\ldots{} ¿Traen comida?
Apéense en este corralón. Entremos y hablemos y comamos\ldots{}

El júbilo de los expedicionarios por verse entre amigos era tan grande,
que no podían expresarlo sino con risas, gritos y exclamaciones
patrióticas. Enterados de que la partida andaba mal de víveres, mandó D.
Fernando a Urrea que franquease todo el repuesto que llevaban, y la
alegría se hizo general. Entraron en un lagar desmantelado, al que
seguían cuadras espaciosas, reconociendo Sabas la casa labrantía de
Zambrana. Mientras acomodaba las bestias y les daba pienso, Urrea iba
distribuyendo pan, queso y vino a la tropa en el corralón. Ibero y D.
Fernando, antes de ponerse a comer, departieron largamente, diciendo el
primero: «También a usted le reconozco. Es usted D. Fernando, el
caballero que trajo de Oñate a las niñas de Castro, y que luego, herido
en un pie, pasó una larga temporada en casa.» Nombrada la familia, no se
hartaba Calpena de pedir informes acerca de ella, y el otro los dio con
mil amores. La Guardia no había caído en poder de los carlistas; pero se
temía que la ocupasen por ser muy débil la guarnición. Las familias
ricas habían salido, siendo de las primeras las niñas de Castro con Doña
María Tirgo y las de Álava. Bien podía el informante dar fe de la feliz
escapatoria, pues él con su gente habíales acompañado hasta el paso del
Ebro, y pudo enterarse de que sin novedad llegaron a Fuenmayor. Doña
María Tirgo, muerta de miedo, proponía que no parasen hasta Cintruénigo;
pero Demetria opinaba que no debían pasar de Logroño, donde estarían
bien seguras.

Era Santiago Ibero un mozo gallardísimo, franco, con toda el alma en los
ojos y el corazón en los labios, cetrino, de mirada ardiente. Nacido en
Páganos de una familia de labradores acomodados, su genio impetuoso, su
ansia de gloria, más potentes que toda razón de conveniencia, habíanle
lanzado a la campaña, antes que por querencia de la profesión militar,
por su amor ardentísimo a las ideas representadas en la bandera de
Isabel. Quería dar su sangre, su vida por la libertad y el progreso, en
los cuales veía fuente inagotable de dichas para la Nación. Con tales
beneficios, España saldría de su apocamiento y pobreza, mejorarían las
costumbres, nos veríamos tan civilizados como los ingleses y tudescos, y
seríamos fuertes, grandes, sabios y ricos. Odiaba el obscurantismo, y
veía en la hipocresía farisaica de los partidarios de D. Carlos la causa
de todos los males que nos afligen y del atraso en que vivimos. Al
exterminio de esta secta nefanda quería consagrar su existencia, todas
las energías de su alma honrada y valerosa. Habiendo visto en Martín
Zurbano, a quien conoció en Logroño, la más feliz encarnación de
aquellas ideas, y admirando en él, además, el coraje, la perseverancia,
la militar pericia, se afilió con entusiasmo en su bandera. Con él
peleaba, y con él moriría, si necesario fuese, por la santa causa de los
\emph{libres}, que era el porvenir glorioso de la Monarquía y de España.

A la media hora de charla, ya eran amigos Ibero y D. Fernando, y este
tuvo conocimiento de la situación de la columna. Los carlistas se habían
apoderado de Peñacerrada, que por su posición topográfica en terreno
montuoso era una fortaleza natural. Fortificados también otros puntos de
la sierra, ocupados pueblos importantes del Condado, quedaba
interrumpida la comunicación de Vitoria con las líneas del Ebro. La
situación era, pues, gravísima, y si no venía Espartero con fuerza
grande a desatar el nudo, sabe Dios lo que sucedería. Según las noticias
del capitán, D. Baldomero se preparaba, y en tanto había mandado al
general Ribero a la parte de Nanclares, mientras D. Martín, en la Rioja
alavesa, molestaba al enemigo todo lo que podía, quitándole raciones y
amparando a los pueblos. Con este fin, ordenó a Ibero que con su columna
limpiase de facciosos los caseríos de la sierra de Toloño, y en ello se
vio el capitán muy comprometido, pues atacado por fuerzas superiores,
había tenido que batirse a la desesperada. Intentaba retroceder hacia la
Rioja alavesa, para reunirse con su jefe; mas no tenía seguridades de
poder conseguirlo. Hallando a su paso en la tarde de aquel día la casa
de labor de Zambrana, en ella se hizo fuerte, con el propósito de
defenderse bien si alguna partida le atacaba. En caso de gran apuro, y
si veía dificultades para retroceder hacia La Bastida, trataría de pasar
el Ebro por el vado de Ircio.

En tanto que Ibero y D. Fernando se comunicaban sus planes y
pensamientos, Iturbide y Zoilo no se apartaban de los de tropa, comiendo
con ellos, contándoles peripecias del sitio de Bilbao, a cambio de las
recientes hazañas de los \emph{zurbanistas}, referidas, la verdad sea
dicha, con disculpable uso de la hipérbole. Aquella tarde se habían
peleado heroicamente con doble número de \emph{serviles}, matándoles al
jefe y cogiéndoles quince prisioneros. Luego tuvieron la desgracia de
que en otro encuentro, en la misma tarde, perdieran ellos tres hombres,
lo que no sintieron tanto como el que se les escaparan los quince
cautivos cuando se disponían a fusilarles, en castigo de su amor al
\emph{retroceso}. Aquel segundo combate había quedado indeciso, sin
grandes ventajas de una parte y otra, perdiendo el contrario dos burros
cargados de cebada, y ellos los prisioneros, que \emph{fue un gran
dolor}. Si se les hubiera quitado de en medio en cuanto fueron cogidos,
no se habrían ido riendo\ldots{} Pero, en fin, como hay Providencia, no
debía desesperarse de volver a cogerles.

A media noche, unos dormían en grupos tendidos en el suelo, otros hacían
guardias en los ángulos exteriores del caserón, y los mejores escuchas
de la partida aplicaban la oreja al suelo, en observación de los ruidos
lejanos. Ibero y D. Fernando se tumbaron en el sitio que mejor les
pareció de la anchurosa cuadra primera; pero el capitán no tenía
sosiego, y de rato en rato se levantaba para dar vueltas por el corralón
y asomarse a las bardas de este, sin poder desechar el presentimiento de
que antes del amanecer le atacarían, con refuerzos, los que en la
funcioncilla última de la tarde habían quedado a media paliza y con
ganas de llevársela entera.

Durmiose en las alternativas de estos temores D. Fernando, teniendo
junto a sí a Urrea y a Sabas, y aún era muy incierta la claridad del
nuevo día, cuando le despertó un rumor vivo, compuesto de voces
corajudas y guerreras. Los facciosos venían, se aproximaban\ldots{}
Silencio, calma, y prepararse todo el mundo.

\hypertarget{xiv}{%
\chapter{XIV}\label{xiv}}

Brincando entró Zoilo en la cuadra, y dijo al capitán: «Denos fusiles,
jinojo, si los tiene, y si no los tiene, déjenos ir a quitárselos a esos
danzantes.» Fusiles había, los quince de los prisioneros fugados, y al
punto dispuso Ibero armar a los dos bilbaínos. «A mí también---dijo D.
Fernando,---y a mis dos escuderos, que no vamos a estar aquí con las
manos cruzadas.» Para todos hubo armas y cartuchos. «Calma, no
atropellarse---repetía el valiente Ibero.---Aunque sean más de mil, no
nos copan, y aún permitirá Dios que se dejen aquí los dientes. Cerrar
todo bien, amontonando en el portalón del camino las piedras que mandé
preparar esta noche, para que no puedan abrirlo. Cerrar también,
dejándola sin parapetar, en disposición de ser abierta, la portalada del
corralón de la noria, queda al campo por nuestra derecha\ldots{} Ya
saben los de la buena puntería que su puesto es arriba, en las ventanas
del pajar que dominan el campo. Fuego sostenido, y mucho ojo,
amigos\ldots Ya saben los ligeros dónde han de situarse: en el corralón
de la noria. Si en la entrada por el camino ponemos piedras, en la otra
parte pondremos carne, para que esta carne me haga una salidita cuando
yo lo ordene. Calma, y fijarse bien en lo que mando\ldots{} Ahora todo
el mundo a su puesto, y apagar luces: hagámonos los dormidos para que
vengan confiados y se dejen abrasar como borregos.»

---Yo me voy con los ligeros---dijo Zoilo,---si el capitán no me manda
otra cosa.

---Y yo con los tiradores---añadió D. Fernando,---pues no es del todo
mala mi puntería. Amigo Ibero, ponga usted en el mejor sitio a mi criado
Urrea, que es gran cazador: al enemigo a quien este eche el ojo, pronto
le verá usted patas arriba. Sabas, ¿tú qué tal tiras? Vente conmigo.

Antes de que D. Fernando y los suyos llegaran al ventanucho en que les
colocó Ibero, ya empezaban los sitiadores a tirar coces a la puerta.
Desde el pajar se les contestó con vivo fuego. Los ligeros, trepando a
la noria, disparaban también sin abandonar el cuidado del portalón.
Ibero recorría los puestos, y tan pronto estaba en el segundo corral
animando a los chicos, como subía para cuidar de que el servicio de
cartuchos se hiciera con prontitud. Sereno en medio del combate, a todos
infundía su valor y confianza. Arreció el fuego desde fuera contra los
huecos del pajar, y el capitán ordenó a los suyos que aprovechasen bien
los tiros, afinando la puntería. Los estragos de la de Urrea se
apreciaban fácilmente viendo cómo se clareaban los grupos enemigos y
oyendo sus vociferaciones; D. Fernando afinaba también, y Sabas, que no
se creía con bastantes ánimos para afrontar el tiroteo, fue destinado
prudentemente al servicio auxiliar de los diestros cazadores. Con doble
juego de fusiles, Sabas y un viejo de la partida cargaban mientras
aquellos, el fusil en la cara, aseguraban con ojo certero la pieza.

Fiados en su número, los sitiadores, que ninguna ventaja adquirían con
el ataque de fusilería, intentaron el asalto, trepando por la parte más
accesible de la tapia. Ibero, que les había calado la intención, bajó
presuroso, después de dar órdenes arriba para arreciar el fuego,
abrasando a los asaltantes todo lo que se pudiera; y sin cuidarse de que
diez o quince penetraran en el patio, dispuso la salida por la portalada
del corral de la noria. Ello se hizo con rapidez y bravura. Como unos
treinta hombres se lanzaron fuera, y la emprendieron a bayonetazos o a
navaja limpia con los sitiadores, sorprendiéndoles y aterrorizándoles de
tal modo en su impetuoso arranque, que con la sola pérdida de tres de
los suyos escabecharon cuádruple número de los contrarios, y a los demás
les impelieron a la fuga. Obedeciendo como máquinas la orden de Ibero,
volviéronse adentro, después de causar el efecto que se proponían, y
atrancaron la puerta con piedras y troncos y cuanto hubieron a mano. De
los que habían saltado, algunos quedaron dentro sin vida, otros lograron
salvarse, y a poco se oyó una voz ronca y frenética que gritaba: «Ibero,
volveremos\ldots» Levantado el sitio, los de arriba vieron al enemigo
retirarse, llevándose sus heridos. Como a cien pasos, dispararon de
nuevo en descarga cerrada; mas Ibero mandó que no se les contestase,
gritando a los fugitivos: «Animales, gastad cartuchos, gastadlos, que yo
reservo los míos para cuando volváis.»

Gozosos celebraban su victoria, y Zoilo parecía demente, del júbilo que
le embargaba, no vacilando en relatar él mismo sus hazañas con infantil
orgullo. Sin la obligación de acatar al jefe, que había mandado a los
ligeros volverse después de la primera embestida, él se habría traído la
cabeza de un faccioso, a quien ya tenía cogido en excelente disposición
para decapitarlo. Reconocido el campo, encontraron dos heridos graves,
que recogieron, y tres muertos propios. Los enemigos eran catorce, que
abandonaron sin cuidarse de darles sepultura. Descansando de la
refriega, elogió Ibero la destreza inaudita de Urrea y la de D.
Fernando. Iturbide se había portado bien entre los ligeros, y Zoilo, al
decir de todos, con extraordinaria bizarría y temeridad. Pronto surgió
en la mente del jefe de la columna el grave problema de la resolución
que debía tomar. ¿Se fortificaban en aquella excelente posición,
aguardando tranquilos las embestidas del faccioso, que de seguro no
tardaría en recalar con mayor fuerza? La solidez del edificio y la
bravura de su gente, reforzada con cinco números, de los cuales tres por
lo menos eran de gran precio, le garantizaban una defensa gloriosa; pero
si la situación se prolongaba, como era de temer, ¿de dónde sacaría
municiones y víveres?

Dificultosa era la salida; pero con todos sus riesgos, les ofrecía menos
probabilidades de una perdición segura. Marchando hacia Miranda, era
menos probable el encuentro de una considerable fuerza facciosa;
marchando hacia el Este, este peligro acrecía, mas lo compensaba la
contingencia ventajosa de encontrar el grueso de la división de D.
Martín. Encaminarse al Ebro para vadearlo y pasar a la Rioja le parecía
desairado: era el recurso último; era imitar a las mujeres y a los
pobres viejos aldeanos que huían de sus hogares. Oír quiso la opinión de
Don Fernando, en quien reconocía un juicio claro y sereno de todas las
cosas, y el caballero, que tan gallardamente había sabido conquistar su
amistad, no titubeó en darle este terminante voto: «Yo que usted, iría
en busca de la peor y de la mejor contingencia, que las dos se le
ofrecen por el lado de Oriente: batirme a la desesperada con fuerzas
superiores, o encontrar el amparo de la división de mi jefe. ¿Quién le
dice a usted que D. Martín, sabedor o sospechoso del conflicto en que
usted se halla, no viene en su socorro?» Esta última razón llevó tal luz
a la mente de Ibero, que ya no hubo más dudas. «Nos vamos ahora
mismo---dijo,---apartándonos del llano, y metiéndonos en las
fragosidades de la sierra de Toloño. Por allí no nos buscarán. Salgamos
sin ruido, en secciones, que no han de perderse de vista.

A la media hora ya estaban en marcha, confiados en su buena estrella,
Ibero fortalecido por su fe ciega en el ideal de \emph{los libres}, que
creía obra de Dios. Aunque odiaba el fanatismo, era creyente y buen
cristiano; y lejos de ver incompatibilidad entre la libertad y el dogma,
teníalos por amigos excelentes, y por amparadores de la Causa, a todos
los santos de la Corte celestial. Grandes fatigas y trabajos sufrieron
en su larga caminata por la falda de la sierra, describiendo curvas
extravagantes para huir de los puntos que suponían ocupados por
destacamentos carlistas. El tiempo se les torció al segundo día,
metiéndose en agua, encharcando la tierra, y convirtiendo en torrentes
las cañadas que descendían de los montes; mas no conceptuaron por muy
desfavorable el temporal, fuera de las molestias que ocasionaba, porque
el continuo llover era como una cortina del cielo que les ocultaba en su
marcha sigilosa, y la humedad del suelo, si a ellos les estorbaba,
quizás en mayor grado entorpecería los pasos del enemigo. En cuatro días
de marcha penosa no tuvieron ningún mal encuentro; al quinto toparon con
una partida inferior en número, que batieron sin dificultad, y el
peligro de que tras ella vendría mayor fuerza, lo sortearon
escabulléndose en dirección contraria a la que habían seguido los
derrotados.

Consumidos los escasos víveres que sacado habían de su fortaleza,
empezaron a sufrir terribles hambres. Merodeaban en los abandonados
plantíos; algunos cazaban; mas los conejos parecían huir también de la
guerra, como su enemigo el hombre. Erizos y otras alimañas encontraron
en la espesura del monte; en una aldehuela miserable, sólo habitada por
cuatro mujeres y dos vejetes, entraron a saco, arramblando por todo lo
que en aquellas pobres viviendas había, algunos panes, cecina y alubias.
Dos cabras fueron después gran hallazgo, y mejor aún unas alforjas
perdidas, con el tesoro de cuatro quesos y algunas cebollas. Con tales
apuros iban viviendo, marchando de noche, ocultos y dispersos de día,
hasta que, sabedores por sus avanzadas de que en una paridera próxima a
Peciña descansaban veinte facciosos, cayeron sobre ellos de madrugada, y
sorprendiéndoles dormidos, a unos mataron, dispersaron a otros,
quitándoles todo lo que tenían. El único que entre ellos quedó
prisionero, con un brazo roto, les dijo que D. Martín, después de dar un
achuchón a los carlistas cerca de Avalos, se había corrido a Leza,
internándose después en la Sonsierra. Arrimados a las asperezas del
monte, siguieron su camino en busca de Zurbano; y por el afán de avanzar
todo lo posible, anduvieron largo trecho en una noche tempestuosa, con
horrísono tronar y golpes de granizo, viendo caer rayos y alumbrarse
toda la tierra con siniestros resplandores. Pero sus templados
corazones, insensibles al miedo, querían ampararse de los accidentes
espantables de la Naturaleza, para recorrer mayor espacio, prefiriendo
los senderos escabrosos e inaccesibles. Por último, más arriba de Leza,
les deparó Dios una columna cristina de tropas regulares, perteneciente
a la división del General Buerens. Estaban salvados.

Provistos de municiones, pues las pocas que llevaban se les habían
inutilizado con la humedad; reparados sus míseros cuerpos con alimento
sano, aunque no muy abundante, y adquirido informe verdadero de la
situación de D. Martín, siguieron en su busca, y al caer de una plácida
tarde le hallaron en un desfiladero por donde pasa el camino de
herradura entre La Guardia y Pipaón. ¡Feliz encuentro, a los doce días
de haber salido de Zambrana, realizando una prodigiosa marcha por país
enemigo! Aunque el mérito de esta no se le ocultaba, Zurbano recibió a
Ibero con una fuerte chillería, pues era su condición mostrar rigor y
displicencia en todo asunto del servicio, sin duda por hacerse respetar
y temer de sus subordinados. Según decía, si hubiera seguido Ibero
puntualmente sus instrucciones, no alejándose de La Bastida más que lo
preciso para picar la retaguardia a la partida del \emph{Zurdo}, no le
habrían pasado tantas desventuras. ¡De buena había escapado! En fin, a
olvidar los desastres, y a repararlos sacudiendo al enemigo todo lo que
se pudiera.

Era Martín Zurbano (a quien se le despegaba el Don postizo) un hombre
tosco y desapacible, de rostro aclerigado, ceño adusto, boca fruncida,
de regular estatura y lentitud parsimoniosa en sus movimientos. Usaba
boina blanca y chaquetón forrado de pieles sin ninguna insignia; sable y
pistolas al cinto. Hablaba incorrectamente y con acento duro, erizado de
interjecciones, lenguaje del valor de aquel tiempo en la milicia
montaraz. A pesar de estas asperezas, y quizás porque en ellas veía la
perfecta imagen del Marte español, Ibero sentía por él amor y
entusiasmo; y aunque sirviendo a sus órdenes quería imitarle en la
rudeza de los modales y en las groseras voces, no siempre lograba el
objeto, pues más que su proselitismo podían su nativa delicadeza y buena
educación. El felicísimo encuentro con Don Martín no les proporcionó
ningún descanso, pues lo mismo fue llegar y juntarse y recibir Ibero la
peluca de su jefe, que se pusieron todos en marcha. No era muy
satisfactoria la situación de los cinco caminantes agregados a la
partida, pues Iturbide iba en estado febril, tendido en un carro; a
Sabas le había salido un grano en el muslo; Zoilo tenía el pescuezo
torcido de una fuerte tortícolis. Los mejor librados eran Calpena, que
padecía extenuación nerviosa por la falta de sueño, y Urrea, que sólo se
quejaba de ganas de comer no satisfechas.

La tremenda contrariedad de no poder comunicarse con su madre puso a D.
Fernando en gran tristeza. Cogido en la trampa de un ejército en
operaciones, tenía que permanecer entre las fuerzas cristinas, pues por
una parte y otra el enemigo ocupaba montes, villas y lugares.
Arriesgadísimo, por no decir imposible, era volver a Miranda con sus
cuatro compañeros, o pasar el Ebro para refugiarse en Logroño, y no
había más remedio que esperar el despejo de la situación y el término
feliz o adverso de aquella campaña. Por todo el camino, en la marcha
fatigosa, no cesaba de pensar que Dios no le había sido hasta entonces
propicio en su expedición, quizás por haber emprendido esta sin lógica
ni criterio, dejándose llevar de las corazonadas del insensato Zoilo,
quizás de inexplicables querencias suyas, que él mismo no sabía definir.
Y llegado a tal punto de confusión, como el que se pierde en un
laberinto sin encontrar salida, no hacía más que interrogarse de este
modo: «¿Y yo a dónde voy? ¿Por qué he venido aquí? ¿Volveré a ver a mi
madre, a mi querido capellán, a mis entrañables amigos de Villarcayo?
¿Habrá dispuesto Dios que deje yo aquí mis pobres huesos? ¿Tendré que
hacer el héroe por fuerza para llegar a serlo de verdad? ¿Es ley
constante que las acciones muy estudiadas y previstas resultan siempre
bien? ¿Es seguro que los actos de impremeditación y de temeridad,
comúnmente tenidos por locuras o necedades, enderezan siempre al mal?
¿Qué caminos llevan a la vida, qué veredas llevan a la muerte? ¿Toda
senda tenebrosa conduce al Infierno? ¿Toda senda iluminada y florida
conduce al Cielo?\ldots{} Si yo tuviese aquí a mi madre para que me
ilustrara en estas dudas, mi tristeza no sería tan honda. Ya que no la
tengo, traeré su pensamiento al mío, y con esta luz veré lo que solo no
veo: la esperanza. Adelante, y sea lo que Dios quiera.»

\hypertarget{xv}{%
\chapter{XV}\label{xv}}

Llegados, entrada la noche, a media legua de Pipaón, pueblo
perteneciente a la hermandad de Peñacerrada (que \emph{hermandades} y
\emph{cuadrillas} son allí las divisiones territoriales), hizo alto la
columna al amparo de unas casas destruidas, y D. Fernando descansó junto
a su amigo Ibero, el cual le dijo que D. Martín tenía órdenes de
destruir, o molestar por lo menos, a todas las columnas carlistas que
llevaran provisiones a Peñacerrada, y, por último, de hacer un esfuerzo
para ocupar a Baroja, lugar al Norte de dicha plaza, y perteneciente a
su hermandad. La tradición designaba aquel territorio con el histórico
título de \emph{Tierras del Conde}, por haber pertenecido en tiempos muy
antiguos a un D. Gómez Sarmiento, repostero del Rey de Castilla D.
Enrique II. Como país montuoso, en los habitantes de la hermandad
dominaban las ideas de \emph{retroceso}, así como en las tierras bajas
crecía lozana la planta de la libertad. Trabajillo había de costarle a
Espartero la destrucción de aquel baluarte que últimamente habían armado
entre peñas los soldados del absolutismo, con la intención bien clara de
dominar los pasos del Ebro y amenazar las puertas de Castilla.

En tanto, D. Martín hizo saber a los cinco individuos de la cuadrilla de
D. Fernando que si querían continuar agregados a la división, y
participar de sus víveres y ampararse de ella, era forzoso que
estuviesen a las agrias y a las maduras, afiliándose resueltamente como
soldados de Isabel II, a lo que accedió el caballero en nombre de todos,
enorgulleciéndose de combatir a las órdenes de Zurbano por la gloriosa
causa de la Reina. En los tiradores de caballería encajaron
admirablemente D. Fernando y Urrea, buenos jinetes y excelentes
escopeteros. Iturbide y Zoilo prefirieron servir como infantes, y Sabas,
que aunque valiente no manejaba el fusil con la necesaria destreza,
pidió que le agregaran a la ambulancia. He aquí, pues, a los cinco
expedicionarios metidos en militar danza por ley de la fatalidad o de la
Providencia, que el nombre no altera el sentido o filosofía del hecho.
Ninguno de ellos sospechaba, al salir de Miranda, que iban a pelear por
Isabel agregándose a su ejército. Pero Dios lo había dispuesto así, sin
duda porque, deseando terminar la guerra, quería que a esto se llegara
echando toda la carne en los respectivos asadores. La incorporación en
las filas fue acogida por D. Fernando sin repugnancia ni entusiasmo,
como un deber impuesto por circunstancias ineludibles, y lo mismo puede
decirse de Urrea, que en todo reflejaba los sentimientos de su amo.
Sabas se resignaba; Iturbe parecía contento, y Zoilo estaba como
demente, poseído de un frenesí de militar gloria.

Quince o más días duraron las operaciones de la brigada y sus veloces
marchas en el quebrado país que separa las \emph{Tierras del Conde} del
territorio de Campezu, los montes de Isquiz, el valle del Ega, los
pueblos de Marquínez y Apellániz. El objeto era interceptar los convoyes
que el carlista traía de Estella, y embarazar toda comunicación de Álava
con Navarra. Brillante fue aquella página militar, y los prodigios de
valor y agilidad que la formaron apenas caben en la historia, que por
hallarse bien repleta de tales hazañas ya no tiene hueco para más. Firme
en su puesto, y atento a su deber, Calpena no se propuso nunca hacer el
héroe, ni señalarse por el desmedido ardor guerrero: cumplía con su
deber, y nada más. En cambio, Zoilo era el propio espíritu de Marte; su
ambición de brillar y distinguirse nunca se saciaba; hallábase poseído
de una loca temeridad; sus hazañas eran, no ya extraordinarias, sino
inverosímiles. La envidia hubo de trocarse al fin en general admiración.

Había D. Martín tomado afecto a Calpena, con quien echaba párrafos
entretenidos en los cortos ratos de descanso, y hablando de Zoilo le
dijo: «¿Pero de dónde ha sacado usted ese diablete? Nunca he visto mejor
madera de militar, ni creo que haya en el mundo quien se le iguale.
\emph{¡Maño!,} en cuanto vea al General he de proponerle para alférez, y
aún me parece poco.» Esto era muy grato a D. Fernando, que, sin saber
por qué, sentía que el bilbaíno ganaba terreno en su corazón. Verdad que
Zoilo le mostraba un afecto sincero; contábale con infantil sencillez
sus actos de heroísmo, y parecía olvidado de todos los asuntos que les
hicieron rivales. Si no hablaba nunca de lo pasado, Calpena hubo de
recordárselo en una ocasión que es forzoso referir.

«Ven acá, chiquillo---le dijo, haciéndole sentar a su lado la noche
antes de incorporarse la brigada al ejército de Espartero.---Quiero
darte la buena noticia de que serás pronto teniente, quizás capitán.
Pero, pues has lucido bastante tus dotes guerreras, en las cuales ya
hemos visto que no tienes semejante, debo decirte que no expongas tu
vida con tan desmedida bravura\ldots{} Tiemblo por ti, hijo. Obligado
estoy a devolverte a tu familia, por compromiso que contraje con mi
conciencia. No me haría ninguna gracia verte espanzurrado el mejor día
en el campo de batalla\ldots{} ¿Y tú no temes morir? ¿No piensas en la
pena de los tuyos cuando sepan que has perecido? ¿No te acuerdas ya de
tu mujer?»

Nublose el rostro de Zoilo al oír esto, y la contestación no se hizo
esperar. «Sí que me acuerdo---dijo al fin.---¡Pues no he de acordarme,
si Aura es mi vida, la vida que he dejado allá\ldots!»

---Pues tienes que volver a su lado y hacerte dueño de su afecto
absoluto, sin alternativas lunáticas, ¿sabes? Yo haré cuanto deseas,
morirme o casarme\ldots{} Todo es cortar la esperanza y hacer
liquidación de lo pasado.

---Ya ve---declaró Zoilo---cómo hemos venido a ser amigos usted y yo.
Desde que nos metimos en la guerra se me fue del alma el rencor contra
usted\ldots{} Porque yo tengo dos vidas, dos amores: mi mujer y la
guerra. Guerreando la quiero más, si más es posible, y se me quitan
todos los resquemores. Valgo yo más que nadie, y no se ofenda\ldots{} Y
también le digo que no tenga cuidado por mí, porque no hay bala que me
mate, ni enemigo que me venza\ldots{} Si me hacen capitán de ejército,
ya no hay quien me separe de la vida militar. Y si consigo curar a mi
mujer y quitarle los malos recuerdos, ¿qué más puedo desear?\ldots{}
Como esas dos cosas quiero, las he de conseguir.

---En cuanto sea posible---dijo Calpena,---hemos de procurar
comunicarnos con nuestras respectivas familias. Tú anunciarás a la tuya
mi muerte o acabamiento, y yo a la mía la conquista de tu amistad. Son
dos buenas noticias, y cada una hará su efecto. Voy pensando, como tú,
que querer es poder. Queramos y podremos.

Poco más hablaron, porque Zoilo, rendido de cansancio, se caía de sueño.
D. Fernando durmió también tranquilamente, y gozoso fue el despertar,
porque recibieron orden de marchar a reunirse con Espartero.

El primer amigo que Calpena encontró en el ejército del Conde de Luchana
fue Juanito Zabala, ya coronel, que mandaba cuatro escuadrones de una
brillantísima caballería, dos de húsares tiradores y dos de lanceros.
Mucho se alegraron uno y otro de verse, y no esperó D. Fernando a que
Zabala le interrogase para contarle el cómo y cuándo de andar en
aquellos trotes. Previo consentimiento de Zurbano, pasaron Fernando y
Urrea al cuerpo adventicio que se había formado con paisanos de Rioja y
con desertores de la expedición de Negri; pero a Zoilo no quiso D.
Martín soltarle, aunque le dieran en oro molido, o sin moler, lo que
aquel endiablado chico pesaba.

Y comenzaron, ¡vive Dios!, vigorosas operaciones contra Peñacerrada. Una
de las divisiones, compuesta de tropas de la Guardia Real, la mandaba el
general Ribero; la otra, que era la tercera del Norte, el General
Buerens. Entre ambos reunían 18 batallones, distribuidos en tres
brigadas por cada división. Mandaba la artillería el brigadier D.
Joaquín de Pont, y la caballería el que ya conocemos. Zurbano se apoderó
de Baroja, y Espartero se posesionó de las alturas de Larrea, que al
punto fueron atrincheradas. Desde allí podía batir el castillo de
Peñacerrada a tiro corto de cañón. Tres días de furiosos combates
precedieron al asalto. Los carlistas, mandados por Gergué, se batían con
indomable valor, intentando destruir las líneas que Espartero iba
formando para emplazar su artillería. Ventajas obtenían los unos,
ventajas los otros, disputándose el terreno palmo a palmo. Los
batallones alaveses hicieron gallarda salida con un empuje que la
caballería de Zabala pudo contener. Y tras aquellos terribles días,
otros tres se emplearon en escalar con vigor de gigantes los muros del
castillo, ganando ahora un montón de piedras, para después perderlo y
volverlo a ganar con horrendo sacrificio de vidas. Incansable, buscando
siempre el primer puesto en el peligro, Espartero era el gran soldado,
el caudillo que de su magnánimo corazón sacaba la increíble fuerza que a
su gente infundía. Creciéndose con las dificultades, cada tropiezo era
escalón donde afianzaba el pie para seguir adelante. Quedó por fin bajo
la enseña de Isabel el formidable castillo, con sus murallas hechas
polvo y sus piedras salpicadas de sangre.

En tan terrible cuanto gloriosa ocasión, D. Fernando, que asistido había
con ardor y curiosidad a todas las peripecias del combate, peleando
también siempre que funcionaba la caballería contra los alaveses, fue
herido en la cabeza y hubo de retirarse. Urrea le llevó a Baroja, donde
pasó un día con las facultades turbadas a causa del golpe, y tres o
cuatro en completa inutilidad para la guerra. Su herida no era grave;
mas no le permitía volver a las andadas en algún tiempo. Pasó dos días
devorado de impaciencia y de sed, asistido del capellán Ibraim y de un
físico muy experto, sin formar cabal idea de las sucesivas peripecias
militares, pues tomado el castillo, obstináronse los carlistas en
defender la plaza a estilo zaragozano, disputando muro por muro y casa
por casa, y fue menester echar contra ellos todo el coraje de acá y la
inagotable energía del jefe y de su tropa. Oía Calpena el continuo
cañoneo, y ansiaba conocer el resultado de tan fiero batallar. Por fin,
una noche entró Urrea en el establo donde yacía, y le dijo: «Peñacerrada
es nuestra, señor. Hemos cogido el hueso, y allá van corriendo hacia
Toloño los perros que lo tenían.» No tardó Zabala en darle las
albricias. Todo era júbilo en Baroja, y la línea desde este pueblo a la
plaza ganada ardía en entusiasmo.

La inquietud mayor del caballero al abandonar su mísero alojamiento era
no saber de Zoilo ni de Sabas, pues Zurbano había salido en persecución
de los fugitivos. Zabala, que también les fue a los alcances, volvió sin
satisfacer las dudas de D. Fernando respecto a sus amigos. Si poco temía
del arrojo de Sabas, no podía desechar la idea de que el bilbaíno pagaba
a la muerte el tributo que su desmedida ambición de gloria le debía. En
estas ansiedades le cogió D. Baldomero, que de Larrea, después de la
entrada oficial en Peñacerrada, trasladó su cuartel a Baroja. Mandole
llamar, y mientras tomaba en el Ayuntamiento un frugal tente-en-pie, del
cual no participó Calpena por la radical inapetencia que sufría,
hablaron de lo humano y lo divino. Enterado el de Luchana de diversos
particulares interesantísimos, y hasta cierto punto novelescos (por
revelaciones que le hizo D. Beltrán no lejos de Medina, en Febrero
último), se arrancó a felicitar al caballero con la confianza militar
que gastar solía, y díjole después: «Pero, amigo mío, ¿en qué estaba
usted pensando cuando consintió que su madre se estableciera en Medina
de Pomar? Si todo aquel país no ha sido hasta hoy de los más castigados,
pronto le veremos arder\ldots{} No, no; allí no está bien. Debió usted
llevarla a Logroño, donde ella y Jacinta se habrían acompañado
lindamente. Allá la seguridad es completa. Nuestra casa es grandísima:
buenos alimentos, buenas aguas. A Logroño han ido a parar muchas
familias de estas hermandades, entre ellas las niñas de Castro, que creo
son amigas de usted.»

Diole el caballero las gracias con efusión, añadiendo que procuraría
trasladar a su madre a Logroño, si la guerra duraba\ldots{}

«¡Que si dura\ldots! Esto no se acaba nunca\ldots{} esto es un bromazo
terrible\ldots---clamó Espartero dando rienda suelta a la franqueza
militar y española, que iguala en la indiscreción a pequeños y
grandes.---¿Y qué quiere usted que pase con el desbarajuste de ese
Gobierno?\ldots{} Yo pregunto: ¿quién aconseja a esa buena señora\ldots?
Cada día más retroceso, más errores, más desconfianza de la libertad y
del pueblo, cuando el pueblo, la masa\ldots{} en fin, no quiero hablar
de esto\ldots{} Usted fíjese\ldots{} ¿Ha visto el país una situación más
desatinada? Les he dicho cuanto hay que decir\ldots{} No hacen caso:
ellos se lo saben todo\ldots{} y ahora nos quieren traer mayores enredos
y conflictos con esa contrarrevolución que han inventado, la bandera de
\emph{Paz y fueros}\ldots{} ¡Otro disparate, Señor! ¡En qué cabeza
cabe\ldots! Créame usted: si el patriotismo no me amarrara a este
puesto, si no creyera yo que me debo a mi patria, al pueblo sano y
liberal, ya me habría ido a mi casa\ldots{} ¡Ah, sí\ldots!»

Asintiendo a todo, D. Fernando aprovechó las franquezas del General para
pedirle que le facilitara medios de enviar una carta a Medina de Pomar,
y tuvo la dicha de que Espartero colmara sin tardanzas sus deseos, pues
al siguiente día pensaba enviar comunicación a Castañeda, que operaba
por allá. Pidió permiso Calpena para retirarse a escribir, y lo hizo con
calma y amor. Desde aquella hora todo fue bien, pues a poco de soltar la
pluma, en el rincón del establo donde había hecho su vivienda, tuvo
razón de Luchu, y al siguiente día le vio llegar tan famoso, radiante de
orgullo, en toda la gallardía teatral de su heroísmo auténtico, contando
sus hazañas sin atenuarlas con modestias anodinas. «Sepa usted, Sr.~D.
Fernando, que D. Martín me ha dicho: «Animal, eres capitán.»

\hypertarget{xvi}{%
\chapter{XVI}\label{xvi}}

Contó luego Zoilo el caso inaudito de Iturbide, que habiéndose portado,
el primer día de ataque al castillo, con toda la decencia militar de un
buen bilbaíno, había ensuciado su reputación y su carrera pasándose a un
batallón alavés. Creyó que los carlistas ganaban; se le aflojaron los
calzones\ldots{} Allá se fue\ldots{} Siempre le había tirado el
servilismo.

«El infeliz---dijo D. Fernando,---ha creído que por caminos de la
facción volvería más pronto a Bilbao.»

---Sabe Dios a dónde irá\ldots{} ¡Otra! Ya me río de pensar que habrá
visto a mi padrino Guergué, tal vez a mi padre, y les habrá dicho que
estoy aquí, en el ejército de Espartero, y que soy capitán, y
que\ldots{}

---Y que eres mi amigo. No serán pocos motivos de confusión para tu
padre.

---Pues hay más. ¡Si parece que esto lo hace Dios, conforme a mi querer,
más fuerte que todas las cosas\ldots! Pues la última vez que estuvimos
juntos Pepe y yo, el jueves por la mañana, nos dieron la noticia de que
usted había caído, en la segunda carga, con una herida mortal en la
cabeza. ¡Jinojo, qué sentimiento! Pasa media hora, y viene Segundo
Corral, y nos larga en seco la noticia: «El pobrecito D. Fernando acaba
de expirar!» ¡Jesús!

---¿Lo creíste?

---Yo no. No creo en la muerte de los que, según mi querer, deben vivir.

---Pero Iturbide se tragó la bola, y a estas horas se lo habrá contado a
D. Sabino, si es que anda todavía con ellos.

---¡Otra!, a mi padre le tiene usted ahora más contento que unas
pascuas, dando gracias a Dios\ldots{}

---¿Por mi muerte?

---Cabal\ldots{} A no ser que crea que yo le maté a usted\ldots{} Todo
es creíble allá\ldots{} Y en este caso, alegrándose, rezará mucho porque
Dios me perdone.

---¡Y tú y yo tan amigos!

---¿Esto qué es?

---Romanticismo, Zoilo. La lógica de las cosas absurdas, la risa del
dolor, la tristeza del placer\ldots{}

---¿Y eso qué quiere decir?\ldots{} ¿Poesía?

---Tal vez\ldots{} Misterios de las almas. Tú dices que querer es poder.
Yo digo que mereces ser dichoso y lo serás\ldots{} Vaya, chico, a tu
obligación, que es tarde. Separémonos. Hasta mañana.

Aquella noche, hecho un ovillo en su pesebre, sintiéndose febril, con
honda ansiedad en su espíritu, agobiado el cuerpo por la debilidad,
rebelde al sueño, el Sr.~de Calpena con esta idea se atormentaba: «¡Si
al fin dispondrá Dios que este loco se salga con la suya!» Efecto de la
fatiga y de la pérdida de sangre, complicadas con añoranzas muy tristes,
se le insubordinó el estómago, rechazando todo alimento, y los pícaros
nervios se declararon en audaz anarquía. En Baroja habría tenido que
quedarse, si no le llevaran en un carro, muy bien asistido por Urrea y
Sabas, que dejó gustoso las armas por el servicio de su querido amo.
Ibero y Zabala le acompañaban todo lo que podían, y Zoilo más de lo que
debiera, descuidándose del servicio, sin miedo a las reprimendas de D.
Martín. En tal estado, y siempre en seguimiento del Cuartel General,
pasó el puerto de Población. Dos días de descanso en Eripán, donde le
deparó Zabala un buen alojamiento, fueron el comienzo de la
recuperación, que había de ser completa dos semanas más tarde en la
histórica y por tantos títulos famosa ciudad de Viana.

Resolvió Espartero quitar al enemigo el único punto fortificado que aún
conservaba en la región alavesa, la villa de Labraza, cabecera de la
hermandad de su nombre en la cuadrilla de Vitoria, guarnecida de viejos
muros y de robustas torres, de las cuales hizo el carlista punto de
apoyo para remediar en lo posible la pérdida de Peñacerrada, y asegurar
sus comunicaciones con Estella. Mientras se disponían los elementos
necesarios para la expugnación de Labraza, pasó Espartero a Viana, donde
estuvo dos días, y de allí a Logroño, ávido de un breve descanso en su
casa. No le vio Calpena al partir; pero tuvo conocimiento de que el
ilustre Caudillo no le olvidaba, por un recado amistoso que Zabala le
transmitió, con estas palabras que de confusión le llenaron: «El
General, además, te ruega que le esperes aquí, a su regreso de Logroño,
pues tiene que hablarte.» Por más que se devanaba los sesos, no acertaba
D. Fernando en el descubrimiento del negocio que con él quería tratar el
conde de Luchana. «¡Hablarme a mí! ¿De qué\ldots?» Y en esta
incertidumbre vivió una semana, aguardando la solución del acertijo, con
el gozo de ver restablecida gradualmente su salud, pues las aguas y los
alimentos de Viana hicieron entrar en razón a su estómago. A los pocos
días de descanso y vida regalona en pueblo tan interesante, pudo montar
a caballo y dar buenos paseos con sus amigos por el camino de Logroño,
hasta llegar a los cerros donde se descubre el curso del Ebro caudaloso,
la mole de la \emph{Redonda} y el caserío y torres de la capital
riojana.

Grata fue la resistencia del caballero en aquel pueblo de tanta
nombradía en los anales de Navarra y de Castilla; disfrutó lo indecible
examinando las señales y vestigios de nobleza en calles viejas y
palacios desmantelados, en las antiquísimas iglesias de San Pedro y
Santa María. Mucho había que leer en aquellas piedras. Los curas del
arciprestazgo y los regidores de la ciudad franqueábanle códices y
papeles interesantísimos, donde vio y gozó históricas hazañas, como la
defensa que hizo el esforzado mosén Pierres de Peralta contra las tropas
del Rey D. Enrique II, y los horrores de aquel memorable sitio en que
las mujeres, así casadas como doncellas, manejaban las \emph{bombardas,
trabucos, cortantes y otras diversas artillerías}. Y fue tal el hambre
que pasaron los vianeses, que viéronse obligados a comer \emph{caballos
e otras fieras inusitadas}, según reza un viejo pergamino. En la guerra
de los Beaumonteses, que arrancó a Viana de la corona de Navarra para
pasarla a la de Castilla, también había mucho digno de perpetuarse para
ejemplo de los presentes. Vio D. Fernando el sepulcro de César Borja,
duque de Valentinois, que allí murió, y los de otros ilustres varones de
aquella tierra.

En estos entretenimientos le interrumpió Sabas, manifestándole que, pues
las queridísimas niñas de Castro-Amézaga se hallaban refugiadas en
Logroño, distante sólo dos leguas cortas, él iría, si su amo le daba
permiso, a visitarlas por su propia cuenta, como Sabas de Pedro, y a
enterarse de si estaban saludables y contentas. Pareciole a D. Fernando
muy atinada la idea de su escudero, y le despachó al instante con la
misión que se expresa, y la añadidura de un recado muy afectuoso de su
parte. Pero ¡ay!, al día siguiente volvió Sabas cariacontecido con la
triste novedad de que no había encontrado a las niñas, pues la señora
Doña María Tirgo, después de una temporadita de residencia feliz en la
capital de la Rioja, había logrado arrastrar a sus sobrinas hasta
Cintruénigo, donde a la sazón pagaban a los Sres. de Idiáquez la visita
que estos hicieron a La Guardia. ¡Ojo al Cristo!

Muy mal le supo el caballero esta desairada vuelta de Sabas; mas cuidó
de disimular la nueva tristeza que a las suyas y a su nostalgia se
añadía. Pasaba las noches entretenido con sus amigos, entre los cuales
la \emph{fiera inusitada} de Ibraim hacía el gasto de los chistes burdos
y sainetescos. Rodaba el tiempo, y todo el afán de Fernando era que
volviese pronto Espartero, que allí le había mandado esperar\ldots{}
¿esperar qué? ¡Oh incertidumbre!\ldots{} Para mayor aburrimiento, pasó
el caudillo una noche por Viana sin detenerse mas que media hora, y
Calpena recibió por el ayudante Serrano Bedoya nueva edición del
recadito de marras: «Que no se mueva de aquí hasta que yo regrese, o le
avise dónde debe ir a encontrarme.»

---Pues, señor, la broma es ya más que pesada---decía Calpena, buscando
medio de entretenerse con nuevos estudios de las antigüedades
vianesas.---Cuanto más libre me creo y más empeño pongo en disponer de
mi persona, más esclavo me encuentro. Mi sino es este, la esclavitud
constante, el arrastrar cadenas\ldots{} de rosas si se quiere; pero
cadenas al fin. ¿Qué habrá en mí para que chicos y grandes me honren con
sus afectos más vivos\ldots? Siento no tener a mano al gran Zoilo, el
filósofo del querer potente, para que me dé su opinión sobre esto.

En tanto que D. Baldomero iba contra Labraza, en Viana corrían voces de
que la tal operación sería de las más sangrientas. Para sustituir a
Guergué, que perdió su valimiento con el desastre de Peñacerrada, Don
Carlos había nombrado general de su ejército del Norte a D. Rafael
Maroto. Este, cogido el bastón, se metió en Estella, ocupándose en
reorganizar los batallones y en proveerlos de lo necesario para una
activa campaña. Desde allí mandó recadito a los de Labraza,
encargándoles que se defendieran hasta morir, que él iría en su socorro,
provocando a Espartero a singular batalla en aquellos campos. Todo
anunciaba una brillantísima página histórica; alguien creía próximo el
último acto y quizá la escena final del drama de la guerra. Pero así
como los dramas suelen flaquear en su desenlace por inhabilidad del
poeta que los compone, los lances guerreros también salen fallidos por
torpeza o desidia de estos poetas de la espada. En resumidas cuentas:
que el de Luchana apretó el asedio; que Labraza se defendió bien, hasta
que no tuvo más remedio que rendirse, sin que de Estella viniese Maroto
con todo aquel aparato de fuerzas que anunció. La esperada lucha
decisiva quedose para mejor ocasión, y Espartero, que había ido con
terribles ganas de romperse el bautismo de una vez y para siempre con su
rival de hoy, ayer compañero de fatigas americanas, volvió grupas, un
tanto descorazonado como militar, como político no descontento de la
prudencia de Maroto y de su pereza en sostener el reto.

Llegó por fin la ocasión que tan vivamente deseaba Calpena, y viendo
entrar a Don Baldomero en Viana al caer de la tarde de un caluroso día
de Julio, no tuvo sosiego para esperar a que el General le llamase, y se
fue a la casa de los Tidones, donde se alojaba, y solicitó audiencia,
que al instante le fue concedida. Sentábase a la mesa D. Baldomero para
cenar con el Arcipreste Don Alonso de Aimar, con el alguacil mayor o
Merino, D. Lázaro Tidón, tres señoras de la familia de Tidón y Asúa, el
General Van-Halen y otros; y convidado Fernando, aceptó gustoso la grata
compañía. Hablando de la guerra, dijo el de Luchana con su franca
llaneza: «No me la dio Maroto\ldots{} Ya me había tragado yo que no
vendría. Le conozco, es muy ladino, y no quiere comprometer el mando,
que deseaba y que no le conviene soltar\ldots» Sin saber cómo, la
conversación recayó en cosas muy distintas de los sucesos militares,
como la calidad de las judías verdes de Viana comparadas con las de
Logroño. Sostenía el vencedor de Peñacerrada, conciliando la justicia
con la galantería, que si al carnero de la merindad de Viana \emph{había
que quitarle el sombrero}, en judías \emph{de riñón} y en pimientos
morrones, \emph{donde estaba} Logroño y su ribera, no había que mentar
hortaliza. ¡Y para que se vean los misteriosos engranajes de la palabra
humana! ¿Cómo pudo ser que del tratado de las alubias pasasen aquellos
señores a la personalidad de César Borgia? Ello fue así, como también lo
es que ninguno de los comensales, incluso el héroe, poseía nociones
exactas de la vida y muerte de aquel afamado cardenal y guerrero,
teniendo Calpena que desenvainar modestamente su corta erudición para
ilustrar al esclarecido senado. No prestó gran atención Espartero a
estas historias añejas, que otras más vivas le solicitaban, y aferrado a
su idea, no cesaba de repetir: «Es muy ladino, muy ladino\ldots»

No pasó mucho tiempo después de la cena sin que la expectación de D.
Fernando quedase\ldots{} a medio satisfacer, pues Espartero, al
conferenciar con él en su despacho, no hizo más que mostrarle los
bordes, digámoslo así, del asunto que tratar quería, reservándose el
cuerpo del mismo. Con su consabida franqueza ruda, que en muchos casos
le resultaba bien, le dijo: «¿Pero a qué tiene usted esa prisa por
volverse a Medina? Un hombre como usted, de sus circunstancias, no puede
estar cosido a las faldas de la mamá.»

---Mi General, he conocido a mi madre hace poco tiempo.

---Ya, ya sé\ldots{} vamos al caso. Usted vale mucho, yo sé lo que usted
vale. No vengamos ahora con modestias ridículas. ¡Entre nosotros\ldots!
En fin, usted es hombre de grandísimo mérito. Lo sé, lo afirmo, y no hay
que desmentirme, ¿estamos? Usted quiere que yo le regale el oído
repitiéndole que es un modelo de caballerosidad, una inteligencia de
primer orden, un joven ilustradísimo\ldots{} Ea, lo digo yo y basta.

---Pues basta, mi General. ¿Y qué más?

---De sus modales y finura de trato, nada hay que decir, pues bien a la
vista están\ldots{}

---Cuando usted acabe de echarme incienso, respiraré.

---No es incienso, es justicia\ldots{} Me habló Urdaneta y otros, otros
amigos que le conocen a usted bien\ldots{} Y para que el hombre resulte
completo, también somos valientes, ¿eh? Me ha dicho Martín\ldots{} Pero
no trato yo ahora de valentías militares; estimo, sí, que sea usted
hombre de corazón, de voluntad bien templada\ldots{}

No exageraba D. Baldomero al manifestarse convencido de los méritos del
joven, pues, en efecto, D. Beltrán le había ponderado, quizás con lujo
de hipérbole, la inteligencia, cultura y dotes sociales del hijo
\emph{extranjero} de Pilar de Loaysa. Quizás estas cualidades eran
agrandadas por el de Luchana en su viva imaginación, que ciertamente la
tenía, como soldado de arranques, de momentos heroicos. «Bueno, señor
mío---añadió poniendo punto final a los elogios.---Convencido de que
usted vale y de que puede prestarme, a mí precisamente no, a la patria,
a España, a la libertad, servicios grandes, no dudo en\ldots{} Decláreme
usted ante todo una adhesión incondicional a los principios que
represento, digo, que representamos todos los leales, que representa la
causa legítima de Isabel II, la causa de la libertad.»

Confirmada por Calpena su profesión de fe política, el de Luchana
prosiguió así: «No cuento con usted para cosas de milicia; le quiero
para una comisión, misión mejor dicho, misión\ldots{} que le comunicaré
cuando estemos perfectamente de acuerdo en las cuestiones preliminares.
Ea, Sr.~D. Fernando, yo no le suelto ya. Si se aflige usted por la
ausencia de su mamá, la traeremos a la Rioja\ldots»

---Mi General, tenga la bondad de explicarme\ldots{}

---No explico más, ¡caramba! Lo dicho, dicho. Le tengo a usted trincado
por los cabezones. Escribiremos a la Condesa si es necesario\ldots{} Yo
me voy mañana a Logroño. No le diré que venga conmigo; pero váyase usted
pasado mañana, cuando guste, y allí seguiremos hablando. Por hoy, ¿eh?,
fijarse bien, como si no nos hubiéramos visto\ldots{} Esto es reservado.
Doy de barato que sobre las buenas cualidades que usted tiene domina la
que de todas es maestra, la discreción, fijarse, la discreción. Y no
digo más. Retírese usted ya\ldots{} Buenas noches. Descansar. Hasta
luego.

Y se fue el caballero a su hospedaje, sabiendo\ldots{} que no sabía
nada, sospechando, queriendo adivinar\ldots{} Toda la noche estuvo
viendo ante sí, en la obscuridad, los ojos de Espartero, negros,
penetrantes, ojos de trastienda y picardía, y su rostro atezado, duro,
que parecía de talla, labradito y con buches, el bigote triangular sobre
el fino labio, la mosca, las patillas, demasiado ornamento de pelos
cortos para una sola cara. La mirada del guerrero le decía más que sus
palabras, y a fuerza de leer en aquella, creyó descifrar el pensamiento
que estas no querían manifestar. «Una misión---se decía.---¿Acaso\ldots?
¿Qué entiendo yo de misiones y tratos y enredos\ldots? ¿Qué quiere hacer
de mí? ¿Un diplomático, un polizonte? Me ha escogido porque cree que la
discreción está en mi naturaleza\ldots{} como hijo del secreto que
soy\ldots{} el secreto mismo. No acepto. Me voy con mi madre.»

\hypertarget{xvii}{%
\chapter{XVII}\label{xvii}}

Dormido con la resolución de no aceptar, despertó con la contraria idea;
que estas mudanzas suelen traer el sueño a nuestro espíritu; y ya no se
ocupó más que en disponer su traslación a Logroño, buscando antes a
Zoilo para saber si pensaba continuar en la columna, o solicitar
licencia y volver al lado de su familia. Este era el anhelo de Fernando,
y esto le dijo, al encontrarle de regreso de un reconocimiento
practicado por Zurbano en el pueblo de Aras. Alegrándose de verle,
expresó el bilbaíno que desde su regreso de Labraza, donde había
cumplido como bueno, sentía que se le iba enfriando el entusiasmo
militar. Harto de gloria y satisfecha su ambición, renacían en él las
querencias de la familia. Dos días y dos noches llevaba ya con el
pensamiento empapado en la memoria de su mujer, a quien dormido y
despierto veía en su mente, anhelando verla con los ojos de la cara,
para recrearse en su belleza y entregarle el alma y la vida. Si su mujer
le quería, y se curaba de aquella maldita enfermedad de recordar a otro
y esperarle, él sería más feliz que los ángeles del cielo, y ninguna
falta le hacía la gloria militar; que esta, sabíalo Dios, la buscó por
dar a su querer una compensación de aquellas amarguras y por llenar los
vacíos de su corazón. No cesaba de pensar que su mujer le echaba de
menos, que indagaba su paradero, que padecía por la ausencia de él
soledad y tristeza\ldots{} «Y de tal modo---proseguía---se me han
clavado en el magín estas ideas, que ya no puedo menos de tenerlas por
cosa cierta y fundada; que lo que yo pienso con gana, sucede, sí, señor,
siempre sucede.

---También yo---dijo Calpena,---de algunos días acá, tengo la corazonada
de que tu mujer se ha curado de esa locura de recordar lo muerto y
esperar lo imposible. Sin ningún dato en que fundarme, lo siento, lo
creo, y en ello me voy afirmando cada día más. Es para ti contrariedad
grande el verte ya cogido en las redes de la Ordenanza y no disponer de
tu persona para largarte a tu casa cuando te diere la gana.

Quedose Zoilo al oír esto muy pensativo, acariciándose la cabeza, sin
que en esta brotase la idea que sin duda buscaba, y al fin, suspirando
fuerte, se consoló de la obscuridad de su entendimiento con estas
expresiones: «En fin, con un querer firme todo se arregla\ldots{}
Volveré a mi casa.»

---Pero ándate con mucho tiento, chico, y no se te pase por las mientes
la idea de la deserción, que podría salirte cara. No juegues con las
leyes militares. ¿Gloria quisiste? Tus triunfos te obligan a la
obediencia. ¿Quieres ir a tu casa, ver a tu mujer? Pues aquí me tienes a
mí para proporcionarte esa satisfacción, a mí, que te saqué de la cárcel
y que adquirí con mi conciencia el compromiso de devolverte a los tuyos
sano y salvo. Prométeme no hacer ninguna locura, pues al ponerte a mi
lado entraste para siempre en el terreno de la razón. ¿Estamos
conformes?

---Conformes, mi General. Así le llamo porque usted manda. Y váyase,
váyase pronto a Logroño, y si está allí su novia, como dicen, cásese con
ella, antes hoy que mañana, aunque para ello tenga que robarla\ldots{}
Si hace falta un amigo de coraje, avise. A casarse, y así estaremos
todos contentos.

---Ni mi novia está en Logroño, ni yo he de robarla, ni ese es el
camino, \emph{Zoiluchu}.

---¿Pues cuál es el camino, señor?\ldots{}

---Esperar obedeciendo.

---Pues obedezco esperando, como soldado de filas.

No hablaron más, y con apretones de manos se despidieron, trasladándose
D. Fernando con sus dos criados a Logroño, a donde llegó muy entrada la
noche. Los oficiales de \emph{Gerona} que iban con él encamináronle al
parador del \emph{Camerano}, en la calle del Mercado, no lejos de la
\emph{Redonda}, iglesia mayor del pueblo, y halló regular acomodo para
sí y su gente; cenó y durmió tranquilo; y como no se le cocía el pan
mientras celebrar no pudiera nueva conferencia con el héroe, al
siguiente día, en cuanto llegó la hora oportuna para visitas, se personó
en el palacio de Su Excelencia, una casona grande y severa, con fachada
de sillería y ornamento barroco en balcones y ventanas. En la puerta se
encontró a varios oficiales que conocía, y en el primer tramo de la
escalera a su amigo Pepe Concha, quien muy contento de verle le
introdujo en el billar, espaciosa sala del entresuelo. A la sazón el
General despachaba con su secretario: era forzoso que Calpena esperase
un rato, el cual resultó breve por la compañía de aquel simpático
oficial, jefe de la escolta, y del ayudante Allende Salazar. A la media
hora subió Fernando al primer piso, y Espartero le salió al encuentro
muy afectuoso. Vestía de paisano, en traje muy ligero por causa del
excesivo calor; y aún no habían concluido los saludos, cuando,
volviéndose hacia una puerta entreabierta, gritó: «¡Jacinta, Jacinta!»
Al conjuro de aquella voz, que era la voz del trueno en los campos de
batalla, y que allí sonaba tan apacible, apareció una dama de excelsa
hermosura, majestuosa en su familiar porte, sin el menor asomo de
presunción en la sencillez casera con que vestía. Al saludo ceremonioso
de Calpena contestaron los dos, marido y mujer, con esa confianza de
buen gusto, propia de personas de viso que gustan de disimular su
superioridad. La dama, más aún que su esposo, poseía un arte magistral
para combinar la llaneza con lo que modernamente se llama distinción, la
gracia con la autoridad. En pie los tres, Doña Jacinta (la etiqueta de
la época obliga a conservarle el \emph{Doña}) dijo festivamente al
caballero: «¿Me acierta usted de quién es esta carta?---y al decirlo
mostraba una que tenía en su mano muy dobladita.---A ver, a ver\ldots{}
¿conoce la letra?»

---Es de mi madre---dijo Calpena mirando el papel que la Condesa de
Luchana puso ante sus ojos.

---Ya hablaremos, ya hablaremos. Tengo que reñirle a usted\ldots{} Así
me lo encargan. Por cierto que es usted el hombre de la mala suerte en
sus viajes. Ayer, ayer mismo pasaron por aquí las niñas de Castro, de
vuelta de Cintruénigo\ldots{} Pero siéntese, D. Fernando. Si tienen
ustedes que hablar, me voy.

---No, no; tiempo hay---dijo el héroe sonriendo.---¿Y qué me cuenta
usted de ese desastre de Morella?

---¿De Morella? No sé una palabra.

---El pobrecito Oraa se ha visto precisado a levantar el sitio.

---¡Qué dolor!---exclamó la dama suspirando, ya sentados los tres.---Lo
he sentido por todos: por la Reina, por el Gobierno, por los liberales,
y principalmente por D. Marcelino\ldots{} Es un hombre muy bueno, un
militar que sabe su obligación, y le quiero de veras.

---Yo también---afirmó el de Luchana.---La empresa no era un grano de
anís. ¡Sabe Dios los entorpecimientos con que habrá tenido que luchar el
pobre Oraa, la falta de recursos!\ldots{} Es la mía: el Gobierno quiere
acabar la guerra, y nos tiene sin raciones, las tropas descalzas. Crea
usted, Calpena, que esos malditos moderados nos llevarán al abismo, si
no se les ataja\ldots{} En fin, este mal paso de Morella, esta retirada
ante Cabrera ensoberbecido\ldots{} nos parte\ldots{} ¡Qué contratiempo,
qué desdicha! Por acá íbamos muy bien; ya usted lo ha visto.

---Crea usted, mi General---indicó Calpena,---que este inmenso litigio
de la guerra civil no se ha de sentenciar en el Centro.

---Se sentenciará en el Norte, convenido\ldots{} pero los sucesos de
allá ayudan o entorpecen, y este resbalón del pobre D. Marcelino\ldots{}
Cuidado que yo le quiero\ldots{} Este resbalón ha de traernos
consecuencias funestas. ¡Qué lástima, Señor\ldots!

---Pero, Baldomero---dijo la Condesa con esa familiar lisonja que tan
bien cae en labios españoles cuando son de mujeres buenas y
amantes,---tú no puedes estar en todas partes.

---¡Yo\ldots!---exclamó el caudillo con modestia, que sin duda no
sentía.---¡Sabe Dios si me hubiera pasado lo mismo, o quizás algo
peor!\ldots{} La guerra es un azar, un compromiso, y por más que uno
ponga de su parte todo lo que tiene dentro, siempre hay algo que no
depende más que del Acaso, de\ldots{}

---Y usted, mi General, ha sabido entenderse con el Acaso.

---¡Oh!, no crea usted\ldots{} También me ha jugado algunas\ldots{}
Pero, la verdad, no hay queja\ldots{}

---No tenemos queja---repitió Doña Jacinta.---Dios no nos
abandona\ldots{} ¡Ay, qué pena! No puedo apartar de mi pensamiento al
pobre D. Marcelino\ldots{} Pero, en fin, dejemos por ahora las cosas
tristes\ldots{} que a D. Fernando tengo yo que decírselas muy gratas,
pero muy gratas.

---Todo lo que usted me diga, señora, me será siempre agradabilísimo.

---¿Está bien seguro de eso?\ldots{} Bueno; luego hablaremos. Váyase
usted preparando.

---Ya lo estoy.

---Y por ahora, dispénseme---dijo levantándose.---Tengo que hacer. No
crea usted: todavía no he acabado de leer la carta\ldots{}

En pie los dos, el visitante y la señora cambiaron frases de donosa
cortesía: ---¡Vaya si hablaremos!\ldots{} Esta noche hará usted
penitencia con nosotros\ldots{} No, no se admiten excusas. ¡Si usted lo
desea!\ldots{} Está usted rabiando porque le hable yo de cierta
persona\ldots{}

---No digo que no.

---Pues para su tranquilidad, le diré que ayer estuvieron aquí las niñas
a despedirse. ¡Si viera usted qué guapa está Demetria!

---Lo creo.

---Y Gracia, no digamos\ldots{}

---También lo creo.

---Pero no creerá que por el lado de Cintruénigo hay nubes\ldots{}

---¿Y truenos?

---Truenos todavía no\ldots{} Vaya, no más por ahora. A las siete, D.
Fernando.

Solo con el Conde, manifestó verdadero ardor porque este acabara de dar
solución al acertijo de Viana. «¿Pero qué prisa tiene usted?---le dijo
Espartero sonriente.---¡Si ahora le vamos a tener secuestrado aquí por
mucho tiempo! Ya le dirá Jacinta esta noche su plan de traernos aquí a
la Condesa\ldots»

La entrada del General Ribero, al que siguió, con minutos de diferencia,
la del brigadier Linaje, cortó la visita, y Calpena creyó discreto
retirarse. Acudió al anochecer a la invitación para la cena, que fue
gratísima, con asistencia del General Van-Halen, del coronel Zabala, del
ayudante Gurrea y de la lindísima Vicenta Fernández de Luco, hermana de
madre de la Condesa, y bastante más joven que esta. Doña Jacinta apenas
pasaba de los treinta, y Vicenta no llegaba a los veintidós. Casó el 41
con Pepe Concha.

Llevó el peso de la conversación el brazo militar, comentando y
discutiendo el desastre de Morella. No obstante disponer Oraa de
veintitrés batallones, doce escuadrones y veinticinco piezas de
artillería, y de contar con los expertos Generales de división Borso,
San Miguel y Pardiñas, no pudo contrarrestar el empuje de Cabrera,
amparado de las fragosidades y quebraduras de aquellos montes
inaccesibles. Según Van-Halen, que conocía bien el Centro y la clase de
guerra que allí se hacía, la culpa del descalabro del buen Oraa era del
Gobierno, que en punible abandono tenía los servicios de administración,
en atraso las pagas, descuidado el vestuario, así como el suministro de
municiones. Debía Cabrera su renombre, más que a sus cualidades de
astucia y arrojo, a la incuria de nuestros gobernantes, que no habían
sabido poner en manos de los defensores de la Reina armas eficaces para
combatirle. De sobremesa, mientras por un lado despotricaban los
caudillos sobre este para ellos sabroso tema, por otro Doña Jacinta y su
hermana platicaban con D. Fernando de la admirable resistencia de la
niña mayor de Castro, en el asedio que nuevamente le ponían los Idiáquez
con ayuda de su fuerte aliada Doña María de Tirgo. De buena tinta sabía
la Condesa que, desesperados los sitiadores de la constancia de la
señorita mayor, habían tratado de entenderse con la menor, creyendo
encontrar en ella ambiciones de ceñir corona de marquesa. Pero la
vivaracha niña quería imitar a su hermana en la vocación de quedarse
para vestir imágenes. De todo ello resultaba que D. Fernando no tenía
perdón de Dios si no cambiaba su actitud circunspecta por otra más
decidida. Sin mostrarse el galán abiertamente contrario a estas ideas,
pues la galantería se lo vedaba, halló medio de rebatirlas aceptándolas
y de hacerlas suyas agregándoles cantidad de ingeniosos \emph{peros},
todo con gran derroche de ingenio y picardía graciosa. Así entretuvieron
la primer noche, retirándose Calpena muy agradecido a tanta bondad, y
ligado ya por cordialísima simpatía a la familia del héroe.

Ningún día dejó de acudir al palacio de la plazoleta de San Agustín. No
siempre pasaba al despacho de Espartero, que a menudo tenía visitas, o
tareas urgentes con Linaje u otro secretario, a las cuales consagraba
largas horas, fumando constantemente puros habanos de los mejores. En
Doña Jacinta observó Calpena el prototipo de la dama casera, pues no
había otra que la igualase en dirigir y conservar en orden perfecto su
casa y servidumbre, sin olvidar por esto las obligaciones sociales.
Inflexible para exigir a todos cumplimiento, era tan ordenancista en su
hogar como D. Baldomero en los campos de batalla. Las comidas se
anunciaban a toque de campana, y ¡ay del que dejara de acudir a su
puesto! El General mismo no se desdeñaba de dar a conocer su miedo a las
severidades de la digna esposa. Era muy sobrio en las comidas, y para él
no había mayor suplicio que estar largo tiempo en la mesa. En días de
convite o de extraordinario, se deshacía en impaciencia, anhelando que
llegase pronto el momento del café y los puros. Ensalzaba las comidas
breves; solía decir que debíamos buscar un medio de ingerir de golpe los
alimentos en el estómago, \emph{como se carga un fusil}.

Cuidábase Jacinta de poner coto a la excesiva largueza del héroe en
socorrer pobres y dar auxilio a necesitados, pues aunque era caritativa,
no gustaba del despilfarro, que aun por generosidad es cosa mala.
Espartero fue hombre que no reclamó nunca del Gobierno las pagas
atrasadas, ni se cuidó de que la Nación le reintegrara las sumas que
anticipó de su bolsillo para dar de comer a los soldados, y así lo hizo
más de una vez, porque era fuerte cosa pretender llevarles a la victoria
con los estómagos vacíos. Los parientes pobres de Granátula y Almagro
habían encontrado en el General una mina inagotable, y los desvalidos de
Logroño no padecían hambre. Si le adoraban los soldados por valiente,
pródigo de su sangre, no le querían menos los pedigüeños por el arrojo
con que vaciaba sus bolsillos. Estos y su corazón estaban siempre
abiertos al heroísmo y a la limosna.

Sin contrariarle abiertamente, procuraba Doña Jacinta reducir su
magnanimidad a límites razonables; mas no alcanzaba en este terreno, la
verdad sea dicha, tantas victorias como él combatiendo a los sectarios
del \emph{retroceso}. Gozaba la excelente señora la simpatía y
admiración de todo el pueblo, por lo bien que sabía manifestar su
superioridad social sin ofender a nadie, porque guardando las etiquetas
era cariñosa y accesible. Adoraba el orden, creía en la eficacia de los
puestos personales, y deseaba que cada cual ocupase el suyo y respetase
los ajenos. Con los humildes sabía ser cariñosa, con los grandes un
poquito encopetada, con todos afable y digna. Su amistad con Pilar de
Loaysa databa de cuando esta se casó y Jacinta era una niña que aún
vestía de corto. En Zaragoza se conocieron, ligándose con entrañable
ternura, a la que siguió más tarde relación continua por correspondencia
cariñosa. Juntáronse años adelante, por muy pocos días, en Pamplona,
cuando Jacinta, soltera todavía galanteada por Espartero, estaba en todo
el esplendor de su hermosura, y ya la Duquesa de Cardeña peinaba canas;
después no se vieron más. El secreto de su amiga lo supo la condesa de
Luchana por la revelación que a Espartero hizo D. Beltrán; y si antes de
conocer a Fernando le estimó, conocido le miraba con afecto fraternal,
como de hermana mayor; y cuando la informó Doña María Tirgo de que era
hijo de un príncipe, le tuvo en mayor aprecio, y vio más claras sus
altas dotes de inteligencia, nobleza y elegancia.

\hypertarget{xviii}{%
\chapter{XVIII}\label{xviii}}

No se habría conformado D. Fernando con la ociosidad en aquella tierra
hospitalaria, si la frecuente correspondencia con su madre no vigorizara
su espíritu. No cesaba la noble señora de recomendarle que prolongase su
permanencia en Logroño, que fuese agradecido a las bondades de Espartero
y su familia, pues le convenía ciertamente estar al arrimo de quien, por
su autoridad militar y la política que iba adquiriendo, parecía llamado
a ser en breve tiempo el árbitro de los destinos de la Nación. «Doloroso
es para mí---le decía,---el verme privada de tu presencia; pero me
consuela de mi soledad el saber dónde y con quién estás, el considerar
reconocido y apreciado tu mérito, principio quizás de las grandezas que
deseo para ti.» Y contestando a la carta en que se le manifestaba el
deseo de Doña Jacinta de traerla a Logroño, decía: «La impresión primera
ha sido de regocijo; pero después la reflexión me ha hecho conocer que
mi presencia podría perjudicarte. Tú no lo creerás así; yo veo las cosas
con frialdad, y no puedo desechar la idea de que por algún tiempo debes
permanecer sin mí al lado de esos señores. Bien sabe Jacinta cuánto le
agradezco sus afectos cariñosos. Pero en su buen juicio comprenderá que
a todos nos conviene mi obscuridad, y que esta es necesaria para que tú
brilles.» Contestaba D. Fernando a estas razones que él no quería
brillar; que ningún bien social podía compensarle de la ausencia de su
querida madre, y que, por tanto, persistía en ir en su busca en cuanto
los caminos se hallasen despejados, para mayor seguridad del regreso.

Notó el caballero que constantemente llegaban a Logroño y conferenciaban
con el General personas diversas, venidas unas de Madrid, otras de
Pamplona, como emisarias del Virrey, general Alaix; otras, de pinta muy
extraña, parecían procedentes del Cuartel de D. Carlos. Entre las caras
madrileñas, algunas reconoció Fernando como significadas en la
patriotería más ardiente. Creyó ver también a D. Antonio González, a
Ferraz, a Sancho y a otros partidarios juiciosos del \emph{progreso}.
Indudablemente, el General apoyaba con decisión la idea que empezó a
llamarse \emph{progresista}, declarándose enemigo del bando moderado y
disparando contra él bala rasa, sin reparar en las manifiestas
concomitancias de este partido con la Gobernadora. Le traía muy inquieto
la protección que esta y su camarilla daban a Ramón Narváez,
permitiéndole organizar el ejército de reserva, como un medio indirecto
de hacer sombra a Espartero y de levantar frente a él un nuevo ídolo
militar. No le gustaban a D. Baldomero estos ídolos secundarios, que
podrían ser dioses mayores el día menos pensado, y la influencia
política que alcanzado había con su victoria no se la dejaría arrancar
¡vive Dios!, a dos tirones. Un día y otro mandaba a Madrid quejas del
abandono del Gobierno; hacía responsables a ciertos y determinados
ministros de las privaciones del ejército; amenazó con su dimisión si no
dejaban sus puestos Mon y Castro, y al fin, con este \emph{modo de
señalar}, dio cuenta del Ministerio del Conde de Ofalia. Nombrado
Presidente el Duque de Frías, poeta y diplomático, Espartero le exigió
que desmembrase el ejército de reserva formado por Narváez, agregando
dos divisiones al de Castilla la Vieja, para contener las facciones de
Merino y Balmaseda;

pidiole que, en reemplazo de Oraa, fuese nombrado Van-Halen general del
Centro. A regañadientes, cediendo a la presión del que dueño se hacía de
todos los resortes, \emph{quia nominor leo}, el buen D. Bernardino,
excelente hombre, prócer ilustre, y ante todo poeta insigne, se
doblegaba y sucumbía por su propio miedo y por los altos miedos
palatinos.

Nunca habló de estas cosas Calpena con el General, quien, en sucesivos
coloquios, fue menos reservado respecto a la índole de la comisión que
confiarle pensaba. Uno de los primeros días de Septiembre, a punto que
el Cuartel General se movía para emprender operaciones de que nadie
tenía conocimiento, dijo Espartero a su amigo, en forma que no admitía
réplica ni excusa, que a seguirle se preparase. Llevado de la
fascinación que el héroe sobre él ejercía, y cediendo además a una
extraña querencia del misterio y a ideas de elevada ambición que le
rondaban la mente, no vaciló en obedecer. Despidiole la Condesa con
afecto maternal, asegurándole que en compañía de su marido no podía
correr ningún riesgo; afirmó él gozoso que nada le importaba exponer su
vida, con tal de ser grato a su ilustre amigo, y partió entre la
comitiva del Cuartel General, llevando a uno solo de sus criados, Urrea.

Por toda la orilla derecha siguieron, sin parar hasta Lodosa, y era
general la persuasión de que se preparaba un ataque a Estella. Al
anochecer de aquel día, 3 de Septiembre, las avanzadas de Espartero se
tirotearon con guerrillas carlistas; pero estas desaparecieron durante
la noche, y el ejército liberal siguió hasta Artajona. Nueva detención,
que en este punto fue más larga, porque recibió el General noticia de un
descalabro de las tropas de Alaix, virrey de Navarra, el cual, empeñado
en duro combate con los carlistas, en el Perdón, fue rechazado con
bastantes pérdidas, resultando heridos el mismo Virrey y su segundo,
Espeleta. Esto y la noticia de que Cabrera, ensoberbecido con el triunfo
de Morella, mandaba una división a engrosar las fuerzas de Navarra,
detuvieron a Espartero en su marcha, si es que esta tenía por objeto
atacar a Estella, lo que no se sabe, pues a nadie comunicó su
pensamiento. Humor endiablado tenía el General en aquellos días, y su
indecisión revelaba la crisis de su ánimo. Dio instrucciones para que D.
Diego de León, que operaba en la Solana, ocupase determinados puntos, y
para que la división de Hoyos hiciese un reconocimiento hacia Los Arcos,
y otras disposiciones tomó, cuyo alcance nadie podía penetrar. Al quinto
día llamó a Calpena, y sin encerrarse con él, paseándose juntos en un
abandonado huertecillo de la casa donde el General se alojaba, hablaron.
La conversación, oída de lejos, habría podido pasar por insignificante,
pues carecía de toda solemnidad y de tonos graves y misteriosos.

---Yo me vuelvo a Logroño a darme otra descansadita---dijo D. Baldomero
con jovialidad;---pero usted, amigo D. Fernando, aquí se queda, y por de
pronto se incorpora a las fuerzas de Diego León. Luego hará usted lo que
le mandaré ahora mismo en pocas palabras. Oído: dentro de un rato se va
usted a su alojamiento, y no se mueve de allí hasta que reciba un recado
mío.

---Bien, mi General.

---Mi recado es lo que menos puede usted figurarse. Consiste en un mazo
de puros habanos, y se lo llevará un arriero\ldots{} No sé si usted le
ha visto\ldots{} Le encontramos en Lodosa con su recua\ldots{} Todo el
ejército le conoce.

---En efecto, le vi, y me dijeron su nombre; pero no me acuerdo.

---Se llama Martín Echaide. Es popular y muy querido en estas tierras.
Tanto nosotros como el enemigo le permitimos franquear las líneas, y
recorrer libremente el país, porque se ha declarado neutral, y sostiene
su neutralidad como un caballero.

---Pero no lo será realmente.

---Me figuro que no---dijo Espartero con acento de marrullería
fina.---El objeto de llevarle los cigarros es para que le conozca a
usted y se fije en su rostro\ldots{} ¡Ah!, no haya miedo de que se le
despinte. Nada le dirá a usted, ni usted a él tampoco, como no sea el
mandarme las gracias por los cigarros.

---Hasta ahora, mi General, la misión que usted quiere encargarme es
facilísima.

---Después no lo será tanto. Se queda usted, como digo, con Diego León,
y en el momento en que Echaide se le presente y le diga: «D. Fernando,
vámonos,» le obedece usted como si yo se lo mandara.

---¿Y para esto, mi General, tendré que disfrazarme de arriero?

---Justo; procurando, naturalmente, la mayor perfección en cara y ropa.
Disfrazará usted también a su criado, que me ha parecido de un tipo muy
para el caso. Con Echaide va usted a donde él le lleve, que le llevará
bien seguro a donde debe ir.

---Faltan ahora las instrucciones fundamentales, mi General, pues
presumo que mi misión no es tan sólo arrear las caballerías del
Sr.~Echaide.

---Ciertamente que no. Ya no es un secreto para usted que este bueno de
Echaide me pone en comunicación con una persona del campo enemigo; pero
las cosas graves que entre una y otra parte se han de tratar no son para
expresadas por Echaide, ni es prudente fiarlas al papel. En estas
embajadas, amigo, no se cruzará ningún papel escrito.

---Ya entiendo, mi General: el papel soy yo, mi buena memoria, y mi
palabra la escritura.

---Justamente. Con su comprensión rápida de todas las cosas me ahorra
usted largas explicaciones. Echaide no es más que el\ldots{} el\ldots{}

---El vehículo; la idea soy yo.

---Exacto. Como nada se escribe, como todo ha de ser verbal, he tenido
que escoger una persona muy inteligente, instruida, que se penetre bien
de mis condiciones, que reciba las del contrario, que las discuta si es
preciso, que transmita fielmente lo que uno y otro digan\ldots{} También
he tenido en cuenta su caballerosidad, su conocimiento de la historia y
de la política. Para decirlo todo, su falta de ambición me agrada, y su
independencia es para mí una garantía de fidelidad. Con que\ldots{}

---Comprendido todo, mi General. Ahora falta que escriba usted en mi
mente su pensamiento con signos bien claros, de modo que yo me penetre
bien y no padezca ningún error al transmitirlo.

---Tengo la seguridad de que ni escrito iría con más claridad. Esta
noche se viene usted por aquí, y le diré mis condiciones para la paz.
Son tan sencillas y tan breves, que caben en un papel de cigarro.
Procure el hombre fijarse bien. Mañana vuelve usted. Paseamos un rato en
este jardinillo y repetiré las condiciones para que se graben en su
memoria. No me escriba usted ni una letra, por los clavos de
Cristo\ldots{} Y por último, nada he de decirle de la reserva, de la
absoluta reserva\ldots{}

---¡Por Dios, mi General\ldots!

---No, no; si estoy bien seguro.

---Pero falta una cosa. Al llegar yo donde está esa persona, ¿cómo
acredito mi calidad de embajador?

---Todo está previsto. Las credenciales que usted ha de presentar son
una sola palabra. Ya lo hemos convenido él y yo: desde Burdeos me lo
propuso.

---¿Una sola palabra?

---El nombre de un pueblo del Perú donde él y yo nos conocimos.
Fácilmente lo grabará usted en su memoria. Mañana se lo diré. Cuando
llegue usted al punto donde ha de celebrar su primera conferencia,
Echaide será su introductor de embajadores. Con que\ldots{}

---¿Me retiro?

---Sí. Hasta la noche.

Retirose Calpena en un grado de excitación indescriptible, la mente
pletórica del sin fin de ideas que en ella despertaba el grave asunto en
que iba a ser actor, y actor histórico con visos de novelesco. Era un
mundo que se le metía en el pensamiento, con imágenes mil fabulosas, con
representaciones de actos en que probaría su valor y su inteligencia,
con ideas elevadas, con fin nobilísimo como era el de la paz. Adelante:
no se avenía con las seguridades que el General le dio de que en su
misión no correría peligro. Sí, sí, que los hubiera, pues los peligros y
la gloria de vencerlos satisfacían los anhelos de su alma generosa más
que una campaña fácil y sin accidentes. Ningún fin alto y grande se
alcanza sin sacrificio, y es forzoso ver en las penalidades la
consagración de toda labor benéfica.

Recibió puntualmente los cigarros; repitió las visitas al General por la
noche y mañana siguiente. Oyó dos veces las instrucciones, mejor dicho,
las condiciones, que estampadas con letras de fuego quedaron en su
memoria; tomó el santo y seña, o mejor, signo de inteligencia; vio
partir al caudillo para Logroño; incorporose al ejército de León, y ya
no hizo más que esperar, clavados los ojos en la imagen borrosa de su
destino.

El diálogo que se transcribe es exacto en sus ideas y sentido,---el
arriero Echaide, rigurosamente histórico.

\hypertarget{xix}{%
\chapter{XIX}\label{xix}}

Muy a gusto se agregó el caballero al ejército de León, y no poco
orgullo sentía de hallarse tan cerca del héroe, cuyas fabulosas hazañas
parecíanle dignas de un Romancero. El creciente influjo político del de
Luchana impuso el nombramiento de Alaix para Ministro de la Guerra, no
obstante su reciente descalabro; y vacante el virreinato de Navarra, fue
designado León para este puesto, que tan bien ganado tenía. Siguiole
Fernando a Pamplona, donde hizo nuevas amistades, muy gratas: Manuel de
la Concha, ya coronel, hermano de Pepe, y que si en la gallarda figura
se le asemejaba, no así en el carácter, que era vivísimo, tirando a
violento, poseído de la pasión militar en sumo grado, y del anhelo de
saber mucho y de practicar lo que aprendía; Domingo Dulce,
distinguidísimo oficial de caballería, muy intrépido; Federico Roncali y
otros. Con ellos pasó buenos ratos en los ocios de Pamplona, que no
fueron largos, porque León, nunca harto de combatir ni saciado de
gloria, salió en busca del enemigo con ansias dementes. Era un hombre
febril, hercúleo, que empezaba en un inmenso corazón y acababa en una
lanza. Se le podrían aplicar los cuatro enérgicos calificativos de
Aquiles: \emph{impiger, iracundus, inexorabilis, acer}.

Encaminose el héroe a Tafalla, buscando camorra a los carlistas. No era
de estos que aguardan las ocasiones más favorables para trabar batalla.
Según él todas las ocasiones eran buenas. Provisto de víveres para tres
días, se lanzó por aquellos campos, como andante caballero, en busca de
lo que saliere, y en Obanos, Legarda y Muruzábal encontró carne enemiga
en que cebar las picas poderosas de sus terribles lanceros. Admiraba
Calpena su gallardía, su varonil rostro, en que relampagueaban los
grandes ojos calenturientos. Los bigotes rizosos del General eran los
mayores y más bellos que en aquel tiempo se conocían. El chacó, con
cimera de plumas ondeando al viento, agrandaba su figura y hacíala
fantástica; su apostura sobre el caballo no tenía semejante. Fascinaba a
la tropa, comunicando a todos, hombres y caballos, su ardor y fiereza.
No le vio Calpena manejar la lanza. La primera hazaña de Belascoaín
había sido algunos meses antes; la segunda, que debía ilustrar su
nombre, fue meses después, en Abril del 39. Cuando se dieron las reñidas
acciones de Sesma y los Arcos en Diciembre del 38, ya D. Fernando no
estaba en el ejército de León, pues un día de Octubre, hallándose
meditabundo en Artajona, rumiando su impaciencia y amargado por las
añoranzas, presentose Martín Echaide y pronunció el conjuro sibilítico:
«D. Fernando, vámonos.»

Como asimismo le dijese que uno de sus hombres marchaba a Logroño con
dos acémilas de vacío, no quiso desperdiciar Calpena tan buena ocasión
de escribir a su madre, y lo hizo despacio y amorosamente, enviando a
Doña Jacinta la carta, con súplica de que por el conducto más rápido la
remitiese.

Ya en marcha, en una aldea próxima a Mendigorría, emplearon gran parte
de la noche en la operación de vestirse de máscara D. Fernando y Urrea,
con las ropas que Echaide traía para el caso, agregando a ellas la
posible alteración de los rostros, en lo que pusieron todo su esmero y
exquisitos primores de arte. Ya D. Fernando había descuidado sus barbas
y cabellos, y en estos aplicó tales refregones de tierra, que pronto
quedaron incultos y enmarañados a usanza salvaje. Lavándose ambos la
cara, si así puede decirse, con polvo del camino, obtuvieron el tono y
pátina de una epidermis horriblemente áspera. Cortose Fernando el
bigote, igualándolo con las barbas, para que todo el rostro quedase como
no afeitado en dos semanas. Cuidaron asimismo de las manos y uñas,
procurando en aquellas la endurecida costra de suciedad, en estas el
luto riguroso, y con un poco de hollín, diestramente aplicado a las
orejas, sienes y carrillos, quedó Calpena hecho un mostrenco tan zafio y
bestial, que no había más que pedir. En Urrea no fue tan necesaria la
transformación, porque su aspecto proceroso y su cara vulgar le
asemejaban a lo que quería ser. Había hecho D. Fernando estudios de
lenguaje, asimilándose un castellano burgalés de los más rudos con dejos
de baturrismo. Bastábale a Urrea con su sonsonete éuskaro, en lo que
poco o nada tenía que fingir. Quedaron, por añadidura, convenidos los
nombres que habían de sustituir a los verdaderos, llamándose D. Fernando
Aquilino Orcha, y más brevemente \emph{Quilino}, natural de Briviesca, y
el otro, Francisco Muno, de la parte de Aramayona. Suponíase, por lo que
pudiera suceder, que Muno había servido cuatro años en la partida de
Lucus, y \emph{Quilino} otros tantos en la de Merino, retirándose del
servicio por la derrengadura que se le produjo al caer del techo de una
ermita en el ataque de Lodosa. Habíale quedado un impedimento del
costado derecho, y la natural torpeza para mover los remos de aquel
lado. Fingía muy bien el caballero la imperfecta andadura, con
ligerísima cojera en que no podía verse la menor afectación.

Componíase la cuadrilla de cuatro sujetos: Echaide, los dos noveles, y
un cuarto arriero, como de sesenta años, a quien de apodo llamaban
\emph{Santo Barato}. Era el arriero jefe cincuentón, de mediana
estatura, tan chupado de rostro, que los carrillos se le juntaban por
dentro de la boca, formando al exterior dos cavernas velludas; los ojos
se le metían hasta el cogote, sin que de ello resultara aspecto de
fiereza, sino más bien como de anacoreta, o como las malas imágenes que
representan a los benditísimos padres del yermo. Su sonrisa de beatitud
convidaba a la confianza. En el cinto de cuero llevaba el rosario de
cuentas negras y pringosas, y un puñal. Era el vestido de los cuatro
calzón corto con peales, chaqueta parda y pañizuelo a la cabeza, las
camisas del más tosco hilo campesino. En suma: a Urrea le faltaba poco
para ladrar; Fernando resplandecía, si así puede decirse, de obscuro
idiotismo y de tosquedad y barbarie. Llevaban cuatro bestias, dos mulos
y dos borricos, mejor apañados que las personas, con sus aparejos en
buena conformidad, y la carga era de pellejos de aceite, algunos
garbanzos, pimentón molido, vinagre y otros artículos de menor cuantía.

Con sus cuerpos y los de sus animales llegaron a Estella al caer de una
tarde de Octubre, metiéndose en una posada próxima al Castillo y al
paseo de los Llanos. Gran aparato de fortificaciones observó Fernando en
todo el contorno de la ciudad. En la escarpa de los picachos de Santo
Domingo y en los altos de Santa Bárbara, todo era baluartes y trincheras
formidables. Hacia la otra parte, en Porfía y sobre el Puy, vio también
cortaduras y reductos. Las puertas de la ciudad por el camino de Puente
la Reina, y en la entrada del paseo, y en las cabeceras de los puentes,
donde arranca el camino de Viana, eran verdaderas fortalezas. En el
centro de la ciudad vio bastante tropa, bandadas de clérigos, corrillos
de oficiales en la plaza frente a San Juan, y en la calle Mayor; observó
el descuido de policía como signo de bárbara guerra, los pisos
desempedrados, formando charcos fétidos; cerrados los comercios, los
establecimientos de pelaires, los talleres de carda de lanas, los
batanes y tintes, en completa paralización y abandono. Recomendole
Echaide que anduviese lo menos posible por la ciudad, manteniéndose en
el parador al cuidado de las bestias, lo que le pareció muy bien, y
pronto hubo de advertir la sabiduría de este consejo, pues en el
parador, y en una próxima tienda de bebidas con algo de comistraje, pudo
observar a sus anchas, sin despertar la menor sospecha, el estado de la
opinión; sólo con poner su oído en las disputas, vio claros los dos
partidos que agitaban el cotarro pretendentil.

En esta parte decían que era de necesidad fusilar a Marato; en aquella,
que no había decencia si D. Carlos no se limpiaba de las alimañas que se
le comían vivo, el cura Echevarría, el capuchino Lárraga, el obispo de
León, Arias Teijeiro y otros tales. Pedían aquí que viniese Cabrera a
enderezar el torcido altarejo de la Causa, pues era el único hombre de
empuje y circunstancias, y allá que la perdición del Rey estaba en los
generales de anteojo y compás, y que los propiamente facciosos que no
sabían leer ni escribir le darían la victoria. En ciertos círculos del
bodegón no se recataban paisanos y militares de hablar pestes de D.
Carlos, que todo lo fiaba de la Virgen, y consultaba sus planes de
guerra con las monjas flatulentas, hartas de bazofia. Los más devotos de
Su Majestad llevaban muy a mal que cuando iban las cosas de la guerra
tan torcidas, y hallándose el país esquilmado y en la miseria, saliese
D. Carlos con la gaita de casarse. ¡Vaya, que tener que aguantar también
Reina, sobre tantas cargas como abrumaban a los pobres pueblos! ¡Y que
no vendría poco finchada la de Beira, ni traerían poca fachenda sus
damas y gentiles-caballeros, todos con atrasadas ganitas de trono y de
parambombas reales, en medio de los desastres y de las inseguridad de la
guerra!

Metían su cucharada en los coloquios \emph{Quilino} y Muno, expresando
las opiniones más contrarias a todo buen criterio, como seres nacidos
para discurrir al tenor de los animales; y así pasaron tres días en
tranquila sociedad y distracciones de bodegón, dando tiempo a que
entregara o colocara Echaide la carga que llevó, y que tomase otra,
consistente en piezas de paño del cuento 24, casimiros y bayetones
estrechos, barriles de vino y algunos trebejos de calderería. Nada
tenían ya que hacer allí. Dos días antes de la llegada de Echaide había
salido Maroto para Alsasua, de donde seguiría hacia Cegama y Oñate. La
misma dirección, por caminos y atajos endemoniados, tomó Echaide con su
cuadrilla, escalando los desfiladeros de Andía, y en todas las ventas y
encrucijadas, así como en los puntos guarnecidos, encontraba el arriero
amigotes, con quienes departía del cisco que tan revueltos traía a
castellanos y navarros. Ningún entorpecimiento hallaban en su marcha por
aquellos vericuetos, porque la solicitud con que Echaide desempeñaba los
encargos, y la forma escrupulosa que sabía dar a su neutralidad, le
garantizaban contra todo recelo. Por la noche, ya le cogiera esta en
alguna venta, desmantelada choza o tejavana, echaba mano a su rosario,
obligando a los suyos a secundarle en sus extremadas devociones. A los
clientes atendía con solicitud, cobrándoles a conciencia, y en el
servicio de todos desplegaba tanta honradez como puntualidad. Jamás
trajo ni llevó soplos referentes a movimientos de uno y otro ejército, y
en ambos tenía protectores y amigos que apreciaban sus raras cualidades
de ermitaño trajinero.

Bajando de los puestos de Aralar hacia Cegama, les cogió un temporal de
nieve y ventisca, que por algunos días les tuvo prisioneros sin poder ir
adelante ni atrás, defendiéndose contra el frío en unas cabañas de
pastores. Hasta las soledades inhospitalarias en que se guarnecían
llegaba el rumor de la ola revolucionaria que por abajo corría. También
allí, viejos que parecían salvajes pedían que descuartizaran a Maroto y
lo echaran a los perros, y soldados errantes que iban a unirse con sus
cuerpos abogaban por que se ahorcase a Guergué con las tripas de Arias
Teijeiro. Con hogueras se defendían los trajinantes del horroroso frío,
que recrudeció la cojera de \emph{Quilino}, obligándole a unos andares
enteramente grotescos. Aprovechando una clara, avanzaron por la
vertiente abajo en busca de mejor abrigo: en una casa en ruinas, donde
se agazapaban media docena de soldados que venían de Ormáistegui, y unos
leñadores míseros, se trabó disputa tan brava sobre quién o quiénes
habían traído el reino a tanta perdición, que no se pudieron contener en
la pendiente de las palabras a los hechos, y algunos palos tocaron a
Calpena, que hubo de aguantarlos con cristiana mansedumbre, porque el
coraje no delatara su condición, tan bien disfrazada. Entre el tumulto,
y mientras se frotaba la parte dolorida, se oyó su voz protestando en
esta forma: «Ridiós, si vus digo que razón tenís más que serafines. Que
afusilen a Maroto, si vedis que no cumple; pero si cumple, escabecen a
los empostólicos que le suerben el seso al soberano Rey\ldots{} Eso vus
digo, y tamién que afusilando, afusilando, al que no ande aderecho,
veredes la faición como una balsica de aceite.»

---Mia tú, \emph{Patarrastrando}; pues que te afusilen, que aderecho no
andas.

---¡Otra!, que me arrimatis con gana. No paicis amigos, ridiós!\ldots{}

---Desapártate, bruto, y no rebuznes de pulítica.

Un tanto repuestos y desentumecidos en Cegama, arrearon para la noble
Oñate, y en ella dieron fondo en un día de lluvia torrencial,
chapoteando en el lodo, caladitos, y con parte del cargamento averiado.
Albergados en un parador de la calle \emph{Zarra}, advirtieron inquietud
grave en el vecindario y en la gente de tropa. La noticia de que habían
sido presos y sometidos a un Consejo de guerra los generales Zaratiegui
y Simón de la Torre, a paisanos y tropas les traía muy alborotados. En
las cuadras del parador vieron a no pocos individuos que se recataban
para leer papeles impresos repartidos por los agentes de Muñagorri, el
escribano de Berástegui, que alzado había la bandera de \emph{Paz y
fueros}. Al siguiente día, despejado ya el cielo y seco el fango de las
calles por un furioso viento, vieron escenas interesantes que revelaban
el gran rebullicio de la opinión y el descontento de unos y otros. Casi
a las puertas de la iglesia mayor, un grupo de soldados insultó a dos
clérigos que salían de sus devociones, y a la entrada de la calle de
Santa María, un grupo alborotaba con amenazas a la Intendencia, por la
detestable calidad de los víveres. Corrían voces de que se habían
interceptado cartas de Maroto a generales de Isabel, proponiendo
condiciones para dar el pasaporte a Don Carlos; mas alguien sostenía con
visos de autoridad que la tal correspondencia era falsa, obra pérfida de
los fueristas de Muñagorri y de otros intrigantes que hormigueaban en la
frontera, protegidos por el Gobierno de Madrid y el Comodoro inglés Lord
John Hay, vulgarmente llamado \emph{Lorchón}.

Y como en Oñate nada tenían que hacer, sabedor Martín de que en un punto
no lejano podrían realizar el fin oculto de su viaje, partieron hacia
Vergara, y a esta renombrada villa llegaron en ocasión que no se cabía
en ella de tanta tropa como entraba por el camino de Durango. Era el
ejército de Maroto.

\hypertarget{xx}{%
\chapter{XX}\label{xx}}

Lo primero que hizo Echaide, después de albergar sus caballerías,
rompiendo como pudo por entre la militar turbamulta, fue dirigirse a
cumplir sus devociones de costumbre ante el célebre \emph{Cristo} de
Montáñez que se venera en la iglesia parroquial de San Pedro de Ariznoa.
Largo rato estuvo allí en compañía de \emph{Quilino} (a quien ya más
comúnmente llamaban \emph{Patarrastrando}), y cuando acabaron de rezar
ante la imagen con extraordinaria edificación, en la misma nave obscura
del templo le dio las instrucciones que creía pertinentes.

\emph{«Patarrastrando,} hijo mío, tú te vas al parador, y allí te estás
como un santico hasta la hora de la cena. Échate a dormir si te parece;
no hables con nadie, que aquí, motivado a estar el Rey, hay soplones y
mequetrefes de la policía. No te fíes de nadie, ni aunque sea sacerdote,
o, pongo por caso, canónigo. Te duermes; después que cenemos te diré a
dónde tienes que ir, con respeto, hijo, con muchísimo respeto.» Puntual
le obedeció D. Fernando, y por la noche, después de cenar, entregole
cuatro botellitas de aguardiente, con encargo de que las llevase a una
señora muy principal del pueblo, llamada Doña Tiburcia Esnaola,
habitante detrás de la iglesia donde habían venerado al \emph{Cristo}.
No tenía pérdida: era un caserón de sillería, con gran escudo cubierto
de negros paños, y en el portal había una imagen de Nuestra Señora,
alumbrada con dos farolitos. Fue \emph{Patarrastrando} con las botellas,
cogidas con muchísimo cuidado para que no se le cayeran en el camino, y
hallada fácilmente la casa, entró, y una moza lozana le llevó por la
bruñida escalera hasta la estancia donde salió a su encuentro una señora
bien vestida, no joven, aunque de buen ver, la cual le mandó poner las
botellas sobre la mesa; y no había acabado de hacerlo, cuando se abrió
una puerta, y en el marco de ella apareció gallarda figura de militar
cincuentón, con bigotes, rostro pálido, rugoso y grave, puro en la boca,
el ceño ligeramente fruncido. El mensajero se acercó pronunciando una
singularísima palabra: \emph{Inquisivi}. Dijo el militar: «pase usted,»
y tras él y \emph{Quilino} se cerró la puerta, quedando todo en
silencio, pues la señora se retiró por otro lado. La casa parecía dormir
con descuidado y dulce sueño.

Descabezaba Echaide el primero de aquella noche en la cuadra del
parador, rodeado de animales y arrieros, ya cerca de las doce, cuando le
tiraron de una pata. Resolviose y dijo: \emph{«Quilino}, ¿eres tú?
Túmbate, hijo, y duerme; o echaremos antes un tercio de rosario si te
parece.» Así lo hicieron, y entre los murmullos del rezo perezoso metían
las cláusulas de un coloquio breve: «¿Despachasteis?»

---Sí. padre.

---¿Tenemos algo más que hacer aquí?\ldots{} ahora y, en la hora de
nuestra muerte\ldots{}

---No, padre.

---Temprano cargamos y salimos. \emph{Amén}.

Y temprano cargaron y salieron, \emph{amén}; que a Echaide no le hizo
mucha gracia la marejada que en la villa advirtió, entre ojalateros y
marotistas, entre la camarilla \emph{impostólica} y los que llamaban
moderados. Hablábase de nuevas prisiones de jefes, de fuertes agarradas
entre la Reina y el Obispo Abarca. D. Carlos se había casado en
Azcoitia, y llevaba consigo a la Reina con séquito palatino muy vistoso,
dentro de la modestia que la guerra imponía. Pero el Infante D.
Sebastián, hijo de la de Beira, se peleaba con Echevarría; y Arias
Teijeiro con Maroto; y este con toda la turba palaciega; y la Reina se
volvía \emph{moderada}; y el Rey quería contentar a todos, y a nadie
daba gusto; y con el nombre de su hijo, el llamado Príncipe de Asturias,
apuntaba un nuevo cisma fundado en la abdicación; y Villarreal y Elío,
famosos caudillos, ponían el grito en el cielo, renegando de los
apostólicos; y S. M. frecuentaba los locutorios de las monjas para
pedirles consejo y oír sus inspirados vaticinios, haciéndose digno de
que se le aplicaran, con más razón que a su hermano, los ridículos
versos de Rabadán:

Las pobrecitas vírgenes claustrales de tratar a su Rey están ansiosas:
don Carlos, con entrañas paternales, ¡ha dado en visitar las religiosas!

Hablando de todo lo observado en Vergara, que era mucho y bueno,
partieron hacia Beasaín, para tomar la vuelta de Navarra, siguiendo
itinerario distinto del que habían traído. Nada les ocurrió digno de ser
contado, sino que uno de los burros enfermó en el paso de Lecumberri
para bajar a Irurzun, y resultando ineficaces los remedios que le aplicó
Martín, maestro en artes veterinarias, el pobre animal entregó su vida a
la inmensidad y su carne a los buitres. Inútiles fueron también las
diligencias para sustituirlo, y, al fin, no hubo más remedio que
malvender parte de la carga del difunto asno, y llevar a cuestas,
repartida entre todos, la restante. Trabajosa fue la expedición en
aquellos días de riguroso invierno, y hasta Puente la Reina, donde
llegaron a primeros de Diciembre, no tuvieron descanso ni abrigo. Pero
la salud no les faltaba, si bien \emph{Patarrastrando} empezó a sentir
verdadero el impedimento muscular que había sido fingido, lo que
felizmente tuvo compostura con los veterinarios remedios que le aplicó
Echaide. En esto, encontraron a León con su ejército, que victorioso
volvía de las acciones de Sesma y Los Arcos. Contaban los soldados
maravillas de audacia del General y heroísmos de su tropa. Animados por
tan feliz suceso, apresuraron los arrieros el paso, para llegar pronto a
la tierra baja, pensando que el palizón recibido por Maroto era parte a
precipitar la solución que todos deseaban. En dos jornadas se pusieron
en Sesma, y al siguiente día pasaron el Ebro por Lodosa, picando hacia
Logroño. A media legua de la ciudad, dijo Echaide a Quilino y Urrea que
se quedasen a dormir en una venta que allí hay, mientras él avisaba al
General del feliz arribo de la embajada: creía complacer a Su Excelencia
dándole ocasión de escoger sitio y hora para recibir a D. Fernando antes
de que este entrara en la ciudad. No iba descaminado el ladino arriero,
pues su precaución agradó mucho al de Luchana, y a la mañana siguiente
mandó recado con el mismo Echaide para que Quilino le esperase en la
Fombera, preciosa finca, propiedad de Doña Jacinta, a corta distancia de
la venta que antes se menciona. Allí pasó el día D. Fernando, y se
entretuvo recorriendo las huertas de frutales y los variados recreos de
tan hermosa posesión, que aun en pleno invierno tenía mucho que admirar.
El arbolado de sombra no desmerecía de la rica colección de peros y
manzanos; espléndido era el corral, bien poblado de aves; y por fin, un
brazo de la Iregua penetraba en la finca, formando en ella como una ría
o lago delicioso, donde su república tenían ánades y patos. Sirvió el
guarda a D. Fernando la comida que al objeto mandaron los señores, y por
la tarde llegaron Espartero y Doña Jacinta, sin compañía de ayudantes ni
de ninguna otra persona, y lo primero fue reír ambos de la pintoresca
transfiguración del caballero, jurando que no le habrían conocido si le
encontraran fuera de aquel sitio. Diéronle luego noticias muy buenas de
Pilar, y con las noticias las cartas que le aguardaban, dejándole que a
su gusto se entregase al deleite de leerlas, o al menos de repasarlas
rápidamente. El rostro del caballero mientras leía revelaba su regocijo
y satisfacción. Su madre gozaba de excelente salud, y aunque
desconsolada por la ausencia de su querido hijo, se alegraba de verle
campeón de noble empresa, propia de un caballero cristiano y español.
Enterado de lo que más vivamente le interesaba, se puso el caballero a
la disposición del General, que ya impaciente aguardaba una pausa en los
afectos filiales. Apartose la Condesa con la mujer del guarda para pasar
revista al ejército de gallinas, y en tanto Espartero y D. Fernando,
paseando despacito, hablaron todo lo que quisieron. Desde lejos se podía
ver el rostro del héroe expresando ya el asombro, ya la ira; oía muy
atento, pronunciando algún monosílabo con vigoroso apretón de quijadas o
arqueo de sus negras cejas.

Imposible transmitir la conversación, que hubo de quedar en vaguedad
incierta, como nebulosa de un suceso histórico. Otras conversaciones se
relatarán; esta no. El oído indiscreto, procurando apoderarse de las
ideas allí manifiestas, sólo pudo coger algún concepto deshilvanado.
«¡Pero ese hombre está loco!---dijo Espartero pisando
fuerte.---¡Pretender que se conserven en la persona de D. Carlos los
honores de Rey\ldots{} y que a la de Beira también la declaremos Reina!
Pero dígame usted, joven, ¿cuántas reinas vamos a tener aquí? La pobre
España será el país de las innumerables Reinas\ldots{} Esto no puede
ser.»

Y después se oyó también este cabo suelto: «No puedo conceder más que el
reconocimiento de la mitad de los grados adquiridos en el ejército
carlista. De Madrid me han venido indicaciones para que reconozcamos la
totalidad\ldots{} pero no puede ser. ¿A dónde vamos a parar? ¿Qué
presupuesto resistirá un Estado Mayor semejante? La guerra nos ha hecho
pobres y la paz nos hará mendigos\ldots{} No puede ser\ldots»

Y por último, cuando ya terminaba la conferencia: «De aquí a mañana
rectificaré algunas de mis condiciones, a ver si recortando yo y
recortando él llegamos a una inteligencia. ¡Qué demonio de hombre! Me
había hecho creer que se hallaba en mejor disposición\ldots{} ¿Pero qué
espera? ¿No teme que los apostólicos, sanguinarios, sedientos de
venganza, llenos de ira y de veneno, la fusilen el mejor día?» Refirió
Fernando lo que en su viaje había observado, la sorda revolución que a
modo de volcán mugía en las entrañas del partido carlista, poco antes
formidable en su potente unidad guerrera y religiosa; mas nada de lo que
dijo fue novedad para el Conde, que por su bien organizado espionaje no
ignoraba nada de lo que ocurría entre el Ebro y el Pirineo. Concluyó el
General diciéndole que se preparase a volver con nueva embajada, pues
una vez iniciado su servicio, no había de renunciar a la gloria que le
reportase. Replicó el caballero que no ambicionaba gloria, si por esto
se entienden los honores y exterioridades que acompañan a los grandes
hechos. Se contentaba con la satisfacción de su conciencia, y si lograba
coadyuvar a obra tan hermosa, de su parte en el triunfo gozaría en la
obscuridad en que pensaba encerrar para siempre su vida.

«¡Qué pena, D. Fernando---le dijo la Condesa,---dejarle a usted aquí tan
solito! Pero ya que se ha impuesto, por amor de la patria, tantos
trabajos y privaciones, habrá hecho buen acopio de paciencia. Ya
cuidaremos de que nada le falte aquí!»

---Con paciencia dicen que se gana el cielo, y con ella he ganado yo el
afecto de ustedes, para mí tan caro.

Despidiéronse muy afectuosos, y Calpena se quedó solito, dueño de aquel
vergel, en cuyas amenas anchuras daba expansión a su espíritu, libertad
a sus pensamientos, para que vagasen de la mente a la naturaleza y de la
naturaleza otra vez a casa. Exploraba el porvenir, tratando de ver la
probable salida de aquel arduo negocio, y ponía en orden todos los datos
y conocimientos adquiridos para deducir de ellos la histórica
resultante. Recordaba la tenacidad de Maroto en el sostenimiento de sus
proposiciones, y no veía fácil que tal dureza se ablandara sin el
castigo de la guerra. Al propio tiempo, si sufría una cruel derrota,
quedaría imposibilitado para negociar, porque los apostólicos le
quitarían el mando y quizás la vida. Veía la situación del General
faccioso erizada de peligros y dificultades, y le admiraba por el tesón
con que afrontarla sabía. No estaba Maroto, no, exento de moral
grandeza, y miraba al interés patrio, tratando de conciliarlo con los
restos, que restos eran ya, del Estado carlista. Con agrado recordó
Calpena el trato franco y ameno del caudillo de las campañas chilenas,
del vencido en Chacabuco. Su despejo manifestábase desde las primeras
expresiones, y su conocimiento del personal del absolutismo revelaba un
observador sagaz. Poco afortunado en los campos de batalla, lo era en la
organización, en adiestrar hombres y componer muchedumbres para la
guerra. Hubiera sido quizás mejor político que militar. Su destino hizo
de él uno de esos hombres que, dotados de amplia fuerza intelectual, no
aciertan jamás con los caminos derechos, y llegan siempre a donde no
querían ir.

Dos días no más permaneció D. Fernando en la deliciosa Fombera, trabando
amistad con patos y gallinas, dando migajas a pájaros y peces, hasta
que, recibidas del General las nuevas instrucciones, que se hizo repetir
para grabarlas bien en su memoria, partió con la cuadrilla al alba de un
día de Diciembre. Con carga de vino, siguieron todo el curso del Ebro,
aguas abajo, para vadearlo por Tronconegro, y tomar allí la dirección de
Salvatierra por La Guardia y Peñacerrada. Lo que menos pensaba Calpena
era pasar por la patria de las niñas de Castro en tan extraña
disposición, y fue para él un rato triste y al propio tiempo placentero
recorrer la villa a media noche, ponerse a la sombra del caserón de
Castro-Amézaga, cerrado a piedra y barro; reconocer también la casa de
Navarridas, la iglesia parroquial y demás sitios que renovaban en su
alma memorias dulces. Contempló largo rato, a la claridad de la luna
creciente, el palacio donde había vivido tres meses, cuidado por los
ángeles, y miraba una tras otra las ventanas, señalando por ellas las
piezas y el interior grandioso, el cuarto donde él dormía, el de las
niñas, el comedor, y hasta se fijó en las tejas, por donde pensaba que
andarían los mismos gatos de su tiempo. Ningún rumor se sentía, fuera
del cantar de gallos en el corral de la casa. Esta dormía con el sueño
del justo\ldots{}

¡Oh, cuánto le embelesó aquella paz, aquel solemne descanso de la vida
laboriosa, de las conciencias puras! ¡La paz! Él la quería, la deseaba
con toda su alma. Por la paz del Reino trabajaba, y si Dios le concedía
también la suya, procuraría, sí, agasajarla dentro de la envoltura más
propia de aquel bien supremo, que era la obscuridad junto a seres
queridos.

\hypertarget{xxi}{%
\chapter{XXI}\label{xxi}}

De su arrobamiento le sacó el amigo Echaide, y salieron arreando para
Peñacerrada. Llevaban, en sentido contrario, el mismo camino que había
recorrido con las niñas en el éxodo de Oñate. ¡Cómo recordaba su
travesía en el carro, y las escenas de Salvatierra, el encuentro con
Serrano, la batalla con el \emph{Jabalí}, la herida, y por fin Aránzazu
con sus habitaciones de mendigos y el humilde sepelio del pobre D.
Alonso! La vieja historia se le presentaba página por pagina, como un
libro repasado al revés.

En Aránzazu les cogió la Noche Buena, y allí la celebraron entre amigos,
que de Echaide lo eran algunos de los leñadores en las ruinas
aposentados. Pudo enterarse Calpena del bienestar que todos debían a las
generosas niñas, y aunque algo habló de esto con sus huéspedes, no quiso
darse a conocer ni repetir la triste historia. Cenaron y bebieron
alegremente arrieros y leñadores, y \emph{Santo Barato}, hombre sin
semejante para toda fiesta y bullanga, cantó villancicos en castellano y
en vascuence, y bailó la jota y el aurresku con mozos y mozas de
Aránzazu, en medio de grande algazara. Aun en aquellas alturas apartadas
del trajín social se oía el resoplido de la profunda revolución de la
Causa, signo indudable del cansancio del País, y de las ganas que tenía
de sacudirse tanto parásito militar, frailesco y político.

La primera parada después de Aránzazu fue en Mondragón, donde Echaide
tenía parientes, una prima hermana casada con el sacristán de la
parroquia, otro primo albéitar, y muchos y buenos conocimientos. Era el
sacristán hombre muy leído, se sabía de memoria las \emph{Gacetas}
carlistas, y estaba al tanto de cuanto pasaba en las regias Cortes,
empezando por la del \emph{legítimo}. Apostólico furibundo, abominaba,
como el Obispo de León, de los generales de anteojo y compás, y en ellos
veía el trastorno y ruina del Reino. Hablaba campanudamente buen
castellano, con ínfulas y tonillo de orador, y creía que la única
imperfección del régimen absoluto era \emph{no tener Cámaras}. Con
buenas y sabias Cámaras, que debían ser presididas por un Obispo, y
sujetas al rigor dogmático, podrían los hombres de estudios ilustrar las
cuestiones; y el Rey desde su real tribuna lo oiría todo, conservando la
libertad de hacer lo que le diere la real gana, que para eso era ungido
de Dios.

Bueno: pues mientras cenaban Echaide y los suyos en casa de los primos
con cierto aparato de limpieza y mejor comida que de costumbre,
disfrutando de tenedores y hasta de mantel, se lanzó Videchigorra, que
tal era el nombre del sacristán, a unas pomposas peroratas que, con ser
enteramente hueras, no cuadraban a la rusticidad de su auditorio.
Calpena le oía con afectada admiración, y el orador observaba en el
rostro de él, como en un espejo, los efectos de su elocuencia. Entre
tanta hojarasca, algo hubo de encontrar Quilino que no le estorbaba para
su conocimiento total de las cosas públicas y de la guerra. Era en
verdad peregrino que, habiendo estado en Logroño tan cerca del hombre
que en aquel tiempo movía los hilos del retablo político, no se hubiese
enterado de la representación dirigida por él a la Reina, documento que
alborotó a España toda. Pero en la soledad de la Fombera, ¿quién había
de informarle de cosas tan graves, como el mismo General no lo hiciese?
Sofocado ya del derroche oratorio, mas sin perder su hinchada serenidad,
Videchigorra decía: «Si hay revolución en nuestro Reino, no es floja
zaragata la que han armado los corifeos de allá. Ahí tenéis al espadón
de los \emph{libres} echando a la titulada Gobernadora un memorial
sedicioso, irreverente, que no es más que la voz de su enojo contra
Narváez, por si le dan o le quitan el mando de cuarenta mil
\emph{pistolos}, los cuales no han cogido el titulado fusil con otro
objeto que desbaratar la preponderancia del \emph{rotulado} Conde de
Luchana\ldots{} ¿Qué es esto? Celos y envidias, señores; verdadero furor
masónico por la dominación. ¿Qué vemos ahí? El nefando \emph{Progreso},
negación de Dios; el execrable culto de la Libertad, negación de la
Virgen\ldots{} ¿Qué quiere el apócrifo General y Conde de engañifa? Pues
quiere la dictadura militar; quiere ser Atila, señores, el azote del
género humano, y venirse luego acá con la guillotina, la Convención, el
culto de los dioses paganos y la libertad de la imprenta. Espartero,
bien lo veis, impone su autoridad a Doña Cristina, y le disputa el
gobierno de las facciones de Madrid, las tituladas Cortes, Ministros,
Oficinas y Arbitrios. El masonismo quiere tener en una mano las arcas
reales, y en otra los soldados que con engaño y violencia defienden el
falso Trono\ldots{} Quiere por medios infernales derribar el Trono
verdadero, que se apoya en el lábaro, y traernos el imperio del error y
del materialismo\ldots{} Pues si por el lado político no es floja la
revoltura de los idólatras de la Constitución, por el lado militar van
de capa caída, y no tardarán en recibir el golpe de gracia. No negaré
que hemos tenido algún tropiezo, como el de Los Arcos, que debió ser
gran victoria y no lo fue por la ineptitud de un Maroto; pero nosotros
al gran triunfo de Morella podemos añadir orgullosos el que ha logrado,
no lejos de Caspe, el invicto entre los invictos, el Macabeo de España,
D. Ramón Cabrera, neto Conde del Maestrazgo. Supisteis, y si no, ahora
lo sabéis, que en los campos de Maella protegió de tal modo el Señor las
armas de nuestros leales, que, a este quiero, a este no quiero, hasta
que se hartaron de matar no dieron paz a los sacros fusiles y a las
cortantes bayonetas. En la refriega cayó muerto el corifeo que les
mandaba, un titulado General Pardiñas, que gozaba fama de temerario, y
los prisioneros fueron mil y cuatrocientos. Quedó el campo de Maella
empapado en sangre de cristinos y cubierto de cadáveres, en lo que se
vio clara la mano del Altísimo y su protección a la divina bandera de D.
Carlos. Nuestra Generalísima merece mayores homenajes y devociones más
pías que la que le tributamos. Adorémosla, reverenciémosla; no apartemos
su imagen de nuestro pensamiento, ni su amor de nuestros corazones.
Seamos macabeos, seamos valerosos y píos, hasta dar cuenta de la hidra,
señores, de la bestia masónica y atea. Y pues hemos cenado en paz y
gracia de Dios, juntándonos en esta honrada casa, vosotros humildes y
sencillos, como los apóstoles, yo más ilustrado que vosotros, yo que os
supero en conocimientos, mas no en fidelidad al Rey ni en entereza para
defenderle; pues hemos cenado con bendición y hasta con cierto regalo,
recemos ahora el rosario santísimo, para que Dios nos mantenga en su
gracia y en la pureza de nuestra fe.

\emph{«Amén,»} dijo Echaide sacando el rosario, y amén repitieron
\emph{Quilino} y los demás, preparándose al acto religioso, tan
favorable a una buena digestión.

No se vieron libres los pobres trajinantes, a la hora del descanso, de
un nuevo chaparrón oratorio del Sr.~Videchigorra, que furioso les siguió
a la cuadra para contarles picardías mil descubiertas por los agentes de
la Superintendencia de policía. Astutos emisarios del masonismo se
habían introducido en el campo carlista, sembrando la discordia con
escritos infames, con falsificadas epístolas, en que se suponían tratos
y contubernios de los leales con la rebeldía de Madrid. El diablo andaba
suelto y con más cara de paz, que le servía para engañar a muchos
incautos. Enmascarados de fueristas venían también los prosélitos de
Muñagorri, titulándose nuncios de paz. ¡Buena paz nos dé Dios! En su
delirio habían concebido el diabólico plan de robar la persona augusta
de D. Carlos en Azcoitia, sorprendiéndole con un centenar de hombres
osados que de Fuenterrabía se embarcarían para Guetaria, y de este
puerto se precipitarían sobre la residencia real en la obscuridad y
silencio de la noche. ¿Pero qué había de hacer Dios más que desbaratar
proyecto tan sacrílego? Bastole al Señor producir entre los infames
regicidas una confusión semejante a la de Babel, de modo que cuando se
congregaban en Fuenterrabía para poner en práctica la villana idea,
viéronse de súbito imposibilitados de comunicarse sus pensamientos,
porque querían decir una cosa y decían otra, y las palabras no salían
nunca conforme a la voluntad, sino expresando lo contrario de lo que
esta disponía. Y hombre hubo además que, creyendo hablar vascuence,
resultaba expresándose en lengua tudesca o polaca, cosa en verdad
inaudita, prodigio sublime con que el Señor justiciero anonadó a los
enemigos de su causa.

\emph{«Amén,»} murmuró Echaide, casi dormido.

Roncaban ya estrepitosamente los demás, con excepción de \emph{Quilino},
que le paró los golpes con una tirada de bostezos, sobre los cuales
trazaba la señal de la cruz. Con esto, Videchigorra se retiró, según
dijo, a escribir una carta urgente, y allá dentro se le sentía charlando
con su mujer. Durmiose el fingido arriero hasta media noche, en que se
levantó para dar aguas a las bestias y aparejarlas, pues querían salir
de madrugada; y hallándose en este trajín, vio que por el patio
adelante, bien iluminado por la luna, avanzaba como fantasma la flexible
figura del parlero sacristán. Tembló el pobre mozo. «Pues eres tú---le
dijo el fantasma---el único que está despierto, a ti confío mi encargo.
Es una carta, hijo; una carta de grandísimo interés, que entregarás en
Durango en la propia mano del señor a quien va dirigida. ¿Sabes leer?
¿Sí? Pues entérate bien del sobrescrito, y que se te grabe en la memoria
el nombre de uno de los más entusiastas defensores de la Religión y del
Rey, \emph{D. Eustaquio de la Pertusa}. No será malo que añada para tu
gobierno las señas del tal sujeto: talla mediana, color moreno, edad
próximamente como la tuya, ojos pequeños y sagaces. Y para satisfacción
tuya y mía, agrego que en ese señor verás a uno de los que con más
ahínco se consagran a la persecución de intrigantes y al descubrimiento
de las perfidias que nos consumen; hombre tan piadoso como valiente y
leal, que daría su vida por el Rey, como la daríamos tú y yo si
necesario fuese\ldots{} porque\ldots{} te diré\ldots{} óyeme.»

Por quitarse de encima la nube dio Quilino su palabra de entregar la
carta en propia mano, y apartose todo lo que pudo, prefiriendo la
sociedad de los burros a la de los oradores. Mas no le valió su
esquivez, porque el otro se le fue encima, brincando por sobre dornajos
y montones de escombros, y le acometió ferozmente con este metrallazo:
«Los que no tengan fe, váyanse con Maroto; los que duden, pónganse
faldas y dedíquense a las faenas mujeriles\ldots»

En esto llegó Echaide, que fue pararrayos de Calpena, porque sobre él
descargó la nube, sin que pudiera defenderse con el rosario, por no ser
ocasión de ello. Partieron al fin de madrugada, y a la salida, por el
camino de Elorrio, fue con ellos el hablador, arreándoles con el látigo
de su palabra. Recomendoles que mirasen bien con quién hablaban, y que
no se dejasen tentar de ningún intrigante; que no acogiesen papeles
impresos, y que si a sus orejas llegaban las \emph{chinchirrimáncharras}
de algún \emph{pacífico fuerista neto}, lo pusiesen en conocimiento de
la autoridad. No tuvo Echaide más remedio que desenvainar el rosario, y
\emph{Santo Barato}, hombre poco sufrido y de malas pulgadas, empezó a
recoger pedruscos con la idea de abrirle el camino del cielo, por un
martirio semejante al de San Esteban.

Dejándole atrás, le vieron hablando con un árbol, hasta que pasaron dos
mujeres, y de parola con ellas se volvió a Mondragón. Ya muy adelantados
en el camino, Echaide, quedándose atrás con Quilino, le dijo: «Nos
guardaremos de dar esa carta del primo Videchi, que, como has visto,
tiene en la cabeza un molinillo, y no piensa ni dice más que disparates.
Conozco a ese Pertusa, que es uno que anda en enredos de los
\emph{fueristas netos pacíficos}; otro más agudo y metidillo no lo hay
acá. Ha engañado al pobre Videchi haciéndole creer que trabaja por lo
\emph{impostólico}. Todos esos tunantes hacen juego doble, y se fingen
lo que no son para trabajar por lo suyo, que es hacer tabla rasa de
estos pequeños reinos y mandar a D. Carlos a tomar aires. La carta de
Videchi no es más que una lista de los \emph{netos} de Mondragón, y otra
de los \emph{ojalateros}, que allí son pocos, y explicaciones de lo que
tiene cada uno y de lo que vale. Debemos, pienso yo, no dar el papel,
que nos pondría en el compromiso de hablar con ese Pertusa, mequetrefe
muy entrometido que querrá entrar en confianzas para curiosear.
Andémonos con tiento, hijo. Nosotros a nuestro trajín, a nuestros
burros, a la buena con todos, sin que nadie pueda decir que quitamos o
ponemos. Dame la carta, y yo me encargo de echarla en el buzón de la
eternidad.

Pareciole muy juicioso a Calpena el acuerdo de su amigo y jefe; mas
desprendiéndose del encargo, no pudo apartar de su mente en todo aquel
día y la siguiente noche la imagen del condenado \emph{Epístola}.

\hypertarget{xxii}{%
\chapter{XXII}\label{xxii}}

Como recuerdo espectral, de esos que pintan y entonan la figura y voz de
personas ausentes, perseguía D. Eustaquio al caballero, quien no podía
menos de admirar la travesura del astuto aragonés. Habríale gustado
penetrar el secreto de sus artimañas, sorprender entre sus ágiles dedos
los hilos que manejaba; observar la sutil hipocresía con que se
infiltraba en la sociedad que quería corromper. La llegada al arrabal de
Pinondo, en Durango, donde se albergaron, borró aquellas impresiones,
que no revivieron hasta el día siguiente por la tarde, en ocasión de,
hallarse el caballero rendido de cansancio y un poco febril. Grande
había sido el ajetreo de entregar y recoger mercancía; como unas quince
veces recorrió cada uno la distancia entre el parador y el centro de la
villa, sin que nada de particular les ocurriese. En retirada iban hacia
su vivienda Quilino y Muno, atravesando por frente a los arcos de la
parroquial de Santa María, cuando vieron salir de esta una luenga
procesión con estandartes y cruces, seguidas de imágenes, y un concurso
inmenso de fieles de ambos sexos, sin que faltaran cantores y un lucido
cleriguicio. Movidos de la curiosidad, aproximáronse los dos arrieros, y
confundidos entre la multitud pudieron admirar la devoción que en los
rostros y actitudes de todo el gentío se manifestaba, y aun hubieron de
sentirse influidos por la masa, que les atraía y les arrastraba sin que
de ello se dieran cabal cuenta. En dos filas larguísimas iban con lento
paso, a un lado y otro del palio, personas de clases diferentes: señores
y pueblo, paisanos y militares, todos con vela encendida, agregando su
voz a la salmodia de los curas. Sin fin de mujeres se agolpaban
fluctuando, onda de paño negro y caras compungidas, y metían también sus
desentonadas voces chillonas en el coro litúrgico. El acto tenía por
objeto impetrar del Altísimo el remedio del mal humano, pidiéndole
expresamente que pusiese fin a las discordias que hacían de su elegido
Reino un campo de Agramante. Cada cual agregaría quizás de su cuenta las
peticiones que creyera más prácticas, como la extinción del
\emph{marotismo}, o la ruina de Muñagorri y su canalla.

Observaba el arriero las caras que iban pasando, graves, mirando al
suelo con beata compostura, y de pronto le dejó suspenso la presencia de
D. Eustaquio de la Pertusa, que marchaba en la devota fila con vela y
escapulario, emulando con los más celosos en devoción y recogimiento.
Mas no podía sostener su papel de clavar en tierra las miradas, y las
esparcía de rato en rato por la muchedumbre, sin quitar de ellas la
expresión santurrona. Viole D. Fernando pasar cerca de sí, y Quilino,
cogiendo del brazo a Muno, apartose de la procesión, abriéndose paso a
fuerza de codazos, pues ya todo lo había visto y no le quedaba nada que
ver.

Antes de llegar a Pinondo, la fiebrecilla que se le había presentado
tomó más fuerza. Intenso escalofrío le corría por todo el cuerpo, y
apenas podía tenerse en pie. Arreglado el mejor lecho que fue posible,
en la cuadra donde todos dormían, se acostó el hombre, perseguido por el
espectro de Pertusa con escapulario y vela, andando al compás de la
procesión con devoto paso y actitud, y echando de soslayo sobre el
gentío el rayo de sus sagaces ojuelos. Y si por el órgano de la vista se
hallaba el buen caballero bajo la sugestión del \emph{Epístola}, por el
oído se le entraban los campanudos discursos de Videchigorra. No podía
su voluntad librarse de ambas visitas espectrales: a Pertusa le tuvo en
su retina toda la noche, y no cesaba de oír el insufrible moscardón,
repitiendo su oratorio zumbido: «¿Qué pretende el corifeo de los
\emph{libres?} La dictadura, tras de la cual vendrá el satánico reinado
de la diosa Razón\ldots{} Pueblos engañados por el masonismo, despertad,
venid\ldots{} Carlos os abre sus brazos amantes; Carlos pío, Carlos
soberano, a todos perdona. Su Reino es la paz, el dogma, la obediencia.»

Pasó la noche intranquilo, apeteciendo bebidas frescas y azucaradas.
Urrea le arropó cuidadoso, dándole de beber a menudo, y se mantuvo a su
lado vigilante. Sin descabezar un sueño hallose al siguiente día más
despejado, y durmió algunos ratos, descansando así de la visión de
Pertusa como de las retóricas de Videchigorra. Pero al caer de la tarde,
hallándose solo en la cuadra, ya invadida por la penumbra, se creyó
nuevamente víctima de su delirio\ldots{} ¿Cómo podía ser esto si los
sentidos del enfermo gozaban de suficiente despejo para no confundir las
impresiones mentirosas con las reales? El individuo que vio acercarse a
su lecho humilde no era una engañosa imagen, sino el propio
\emph{Epístola}, en su natural ser, todo vivacidad, agudeza y travesura.

«No se me esconda, Sr.~D. Fernando---le dijo cauteloso, bien seguro de
que nadie le veía.---Le conocí en la procesión, a pesar del bien
dispuesto disfraz. Un poco difícil me ha sido después dar con usted;
pero guiado por mi olfato finísimo, ya lo ve\ldots{} he descubierto a mi
hombre.»

Creyó Fernando de malísimo augurio semejante encuentro, y habría dado
cualquier cosa de valor por que el \emph{Epístola} que veía fuese
creación de la fiebre. Sintió impulsos de agarrar el palo que próximo al
lecho tenía, y ahuyentar a garrotazo seco la importuna imagen, por
desgracia muy real; pero luego estimó peligroso este procedimiento, por
el escándalo que ocasionar podría. Dejó pasar un rato; y mientras el
entrometido aragonés se despachaba a su gusto con demostraciones de
cordial amistad y respeto, discurrió qué resortes emplearía para
librarse de él, o por lo menos para alejarle sin comprometer el
incógnito riguroso que quería guardar.

«Mire, D. Eustaquio---le dijo,---si cree usted que yo vengo en esta
traza con algún fin de intriga política, se equivoca grandemente; y como
me contraríe y me salga con alguna necedad que estorbe mis planes, sepa
que no lo sufro, pues no soy hombre que se deja burlar por el primero
que llega. Yo le aseguro que si no me guarda las consideraciones que
debe a mi persona y al disfraz que he tomado, por motivos y razones que
nada tienen que ver con el carlismo, yo le aseguro, repito, que si no se
conduce usted, con respecto a mí, como si no me hubiera visto, le haré
entender lo que es discreción y delicadeza, en caso de que me convenza
de que no lo sabe.»

---¡Pero, D. Fernando, si yo\ldots! No se sulfure, óigame\ldots{}

---No tengo que oír nada. Usted es quien tiene que andar con tiento,
pues al menor descuido le meto una bala en el cráneo y me quedo tan
fresco.

---¡Pero, señor, ilustre señor\ldots{} si no me ha dejado explicarme!
¿Cómo puede suponer que yo me acerco a usted con intenciones que no sean
leales, y con todo el respeto que usted se merece? Por Dios, devuélvame
su estimación, que en un momento de desvarío parece negarme. Créame,
señor: no me ha pasado por el magín que se haya usted puesto en esa
facha para fines y enredos políticos; eso se deja para los desdichados
que no tienen qué comer, como un servidor\ldots{} En cuanto le vi a
usted, mi finísimo olfato y mi penetración, que nunca fallan, me dijeron
que el Sr.~D. Fernando anda en estas comedias por cuestión de amores.
Con esta idea, créalo, hallé fácil explicación a su presencia en
Durango\ldots{} ¡Como que esperaba verle a usted por acá, cambiado de
rostro y vestimenta! He aquí la razón de haberle reconocido al primer
golpe de vista.

---Pues ya que su penetración por esta vez ha dado en el clavo, pues de
amores se trata y por amores vengo, suspendamos aquí la conversación, y
váyase por donde ha venido, que yo en mis soledades vivo, y con ellas me
basta para lo que me propongo. Sea usted discreto y déjeme.

---¿Está bien seguro, señor, de que no me necesita?

---Segurísimo.

---Piénselo, piénselo, y si en ello se confirma, me retiraré con la
promesa y palabra que doy de respetar fielmente su secreto. Pero yo
confío en que un poco de reflexión le convencerá de que puedo serle de
grande utilidad en su empresa, por no decir aventura.

---Paréceme que no, Sr.~D. Eustaquio. Nada puede usted hacer en obsequio
mío.

---¿Ni aun allanarle algún camino\ldots{} decirle lo que ignora,
señalarle el punto donde encontrará el cazador la res en cuyo
seguimiento viene?

Los ojuelos penetrantes del \emph{Epístola} turbaron a D. Fernando, que
no supo ya en qué actitud ponerse, ni si tomar o no en serio el orden de
ideas a que el astuto aragonés quería llevarle. Picado de la curiosidad,
y no queriendo ser menos agudo que su interlocutor, le dijo:

«Agradeciéndole sus buenos deseos de servirme, debo manifestarle que sus
informaciones llegan tarde, pues ya sé todo lo que me conviene saber.»

---En ese caso, señor mío, nada tengo que añadir, sino que me perdone lo
que creerá oficiosidad. Si usted sabe dónde ha de encontrar a la dama,
el cómo y cuándo de poder verla y hablarla, resulto, en efecto,
inútil\ldots{} No obstante\ldots{}

---¿Qué?\ldots{}

En el colmo de la confusión, y viéndose en un terreno desconocido, D.
Fernando no sabía qué postura tomar. Pertusa, atravesándole con su mirar
fino, prosiguió así:

«Permítame que le haga una pregunta: ¿la vio usted ayer tarde en la
procesión?»

Afirmándose en el nuevo terreno, que aún no conocía, Calpena respondió
con intención capciosa: «Sí, señor, la vi.»

---Iba con Doña Prudencia. D. Sabino formaba en la fila, dos cuerpos
delante de mí.

---Todo lo observé, sí señor---aseguró Don Fernando haciéndose cargo del
nuevo terreno a que su destino le traía, por mediación de aquel
diabólico sujeto.---¿Para qué tengo yo los ojos en la cara, Sr.~D.
Eustaquio?

---Naturalmente: lo que no ven los ojos de un enamorado no lo ve el
mismo sol. ¿Y sabe usted también la residencia de la hermosísima Doña
Aura?

---Sí, hombre, sí\ldots{} ¿Cree usted que yo he venido aquí a perder el
tiempo?

---Pues si todo lo sabe, no soy un amigo útil, sino un visitante
fastidioso, y con la venia del Sr.~D. Fernando me retiro.

Mirándose un rato en silencio, rivalizando los ojos de uno y otro en
penetración y picardía; y como Pertusa repitiese su ademán de retirarse,
le agarró Calpena por el faldón, diciéndole: «Aguárdese usted un
rato\ldots{} Deje que me levante\ldots{} Estoy un poco enfermo; pero no
es nada\ldots{} puedo salir\ldots{} Hablaremos en la calle\ldots{} aquí
no conviene.» Vistiose presuroso el caballero; dio algunas vueltas por
la estancia y las cuadras próximas para cerciorarse de que no le
observaban sus compañeros de arriería, y echose a la calle precedido del
aragonés. Ya era de noche.

«Vámonos por estos callejones---dijo el caballero guiando,---que no nos
conviene encontrar gente conocida, y hablaremos\ldots{} Pues sí, Sr.~de
la Pertusa, si usted me descubre el nido de ese lindo pájaro, practicará
una de las obras de misericordia: enseñar al que no sabe.»

---¿No decía yo que podría serle de gran utilidad? Al fin me salí con la
mía. Por lo que veo, usted supo que la familia reside en Durango.

---Eso sí\ldots{} pero ignoraba\ldots{}

---Su casa. Ahora mismo vamos allá; pero tomémoslo con calma, que es
lejos, al otro lado de la población, en el barrio de Curuciaga.

---Aunque sea en el fin del mundo, vamos allá.

---Pues sí, D. Fernando: cuando le vi a usted, mi primera idea fue
suponer que venía con algún intríngulis político. Hoy por hoy, conspiran
aquí hasta las piedras\ldots{} Después me acordé de haber visto a Doña
Aura, y dije: «No, no: este viene a la querencia antigua\ldots{} Es
natural.»

---¿Y qué sabe usted de Zoilo?

---Que desde lo de Peñacerrada no se tiene de él noticias buenas ni
malas. Está loco. ¡Miren que meterse a guerrear en la partida o división
de Zurbano!\ldots{} No me sorprenderá que venga el mejor día el relato
de su muerte.

---¿Se supo por Iturbide que Zoilo se batió en Peñacerrada?

---Sí, señor, por Pepe Iturbide, que se pasó a los alaveses, y con ellos
estuvo hasta que su padre y los amigos le cogieron y se le llevaron a
Bilbao.

---Muy bien. Dígame otra cosa: ¿trata usted a D. Sabino Arratia?

---¡Anda!\ldots{} somos amigos. Y pues no debo escatimar a usted mi
confianza para merecer la suya, le diré\ldots{} Sé que hablo con un
caballero, y que mis informaciones quedarán entre los dos.

---Hágase usted cuenta de que habla con esa pared.

---Pues D. Sabino es de los que ha logrado traer a la devoción de mi
Causa\ldots{}

\emph{---Paz y fueros\ldots{}}

---Bajito, que aquí cada pedrusco es una oreja. D. Sabino es mío, y no
quiere más que el acabamiento de esta estúpida guerra, y que se vaya
\emph{Isidro} a que le mantenga el Rey de Francia.

---¿Entra usted en casa de D. Sabino?

---No, señor: nos hemos visto y hablado en casa de un amigo común,
también de los de acá.

---¿Qué otras personas de la familia de Arratia, a más de Aura y
Prudencia, están aquí?

---Ninguna más. El venirse a Durango es por averiguar el paradero de
Zoilo, pues se dijo que había caído prisionero en una acción que se dio
el mes pasado en la parte de Campezu o de Contrasta, no estoy seguro.

---¿Y trajeron acá los prisioneros?

---Algunos\ldots{} Pero entre ellos no ha parecido Zoilo.

Interrogado acerca de Ildefonso Negretti, si era difunto o había sanado
de sus trastornos de cabeza, nada pudo contestar D. Eustaquio. En esto,
atravesaron todo el pueblo, y pasado un camino campestre entre paredes
de piedra seca, franqueando después un llano pantanoso, en el cual
vieron dos lóbregos edificios y una iglesia negra, cuya espadaña se
recortaba sobre el cielo azul estrellado, llegaron a Curuciaga, barrio
compuesto de dos docenas de casas esparcidas entre huertas, prados y
arroyos. La noche era serena y fría, y sobre todos los objetos extendía
el relente una humedad glacial. Embozado en su manta, D. Fernando sentía
calor, y el corazón le palpitaba furiosamente. Parándose, Pertusa le
dijo: «¿Ve usted esta tapia con portalón? ¿Ve usted más allá, dentro del
espacio cerrado, el cuerpo alto de una casa grandona? Pues aquí viven, y
ahora están cenando. Por esta otra parte se ve la luz del
comedor\ldots{} Allí, allí están\ldots{} Pero que no se le pase a usted
por las mientes llamar ahora, ni\ldots{} En fin, como ignoro sus
intenciones, no sé qué debo aconsejarle\ldots{} No hemos venido, pienso
yo, más que a explorar el terreno, a conocer las posiciones del enemigo,
el grado de resistencia de la plaza\ldots{} ¿No es eso?»

Completamente abstraído, cual si no viviera ya su espíritu en este
mundo, D. Fernando no decía nada, y por los dos hablaba el otro. La
viveza y locuacidad del aragonés se anticipaban a las ideas del que
parecía privado del don de la palabra. Las miradas, el alma toda del
caballero, se anegaban en aquel iluminado espacio cuadrangular, ventana
de un aposento donde había personas vivientes, pues había luz. Y
aquellas personas, que él a una sola redujo, la soberana persona
fundamental, ¿qué haría, qué diría, qué pensaría?

\hypertarget{xxiii}{%
\chapter{XXIII}\label{xxiii}}

«Ya voy entendiéndole, señor---dijo Pertusa, cuya grande agudeza
sorprendía los pensamientos del caballero.---Lo que usted quiere saber
ahora es si podremos hacer un reconocimiento del interior de la casa, de
sus entradas y salidas, de los espacios y rincones de la huerta
delantera y del corral; todo ello desde alguna de las casas próximas. Si
tal es su deseo, le diré que, dejando pasar la noche, podremos observar
cuanto nos diere la gana por esta parte de acá\ldots{} Véngase\ldots{}
deme la mano\ldots{} saltemos este pedazo de pared destruido\ldots{} por
esta otra parte hay una casita, que también tiene huerta. ¿La ve? Un
tejado con abolladuras, y bajo el alero un balcón jorobado y un
ventanico tuerto. Pues aquí se albergan dos señoras petisecas que hace
treinta años eran poderosas y ahora viven de la caridad\ldots{} Son
amigas mías, furibundas apostólicas, que adoran a D. Carlos y le ponen
velas\ldots{} ¿Pero esto qué importa? Mañana vendremos, y mediante una
limosna nos franquearán su vivienda para hacer de ella la garita o
atalaya más cómoda que se pudiera imaginar\ldots{} Y ahora, vámonos,
Sr.~D. Fernando, que el rondar es peligroso en estos tiempos y en estos
barrios extraviados. Los espías hormiguean. Todo el suelo que pisamos
dentro y fuera de Durango, mejor dicho, todo el territorio de Vizcaya y
Guipúzcoa, está minado\ldots{} hablo figuradamente\ldots{} y las minas
cargadas, no con pólvora, sino con ideas y sentimientos, reventarán
pronto. Ya no es fácil encontrar dos carlistas que piensen del mismo
modo en las innúmeras cuestiones que agitan la Causa. Quizás, quizás
exista la unanimidad en la idea de que \emph{Isidro} no sirve para el
caso. Las ilusiones de esta buena gente caen por el suelo. Vámonos de
aquí poquito a poco, y por el camino seguiremos hablando, ya digo, con
cautela, que ahora no hay palabra segura, ni sílaba que no comprometa.»

Como se había dejado llevar, dejose traer Calpena, sin oponer réplica ni
comentario a los dichos de su compañero. Andando, miraba a las
estrellas, lo que no dejó de ocasionarle algún tropezón, cuyas
consecuencias evitaba cuidadosamente el \emph{Epístola} echándole una
mano. Llegados al centro, rompió el silencio D. Fernando con estas
palabras: «Quedemos, amigo Pertusa, en reunirnos mañana temprano, y
fijemos para el caso la hora y sitio más convenientes.»

---¿Sitio? El pórtico de Santa María. ¿Hora? La que usted quiera, pues
para mí todas son iguales\ldots{} Ya que entre los dos se establece la
confianza, le diré que desde esta tarde ha empezado a faltarme la
seguridad que aquí disfrutaba yo, que si antes no inspiraba sospechas,
ahora me tienen entre ojos, no por descuido mío, sino por soplos
indecentes\ldots{} Me ha entrado un grandísimo miedo de estos infames
polizontes, y no me encuentro con ánimos para volver esta noche a mi
casa. Antes de salir en busca de usted di fuego a todos los papeles cuya
conservación no creía de importancia, y los que no debo destruir los he
dado a guardar a un amigo de toda confianza, veterinario, el cual se
avino a prestarme este favor, a condición de que albergaría mis papeles,
mas no mi persona\ldots{} en fin, que no puedo contar con que me deje
pasar la noche en su casa. Seamos claros como buenos amigos, y
confiémonos el uno al otro sin reparo alguno. Yo pensaba que usted, a
cambio del precioso servicio de ojearle a Doña Aura, me concedería el
amparo de admitirme en la cuadrilla de arrieros, al menos hasta salir a
cuatro leguas de Durango por una parte u otra, mejor por la parte de
Elorrio, Mondragón y Vergara\ldots{} ¿Qué dice?\ldots{} ¿Es atrevimiento
lo que pido?

No dio contestación D. Fernando a la propuesta del \emph{Epístola},
porque al punto de oírla vio los gravísimos inconvenientes de acceder a
ella. Sin duda Echaide no permitiría que semejante pájaro se les
agregara, ni el caballero tampoco habría de consentirlo. Detestable
compañía era la de D. Eustaquio, pues si por nada del mundo se le debía
dar conocimiento del contrabando que los arrieros llevaban, tampoco a
estos convenía correr la suerte del conspirador \emph{fuerista}, ni
exponerse a participar de los palos y encierros con que le amenazaba la
Superintendencia. Visto así por D. Fernando con toda claridad, se
apresuró a cortarle los vuelos, sin meterse en explicaciones, que
verdaderas serían indiscretas, y mentirosas le repugnaban. «Con nosotros
no puede usted venir, amigo Pertusa---le dijo,---ni en la posada donde
estamos, y cuyo dueño es furibundo apostólico, debo yo albergarle. Lo
más prudente es que nos separemos esta noche. Yo me voy a mi casa, y
usted se guarecerá donde pueda hasta el amanecer\ldots{} ¿Qué dice? ¿Por
qué suspira? ¿Es que no halla sitio seguro donde pasar la noche? ¿Tiene
usted miedo?\ldots»

---Sí señor, un miedo horroroso; no puedo ocultarlo.

---En ese caso, no es hidalgo que yo le abandone, siendo su deudor por
el servicio de esta noche y por el que me prestará mañana. Pasaremos
juntos las horas que faltan para la salida del sol, y tempranito
buscaremos medio de introducirnos en la casa de las señoras vecinas de
D. Sabino Arratia.

---Eso haremos, sí, señor\ldots{} ¡Ay!, me tranquiliza el verle a usted
junto a mí toda la noche. Dígame, señor: ¿lleva por casualidad armas?

---Hombre, no: en el parador dejé las pistolas.

---¿Por ventura lleva dinero?

---Eso sí\ldots{} alguno llevo.

---¡Ay, qué alivio!---exclamó el \emph{Epístola} recobrándose de su
pavura.---Arma formidable es el dinero, y en ocasiones más eficaz para
la defensiva que las piezas de a veinticuatro. Puesto que usted posee
proyectiles del precioso metal, ya me vuelve el alma al cuerpo: ha de
saber que entre mantenerme con miseria y atender a los gastos de mi
comisión, se me han ido hace dos días los últimos maravedises. Ahora nos
volvemos hacia Curuciaga, y pediremos albergue en un bodegón de las
últimas casas de la villa, en el cual suelo comer algunas noches. Los
dueños de él son buena gente, y tienen trato con la policía; pero los
pajarracos que van por allí son de esos que venderían a \emph{Isidro}
por un pedazo de pan: tal es el hambre a que les tiene reducidos el
titulado ministro de Hacienda. En cuanto vean ellos el \emph{in utroque
felix}, caen atontados. Bastará con media onza para cada uno en el caso
de que se nos presenten\ldots{} Vámonos por este callejón a salir al
campo, que los caminos solitarios son los menos peligrosos.

Siguiole D. Fernando, y ya en descampado, franqueando cercas y cruzando
prados, se le soltó más la lengua al \emph{Epístola}, ya repuesto de sus
angustias por la compañía de un señor benévolo y rico, aunque no lo
pareciese por el artificio de su plebeya facha. «Somos felices, Sr.~D.
Fernando---decía, ayudándole a saltar zanjas y a romper zarzales,---y
podrá usted, en todo el día de mañana, dar fin a su aventura, que
entiendo es de las más bonitas que pueden presentarse a un hombre de su
calidad. En la tienda de Zubiri nos recogeremos para pasar la noche, y
en cuanto aclare el día nos colamos en la casa que ha de ser atalaya
nuestra, vivienda de dos señoras que se alegrará usted de conocer, la
una un tanto poetisa y con su poco de latín, la otra muy pagada de su
finura y cháchara social, ambas sesentonas, y aún me quedo corto, muy
gustosas de recordar sus tiempos de grandeza, que deben de ser los de
Maricastaña. Le bastará a usted correrse con media onza, que será para
ellas como si en la casa se les metiera el Espíritu Santo. No son
vizcaínas, sino navarras, de la parte de Cintruénigo, huérfanas de un
general de la guerra del Rosellón, y en su tiempo tuvieron aquí mucha
propiedad, que perdieron por mala cabeza del marido de una de ellas, D.
Gaspar de Oñabeitia. Aquí se las conoce por \emph{las niñas de
Morentín}, nombre que les daban el siglo pasado, y que viene
perpetuándose de generación en generación. Hemos de inventar un bonito
ardid para darles la media onza, pues como limosna de un desconocido no
han de aceptarla, y ello será preciso fingir una carta del propio
\emph{Isidro}, o de Arias Teijeiro, lo que yo puedo hacer muy
lindamente, porque \emph{domino} la letra de casi todos los señores de
la cámara y camarilla, en la cual carta se les dirá que por premio de su
devoción al Soberano y de su lealtad bien probada, se les manda aquel
recuerdito, que también podrá ser un \emph{pequeño óbolo} de S. M. la
Reina\ldots»

Replicó a esto D. Fernando que pues las señoras niñas eran naturales de
Cintruénigo, y en esta villa navarra tendrían lejana parentela y quizás
relaciones, no era preciso que D. Eustaquio se molestara en fingir
cartas del Rey ni de sus adláteres: más eficaz sería, para el objeto de
cohonestar la limosna, un artificio que al caballero le pasaba por las
mientes. En ello se convino, y llegados al lugar donde debían pasar la
noche, llamó Pertusa, les abrió una mujer gorda, soñolienta, y entraron
a ocupar dos camastros en la trastienda, entre pellejos de aceite y de
vino, sacos de maíz y haces de hierba. Descansaron sin que nadie les
molestase, y por allí no recaló ningún polizonte ni persona alguna que
intimidarles pudiera. Durmió Pertusa, veló el caballero, recalentándose
el pensamiento con ideas resucitadas que se peleaban con las novísimas,
y al amanecer, el \emph{Epístola}, después de platicar en la tienda con
el patrón, fuese a D. Fernando y le dijo gozoso: «Por milagro de Dios
nos hemos librado de la canalla, señor mío, y para mayor seguridad, si
hemos de pasar el día en estos arrabales, no será malo que demos al
bueno de Zubiri una de las medias onzas que destinábamos a los podencos
del absolutismo. Untándole así los hocicos a este buen hombre, que,
entre paréntesis, me estima, le tendremos a nuestra devoción para negar
que hemos pasado aquí la noche, si preciso fuere, y despistar y
confundir a la maldita Superintendencia.»

A todo se prestó Calpena, pues aunque comprendía que las sutilezas de D.
Eustaquio no tenían más objeto que tomarle por proveedor de sus
necesidades y alivio de sus deudas, quería recompensarle con favores
positivos su ayuda en aquella campaña. Además, los ingeniosos arbitrios
del aragonés le hacían mucha gracia; daba con gusto la media onza, y
bastante más, por verle desplegar tanto donaire y travesura. Acertados
anduvieron los que de él habían hecho un instrumento de conspiración,
que otro más cortado para el caso no se encontrara en toda la redondez
de la tierra. Serían las ocho de la mañana cuando, previos los informes
y advertencias que Pertusa creyó útiles para entenderse fácilmente con
las niñas de Morentín, a la casa de estas fueron en derechura, tramando
por el camino la fingida historia que debía justificar el soborno y
darle apariencias delicadas. Llamó D. Eustaquio al portalón, y abierto
este por la niña mayor, viéronse en un corral poblado de hermosas
gallinas. Ambas \emph{niñas} se ocupaban en aquel menester, y mientras
la una reconocía con hábil dedo a las aves que debían poner aquel día,
la otra les daba la pitanza de berzas cocidas con salvado, y les
renovaba el agua, y les arreglaba los nidos.

Eran muy parecidas las dos damas: pequeñas, vivarachas, limpias, con sus
pañuelos a la cabeza a estilo bilbaíno, dejando ver sobre las orejas
mechones de purísimas canas; vestidas humildemente, chapoteando en el
fango del corral, con almadreñas, que hacían un \emph{clo-clo} muy
campesino, eco celtíbero sin duda que nos trae los rumores de antaño al
través de cientos de siglos. Doña Marta y Doña Rita acogieron a los dos
mozos con recelo, sobre todo a Calpena, cuya traza no era en verdad muy
tranquilizadora. Mandáronles subir, y soltando las almadreñas fueron
ellas por delante, venciendo con ligereza impropia de su edad los
gastados peldaños de una escalera que marcaba los pasos con gemidos. Lo
primero que vio Don Fernando al entrar en la estancia principal, que
bien merecía el nombre de sala, fue un primoroso altar con multitud de
imágenes vestidas y angelitos desnudos, estampas varias, todo ello
resguardado de las moscas por tules verdosos, y profusión de flores de
trapo con infantil arte dispuestas, y papeles que imitaban el brillo de
la plata y el oro, y rizadas velas sin encender. En el centro de la
mesa, cubierta de blanco paño con encaje había un gran vaso lleno de
agua en sus dos tercios inferiores, lo demás de aceite. En este flotaba
una cruz de lata con puntas de corcho, y en el centro de la cruz ardía
una lucecita modesta, familiar, diminuta, que difundía en torno de sí,
con su débil claridad, cierta confianza dulce y plácida, como un ángel
doméstico representado en la forma más humilde.

En cuanto abocó en la estancia, dándose de hocicos con el altarito, cayó
de hinojos D. Eustaquio, y sus expresivas demostraciones de piedad
maravillaron y entontecieron a las dos señoras. Calpena, con menos prisa
y devoción no tan ferviente, se arrodilló también, y mientras rezaba
entre dientes, observó que en lo más bajo del altar, cubriendo la peana
que sostenía la imagen de Cristo, campaba el retrato de Carlos V,
mediana estampa de colorines. La graciosa lucecita iluminaba el rostro
antipático del Rey (que si algo expresaba era lo contrario de la
inteligencia) y su busto exornado de cruces y bandas. Rezaron también
las dos \emph{niñas}, y una de ellas no quitaba los ojos de D. Fernando,
como si las facciones de este no le fueran desconocidas, o si algo
quisiese deletrear en ellas. Y al verle persignarse y ponerse en pie, se
apresuró a decir: «Si no me engaño, el señor es de Cintruénigo.»

\hypertarget{xxiv}{%
\chapter{XXIV}\label{xxiv}}

---No soy de Cintruénigo, sino de Ablitas---replicó D. Fernando muy
cortés, olvidado del lenguaje baturro que en aquella tierra fingía, y
adoptando su natural dicción,---y traigo para las señoras un encargo del
señor D. Beltrán de Urdaneta, mi amo.

Mudas de asombro, las dos damas hicieron intención de santiguarse, y
después cruzaron las manos. Entretanto, Calpena pensaba que era muy
conveniente abordar sin circunloquios el asunto, para ganar tiempo, para
inspirar confianza.

«¡Jesús mío\ldots{} Beltrán\ldots! ¿Pero es cierto? ¡Acordarse de
nosotras Beltrán!»---exclamó la una mirando a la otra.

---¡Beltrán, ay!\ldots{} ¡Si no le hemos visto desde el año 5,
cuando\ldots! ¡Qué confusión en mi cabeza!

---Sí, mujer: ¿no te acuerdas? En Noviembre del año 5. Estando nosotras
en Tudela, fue a comunicarnos, por encargo de padre, la triste noticia
de la muerte de nuestro hermano D. Luis en Trafalgar.

---¡Oh, Beltrán, Beltrán!\ldots{} Hace cinco años, a la muerte de
Fernando llamado VII, supimos que vivía el primer noble de Aragón, y que
andaba un tanto decaído de intereses.

---Pues aún vive y está bueno---dijo Pertusa, conforme a la lección que
su amigo y él llevaban bien aprendida.

---Y su decaimiento de fortuna---añadió Calpena, aceptando el asiento
que las señoras le señalaron---se ha trocado ahora en grandeza y
abundancia, porque, verán ustedes\ldots{} ¡qué suerte de hombre!, un tal
Francisco Luco, que en la guerra del Maestrazgo perdió a sus hijos, dejó
a D. Beltrán por heredero de todas su riquezas, consistentes en
cincuenta o sesenta ollas de dinero\ldots{} no recuerdo el
número\ldots{} sepultadas en diferentes puntos. Desenterradas lleva ya
como unas cuarenta y pico, y el dinero lo vamos transportando a
Cintruénigo, donde hay una estancia no más chica que esta llena de sacos
de onzas y medias onzas\ldots{}

Las dos niñas se miraban absortas, y luego se pasaban la mano por la
cara como dos gatitos que se relamen limpiándose los hocicos. No
acababan de creer lo que oían, maravillas de cuentos infantiles.

---Y como es D. Beltrán caballero muy hidalgo y generoso, hecho a mirar
por las desgracias ajenas antes que por las propias, decidió repartir la
mitad de aquellos caudales entre familias de su conocimiento que se
hallan faltas de recursos. Cuatro criados del Sr.~D. Beltrán andamos en
este trajín del reparto, y a mí me ha tocado la tierra de Vizcaya, y
todo el señorío pobre que traigo en esta lista\ldots{}

Diciendo esto, sacó el papel en que trazado habían una luenga cáfila de
nombres y pueblos, y después de mostrarlo a las señoras, que en su
aturdimiento y estupor apenas pudieron enterarse de lo que veían, echó
mano al cinto y dio a luz una onza. Momentos antes había pensado,
generoso, duplicar la cantidad presupuesta, por la profundísima lástima
con algo de respeto que la digna pobreza de las \emph{nenas de Morentín}
le infundía.

«Esto es lo que corresponde a las señoras, según mi lista. Pero podrá
tocarles mayor cantidad, pues el amo me encargó que lo resultante de las
partidas fallidas lo repartiese a la vuelta entre los existentes. A
muchos no les hallo; otros han muerto, dejando algún acomodo a sus
familias\ldots»

Cogió Doña Marta la onza no sin cierto recelo; pasó después la hermosa
pelucona a las manos de Doña Rita; la miraron y remiraron por un lado y
otro. De una mano que la sobaba pasaba a otra que la movía para ver el
reflejo. ¿Creyeron las señoras la burda historia tramada por los dos
hombres? Si estos no la inventaron mejor y más fina, fue porque no lo
creían necesario. Una de las \emph{niñas}, la que, según los informes de
Pertusa, hipaba por la poesía y el latinismo, se tragó sin esfuerzo el
voluminoso embuste; la otra, más práctica y reflexiva, debió de ponerlo
en cuarentena; pero esta divergencia de impresiones no impidió la
unanimidad de aceptar y guardar la onza, expresando gratitud al
mensajero y pidiéndole noticias de la familia de Idiáquez. Diolas
cumplidísimas D. Fernando, y agregaron las señoras que habían tenido
cuatro años antes carta de Doña Juana Teresa, mandándoles regalitos y un
delicado socorro metálico, que agradecieron con toda su alma;
escribieron ellas, y hasta la fecha no habían vuelto a tener noticia.
Amplió Calpena sus informes con pormenores mil de las familias de
Cintruénigo y Villarcayo, edad y referencias de los nietos; y después de
oírle atentas y gustosas las dos \emph{nenas}, dijéronle que observaban
cierta discordancia entre su traje y su manera de producirse, la cual
más bien parecía de caballero bien educado. A esto acudió Pertusa con la
manifestación de que el mensajero de D. Beltrán había cursado estudios
mayores en Tarazona, continuando, no obstante su mediana ilustración, al
servicio de casa y familia tan alcurniada.

Tomó luego la palabra D. Fernando para contar cómo el Sr.~de Urdaneta,
que había recorrido media España con la expedición Real, al absolutismo
pertenecía en cuerpo y alma, y ya se le indicaba para Ministro universal
de Carlos V el día no lejano del triunfo y salvación del Reino.
Profesando él las mismas ideas que su amo, podía correr libremente por
el señorío de Vizcaya, sin más precaución que la de alterar un poco su
facha, y hacerla más grosera y tosca, con el fin de que nadie le
supusiera portador de cantidades relativamente cuantiosas. Al llegar a
este punto, parecieron ambas más tocadas de credulidad: a Pertusa le
conocían por sectario furibundo de la realeza carlista; el otro, que
entonces veían por primera vez, parecióles más fino y apersonado que su
compañero, a pesar del pelaje humilde. Recayó suavemente la conversación
en los negocios de la facción, mostrándose Calpena tan entusiasta, que
su fanatismo daba quince y raya al de los más feroces. Tronó contra
Maroto, viendo en su doblez el origen de las desdichas del Reino;
ensalzó hasta las nubes a D. Pedro Abarca, Obispo de León, que debía ser
canonizado por valiente apóstol de la causa de Dios; igualmente
encareció los sublimes talentos de Echevarría, Padre Lárraga y Arias
Teijeiro, y terminó sosteniendo que San Fernando, San Luis y San
\emph{qué sé yo qué} eran soberanos de alfeñique en parangón de la
extraordinaria majestad y grandeza de Carlos V.

Por fin, viendo a las dos \emph{nenas} tan complacidas, amansadas ya y
bien dispuestas para la última suerte, acometieron esta, tomando la
iniciativa el ladino Pertusa. Uno y otro amigo se hallaban fatigadísimos
de la caminata que habían hecho a pie desde Elorrio, y pedían a las
señoras hospitalidad sólo por el día, ofreciendo marcharse a la noche,
pues les era forzoso continuar su viaje hacia Bilbao, llevado el uno por
comisiones graves de la real Superintendencia, el otro por los encargos
que de Cintruénigo traía. Al pronto, las dos \emph{nenas} se mostraron
recelosas, balbuciendo excusas; pero tan expresivo lenguaje usó el
\emph{Epístola} para convencerlas, y con tanta nobleza y franca
cordialidad apoyó el otro las demostraciones de su compañero, que
hubieron de ceder, siempre con un poquito de escama. Agregada por
Pertusa la indicación de que pagarían con largueza el gasto de una
modesta comida, dijeron Doña Marta y Doña Rita que muy frugal tenía que
ser, pues en su despensa no había más que huevos, algo de pan y alubias.
¡Magnífico! Pedir más era gollería.

«Mi compañero Blas---dijo D. Eustaquio, percatándose de la necesidad de
bautizar a su amigo,---está más cansado que yo, y agradecería mucho a
las señoras que le permitieran tumbarse en cualquier aposento de los que
en la casa tienen para guardar trastos inútiles.» Tanta labia y
metimiento desplegó en ello el astuto aragonés, que pasado un rato se
hallaba D. Fernando en un cuarto próximo a la sala, con ventanucho que
dominaba la huerta de la cercana finca. Era una pieza de techo bajo,
atestada de rotos muebles y cachivaches, vestigios luctuosos del antiguo
esplendor de las de Morentín, y no fue difícil improvisar en ella sobre
un arcón vacío, al que se agregó una silla, cubriéndolo todo con mantas,
un camastro de relativa comodidad. Encerrado el caballero en aquel
cuchitril, pudo disfrutar a sus anchas del beneficio de la ventana,
principal objetivo de aquella improvisada comedia. El hueco de piedra,
como de una vara en cuadro, se dividía en cuatro vanos por gruesos
barrotes en cruz. Excelente era el miradero, segura la atalaya, pues
desde allí no sólo se veía todo el huerto vecino, sino algo del interior
de la casa por las abiertas ventanas de esta. Ávido se asomó el
caballero, y un rato permaneció sin ver a nadie.

Siglos le parecieron los minutos: apoyado su pecho en el muro, su
corazón rebotaba contra este, marcando las ansias que transcurrían antes
que la curiosidad fuese satisfecha. Por fin vio una criada, que al
parecer se ocupaba en la limpieza de habitaciones. Un anciano con
almadreñas atravesó la descuidada huerta, en cuyo suelo crecían hierbas
lozanas. Entretuvo el caballero su angustiosa expectativa examinando los
frutales sin hoja, los añosos perales de rugosos troncos arrimados a la
tapia en forma de espaldera, los manzanos escuetos, las higueras
derrengadas, la vieja parra de torcida y áspera cepa, agarrándose a la
pared de la casa, y enganchando en el balcón sus sarmientos más altos.
Junto al muro medianero, entre el corral de Morentín y la huerta de
Arratia, debía de existir un pozo que D. Fernando desde su atalaya no
podía ver; y junto al pozo había sin duda pila de lavar, porque a los
oídos del vigía llegaba rumor de chapoteos en el agua, el golpetazo de
la ropa sobre la piedra, y una voz de mujer canturreando bajito. En
estas observaciones le cogió una súbita sorpresa, que fue como un
rayo\ldots{} En la ventana de la izquierda apareció Aura\ldots{} D.
Fernando, ¡caso inaudito!, tardó algunos segundos en conocerla, en
cerciorarse de que era ella, y más que por el rostro y figura, la
reconoció por la voz, cuando dijo a la mujer que lavaba: «María, por
Dios, ¡qué calma!\ldots{} Ven pronto.» Desapareció de la ventana,
mientras la mujer hacia la casa corría.

Dudó el caballero si lo que había visto era realidad o visión engañosa.
Y de tal modo quedó estampada en su mente la imagen, que continuaba
fijando los ojos en la ventana, no convencido aún de que estaba el marco
vacío. ¿Había ganado o perdido en hermosura la romántica moza? Imposible
discernirlo. Sólo era indudable para él que había engrosado sin perder
su esbeltez y gallardía. El color había cambiado: era más morena; hasta
llegó a parecerle negra. La impresión recibida fue como una serie de
impresiones muy rápidas, de centésimas de segundo; la luz vibrante
cambiaba el color y las líneas. ¿Había visto una imagen temblorosa en
ráfagas del aire?\ldots{} Pasó algún tiempo, durante el cual introducía
el caballero su mirada por las ventanas, como el ladrón que prueba las
ganzúas en ojos de llaves. Creyó sentir la incomparable voz; mas no pudo
entender si reñía o lanzaba notas de júbilo\ldots{} El sol despejó las
neblinas, y se presentaba un hermoso día de invierno. Abrigada por sus
altas tapias, la huerta debía de tener un temple muy grato, y la faja
meridional, bien asoleada, ofrecía en las callejuelas que separaban los
bancales un piso firme y seco. Apareció un gallo pintado con dos
gallinas, y escarbaba descubriendo bichos que entre sus damas repartía.
Un gato vino después, que se paseó con parsimonia inglesa entre las
coles respigadas, buscando ratoncillos campestres; un perro de cuatro
ojos, negro y con las patas amarillas, se dirigió hacia el pozo, después
hacia la casa, grave y meditabundo, y se tendió al sol junto a la cepa.
Pensó Calpena que todas aquellas apariciones de animales anunciaban
nueva sorpresa. La primera que sobrevino no fue muy agradable, pues
consistió en una mujerona alta y bigotuda, que no podía ser otra que
Prudencia, la cual surgió por la derecha dando voces a otra mujer, en
tono displicente. Era cosa de tendederos de ropa, de cuerdas quitadas de
su sitio para amarrar un burro en la pradera, de palitroques caídos y
que debían ser repuestos. Retirose por el forillo derecho encargando que
no faltase leña para la tarde. Su voz desentonada continuó largo rato
sonando a la otra parte de la casa, donde sin duda estaban la cocina, el
corral y leñera. A poco de esto abriose la puerta central de la fachada
que observaba Calpena, la que a un lado tenía la parra y encima el
balcón. Abriola una mujer que barrió las baldosas del umbral y el
empedradillo delantero. El corazón del galán, golpeando furioso contra
la piedra del ventanucho en que se apoyaba, le decía que por aquella
puerta saldría pronto la mayor belleza del mundo\ldots{}

Pasó un siglo\ldots{} En las medias horas veía el caballero piezas
enormes, tiras sin fin de una eternidad que se desarrollaba ante su
espíritu. Oyó rumor de cháchara, risas que indudablemente eran de ella.
Ningún reír humano podía confundirse con el reír de Aura, y pensándolo
así, el caballero apretaba con ira el barrote cruzado de su atalaya,
porque era en verdad muy inconveniente que ella estuviese tan
regocijada, mientras él se estremecía de dolor, amargado por los
recuerdos. ¿Qué motivos tenía para tales esparcimientos del ánimo
gozoso? ¿No estaba su marido ausente?\ldots{} ¿Acaso habían llegado
noticias de él? Era muy probable que nada se supiese, y que continuaran
en la familia los temores y sobresaltos por la suerte del atrevido mozo.
No estaba de más que la esposa, que bien podía ser viuda ya, mostrase un
poquito de gravedad y compostura. En estas ideas le cogió un estupor,
una emoción inexplicable. No veía nada, y veía un mundo salir por
aquella puerta. Más bien temía, sospechaba, por misterioso aviso de su
corazón, la presencia de un caso, de un hecho monstruoso y al propio
tiempo bello, sublime quizás. «Ya viene,» se dijo; y diciéndolo vio que
Aura salía con un niño en brazos.

\hypertarget{xxv}{%
\chapter{XXV}\label{xxv}}

Salió con un niño en brazos\ldots{}

Salió con un niño en brazos. Sólo diciéndolo más de una vez se expresa
la tardanza del observador en darse cuenta de aquel caso natural, tan
natural que ya en los últimos nimbos de su pensamiento lo había
previsto. Pero tardaba en creerlo, y mirándolo, viendo a la madre, como
nunca hermosa; viendo al chiquillo, que parecía robusto, alegre, deseoso
de vivir, hubo de añadir a la evidencia la confirmación de la palabra, y
dijo: «Es ella con su niño, con su niño\ldots{} porque suyo es\ldots{}
Se le ve que es suyo.»

Venía Doña Aura mal vestida, y un tanto despechugada, señal de haber
dado la teta poco antes. No hacía más que saltar al chiquillo, que al
sentirse bañado del aire y del sol empezó a echar unas carcajadas
graciosísimas, elevando sus manos rojas. Saltaba en los brazos, y ella
le decía mil ternuras, y a estas seguían tantos, tantos besos, que el
chico protestaba, prefiriendo los saltitos al refregón pegajoso de los
labios de su madre. Avanzó hacia el lavadero; pudo verla D. Fernando a
una distancia como de seis varas, y reconocer su hermosura, no
disminuida, sino antes bien realzada por nuevas bellezas\ldots{} El
color era más moreno; pero en su tez resplandecía la salud; su seno, más
abultado, hacía resaltar la flexibilidad de su talle. El chiquitín
parecía de cinco o seis meses, de notable desarrollo y viveza\ldots{}
Por un momento se vio D. Fernando sorprendido por la idea de que el niño
se le parecía\ldots{} ¡Qué disparate! Era su pena, que al desgajarse en
aquella inmensa emoción, fluctuaba entre lo inconsolable y los consuelos
comunes, impropios de un criterio sano. Observándole bien, vio que el
niño era el retrato de Zoilo; tenía los ojos de su padre, y en ellos la
chispa del querer fuerte.

Dio Aura la vuelta por entre las coles, y mostraba a su hijo el gallo y
las gallinas, queriendo que entrara en conversación con ellas por el
lenguaje de \emph{pipís}\ldots{} «¡Y esta es la mujer que hace un año
andaba loca por los caminos---pensó D. Fernando,---corriendo tras el
problema de su vida! ¡Y al fin la Naturaleza se lo ha resuelto de un
modo muy contrario a sus deseos de entonces! ¡Oh Dios, oh grandeza del
tiempo y de la realidad! Pensé encontrar una lunática, y me encuentro la
razón misma. Creí encontrar una enferma, y me encuentro una madre. Se ha
curado dando vida a otro ser. Este caballero de meses, este nuevo
Arratia, nos ha conquistado a todos, nos ha devuelto a todos la vida, la
calma, la salud, quitándonos de los puestos que habíamos tomado en el
terreno antiguo, para ponernos en nuevo terreno. ¡Oh vida, oh
naturaleza!\ldots{} ¡Y nosotros, enfatuados con la idea de buscar la
solución en nuestras pasiones, en el juicio nuestro, cuando nuestro
juicio no es más que un pobre ciego sin lazarillo!\ldots{} Debo hacerme
justicia, diciendo que yo había previsto este caso; sí, lo había
previsto\ldots»

Fuera por lo que fuese, ello es que D. Fernando, lastimado por lo mismo
que admiraba, apartose del ventanucho y se sentó, sosteniéndose en las
manos la cabeza, que por la gran pesadumbre de sus ideas difícilmente se
conservaba erguida. Largo rato permaneció en aquella postura, viendo
pasar por la obscuridad de su pensamiento una triste procesión de
imágenes, el maravilloso hallazgo de Aurora Negretti en casa de la
diamantista; el rostro de esta, trasunto de María Antonieta
guillotinada; las figuras burlescas de Milagro y Maturana, y por fin la
persona de Aura en distintos aspectos, siempre hermosa, interesante,
espiritual, resplandeciente de ingenio y hechicera gracia\ldots{} Vio la
escena de Bilbao, la horrible decepción, que parecía desenlace
trágico-tonto y no lo era, pues el verdadero desenlace lo había traído
aquel lindo mocoso, que acababa de tomar el pecho y pronto a tomarlo
volvería. Las rebeldías de ella, sus dudas horrorosas causantes de
locura, ya no eran más que el recuerdo de una dolencia curada, sin dejar
ningún rastro. Nada de aquel trastorno podía volver. El chiquillo era el
médico, era también el amo, y su existencia a todos imponía vida nueva y
nueva conducta.

Al asomarse de nuevo, Aura estaba sentadita en un banco de piedra frente
a la casa, dando de mamar a la criatura. Veíala de espaldas, frente a
Prudencia, que en pie exhibía su figura procerosa a la admiración del
observador. Este la encontró vulgar, antipática. No podía menos de
odiarla; a todos perdonaba D. Fernando menos a la tarasca intrigante,
autora de tantas desdichas. Y al fin no había manera de negarle el
triunfo\ldots{} ¿Habría sido aquella mujer instrumento de la
Providencia?\ldots{} También se hizo el caballero esta pregunta, y por
cierto que no supo qué contestarse. ¡Estaría bueno que la obra de
Prudencia fuera la mejor, la más lógica, y que los equivocados fuesen
los demás y no ella. ¡Oh tiempo, juez y maestro, definidor augusto,
eternamente sabio!\ldots{}

Ocurrió después que asomadas a su balcón las \emph{niñas} de Morentín,
Aura las vio, y ya tapado el pecho y el chico harto, se vino hacia esta
parte saludándolas con mucho afecto. «¡Rey!\ldots{} mira, mira las
nenas\ldots» Y las nenas le decían mil ternezas, y a ella otras tantas.
«¡Qué guapa está usted!\ldots{} ¡Ay!, cada día más hermosa, rebosando
salud\ldots{} Y el cachorro como una bola de manteca\ldots{} ¡Hija, qué
bien lo cría usted\ldots{} da gusto verle, qué guapín!\ldots{} vaya unos
ojos asustadicos. Parece que quiere decirnos algo\ldots» Y Aura repetía:
«Es un pillo: no saben ustedes lo tunante que es\ldots{} Pero malo, malo
de verdad.» Luego los besos restallaban como cohetes. Fernando se retiró
otra vez con el corazón traspasado. Tanto besuqueo le lastimaba.

No tardaron en entrar en el aposento Don Eustaquio y Doña Marta. «¿Pero
qué le pasa a usted?---le dijo esta.---Parece que ha llorado.»

Sí, señora. Padezco una enfermedad muy rara: ello es cosa antigua en mí.
Empiezo con dolor de corazón, y acabo echando un poco de agua por los
ojos. Agua, nada más que agua.

Le compadeció la señora, asegurando que para males de tal naturaleza no
había mejor remedio que el comer. Pronta estaba ya la comida, que era de
las más elementales: tortilla y un plato hecho al horno por Pertusa, con
pan, huevos, tocino, alubias, queso y castañas. Era D. Eustaquio un gran
cocinero, que sabía improvisar manjares exquisitos con las provisiones
de la despensa más pobre. A comer, y a dejarse de penas y de echar agua
por los ojos.

Comiendo en modestísima mesa, con pobre y muy blanco mantel, vajilla
desportillada y cubiertos desiguales, pero todo limpio como el oro,
charlaron de diferentes cosas. La conversación se inició con el tema de
la familia de Arratia, diciendo las señoras que trataban a Doña
Prudencia y su sobrina sin otro motivo que el de la vecindad. De Aura
sabían que a poco de casarse padeció una endiablada enfermedad nerviosa,
\emph{a consecuencia de un susto}; se le trastornó el sentido tan
gravemente que no podían sujetarla, y se lanzó a los caminos, buscando a
un príncipe imaginario, héroe de los cuentos infantiles. Recogida por la
familia, siguió a su locura una temporada de sosiego y de armonía
matrimonial; y al fin, ya estaba la guapa moza curada del modo más
feliz, sólo por la virtud de su alumbramiento, \emph{que le hizo
revolución en la naturaleza}, y por el gozo que le daba el verse madre
de tan precioso niño. Mas como nunca hay dicha completa, la familia
lloraba la ausencia del hijo, sobrino, esposo y padre, el cual era un
valentón a lo D. Quijote y una cabeza desclavijada. Quince meses o más
iban transcurridos desde que se lanzó con otro loco bilbaíno en busca de
aventuras, y a la fecha no se tenían de él noticias directas. Sabían que
estuvo preso en la cárcel de Miranda; que luego le cogieron y embaucaron
los cristinos, afiliándole en sus infames ejércitos, infortunio grande,
¡ay!, pues más vale la muerte que el pecado y desdoro de pelear contra
Dios. Añadieron que las últimas noticias, recogidas de la misma Aura la
tarde anterior, eran que el Zoilo vivía y andaba \emph{con ese Zurbano},
luciendo su bravura, y que D. Sabino había salido nuevamente en su
busca, para rescatarle del cautiverio cristino y traerle a su familia y
a las dulzuras de su hogar. La tal Aurora era una madraza, sin más
demencia que el amor de la criatura, y como esta viviera, no había que
temer nuevos arrechuchos. Así lo aseguraba la sabia Prudencia, cuya
cabeza reunía la ciencia de veinte doctores. Todo su afán era recobrar a
Zoilo, quitándole de la cabeza las locuras guerreras, y cuidándole para
padre, pues convenía traer al mundo tres o cuatro criaturas más, con lo
que se aseguraba la conformidad y curación de la mujer. El matrimonio
viviría pacífico y dichoso, y mientras más fecunda fuese Doña Aura, más
y más felicidades vendrían sobre la familia.

Oyó estas cosas Calpena cuidando de ocultar el interés que en él
despertaban. Por no infundir sospechas no preguntó nada referente a
Ildefonso Negretti, y siguió a las niñas en el sesgo político que dieron
a la conversación. «No puedo creer---dijo Doña Marta,---lo que ayer
oímos: ese fantasmón de Maroto ha separado a trescientos oficiales sólo
porque pertenecen a la \emph{divina intransigencia}, que es el partido
de S. M.»

---Pues créanlo---dijo el \emph{Epístola},---que del D. Rafael no hay
que esperar cosa buena.

---Y mientras no le quiten de en medio---añadió D. Fernando,---no se
enderezará la Causa, que está bastante torcida, como una torre que se
quiere caer.

---¡Caer no, Jesús!---exclamó Doña Rita echando lumbre por los
ojos,---que aún tiene el Rey a su lado muy firmes puntales. El señor
Arias Teijeiro, que en cuanto habla parece inspirado por el Espíritu
Santo, ha dicho: «Señor, los \emph{brutos} llevarán a V. M. a Madrid.»

---Y los brutos---agregó Doña Marta,---son los limpios de corazón y al
propio tiempo valientes y arrojados; que el arte de las armas es por
naturaleza rudo y se da de cachetes con las letras; y el heroísmo no
casa con esas matemáticas que traen acá los militronches de planitos y
anteojo.

---Ello es que la Causa, señoras---dijo Calpena suspirando,---anda
revuelta, y los que adoramos al Rey vivimos con el alma en un hilo. Y
ahora, para afligirnos más, nos salen con que la sacra y católica Reina
también se tuerce, queriendo transacción, que es decir \emph{¡viva
Maroto!}

---Eso sí que no lo creo aunque me lo aseguren frailes capuchinos---dijo
Doña Marta palideciendo.---¡La Reina, la señora Reina\ldots{}
transacción\ldots!

---Es que anda por ahí una nube de pillos---afirmó Pertusa,---pagados
por Muñagorri o por Espartero, que sirven al demonio echando a volar
mentiras. A mí me han dicho ayer que Maroto aseguró a Su Majestad que le
aceptarán los liberales si les concede una chispita de Constitución y
unas miajas de libertad de la imprenta.

---Sí, sí: con eso y con que se declarara que no hay Dios, ya estábamos
todos iguales. Una de dos: o Maroto dimite, o le arrancarán de las manos
el bastón. Para esto se necesita un hombre.

---Un faccioso de ley.

---¿Qué hombre hay aquí capaz de colgarle el cascabel al gato?

---Hay uno, sí: Guergué.

---Pues Guergué---dijo Pertusa dándole mucha importancia,---y otros dos
espadones de mucho brío que no quiero nombrar\ldots{} en fin, los
nombro, pero bueno es que guardemos reserva\ldots; pues Guergué y los
generales D. Francisco García y D. Pablo Sanz le tienen armado el cepo a
D. Rafael, y ustedes han de verle pronto cogido por una pata, ya que por
la cabeza\ldots{}

Como el que despierta de un sueño, Don Fernando recayó de súbito en la
realidad de sus obligaciones, diciendo: «El tiempo vuela\ldots{} ¿Qué
tenemos que hacer aquí?»

Miráronle con asombro las \emph{niñas}, pues más le creían perezoso que
impaciente, y una de las dos (no consta cuál) le preguntó si había de
distribuir en el propio Durango más partijas del donativo de su señor.
Con el tumulto que en su mente habían levantado las recientes emociones,
se le fue de la memoria el embuste urdido para justificar su entrada en
la casa; y al caer en la cuenta de la torpeza con que contestó a la
\emph{niña}, no se cuidó de enmendarla.

«Muy agradecidos estamos a la hospitalidad de las señoras---dijo;---pero
tenemos mucho que hacer, y nos retiramos.»

Mirábale Pertusa, queriendo penetrar el motivo de aquella súbita
retirada; y por no aparecer desacorde con su compañero, repitió:
«Tenemos, sí, mucho que hacer. Es mediodía.» Y las \emph{niñas}
desconfiadas, alzando manteles y recogiendo loza, dijeron: «Entendimos
que en casa permanecerían hasta la noche\ldots{} La verdad, pensábamos
que querían ocultarse, y ni sabíamos ni pretendemos saber el
motivo\ldots{} Pero, pues no hay ocultación, más vale así.»

---Bien podemos---dijo D. Eustaquio,---andar por todo el pueblo con
nuestras frentes muy altas, pues aquí, que yo sepa, no ha tendido sus
redes el \emph{marotismo}\ldots{} Y si las señoras no lo llevan a mal,
volveremos, y nos darán la satisfacción de leernos algunas de las
composiciones poéticas, producto del ingenio de mi señora Doña Marta.

---¡Ay, no, no, D. Eustaquio, por Jesús vivo!---exclamó ruborizada la
señora, en la puerta de la cocina, secando un plato que acababa de
fregar.---El pobre ingenio mío no merece tales honores. Si me entretengo
a ratos perdidos en jugar con las musas, hágolo para mí misma, para
nosotras, o para personas sencillas, no para que se rían de mí los
ilustrados, porque usted, Pertusa, tiene estudios, y el señor, por bien
que lo disimule, no es lo que parece.

---Sea yo lo que fuere---declaró D. Fernando sonriendo,---tendré mucho
gusto en oír los versos de la señora. Se me ocurre que si quiere usted
dar las gracias a D. Beltrán, lo haga en una linda décima, como es uso y
costumbre en las personas agradecidas que saben metrificar.

---¡Oh!\ldots{} ¡qué compromiso! ¡Por Dios, Blas!\ldots{} Pues no es
floja encomienda la que usted me da.

---Y ello, la verdad, no puede ser más razonable---agregó la otra,
ruborizándose también por cuenta de las dotes poéticas de su
hermana.---Sí, Marta: compón la decimita, que ha de ser muy grata al
Sr.~de Urdaneta.

---Y esta tarde---afirmó D. Fernando,---volveremos nosotros a recogerla.
Ea, que no perdono la décima. No valen modestias aquí. Y si quiere usted
componer otra a la Majestad del augusto Monarca, será miel sobre
hojuelas.

---Tema---dijo Pertusa:---Carlos el Grande corta las cabezas de la hidra
\emph{marotista} para fundar sobre ellas su trono.

---¡Ay, ay, ay, qué magno asunto!\ldots{} Eso no es para mí. Señores,
no, no\ldots{} Mi lira es un guitarrillo humilde\ldots{} Para eso se
necesita trompa\ldots{} y lo que es trompa\ldots{} no, eso no me ha dado
Dios.

---Pues con trompa o con guitarra---dijo Fernando, ansioso de
salir,---las décimas estarán listas para cuando volvamos. Señoras,
dispénsennos\ldots{} Hacemos falta en otra parte.

Aún quiso D. Eustaquio, bromeando, entretener algunos minutos; pero a
Calpena se le caía la casa encima; quería salir pronto, huir, ponerse
lejos. Cogió por un brazo a su compañero, y repitiendo las cortesanías
se despidió de las señoras, que hasta la salida les acompañaron,
insistiendo Doña Marta en empequeñecer sus facultades poéticas, y en
ponderar la magnitud del literario compromiso en que sus huéspedes la
ponían. Cuando se cerró el portalón dejando dentro las dos caras de
gatitas blancas y relamidas, D. Eustaquio preguntó a su compañero si
volverían, y la respuesta fue: «Como el humo. Cumplido el objeto que
aquí nos trajo, doblemos esta hoja; y adiós para siempre las
\emph{niñas} de Morentín, adiós su casa\ldots{} y su vecindad. Historia
pasada\ldots{} mundo concluido.»

\hypertarget{xxvi}{%
\chapter{XXVI}\label{xxvi}}

No menos entrometido que curioso, ardía el aragonés en impaciencia por
conocer las intenciones de su amigo y el estado de la que juzgó aventura
de amor. «¿Pero qué, señor D. Fernando, no entramos en la casa de
Arratia? ¿No hemos venido a sorprender y llevarnos a la hermosa mujer
con niño y todo?

---Cállate la boca, simple. Da por terminada la aventura, y no hagas
preguntas a que no he de responder. Alejémonos pronto de este barrio, al
cual no he de volver en todos los días de mi vida.

---¿De modo que\ldots?

---Chitón.

---¿Y ahora?

---Ahora, yo haré lo que me acomode, y tú callarás. ¿Cómo quieres que te
tape la boca: con dos onzas para que acabes de pagar tus deudas, o con
una morrada de las mejores?

---Prefiero la primera de las dos mordazas presupuestas; y aunque en
todo caso mi silencio ha de ser profundísimo, mi felicidad será mayor si
a las dos onzas agrega vuestra señoría una media más.

---Bueno\ldots{} Ya sabes que ahora nos separamos, que no has de pensar
en seguirme, ni en buscarme, ni menos en hablar a nadie de mí.

---Conforme. No necesita encargarme la discreción, pues soy agradecido,
y aunque a veces no lo parezca, \emph{caballero también soy}, como dijo
el otro\ldots{} Si estas razones no bastaran para garantizar mi
fidelidad, hay otra, señor, y es que los dos trabajamos por la misma
causa.

---¿Tú qué sabes? Mi causa nada tiene que ver con la cosa pública.

---Es deber de usted afirmarlo así, y nada contesto; pero si D. Fernando
cumple reservándose, yo cumplo callando lo que mi finísimo olfato me
enseña.

---¿Qué?

---Que andamos en hociqueos con Maroto.

---¿Quién, tú?

---Usted\ldots{} Mis papeles son inferiores; pero a un mismo fin vamos
todos. Con que\ldots{}

---Estás en un error grave.

---Separándonos ahora, yo apostaría\ldots{} que nos encontraremos en
Vergara.

---¿A que no? Yo me voy en busca de Zoilo Arratia, y hasta el fin del
mundo no pararé mientras no le encuentre.

---Pues no irá usted al fin del mundo, sino a Campezu, que por allí anda
Zurbano.

---Abreviemos, que tengo prisa. ¿En dónde te entrego las dos onzas y
media?

---Lleguémonos a la tienda de Zubiri, cuatro pasos de aquí.

Pasado un rato, alejándose de la tienda, repitió D. Fernando sus
amonestaciones acompañadas de una despedida terminante. «Si quieres ser
mi amigo, demuéstrame con hechos que mereces serlo. No me sigas; no me
busques; no hables de mí.»

---Ni sigo, ni hablo, ni busco; pero sí veo\ldots{} y callo.

---Es que si no callaras, no habría de faltar quien te cerrara la boca
para siempre.

---Comprendido.

---Y vete a donde quieras.

---No hago misterio de ello. Voy a Vergara, donde encontraré no pocos
amigos, oficiales de Maroto.

---Ándate con tiento.

---Cuide usted de su pelleja.

Y con un adiós afectuoso y apretones de manos se despidieron, corriendo
D. Fernando hacia el parador de Pinondo, en cuya puerta le aguardaba
Urrea, loco ya de impaciencia y zozobra, después de pasarse la noche y
el día recorriendo las calles del pueblo y todos sus arrabales. No tenía
por qué darle el caballero explicaciones de su ausencia, y entrando en
busca de Echaide, que también estaba con el alma en un hilo, hubo de
soportar resignado la reprimenda que el digno jefe de la cuadrilla se
permitió echarle, valido de la confianza y llaneza que con él gastar
solía en la dura vida de caminantes. El estupor del buen arriero subió
de punto cuando \emph{Quilino} le manifestó severamente su propósito de
trasladarse al territorio donde operaba Martín Zurbano. Halló por fin el
otro fácil modo de conciliar todas las obligaciones, pues despachado
primero el asunto capital en Vergara o Tolosa, tomarían la vuelta de
Salvatierra, para franquear los montes de Andía y bajar a Campezu, que
no era mal camino para Logroño. De acuerdo en esta transacción,
preparáronse para la madrugada siguiente. Pasó D. Fernando muy mala
noche, con ardores de fiebre, atormentado por la persistencia de las
emociones de aquel día. Con más intenso colorido y acentuación más viva
que en la realidad, se le reprodujeron las escenas y figuras observadas
desde la atalaya; de tal modo se poseían de ello su espíritu y su
naturaleza toda, que le dolía la mano derecha de tanto apretar el
barrote que partía en cuatro la luz del ventanucho. Y ya de camino, al
romper el día, sacando fuerzas de flaqueza para seguir a sus compañeros,
continuaba el horroroso dolor de la mano\ldots{} empuñando la cruz de
hierro.

Vergara, donde entraron a media tarde, rebosaba de gente, así militar
como paisana. No sólo había llegado Maroto con su ejército, sino D.
Carlos con todo el matalotaje de su corte vagabunda. Clérigos y frailes
discurrían en grupos, reforzados con señorones administrativos, que
vivían sobre el país, justificando su existencia con el consumo de tinta
y papel en inútiles escritos. Corrillos de oficiales obstruían los
lugares de mayor tránsito: en unos se advertía la intranquilidad, en
otros la tristeza. Cualquier observador que conociese el personal habría
podido advertir que los amigos de toda la vida no se hablaban ya, y se
dirigían miradas recelosas. \emph{Quilino} y \emph{Santo Barato}
anduvieron por calles y plazas, respirando los aires de discordia que
por todas partes corrían. Gran tumulto de gente les atrajo hacia la
iglesia de San Pedro. El Rey con su rebaño apostólico salía de Palacio
para ofrecer al Cristo sus soberanos respetos, y la multitud a su paso
se agolpaba. Bien pudo apreciar Calpena la diferencia entre los
entusiasmos cariñosos que había visto en Oñate y la frialdad de Vergara.
Aún le respetaban; ya no le querían; y por entre la doble fila de sus
vasallos, a quienes congregaba la curiosidad antes que el amor, pasó
Carlos V saludando más severo que amable; que así creía representar
mejor la majestad del derecho divino. Su rostro no ofrecía ninguna
alteración: era un rostro de efigie inexpresiva, de esas que no dicen
nada al devoto que las adora. Su mirada resbalaba en la superficie de
las cosas, y los vasallos no veían en ella más que un convencimiento
tenaz y un fatalismo irreductible. Ni alegría ni tristeza pusieron nunca
sus resplandores en aquel rostro apagado, semejante a los rayos de luz
fingidos con madera y estofa en los retablos churriguerescos. No iba con
él la Reina, que se había quedado en Azpeitia, un tanto aburrida y
descorazonada por el mal giro que tomaban las cosas. Arias Teijeiro
miraba al suelo, Valdespina parecía distraído, y el Padre Echevarría
desafiaba a la multitud con miradas altaneras. Mediano rato duró el acto
piadoso del \emph{Presidiente} en la capilla del Cristo, y de allí se
fue a visitar a las monjas clarisas, cuya priora le fascinaba por el
optimismo de sus juicios y por la gravedad de sus sentencias. Esta
ilustre señora fue la que le dijo que confiara en los \emph{brutos}, que
así como los Apóstoles, sin saber leer ni escribir, habían \emph{sacado
triunfante} la Iglesia de Cristo, D. Basilio y Balmaseda y todos los
\emph{lerdos} de la Causa pondrían en el trono de Madrid al legítimo
Rey.

De vuelta a Palacio, ya cerrada la noche, fue a visitarle Maroto, que
entró con su Estado Mayor, apretando los dientes y atusándose los
bigotes, movimientos en él habituales. Algunos días después fue del
dominio público lo que hablaron D. Carlos y el Caudillo. Pretendía este
que el Rey separase de su lado a los más rabiosos intransigentes; que
cambiara sus ministros por otros menos furibundos y destemplados; que
llamase al orden a los militares y altos funcionarios que abiertamente
conspiraban contra el general jefe de Estado Mayor (que este era el
título de Maroto), y amenazó con sentar la mano a los rebeldes si el Rey
no lo hacía. Como siempre, D. Carlos contestó lo que le inspiraban su
indecisión y pusilanimidad, que sí y que no, y que \emph{ya se
proveería}. Odiaba cordialmente a Maroto, no por mal militar, que no lo
era, ni por desafecto a su causa, sino porque en cierta ocasión de
apuro, atravesando la frontera de Portugal, había soltado D. Rafael en
los regios oídos la interjección más común en bocas españolas, desacato
que el meticuloso Rey no perdonó nunca; pero como le temía tanto como le
detestaba, ni tuvo corazón para quitarle el mando, ni agallas para
entregarle su camarilla.

Esperó Echaide la hora que le pareció más conveniente para mandar a
\emph{Quilino} con el encargo de un barrilito de aceitunas consignado a
la señora Doña Tiburcia Esnaola. Las nueve y media serían cuando partió
el mozo al desempeño de su comisión; como la primera vez, se le franqueó
la puerta, y una criada le introdujo en la estancia donde encontró a la
misma señora, sentadita en el propio canapé. No había puesto aún el
hombre sobre la mesa, al pie del velón, lo que llevaba, cuando la señora
le mostró un papel no más grande que el de un cigarrillo. Con tinta vio
escrita la palabra que servía de contraseña: \emph{Inquisivi}; y debajo,
con lápiz: \emph{Aquí no puede ser}. \emph{Váyase a Estella.}

«¿Se ha enterado usted?» dijo la señora; y ante la respuesta afirmativa
del mozo, rompió el papel en pedazos muy chiquitos.

Con lo dicho queda explicada la salida presurosa de la expedición
arrieril camino de Oñate, para pasar a Salvatierra. Daba prisa D.
Fernando, a pesar de sentir muy quebrantada su salud, y era el más
diligente en arrear por aquellos caminos, pues se le había metido en la
cabeza que siguiendo la ruta de Campezu o de Contrasta le sería fácil
encontrar la brigada de Zurbano, objeto por entonces de su más ansioso
interés. El tiempo se les puso frío y seco, y en Salvatierra hallaron
las aguas cubiertas de hielo durísimo, y los caminos pulimentados por la
humedad cristalizada. Con esto se le agravó al pobre Calpena el
quebranto de huesos que desde Durango traía, viéndose obligado a pedir
fuerzas a su animoso espíritu para continuar el viaje. Faldeando la
sierra de Andía, en dirección de Rióstegui, Urrea le llevó a cuestas por
un empinado sendero, y al fin determinó Echaide desocupar de carga a uno
de los mulos, para transportar al enfermo con relativa comodidad de
todos. Renegaba D. Fernando de su naturaleza, que había creído más
resistente y a prueba de trabajos, y a Dios pedía las ágiles patas del
lobo, o el vuelo de las águilas, franquear sin cansancio aquellos
vericuetos. En los descansos nocturnos, la fiebre le acometía con furia,
y a fuerza de abrigo, verdaderos montes de lana que acumulaban sobre él
sus compañeros, se iba defendiendo. Por fin, en Ulibarri se sintió
mejorado, y la blandura que sobrevino, derritiendo los hielos, fue un
bien para todos, hombres y animales.

Al bajar a Orbizo tuvieron las primeras noticias de Zurbano: días antes,
la helada crudísima le obligó a retirarse a la Solana, y por allí
andaba, entre los Arcos y Dicastillo, aguardando que abonanzara el
tiempo para reanudar las operaciones. Siguieron los cuatro en el rumbo
indicado, y al llegar a Espronceda encontraron una columna de la brigada
de D. Martín, que salió poco después de entrar ellos en el pueblo, sin
que pudieran adquirir las noticias que deseaban. Para dar reposo a D.
Fernando y evacuar con la debida prontitud la diligencia que les
desviaba de su itinerario, determinó Echaide dejar al caballero en
Espronceda con Urrea, bien acomodados en casa de un amigo, y adelantarse
él con \emph{Santo Barato} hasta Muez o los Arcos, para indagar si
Arratia continuaba en la división o se le habían llevado los demonios.
Poco afortunado el primer día, tropezó al segundo con Ibero, por quien
supo que en una acción cerca de Nazar había caído prisionero el Capitán
bilbaíno con otros diez. Conducidos a Estella, Zurbano había propuesto
un canje, sin resultado. Se ignoraba la suerte de los once cautivos,
héroes y mártires. Cuando volvió Echaide con nuevas tan tristes, la
pesadumbre del caballero fue extremada. Creyó a Zoilo perdido para
siempre; vio frustrado el soberbio plan moral que era su ilusión más
risueña: devolver a \emph{Luchu} a su familia, y reconstruir esta sobre
bases inconmovibles. La pasmosa suerte del bilbaíno le había hecho al
fin traición, y sus teorías del querer firme fallaban por primera vez.
Algún dato más, recogido de los labios de Ibero, añadió Echaide, a
saber: que dos días antes se presentó el padre de Arratia en la brigada,
con salvoconducto en regla y cartas de recomendación de Van-Halen y
Buerens, y que sabedor del desgraciado caso, había partido para Estella
en busca de su amigo Guergué, por cuya mediación esperaba libertar al
pobre chico si no le habían quitado la vida. Desorientado en sus ideas,
lleno de acerbas dudas, mandó D. Fernando picar hacia Estella sin
dilación. Tres nombres giraban en su mente describiendo círculos de
fuego: Maroto, Zoilo, D. Sabino.

\hypertarget{xxvii}{%
\chapter{XXVII}\label{xxvii}}

Al pasar por Irache, ya próximos a la ciudad, supieron que Maroto había
entrado algunas horas antes, y que alborotados pueblo y milicia, se
esperaba una colisión sangrienta entre los dos bandos que se disputaban
la opinión y el imperio. Llegados al puente que da ingreso a la ciudad
frente a San Pedro, vieron mucha tropa en las inmediaciones del
castillo. Hallando cortado el paso para el parador, hubieron de dar un
gran rodeo por la ciudad para dirigirse a los Llanos, y al pasar por la
plaza vieron muchedumbre de soldados que a paso de carga traían a un
clérigo amarrado codo con codo, entre vociferaciones brutales y
despiadadas. No tardaron en saber que el tal no era sacerdote, sino el
General D. Francisco García, que se había disfrazado con sotana y manteo
para escapar. Minutos después vieron conducido entre bayonetas a un
hombre pequeño y rechoncho, de fiera catadura, cabello hirsuto, ojos
sanguinolentos, la boca espumante. «Es Guergué---dijo Echaide en voz
baja.---¡Mal día para los \emph{impostólicos}!\ldots» Con no poca
dificultad, por causa del gentío que azorado corría de una parte a otra,
lograron ganar el parador, y allí supieron que los cabecillas
apostólicos, ayudados de paisanos y clérigos, tenían preparada una
sublevación contra Maroto, habiendo seducido previamente a dos
batallones navarros que al aproximarse aquel salieron a tomar
posiciones. En la entrada de Estella por los Llanos y por el camino de
Puente la Reina, habían comenzado a levantar barricadas; pero D. Rafael
anduvo más listo, presentose como llovido del cielo, y tomó medidas
perentorias y radicales en el momento mismo de poner el pie en la
ciudad.

¿En qué se fundaron los netos para proceder así contra el General? Se
habían interceptado papeles en que Maroto y Espartero concertaban la
paz, transigiendo el uno en el reconocimiento de grados, el otro en
aceptar un poquito de Constitución con algo de libertad de conciencia.
Estos papeles existían y se mostraban de mano en mano; mas eran falsos,
obra de los calígrafos del absolutismo, o de los fueristas de Muñagorri.
Ello es que Maroto puso corto espacio entre su llegada y el acto
audacísimo de meter mano a sus enemigos, cogiéndoles en sus domicilios,
en la calle, o donde quiera que se les encontraba. No les dio tiempo a
nada, y en un instante se les cambió la festiva tramoya en trágico
desenlace, las burlas en veras. Pasando el General por la calle Mayor
para dirigirse a la Merced, desde un balcón fue saludado con risas y
chacota. Media hora después, en aquella misma casa era preso el
intendente D. Javier de Uriz, rabioso apostólico. A las cuatro horas de
la entrada de D. Rafael, ya estaban en el castillo los Generales
Guergué, García y Sanz, el Brigadier Carmona, el Intendente Uriz y el
oficial de la Secretaría de Guerra, D. Luis Ibáñez. Cogidas las seis
cabezas del motín, no se entretuvo Maroto en futesas de procedimientos
jurídicos y militares. Sin consejo de guerra, sin auxilio religioso, sin
otro trámite que cargar los fusiles y formar el cuadro, fueron pasados
por las armas de dos en dos. Allí quedaron las seis cabezas de la hidra
hechas pedazos. El estupor no les dio tiempo ni aun para protestar del
bárbaro suplicio. Se enteraron cuando se les mandó ponerse de rodillas.
Nadie se cuidó de vendarles los ojos. Guergué gritó: \emph{viva el Rey},
\emph{viva la religión}; en el rostro del intendente se mezclaron las
lágrimas con la sangre. Los demás gritaron: «¡canallas, traidores!» y
todo acabó.

Retenes de tropa recorrían las calles, y aquí y allí continuaban
haciendo prisioneros. Mudo, paralizado de terror, el vecindario se
refugiaba en sus casas atrancando las puertas. Cerráronse los comercios;
no se veía un clérigo en las calles, y algunas iglesias se incomunicaron
con los fieles devotos. Ordenó Echaide a los suyos que no saliesen, y en
las cuadras del parador, en el despacho de bebidas y en los comedores
próximos, los parroquianos habituales no volvían aún del susto, ni
osaban expresarse con la libertad de otros días. Llegada la noche, la
ciudad ofrecía un aspecto terrorífico: con sus tinieblas y su silencio
parecería una ciudad muerta si los ruidos de tropa no dieran señales de
vida, semejantes a una palpitación febril.

Mientras llegaba la ocasión de acudir a la cita que se le había dado en
Vergara, Don Fernando no perdía ripio para buscar el rastro al padre de
Zoilo, suponiéndole en Estella, y a cuantos guipuzcoanos o vizcaínos vio
en el parador interrogaba, añadiendo que traía un encargo para dicho
sujeto. Por fin, después de mil indagaciones inútiles, dio con un
vizcainote inválido, buen bebedor y atrozmente sedentario, por obligarle
a ello su obesidad y su pierna izquierda, que era de acebuche. Resultó
que el tal había visto el día anterior al D. Sabino Arratia, con quien
tuvo algún conocimiento en Bermeo y Elorrio, y hablaron un rato breve,
lo bastante para enterarse de que venía en seguimiento de uno de sus
hijos, prisionero. «Mas ahora caigo---añadió el cojo,---en que no será
fácil que le encuentres. Era, según me dijo, amigo y compadre de
Guergué, de quien esperaba la salvación del mozo, y muerto el General de
este modo \emph{trígico}, el pobre señor se habrá metido siete estados
bajo tierra, o habrá echado a correr huyendo de la chamusquina. Yo me le
encontré saliendo de la parroquia de San Miguel, a punto de que él
entraba. ¿Sabes?, es la iglesia que está en un alto, en el centro del
pueblo. Nos conocimos; el hombre se echó a llorar, porque es muy
\emph{lagrimero}. Me dijo que si el hijo, que si Guergué, que si tal, y
nos despedimos: él entró a rezar\ldots{} Es aquella la iglesia que más
le gusta, por ser la más recogida\ldots{} Allí se pasa todo el tiempo
que le dejan libre sus diligencias. Como no le cojas en San Miguel, en
Estella no le busques.»

Tempranito se fue Calpena a la mencionada iglesia, y el toque de misa
que oía, cuando a ella se aproximó, alegraba su corazón. Entró,
admirando la severa puerta románica y el interior sombrío, que
impresionaban por su riqueza arqueológica y por su ambiente sepulcral,
con olor de tierra húmeda y de ataúdes podridos. Sólo dos ancianas oían
misa: no había más varones que el cura y monaguillo\ldots{} Salió D.
Fernando, y por aprovechar la mañana dirigiose al Santuario del Puy, al
que por larga cuesta se asciende desde el hospital próximo a San Miguel.
También en el Puy tocaban a misa; vio que algunas viejas y un mendigo
entraban delante de él. Cobró esperanzas, deseó con viveza encontrar lo
que buscaba, imitando el querer ardiente de Zoilo, y por aquella vez no
fue ineficaz la efusión grande de su espíritu, porque a poco de entrar
en la iglesia, y cuando sus ojos se habituaron a la obscuridad que en
ella reinaba, distinguió un bulto, un hombre de rodillas, al cual sin
mayor examen tuvo por el propio D. Sabino Arratia. No se movía el pobre
señor, que más bien parecía fúnebre estatua, y a ratos se llevaba el
pañuelo a los ojos como para limpiarlos de la humedad luctuosa que de
ellos afluía. Oyó la misa con suma devoción; oyéronla Calpena y los
demás en corto número asistentes al acto, y cuando este terminó y hubo
visitado tres altares el señor desconocido, se le acercó D. Fernando, y
a boca de jarro le dijo: «¿Es usted D. Sabino Arratia?»

---Yo no\ldots{} no, señor---replicó muy asustado el tal.---¿Qué quiere
usted?\ldots{} ¿qué se le ocurre?

---No se me ocurre más sino que es usted D. Sabino Arratia---añadió
Calpena, que en el parecido con \emph{Luchu} le reconocía,---y hace
usted mal en negármelo, porque soy su amigo y no le causaré daño alguno.

---Pues sí\ldots{} yo soy\ldots{} Ya ve usted\ldots{} Con estas
cosas\ldots{} ¡Ay de mí!---dijo el bilbaíno sollozando y acudiendo a sus
ojos con el pañuelo.---¿Puedo saber quién eres?\ldots{} ¿quién es
usted?\ldots{} porque aquí estamos todos con el alma en un hilo\ldots{}
y aun dudamos si somos vivos o muertos.

---Estamos vivos. ¿Y Zoilo\ldots?

---Vivo también.

---¿Dónde?

---Aquí, en el Santo Hospital\ldots{} ¿Es usted su amigo?\ldots{}
¿Conoces a \emph{Luchu}?\ldots{} Salgamos si le parece.

---Salgamos, sí señor.

---Somos amigos. Ya comprendo la terrible situación de usted. Vino aquí
fiado en la amistad de Guergué, que era su compadre, padrino de Zoilo, y
allí donde creía encontrar usted un protector\ldots{} encuentra un
cadáver\ldots{}

---¡Pero has visto qué crueldad, qué salvajismo! ¡Ay!, no comentemos.
¿Puedo saber quién es usted?

---Un amigo de Zoilo, que le sacará del hospital, de la prisión, o de
dondequiera que se halle.

---¡Oh, señor\ldots!---exclamó D. Sabino, que con sus ojos llorantes se
quería comer el rostro del caballero.---Prisionero y enfermo está, ¡qué
dolor de hijo! Todo por su temeridad\ldots{} ¡Qué cabeza, señor!

---¿Le ha visto usted?

---¡Si no me ha dado tiempo ese condenado Maroto fusilándome!\ldots{} a
mí no\ldots{} a Guergué, el mejor de los hombres, el amigo más
cariñoso\ldots{} Pero dime tú, diga usted, ¿es este el mundo criado por
Dios, o es otro que nos han traído del infierno? Yo digo que están
condenados cuantos sostienen esta guerra, reyes y reinas, archipámpanos
y ministriles\ldots{} ¡Qué dolor! Y todo por un papelito, la Pragmática
Sanción\ldots{} ¿Estamos todos locos, o somos tontos de remate? En ello
pensaba yo mientras oía la santa misa\ldots{} ¿Acaso sabes tú, sabe
usted, en qué vendrá a parar esto? Aquí tienes a un hombre que se
aguantó todo el sitio de Bilbao a pie firme, padeciendo aquellas
terribles hambres, hijo, y el continuo caer de bombas. Pues terminado el
sitio, y cuando en el pueblo entró la felicidad, para mí y para mi
familia empezaron las mayores desdichas que es posible imaginar. No
puedo recordarlo sin que se me llenen los ojos de lágrimas.

---Volvamos a lo presente. ¿Desde cuándo no ve usted a Zoilo?

---Desde que sin mi permiso, y contra la voluntad de toda la familia, se
lanzó a \emph{quijotear}, en Octubre del 37, siendo en sus aventuras tan
desgraciado, que al intentar la primera se ganó cinco mesecitos de
cárcel\ldots{} Después se me mete con los cristinos. Siempre fue el
chico muy guerrero, con grandísima disposición para las armas, y una
valentía y una terquedad que más parecen divinas que humanas\ldots{}
Pues, como digo, me le cogen los cristinos, y ya está loco el
hombre\ldots{} Tan pronto acudo a consolar a la familia, como a
perseguir y a rescatar a mi caballero, y en este trajín se me van meses
y meses\ldots{} Parezco yo también un Tío Quijote, buscando lo que no
hallo, y recibiendo en todas partes sofiones y descalabraduras\ldots{}
Si a usted le parece, sentémonos en esta piedra, que estoy desfallecido.
Pues verás, verá usted\ldots{} Hasta Julio del año pasado no supimos que
estuvo mi hijo en la acción de Peñacerrada. Yo me hallaba entonces en
Vitoria aguardando una ocasión de abocarme con el pobre Guergué\ldots{}
También le digo que si mi Zoilo es más guerrero que el propio Marte, a
mí no me ha llamado Dios por ese camino, y nada me turba y descompone
tanto como los espectáculos de lucha y muertes. Tiemblo al oír tiros, y
si me aproximo a un campo de batalla, éntrame sudor de agonía\ldots{} Ni
con cien salvoconductos me atrevía yo a penetrar entre las \emph{hordas}
de Zurbano\ldots{} Me acercaba, y retrocedía\ldots{} Mejor me acomodaba
entre carlistas, porque siempre me tiró de ese lado mi fervor
religioso\ldots{} la verdad, te digo la verdad\ldots{} Si mi Zoilo se
hubiera metido a guerrear por la Fe, fácil me habría sido cogerle y
retirarle de la milicia; pero entre cristinos no me hallo\ldots{} no
respiro\ldots{} El aire que anda entre ellos me huele a libertad de
cultos, libertad de la imprenta y pueblo soberano\ldots{} No, no\ldots{}
Mil veces pensé abandonar al chico, dándole por perdido para siempre;
mil veces me llamó el amor que le tengo, y volví a rondarle, siempre
medroso, siempre desconfiado\ldots{} Dios me decía: «ve por él y sácale
de la sentina.» y yo iba a la sentina y me acercaba, y tenía
miedo\ldots{} y\ldots{} Por fin, desesperado, me aboqué con el General
Van-Halen, el cual me agregó a un convoy que llevaba socorros a Zurbano.
Vi a este en Dicastillo; me echó muchos \emph{ajos}, me trató con
desprecio, ensalzando a mi hijo, y llamándome obscurantista y
retro\ldots{} no sé qué. Pero, en fin, diome las noticias que deseaba, y
a Estella me vine. Por llegar, mira tú qué suerte, me entero de que
Zoilo está en el hospital\ldots{} «Esta es la mía,» dije para mí; y me
fui en busca de Antonio Guergué\ldots{} De chicos jugábamos en los
Cantones de Bilbao\ldots{} Encontrele muy inquieto\ldots{} ¡Toma, como
que estaba urdiendo el golpe para hundir a Maroto! Con mal cariz me
dijo: «mañana.» ¡Mañana! Aquel mañana de Guergué fue ayer, hijo, y
¡pum!, fusilado\ldots{} y yo muerto de ansiedad, de miedo\ldots{} lo
diré todo, muerto también de hambre\ldots{} ¡ay dolor!\ldots{} Si eres
caritativo, como parece, y no temes andar por la ciudad, llévame a donde
yo tome algún alimento, pues desde ayer por la mañana no ha entrado en
mi cuerpo cosa caliente ni fría.

Compadecido del infortunio, así como de la flojedad de ánimo del pobre
señor, D. Fernando le agarró el brazo para llevársele a su posada. Por
el camino, a pesar del tranquilo continente del que ya se había
constituido en su protector, no se recobraba de su horrible susto el
buen Arratia, receloso de cuanto veía, temiendo engaños y traiciones.
«Bien comprendo---decía,---que eres, que es usted marotista, y no me
pesa. Si me apuran, no creo lo que ayer se decía de tratos nefandos para
que D. Carlos nos dé la libertad de conciencia. Y pues Maroto ha venido
a ser el amo, tráiganos una paz decente, con la religión sobre todo, y
debajo de la religión el rey o reina que nos quieran poner\ldots{} ¿A
dónde me llevas? ¿A tu casa? Si eres militar, ¿por qué vistes de
carbonero, y si eres carbonero, dónde demonios has conocido a Zoilo, y
por qué te interesas por él?\ldots{} Párate un poco, que me canso
horriblemente\ldots{} Ya estamos en la plaza\ldots{} Por aquí llevaron
al pobre Guergué como se lleva un cerdo a la matanza, ¡ay!, y al General
García vestido de sacerdote\ldots{} Al verles, creía que de terror me
moría\ldots{} Otra cosa: ¿cómo te llamas?\ldots{} ¿Cuál es la gracia de
usted?\ldots{} Perdona: con el hambre que tengo, hasta se me olvida la
buena educación\ldots{} Sigamos otro poco. ¿Falta mucho todavía? Ya no
puedo tenerme\ldots{} Pues sí, hijo mío: venga pronto la paz, sea como
quiera, con tal que no toquen a la religión sacratísima, ni al clero, ni
a sus bienes raíces, ni nos metan en casa la libertad de pensar\ldots{}
¡Ay, qué ganas de llorar! Deja que me seque los ojos\ldots{} Pues tan
extenuado me encuentro, que ahora daría yo todos los dogmas por unas
sopas de ajo bien calientes, con chorizo\ldots{} ¿Falta mucho?

Pronto llegaron, y lo primero que hizo D. Fernando fue ponerle delante
cuanta comida encontró, y bebida sin tasa. Gozaba viéndole comer, y el
hombre se mostró muy agradecido, y con mayor luz en la mollera para dar
a sus pensamientos claridad y fácil expresión\ldots{} «¡Oh, qué bueno es
Dios---exclamaba mirando al techo, por no haber allí cielo que
mirar,---y qué excelente cordero es este!\ldots{} Cuando más
desconsolados vivimos, se nos aparecen las buenas almas. Es usted un
ángel, Aquilino, un ángel sin alas. Repito que no me asusta Maroto, y
que bendeciré la paz que nos traiga, si no vienen con ella libertades de
pensar\ldots{} El dogma sobre todo\ldots{} Vino de ley es este,
¿verdad?»

Satisfecha el hambre, se caía de sueño, como quien pasara la noche
anterior al raso, sin atreverse a entrar en su vivienda, que era la
misma donde el pobre General García se había disfrazado de cura. Llevole
Calpena a un camastro, donde le dejó bien arropadito, sin cuidarse más
de él, porque otras graves obligaciones le llamaban. Echaide y el mozo
se miraron, añadiendo pocas palabras a lo que con los ojos se decían.
Había llegado la hora. Fuéronse los dos a la residencia de Maroto sin
rodeos ni precauciones, que en tal ocasión no se necesitaban; quedose a
la puerta Echaide, y entró Quilino con una caja de puros, abierta,
dentro de la cual había puesto un papel que en gordos caracteres decía:
Inquisivi.

\hypertarget{xxviii}{%
\chapter{XXVIII}\label{xxviii}}

Recibió el General a D. Fernando familiarmente en una gran pieza donde
tenía su lecho y una mesa de escribir. Habíase levantado poco antes, y
aún estaba la cama revuelta. Junto a una de las ventanas veíanse, sobre
derrengada mesilla, la navaja y trapos de barba, llenos de jabón, señal
de que Su Excelencia acababa de afeitarse. En la cómoda cercana estaba
el servicio de chocolate, el cangilón rebañado, migas de bollos y la
servilleta sucia. Vestía D. Rafael levita vieja militar con el cuello
desabrochado, dejando ver la camisa de dormir, pantalón azul y unas
enormes pantuflas de abrigo que cuadruplicaban las dimensiones de sus
pies. A poco de entrar Calpena, y despedido el asistente, se echó un
capote por los hombros, y sentose a la mesa de despacho, donde tenía
papeles a medio escribir, picadura esparcida y cigarrillos recién
hechos. Sentados frente a frente, el emisario de Espartero expuso las
condiciones de este, que oyó el carlista con atención y sonrisa
marrullera, y al terminar se produjo un silencio que a Calpena le
pareció larguísimo: el General, recogiendo aquí y allí la picadura, y
aprovechándola minuciosamente, tardó en formular la respuesta, que había
de ser solemne por tratarse en ella de los destinos de la infeliz
España.

«Ya no estamos en la situación de hace dos meses---dijo al fin, mirando
al mensajero en las pausas.---Entonces no tenía yo fuerza\ldots{} me
refiero a la fuerza moral\ldots{} y ahora la tengo. Ya se habrá usted
enterado de la justiciada que hice ayer. No había más remedio. Me
importa poco que D. Carlos refunfuñe. Al fin me dará la razón, cuando yo
consiga, y lo conseguiré, librarle del cautiverio en que le tienen
cuatro clerigones y cuatro buscavidas. No descansaré hasta no hacer la
limpia total\ldots{} Pero vamos al caso: decía que ahora tengo fuerza, y
procuraré mejorar todo lo posible, si hacemos la paz, la situación
ulterior de ese Rey que tan ingrato es para mí. Puesto que todo puedo
decirlo, y lo que a usted diga es como si lo hablara con el propio
Baldomero, sepa que la Reina y su hijo D. Sebastián ven las cosas de un
modo más razonable que D. Carlos\ldots; naturalmente, poseen luces,
criterio, que Dios no ha concedido a S. M\ldots{} y hoy por hoy se
contentarían con el reconocimiento de los derechos de D. Carlos,
abdicando este en su hijo y en Isabel juntamente\ldots{} ¿Conoce usted
la historia de Inglaterra?»

---Un poco. El caso es como el de Guillermo y María.

---Justo: sólo que lo que allí hizo el Parlamento, aquí lo haría D.
Carlos en nombre de Dios. Pues bien: sepa Espartero que en este punto no
cedo ni un ápice, ¡porra!, pues así lo he concertado con la de
Beira\ldots{} Claro que el pobre D. Carlos es ajeno a todo; pero ¡qué ha
de hacer el buen señor más que conformarse!

---Mi General, desde luego aseguro a usted que esa combinación no ha de
aceptarla mi poderdante. De ella resultará una familia real gravosísima,
con toda esa plaga de reyes padres y reyes madres\ldots{} Y luego, ¿en
qué condiciones ejercerían el Poder Real Isabel y Carlitos?

---Como los Reyes Católicos, mancomunadamente, firmando juntos, pues si
en aquel matrimonio se casó Aragón con Castilla, en este se casan y
conciertan dos ramas igualmente legítimas, para bien de la Nación y para
establecer una paz duradera. Creo yo que esto es muy patriótico.

---Será muy patriótico; pero imposible en la práctica. Delo usted por
rechazado.

---Muy pronto lo asegura---dijo Maroto dándole un cigarrillo que acababa
de liar.---Si Espartero me acepta esto, admito yo sin más discusión lo
referente al reconocimiento de grados tal como él lo propone\ldots{} y
hemos concluido\ldots{} Fíjese usted en que tengo fuerza, y ahora no
hemos de estar arma al brazo. Mis soldados anhelan batirse; yo también.
Aquí faltaba unidad; yo acabo de hacerla, ¡porra!; y sin necesidad de
que venga en mi ayuda ese loco de Cabrera, que para nada me hace falta,
intentaré bajarle el tupé al amigo Espartero. Él vale mucho; hace tiempo
le conozco\ldots{} Pero nuestras discordias le han ensoberbecido; los
laureles de Peñacerrada los debió a la ineptitud de Guergué y a lo
desordenado que estaba aquel ejército. Batallones hubo allí enteramente
a mi devoción; otros padecían la rabia apostólica. Yo he curado esa
rabia, ¡porra!, y mi ejército es mío; todo él respira con mi
aliento\ldots{} De modo que\ldots{} En fin, dígame usted algo.

---¿Sobre qué, mi General?

---Sobre estos propósitos míos de aplacarle un poco los humos a su amigo
de usted, ¡porra!

---Pues mientras no se llegue a la paz, ninguna contingencia de la
guerra podría causarme asombro, ni sobre ellas tengo por qué anticipar
opiniones. Buen militar es usted, y del arrojo de sus soldados nada he
de decir, pues reconocido está por todo el mundo. Podrá suceder que
alcance usted una victoria con que se olvide el desastre de Peñacerrada;
podrá suceder lo contrario\ldots{} ¿Quién lo sabe? Si se me permite una
opinión radical, diré que ya han demostrado unos y otros su valor; que
España no desea mayores pruebas de pericia militar y de personal
bravura. Hemos llegado a ese punto del duelo en que se impone la
cesación de los golpes y el abrazo de los combatientes. Los jueces del
terrible lance han visto maravillados la entereza heroica de los dos
caballeros; estiman como de igual importancia las terribles heridas que
uno y otro se han hecho; el juicio de Dios está cumplido, y la sentencia
no puede ser otra que la conservación de las vidas de entrambos. No hay
más remedio que envainar los aceros. La paz se impone. ¿Qué quiere
usted?, ¿convertir a España en sepulcro de dos inmensos cadáveres? Pues
España no quiere eso: anhela vivir, y el obstinarse en que muera, en que
muramos todos, paréceme una terquedad salvaje\ldots{} Formule usted de
un modo más práctico el artículo referente a la familia real y a la
situación de cada príncipe después del convenio, y la paz, tal creo yo,
tardará lo que tardemos en concertar la entrevista final de Maroto y
Espartero. Se ha de mirar antes por los fueros de España y de la
humanidad que por los intereses de tanto y tanto príncipe, que con sus
pretendidos derechos están desangrando a la raza, y nos la dejarán
anémica.

---Pues si en los derechos de príncipes, ¡porra!, hay que quitar
\emph{jierro}, ¡porra!, empiecen ustedes por dar carpetazo a los de
Isabel.

---Eso no puede ser.

---¡Ah!\ldots{} ¿Con que no puede ser? Pues lo mismo digo yo de los de
D. Carlos\ldots{} Ya lo ve usted: volvemos al principio, y nos
encontramos en Septiembre del 33, ante el cadáver de Fernando VII, que,
entre paréntesis, era una mala persona.

---No divaguemos, mi General.

---No divaguemos. Conste que no puedo ceder en la combinación propuesta
por mí. Reinarán Isabel y Carlos, o Carlos e Isabel, \emph{tanto monta},
con iguales derechos, con iguales prerrogativas\ldots{}

---Anticipo a usted que Espartero rechazará la combinación.

Pues antes que ceder en ello, cedería yo en lo del reconocimiento de
grados, aunque se que daría un disgusto a muchos personajes de acá, que
esperan las paces para saber la paga que han de cobrar\ldots{}

---No divaguemos. Me voy descorazonado, temeroso de que el de Luchana me
acuse de no haber sabido expresar su pensamiento. En nombre suyo rechazo
la organización estrambótica y complicada del Poder Real, que sería
lanzarnos a la mayor confusión y desconcierto. Piénselo usted, mi
General, y aguardaré hasta mañana.

---Lo he pensado bien---dijo el Caudillo dando un puñetazo en la
mesa.---No puedo yo, Rafael Maroto, tirar a los pies del caballo de
Espartero los derechos de D. Carlos.

---Pues ya verá usted\ldots{} ya verá, permítame que se lo diga, el pago
que le dará D. Carlos por esa transacción a la inglesa, a la
protestante. Todo lo que no sea reinar él solo, con poder absoluto,
brutal, le parecerá el triunfo de la revolución y de la herejía\ldots{}

---¡Ah, lo sé!\ldots{} pero yo cumplo con mi conciencia, ¡porra!, y hay
otras personas en la familia de S. M. que no se han puesto en esa
actitud intransigente por no estar dominadas por un cleriguicio loco, ni
por la cáfila de parásitos\ldots{} En fin, no puedo ceder en esto. Si él
no cede tampoco, sea lo que Dios quiera\ldots{}

---¿De modo que es cosa cerrada? ¿Puedo retirarme?

---Cerrada es\ldots{} pero no se vaya usted tan pronto. Quiero
obsequiarle con una copita\ldots{}

Levantose Maroto; de una próxima alacena sacó botella y copas, y al
dejarlas en la mesa, requiriendo después su capote, que se le caía,
dijo: «Ya sé que no pierde usted ripio, y que aprovecha estas embajadas
para distraerse con alguna conquistilla\ldots{} Cosa muy natural\ldots{}
Crea usted que no se mueve la hoja en el árbol en todo este país sin que
yo lo sepa.»

---Ya, ya veo que hay más polizontes que criminales, señal cierta de un
estado moribundo. Pero si todo lo que su policía le cuenta es tan
verdadero como mis conquistas, está usted muy mal servido, mi General.

---¿De veras? Por eso les digo yo: \emph{et surtout, point de zéle},
¡porra!\ldots{} Va usted a probar un vinito que me ha regalado nuestra
excelsa Soberana.

---¿Cuál? Porque, según la cuenta de usted, el arreglo de Reinas nos ha
de resultar muy parecido a las monteras de Sancho: una Reina para cada
dedo.

---Ya veremos eso\ldots{} Convinimos en no discutir más ese
punto\ldots{} Este vino me lo regaló la princesa de Beira, hoy Reina de
Castilla.

---Pues si usted no me riñe, bebo a la salud de Isabel II.

---Yo también, que una cosa es la galantería y otra la convicción
política.

En el momento en que el General bebía, le vio Calpena tan claro, como si
todo su interior gráficamente en signos externos se mostrara. El mirar
vivo del carlista y su rostro inteligente se iluminaron, si así puede
decirse, con la bebida, y se le transparentó el alma. Recordó D.
Fernando la frase que oyó a Espartero en Viana: «es muy ladino, muy
ladino,» y como tal se le manifestaba en la entrevista de Estella.
Estrenando los puros de la caja traída por Echaide, y divagando los dos,
entre humo, sobre asuntos familiares y sin importancia, formuló Calpena
de este modo la situación psicológica de D. Rafael Maroto en aquel
instante de la historia. «Ya te veo, ya te veo claro. Hace dos días te
habrías entregado a Espartero sin condiciones. No tenías fuerza; ahora,
por virtud del golpe de mano de ayer, la tienes y grande; te has
crecido, te sientes capaz de imponerte a D. Carlos y de manejarle como a
un títere. Naturalmente, ahora no te conformas con aceptar las
condiciones de paz que el otro quiere poner, sino que aspiras a que él
acepte las tuyas. El orgullo de tu éxito reciente te trastorna la
cabeza; sueñas con obtener una victoria, que te pondría en condiciones
excelentes para dictar luego los artículos del convenio de paz. Todo eso
que propones referente a las ramas dinásticas y al modo de organizar el
Poder Real, no es más que un expediente dilatorio. Conoces, como yo, lo
disparatado de semejante idea; pero tu cálculo revela tu agudeza:
mientras voy con tu mensaje y vuelvo con la negativa, te preparas,
eliges una posición ventajosa, das una batalla, la ganas, destrozas el
ejército de la Reina, y ya eres el hombre culminante, único, que tiene
en su mano la clave de los destinos de la Nación. Eso piensas, ese es el
ensueño forjado por tu travesura, por tu marrullería, que no le va en
zaga a la de tu rival\ldots»

De esta meditación le sacó bruscamente D. Rafael, diciéndole con
picardía: «Caviloso estáis\ldots{} No se devane los sesos por
adivinarme, ¡porra!\ldots{} Cuando vea usted a Espartero le dice que,
aunque enemigos políticos, le quiero bien, y deseo darle un abrazo.
Bueno. Hablemos de otra cosa. Ándese usted con cuidado con las mujeres
navarras, que todo lo que tienen de bonitas lo tienen de fanáticas. Rara
es la que no está afiliada en la policía, mejor dicho, en la masonería
apostólica. Le venden a uno con toda la gracia del mundo.»

---Descuide usted, mi General\ldots{} ya he previsto ese peligro\ldots{}
Y si le parece, me retiraré ya.

---Hijo, sí: yo tengo que hacer. ¿Lleva usted bien aprendida la lección?

---Tan bien aprendida que no se me olvidará ni una coma\ldots{} Y por
último, mi General, tengo que abusar de su bondad pidiéndole un favor en
asunto completamente extraño a estas embajadas.

---Venga pronto.

---Es cosa sencillísima.

---Aunque fuese oro molido. Venga\ldots{} ¿De qué se trata? Ya\ldots{}
de poner en libertad a un prisionero. Y yo, si usted no se enfada, le
pregunto: «¿quién es ella?»

---Aquí no hay \emph{ella}\ldots{} En fin, cuento con su benevolencia
para una obra de caridad.

---Bien, hombre, bien; me gustan a mí los caballeros caritativos. Pero
le advierto que yo lo he sido demasiado, y por ello no estoy donde me
corresponde, ¡porra! Pero, en fin, venga.

Expuso D. Fernando su pretensión, a la que accedió gustoso el General,
extendiendo de su puño y letra una orden \emph{a raja tabla}, de esas
que, en nuestro sistema de Gobierno, enteramente personal, tienen más
fuerza que la ley. Diole el caballero las gracias; despidiéronse con
vivos afectos, expresando los dos la esperanza de llegar en la próxima
entrevista a una concordia lisonjera, y Calpena salió, si pesaroso por
no haber obtenido ventaja en el asunto de interés político, contentísimo
de su feliz éxito en el privado.

En la calle le esperaba Echaide, que le preguntó: «¿Tienes que
volver\ldots? \emph{¿Acabatis\ldots?} ¿Nos vamos?»

---Todavía no: tengo que hacer algo aquí.

---¿Cosa de\ldots?, vamos, por el aquel de la paz.

---Sí, hombre, por el aquel de las paces, de las benditas paces.

\hypertarget{xxix}{%
\chapter{XXIX}\label{xxix}}

Profundamente dormido halló a D. Sabino en el parador, tumbado boca
arriba, rígido, cruzadas las manos, el rostro ceñudo y cadavérico. Creyó
por un instante que había pasado a mejor vida el infeliz; pero un
suspiro y una voz gutural le convencieron de que vivía y soñaba. Un rato
aguardó, por no turbar su descanso; pero al fin, obligado por la
urgencia del asunto, determinose a despertarle, dándole fuertes
sacudidas y voces. «No, no, Antonio Guergué---murmuraba con torpe voz el
bilbaíno.---No te conozco ni te he visto en mi vida\ldots{} Me estás
comprometiendo\ldots{} Yo no me meto en nada.» Fijando los ojos en D.
Fernando, le observó con asombro primero, con alegría después, viniendo
por esta gradación a la realidad. Y estirando brazos y piernas en largo
desperezo, dijo claramente: «¡Oh, tú!\ldots{} señor\ldots{} bien\ldots{}
Muchas gracias\ldots{} Yo bueno\ldots{} ¿y en casa?»

Díjole el caballero que era un hecho la liberación de su hijo, y que se
levantara y fuera al hospital para sacarle; mas tan torpe de
entendederas se hallaba el desdichado señor, que no se hizo cargo de la
feliz nueva, o por demasiado feliz no le daba crédito. «No habrá paz, no
volveremos a ver paz\ldots---decía.---Moriremos todos\ldots{} El amigo
nos engaña, y el enemigo se disfraza de amigo para vendernos. Tú,
\emph{marotista}, ¿qué nos traes? La libertad de cultos, y el que cada
uno piense lo que quiera, haciendo mangas y capirotes del dogma
sacratísimo. Esto no lo podemos admitir los creyentes. Mi amigo, llame
usted a otra puerta\ldots{} Con libertad de la conciencia no queremos
paz\ldots{} ¿Qué paz ni qué porquería? Es una paz pringada\ldots{} No,
no. Lo primero es el dogma, después los fueros, y luego, arréglense los
reyes y príncipes como gusten para ver quién calienta el Trono\ldots{}
¿Cuál es mi Soberano? Dios\ldots{} Dios mi \emph{Pretendiente} y mi
\emph{absoluto}\ldots{} Esto digo.»

Y volviéndose del otro lado, cogió nueva postura para seguir durmiendo:
su quebranto de huesos era enorme, su sueño atrasado de muchos días. No
viendo la posibilidad de hacer comprender al desdichado bilbaíno lo
perentorio del caso ni la solución tan fácilmente conseguida, decidió
abandonarle a su descanso y proceder por sí mismo. Antes de dar paso
alguno hubo de consultar con Echaide, el cual le aconsejó que no diese
la cara en asuntos de presos liberados, ni presentase por sí mismo la
orden del General. Convinieron en que Urrea desempeñaría muy bien la
diligencia, y así se dispuso, personándose el guipuzcoano en el
hospital, donde ninguna dificultad encontró; y al caer de la tarde,
entre dos luces, viéronle entrar en el parador, trayendo a Zoilo del
brazo, tan extenuado que daba dolor verle, lívido el rostro, la cabeza
liada en un sucio pañuelo; flojo de piernas, trémulo de palabra; el pelo
caído en algunas partes de su cráneo como si le arrancaran o se
arrancara mechones; un brazo inválido, con magulladuras lastimosas; y en
tan mísero estado de ropa, que las enjutas carnes se le veían por
distintas claraboyas de la chaqueta y del pantalón.

Metiéronle en un cuarto alto que les proporcionó el posadero, y allí le
rodearon Echaide y D. Fernando, a quien al punto y sin vacilar
reconoció, diciéndole: «No se me despinta, no, el caballero, aunque se
ponga en esa facha\ldots{} Y no he de meterme en averiguar por qué viste
como viste, que eso es cosa suya y no mía\ldots»

---¿Tienes hambre, Zoilo?

---Estoy como cuando salí de la cárcel de Miranda, desganado de rabia, y
enfermo de mala suerte. Ya me creí difunto, y cuando me sacó este buen
hombre creí que me llevaban a enterrar.

---Dinos una cosa. ¿Cómo te dejaste coger prisionero? ¿No te valió en
aquel caso tu querer fuerte?

---Es la primera vez que me ha fallado\ldots{} Pero algún día había de
ser\ldots{} Tanto va el cántaro\ldots{}

---Eso te decía yo, y no querías creerme. No hay que fiar tanto de la
suerte y del arrojo\ldots{} Aprenderás ahora, y vivirás dentro de la
razón\ldots{} ¿No me preguntas por tu familia?

Fijó Zoilo una mirada estúpida en D. Fernando, y tan sólo dijo: «¡Mi
familia!\ldots{} ¡Qué lejos se han quedado! ¿Cuántos años hace que no sé
de ellos ni ellos de mí?\ldots{} ¿Se han muerto?»

---Hombre, no: todos viven y están buenos. Sosiégate, descansa, y no te
descuides en tomar alimento. ¿Qué quieres?

---Agua\ldots{} No, no: vino.

---Aquí lo tienes. Entona ese cuerpo.

---Y mi padre, ¿vive también?

---Como tú y como yo.

---¿Mi mujer\ldots?

Al decirlo se le llenaron de lágrimas los ojos, y se dio un fuerte
puñetazo en la rodilla, cual si quisiera rompérsela.

---Tu mujer\ldots{} tan famosa\ldots{} esperándote\ldots{} Recuerda los
meses que han pasado desde que no te ha visto.

---Ya no se acordará de mí\ldots{}

---¿Tú qué sabes? Dime otra cosa: ¿se te ha pasado la borrachera de la
gloria militar?

---Sí, señor\ldots{} Estuve loco\ldots{} De tanto querer cosas grandes,
parece que se me ha gastado el alma, y en estos días, ¿sabe usted lo que
quería?: morirme.

---¿Y esperabas ver a tu mujer en el cielo?

---En el cielo, sí; ¿pues dónde había de verla si yo me moría\ldots?
Digo la verdad, señor: no me cabe en la cabeza que mi mujer esté en la
tierra.

---Pues en la tierra está. Procura reponerte, y la verás pronto, y de
ella no te separarás en lo que te reste de vida.

Rompió de nuevo en llanto, y Calpena, para curarle la aflicción, que
parecía un achaque hereditario, le administró comida, un par de huevos,
un pedazo de carne. No recibió con repugnancia la medicina el bruto de
\emph{Luchu}, y a la media hora de este tratamiento ya era otro. La
locuacidad se despertó en él, y cuando su amigo le hablaba de Aura, el
contento daba rosados tintes a su rostro demacrado, luz a sus ojos.
Queriendo activar la reparación psicológica, ya que la física iba por
buen camino, llevole D. Fernando a otros asuntos muy apartados del
familiar y doméstico que tan hondamente le convenía. Pedido informe de
las operaciones de Zurbano en el tiempo que no se habían visto, refirió
Zoilo, no sin trabajo, en cláusulas entrecortadas, la campaña laboriosa
en los montes de Bedaya, la arriesgada correría por Treviño y valle de
Cuartango, la defensa gloriosa de Subijana, la acción indecisa,
sangrienta cual ninguna, de Avechuco, en la que tuvo la desgracia de
caer prisionero; agregó sus desdichas en el largo \emph{via crucis}
hasta Estella, donde le tuvieron trabajando más de un mes en las
fortificaciones de Santo Domingo, con hambre y palos, hasta que,
acometido de unas terribles calenturas, se vio luengos días entre la
vida y la muerte. Concluido su relato, comió con más gana, y le mandaron
acostarse. En los aposentos de abajo continuaba D. Sabino en su
reparador sueño, empalmando una noche con otra.

En tanto, preparaban los arrieros su salida, señalada para el día
siguiente; al amanecer subió D. Fernando al cuarto de Zoilo, y
hallándole despierto, bastante aliviado de su postración, y con los
espíritus en buena conformidad, no quiso dilatar el darle conocimiento
de lo que creía más interesante. «Hola, \emph{Zoiluchu}, parece que
vamos bien. Con un par de días en tu casa, al lado de tu mujer, te
pondrás como un roble. En tu familia, te lo aseguro, encontrarás una
novedad, una estupenda novedad.»

---¿Mala o buena? No me encoja el corazón más de lo que lo tengo.

---Hombre, no: si quiero ensanchártelo. Necesitas ahora querer más de lo
que querías, amar más de lo que amabas.

---¿Más? Imposible. Si mi mujer está buena y no me recibe con despego,
soy feliz.

---Está totalmente buena, curada para siempre con una medicina que le ha
dado Dios. ¿No caes en ello, bárbaro? ¿A qué pones esa cara
estúpida?\ldots{} ¿No se te ha ocurrido que en los diez y seis meses que
has faltado de tu casa, ya por tus borracheras de gloria, ya por el
castigo que Dios ha dado a tu orgullo; no se te ha ocurrido, pedazo de
alcornoque, que en tan largo tiempo podían ocurrir novedades en tu
familia?

---Sí, señor\ldots{} pensaba yo\ldots{} lo vengo pensando desde que
estábamos frente a Peñacerrada.

---¿Qué?

---Que mi mujer\ldots{}

---Sí, hombre; tienes un hijo\ldots{} Has vivido diez y seis meses
soñando, y en tanto tu mujer, buena parroquiana de la naturaleza y de la
realidad, ha sabido cumplir sus deberes de esposa. En Durango la tienes
hecha una madraza\ldots{}

---¡D. Fernando!---exclamó Zoilo cerrando los puños.---No gaste conmigo
esas bromas. ¡Mire que\ldots!

---¡Broma que tú seas padre! ¿Pues para qué te has casado, animal?

---Para eso.

---Justamente, para eso.

---Pues allí tienes, en Durango, a tu cara mitad loca con su hijo, digo,
loca no, cuerda, enteramente cuerda y bien curada de sus arrechuchos, y
esperándote, esperándote, hombre, para que seas feliz con ella y con el
crío\ldots{}

---¡D. Fernando, mire que\ldots!

---La edad del chiquillo no la sé seguramente; sólo me consta que es
rollizo, guapote, y como tú, querencioso de vivir. ¿Qué? ¿No lo crees?
Pues en Estella está tu padre, que no me dejará mentir. ¿Tampoco crees
que está aquí tu padre? ¿Y si te le presento antes de diez minutos?
Aguárdame.

Salió D. Fernando, dejándole en tal confusión, que no sabía el hombre si
tirarse al suelo, o coger el techo con las manos. No tardó en volver el
caballero con D. Sabino, al cual agarraba por un brazo para tirar de él,
ayudándole a vencer los empinados peldaños. Al entrar en el cuarto, el
viejo Arratia decía: «¿Cómo cinco meses? Siete meses y seis días, si
usted no manda otra cosa, pues nació mi nieto el 13 de Julio, día de San
Anacleto, papa, y de San Salutario, mártir.»

El encuentro de hijo y padre fue tan solemne y patético como si cada
cual viese al otro resucitado. Se abrazaron, y D. Sabino inundó a Zoilo
con el raudal de su llanto salido de madre. Al hijo le faltó poco para
perder el conocimiento, de la fuerza de la emoción, y viendo confirmada
la noticia de su paternidad y de la mental reparación de Aurora,
entregose a una alegría delirante y como fantástica: primero se colgó de
una viga del techo, al cual alcanzaba puesto de pie en la cama; hizo
allí varias suertes acrobáticas de singular mérito, y después se lanzó a
gran distancia, andando un trecho con las manos, las patas en el aire.

«Nada tengo que hacer aquí---dijo D. Fernando,---y me voy. Pueden
descansar hijo y padre en este mesón el tiempo que les convenga.»

---¡Descansar!---exclamó D. Sabino aleteando con los brazos, como si le
contagiase el frenesí gimnástico de su hijo.---Nos iremos a escape, si
el \emph{marotismo}, que es ahora el amo, nos proporciona un
salvoconducto.

Recibiendo de manos de Calpena el pasaporte en toda regla, hijo y padre
se abrazaron de nuevo. D. Sabino, que creía en los milagros pasados,
pero no en los presentes, amplió su fe milagrera, declarando prodigiosas
y sobrehumanas las felicidades que llovían sobre él. Mayor fue su
asombro, que hubo de traducirse en religioso entusiasmo, cuando el
posadero le notificó que podía disponer de un mulo y un borrico, sin
ningún estipendio, con la sola obligación de entregarlos en Durango en
el punto que se les designaba. Dinero para el viaje también les fue
suministrado, lo que les vino de perillas, pues Zoilo no tenía blanca, y
la bolsa de D. Sabino había venido a una flaqueza casi equivalente al
vacío. Prorrumpió el vizcaíno en exclamaciones bíblicas con solemne
acento, que fue de gran edificación en la posada. «Señor, no hay lengua
que entone tus alabanzas\ldots{} Tu mano desciende a nuestro muladar, y
henos aquí vestidos de luz\ldots{} En tu misericordia con estos tristes,
veo la señal de que envías la paz al mundo. Glorifiquemos a Jehová
paternal, a Jehová pacífico\ldots{} ¡Hosanna!\ldots{} ¡Bendita sea tu
paz, Señor, que ha de venir sin libertad de cultos ni libertad de la
imprenta!\ldots{} ¡hosanna!»

En la exaltación de su júbilo, llegó a creer Sabino que el misterioso
arriero bienhechor no era persona de este mundo, sino un ángel tiznado,
un ordinario celestial que traía encargos del cielo para repartir entre
los mortales, preparando el reinado de la paz. Aparte hizo D. Fernando a
Zoilo advertencias muy oportunas, dictadas por un prudente recelo.
«Chico, no hagas la tontería de decir a tu padre quién soy.»

---Comprendido\ldots{} No debe saberlo\ldots{} ¿De modo que el Sr.~D.
Fernando se ha muerto?

---O se ha casado, que es lo mismo.

---Bien, hombre, bien\ldots{} Déme usted otro abrazo\ldots{} ¡Qué gusto!
¿Y cuántos hijos tiene ya?

---¡Hombre, todavía\ldots!

---Es verdad\ldots{} Todavía es pronto. Pero tendrá muchos\ldots{} como
yo.

---Sí\ldots{} muchísimos. Procura tú largar uno cada año\ldots{} Vaya,
adiós. Yo tengo prisa.

Y al partir, dejándoles en disposición de hacer lo propio, sintió la
tristeza que acompaña al acto de enterrar un muerto querido. Sobre una
parte principalísima de su existencia ponía la losa con epitafio harto
breve: \emph{Aquí yace}\ldots{} Las letras borrosas, ilegibles, que
decían y no decían un nombre, parecían sepultar más lo sepultado, y
ponerlo más hondo, y hacerlo más muerto.

\hypertarget{xxx}{%
\chapter{XXX}\label{xxx}}

Sin tropiezo ni accidente alguno llegaron los cuatro asendereados
hombres a Logroño, y la primera diligencia de Echaide fue dar aviso al
General para saber si era su gusto recibir al embajador en la Fombera o
en otra parte. La contestación fue que el caballero podía despintarse
ya, soltar el disfraz, presentándose en el palacio de la plazuela de San
Agustín lo más pronto posible. Toda una tarde y parte de la mañana
siguiente empleó D. Fernando en la tarea de volver de aquel estado
rústico al de persona fina, pues tan dura era la costra de su figurada
barbarie, que para romperla y rasparla fueron menester muchas aguas y
restregones muy fuertes. Por fin, restaurado el hombre, entró muy
satisfecho en la casa de sus nobles amigos. Después de una corta espera
en el billar, tuvo el gozo de ofrecer sus respetos a Doña Jacinta, que
le encontró muy negro, quemado del sol y de los aires fríos; pero con
aspecto de salud y robustez. Diole las cartas de su madre que allí le
aguardaban, y comprometiéndole para la comida de aquel día, se retiró
para que leyera. Así lo hizo, primero repasando los plieguecillos con
avidez, luego despacio y enterándose de todo. El caballero se sentía
dichoso, y no se contentaba con echar a volar el pensamiento hacia
Medina de Pomar: quería irse todo entero y descansar de tantas fatigas
junto a la persona que más amaba en el mundo.

Hasta la hora de comer no vio a Espartero, que aquel día tuvo tarea
larga en su despacho. Le saludó muy afectuoso, presentándole después al
jefe político interino de Logroño, D. Joaquín Berrueta, a quien debía el
General su conocimiento con el arriero Echaide. Probablemente aquel
señor estaría en el secreto; pero no hablaron sílaba de tal asunto. Los
convidados, a más de Berrueta y de Fernando, eran Pepe Concha y D.
Leopoldo O'Donnell. Nunca estuvo D. Baldomero tan impaciente porque la
comida acabase pronto: saltaba en su asiento; miraba con inquietud el
traer y llevar de platos. Por fin, escaldándose vivo con el café, que
tomó muy caliente, se levantó y dijo: «¡Qué calor hace aquí! Venga
usted, D. Fernando.» En el próximo billar, donde se cruzaron con el
criado que traía el braserillo para encender los cigarros, dieron lumbre
a los suyos, y por una escalerilla de piedra que en dicha pieza existía
bajaron al jardín, como de treinta varas en cuadro, poblado de
corpulentos árboles con una fuente en el centro. Paseándose en la parte
más asoleada, dio cuenta Calpena de su segunda entrevista con Maroto, y
ello fue motivo para que el de Luchana montara en cólera y dijese: «Toda
esa componenda de reyes y príncipes es una farsa. Lo mismo le importan a
él las ventajas que puede obtener la familia de D. Carlos que la
carabina de Ambrosio\ldots{} Lo que quiere es confundirme, acabarme la
paciencia\ldots{} Pero ya, vera quién es Baldomero Espartero.»

Pedida venia por D. Fernando para exponer el juicio que había formado de
la situación psicológica del caudillo faccioso en el momento de la
entrevista, trazó la figura moral e intelectual completa, tal y como él
la había visto. La cara de Espartero revelaba su conformidad con el
retrato, en que veía una obra maestra de observación penetrante. «Es
usted---le dijo cariñoso,---un gran conocedor del corazón humano, y
podía dedicarse a escribir Historia. Me trae usted un Maroto vivo con el
pensamiento pintado en la cara. Es cierto, sí\ldots{} este es el hombre.
Se ha ensoberbecido con el golpe de Estella; pretende ahora tener un
chiripón a mi costa, y si lo consiguiera podría dictar a su gusto la
paz, esa paz con fueros de un lado, y de otro la caterva de Príncipes
consortes y de Reinas viudas\ldots{} Dejémosle en esa ilusión, para que
el trastazo que le voy a dar le coja en el Limbo\ldots{} ¡Pobre
Maroto!\ldots{} En fin, vámonos arriba. Esta noche venga usted a cenar,
y seguiremos charlando.

De lo que hablaron en la cena, pudo colegir D. Fernando que el ejército
del Norte se ponía en marcha. Dadas las órdenes aquella noche, oyose de
madrugada el trompeteo de la caballería. Los jefes que mandaban tropas
acantonadas en los pueblos a lo largo del Ebro, entre Logroño y Miranda,
salieron también. Hablando con Espartero, Calpena se aventuró a decirle:
«Mi General, por la dirección de las tropas, el traslado será en el ala
izquierda y líneas de Balmaseda, plan felicísimo para mí si me permite
acompañarle.»

---No le permito, sino que le mando venir conmigo. Falta la mejor parte
de la misión, caballero D. Fernando, la más delicada y difícil. En
premio de sus buenos servicios, le llevo a ver a su madre. No crea usted
que la sorprenderá\ldots{} Ya lo sabe\ldots{} ya le espera. Tienen las
mujeres una policía y un espionaje que vale un mundo. Si quiere usted
adelantarse, váyase con Ribero, que llegará antes que yo.

Gozoso replicó el caballero que, a pesar de su vivísimo afán de llegar
pronto, prefería seguir al Cuartel General. Despidiose de Doña Jacinta y
de Vicentita con vivos afectos, así como de todas las personas con
quienes había hecho amistad en la casa. Sentía un inmenso regocijo, y se
creyó compensado de tantos afanes y sufrimientos con las alegrías de
aquella marcha en dirección de sus amores. Medina de Pomar, Villarcayo,
se le presentaban luminosos, como estrellas refulgentes marcando la meta
de su destino, y hacia la derecha del sendero distinguían también un
resplandor lejano sobre las lomas de la Rioja alavesa. Alguna luz
brillaba constante, inextinguible, del lado de La Guardia.

No habían llegado aún a Fuenmayor, cuando topó con su amigo Ibero, que
de la brigada de Zurbano había pasado a la división de Alcalá, con
adelanto considerable en su carrera, pues era ya primer comandante con
grado de teniente coronel, y mandaba el segundo batallón de Luchana.

En cuanto se vieron, concertaron el ir juntos en las marchas. Ibero se
manifestó a D. Fernando muy orgulloso de sus éxitos recientes, y al
compás de los adelantos de jerarquía iba creciendo su entusiasmo por la
Libertad y el Progreso, ideales hermosos, que exigían el sacrificio de
cuanto existe en el hombre, menos el honor. Tan penetrado se hallaba el
valiente Ibero de estas ideas, que no vaciló en confiar a su amigo la
repugnancia de que terminara la guerra por tratos y componendas con los
facciosos, reconociéndoles grados, e igualándoles con los que habían
derramado su sangre por Isabel. Esto era inconveniente, indecoroso,
inmoral; hacer concesiones al \emph{retroceso} era reconocerle como un
Estado. Transigir con él era una declaración de impotencia. No, no mil
veces: los soldados de la Libertad debían perecer antes que terminar la
campaña por otro medio que el hierro y el fuego. Si se quería establecer
una paz durable, era forzoso descuajar el carlismo, y abrasar toda
semilla, para que ningún tiempo ni ocasión pudiera germinar de nuevo.
Con los elementos que a la sazón poseía la Libertad, debía emprenderse
la extinción completa, radical, de aquel bando execrable que pretendía
implantar el despotismo asiático, la superstición y la barbarie. «Que en
todo el siglo y en los siglos que sigan no se oiga hablar más de
Pretendientes, ni de clérigos salteadores, ni de fanatismo, ni de estas
antiguallas odiosas. Como así no se acabe, como sólo nos contentemos con
cortar al monstruo una de sus cabezas, y luego le demos de comer por las
bocas que le queden, no conseguiremos nada, y la Libertad morirá con
vilipendio, amigo mío. Esto pienso, esto aseguro, y mientras viva
pensaré lo propio, a fe de Santiago Ibero.»

No dejaron de producir efecto en el ánimo y en la inteligencia de D.
Fernando las razones de su amigo. Pero se apresuró a rebatirlas con
suavidad, haciéndole ver que el carlismo era una fuerza social, difícil
de destruir. La fatalidad había traído a esta pobre Nación a un dualismo
que sería manantial inagotable de desdichas por larguísimo tiempo. La
idea absolutista, la intransigencia religiosa hallábanse tan hondamente
incrustadas en los cerebros y en los corazones de una gran parte de los
hijos de España, que era ceguedad creer que podrían ser extirpadas
\emph{de un tirón}. Dios había sido poco benigno con España, poniéndola
en manos del mayor monstruo de la historia, Fernando VII, que sobre ser
déspota sin talento, no supo establecer con firme base la sucesión a la
Corona. La herencia de este hombre funesto había de ser insufrible carga
para la Nación; su testamento ponía los pelos de punta. Dejaba a su país
un semillero de guerras, discordancias irreductibles entre los
españoles, un Estado siempre débil, una Monarquía fundada en la
conveniencia antes que en el amor de los pueblos, una religión
formulista, una paz armada, métodos de gobierno con carácter
provisional, como si nunca se supieran las necesidades que habían de
traer el día de mañana. ¿Era conveniente la transacción, aun siendo mala
cosa? Sí, porque con ella, si España no mejoraba, al menos viviría, y
los pueblos rehúsan la muerte aún más que las personas. Si no fueron
estas las razones que a las de su amigo opuso Calpena, debieron de ser
muy parecidas. Una y otra vez, en el curso de la marcha, hablaron del
mismo asunto, abominando el uno de los arreglos, y defendiéndolos el
otro como el médico que aplica los calmantes en un incurable mal.

A los cuatro días de la salida de Logroño, llegaban a las tierras altas
de Burgos, y Calpena, con permiso del General, se dirigió a Medina,
donde tuvo la inefable dicha de abrazar a su madre y a los Maltaras, que
en aquella villa y en el palacio de la Condesa habían buscado refugio.
Todo habría sido venturas para el caballero sin la pena de ver a la niña
mayor atacada de la pícara dolencia pulmonar constitutiva en los hijos
de Valvanera, y a uno de los pequeños enflaquecido y transparentado como
si la tierra le reclamase. Para colmo de infortunio, el insigne D.
Beltrán, perdido de la vista, había caído en gran tristeza y
abatimiento, que agriaba su carácter y le despojaba de las amenidades
que embellecían su trato. No se conformaba el buen aristócrata con aquel
bajón impuesto por su naturaleza ya gastada y caduca; protestaba, quería
suplir las fuerzas corporales con energías de concepto y alardes de
temeridad, y D. Fernando agotaba su ingenio para producir en él una
dulce componenda entre la esperanza y la resignación. En cambio,
encontró a D. Pedro bastante fuerte, sin nuevas amenazas de la dolencia
que le postró en Vitoria, muy bien adaptado a la cómoda existencia de
capellán palatino. La Condesa gozaba, según dijo, de una salud perfecta,
como nunca la disfrutó, y se animaba grandemente viendo su casa tan bien
poblada de amigos cariñosos. Todo lo regía y gobernaba con actividad
casera, cuidando de que sus numerosos huéspedes estuviesen contentos y
los enfermos atendidos como en su propia casa. Con ella se franqueó el
hijo en secretas conversaciones, refiriéndole sus embajadas, y
comentando los dos el probable giro de aquel negocio, según lo que
resultara de la campaña emprendida. El último esfuerzo de Marte traería
la paz, dando este nombre a un armisticio de algunos años o lustros. Los
que vivieran mucho verían extrañas cosas. Y como ante todo ansiaba ver
D. Fernando la grande empresa de Espartero y su gente ante las líneas de
Ramales, una vez consagrados tres días a las más puras satisfacciones de
su espíritu, abandonó las ociosas alegrías junto a su madre, para
meterse en el fiero trajín de la guerra.

\hypertarget{xxxi}{%
\chapter{XXXI}\label{xxxi}}

Cerca de Agüera encontró D. Fernando al coronel inglés Wilde, a quien
había conocido en Logroño. Comisionado por el Gobierno de su país para
estudiar la guerra, habíala seguido en todos sus accidentes desde
Peñacerrada, compartiendo las fatigas y aun los peligros de nuestros
soldados. Era persona muy simpática, instruida, de finísimo trato, y
habiéndose propuesto con tenacidad sajona dominar la lengua de Castilla,
andaba ya muy cerca de conseguirlo sin perder su nativo acento. Con él
iba un capitán de la misma nación, que no había podido vencer aún, por
el corto tiempo que llevaba en España, las dificultades elementales de
nuestro idioma, y lo destrozaba graciosamente sin miedo al disparate,
ávido de aprender, como se aprenden todas las cosas: errando. Ingleses y
españoles celebraban la ocasión que les unía, y se concertaron para
presenciar juntos las peripecias de la \emph{campaña de Occidente}, como
decía Wilde. Formando un cuerpecillo militar de siete hombres (con el
criado de Calpena y los ordenanzas que el General había puesto al
servicio de los extranjeros), se colaron en el teatro de la guerra, y su
primer paso fue aproximarse a D. Leopoldo O'Donnell, que había sucedido
a Van-Halen en el cargo de Jefe de Estado Mayor. Causaba espanto ver las
posiciones ocupadas por los carlistas en los montes que rodean a Ramales
y Guardamino; imposible parecía que de tales alturas pudiera ser
desalojado un enemigo intrépido, que con tiempo supo plantarse allí, al
amparo de rocas ingentes. Allí el arte militar semejaba al instinto
guerrero de las bestias feroces. Hablando los ingleses con O' Donnell,
que por la pinta y la seriedad flemática parecía más inglés que ellos,
dijéronle: «¿Pero están ustedes seguros de poder ganar esos picachos, si
en ellos los lobos tendrán que mirar dónde ponen la pata?»

---No estamos seguros de llegar arriba, coronel---replicó D. Leopoldo
con la sonrisa que ponía en sus labios, así para los dichos triviales
como para los que precedían a los grandes hechos;---pero subiremos hasta
donde humanamente se pueda. Mis soldados no miden los caminos con la
vista, sino con los pies, y no se hacen cargo de los peligros sino
después de estar en ellos.

---Los que hemos visto la subida de Banderas---indicó D.
Fernando,---estamos curados de asombro.

---Lloverán piedras seguramente---quiso decir el capitán inglés
mezclando de un modo pintoresco las hablas española y británica.---La
ventaja del enemigo es que no necesita gastar pólvora ni proyectiles.

---Eso lo veremos---dijo D. Leopoldo.---Señores, con Dios. No puedo
entretenerme.

---General, a sus órdenes. ¡Gloria a Dios en las alturas!

---Y paz en la tierra, \emph{etcétera}\ldots{} ¿La paz dónde está?

---Donde menos se piensa\ldots{} aquí.

Siguieron faldeando el cerro, y a cada paso encontraban fuerzas
acantonadas. Se había dispuesto que la división del General Castañeda
con las tropas de O'Donnell disputara a los carlistas las alturas del
Moro y el Mazo, empresa que parecía fabulosa. Toda la tarde de aquel día
la empleó la partidilla hispano-inglesa en enterarse de las posiciones
del ejército constitucional: Ribero, con la Guardia, hallábase en la
loma de Ubal, en observación de Maroto, que ocupaba el valle de
Carranza. A Espartero no pudieron verle; pero se aproximaron a sus
avanzadas en el camino de Ramales a la Nestosa. Pasaron la noche en la
falda de Ubal, entre oficiales del 3.º de la Guardia, y al amanecer del
día siguiente, 27 de Abril, salieron en la dirección que se les indicó
como más conveniente para encontrar a O'Donnell; pero no lograron su
propósito, pues el que Wilde llamaba \emph{el gran irlandés} habíase
remontado en la vertiente de la peña del Moro hasta una altura en que
era muy difícil alcanzarle ya. El tiroteo que desde las ocho empezó por
diferentes puntos obligoles a buscar algún abrigo: procuraron guarecerse
de las balas, ya que no podían hacerlo de la lluvia de piedras. En una y
otra eminencia, el Moro y el Mazo, el vigoroso ataque subiendo era un
prodigio de agilidad y serena bravura. La roca erizada de picos,
ofreciendo a cada paso accidentes difíciles de franquear, cortaduras,
grietas, cresterías inabordables, centuplicaba las fuerzas absolutistas
y disminuía las liberales. Pero lo inverosímil se hizo verdadero poco
después del mediodía. Castor Andéchaga y Simón de la Torre no supieron
sacar partido de sus admirables posiciones, y se las dejaron quitar,
cumpliendo con una resistencia formal de dos horas. ¿Qué fue?
¿Cansancio, escepticismo, deseos de acelerar el desenlace que preveían y
deseaban? Aun admitida esta causa del desfallecimiento de los facciosos,
siempre era grande el mérito de los soldados de Isabel, que treparon por
aquella escalera de piedras cortantes, con un precipicio en cada
peldaño.

Faltaba un hueso muy duro que roer, pues los demonios de la facción
habían fortificado una cueva que dominaba el camino entre la Nestosa y
Ramales. Una pieza de a cuatro, que disparaban con metralla, era el
monstruo de aquella caverna, apostado en su boca.

Allí no escapaban hombres ni ratas. Alentado D. Baldomero por la toma de
las alturas del Moro y el Mazo, decidió apoderarse de la cueva, y
embocando hacia ella ocho piezas de artillería, que fueron como otros
tantos perros que atacaron al monstruo, y soltándole además de lo más
granado de la tercera división, hizo polvo al guardián formidable. Día
bien aprovechado fue aquel: Espartero debió marcarlo con piedra blanca,
pues entre sol y sol, peleándose con las montañas más que con los
hombres, disputó y obtuvo los baluartes que convertían en gigantes a sus
poseedores. Con esto les hizo pigmeos, y él adquiría una talla que le
igualó a la que había sido enemiga y era ya su aliada, la Naturaleza.

No pudieron los ingleses, con su agregado español, presenciar el ataque
a la cueva, porque cuando llegaron al Cuartel General ya estaba todo
concluido; pero lo oyeron relatar a Echagüe, capitán de Guías del
General, y a un oficial de artillería, Osma, ambos partícipes de la
gloria de aquella jornada. Al anochecer acompañaron a los vencedores a
la cima de Ubal, donde Espartero mandó construir un reducto, cuyos
trabajos se emprendieron sin dilación, alardeando todos de incansable
actividad. Favorecíales una noche espléndida, que en aquellas alturas,
dominando valles y montes, era de una majestad y belleza incomparables.
En amenas pláticas la pasó D. Fernando con sus amigos Echagüe y Dulce,
pronosticando glorias y venturas, brillantes acciones de guerra,
precursoras de una dichosa paz. Al día siguiente bajó con los ingleses a
Bolaiz, visitaron la famosa cueva, hicieron alto en todos los puntos
donde encontraban oficiales conocidos, aquí Gándara, allí Linaje y
Urbina. En Los Valles ofrecieron sus respetos al General Jefe, a quien
hallaron contento, en estado de excelente salud, disponiéndose a
embestir y ganar los fuertes de Ramales y Guardamino, con lo cual les
\emph{aventaría} (era su expresión habitual), obligándoles a replegarse
a las guaridas de Vizcaya y Guipúzcoa.

A su amigo Ibero le encontró Calpena un tanto melancólico por no haber
entrado en fuego en los combates del 27. Era de los que cuando no
pelean, viendo pelear a sus compañeros, se juzgan ofendidos y hasta
cierto punto despojados de lo que les pertenece. Hablando de esto y de
las próximas luchas, las conversaciones venían a parar en cálculos
diversos sobre lo que haría Maroto con sus veinticuatro batallones
apostados en el valle de Carranza. ¿Aceptaría el reto de su grande
enemigo? En la previsión de que se presentase por Gibaja, reforzó
Espartero el extremo de su ala izquierda, tomando posiciones y
fortificándolas bajo el fuego de las guerrillas enemigas.

En los primeros días de Marzo rompieron fuego las baterías contra
Ramales, y avanzaron los batallones. No fue todo a pedir de boca, que
algunos cuerpos retrocedieron, aunque sin desorden, y lo que se ganaba
en una hora en otra se perdía. Pero a media tarde, los defensores del
fuerte, viéndose amenazados por diferentes puntos y desmontada la
artillería, se retiraron precipitadamente a Guardamino, situación más
áspera, más defendida de la Naturaleza, y allí se encastillaron con la
seguridad de que el hueso era de los que no podían roer los liberales
sin dejarse en ellos los dientes. Ya se vería esto.

En efecto: no era blando el hueso, y dos días estuvo Espartero bregando
con él sin obtener grandes ventajas. Pero el día 11, cargado ya el
hombre de perder soldados, y movido de su valor impaciente, que no
admitía largas dilaciones para satisfacer su anhelo, dispuso un ataque
simultáneo contra todos los puntos en que presentaba el enemigo mayor
resistencia, y con sus intrépidos Guías, el 2.º de Luchana y la escolta,
dio una de esas cargas que hacen memoria en los fastos militares. El
mismo peligro corría D. Baldomero que el último de sus soldados, pues el
avance fue a la desfilada, bajo el fuego mortífero de los fuertes y de
las trincheras abiertas por los carlistas en montes altísimos, que en
algunos pasos ofrecían una verticalidad aterradora. Electrizados por la
presencia y la actitud arrogante del Caudillo, los soldados avanzaban
husmeando la victoria, gozándola antes de obtenerla. Algunos caían, es
verdad; pero los más andaban bien derechos. En lo mejor de la marcha,
vio Espartero que una compañía bajaba en retirada; pero con unas cuantas
voces, que si en otra ocasión podían ser innobles, en aquella eran la
más gallarda de las imprecaciones poéticas, les obligó a \emph{volver
caras}. Adelante todo el mundo, sin miedo a la muerte; que allí no había
que pensar en cosas tristes, sino en la grande alegría de arrojar al
enemigo al otro lado de los montes, a la corriente del Cadagua\ldots{}
Adelante, pues, y vengan balas. Llegaron a un punto en que la
desigualdad del terreno no permitía funcionar a la caballería. Los
individuos de la escolta pidieron permiso para desmontarse y acometer a
pie los parapetos desde donde los facciosos les abrasaban a tiros. Fue
concedido el permiso, que Espartero no negaba nunca para los actos de
temeridad loca. Los jinetes sin caballos no pudieron tomar a la primera
embestida los parapetos; pero su ejemplo enardeció a los menos
decididos, su locura se comunicó a los más sensatos, y a la segunda
embestida los carlistas abandonaron la indomable almena natural en que
peleaban. En tanto, Linaje les daba un fuerte achuchón por la parte de
Cibaja, y viéndose amenazados por el flanco, se retiraron de todo el
monte, quedando Guardamino entregado a su propia fuerza. Mas era por
naturaleza tan robusto, que a la intimación de Espartero para que se
rindiese, contestó con un no redondo y procaz.

Era ya cuestión de tiempo y paciencia el someter a tan fiero gigante,
emplazando en las alturas toda la artillería de que Espartero podía
disponer, y haciendo polvo con cañoneo constante la armadura de roca que
el coloso vestía. Incansables, comenzaron por la noche la operación de
subir las piezas; pero al amanecer del 12, hallándose el general en una
ermita desmantelada donde pasó la noche, sin otro alimento que un pedazo
de pan y un chorizo que llevaba en sus pistoleras, por cama la dura
peña, por descanso la impaciencia ansiosa, recibió un parlamentario de
Maroto con las condiciones para rendir el fuerte. Proponía que la brava
guarnición de Guardamino, prisionera de guerra, fuese canjeada por igual
número de liberales que los carlistas tenían en sus depósitos. Invocaba
Maroto la humanidad, y por humanidad accedió D. Baldomero a lo que su
rival le pedía. Todo el día duró el ir y venir de parlamentarios desde
Carranza a la ermita, porque el Gobernador del fuerte no quiso rendirse
sin que su General se lo ordenase directamente; pero al fin ello se
arregló, y las comunicaciones mediadas entre ambos caudillos fueron
afectuosas por todo extremo. Entregose, pues, Guardamino con su
artillería, municiones, pertrechos y víveres. Los rendidos fueron
inmediatamente enviados al cuartel de Maroto, que no tardó en pagar la
carne facciosa con igual peso y medida de carne liberal. Alardearon uno
y otro de hidalguía y generosidad. La victoria de Espartero fue de las
más grandes que obtuvo en su gloriosa vida. En la elocuente orden del
día que dio a las tropas les dijo: «El enemigo no quiso aceptar vuestro
reto para una batalla general. Encastillado en sus formidables
posiciones, allí quería que se estrellase vuestro arrojo. Allí os
conduje. Allí vencimos. Allí completamos su ignominia.»

\hypertarget{xxxii}{%
\chapter{XXXII}\label{xxxii}}

La brillante hazaña de Espartero sobre Guardamino fue presenciada por
los caballeros de la trinca anglo-española. Marcharon en la retaguardia
de la escolta, de tal modo fascinados, que no advirtieron el peligro
hasta que no se hallaron en la imposibilidad de evitarlo. Tuvieron la
suerte de salir ilesos, con excepción de Urrea, que recibió un balazo en
el muslo, sin que le tocara el hueso. Perdió alguna sangre, continuó a
caballo, y al fin de la jornada le curó veterinariamente un práctico del
escuadrón. Hasta el día 13 no tuvo Calpena noticias de Ibero, que había
sabido hartarse del manjar de su gusto: peligro, temeridad, gloria.
Entre él con los de Luchana, y Echagüe con los Guías, habían tomado los
parapetos que decidieron la victoria\ldots{} El hombre no cabía en su
pellejo. No quería grados, no buscaba recompensas. Bastábale el gozo de
haber empujado a la Libertad hacia las altas cimas donde debía tener su
asiento, de haber arrojado hacia los valles cenagosos al \emph{monstruo
del obscurantismo}.

Maroto se internó en Vizcaya; Espartero, fijando en Ramales su Cuartel
General, dio descanso a sus tropas antes de emprender la ocupación del
país vasco-navarro, contando con el desaliento del enemigo y con la
descomposición y ruina de su antes poderosa unidad. Pasado el temporal
de agua que en lo restante de Abril y principios de Mayo entorpeció los
movimientos, avanzó el ejército cristino hacia Orduña, que fue ocupada
sin disparar un tiro. Con pretexto de tratar de un nuevo canje de
prisioneros, envió el de Luchana a su rival un parlamentario, al cual
acompañaban el coronel Wilde, encargado por su Gobierno de hacer cumplir
el convenio Elliot, y dos o tres personas más, afectas al servicio del
militar extranjero. Recibioles Maroto un tanto displicente. Expuso el
parlamentario, Brigadier Campillo, lo referente al canje; el inglés hizo
presente su propósito de trasladarse a Tolosa para someter al
\emph{elevado criterio} del Rey los deseos del Gabinete británico,
inspirados en sentimientos de humanidad y justicia; disuadioles Maroto
de esta idea, brindándose a dar cumplimiento por sí mismo al convenio
Elliot, pues poder y autoridad tenía para ello; y una vez retirados de
su presencia los mensajeros con sus respectivos secretarios, mandó
recadito al caballero español que en calidad de intérprete al coronel
Wilde acompañaba. Encerrándose con él a media noche en la destartalada
estancia del caserón donde tenía su alojamiento, solos, sin más luz que
la del candil que alumbraba un cuadro negro de las ánimas del
Purgatorio, hablaron lo que a renglón seguido con la posible fidelidad
se reproduce:

«He leído la carta de Espartero que usted me trajo---dijo Maroto,
paseándose, las manos en los bolsillos,---y empiezo por decirle que no
me parece bien el abandono del disfraz, ¡porra!\ldots{} aunque me sea
muy grato verle a usted en su porte de caballero distinguido y llamarle
por su verdadero nombre\ldots{} Pero no es prudente, no. Estamos, estoy
rodeado de espías infames\ldots{} Tome usted asiento.»

---No tema usted por mí, General---dijo Calpena, siguiendo a Maroto en
su paseo:---yo sabré guardarme\ldots{} y vamos al asunto.

---Pues al asunto. Veo que su jefe de usted está bien enterado como yo
de las intrigas de los apostólicos contra mí.

---Europa entera conoce la rabia vengativa y el furor venenoso de ese
bando que, aun después de vencido, se revuelve contra el hombre fuerte
que le apartó del Rey\ldots{}

---Todos los que D. Carlos desterró por exigencia mía\ldots{}
naturalmente, tuve que cuadrarme\ldots{} plantear la cuestión en el
terreno de la dignidad: O ellos o yo, ¡porra!\ldots{} pues todos
aquellos que eran la perdición y el descrédito de la Causa, en la
frontera trabajan contra mí, con mil enredos y calumnias\ldots{} Lo que
yo digo: no necesitan volver a ganar el corazón del Rey, porque lo
tienen bien ganado. Carlos V les ama y a mí me detesta. Eso lo sé, lo he
visto muy claro. S. M. cedió a mi exigencia, porque no tenía corazón
para resistirme. Yo apelaba a su dignidad, a su conveniencia, y a falta
de estas, encontré su miedo\ldots{} Pero el miedo aplaza, no resuelve.
Estamos lo mismo: el Rey no se apea ni se apeará del burro de su
intransigencia apostólica y absolutista\ldots{} ¿Y sabe usted que ese
danzante de Arias Teijeiro, en vez de largarse a Francia como el Rey le
ordenó, se fue al Maestrazgo? Allá le tiene usted reconciliado con Cala,
a quien acusó de venal, y partiendo un piñón con Cabrera. Entre todos
arman grandes tramoyas contra mí. Nada conseguirán mientras yo tenga
junto al Rey a mi gran aliado, el miedo; pero el día en que S. M. se
recobre del susto que le di, y apoyado se vea por los \emph{brutos}, que
así califican a la fidelidad, perderé mi mando, y creo que la vida con
él\ldots{}

---La situación de usted, mi General, es harto difícil. Las
circunstancias, los hechos, con su lógica incontrastable, imponen a
todos la paz\ldots{}

---La paz\ldots{} Venga pronto, si ha de ser honrosa, como yo puedo
admitirla y proponerla\ldots{} Sentémonos, señor mío\ldots{} Y ahora que
me acuerdo. Felicite usted en mi nombre a Espartero por el nuevo título
que le ha concedido su Reina: \emph{Duque de la Victoria}\ldots{} Es
hermoso, y hasta cierto punto me lo debe a mí. No debe olvidar que le
abandoné voluntariamente las posiciones de Ramales y Guardamino, por
evitar el derramamiento de sangre\ldots{}

---Me permitirá usted, mi General, que no exprese ninguna opinión sobre
los hechos militares del pasado mes\ldots{} Y no es porque no los
conozca; que observé al ejército en todos sus movimientos, y seguí al
Duque en su prodigiosa marcha sobre Guardamino.

---El fuerte hubiera resistido mucho tiempo. Se rindió porque yo se lo
ordené.

---Cierto; pero\ldots{}

---Pero\ldots{} No discutamos. Sólo digo que el título de Duque de la
Victoria en gran parte me lo debe a mí D. Baldomero, ¡porra!\ldots{}
Reconozco que es un militar valiente y un hombre honrado, que desea el
bien de su patria\ldots{} Yo también, ¡porra!, yo, sin llamarme Duque,
quiero la felicidad de España.

Nervioso y exaltado, Maroto se levantó a poco de sentarse, diciendo con
fuertes voces:

«Y me hará el favor de advertir a su jefe que no me mande parlamentario
militar, so color de canje de prisioneros. Esto me compromete, ¡porra!
No tardan mis enemigos en llevar el soplo a Tolosa\ldots{} Que si
andamos en arreglos, que si vendo al Rey\ldots{} No, no quiero
parlamentarios. Siempre que llega uno, tengo que dar a mi ejército una
orden del día echando sapos y culebras\ldots{} ¡porra!\ldots{} para
disimular el mal efecto\ldots{} Y vamos al asunto.»

---La ingratitud del Rey es tan manifiesta, lo mismo que su tenacidad en
sostener el retroceso y la barbarie, que no insistirá usted, así lo
creo, en las condiciones que me manifestó en Estella referentes a la
familia Real.

---No, no insisto en ello; renuncio a mi propósito del enlace de los
hijos; renuncio a conservar a D. Carlos las preeminencias de Rey
padre\ldots{} Que se vaya al extranjero, con título y calidad de Infante
aburrido y de Pretendiente chasqueado, a comerse la pensioncilla que se
le dará para que viva con decoro\ldots{} No merece otra cosa; no ha
nacido para más; aún saca más de lo que le corresponde por su menguada
inteligencia\ldots{}

---Espartero contaba con esta rectificación de las antiguas ideas de
usted, y una vez de acuerdo en cosa tan importante, espera que la
conformidad en los demás puntos no se hará esperar.

---Poco a poco---dijo el carlista, súbitamente acometido de una gran
agitación.---Si cedo en lo de las personas Reales, no puedo ceder en los
principios, pues no pretenderá Espartero que yo le entregue todo, la
fuerza y las ideas\ldots{} Eso no sería transigir: sería por mi parte
una debilidad vergonzosa\ldots{} ¿Qué quiere ese hombre? ¿Dejarme a mí
un papel ridículo, y conservar él la gloria de la pacificación? Dígame
usted: ¿qué papel hago yo, entregando mi ejército al masonismo y a la
impiedad revolucionaria? Eso no puede ser, y no será\ldots{} Antes
moriremos todos\ldots{} Asegure usted a su General que no suscribiré
nunca una paz que no vaya fundada en un régimen político mucho más
restringido que el existente.

---Pues el General Espartero---declaró Calpena con solemnidad---pone por
condición primera que se ha de conservar el régimen político existente,
la Constitución del 37, con todas sus consecuencias\ldots{} ¿Le parece a
usted justo que después de la sangre derramada por la libertad,
ofendamos la memoria de los hombres heroicos que por ella han perecido?
¿Qué quiere usted? ¿Que el representante de las ideas liberales acepte y
patrocine el absolutismo? Eso no será transacción. Será entregar nuestra
bandera al enemigo vencido para que la pisotee.

---Pues quédese cada cual con su bandera, y perezcamos todos---ritó D.
Rafael, no ya agitado, sino furibundo.---Sepa Espartero que trata con un
General que manda fuerza considerable, no con un monigote sin decoro ni
vergüenza. Corra la sangre; no haya humanidad ni compasión. Lo que no se
hace por un Rey inepto, lo haremos por la defensa de los grandes
principios.

---Veo, señor mío, que, obedeciendo a un destino fatal, será usted el
instrumento del obispo de León, de Arias Teijeiro y del clérigo
Echevarría. Usted les detesta, y al propio tiempo les ampara. Ellos
pregonan la cabeza de Maroto, ignorando que al matarle, matarían a su
mejor amigo.

---No, no defiendo yo el absolutismo---ritó Maroto fuera de sí, con
fuertes voces,---ni las ideas de esa canalla. Defiendo un régimen
templado, en que el Rey gobierne inspirándose en las necesidades
positivas de los pueblos; un régimen sin tiranía del Soberano ni
alborotos de los súbditos, con la unidad católica bien garantizada y los
clérigos levantiscos bien sujetos; un régimen en que puedan hacerse oír
los hombres ilustrados y callen los ignorantes y díscolos; un régimen de
justicia, de gobierno paternal, con el consejo de un escogido número de
personas graves que ilustren al Rey y enfrenen a la plebe\ldots{} Eso
quiero, eso propongo, y sin eso no habrá paz, no puede haberla,
porque\ldots{} denme todo lo que quieran, mi destitución, mi muerte;
pero no pidan a Rafael Maroto que firme una paz a gusto de los masones y
comuneros. Eso no puede ser\ldots{} Yo le suplico a usted que no me
contradiga, ¡porra!

---Bueno, mi General\ldots{} Realmente, yo no contradigo a usted: no
hago más que exponer las que creo ideas y propósitos de la persona en
cuyo nombre hablo. Siento infinito volver allá con la triste obligación
de comunicar el fracaso definitivo de las negociaciones.

---Pues comuníquelo usted\ldots{} No hay paz, no puede haberla---dijo
Maroto desplomándose en la silla, por una cesación súbita de aquel
frenesí nervioso.---¿Qué me importa? Si todo se hunde y se lo lleva el
diablo, no es por culpa mía. Es culpa del señor Duque nuevo, que quiere
arreglar todo a su gusto, para su sola gloria y provecho, dejándonos a
los demás como trapos\ldots{}

---No es eso: perdone usted\ldots{}

---Es eso\ldots{} y no me contradiga. Como trapos\ldots{} ¡Bonito papel
quiere asignarme!\ldots{} ¡Y él, ¡porra!, el héroe, el pacificador, el
niño bonito, el niño mimado!\ldots{} Pretende el mangoneo universal, y
ser el amo, y traernos a todos cogidos de la nariz\ldots{} ¡Ay!

Este ¡ay! fue una exclamación dolorosa, como punzada en el corazón, el
lamento de una naturaleza profundamente herida. «¡Ay!---repitió
oprimiéndose el costado.---Puede usted creerme: deseo una muerte
repentina que ponga fin a mis sufrimientos. No era esto lo que yo
presentía, lo que yo soñaba al venir al carlismo. No era esto, no, lo
que me impulsó al abandono de las posiciones de Ramales. Pensé yo que
Espartero me comprendería, que sería generoso\ldots{} Pero su egoísmo
está bien manifiesto: quiere una paz que sea para él un triunfo, y un
oprobio para mí\ldots{} Lo peor es que\ldots{} Siéntese usted: aún
tenemos algo que hablar.»

Con acento quejumbroso, de hombre enfermo, de un alma sumida en acerba
pena, prosiguió así: «Y a pesar de todo, créame usted, deseo la
paz\ldots{} sí, señor, la deseo como soldado y como español\ldots{}
porque yo amo a mi patria\ldots{} Bien sabe Dios que el absolutismo mío
no es el régimen absurdo y tenebroso que predican los clérigos de Oñate.
Espartero me conoce\ldots{} No quiera hacer de mí un monigote\ldots{} Si
en ello se empeña, no habrá paz, y España se acabará\ldots{} Más quiero
verla muerta que en brazos del masonismo y de la revolución.

---Espartero---dijo Calpena compadecido del General carlista, por el
lastimoso estado a que le habían traído sus errores---no pretende
humillar a usted, ni apropiarse la gloria de este bien tan grande: la
gloria será de los dos, para los dos la inmensa gratitud de España.

---Así debiera ser\ldots---murmuró el carlista con emoción, que afeminó
por un instante su voz varonil y guerrera.---Nadie me gana en el amor a
este terruño donde hemos nacido\ldots{} En mi larga vida militar y
política no he tenido otro móvil que el bien de los españoles\ldots{}
Pero los buenos deseos son una cosa, y los buenos caminos otra\ldots{}
Cuestión de suerte, amigo mío; cuestión de acertar o no en los primeros
pasos\ldots{} ¡Oh, pues si yo lograra que España dijese: «a Maroto debo
la paz!» Pero no me caerá esa breva, ¡porra! La fatalidad dice que
no\ldots{} que no\ldots{} la fatalidad me ha tomado entre ojos\ldots»

En la pausa que siguió a estas palabras, D. Fernando vio al General
agobiado en el sillón, los codos en las rodillas, el rostro en las
palmas de las manos, y respetó su dolor guardando silencio. Después sacó
D. Rafael del bolsillo del capote un pañuelo grandísimo, y se sonó con
estrépito. Tenía los ojos encendidos y húmedos.

«Mi General---le dijo Calpena, aprovechando con delicadeza la emoción
que observaba,---me detendré aquí todo el tiempo que sea menester, si de
la espera resulta que puedo llevar una proposición de concordia. Piense
usted en ello un día, dos; considere su situación, la ansiedad del país,
el deseo de todos los partidos\ldots»

---¡Pero si estoy ya loco de tanto pensarlo!\ldots{} No, no pienso más.
Ya es cuestión de decidirse, de escoger la primera carta que salga.

Suspirando, volvió a su inquieto pasear por la estancia. De pronto se
paró ante Calpena, diciéndole: «Puesto que no tiene usted prisa de
volver a Orduña, ayúdeme a buscar una solución decorosa para mí. Verá
usted lo que se me ocurre\ldots{} Tenga paciencia, y hablaremos algo
más.»

\hypertarget{xxxiii}{%
\chapter{XXXIII}\label{xxxiii}}

Dirigiose a la cómoda en que estaba el candilón, el cual, dicho sea por
respeto a la puntualidad histórica, había dejado extinguir una de sus
dos mechas, manteniendo encendida la otra por puro compromiso, al
parecer, pues bien se le conocían las ganas de dormirse en la
obscuridad. D. Fernando miró al General, que revolvía papeles en el
cajón primero de la cómoda, y tras él veía también mal alumbradas por la
luz dormilona las pobrecitas ánimas del Purgatorio, sus cuerpos desnudos
entre llamas rojizas. ¡Con qué gusto las habría sacado de aquel
martirio, extrayendo al propio tiempo al pobre General, que en las
llamas de su ansiedad e irresolución ardía!

«Verá usted---dijo D. Rafael, hallando lo que buscaba y volviendo el
rostro hacia el mensajero de su rival:---aquí tengo una carta
interesantísima. No haré con usted misterio de su contenido ni de la
persona que la firma: es un amigo íntimo de Simón de la Torre y mío. En
ella se me propone una entrevista con el Comodoro Lord John Hay, el cual
tiene instrucciones de su Gobierno para proponer a Espartero y a mí
fórmulas de paz.»

---Debo decir a usted que a mi jefe no le gusta que los extranjeros
medien en este asunto. Notaría usted que el coronel Wilde no pronunció
una palabra de condiciones de arreglo. También debo decirle, General,
que a Espartero no le supo bien que usted cambiara comunicaciones con el
mariscal Soult sobre este negocio. Es muy delicada la intervención
extranjera, así en la guerra como en la paz, porque casi siempre los
poderosos que nos prestan servicio tan eminente lo cobran después con
una pesada injerencia política y diplomática.

---Es verdad; pero yo no puedo negar al Comodoro la entrevista que me
propone. Sólo que no sé dónde ni cómo celebrarla. Bien podría servirme
de pretexto la orden que a León ha dado Espartero de quemar las mieses
de Navarra. Esto es una violación del tratado de Elliot.

---¿Ha contestado usted a La Torre que acepta la entrevista?

---No, porque de nadie me fío ya. No me determino a enviar una carta de
tanta gravedad por mano de carlista: la traición y el espionaje tienden
aquí sus redes que es un primor.

---¿Y no hay un hombre leal que establezca la comunicación verbalmente?

---No le hay, o al menos yo no le veo junto a mí---replicó Maroto con la
desconfianza pintada en su inquieto mirar.

---Permítame usted que le diga, mi General, que en el recelo y
suspicacia que me manifiesta veo una enfermedad del ánimo, efecto de su
singularísima situación entre la guerra apostólica y la paz nacional;
veo el delirio persecutorio, que usted logrará vencer mirando con más
serenidad cosas y personas.

---Puede que tenga usted razón\ldots{} Déjeme seguir: Simón de la Torre
y yo estamos de acuerdo; el amigo que nos comunica es un joven bilbaíno
muy simpático, que ha servido con Córdova y con Espartero\ldots{}

---¡Oh, qué luz, mi General!\ldots{} ¿Es acaso Pedro Pascual Uhagón?

---¿Amigo de usted, por ventura?

---Sí señor\ldots{} Yo sabía que andaba por aquí; me constaba su amistad
con Simón de la Torre\ldots{} En fin, ¿quiere usted que yo me vea con
Uhagón?\ldots{} ¿Dónde está?

---Muy cerca de aquí: en Amurrio.

---Pues allá me voy. ¿Debo decirle que está usted dispuesto a celebrar
la entrevista con el Comodoro?

---Justo; ¿pero dónde nos encontramos, Señor?\ldots{} ¿Debemos reunirnos
por casualidad, o por reclamo del inglés, para tratar de la cuestión de
las mieses incendiadas?

---Deje usted a mi cuidado el determinar la entrevista de una manera
lógica, en forma que le ponga a usted a cubierto de toda sospecha.

---Si así lo hiciere, me prestaría un servicio inmenso en las actuales
circunstancias\ldots{}

---¿Con que en Amurrio? Cuente usted con que mañana comemos juntos Pedro
Pascual y yo; cuente con que un día de estos se verá usted sorprendido
por Lord John, y obligado aparentemente a conferenciar con él\ldots{} Y
cuente con que las proposiciones del inglés diferirán poco de las de
Espartero\ldots{}

---Pero la sanción de una potencia extranjera, amigo mío, es alivio
grande de la responsabilidad\ldots{}

---Convenido. Luego veremos el grado de desinterés de la gestión
inglesa\ldots{} En fin, mi General, viva la paz, \emph{aunque viva con
su Pepita}\ldots{}

---Eso, eso---dijo Maroto, riendo por primera vez en la conferencia de
aquella lúgubre noche,\emph{---que viva con su Pepita.} Y ahora\ldots{}

---Sí: debo retirarme.

---Que no se le olvide felicitar a Espartero por su ducado.

---Lo agradecerá mucho.

---Sí, sí: los dichosos agradecen los plácemes de los tristes---dijo D.
Rafael sin ocultar su pena inmensa.---Con que buenas noches. No tengo
vino superior con que obsequiarle.

---Ya beberemos pronto a la salud de España pacificada. No me detengo.
Querrá usted dormir; yo también.

---Yo no duermo.

---Descansar, por lo menos.

---Tampoco.

---Ya vendrán para todos el descanso y la tranquilidad.

---Dios lo quiera.

---¡Ánimo, sinceridad, patriotismo! Adiós, mi General.

---Adiós. Le deseo lo que yo no he tenido nunca: buena suerte.

---La tendremos\ldots{} ¿Qué hace falta? El corazón siempre por delante.

---¡Ay!\ldots{} Eso se dice, eso se intenta\ldots{} pero no siempre el
corazón se pone donde quiere, donde debe\ldots{} Adiós.

Salió Calpena de la triste casona; palpando paredes se encaminó a su
alojamiento, y lo primero que hizo fue dar órdenes para partir de
madrugada. El coronel Wilde y el Brigadier Campillo dormían
profundamente; procuró hacer lo propio, y al romper el día trotaban los
seis desandando el camino que habían traído. Las diez serían cuando las
avanzadas del ejército liberal les indicaban la proximidad de Amurrio.
Dijo D. Fernando a sus compañeros que si no querían esperarle en aquel
pueblo, donde una diligencia importante le detendría, siguieran a
Orduña. Divididas las voluntades, el Brigadier determinó encaminarse sin
demora al Cuartel Real, y Wilde se quedó, pues no había para él compañía
más grata que la del caballero español. No vaciló este en ponerle en
autos del asunto que motivaba su detención en Amurrio: uno y otro, cada
cual en su esfera, trabajaban por la paz, y solían comunicarse una parte
de sus secretos. La primera diligencia fue tomar lenguas del paradero de
Uhagón, también del inglés amigo, y sin grandes molestias dieron con él
en la casa de Zárate, donde estaba en gran parola, \emph{inter pocula},
con Ibero y otros oficiales, entreteniendo los ocios con historias
picantes y libaciones de chacolí. En el mismo hospedaje se metieron
Calpena y Wilde, formando alegre compañía, y al poco tiempo de sociedad,
ya se habían trazado los conspiradores de la paz el plan más acertado
para llevar adelante las vistas entre el Comodoro y el General de D.
Carlos. Por desgracia, Lord John se hallaba por aquellos días en Bayona;
Pedro Pascual tenía que trasladarse a Bilbao, buscar embarcación que le
llevase a Francia, y volver luego con el Comodoro. Convinieron en que
Wilde le acompañaría en la expedición marítima, mientras a Orduña pasaba
D. Fernando para dar cuenta al General. Algunos días retuvo el Duque de
la Victoria a su amigo, no sólo porque descansase, sino por creer que en
el estado de las negociaciones convenía dar largas a Maroto, para que su
turbado ánimo, con la tremenda crisis del carlismo, viniese a mayor
decaimiento y desorden más grande. La primera comisión que D. Baldomero
dio a su fiel servidor después de aquel descanso fue llevar a Maroto las
cartas de los emigrados apostólicos, que interceptadas por el Gobierno
fueron impresas en la \emph{Gaceta de Madrid}. Por ellas se veía que el
partido intransigente, a quien el Rey con fingida corrección había
separado de su gracia, se mantenía con este en inteligencia clandestina.
Por miedo a Maroto, había decretado D. Carlos el destierro de los
clérigos Echevarría y Lárraga, de Marco del Pont y Arias Teijeiro; pero
no tardaron estos en ponerse de nuevo al habla con su señor, tendiéndose
desde la frontera a la Corte un hilo de conspiración que no fue el paso
menos interesante de aquella tragicomedia.

Volvió, pues, D. Fernando al Cuartel de Maroto, acompañado de Ibero en
calidad de parlamentario militar para un nuevo canje, y halló muy
desconcertado del entendimiento al General sin ventura, variando de
opiniones y actitudes a cada instante, pasando bruscamente del ardiente
furor al desmayo mujeril. Ya tenía conocimiento, cuando el mensajero le
mostró la \emph{Gaceta}, de los tratos que sostenían los emigrados con
el Rey absoluto, y a este propósito le hizo Calpena, con seguro
conocimiento de la humanidad, estas profundas observaciones: «Vea usted,
mi General, cómo se reproducen en la historia los mismos efectos cuando
las causas no varían, y cómo se repiten los hechos cuando las personas
no cambian. En D. Carlos tiene usted la imagen viva de su hermano
Fernando VII: son los mismos perros con el mismo Toisón de Oro al
cuello, y perdóneseme la comparación. Diferentes parecían uno y otro
hermano, y son el mismo sujeto repetido en el tiempo, desmintiendo a la
muerte. Si discrepan en cualidades secundarias, en lo principal son
idénticos, y proceden de igual manera. La situación en que el estadillo
carlista se encuentra es la misma del Estado español en aquellos famosos
años del 20 al 23. La pesadumbre y la barbarie del absolutismo han
traído una revolución, y esa revolución, esa protesta contra el régimen
tiránico y clerical, Maroto a pesar suyo la representa. Por una serie de
circunstancias, la fuerza ha venido a estar en manos de usted. El Rey no
supo serlo absolutista, no sabe serlo tampoco liberal, y doy este nombre
al partido \emph{marotista} o \emph{de transacción}, para establecer un
término relativo que facilite mi argumento. Liberal es usted, aunque no
quiera confesarlo; liberales son Simón de la Torre, Zaratiegui y aun el
mismo Elío, por extraño que parezca. Digamos que han admitido un átomo
de la idea liberal: en ese átomo está todo lo sustancial del principio.
Pues bien: D. Carlos ha venido a ser prisionero de usted; tiembla de
miedo viéndose sometido a la fuerza que odia; aparenta ceder; aun dice
\emph{marchemos y yo el primero}\ldots{} Por intimación de usted, separa
de su lado a su camarilla; destierra muy contra su voluntad a los que
cree sostenedores de su soberanía absoluta; pero continúa entendiéndose
con ellos, dándoles ánimos para que conspiren, adquieran fuerza y vengan
a libertarle. ¿Duda usted esto? ¿Cree la pintura recargada y violenta?
Su silencio y su mirada me dicen que no. Pero si aún duda, pronto ha de
ver cuán fundado es este juicio mío. ¿Recuerda usted la sublevación de
los voluntarios realistas? ¿Recuerda las partidas levantadas por
clérigos y frailes salteadores? Pues pronto hemos de verlas
reproducidas. El bando apostólico, apoderándose de los soldados que
usted manda, levantará la bandera del absolutismo neto y rabioso contra
la transacción que este ejército representa. Harán creer a los pueblos
que usted secuestra al Rey, que tiene embargado su real ánimo\ldots{} Y
por fin, y esto es lo más triste, esa bandería furibunda vencerá por
lógica ley al partido de la moderación, y Maroto será tratado no como un
hombre que mira por el bien de su patria, no como un General que sirve
intereses superiores a los de una persona, sino como un vulgar
ambicioso, y le impondrán pena infamante. Por muy extraño que parezca,
será usted, en su papel político y en su fin desastroso, muy semejante
al infortunado Riego. Le llevarán a la horca en un serón arrastrado por
un burro\ldots{} y\ldots{}

---Cállese usted\ldots---dijo Maroto apretando los puños y despidiendo
lumbre por los ojos,---que si algo hay de verdad en el paralelo que
hace, no puedo admitir mi semejanza con Riego.

---Ya lo veremos.

---Yo sabré morir con dignidad.

---No lo dudo. Pero es lástima que usted muera, pudiendo vivir con honor
y hasta con gloria, facilitando la obra de la paz.

Poco más hablaron; Maroto se volvió muy taciturno, sumergiéndose en sus
melancolías. Luchaba fieramente ¡infeliz hombre!, con el turbio,
revuelto oleaje de su destino, más embravecido cuanto más en él
pataleaba.

\hypertarget{xxxiv}{%
\chapter{XXXIV}\label{xxxiv}}

Fue un hecho, al fin, a fines de Julio, en Miravalles, la entrevista de
Maroto con Lord John Hay. No se halló presente Calpena; pero por su
amigo Uhagón supo después que no habían llegado a un acuerdo. Quizás
Maroto, harto ya de guerra, y deseando ponerle fin a todo trance para
salvar su honor militar y su vida, habría dado asentimiento a las
condiciones presentadas por el inglés, muy semejantes a las de
Espartero; mas no podía por sí solo cerrar trato sin el asenso de los
demás jefes, encariñados con la paz, pero más exigentes en punto a
condiciones. Necesitaba tomarse tiempo para traer las demás voluntades
al punto de cansancio y desesperación en que ya estaba la suya, y
propuso a Espartero, por conducto del Comodoro, la suspensión de
hostilidades. De la respuesta del Duque de la Victoria a esta martingala
de su rival sí fue testigo D. Fernando, el cual vio con gusto que el
criterio del Duque no difería del suyo. Nada de armisticio. Maroto,
juzgándose impotente ya para presentar batalla, no quería más que ganar
tiempo, esperando del acaso una solución menos terrible para él que la
que anunciaba la realidad. Volvió, pues, el inglés al Cuartel carlista,
en Arrancudiaga, y expresó a Maroto la negativa de Espartero, y su
propósito de reanudar sin demora las operaciones. He aquí la razón de la
marcha del ejército liberal desde Amurrio a Vitoria por el desfiladero
de Altuve. Ocasión tuvo el carlista, en aquel paso peligroso, de
contener a su rival y aun de batirlo; mas no quiso o no supo
aprovecharla. Sólo algunas guerrillas molestaron a Espartero en Altuve;
y cuando entraba en Vitoria, casi sin disparar un tiro, los facciosos
abandonaron el puente fortificado de Arroyabe, corriéndose hacia las
líneas atrincheradas de Arlabán y Villarreal.

Decidido siempre y con sus ideas bien claras, como turbias eran las del
otro, atacó Espartero resueltamente, no dándole tiempo a prepararse.
Maroto aceptó aquel combate, como el suicida que ve en la segura muerte
la única solución del conflicto que le agobia. La proclama que dio a su
ejército era el lenguaje de la impotencia y el orgullo, y estos
sentimientos se comunicaron a la tropa carlista, que en aquella jornada,
como en otras muchas, desplegó un valor heroico, una grandiosa entereza.
Porfiado cual ninguno fue el combate: de una parte y otra se desarrolló
toda la fuerza espiritual y física que siempre fue D. de los soldados
españoles en las grandes apreturas de la guerra. Perecieron aquí y allá
valientes en gran número. Venció al fin el que tenía razón: Espartero
fue dueño de Villarreal. De las alturas de Arlabán desaparecieron los
carlistas como una nube empujada por el viento, y escabulléndose por las
tristes hoces de Aránzazu, caían sobre Oñate y los valles guipuzcoanos,
cuna y sepulcro de la Causa.

Antes de la gloriosa ocupación de Villarreal por Espartero, supo este
que en el campo enemigo, por la banda de Navarra, ocurrían sucesos
graves, que, confirmando la rápida gangrena del cuerpo lacerado del
absolutismo, venían a favorecer los planes de pacificación. Algunas
compañías de los batallones 5.º y 12.º de Navarra se sublevaron en
Irurzun al grito de \emph{Viva el Rey}, \emph{mueran los traidores},
\emph{abajo Maroto}. Era la enfermedad histórica de la Nación, la
protesta armada, manifestándose en la Monarquía absoluta de Oñate como
en el régimen constitucional de Madrid. La ineptitud y doblez de los
hijos de Carlos IV, tan semejantes en su soberbia como en su incapacidad
para el gobierno, eran quizás la causa determinante de aquella dolencia
que con el tiempo había de corromper la sangre nacional. El Rey tenía
una cara para los transaccionistas y otra para los apostólicos.
Creyérase que Fernando y Carlos eran el mismo hombre. Pues bien: los
sublevados de Irurzun encamináronse a Vera, soliviantando a los pueblos
del tránsito; diéronse allí la mano con los emigrados, que dejaron de
serlo, pasando la frontera. El Obispo Abarca, Gómez Pardo, el cabecilla
o General D. Basilio, y el famoso canónigo y confesor Echevarría,
constituyéronse en autoridad revolucionaria, en nombre de Carlos V. Era
como una sombra de la Regencia de Urgell. ¡Tristes amaneramientos de la
Historia!

Lo primerito que se les ocurrió a los sediciosos, demostrando en ello
buen tino, fue nombrar su Comandante General; y aunque entre ellos
estaba D. Basilio, hombre de guerra, recayó la elección en el Canónigo,
quien de confesor de S. M. pasó a Jefe de Estado Mayor de la
Generalísima. Empuñó el hombre su bastón, y pasada revista a las tropas
con una felicísima mezcolanza de unción y marcialidad, largó su
correspondiente proclama, poniendo a Maroto a los pies de los caballos,
y procurando levantar el decaído espíritu de aquellos pueblos infelices,
honrados, inocentes, que habían hecho por la realeza de Carlos Isidro el
sacrificio de su sangre y su hacienda. Pero los pueblos, la verdad sea
dicha, no respondieron con el calor que se esperaba a la invocación del
clérigo metido a Macabeo. La fe en un Rey que no sabía gobernar ni
combatir se debilitaba rápidamente. Paces querían ya, aunque no se les
hablaba tanto de religión, que bien segura veían por todas
partes\ldots{} porque, verdaderamente, si tan \emph{partidario} de D.
Carlos era Dios, ¿a qué consentía los avances de Espartero y los
palizones que este venía dando a los caballeros del Altar y el Trono?

Y no se paraba en barras el Conde-Duque, seguro ya de ganar la partida.
Desde Villarreal de Álava, avanzó hacia el fuerte de Urquiola, donde fue
muy débil la resistencia. Sabedor de que su rival ocupaba a Durango con
fuerzas considerables, allá corrió dispuesto a batirle; pero Maroto, ya
en el grado último de turbación y azoramiento, le abandonó la villa,
marchándose a Elorrio. Hizo, pues, Espartero entrada triunfal en
Durango, y la animación y el orgullo de sus tropas, vencedoras sin
disparar un tiro, contrastaban con el desmayo y tristeza de los
batallones guipuzcoanos.

No estará de más decir que no fue para el Sr.~de Calpena motivo de gozo
la entrada en Durango. Temía que el encuentro de los Arratias le
produjese una situación penosa, y que los recuerdos apagados se avivasen
con la presencia de personas que no quería ver más en lo que le restara
de vida. Por fortuna suya, en el retraimiento que se impuso,
encarcelándose y entreteniendo sus ocios con lecturas, le descubrió el
sabueso de más fino olfato que por aquellos Reinos andaba: el sagacísimo
D. Eustaquio de la Pertusa, que una mañana se le apareció como por
escotillón, sirviéndole el chocolate, según testimonio del propio D.
Fernando en sus Memorias escritas y no publicadas. Adivinando el motivo
de la encerrona de su noble amigo, el astuto conspirador se apresuró a
tranquilizarle refiriéndole que todos los Arratias de ambos sexos habían
levantado el vuelo hacia Bilbao, en cuanto se agregaron a la familia
Zoilo y su padre. ¡Memorable día de abrazos y besos, reconciliaciones y
extremos de cariño! Felices parecían todos al emprender la marcha hacia
sus lares, y tan embobada con la criatura iba la juvenil pareja, que era
lógico esperar se cumplieran los deseos de Doña Prudencia, la cual no se
contentaba con menos de una criatura por año. La fecundidad de la guapa
moza garantizaría su dicha y la paz del matrimonio. Para D. Fernando
fueron estas referencias como si la sepulcral losa, que en el cementerio
de su corazón guardaba sus primeros amores, se levantase y se volviera a
cerrar. Trató de asegurarla bien, soldándola o claveteándola con buenas
razones, y trazó sobre ella con escoplo más firme las tres fúnebres
letras R.I.P.

Luego entró D. Eustaquio en informaciones muy interesantes de la
trapatiesta apostólica. Por un lado, D. Carlos no quería indisponerse
con Maroto, a quien creía capaz de un regicidio; por otro, alentaba a
los que en rigor de ley eran rebeldes. Para negros y blancos tenía una
palabra benévola. Él lo había visto, él, D. Eustaquio de la Pertusa;
nadie se lo contaba. Desde Lesaca mandó D. Carlos un recadito secreto al
Canónigo General, y este, bien disfrazado, fue a verle, y toda una media
noche pasaron conferenciando. Suponía \emph{el Epístola} que el objeto
del conciliábulo no era otro que ver el modo y ocasión de armar una
ratonera en que coger descuidado a Maroto, y hacer con él luego el mayor
y más ruidoso escarmiento de traidores. Al propio tiempo, Zaratiegui,
encargado por Maroto de sofocar la insurrección de los batallones
navarros, se situaba en Etulaín, decidido a liarse con ellos. Y el
General Elío, que también quería paces, mandaba al campo insurrecto a un
frailazo llamado Guillermo, marotista por excepción, para que arengase a
los navarros y les trajese a la disciplina, todo ello invocando siempre
el Altar y el Trono, que ya casi no tenían forma, de tanto como los
manoseaban, de tanta saliva como ponían en ellos los labios de los
oradores. Pero el buen fraile no sacó de sus prediques más fruto que una
ronquera penosa y el desaliento con que volvió y dijo a Elío que fuera
él a convencerles. En tanto, ¿qué hacía D. Carlos? Inalterable en su
doblez medrosa, largaba otra proclamita, diciendo horrores de los
rebeldes, llamándoles \emph{puñado de extraviados}, y amenazándoles con
destruir por sí mismo aquel germen de \emph{cobarde} y \emph{vil
traición}. En las cartas que se cruzaron entre Maroto y el canónigo
Echevarría, este le llamaba con todo desenfado \emph{traidor} y
\emph{asesino}.

Informado el Duque de estos hechos, mandó a Calpena que fuese al Cuartel
General de Maroto y allí se instalara, valiéndose de cualquier arbitrio,
con objeto de vigilar sus actos e influir en sus resoluciones, pues del
estado de trastorno en que se hallaba, todo podía temerse. Al propio
tiempo llevaba el encargo de anunciarle la proposición de entrevista,
que muy pronto se haría oficialmente por conducto de un parlamentario.
Si no la aceptaba, se le atacaría con esfuerzo combinado en toda la
línea, obligándole a una capitulación en que no le sería fácil obtener
las ventajas que él y sus compañeros obtendrían del convenio proyectado.

Con estas instrucciones partió D. Fernando a Salinas acompañado de Urrea
y de Pertusa, que se agregó muy contento a la embajada, estimando que su
concurso había de ser eficaz para el caballero, por su gran metimiento y
sus amistosas relaciones en el campo marotista. Poco antes de que los
tres llegaran a Salinas, había salido Maroto para Mondragón;
siguiéronle, agregándose a la retaguardia sin ningún cuidado, pues
\emph{el Epístola} era en aquel ejército como de casa, y el día próximo
alcanzaron al General no lejos de Vergara, por donde pasaron sin
detenerse. Iba Maroto decidido a refrenar en Lesaca la insurrección
apostólica, y a colgar de un alcornoque al canónigo Echevarría,
enracimado con otros clérigos y bárbaros caciques. Pero al llegar a
Villarreal se encontró D. Rafael con una novedad que hubo de causarle
tanta sorpresa como disgusto. Entraba su vanguardia en el pueblo por el
lado de Anzuola, y por el de Zumárraga comparecía la guardia de honor de
D. Carlos. Detrás venía la brigada del Cuartel Real, con el propio Rey,
procedente de Villafranca. A regañadientes, y con el cuerpo lleno de
bilis, Maroto no tuvo más remedio que afrontar la presencia de su señor,
y se llegó con su Estado Mayor a recibirle, creyendo que allí
permanecería. Pero D. Carlos no hizo más que una parada momentánea, sin
apearse del caballo; y al recibir los homenajes de su General, pálidos
ambos como difuntos, recelando el uno del otro, le dijo: «Sígueme: voy a
Anzuola\ldots» Automáticamente, sin darse cuenta de lo que hacía, se
agregó a la escolta, y siguieron Rey y vasallo silenciosos hasta cerca
de Descarga. Allí paró un instante D. Carlos, y llamando a su lado a
Maroto, repitió: «Sígueme hasta Anzuola. Tenemos que hablar.» Maroto,
que había dejado en Villarreal su escolta y ayudantes, presintió que se
le quería llevar a una encerrona. Se vio fusilado ejecutiva y
cruelmente, en el estilo sencillísimo que él empleara con Guergué, y
evocando su entereza contestó al hijo de Carlos IV: «Señor, los cuerpos
están formados y tengo que darles una orden muy precisa.» Y sin añadir
otras razones, ni aguardar las que el Rey pudiera darle, volvió grupas,
caminito de Villarreal. De lejos, alzando la voz, queriendo ser
enérgico, y sin dejar de ser tímido, el Pretendiente le dijo:
«Cuidado\ldots{} que te espero en Anzuola.» Con un movimiento de cabeza
respondió Maroto que sí, y se alejó al trote, difiriendo la entrevista
para la vuelta, que sería \emph{la del humo}.

\hypertarget{xxxv}{%
\chapter{XXXV}\label{xxxv}}

Hasta el día siguiente muy temprano no pudo ver D. Fernando al General,
porque se encerró en su alojamiento con órdenes de no dar paso a nadie.
¿Qué hacía?, ¿qué pensaba? Le atormentaba el cruel dilema de obedecer a
su señor o volverle la espalda para siempre. Antes de ser recibido, supo
Calpena que había pasado la noche en cama con alta calentura, privado a
ratos de conocimiento. Al entrar el caballero en la alcoba de Maroto,
tardó un instante en conocerle: tan desfigurado estaba por los
sufrimientos. Además, acababa de afeitarse quitándose el bigote. Su cara
parecía otra, por efecto de esta mutilación, del color cárdeno de sus
ojeras, de las arrugas que surcaban su piel amarilla, del desordenado
cabello. Había envejecido diez años, perdiendo su gallardía militar. Al
ver a D. Fernando, le dijo: «Hola, \emph{Inquisivi}\ldots{} ¿Otra vez
por acá?»

---Sí, mi General: otra vez aquí con la esperanza de ser a usted útil, y
de servir, no a mi partido, sino a mi patria.

Abordando el asunto, notó Fernando un grave desorden en las facultades
del Caudillo, que tan pronto expresaba sus anhelos de paz como su
repugnancia del dictado de traidor que en el Cuartel Real se le
aplicaba. La proposición de entrevista le puso en un estado de inquietud
epiléptica. Llevándose las manos a la cabeza, con voces roncas,
destempladas, replicó: «No puede ser\ldots{} Me comprometen\ldots{} ¡El
Rey\ldots! Soy General de Carlos V, soberano legítimo\ldots{} ¿Usted qué
opina? ¿Debo ir a la entrevista?\ldots{} ¿Acaso irá Simón de la Torre?»

---Creo que sí---dijo Calpena, juzgando de gran efecto la afirmativa.

---Pues que sea suya la responsabilidad. ¿Y asistirán también los
ingleses? ¡Malditos ingleses!\ldots{} Yo no, yo no puedo ir\ldots{} Lo
consultaré con D. Carlos. A nadie conviene más la transacción que a
nuestro pobre Rey, ese bendito, ese bendito\ldots{} Pero no, no: antes
tengo que colgar de un alcornoque al Canónigo\ldots{} Sin eso no hacemos
nada\ldots{} Y de otro alcornoque a D. Basilio, y empalar al malvado
Teijeiro\ldots{}

No había manera de sacarle de este círculo de ideas. Descompuesto y
contradiciéndose a cada instante, ordenó que se preparara su escolta,
reforzada con la mejor caballería de su ejército, y sin tomar ningún
alimento, montó a caballo y se fue al Cuartel Real. Regresó al
anochecer; en Villarreal se aseguraba que Maroto había presentado su
dimisión al Rey; que este, poco menos que llorando, le había dicho:
«¿Con que ahora me vas a abandonar?\ldots» Algo enternecido también, D.
Rafael se deshizo en demostraciones de lealtad, manifestándose dispuesto
a sacrificarse por la Causa\ldots{} Esto se decía, y sobre ello
endilgaron comentos mil D. Fernando y Pertusa, con los oficiales que les
hacían coro en la cantinela de la paz. Convenían todos en que no era
fácil entender a Rafael Maroto, monstruoso enigma en que se reunían
todas las complejidades psicológicas. Decía \emph{el Epístola} con sutil
ingenio: «Esta mañana, después de una horrible noche de insomnio y
fiebre, el General debió de saltar del lecho con una idea
salvadora\ldots{} Así me lo figuro yo, y así tiene que ser\ldots{} Pues
saltando del lecho cogió la navaja de afeitar\ldots{} Por un momento
pensó en degollarse, la mejor solución de sus horribles dudas\ldots{}
Después pensó otra cosa quizás más práctica\ldots{} escapar a la
calladita, vestido de cura\ldots{} Por eso se quitó el bigote. No tiene
otra explicación.»

No pareció mal a los amigos presentes la versión del \emph{Epístola}, y
convinieron con Calpena en que todos, Rey, General y Canónigo, habían
perdido el juicio. El carlismo había venido a ser un campo de orates. Al
día siguiente dio un súbito cambiazo la voluntad indecisa del desdichado
caudillo, y en vez de dirigirse a Lesaca, según lo convenido con el Rey,
se encaminó a Elgueta. No bien entraron en este pueblo, supo D. Fernando
la llegada de su amigo Zabala, ya brigadier, que con el carácter de
parlamentario venía de parte del Duque de la Victoria. Negose Maroto a
recibirle; trabajó Calpena por lo contrario, empleando más de una hora
en argüirle con cuantos resortes lógicos creía propios del caso, y al
fin accedió el General gruñendo: «Pues sea, y acabemos de una vez,
¡porra!\ldots» El día 25, a las seis de la mañana, se reunían en la
venta de Abadiano, entre Durango y Elorrio, D. Baldomero Espartero con
el Brigadier Linaje y el coronel inglés Wilde, representado la idea
constitucional, y por la idea absolutista D. Rafael Maroto y el General
Urbistondo, jefe de los batallones castellanos. La magna cuestión de los
Fueros trajo el desacuerdo de los conferenciantes, porque los carlistas
pedían que se reconociese el régimen foral en toda su pureza, y
Espartero no quería comprometerse a tanto, dejando el grave asunto a la
resolución de las Cortes. Manifestose Linaje contrario a los Fueros,
sosteniendo que el fanatismo había sido el único móvil del levantamiento
carlista; cruzáronse agrias contestaciones entre Linaje y Urbistondo, y
entre el jefe de los castellanos y Maroto, pues este, al llevar a su
compañero a la conferencia, le había manifestado que, en las
negociaciones preliminares, ambas partes estaban conformes en el
reconocimiento incondicional de los Fueros. Negolo Espartero,
atribuyendo la idea de su rival a mala inteligencia. Al cabo de tanto
discutir se separaron en desacuerdo. No había paz, no podía España
disfrutar de este inmenso bien.

Cuando se retiraban, cada cual por su lado, llegó D. Simón de la Torre,
que fue en seguimiento de Espartero, y alcanzándole cerca de Durango, se
declaró dispuesto, con los ocho batallones de su mando, a transigir
resueltamente sin regatear ninguna condición. En tanto, volvió Maroto a
Guipúzcoa dando tumbos, que no de otra manera puede expresarse la
inseguridad de sus movimientos, reflejo de la horrible lucha de su
espíritu, y en la villa de Elgueta se encontró nueva sorpresa y
emociones tan vivas, que ellas bastarían a quitarle el seso si alguno en
aquella ocasión le quedara. De improviso se presentó el Rey con su
escolta en el Cuartel General, y antes de que Maroto pudiese tomar
resolución alguna, mandó formar los 14 batallones para pasarles revista
y arengarles. Así se acordó en una junta celebrada por Carlos V el día
anterior, al tener conocimiento de la entrevista de Abadiano. Había
llegado el instante en que el Rey lo era de hecho, y como tal procedería
con soberana entereza y celeridad. Pronto vería el mundo si merecía la
corona. Revistar a las tropas que formaban el núcleo de su ejército;
presentarse a ellas, no sólo como Rey, sino como Generalísimo, asumiendo
el mando directo; destituir en el acto al desleal caudillo, y aplicarle
sin consideración sumariamente la pena que le correspondía, era un acto
propio de Monarca guerrero. Si el programa se cumplía, ¡qué hermosa
solución de los enmarañados problemas pendientes, qué gallarda manera de
cortar el nudo que en vano con su estira y afloja había querido desatar!

Ante el aparato que en torno al Soberano se desplegaba, Maroto se vio
perdido, se sintió fusilado\ldots{} De su cráneo a su olfato descendía
el olor de pólvora. Para mayor solemnidad del acto, presentábase el Rey
de gran uniforme, con todas sus cruces, bandas y collares, radiante de
inepta vanidad, y le acompañaban su hijo Carlitos, Príncipe de Asturias;
el Infante D. Sebastián y los Generales Eguía, Valdespina, Villarreal y
Negri\ldots{} Formaron las tropas. La expectación era para algunos como
si esperaran el fin del mundo\ldots{} Rompió al fin el Rey en una
perorata que llevaba bien aprendida; pero su voz no vibraba, no sabía
llegar a los oídos lejanos, no era instrumento para conmover y
entusiasmar a las muchedumbres. Se observaban en su rostro y en su
actitud los inútiles esfuerzos para ponerse en la situación que el grave
caso exigía, para desempeñar airosa y noblemente el papel de Rey, para
imitar la marcial fiereza, la grandiosa altivez de los más célebres
capitanes en circunstancias como las de aquel momento. Oyeron los más
próximos algunos conceptos en que el hijo de Carlos IV evocaba las
sombras de César y Aníbal; algo dijo luego de los cántabros indomables,
de Roma, señora del mundo\ldots{} No dejó de causar sorpresa que
omitiese la rutinaria invocación a la Generalísima, Nuestra Señora de
los Dolores. No estaba sin duda la Causa absolutista para
tafetanes\ldots{} Por fin, viendo el buen señor que no producía el
efecto que se proponía, y conociendo que ni su acento ni su ademán
respondían a la majestad que intentaba poner en ellos, se comió la mejor
parte del preparado sermón, y fue derecho en busca del efecto final.
«Hijos míos---exclamó ahuecando la voz todo lo que pudo,---¿me
reconocéis por vuestro Rey?» La contestación fue un «¡Sí, sí\ldots{}
viva el Rey!» que corrió, extinguiéndose en las filas lejanas. «¿Y
estáis dispuestos---añadió,---a seguirme a todas partes, a derramar
vuestra sangre en defensa de mi Causa y de la Religión?»

Silencio en las filas. No se oyó ni un murmullo ni un aliento. El
General Eguía, alzándose sobre los estribos, y poniéndose rojo del
esfuerzo con que gritaba, dio varios vivas que fueron contestados
fríamente. De las segundas filas vino primero un rumor tímido, después
exclamaciones más claras, por fin estas voces: «¡Viva la paz, viva
nuestro General, viva Maroto!»

---¡Voluntarios!---ritó entonces D. Carlos, y en ocasión tan crítica la
dignidad brilló en su rostro\ldots{} Al fin descendía de cien
Reyes.---Voluntarios, donde está vuestro Rey no hay General
alguno\ldots{} Os repito: ¿queréis seguirme?»

Silencio sepulcral. El Brigadier Iturbe, jefe de los guipuzcoanos,
acudió a remediar con un pérfido expediente la desairada, angustiosa
situación del Monarca. «Señor---le dijo,---es que no entienden el
castellano.» Y D. Carlos, tragando saliva, le ordenó que hiciera la
pregunta en vascuence. Pero Iturbe, que era de los más comprometidos en
la política marotista, formuló la pregunta con una alteración grave:
\emph{¿Paquia naidezute, mutillac?} (¿Queréis la paz, muchachos?) Y con
gran estruendo respondió toda la tropa: \emph{¡Bai jauna!} (Sí, señor.)

Debió D. Carlos sacar su espada y atravesar con ella al brigadier
guipuzcoano, castigando en el acto la grosera, irreverente burla. Volvió
la cara lívida, y vio tras sí a Maroto, que de su mortal zozobra se
recobraba viendo convertido en sainete el acto iniciado con trágica
grandeza. D. Carlos, incapaz de arranque varonil, tuvo dignidad. Dijo a
los de su escolta: «estamos vendidos;» y sin más discursos, ni
pronunciar ligera recriminación, volvió grupas y picó espuelas, saliendo
al galope por el camino de Villafranca, con la reata de Príncipes y
Generales y la menguada escolta. Corrieron, corrieron sin respiro,
temerosos de que los sicarios de Maroto fueran en su seguimiento.

\hypertarget{xxxvi}{%
\chapter{XXXVI}\label{xxxvi}}

Testarudo como él solo, D. Carlos no se daba ni en tales extremidades
por vencido, y apenas llegó a Villafranca, jadeante, llamó a Consejo a
sus adictos, los Generales que le acompañaron en la fracasada escena de
Elgueta, el Padre Cirilo de Alameda, el Barón de Juras Reales, Erro y
Ramírez de la Piscina, algunos de los cuales aún se llamaban Ministros.
Opinaron casi unánimemente que S. M. debía situarse en punto cercano a
la frontera, para poner a salvo su sagrada persona en el desecho
temporal que la Causa corría. Trabajillo le costaba al buen señor
determinarse a partir arrojando en las puertas de Francia su corona, y
acariciaba el ensueño de reunir algunos batallones navarros y alaveses
que le llevaran en procesión al Maestrazgo, donde aún tenía un ejército
y un General incorrupto y valiente: Cabrera. Estimaron todos peligrosa
la marcha al Centro; pero le dejaban consolarse con esta ilusión.
Aferrado a su realeza, D. Carlos enderezó nueva proclama a sus míseras
tropas, en la cual les hablaba de la \emph{traición más infame que
habían visto los nacidos}, y concluía llamándoles héroes, y dando vivas
a la sacra Religión. ¡Bueno estaba el país para estos suspirillos!

En tanto, Maroto, después del triunfo de Elgueta, caía en gran
postración, atormentado por su conciencia, y procurando en vano salir
limpio y airoso de la charca en que se había metido. Calpena y Uhagón,
que acudieron a su lado el 26, un día después de la famosa revista, se
maravillaron de verle en un grado increíble de turbación y apocamiento.
Poco le faltaba para llorar; sus conceptos habían quedado reducidos a
una exclamación maníaca: no decía más que: «No soy traidor\ldots{}
Maroto no pasará a la Historia con un dictado infamante\ldots{}
Convencido estoy de que el absolutismo es imposible\ldots{} Pero no
cedo, no cedo, si no me dan los Fueros íntegros, la gloria de este país.
Maroto no es traidor. Maroto es un hombre honrado, un buen
español\ldots{} ¡Ay del que lo ponga en duda!»

Toda la tarde y parte de la noche permanecieron a su lado los dos
amigos, arguyéndole con habilidad, sin lastimar su amor propio, antes
bien fundado en este todo el trabajo sugestivo con que querían llevarle
a la aceptación incondicional del Convenio. ¿Qué otra solución podía
soñar? ¿Qué esperaba, qué temía? Retiráronse en la creencia de que le
dejaban convencido, pues esperanzas de ello daban sus expresiones
conciliadoras; pero D. Fernando, que ya conocía su indecisión y el
confuso laberinto a que había llegado su voluntad, no las tenía todas
consigo\ldots{} Repetida por la mañana la visita, le encontraron
escribiendo una carta. Despidioles el General con acritud. La carta que
escribía era la famosa retractación dirigida a D. Carlos, en la cual le
decía: \emph{Nunca es más grande un Monarca que cuando perdona las
faltas de sus vasallos\ldots{} D. Eustaquio Laso presentará a Vuestra
Majestad los sentimientos de mi corazón para que se digne dirigirme las
órdenes que fuesen de su agrado. }

Ignoraban Calpena y su amigo esta humillación increíble; mas del
trastorno de Maroto tuvieron prueba clara cuando se llegó a ellos un
ayudante con el recado conminatorio de que si los caballeros y el
llamado \emph{Epístola} no se largaban pronto del Cuartel General, se
les mandaría fusilar. No eran cobardes: no perdieron la serenidad con
esta brutal amenaza; mas la prudencia les aconsejaba ponerse en salvo, y
a ello se disponían, cuando llegó D. Simón de la Torre, que, informado
de los desvaríos de Maroto, les tranquilizó con respecto a sus vidas.
Conferenciaron los dos jefes, y por la noche salieron con sus fuerzas
reunidas en dirección de Azpeitia. Los tres paisanos ignoraban a qué
razón militar o política obedecía tal movimiento, y no se ocuparon más
que de seguir a las tropas, acogidos a la caballerosidad e hidalguía del
simpático La Torre. En Azpeitia se les dijo que Espartero avanzaba
triunfalmente por el interior de Guipúzcoa; que había entrado en
Vergara, donde te acogieron con ardientes demostraciones en favor suyo y
de la paz. De Vergara pasó a Oñate, y la vieja Corte le recibió con
palmas. Dirigiose Maroto a Villarreal, donde como llovido se le presentó
al conde de Negri con una orden del Rey para que le entregase el mando.
Al recibir D. Carlos la carta palinodia, habíala estimado como la mayor
prueba de traición y perfidia. Los de la camarilla vieron en aquel paso
un ardid diabólico para aproximarse al vencido Monarca, apoderarse de su
persona y entregarla en trofeo a los constitucionales para un sacrificio
que fuera digno epílogo de guerra tan sangrienta. Rompió el Soberano la
carta del vasallo infiel, y mandó a Negri a desposeerle del mando,
determinación ridícula en situación tan extremada. Como era natural,
tanto Maroto como La Torre acogieron al conde de Negri con escarnio de
su persona y de quien tal comisión le daba. Salió de estampía el buen
Conde, que al volver al lado de su triste Rey, le dio con la respuesta
de los que fueron sus Generales franco pasaporte para Francia.

Ante la irresistible presión de este suceso, Maroto confió
decididamente, al parecer, a sus compañeros La Torre y Urbistondo la
misión de llevar a Oñate su conformidad con el Convenio, tal como se le
había presentado en Abadiano. \emph{¡Alleluia!} La paz era un hecho. Al
despedirse para tan grato mensaje, Don Simón reconcilió a sus amigos con
el jefe, que sin acordarse ya de que había pensado fusilarles, les
convidó a comer muy afectuoso. Durante el día, observáronle más sereno y
en vías de recobrar su equilibrio; mas por la noche advirtieron de nuevo
en él cierta intranquilidad, y una insistencia monomaníaca en hablar de
fueros \emph{netos}, intangibles. Temerosos de un nuevo cambiazo del
veleidoso General, trataron de explorar su pensamiento. «Por mi
parte---les dijo,---a todo estoy dispuesto, y cuando me traigan de Oñate
el Convenio cuyas bases he admitido, lo firmaré\ldots{} Pero dudo que
algunos cuerpos de mi ejército, principalmente los guipuzcoanos, lo
acepten\ldots{} De modo que no hemos hecho nada, y la guerra
continuará.» A esto arguyó Calpena que antes de proceder a la solemne
ratificación de lo tratado, debía el General conferenciar con los jefes
y oficiales, uno por uno, y darles cuenta de las condiciones de paz a
que todos debían someterse.

«Háganlo ustedes»---dijo Maroto, revelando en su tono y en su actitud
una indolencia que llenó de asombro a los dos amigos.

---Pero, General---le contestaron,---¿qué autoridad tenemos nosotros
para convencer a las tropas vizcaínas y guipuzcoanas de que, ante el
bien inmenso de la paz, deben contentarse con la fórmula vaga del
reconocimiento de Fueros?

---No es tan vaga. Se estipula que Espartero propondrá a las
Cortes\ldots{}

---Pero eso, sea poco, sea mucho, es lo que el Duque les concede, y
deben saberlo. Usted, su Jefe, que ha de firmar por todos el pacto, está
en el caso de instruirles\ldots{}

---Mi cansancio es tal, amigos míos, que ya no sé cómo valerme, ni halla
mi pensamiento voces con que producirse\ldots{} Hay momentos en que me
creo sin vida\ldots{}

---Pero el trabajo restante, para llegar a un fin glorioso, es breve y
fácil, mi General.

---Fácil no, ¡porra!

¡Cualquiera le convencía! Llegaron de Oñate los comisionados La Torre, y
Urbistondo con Zabala y Linaje, portadores del Convenio, que Maroto
firmó sin ninguna dificultad. Al propio tiempo traían la comisión de
proponerle que al día siguiente, 30 de Agosto, se reunirían en Vergara
los dos ejércitos, con sus caudillos a la cabeza, para dar forma solemne
a la grande obra de la reconciliación. A todo asintió D. Rafael, que
aliviado parecía de un peso abrumador.

Uhagón y Calpena pasaron el día recorriendo los cuerpos, en que tenían
no pocos amigos, y hablando con unos y otros campechanamente. Si en
todos reconocían la satisfacción y júbilo por ver terminada la odiosa
discordia, causoles no poca inquietud el observar que los soldados y
oficialidad carlistas descansaban en el engaño de que el pacto reconocía
los Fueros en toda su integridad, y que así se declaraba de una manera
explícita. Maroto les tenía en esta persuasión, pues nada en contrario
les había dicho desde la ineficaz entrevista de Abadiano. Era, pues,
indudable que surgirían en el momento que se creía final nuevas
complicaciones, quizás un gravísimo conflicto, por la indolencia del
General, por su falta de carácter y de resolución para presentar los
hechos como realmente eran. ¡Torpeza insigne, abandono de autoridad!

Sobresaltado, temeroso de ver perdido en un instante el ímprobo trabajo
de tantos meses, creyó D. Fernando que debía prevenir a Espartero de lo
que ocurría, evitándole un triste desengaño al llegar a Vergara, donde
contaba con la presencia y conformidad del ejército carlista. Pensado y
hecho: de madrugada montó a caballo, y seguido de Urrea y Pertusa se fue
al encuentro de su General, a quien halló a media hora de Vergara. No
daba crédito D. Baldomero a la triste realidad que le comunicó su amigo,
y ante la insistencia de este, más de un cuarto de hora estuvo echando
ternos, y maldiciendo la hora en que entabló negociaciones con hombre
tan inseguro y tornadizo. En efecto: poco antes de entrar el Duque en
Vergara, llegó Maroto, sin más compañía que la del General La Torre y
algunos oficiales de su Estado Mayor. Y los 21 batallones y los tres
escuadrones que debían figurar como convenidos, ¿dónde estaban? Sin
pérdida de tiempo avistose Espartero con su antagonista, el cual hubo de
contestar a la anterior pregunta, con turbado acento, que las tropas se
negaban al cumplimiento de lo pactado mientras no se reconociesen los
Fueros provincianos en toda su integridad. Según esto, Maroto declaraba
a su ejército en rebeldía, y se presentaba él solo, \emph{con cuatro
gatos}; y él solo reconocía los derechos de Isabel, dejando en el aire
la obra de la paz, y a las tropas apartadas de toda reconciliación.

«A este hombre hay que dejarle---dijo D. Baldomero, luego que Maroto,
afectado de gran postración, se retiró a descansar.---Imposible hacer
carrera de él\ldots{} ¡Qué hombre, santo Dios! Verdad que su situación y
los contratiempos que ha sufrido son para trastornar la cabeza más
firme.» En esto, La Torre se apresuró a manifestar a Espartero con
gallardo arranque que él se comprometía, en el término de veinticuatro
horas, a convencer a los vizcaínos o morir en la demanda. No descansó
Maroto, pues su conciencia y sus embrollados pensamientos no se lo
permitían, y llamando a Calpena, como se llama a un confesor en la
última hora, le dijo: «Hágame el favor de comunicar al coronel Wilde
que, no creyéndome seguro aliado de Espartero por haber venido aquí sin
tropas, me acojo al pabellón inglés.» A esto respondió el caballero que
no necesitaba añadir a sus errores la mengua de ampararse a una nación
extranjera; bien seguro estaba en el Cuartel General del Duque de la
Victoria, toda vez que reconocía la legalidad por este representada. En
tanto, los bravos generales carlistas La Torre, Urbistondo y el
Brigadier Iturbe, con riesgo de sus vidas, tratarían de reducir a las
tropas a la aceptación de lo tratado, después de darles conocimiento del
artículo 1.º del Convenio\ldots{}

«¿Y cómo queda redactado al fin?---dijo Maroto vivamente---Ya no me
acuerdo.»

---Poco más o menos dice: \emph{Artículo 1.º El General Espartero
recomendará con interés al Gobierno el cumplimiento de su oferta de
comprometerse formalmente a proponer a las Cortes la concesión o
modificación de los fueros}.

---¿Y las Cortes\ldots? Claro, las Cortes\ldots{} Me parece bien\ldots{}
Buenos tontos serán esos pobres muchachos si no aceptan, si no fían
resueltamente en la promesa del Duque, de cuya caballerosidad nadie
puede dudar\ldots{} Por mi parte, no escatimaré ningún sacrificio.
Hágame el favor de llamar a mi ayudante, D. Enrique O'Donnell, para
dictarle algunas órdenes. Aún soy General en Jefe de mi ejército, del
ejército Real, desde hoy incorporado al de la Nación.

\hypertarget{xxxvii}{%
\chapter{XXXVII}\label{xxxvii}}

Mientras La Torre trabajaba por reducir a los vizcaínos, Urbistondo
hacía lo mismo con los castellanos. No tuvo igual fortuna Iturbe con los
de Guipúzcoa, que enterados de la vaga promesa consignada en el artículo
primero, se negaron a suscribir el Convenio, gritando \emph{¡traición,
traición!}; y declarados en franca rebeldía, manifestáronse dispuestos a
unirse con D. Carlos. Al fin pudo Iturbe contenerles en Descarga.
Urbistondo situó fuerzas castellanas en la carretera, con objeto de
observar a los guipuzcoanos, y corrió en busca de Maroto para que
saliese al frente de ellos y con su autoridad les redujera. Era la noche
del 30, y D. Rafael, que estaba en cama, dolorido, incapaz para toda
acción, dijo a Urbistondo que se entendiese con Espartero. Así lo hizo.
Se convino en no contar para nada con D. Rafael, que se había echado en
el surco, como hombre históricamente concluido, y no hubo más remedio
que intentar la pacificación de los guipuzcoanos, comprometiendo entre
ellos la vida, catequizando uno por uno a jefes y oficiales, sin reparar
en la clase de argumentación con tal de llegar al fin deseado. En esto
se empleó toda la noche del 30; al fin, el 31 de madrugada desfilaban
hacia Vergara los batallones reacios precedidos de cuerpos castellanos,
para que la moral de estos fuese para todos ejemplo provechoso, y así,
con más maña que fuerza, empleando sin cesar la palabra convincente,
cariñosa, paternal, que igualaba al jefe con el soldado, fueron
aproximándose al redil.

Era este un extenso campo a la salida de la villa, entre el río Deva y
el camino de Plasencia. Allí formó muy de mañana el ejército de
Espartero, y ante él fue desfilando la división castellana, con su jefe
el General Urbistondo. Maroto, que parecía resucitado, a juzgar por la
repentina transformación de su continente, que recobró su gallardía, así
como el rostro la expresión confiada y el color sano, ocupó su puesto;
al punto apareció con su brillante Estado Mayor el Duque de la Victoria,
y recorridas las líneas, cautivando a todos con su marcial apostura y la
serenidad y contento que en su rostro se reflejaban, mandó a sus
soldados armar bayonetas; igual orden dio Maroto a los suyos. Espartero,
con aquella voz incomparable que poseía la virtud de encender en los
corazones la bravura, el amor, el entusiasmo y un noble espíritu de
disciplina, pronunció una corta arenga perfectamente oída de un lado a
otro de la formación, y terminó con estas memorables palabras:
\emph{Abrazaos, hijos míos, como yo abrazo al General de los que fueron
contrarios nuestros.} Juntáronse los dos caballos; los dos jinetes,
inclinando el cuerpo uno contra otro, se enlazaron en cordial apretón de
brazos. Maroto no fue de los dos el menos expresivo en la efusión de
aquella concordia sublime. En las filas, de punta a punta, resonó un
alarido, que parecía explosión de llanto. No eran palabras ya, sino un
lamento, el ¡ay! del hijo pródigo al ser recibido en el paterno hogar,
el ¡ay! de los hermanos que se encuentran y reconocen después de larga
ausencia. Era un despertar a la vida, a la razón. La guerra parecía un
sueño, una estúpida pesadilla.

Se había dispuesto que las divisiones vizcaínas y guipuzcoana entrasen
en el campo del convenio después de comenzado el acto, para que la
solemnidad de este y su ternura influyesen en el ánimo de los reacios, y
el efecto correspondió a lo que Espartero y Urbistondo con tanta
habilidad y conocimiento del humano corazón habían dispuesto. Las tropas
guiadas por La Torre como las conducidas por Iturbe, se vieron envueltas
en la inmensa atmósfera de fraternidad que ya se había formado. Los
corazones respondieron con unánime sentimiento. No podía ser de otro
modo. La idea de unidad, de nacional grandeza, de moral parentesco entre
todas las razas de la Península, ganó súbitamente los entendimientos de
castellanos y éuskaros, y ya no hubo allí más que abrazos, lágrimas de
emoción, gritos de alegría, aclamaciones a Espartero, a la Constitución,
a Isabel II, a Maroto, a la Religión y a la Libertad juntamente, que
también estas dos matronas se dieron de pechugones en aquel solemne día.

\hypertarget{xxxviii}{%
\chapter{XXXVIII}\label{xxxviii}}

En los mismos 30 y 31 de Agosto, D. Carlos continuaba emitiendo
proclamas desde Andoaín y desde Lecumberri, en las cuales hablaba
\emph{del rebelde Espartero} como de un enemigo insignificante; echaba
la culpa de sus desgracias a la intriga, a las malas artes de los
pérfidos; delataba \emph{planes maquiavélicos} de los dos Generales
\emph{compañeros en las revoluciones de América}; atribuía la defección
de Maroto al \emph{oro que había recibido de los constitucionales}, y,
por fin, hacía postrer llamamiento a sus fieles súbditos para que se
acogieran a \emph{su paternal benevolencia}, ofreciendo olvido de lo
pasado si volvían a la defensa del Trono y la Religión. A los leales les
llamaba \emph{la más preciosa joya de su corona}. ¡Y con estas retóricas
sermonarias, con este lamentar de pastores, pretendía el pobre hombre
congregar de nuevo su disperso rebaño! La desbandada se inició al tener
conocimiento del abrazo de los Generales, que fue tiernísima
reconciliación de los dos ejércitos. El \emph{sálvese el que pueda}
resonó en los valles, que había ensordecido el estruendo guerrero de
seis años de lucha fratricida. Cada cual pensó en salvar lo que poseía,
y en último caso la pelleja, que es \emph{la más preciosa joya} de cada
mortal. Los restos de los sublevados de Irurzun y Vera, de aquel
flamante ejército apostólico y neto, que, levantando bandera por la
integridad de los derechos de Carlos, puso a su frente al canónigo
Echevarría, se desbordó en la más horrible desmoralización,
convirtiéndose los valientes navarros en vulgares ladrones y desalmados
homicidas. So color de castigar traidores, acosaban a los infelices
\emph{ojalateros}, que iban buscando su salvación por los caminos de
Francia, y les arrebataban cuanto tenían. El pillaje y el asesinato, la
persecución de hombres y el atropello de infelices mujeres fueron la
campaña postrera de aquellos degenerados vestigios de un grande
ejército. El mismo Echevarría estuvo a punto de perecer a manos de sus
soldados ebrios; D. Basilio y Guibelalde, puestos en capilla, escaparon
de milagro. Menos dichoso el General González Moreno, de lúgubre
memoria, el \emph{verdugo de Málaga}, caudillo inepto en Mendigorría,
hombre de quien puede decirse que fue una de las más negras fatalidades
del bando carlista, pereció cerca de Urdax, de un modo desastroso y vil,
digno término de una ruin vida. Dieron en creer los forajidos que iban
llenas de dinero las cajas que el General llevaba en su presurosa fuga,
y \emph{como a un cerdo} (así lo cuenta un testigo presencial) le
mataron en medio de las calles.

La que aún se llamaba Corte, el fracasado Rey y los fieles que le
seguían continuaban en Elizondo sin saber dónde meterse ni por qué
resquicios escurrir el bulto. Incansable, corrió allá Espartero; D.
Carlos oyó el galopar de su caballo, y acercose más a la frontera. Allí
quemó el absolutismo su postrer cartucho. El batallón cántabro, último
en la fidelidad, primero en el valor, defendió con estoica bravura las
posiciones de Urdax contra las fuerzas triplicadas que allí mandó el
Duque de la Victoria. Batiéndose con desesperación, mártires de la fe
del deber, los cántabros pudieron decir a su expugnador: \emph{morituri
te salutant}. Una columna de cazadores y una sección de tiradores de la
Princesa, mandados por Zabala, dominaron el terreno, dando por terminada
la acción, y con ella la guerra del Norte. Antes de que sonaran los
últimos tiros, montaron a caballo el Rey, la Reina y demás personas de
la familia y servidumbre, y a todo correr emprendían la fuga sin parar
hasta Francia. Había entrado Carlos seis años antes por el mismo boquete
de la frontera, siendo recibido por Zumalacárregui; se retiraba
escoltado por algunos números de su guardia, solo, triste, más abatido
que desengañado, sin ninguna gloria personal. La corona de la dignidad
con que supo sobrellevar su destierro fue la única que poseyó en su
vida.

D. Fernando Calpena y D. Santiago Ibero, testigos de la última refriega
con los valientes cántabros, admiraron el tesón de estos y les colmaron
de alabanzas. De regreso al Cuartel General de Elizondo, expresaron los
dos amigos su alegría por la terminación feliz de tan dura, enconada
campaña, y cada cual dijo lo que le sugería su conocimiento de hombres y
cosas.

«Hemos acabado una guerra---declaró Ibero con melancolía,---y yo me
felicito de este descanso que pronto disfrutaremos. Un descanso, por
corto que resulte, siempre es de agradecer. Pero le diré a mi amigo con
franqueza que no creo en la paz\ldots{} Soy ateo de esta religión que
ahora fanatiza a mis compatriotas\ldots{} No creo, no creo\ldots»

---Yo tampoco. La grande obra de nuestro General es una tregua que
debemos alargar todo lo que podamos. Las treguas son necesarias. Así nos
prepararemos para dar al problema, en otro día, solución más segura y
radical.

---Yo estoy triste\ldots{} no sé por qué\ldots{} Lo diré sin
rebozo\ldots{} Me gustaba el delirio, la barbarie, la guerra, en fin.

---Es realmente un estado muy vital, y además interesante y pintoresco.

---Si vivimos, no envejeceremos en la paz.

---Seremos siempre jóvenes, es decir, guerreros.

---El Convenio, el abrazo, no son más que la fórmula del cansancio.

---Del descanso, querrá usted decir.

---Eso. Se nos permite echar una siesta en día caluroso, el día del
siglo.

---Durmamos un poquito.

---Y descansemos, que buena falta nos hace.

\medskip
\begin{center}
\line(1,0){50}
\end{center}
\medskip

En la opinión del carlismo, quedó Maroto como el prototipo de la
traición y la perfidia. No era justo. A sus defectos, con ser grandes,
toca menos responsabilidad que a su destino cruel, y a la disparidad
entre su carácter y el personal absolutista, entre sus ideas y la causa
que defendió. El brazo eclesiástico, firme apoyo de la facción
(descoyuntado en Vergara, recompuesto después), no perdonó a Maroto su
cooperación en la obra de la paz, como se verá por este hecho
rigurosamente histórico. Recompensado por el Gobierno de Isabel con un
alto cargo militar, residió D. Rafael algún tiempo en España. Su hija
Margarita, joven de acrisoladas virtudes, que no se descuidaba en sus
prácticas religiosas, fue a confesar una mañana, una tarde (no importa
la hora), en una iglesia que no hace al caso. Cumplió serena y contrita,
declarando sus pecados, que no debían de ser graves, y cuando terminaba,
le preguntó el sacerdote su nombre. La pobre niña, tímida y pura, ¿qué
había de hacer? Se lo dijo\ldots{} Lo mismo fue oírlo el cura que de un
bote se levantó iracundo, y con destempladas voces la despidió,
negándose a darle la absolución. Atribulada, llorosa, salió la penitente
de la iglesia y no paró hasta su casa. ¿Se pone en duda este hecho? Pues
de él puede dar testimonio Doña Margarita Maroto, viuda de Borgoño,
anciana respetabilísima, que aún vive. Reside en Valparaíso.

\flushright{Santander-Madrid, Octubre-Noviembre de 1899.}

~

\bigskip
\bigskip
\begin{center}
\textsc{fin de vergara}
\end{center}

\end{document}
