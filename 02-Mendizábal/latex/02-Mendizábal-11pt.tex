\PassOptionsToPackage{unicode=true}{hyperref} % options for packages loaded elsewhere
\PassOptionsToPackage{hyphens}{url}
%
\documentclass[oneside,11pt,spanish,]{extbook} % cjns1989 - 27112019 - added the oneside option: so that the text jumps left & right when reading on a tablet/ereader
\usepackage{lmodern}
\usepackage{amssymb,amsmath}
\usepackage{ifxetex,ifluatex}
\usepackage{fixltx2e} % provides \textsubscript
\ifnum 0\ifxetex 1\fi\ifluatex 1\fi=0 % if pdftex
  \usepackage[T1]{fontenc}
  \usepackage[utf8]{inputenc}
  \usepackage{textcomp} % provides euro and other symbols
\else % if luatex or xelatex
  \usepackage{unicode-math}
  \defaultfontfeatures{Ligatures=TeX,Scale=MatchLowercase}
%   \setmainfont[]{EBGaramond-Regular}
    \setmainfont[Numbers={OldStyle,Proportional}]{EBGaramond-Regular}      % cjns1989 - 20191129 - old style numbers 
\fi
% use upquote if available, for straight quotes in verbatim environments
\IfFileExists{upquote.sty}{\usepackage{upquote}}{}
% use microtype if available
\IfFileExists{microtype.sty}{%
\usepackage[]{microtype}
\UseMicrotypeSet[protrusion]{basicmath} % disable protrusion for tt fonts
}{}
\usepackage{hyperref}
\hypersetup{
            pdftitle={MENDIZÁBAL},
            pdfauthor={Benito Pérez Galdós},
            pdfborder={0 0 0},
            breaklinks=true}
\urlstyle{same}  % don't use monospace font for urls
\usepackage[papersize={4.80 in, 6.40  in},left=.5 in,right=.5 in]{geometry}
\setlength{\emergencystretch}{3em}  % prevent overfull lines
\providecommand{\tightlist}{%
  \setlength{\itemsep}{0pt}\setlength{\parskip}{0pt}}
\setcounter{secnumdepth}{0}

% set default figure placement to htbp
\makeatletter
\def\fps@figure{htbp}
\makeatother

\usepackage{ragged2e}
\usepackage{epigraph}
\renewcommand{\textflush}{flushepinormal}

\usepackage{indentfirst}

\usepackage{fancyhdr}
\pagestyle{fancy}
\fancyhf{}
\fancyhead[R]{\thepage}
\renewcommand{\headrulewidth}{0pt}
\usepackage{quoting}
\usepackage{ragged2e}

\newlength\mylen
\settowidth\mylen{……………….}

\usepackage{stackengine}
\usepackage{graphicx}
\def\asterism{\par\vspace{1em}{\centering\scalebox{.9}{%
  \stackon[-0.6pt]{\bfseries*~*}{\bfseries*}}\par}\vspace{.8em}\par}

 \usepackage{titlesec}
 \titleformat{\chapter}[display]
  {\normalfont\bfseries\filcenter}{}{0pt}{\Large}
 \titleformat{\section}[display]
  {\normalfont\bfseries\filcenter}{}{0pt}{\Large}
 \titleformat{\subsection}[display]
  {\normalfont\bfseries\filcenter}{}{0pt}{\Large}

\setcounter{secnumdepth}{1}
\ifnum 0\ifxetex 1\fi\ifluatex 1\fi=0 % if pdftex
  \usepackage[shorthands=off,main=spanish]{babel}
\else
  % load polyglossia as late as possible as it *could* call bidi if RTL lang (e.g. Hebrew or Arabic)
%   \usepackage{polyglossia}
%   \setmainlanguage[]{spanish}
%   \usepackage[french]{babel} % cjns1989 - 1.43 version of polyglossia on this system does not allow disabling the autospacing feature
\fi

\title{MENDIZÁBAL}
\author{Benito Pérez Galdós}
\date{}

\begin{document}
\maketitle

\hypertarget{i}{%
\chapter{I}\label{i}}

Al anochecer de aquel día, el \emph{no sé cuántos} de Septiembre del año
35 (siglo XIX), llegó puntual al parador de \emph{no sé qué}, calle de
Alcalá, entre la Academia y las Monjas Vallecas, la diligencia, galerón
o quebrantahuesos ordinario de Zaragoza, que traía los viajeros de
Francia por la vía de Olorón y Canfranc, único portillo que dejaban
libre en aquellos tristes días los porteros del Pirineo, \emph{vulgo}
facciosos.

No bien pararon las ruedas del polvoriento armatoste, fue cercado de
gentes diversas: por una parte, familia o amigos de los pasajeros; por
otra, intrusos, ganchos o buscones enviados por fondas y posadas. Con
este contingente y los viajeros que iban bajando perezosos, según les
permitían sus remos entumecidos, se formó al instante un apelmazado y
bullicioso grupo. Produjéronse rumores diferentes: aquí salutaciones
cariñosas; allí el restallido del besuqueo y los palmetazos del
abrazarse; acullá ofertas importunas de pupilajes cómodos y baratos.
Entre tantos viajeros, sólo uno no tenía quien le esperase: nadie se
cuidaba de él ni le decía \emph{por ahí te pudras}, como no fueran los
moscones de las casas de huéspedes. Era el tal un joven de facciones
finas y aristocráticas, ojos garzos, bigotillo nuevo, melena rizosa y
negra, que sería bonita cuando en ella entrara el peine y se limpiara
del polvo del camino. Su talle sería sin duda airoso cuando cambiara el
anticuado y sucio vestidito de mahón por otro limpio, de mejor corte. En
lo más claro del grupo quedose como atontado palomino, contemplando el
bullanguero tropel de gente descuidada y ociosa que por la calle a tales
horas discurría. ¡Pobrecillo! Solo y sin maestro ni amigo a quien
arrimarse, se lanzaba en aquel confuso laberinto; sin duda entraba
gozoso y valiente, con la generosa ansiedad del mozuelo de veinte años a
quien ha quitado el sueño y las ganas de comer, en las aburridas
soledades de la aldea, la visión de la Corte y de sus placeres y
grandezas, tal y como las aprecian desde lejos los que empiezan a vivir,
los que se hallan en pleno retoñar de ideas tempranas, producto fresco
de las primeras lecturas, de las primeras pasiones, de la ambición
primera, que tanto se parece a la tontería.

Embobado, como digo, estaba el hombre, contemplando el ir y venir de
vagos bien vestidos, cuando le hizo volver en sí una voz bronca y
desapacible que en el corro gritaba: «¡D. Fernando Calpena! ¿Quién es
Don Fernando Calpena?»

---No vocee usted tanto, que soy yo---dijo el mancebo, un tanto
asustadico.---¿Qué se le ofrece?

---Véngase conmigo, señor---replicó el otro, como sin ganas de entrar en
explicaciones.---Tengo el encargo de llevar a usted a una casa de
huéspedes.

---¿Encargo?, ¿de quién?\ldots{} ¿Se puede saber?

---Del Sr.~D. Manuel, el segundo jefe de la Superintendencia.

---¿D. Manuel?\ldots{} A fe que no le conozco.

Recordando haber oído ponderar lo que abundan en Madrid los ladrones,
pícaros y toda la caterva de gente perdida y maleante, tuvo Fernandito
algo de miedo, y miró con recelo al que parecía, si no protector,
mensajero de desconocidas influencias tutelares; y en verdad que el
pelaje, la carátula y el vocerrón de aquel sujeto no eran para infundir
tranquilidad. El desconocido distinguiríase entre mil por la pátina de
su cara sudosa, afeitada de ocho días; por los ojos ribeteados de
bermellón; por la boca desmedida y los labios con hemorroides; por los
ojos de carnero moribundo; por la ropa, que habría sido decente en otro
cuerpo y en remotas edades; por el sombrero de copa, que su oficio le
obligaba a usar, y era de catorce modas atrasado. Rasgo final: usaba
bastón de nudos con gruesa cachiporra.

«¿Y el equipaje del señor?\ldots»

---Ya lo han bajado\ldots{} Vea usted aquel baúl largo, forrado de
cabra\ldots{} así, con poco pelo\ldots{} No podremos llevarlo hasta que
no me lo despachen los de la Aduana.

---¡Los de la Aduana!---exclamó con visible desdén el de la
cachiporra.---¡Pues no faltaría más sino que abrieran el cofre del
señor!\ldots{} Traigo bula para que den paso franco a todo.

Y al punto se metió por lo más apretado del grupo, repartiendo codazos a
un lado y otro; llegándose al de la Aduana, le dijo no sé qué
frasecillas enigmáticas, y no fue preciso más para que el equipaje del
Sr.~De Calpena quedase libre y exento de toda impertinencia fiscal. Un
momento después Don Fernando y su acompañante, precedidos de un mozo de
cuerda con el baúl a cuestas, se alejaban del parador calle abajo.

«Estamos a cuatro pasos del domicilio, señor. Esta calle por donde ahora
entramos es la \emph{Angosta de Peligros}\ldots{} Aquella de enfrente es
\emph{Ancha} de lo mismo, a saber: de los peligros. Váyase enterando si,
como parece, es esta la primera vez que viene a los Madriles.»

---Es la primera vez\ldots{} Por más que rebusco en mi memoria---dijo el
D. Fernando caviloso y otra vez inquieto,---no caigo en quién pueda ser
ese D. Manuel que ha dado a usted el encargo de recibirme y alojarme.

---D. Manuel de Azara.

---¿De Azara?\ldots{} Ese apellido me suena, sí, me suena\ldots{}
pero\ldots{} vamos, que no le conozco ni le he visto en mi vida, así
Dios me la conserve. Y usted\ldots{} ¿tendría la bondad de decirme su
gracia?

---Mi gracia, como quien dice, mi nombre, es Filiberto Muñoz. Aunque
nací en Consuegra, soy \emph{orundio} de Extremadura, y\ldots{}

---O me equivoco mucho, o es usted de la policía.

---En ella serví durante los \emph{tres años}; pero en la \emph{ominosa
década}, como decimos por acá, quedé cesante, y tuve que arrimarme a los
teatros y a la compañía de Luna para poder vivir malamente. El 33, no
quería reconocer el Gobierno la tropelía que se había hecho conmigo;
pero fui repuesto, gracias a que me agarré a los faldones de mi paisano
D. Manuel José Quintana, de cuyos padres el mío\ldots{} mi padre quiero
decir\ldots{} era muy amigo\ldots{} o más claro, que le castraba los
cochinos, con perdón de usía\ldots{} Ea, ya entramos en la calle de
Caballero de Gracia, donde está su alojamiento. Por aquí, señor. Es
aquella casa donde está el reverbero\ldots{} dos puertas más allá del
quitamanchas. Ya estamos. El portal es antiguo; pero muy decente, y en
él no está permitido hacer aguas, porque en el principal vive el dueño,
que es un señor consejero, pariente del señor subdelegado, ya
sabe\ldots{} Olózaga.

Subieron al segundo piso y penetraron en la casa, que era de las
llamadas de huéspedes, decentísima, lo mejor del ramo, pues en ella no
se entraba más que por recomendación, y rara vez pasaba de cuatro el
número de los favorecidos. Recibioles afablemente el dueño, que ya
esperaba al señor de Calpena, y le llevó derechamente a la habitación
que preparada para él tenía. Hallose el joven en un gabinete muy lindo,
en aquellos tiempos casi lujoso, con alcoba estucada, buenos
muebles\ldots{} Vamos, que creía ser víctima de un error; que le habían
tomado por otro; que aquel hospedaje y el servicio del polizonte y todo
lo que le ocurría, no era por él ni para él. Pero mientras el error
durara, juzgaba práctico aprovecharse. Adelante, pues, con la aventura:
siguiera el \emph{quid pro quo}, que tiempo habría de que el acaso o la
realidad lo deshicieran.

Mostrole el patrón todas las partes del aposento, diciéndole: «Tengo mi
casa montada a la inglesa, conforme a los últimos adelantos. Vea
usted\ldots{} cordón para tirar de la campanilla; lavabo con su cubo,
jofaina y demás; alfombrita delante de la cama; percha con su cortina
para resguardar del polvo la ropa\ldots{} en fin, progreso, finura. Y
como punto céntrico, no hallará usted nada mejor que esta casa. Aquí
está usted cerca de todo. Dos pasos más arriba, la Red de San Luis, con
tanto comercio. En la calle de atrás, la fonda de Genieys; más abajo el
Carmen Descalzo, donde tiene usted misa a todas horas. En la calle de
Alcalá, que es a dos pasos, las Señoras Calatravas, las Señoras
Vallecas, la Embajada inglesa\ldots{} En fin, cerca tenemos también las
\emph{Niñas de Leganés}\ldots{} la casa de las \emph{Siete chimeneas},
que por mi cuenta son ocho, y cuanto bueno hay en Madrid\ldots{} Para
que nada falte, en esta misma calle tiene usted la casa de baños de
Monier, que es, según dicen, de las mejores de Europa, como que en ella,
por seis reales, puede un cristiano lavarse\ldots{} de cuerpo entero.»

Encantado de su vivienda y de su barrio estaba el buen D. Fernando, y
aunque ignoraba de dónde y de quién le venían tantas dichas, iba muy a
gusto en el machito, y no pensaba más que en arrear en él mientras
durase la ganga. Por de pronto, urgía pagar al mozo; y en cuanto al
desconocido que salió a encontrarle, no parecía hombre que desdeñara una
gratificación si delicadamente se le ofrecía. De ambas cosas habló D.
Fernando a su hospedero, el cual, con aires de gran señor, le contestó
que todo estaba pagado, y que el Sr.~de Calpena no tenía que ocuparse de
nada, como no fuera de pedir por aquella boca cuanto le dictasen su
necesidad y sus antojos.

«Pues, señor---dijo para sí el mancebo, después de dar las
gracias,---sin duda estoy soñando, o me equivoqué de camino y en vez de
ir a Madrid, me he metido en Jauja. Porque esto de que le reciban a uno
desconocidos emisarios del diablo o de las mismísimas hadas, y le saquen
el equipaje sin registrar, y le traigan a este lindo aposento, y no
cobren nada, y desaparezcan por escotillón mozos y servidores cuando uno
echa mano al bolsillo para darles la propina\ldots{} esto, vamos, esto
que a mí me pasa, no le ha pasado a ningún nacido en sus primeros pasos
por una capital grande o chica. Aquí hay algo, y vuelvo a temer que,
tras de tantas venturas, venga una triste y quizás trágica sorpresa.
Mucho ojo, Fernando, y trata de sondear al patrón, que tal vez posea la
clave del acertijo.»

«Siento mucho---dijo en voz alta, sentándose en la butaca y observando a
su patrón de los pies a la cabeza,---que haya usted dejado marchar a ese
hombre sin que yo le dé una gratificación por haberme traído aquí.»

---Déjele usted, que ya, ya se la darán, y más de lo que merece.

---¿Pero quién, por Cristo?\ldots{} ¿Por quién vengo yo aquí? ¿En qué
manos estoy?

---En buenas manos, caballero---afirmó el patrón con sonrisa tan
benévola y franca, que el desconcertado joven no tuvo más remedio que
creerle.

---Ese sujeto, ¿es de la policía?

---Sí, señor.

---¿Y por mandato de quién sale a mi encuentro la policía?

---No sé, señor\ldots{} Yo que usted, francamente, me cuidaría de coger
la fruta que me cae entre las manos, sin meterme en averiguar quién
plantó el árbol que la da tan rica.

Calló D. Fernando, sin dejar de mirar a su aposentador como se mira un
jeroglífico.

«Ese hombre se llama Muñoz\ldots»

---Y por mal nombre \emph{Edipo}, porque fue, según dicen, del
teatro\ldots{}

---Pues, la verdad, me disgusta que se haya ido sin que yo le dé
siquiera las gracias, sin obtener de él una explicación de este
misterio\ldots{} ¿Quién le mandó?\ldots{} ¿Cómo sabía mi llegada, mi
nombre?

---Él lo explicará cuando vuelva, señor\ldots{}

---Al menos, me dirá usted, como dueño de la casa, qué tengo que pagarle
por este cuarto---añadió Calpena impaciente y un tanto
nervioso.---Podría ser que el precio fuese superior a mis recursos, y
tuviera yo que buscar alojamiento más arreglado.

---Si por más arreglado entiende más barato, caballero, no lo encontrará
ni en los cuernos de la luna, que el colmo de la baratura es el no pagar
nada. Quiero decir que\ldots{}

---¿Pero quién, Señor?\ldots{} Esto me vuelve loco\ldots{} ¿Se ríe
usted? O juega conmigo, o aquí hay gato encerrado.

---¡Encerrado\ldots{} aquí! Yo le juro al señor que el único que tenemos
en casa, y se llama \emph{Zumalacárregui}, es un gato de buena crianza,
que no se mete a deshora en las habitaciones de mis huéspedes.

---Ya que no otra cosa---indicó D. Fernando, rindiéndose a la bondad
marrullera del patrón,---dígame usted su gracia, y\ldots{}

---Mi gracia es Mendizábal\ldots{}

Al oír este nombre se le crisparon los nervios al joven forastero, que
se puso en pie, acercándose al dueño de la casa para verle mejor y
examinarle. Era este de espigada estatura, representando cincuenta años,
de rostro agradable, con patillitas, corbatín, el cuerpo enfundado en un
levitón alto de cuello y larguirucho de faldones. Al verle reír, entró
más en cuidado Calpena, y se aumentaron las confusiones que desde su
novelesca entrada en la Villa del Oso embargaban su espíritu.

«Me río porque\ldots{} verá usted---dijo el patrón.---No es que yo me
llame propiamente Mendizábal. Mi apellido es Méndez. Pero como el Sr.~D.
Juan Álvarez y Méndez, el grande hombre que ha venido de las Inglaterras
a meternos en cintura y a salvar al país, se ha variado el nombre,
poniéndose \emph{Mendizábal}, que tan bien suena, yo\ldots»

---Usted, por no ser menos\ldots{} ya.

---Y digo más: bien podría resultar que D. Juan de Dios Álvarez y un
servidor de usted fuéramos parientes, pues Méndez somos los dos: él hijo
de Cádiz, yo, de San Roque, frente a Gibraltar. ¿Quién me asegura que no
seamos ramas del mismo tronco? Porque eso que cuentan de que el
Sr.~Álvarez y Méndez no viene de casta de cristianos viejos, es
calumnia, señor; cosas que inventa la maldad del absolutismo para
rebajar a los patriotas\ldots{} En fin, que como mis compañeros de
oficina ven en mí a un partidario furibundo del señor Ministro nuevo, me
han puesto el remoquete de Mendizábal, y así me dejo llamar, y me
río\ldots{} me río\ldots{}

\hypertarget{ii}{%
\chapter{II}\label{ii}}

---Según eso, es usted empleado.

---Para todo lo que el señor guste mandarme, me tiene de portero en el
Ministerio de Hacienda. Miliciano nacional de artillería en el glorioso
trienio, fui colocado por el señor Feliu. Quedé cesante el 23. Diez años
después me repuso el Sr.~D. Francisco Javier de Burgos, que entró en
Fomento el 21 de Octubre del 33.. En 7 de Febrero del año siguiente pasé
a Hacienda con el Sr.~D. José de Imaz; me conservó en mi puesto el señor
Conde de Toreno, que entro el 15 de Junio, y allí me tiene usted\ldots{}
Pero estoy entreteniendo al señor más de lo regular, sin pensar que se
aproxima la hora de la cena. Antes querrá quitarse el polvo del camino y
lavarse cara y manos. Voy por agua, pues creo que tenemos el jarro
vacío\ldots{} Efectivamente\ldots{} ¡Y tanto que les encargué\ldots!
¡Cayetana!\ldots{} ¡Delfina!

Salió presuroso, llamando a su esposa e hija, y a poco se presentaron
estas con el agua y toallas limpias. Era la patrona regordeta y
vivaracha, bastante más joven que su marido; mala dentadura, pecho
vacuno, que el corsé levantaba a las alturas de la garganta; el habla
gallega, manos de cocinera. La niña, tímida y rubicunda, habría sido muy
bonita si no torciera terriblemente los ojos. Precedíalas el risueño
padre, que, al presentar a la familia, volvió a soltar la vena de su
verbosidad.

El Sr.~D. Fernando traería, según él, buen apetito. Pronto se le
serviría la cena\ldots{} Casa más sosegada no se encontraba en todo
Madrid, y como no admitían sino huéspedes recomendados, nunca tenían más
de cinco o seis, y a la sazón, por ser verano, tan sólo dos, sin contar
al Sr.~D. Fernando, los cuales eran personas de mucho asiento y
formalidad. A la hora de la cena les conocería el nuevo huésped, y
trabaría con uno y otro sujeto relaciones cordiales\ldots{} Dejáronle al
fin para que se lavase, y despojado de su trajecito de mahón, se ocupó
el huésped en sacar del baúl la única ropita decente que traía, y camisa
y corbata, para vestirse con toda la decencia compatible con su escaso
peculio. Durante las operaciones de lavoteo y vestimenta, no cesaba de
pensar en la ventura inesperada y misteriosa con que entraba en Madrid,
y entre otras cosas que habrían revelado su confusión si las pasara del
pensamiento a los labios, se dijo: «Es mucho cuento este. Se empeña uno
en ser clásico, y he aquí que el romanticismo le persigue, le acosa.
Desea uno mantenerse en la regularidad, dentro del círculo de las cosas
previstas y ordenadas, y todo se le vuelve sorpresa, accidentes de poema
o novelón a la moda, enredo, arcano, \emph{qué será}, y manos ocultas de
deidades incógnitas, que yo no creí existiesen más que en ciertos libros
de gusto dudoso\ldots{} Pues, señor, veamos en qué para esto, y Dios
quiera que pare en bien. No las tengo todas conmigo, ni me resuelvo a
entregarme a esta felicidad que me sale al encuentro abriéndome los
brazos, pues suelen los salteadores de caminos disfrazarse de personas
decentes y benéficas para sorprender mejor a los viajeros. Vigilemos,
vivamos alerta\ldots»

Cenando migas excelentes con uvas de albillo, peces del Jarama fritos, y
chuletas a la \emph{papillote}, hizo conocimiento con los dos huéspedes
que la suerte le deparaba por compañeros de vivienda, y en verdad que
tal conocimiento fue un nuevo halago de la escondida divinidad que tan
visiblemente le protegía, porque ambos eran agradabilísimos, instruidos,
graves y de perfecta educación. El uno frisaba en los cincuenta años, y
en las primeras frases del coloquio se declaró manchego y patriota. Su
locuacidad no molestaba; antes bien, instruía deleitando, porque narraba
los sucesos y exponía las opiniones con singular donaire y una
prolijidad pintoresca. Debía de tener muchas y buenas amistades con
personas en aquel tiempo de gran viso, porque al nombrarlas empleaba
casi siempre formas familiares.

Cuando Delfinita le servía las truchas, volviose a ella con viveza,
diciéndole: «No me han enterado ustedes de que hoy estuvo aquí
Salustiano dos veces.»

---¡Ah!, sí\ldots{} no me acordaba\ldots---replicó la niña de la
casa.---¡Y que no se puso poco enojado la segunda vez, porque usted no
estaba!

---¡Si ya le he visto, criatura! Por fin dio conmigo en el Café Nuevo,
donde me había citado mi tocayo Nicomedes para leerme dos artículos de
filosofía, una comedia en verso y un proyecto de Constitución\ldots{}

---Dispénseme---dijo Calpena, que pronto empezó a tomar confianza:---ese
Salustiano, ¿es Olózaga?

---El mismo. Le nombran Gobernador de Madrid\ldots{}

---Subdelegado---apuntó el otro huésped, de quien se hablará
después,---que así se llaman ahora.

---Tanto monta, amigo Hillo\ldots{} La denominación que se adoptará como
definitiva es la de \emph{jefes políticos}. Por de pronto, empleemos la
acepción que más fácilmente comprende el pueblo:
\emph{gobernadores}\ldots{} Pues pretende Salustiano llevarme de
secretario; pero\ldots{} no en mis días. Mientras yo no vea clara la
situación, mientras no vea un Gabinete decidido a marchar adelante,
siempre adelante, enarbolando resueltamente la bandera del progreso, no
me cogen, no me cogen\ldots{} Nicomedes piensa lo mismo\ldots{}

---Oí decir esta tarde en el despacho de los Toros---indicó tímidamente
el segundo huésped,---que sería secretario ese joven, tocayo de usted,
que acaba de citar\ldots{} Pastor.

---Atrasados están de noticias en el despacho de Toros, mi querido
Hillo. Será secretario del Gobierno de Madrid mi amigo Manolo Bretón.

---¿El poeta\ldots{} el autor de Marcela?---preguntó Calpena con vivo
interés.

---El mismo. Y añadiré que a mí me lo debe---afirmó con cierta fatuidad
de buen tono el que llamamos \emph{primer huésped}, y ahora Don
Nicomedes. Conviene declarar, ante todo, que no es Pastor Díaz. El
huésped de la casa de Méndez no ha pasado a la historia, aunque en
verdad lo merecía, por la agudeza de su entendimiento y la variedad de
sus estudios. Menos años contaba entonces el Nicomedes que después
adquirió celebridad como político y publicista: ambos se hallaban
ligados por estrecha y cordial amistad. El más joven hizo carrera
literaria y política; el más viejo se fue a la Habana en tiempo del
general Tacón, y murió de mala manera bajo el mando de Roncali. Apenas
ha dejado rastro de sí, como no sea el descubierto con no poca
diligencia por el que esto refiere; rastro apenas visible, apenas
perceptible en el campo de la historia anónima, es decir, de aquella
historia que podría y debería escribirse sin personajes, sin figuras
célebres, con los solos elementos del protagonista elemental, que es el
macizo y santo pueblo, la raza, el \emph{Fulano} colectivo.

Bueno. Diré algo ahora del segundo huésped, clérigo enjuto y amable, que
entraba siempre en el comedor tarareando, y a veces tocando las
castañuelas con los dedos, lo que no quiere decir que fuera un sacerdote
casquivano, de estos que no saben llevar con decoro el sagrado hábito
que visten. La jovialidad del bonísimo D. Pedro Hillo, natural de Toro,
era enteramente superficial, y a poco que se le tratara, se le veían las
tristezas y el amargo desdén que le andaba por dentro del alma, como una
procesión interminable. Por lo demás, no se ha conocido hombre de
costumbres más puras ni en la clase eclesiástica ni en la civil; hombre
que, si no derramaba el bien a manos llenas, era porque no se lo
permitía su mediano pasar, cercano a la pobreza; incapaz de ofender a
nadie de palabra ni de obra; comedido en su trato; puntual en sus
obligaciones; religioso de verdad, sin aspavientos. No tenía más falta,
si falta es, que gustar locamente de las funciones de toros. Su
principal ciencia, entre las poquitas que atesoraba, era el entender del
arte del toreo y mostrar profundo conocimiento de sus reglas, de su
historia, y poder dar sobre tales materias opiniones que los devotos del
cuerno oían como la palabra divina. Pero dígase en honor de D. Pedro
Hillo que, lejos de la intimidad con otros taurófilos, no alardeaba de
su conocimiento, ni usaba nunca los groseros terminachos que suelen ser
lenguaje propio de esta singular afición. Como se disimula un ridículo
vicio, disimulaba el buen curita su autoridad en materia de quiebros,
pases y estocadas.

Y para que se vea un ejemplo más de las complejidades del humano
espíritu, sépase que a este saber de cosas triviales unía Don Pedro de
otro de más sustancia. Era un apreciable retórico, de la escuela de
Luzán y Hermosilla; había practicado durante más de veinte años el
magisterio del arte de hablar bien en prosa y verso, y orgulloso de
estos conocimientos, trataba de lucirlos siempre que podía.

Se ignora por qué dejó el bueno de Hillo, primero su cátedra del Colegio
Mayor de Zamora, después el cargo de preceptor de los niños del señor
Duque de Peñaranda de Bracamonte. Lo que sí se ha podido averiguar es
que en Septiembre de 1836 pretendía una cátedra de la Universidad
Complutense, y que en aquella fecha llevaba año y medio de inútiles
pasos y gestiones sin obtener más que buenas palabras. Eso sí: ni se
cansaba de pretender, ni los desaires y aplazamientos marchitaban sus
ilusiones, ni le rendía el fatigoso y tristísimo \emph{vuelva usted
mañana}.

Dígase también, para completar la figura, que D. Pedro profesaba o
fingía, en política, un escepticismo inalterable, rara condición en
aquellos tiempos de lucha. Conocimiento y amistad tenía con personas de
una y otra bandera; pero de nada le valían, sin duda por causa de su
timidez, o por la vaguedad de sus opiniones, que tal vez le hacía
sospechoso a tirios y troyanos. Los patriotas le miraban con recelo
creyéndole arrimado al carlismo, y la gente templada le tenía por afecto
a las logias. Por esto decía él, empleando la palabra griega que
significa moraleja: «\emph{Epimicion}: quien navega entre dos aguas, no
llega nunca a una cátedra.»

El primer huésped, D. Nicomedes Iglesias también pretendía; mas no era
fácil traslucir el objeto de sus desatentadas ambiciones. Cosa extraña:
Hillo hablaba poco, y sus propósitos y deseos se traslucían a las
primeras palabras. Por los codos hablaba Iglesias y después de oírle
perorar tres horas con gracia y facundia prodigiosa, nadie sabía lo que
pensaba, ni qué planes o enredos se traía. No disimulaba el radicalismo
de sus ideas, el cual no era obstáculo para que cultivase el trato de
casi todas las notabilidades de aquella turbulenta generación, siendo su
mayor intimidad con los exaltados. Toda la tarde estaba fuera de casa,
menos cuando daba cita en ella a un par de compinches, pasándose las
horas muertas de conciliábulo a puerta cerrada. Después de cenar se
echaba invariablemente a la calle, y no volvía hasta la madrugada;
levantábase a la hora de comer, y al encontrarse en la mesa con su amigo
D. Pedro, bromeaban un rato. El presbítero tenía siempre algo que decir
de las nocturnidades de su compañero; pero sin traspasar nunca los
límites de una discreta confianza inofensiva: «¿Qué hay por la
\emph{casa de Tepa}?\ldots{} Anoche, amigo Nicomedes, debieron ustedes
tratar de ir disolviendo juntitas, para que no se enfade D. Juan de Dios
Álvarez\ldots{} Mucho tuvieron que discutir anoche los del \emph{rito
escocés}, porque entró usted cerca de las cuatro\ldots{} ¿Y qué se sabe
del ínclito Aviraneta? ¿Le sueltan, o le hacen ministro, o le ahorcan?»

Contestaba el otro a estas pullas inocentes con gracia y mesura, sin
soltar prenda, ni clarearse más de lo que le convenía. Desde la primera
cena simpatizó Calpena con sus dos compañeros de casa, y singularmente
con el clérigo Hillo. El agrado que la conversación de este le causaba
aumentó tan rápidamente, que al segundo día eran amigos, y ambos creían
que su trato databa de larga fecha. Verdad que los dos eran clásicos en
lo literario, templados o neutrales en lo político, de pacífico y blando
genio, amantes de la regularidad y del vivir manso, sin emociones;
semejanza que un atento observador habría podido apreciar, no obstante
las diferencias que la edad marcaba en uno y otro. Había, sin embargo,
momentos en que Calpena se expresaba como un viejo, y D. Pedro como un
muchacho.

El segundo día de hospedaje, desayunándose juntos, hablaron de política,
que era en aquel tiempo la usual, la obligada comidilla, lo mismo al
almuerzo que a la cena. «¿Qué le parece a usted, amigo D.
Fernando?---dijo Hillo.---¿Nos cumplirá ese Sr.~Mendizábal todo lo que
nos ha prometido? Porque ya ve usted si ha venido con ínfulas. Que
acabará la guerra carlista en seis meses, y que para entonces no veremos
un faccioso ni buscándolo con candil. Que pondrá término a la anarquía,
cortando el revesino a todas las juntas. Que arreglará la Hacienda, y
pronto rebosarán las arcas del Tesoro. Que hará de la España una nación
tan grande y poderosa como la Inglaterra, y seremos todos felices y nos
atracaremos de libertad y orden, de pan y trabajo, de buenas leyes,
justicia, religión, libertad de imprenta, luces, ciencia, y, en fin, de
todo aquello que ahora no comemos ni hemos comido nunca.»

\hypertarget{iii}{%
\chapter{III}\label{iii}}

---Yo, amigo Hillo, no entiendo este endiablado Madrid, ni puedo darle a
usted una opinión sobre lo que me pregunta. Aún no he tomado tierra.
Ahora vengo de Francia, y allí, puedo asegurarlo, los españoles que he
conocido se hacen lenguas del Sr.~Mendizábal, y ven en él a un hombre
extraordinario, providencial, que ha de regenerar la España.

---¡Viene usted de Francia!---exclamó Hillo picado de curiosidad
ardiente.---Y en Francia ha dejado a sus padres\ldots{}

---Yo no tengo padres. No los he conocido nunca.

---Entonces tendrá usted tíos.

---Tampoco. Yo me crié en Vera, en casa de un sacerdote, que murió hace
tres años. Sus hermanos me mandaron a París, a una casa de comercio. Un
año he vivido en la capital de Francia. Después pasé a Olorón\ldots{}

---Pero es usted español, seguramente.

---Creo que sí\ldots{} digo, sí: español soy.

---Habla usted nuestra lengua con gran corrección.

---Lo mismo hablo el francés.

Más avivada a cada momento la curiosidad del buen clérigo, arreció en
sus preguntas: «Y dígame, si no hay inconveniente en que yo lo sepa:
¿viene usted a estudiar una carrera, o a ocupar una placita en nuestra
administración?»

---Vengo a buscarme una manera de vivir honrada y modesta.

---¿Tiene usted aquí familia, parientes, amigos\ldots?

---No lo sé\ldots{} Creo que no\ldots{} creo que sí.

---Traerá usted cartas de recomendación.

---No, señor\ldots{} Mis tíos (y llamo tíos al hermano y parientes del
cura de Vera, en cuya casa me he criado) enviáronme a Madrid, sin
decirme más que lo que va usted a oír: «Anda, hijo, que aquí no saldrás
nunca de la pobreza oscura, y allá\ldots{} allá puedes encontrar
protecciones donde y cuando menos lo pienses.» Me hicieron el equipaje
con la poca ropa que tenía, me costearon el viaje, diéronme algo para
los primeros días, y aquí me tiene usted\ldots{}

---Esperándolo todo de la suerte, de lo desconocido\ldots{} ¡Ah, señor
de Calpena, usted pitará! No le faltarán contratiempos, afanes; pero no
es usted, me parece, de los que se ahogan en este piélago. Y dígame otra
cosa: ¿ese buen párroco de Vera\ldots?

---Un gran humanista, señor, más versado en los clásicos latinos y
griegos que en Teología y Cánones.

---Bien se le conoce a usted, en su manera de expresarse, la sabia mano
que le ha pulimentado.

---Sabía mucho mi padrino---dijo D. Fernando con tristeza;---y aunque él
se esforzó en darme todo su saber, yo no he tomado sino parte mínima.

---¿Modestia tenemos? Pues a mí me da en la nariz, Sr.~D. Fernandito,
que usted ha de ser un grande hombre. Este tarambana de Nicomedes me
aseguraba ayer que el porvenir será de los románticos, así en literatura
como en política. Yo sostengo lo contrario. La sociedad se va hartando
de contorsiones y de hipérboles, y el clasicismo, la corrección, la
serenidad, la devoción de las buenas reglas, han de gobernar el mundo.
¿No cree usted lo mismo?

D. Fernando, profundamente abstraído, fijaba sus ojos en el ya vacío
pocillo de chocolate.

«Yo no puedo tener opinión, no acierto aún a formar juicio de
nada---murmuré al fin:---soy un chiquillo.»

---Pues lo dicho\ldots{} No sé por qué me figuro que entrará usted en
esta diabólica villa con pie derecho. En todas las cosas y casos de la
vida\ldots{} esto es observación mía, que no me falla\ldots{} los
primeros pasos dan la norma de la suerte total.

---Pues si es así, amigo Hillo---dijo Calpena, revelando en su agraciado
rostro más confusión que alegría,---yo he de ser el niño mimado de la
fortuna, porque en mis primeros pasos en Madrid no piso más que flores.

---Bien, hombre, bien: hay hombres predestinados a la dicha, como los
hay al sufrimiento, y de estos, alguno conozco yo, sí, señor, y más de
lo que quisiera\ldots{} Y puedo asegurarle que no siento envidia de
usted, siendo, como soy, desgraciado \emph{a nativitate.} Créame: el
suelo que yo piso es todo abrojos y guijarros cortantes\ldots{} Pero
ando\ldots{} ando siempre, y adelante. Lo repito: no soy envidioso, y
cuando veo a un hombre con suerte, me alegro, le doy mis plácemes, y
digo: «Bendito sea Dios que, por hacer de todo, también hace seres
felices.»

---No estoy yo seguro de serlo, ni me fío de estas venturas, que bien
podrían ser engañosas, traicioneras.

---No digo que no\ldots{} Pero cuando viene la dicha, hay que tomarla
sin remilgos. La Fortuna, deidad caprichuda, descaradota, se muestra más
liberal con los que no se asustan de sus favores. Los modestos y
encogiditos no le entran por el ojo derecho. Sea usted arrogante,
acometedor; confíe en sí mismo y en su estrella; láncese sin miedo,
\emph{arrancando}, a toda clase de empresas, ya políticas, ya
literarias, ya mercantiles, que de fijo en todas alcanzará la meta.
Ejemplos, aunque no muchos, tiene usted aquí de hombres privilegiados,
que nacieron en la mayor humildad, y luego mansamente, sin hacer nada
por sí, se ven levantados del polvo, y conducidos por manos de ángeles a
los cielos de la prosperidad y de la gloria. Vea usted a este señor
Mendizábal, que se nos ha entrado por las puertas de España. Le
encargaron a Inglaterra para Ministro de Hacienda, como se encargan los
niños a París, y por llegar, con la sola fuerza de su desahogo, que se
impone a todo el mundo, se ha calzado la Presidencia del Consejo y
cuatro Ministerios. ¿Y quién es Mendizábal? Un hombre sin estudios, que
no aprendió más que a leer y escribir, y algo de cuentas. ¿Pues qué es
esto más que suerte? Y los afortunados ¿qué son sino hombres que se
pasan el mundo por debajo de la pata, y han tirado la modestia y los
miramientos, como se tira la careta de trapo que molesta y acalora el
rostro?

---No estamos conformes---dijo D. Fernando, más comedido en sus pocos
años que el viejo Hillo,---en esa manera de apreciar las causas del
éxito en la vida pública. Además, no admito que el Sr.~Mendizábal sea
hombre tan ignorante, ni que carezca de autoridad para desempeñar uno,
dos o media docena de Ministerios. Cierto que no sabe latín; pero es muy
práctico en asuntos mercantiles. Dígame usted, con la mano puesta en el
corazón, si cree que para gobernar a los pueblos es indispensable tratar
de tú a Horacio y Virgilio.

---¡Qué sé yo!\ldots{} Una pasadita de Cicerón no les viene mal a los
señores que andan en la política. Pero, en fin, concedo\ldots{}

---Preveo el argumento que usted va a emplear ahora mismo, y me anticipo
a refutarlo.

---Bien, hombre, bien---dijo gozoso D. Pedro, sintiéndose maestro de
Humanidades.---Ha empleado usted con verdadera elegancia una forma de
raciocinio que los retóricos llamamos \emph{prolepsis}\ldots{} Eso es:
anticiparse a la objeción, prevenir los argumentos del contrario,
refutarlos antes que los emita\ldots{}

---Justamente; y usted ahora, con maestría indudable, ha empleado la
\emph{expolición} o \emph{amplificación}\ldots{}

---Que también llamamos \emph{conmoración}\ldots{} ¿no es eso?

---Y que cuando degenera en abuso se denomina \emph{tautología} y
\emph{perisología}\ldots{} Volviendo a mi prolepsis, prosigo. Usted me
dirá que, si no es necesario saber latín para regir a las naciones,
tampoco estriba la conciencia de gobierno en el arte o manejo de los
negocios mercantiles; es decir, que si mal nos gobiernan los humanistas,
no lo harán mejor los comerciantes.

---Efectivamente.

---A eso respondo que el Sr.~Mendizábal no es un simple mercader, de
esos que compran y venden géneros: es, si se me permite decirlo así,
comerciante político, y no me busque usted en este concepto la
\emph{anfibología}, que no la hay. Comerciante político quiere decir: el
que entiende de manejar el crédito de los países y distribuir su
Hacienda, de imponer y recaudar tributos\ldots{}

---El Sr.~Mendizábal era el año 23 un traficante gaditano; menos aún,
dependiente en la casa del Sr.~Bertrán de Lis, y se metió a contratista
de las provisiones del Ejército, con lo cual hizo su pacotilla en pocos
años.

---Sus opiniones avanzadas y la viveza de su genio, le arrastraron a la
empresa de abastecer al Ejército y Marina en condiciones tales, que su
servicio fue, más que negocio, un caso de abnegación y patriotismo.
Todavía no se han liquidado aquellas cuentas, y las ganancias de D. Juan
de Dios, si las tuvo, están aún en poder de la nación.

---Porque usted lo dice lo creo\ldots{} Persona de mi mayor confianza me
ha contado a mí que Mendizábal, allá por el año 20, era en Cádiz un
muchachón alborotado, bullanguero, de una intrepidez loca para las
aventuras políticas. Él y otros tales no hacían más que conspirar en
logias y cuarteles para que volviese la Constitución del 12, y destronar
al Rey o convertirlo en un monigote.

---Es verdad.

---Y que trabajó por la bandera que defendían Riego, Arco, Agüero,
Quiroga\ldots{}

---También es cierto. Todas aquellas trapisondas salían de la Masonería,
que ahora es una vieja pintada, y entonces era una mocetona llena de
vida y seducciones, con las cuales enloquecía a la juventud.

---No me disgusta la imagen, señor mío. Adelante.

---En Cádiz existía lo que llamaban el \emph{Soberano Capítulo} y el
\emph{Sublime Taller}, y qué sé yo qué. De estos talleres y capítulos
salían las conspiraciones para sublevar el Ejército y derrocar la
tiranía; de allí las trifulcas, las asonadas, los ríos de sangre\ldots{}
Mendizábal era masón, que en aquel tiempo era lo mismo que decir
\emph{político}. Si quiere usted más noticias, pídaselas a D. Arturo
Alcalá Galiano, que anduvo con él en aquellos trotes; al Sr.~Istúriz, a
D. Vicente Bertrán de Lis\ldots{}

---De donde se deduce, amigo Calpena---dijo el clérigo suspirando
fuerte,---que el que pretenda en estos tiempos ser algo o conseguir
alguna ventaja, aunque esta le corresponda de justicia, y lo intente sin
agarrarse previamente a los faldones o a las faldas de esa gran púa de
la Masonería, es un simple o un loco.

---No diré yo tanto. Las cosas son como son.

---Tenga usted presente que hay logias liberales y logias absolutistas.
Las primeras conspiran; las segundas también. Unas y otras introducen
individuos suyos en la contraria, fingiéndose amigos, para sorprender
secretos.

---Sí, sí; y se pelean en las tinieblas de los ritos nefandos. De las
unas salen los ejércitos sediciosos, que todo lo destruyen y profanan;
de las otras los tribunales sanguinarios que levantan la horca. Así vive
España\ldots{} hoy te fusilo, mañana te ahorco.

---Y vea usted. Si el 24 hubiera sufrido D. Juan de Dios la suerte de su
compinche Riego, hoy no tendríamos la dicha de que ese señor nos
arreglara la Hacienda y nos hiciera juiciosos y ricos.

---Porque escapó a Inglaterra.

---Le llamaba la banca más que la política.

---Se estableció en un país grande y libre, donde forzosamente había de
aprender muchas cosas sólo con tener ojos y ver, sólo con tener oídos y
oír.

---Sí, porque en los libros me parece que poco aprende su ídolo de
usted. Le llamo así porque veo, amigo Calpena, que es usted de los
devotos furibundos del \emph{hombre nuevo}, y que conoce su vida y
milagros, entendiendo por milagro lo que dicen ha hecho en Portugal.

---Algo sé del Sr.~Mendizábal\ldots{} Más de lo que usted piensa.

---¿Andan por el extranjero biografías del grande hombre?

---No he leído ninguna.

---¿Pues quién se lo ha contado?

---Él mismo.

---¡Le conoce usted\ldots{} le trata!

Al ver en el rostro de Calpena la sonrisa plácida y el movimiento
afirmativo con que a su pregunta respondía, Hillo se quedó suspenso de
estupor, de admiración\ldots{} No daba crédito a tan inaudito caso de
precocidad. ¡Tan joven, y haber tratado a Mendizábal, charlar con él,
quizás poseer su confianza! Desde aquel momento vio el clérigo en su
amiguito un ser extraordinario, misterioso. Aumentaban su fascinación la
procedencia extranjera del joven; el no saberse quién era; la atención y
exquisitos cuidados que le prodigaban los patrones, recatando
sigilosamente el nombre de las personas que habían recomendado al nuevo
huésped; la educación exquisita de este; su aire, belleza y modales
aristocráticos\ldots{} y, sobre todo, haber tratado a Mendizábal, y oír
de él mismo la narración de episodios históricos y lances personales. D.
Pedro se levantó de su asiento impulsado de la sorpresa, que como un
resorte le movía, y dio pasos desordenados, repitiendo: «¡Le conoce, le
ha tratado!\ldots{} Dígame, cuénteme: no deje que me abrase la
curiosidad.»

\hypertarget{iv}{%
\chapter{IV}\label{iv}}

---Allá voy---dijo Calpena indicando a su amigo que se
sentara.---Paréceme haber contado a usted que los hermanos de mi padrino
me mandaron a París a instruirme en el comercio y la banca. Empecé a
trabajar, digo, a aprender, en la casa de comisión de Reischoffen y
Bloss, alsacianos, donde sólo estuve tres meses, pasando después a la
célebre casa de banca de Ardoin, que opera por millones de millones, y
hace empréstitos a las naciones apuradas, negociando con los Estados y
con los Reyes, con los Gobiernos y hasta con las revoluciones. En fin,
esto es largo de contar. Allí estaba yo muy bien. Llevaba toda la
correspondencia de la América española; me daban regular sueldo, y el
principal me distinguía y me trataba con mucho miramiento. Un día de
Febrero vimos entrar a un señor alto y bien parecido, de ojos negros,
cabello rizado, patillas cortas, muy elegante y pulcro. Al punto corrió
la voz entre los dependientes: «Es Mendizábal, el gran Mendizábal, el
restaurador de la Monarquía legítima en Portugal\ldots» Entró en el
despacho del Barón, nuestro jefe, y a la media hora este me
llamó\ldots{}

---Para presentarle al Sr.~D. Juan de Dios.

---No, señor; para mandarme que le acompañara por las calles de París,
que yo conocía perfectamente, y el Sr.~Mendizábal no. Tenía que ir a la
casa Erlanger, \emph{Rue Drouot}, muy cerca de la nuestra,
\emph{Chaussée d'Antin}. Cojo mi sombrero, y me pongo a la disposición
del hombre grande, en cuya compañía salí muy orgulloso. Por la calle me
hizo mil preguntas: quién era yo, cómo se llamaban mis padres, cuánto
tiempo llevaba de residencia en París y de aprendizaje en casa de
Ardoin. Yo le contesté como pude, y al llegar a las oficinas de Erlanger
me mandó esperar para que le condujese a otra parte.

---Nada, que le cayó usted en gracia---dijo Hillo restregándose las
manos.---Así se empieza, así.

---Al salir de la visita me preguntó si sabía yo cuál era la mejor casa
de París en guantes y perfumería, y le indiqué Damiani, en el bulevar
Saint-Denis. Tomó el hombre un coche de alquiler, que allí llaman
\emph{fiacres}, y fuimos de compras. Debo decirle a usted que es algo
presumido, y que gusta de acicalarse y lucir su buena figura. De la
guantería fuimos a comprar un maletín de mano para viaje, con muchos
compartimientos y algún secreto para papeles reservados. Compró también
un calzador, tirantes y algunas otras baratijas que no recuerdo. Dejome
en mi escritorio, y él se fue a su hotel, en la \emph{Rue de l'Arcade},
mostrándose en la despedida tan fino y al propio tiempo tan llano, que
yo estaba encantado. Díjome que, siempre que no le convidasen, comería
en el Palais Royal, en casa de Very, y se dignó invitarme, excusándome
yo todo turbado y confuso.

---Esto se llama caer de pie, amigo mío, o nacer en Jueves Santo. Siga
usted, que me parece que aún falta algo.

---Verá usted. A los dos días mandó un recado a mi principal, pidiéndole
un buen amanuense español que escribiese corrido, con buena letra y
mejor criterio. El Barón me eligió a mí, y aquí me tiene usted,
encerrado con el Sr.~Mendizábal en una cómoda estancia del hotel
\emph{Meurice}, los dos frente a frente, con una mesa por medio, él
dictando y yo escribiendo. Hombre más incansable no he visto en mi vida.
Cinco horas me tuvo con la pluma en la mano. Dictó una larguísima carta
a Martínez de la Rosa, otra al Conde de Toreno, y dos o tres a personas
para mí desconocidas. Él estaba en bata, una bata elegantísima, y
zapatillas de terciopelo, con las que lucía su pie pequeño, que parece
de mujer. Casi era preciso escribir taquigrafía para poder seguirle.
Expresaba su pensamiento con rapidez; rectificaba pocas veces; no se
paraba en el estilo; iba derecho al asunto y a la idea, sin cuidarse de
la forma. Mandome volver al día siguiente, y me dictó tres o cuatro
decretos, uno de ellos suprimiendo las órdenes religiosas y haciendo
tabla rasa de todos los frailes, monjas, clérigos y beatas que hay en
estos reinos, estableciendo la reversión de todos los bienes al Estado
para venderlos\ldots{} y ¡qué sé yo!

---¡María Santísima! Pero eso sería broma.

---¿Broma? Ya verá usted las que gasta ese sujeto. No habíamos concluido
aquella degollina de frailes y la repartición de sus riquezas, cuando
entró un señor inglés, que debía de ser diplomático, pariente, sobrino,
hijo quizás del embajador en Madrid, que no sé cómo se llama.

---\emph{Mister} o \emph{sir} Jorge Williers. Adelante.

---Y hablaron en inglés, y no entendí una palabra\ldots{} Bueno: pues en
esto son anunciados tres españoles, y D. Juan les manda pasar. ¡Ay, qué
alegría, qué abrazos, qué maravillas, hablando todos a un tiempo!
Evocaban recuerdos de la juventud, alababan lo pasado, denigraban lo
presente con saña y cuchufletas\ldots{} La conversación fue continuada
en castellano, después de hacer Mendizábal con gran ceremonia la
presentación del inglés a los españoles, y viceversa. Pregunté al Sr.~D.
Juan si debía retirarme, y me mandó que me quedara, lo que me supo muy
bien. ¡Qué gusto estar mano a mano con aquellos señorones, calladito,
oyendo todo lo que decían, que era sabroso, picante y muy instructivo,
pues yo poco o nada sabía de España! Mandó D. Juan al mozo que sirviese
vino de Porto, y con esto las lenguas se soltaron aún más de lo que
estaban.

---Recordará usted los nombres de esos tres españoles, que de fijo
hablarían pestes de su patria.

---Los nombres no los recuerdo; las caras, sí: de seguro son personajes
de acá, y puede que alguno esté hoy en candelero. El uno puso de vuelta
y media a ese Martínez de la Rosa; el otro no dejó hueso sano al Conde
de Toreno, que entonces era Ministro, y el tercero le hincó el diente
venenoso a la Reina Cristina y a su marido D. Fernando Muñoz.

---¡Lástima que usted no se fijara en los nombres!

---Continúo. Pues hablando, hablando de lo revuelto que está todo, de lo
mal que gobiernan los que gobiernan, de las cosas gordas que se
preparan, la conversación recayó en los asuntos de Portugal, y uno de
ellos dijo que en Lisboa había salido un folleto poniendo de oro y azul
a Mendizábal, y negando que tuviera arte ni parte en la restauración de
Doña María de la Gloria. Armose entonces gran tremolina. D. Juan Álvarez
daba golpes en el brazo del sillón, acusando de envidiosos y
calumniadores a algunos españoles residentes en Portugal; indignose el
inglés, echando venablos en su lengua, y los otros atribuían todo a
intrigas de los \emph{moderados} (no sé qué gente es esta que aquí
llaman \emph{moderada}), por arrojar lodo a la figura del grande hombre
que se indicaba ya como el único que podía enderezar al país. No sé cuál
de ellos manifestó no estar al corriente de lo de Portugal, por haber
vivido fuera de la península durante los años de aquellas
tremolinas\ldots{} (paréceme que el tal es militar y de los que aquí
llaman \emph{ayacuchos}), y entonces D. Juan Álvarez, a instancias de
todos, refirió puntualmente las grandes empresas a que prestó su
auxilio.

---Y se despacharía a su gusto, abultando los peligros, y presentándose
como enviado de la Providencia divina.

---Sólo puedo asegurarle a usted que en lo que relató se ve la verdad,
así como una energía pasmosa, fecundidad de arbitrios, recursos
ingeniosos, entusiasmo para encender más la voluntad, maña para suplir a
la fuerza. Lo que sí me pareció notar es que el buen señor se regodea
contando sus empresas: gusta de hablar de sí mismo y de hacer ver que
sin él no se hubiera hecho nada, lo que en muchos casos parecía verdad.

---Psh\ldots{} todo se redujo a proporcionar a D. Pedro un
empréstito\ldots{} Sin dinero no se hacen revoluciones. Mendizábal, por
su metimiento en las casas mercantiles de Londres, fácilmente levantaba
fondos para quitar y poner reyes. Si para echar a los reyes se necesita
dinero, el volver a traerlos cuesta mucho más. No anda sin unto el carro
de las restauraciones.

---Perdone usted. Mendizábal hizo bastante más que proporcionar a D.
Pedro los cuartejos que necesitaba. Ya comprende usted que mientras el
grande hombre refería sus hazañas, yo ni le quitaba ojo ni perdía
sílaba. Todo lo oí, y se me ha quedado bien presente\ldots{} Hizo
verdaderos prodigios, y se mostró gran financiero, gran político, y
hasta gran militar, con unas facultades de organización que ya las
quisieran más de cuatro\ldots{} D. Pedro y su hija se habían refugiado
en las islas Terceras, y allí pasaban su triste vida mirando al Cielo,
esperando su salvación de la Providencia. Pero esta no les hacía maldito
caso, y los ingleses, a quienes el buen Emperador brasileño pedía
recursos, no soltaban ni un chelín. En una de sus excursiones a Londres,
el aburrido D. Pedro y Mendizábal se conocieron. Don Juan le dio
alientos; le indujo a perseverar en su empresa, minando la tierra para
procurarse hombres y pecunia, ambas cosas necesarias para conquistar
reinos, y empezó por facilitarle un empréstito de la casa Ardoin, mi
casa, señor Hillo, la casa donde fui triste aprendiz con ciento
cincuenta francos de sueldo al mes\ldots{} Cien mil libras esterlinas
entraron en el bolsillo de D. Pedro, y con ellas renació la esperanza de
sentar en el Trono a la niña. El hombre se metió de hoz y de coz en la
causa portuguesa, y no habría hecho más si Doña María de la Gloria fuera
su propia hija.

---Bien, bien: así han de ser los hombres.

---En un santiamén compró dos fragatas por cuenta de la Regencia, que
tal era el Gobierno constituido por D. Pedro en la capital de las
Terceras. Advierta usted que en estas compras empleaba sus recursos, sin
más garantía que una palabra del Emperador. Adquiridos los barcos,
agenció en la City más dinero, más, y en seguida, a buscar hombres,
soldados. Mientras en las Terceras se organizaban unos seis mil, en
Plymouth, puerto de Inglaterra, se alistaban más. Mendizábal, que en
todos estos asuntos ponía siempre una vehemencia y un ardor increíbles,
y así lo declara él mismo, no tenía sosiego\ldots{} Creo yo que las
empresas políticas le seducen, le enloquecen; pone en ellas toda su alma
y una actividad febril\ldots{} El hombre se multiplicaba. Sus propios
asuntos perdían para él todo interés. No vivía más que para la Monarquía
liberal portuguesa. Él mismo lo dice: «Cuando se le enciende el
patriotismo no vive, no desmaya hasta conseguir lo que se propone.» Cien
vidas propias daría él por exterminar a los sectarios del usurpador
absolutista D. Miguel, que es allí lo mismo que aquí nuestro D. Carlos
María Isidro\ldots{} No contento con los alistamientos que había hecho
en Inglaterra con ayuda del Duque de Palmela, se planta en Bélgica, y en
cuatro días, auxiliado por su amigo el general Van Halen, busca y
encuentra, organiza y equipa un regimiento de mil flamencos con sus
jefes y todo\ldots{} En Ostende les embarcaron en un buque de vapor
fletado en Londres, y reunidos en Plymouth con los ingleses y
portugueses, zarpó la expedición contra Oporto, mandada por el mismo D.
Pedro. Dominaban en Oporto los liberales, por lo que no le fue difícil
al padre de Doña María la ocupación de aquella capital. Pero el D.
Miguel acudió con mucha tropa, puso cerco a la plaza, y si bien no pudo
entrar en ella, tampoco los \emph{mariistas} podían salir. Allí hubiera
sucumbido D. Pedro, si Mendizábal, desde Londres, no le animara a la
resistencia ofreciéndole nuevos auxilios. ¿Qué hizo el hombre? Pues
buscar más dinero; reunir más soldados; formar al propio tiempo una
escuadra, cuyo mando se ofreció al célebre almirante inglés Napier.
Escuadra y segundo ejército debían operar en los Algarbes, para sublevar
en pro de la Reina a las poblaciones del Sur, y atacar por retaguardia
el ejército miguelista. Todo se hizo tal y como lo había dispuesto D.
Juan\ldots{} La segunda expedición se dirige a Oporto, donde refuerza a
los combatientes asediados por D. Miguel; después parten dos mil hombres
a los Algarbes, desembarcando felizmente. Allí se pasan a los liberales
algunas tropas del absolutismo: entre todas invaden el Alentejo. La
escuadra mandada por Napier desbarata la miguelista en el Cabo de San
Vicente; D. Pedro sale de Oporto y bate a D. Miguel. Replegándose a
Lisboa, recibe éste otro achuchón tremendo de las tropas liberales, y ya
tenemos al Emperador entrando triunfante en su capital, a la niña Doña
María de Braganza en el Trono, y al D. Miguel escapando para el
extranjero como alma que lleva el diablo.

---Y hecho todo eso, que si es como usted lo cuenta, no dudo en
calificarlo de maravilloso, el D. Juan Álvarez se volvió a su escritorio
de Londres tan fresco, a contar millones, calcular empréstitos, extender
letras de cambio, mirando dónde salta otra reina que socorrer, y otro
usurpador malsín a quien poner en la puerta.

---Que no faltan, como usted ve.

---Pero Portugal es chico: puedo compararle a un juguete, para estas
cosas de revoluciones y quita y pon de tronos. Ahora veremos cómo se las
arregla aquí el gaditano; aquí, donde salimos de una zaragata para
entrar en otra, donde nos peleamos por los derechos a la Corona, por las
Juntas, por la Milicia Urbana, por una letra de más o de menos en la
Constitución, y por lo que dicen o dejaron de decir Juan y Manuela.
Vamos a ver a los hombres guapos; a los salvadores de sociedades; a los
que sacan el dinero de debajo de las piedras para equipar soldados; a
los genios, como ahora se dice; a los que calman las olas
revolucionarias con el \emph{quos ego}\ldots{} del amigo Neptuno.

---Adelante: va muy bien. Está usted empleando una forma de ironía muy
bella. Es lo que llamamos \emph{cleuasmo}.

---Dispense usted. Esta forma irónica se llama \emph{carienteísmo}.
Consiste, y bien lo recordará usted; consiste\ldots{}

---Sea lo que fuere, amigo Hillo, mi parecer es que Mendizábal no ha
venido aquí por ambición, sino por patriotismo. Oí contar que se hallaba
muy tranquilo en Londres cuando recibió el nombramiento de Ministro de
Hacienda, que le dejó estupefacto.

---Y estupefacto se ha venido aquí por Portugal; y en cuanto llegó a
Badajoz, empezó a largar decretos\ldots{} Bueno: le concedo a usted que
esto sea patriotismo; pero es un patriotismo\ldots{} romántico, y lo
romántico sepa usted que a mí no me gusta. En literatura me apesta, y a
ese francés que llaman Víctor Hugo le mandaría yo cortar el pescuezo: en
política tengo por más funesto aún el romanticismo.

---Puede que esté usted en lo cierto; pero el Sr.~Mendizábal es ante
todo hacendista, y en esto no creo yo que quepan romanticismos. Los
números ¡ay!, los números, amigo mío, son clásicos.

---Allá lo veremos; y pues ya tenemos al hombre con las manos en la
masa, pronto hemos de saber si yo me equivoco o se equivoca usted.

---Yo no profetizo: yo espero, y\ldots{}

---¿Cree usted firmemente que D. Juan Álvarez enderezará esta
desquiciada nación?

---No lo aseguro; pero confío en que lo hará.

---Pues yo no.

---¿En qué se funda?

---No dudo que le sobren buena intención, voluntad firme, actividad,
talento; pero\ldots{}

---¿Pero qué?

---Que con sus buenas cualidades incurrirá en el defecto de todos los
ilustres señores que nos vienen gobernando de mucho tiempo acá. Talento
no les falta, buena voluntad tampoco. Y fracasan, no obstante, y
continuarán fracasando unos tras otros. Es cuestión de fatalidad en esta
maldita raza. Se anulan, se estrellan, no por lo que hacen, sino por lo
que dejan de hacer. En fin, amiguito, nuestros mandarines se parecen a
los toreros medianos: ¿sabe usted en qué? Pues en que no
\emph{rematan}\ldots{}

---¿Qué significa eso?

---No se ría usted del toreo, arte que me precio de conocer, aunque no
prácticamente. Y sepa usted, niño ilustrado, que hay reglas comunes a
todas las artes\ldots{} De mi conocimiento saco la afirmación de que
nuestros ministriles no \emph{rematan la suerte}.

---¿Y cree usted que Mendizábal\ldots?

---Hará lo que todos. Empezará con mucho coraje, y un trasteo de primer
orden\ldots{} pero se quedará a media suerte. Usted lo ha de ver\ldots{}
Que no remata, hombre, que no remata\ldots{} Y créame usted a mí:
mientras no venga uno que remate, no hemos adelantado nada.

\hypertarget{v}{%
\chapter{V}\label{v}}

Alejose hacia su cuarto, accionando festivamente, y en dirección al suyo
iba también Calpena, cuando le detuvo el patrón señor Méndez, y le dijo
entre risueño y respetuoso:

«Ahí tiene usted el sastre.»

---¿Qué sastre?

---Pues el cortador mayor del Sr.~Utrilla, que viene a tomarle medida.
Le mandé pasar a la sala, donde espera hace un cuarto de hora.

---Ese señor se equivoca. Yo no he llamado a ningún sastre.

---Aunque no le haya usted llamado, él viene, y cuando viene, él sabrá
por qué. Déjese tomar medida, y que le hagan cuanta ropita necesite para
ponerse bien guapo.

---¿Pero está usted loco?\ldots{} ¿No hay más que encargar ropa? Y
luego\ldots{} Sr. Méndez\ldots{} luego vienen las cuentas, ¿y qué
hacemos? ¿Soy acaso un Sr. Mendizábal, que con cuatro rasgos de pluma
fabrica millones?

---Las cuentas no son cuenta de usted, sino de quien las pague. Entre el
señor en su cuarto, y escoja las telas, y déjese que le midan el cuerpo
a lo largo y a lo ancho\ldots{}

---Que pase ese hombre---dijo Calpena prestándose a todo, con la
esperanza de salir de la confusión en que, desde su venturosa llegada a
Madrid, vivía.

En presencia del oficial, hombre finísimo, colorado y regordete, que iba
cargado de muestras de diferentes paños, D. Fernando no pudo resistir a
la fascinación que ejercía sobre él, joven y gallardo, la idea de
vestirse elegantemente. Ante todo quiso saber cómo y por qué los
afamados sastres acudían en busca de parroquia sin que nadie les
llamase; pero sus interrogaciones prolijas y capciosas no lograron
aclarar el enigma. «Mi principal, el señor Utrilla---le dijo aquel
relamido sujeto,---me ha mandado acá con muestras y encargo de tomar a
usted medida para diferentes piezas. Hubiera venido él en persona con
mucho gusto; pero está malo de un pie, y hoy no puede salir de casa. De
quién ha recibido las órdenes para estas hechuras, yo no lo sé, señor
mío, ni es cosa que me corresponde averiguar.»

---Pues yo---afirmó Calpena,---no me dejo medir el cuerpo mientras no
sepa\ldots{} ¿Será tal vez alguna broma impertinente?

---Eso, de ningún modo\ldots{} Utrilla no se presta a tales
bromas\ldots{} Crea usted que, cuando me ha mandado aquí, es porque ha
recibido órdenes de personas que saben el cómo y por qué de lo que
encargan. Con que\ldots{} tomemos esos puntos, y no piense usted en nada
más que en vestirse como le corresponde.

---Accedo, sí, señor---replicó D. Fernando en el tono de quien se presta
a seguir un bromazo de buen género, y seducido además por la idea de ver
realizada su ilusión juvenil de vestir buena ropa.---¿Sabe usted el
cuento del perrito y del trasquilador?

---Sí, señor---dijo el otro, ayudándole a quitarse levita y
chaleco.---Es un cuento viejísimo\ldots{}

---Pues ahora mida usted todo lo que quiera, y hágame todas las prendas
de vestir que haya dispuesto\ldots{} el amo del perrito.

---Me han dicho que dos levitas, fraques, un traje de mañana\ldots{}
cuatro pares de pantalones variados.

---Ande usted, maestro\ldots{} Y si quiere dejarle borlita en el rabo,
déjesela usted.

---La ropa más precisa para un \emph{joven} introducido en sociedad.
¿Qué menos? ¡Ah!, me olvidaba. También le haremos capa de sedán
finísimo, con forros de piel de chinchilla.

---Me parece muy bien\ldots{} ¿Y las levitas, cómo han de ser?

---El Sr.~de Utrilla acaba de llegar de Londres\ldots{} Precisamente al
bajar de la diligencia se estropeó el pie. Pues ha traído las últimas
novedades que se han puesto al uso en aquella capital. Las levitas son
ahora cortas y de poco vuelo en los faldones; pero siguen muy
entalladas, marcando bien la cintura. Las que ha traído el
Sr.~Mendizábal, y que tanto llaman la atención, son ya antiguas, y en
Londres no las usan más que los \emph{lores}, que es como si dijéramos
los señores próceres protestantes, que tienen asiento en lo que llaman
Parlamento inglés, o sea en las Cortes liberales de allá.

---Hombre, bien\ldots{} ¿Con que entalladas y de faldón corto?

---Menos largo que el año pasado---dijo el sastre, tomando y anotando
las medidas con singular presteza.---Los cuellos son ahora más largos, y
bien caídos sobre los hombros; los botones grandes\ldots{} Haremos una
de las levitas, si a usted le parece, con cordones a la húngara\ldots{}

---Perfectamente. Despáchese usted a su gusto\ldots{} ¿Y los paños?

---Fíjese usted en este color verde obscuro, que es la gran novedad que
ha traído Utrilla. Se llama \emph{Lord Grey}, y es el gran \emph{furor}
en Londres.

---Pues hagamos \emph{furor} aquí\ldots{} Pero las dos levitas no serán
iguales.

---Haremos azul gendarme, \emph{Conde Orsay}, la de cordones. ¿Qué le
parece?

---Acertadísimo\ldots{} ¿Y cuándo podré estrenar?

---Lo activaremos todo lo posible\ldots{} Tenemos mucho trabajo, y
velamos para servir a tantísima parroquia.

---Pero no me dejarán ustedes para lo último, como parroquiano
pobre\ldots{}

---Será usted de los primeros\ldots{} Y que tiene un talle de primer
orden, y una forma de cuerpo que no hay más que pedir. Le caerá a usted
la ropa que ni pintada.

---Y en fraques, ¿qué se lleva?

---Los fraques son ahora sin cartera; faldones nada de anchos, y los
cuellos de la misma forma que las levitas. El Sr.~Mendizábal los trae
negros, verdaderamente \emph{fachonables} por el corte y lo bien
sentados.

---¿Y el mío será también negro?

---No, señor: a usted, por la edad, le corresponde\ldots{} café claro.

---¡Magnífico!\ldots{} Y en pantalones ¿qué tenemos?

---Sigue la moda de las telas escocesas; pero sin exagerar el tamaño de
los cuadros. Haremos a usted dos \emph{patencur}, y dos más ligeritos:
uno negro para entierros, y otro claro. Se llevan estrechos, sin tocar
en el extremo. Chalecos, se le harán a usted seis: dos de seda en claro,
uno en obscuro, dos \emph{piqué} y uno escocés.

---¡Maravilloso! Y en tanto que me confeccionan todo eso, me estaré en
casa, escondidito, leyendo las \emph{Mil y una noches}, única lectura a
que debo aplicarme ahora para hacerme a estas sorpresas\ldots{} Adiós,
maestro\ldots{} Y que se esmeren en el corte\ldots{} ¿Cuándo probamos?
Estoy aquí a su disposición todo el día. ¿Pues cómo voy a salir a la
calle con estos adefesios de ropa que he traído de mi pueblo?\ldots{}
Vaya con Dios\ldots{} y no me olvide, maestro.

Retirose el sastre, y D. Pedro Hillo, que acechaba en la puerta
aguardando que el joven estuviese solo, entró de rondón con los brazos
abiertos, diciendo muy gozoso: «Pero, niño, ¡le regalan ropa elegante, y
todavía gruñe! Rarísimos son en el Universo estos fenómenos de salirle a
uno sastres ex-machina, que le miden, le cortan, le cosen, y después no
cobran. Casos tales acaecen sólo de siglo en siglo, y hay que saber
aprovecharlos. \emph{¡Oh fortunate nate!} Yo, que para hacerme una
sotana tengo que ahorrar seis meses en la comida, le declaro a usted
simple de solemnidad si no acepta calladito esas mercedes anónimas. Por
la sagrada orden que profeso, declaro también que a mí no me ha pasado
jamás cosa semejante, y que las deidades misteriosas y las manos ocultas
no han existido para mí. A usted me arrimo, por si se me pega algo y
halla en su ventura mi desventura algún remedio. Ya, ya sé\ldots{} me lo
ha dicho Méndez, que anoche recibió usted un abultado pliego. Abrió, ¿y
qué era? Billetes para los teatros del Príncipe y la Cruz. Dígame: ¿no
ha recibido también para los Toros?»

---Todavía no---dijo Calpena sonriente;---pero por lo que voy viendo, ya
no dudo que los tendré la víspera de la primera corrida. Y como de los
teatros mandan dos, para que vaya con algún amigo, iremos juntos a la
plaza.

---Ya le mandarán también, cuando empiece el tiempo de las máscaras,
para los bailes de \emph{Trastamara} y del \emph{Café de Solís}. Pero a
eso no podré acompañarle\ldots{} Le daré consejos, porque de fijo han de
salirle aventuras y le acosarán mascaritas\ldots{}

---Ya adivino sus consejos.

---¿A que no?

---Que remate la suerte.

---No, no es eso, sino todo lo contrario. Que se prevenga contra las
celadas que pudieran tenderse a su voluntad honesta, virginal. Este
Madrid es muy malo. No se fíe usted de las caras tapadas.

---De las manos ocultas debo fiarme, según dice.

---No es lo mismo. Esa mano desconocida le viste a usted, le da de
comer, atiende a sus necesidades. Las caritas encapuchadas podrían hacer
lo contrario: desnudarle, quitarle el pan de la boca y reducirle a la
ruina y la miseria. Existirán tal vez, ¿quién asegura que no?, manos
escondidas que quieran perderle, como las hay que trabajan por su bien.
Lo primero que usted debe hacer es averiguar en qué cielo habita esa
deidad misteriosa, para poder rezarle y pedirle lo que le convenga.

---¿Qué le pediría usted para mí si estuviese en mi lugar?

---Lo primero, un destino de Hacienda o de \emph{lo Interior} con doce
mil realetes\ldots{} Y puesto a pedir, yo que usted pediría también la
cátedra de Alcalá para un amigo.

---Para usted eso y mucho más.

---Las manos mágicas deben extender sus caricias a los buenos amigos. A
Roma con Santiago he revuelto yo para conseguir esa humilde plaza, y
aquí me tiene usted esperando a que San Juan baje el dedo. Si hubiera
para mí una mano oculta, esa mano, en medio de las tinieblas de lo
incógnito, me daría una bofetada. Estoy dejado de la mano de Dios, por
lo que voy creyendo que Dios está en todas partes menos en las oficinas,
y que, si acaso está, no tiene en ellas la mano, sino el pie.

---No hay que desmayar. Hagamos un trato. Búsqueme usted a la persona
que ha mandado a Utrilla tomarme medidas, y si me la encuentra, prometo
a usted solemnemente que el primer favor que pediré a mi desconocida
providencia es esa colocación que usted desea\ldots{} esto en el caso de
que nos resulte influyente.

---¡Influyente!\ldots{} ¡Por Dios, D. Fernandito, no me venga usted con
inocencias! Esa persona desconocida tiene que ser muy alta, pero muy
alta.

---¿En qué lo conoce?

---A ver\ldots{} pronto, enséñeme usted la carta en que venían las
localidades de teatro.

---No es carta\ldots{} Es un pliego cerrado con obleas\ldots{} Aquí lo
tiene usted.

---A ver, a ver\ldots{} ¡San Canuto, qué papel más fino!\ldots{} Este
papel, puede usted asegurarlo, no se encuentra en ninguna tienda de
Madrid\ldots{} ¿Y la letra del sobre?\ldots{} ¡Ay qué letra, San
Bartolomé! ¿Es de mujer? ¿Es de hombre?\ldots{} Sr.~D. Fernando, no se
asuste de lo que voy a decirle. La mano que ha escrito esto es de sangre
real.

---¡Atiza!

---¡De sangre real!\ldots{} Y si no, al tiempo\ldots{} ¡Ay, Sr.~D.
Fernandito de mi alma, allá va una profecía! Déjeme usted ser profeta, y
adivino, y augur, y brujo, si usted quiere. Antes de cuatro días recibe
usted, como llovido del cielo, el nombramiento\ldots{} de\ldots{}

---¿De qué?

---Vamos\ldots{} de Caballerizo Mayor del Reino, digo, de
Palacio\ldots{} Y si no es esto, será de otra cosa de mucha categoría.

Rompió a reír Calpena, y dijo a su amigote:

«Pero, Sr.~D. Pedro, ¿somos clásicos o no somos clásicos?»

---Sí, sí, tiene usted razón: no desvariemos, ilustre joven; pero por de
pronto, yo, el más desgraciado de los nacidos, quiero hacer constar que
anhelo ser su amigo de usted. Sí, sí: seamos amigos; déjeme usted
arrimarme al ser más afortunado, más resplandeciente de felicidad que he
visto en mi vida. Es usted el sol, y yo me muero de frío.

---Bueno, seamos amigos---replicó D. Fernando, no sin cierta
emoción.---Y pues el día está hermosísimo, vámonos de paseo, y le
contaré a usted muchas cosas que ignora, y que quizás le hagan
rectificar sus juicios acerca de mí como depositario de la dicha
terrestre. Diré a usted quién soy, de dónde vengo, por qué estoy en
Madrid\ldots{}

---Todo eso me interesa extraordinariamente\ldots{} Ya me lo contará
usted otro día; hoy no puede ser\ldots{} Ni usted ni yo debemos salir
hoy. Nos estaremos aquí toda la mañana acechando a Iglesias.

---¿Pero Iglesias no duerme aún?

---Aún estaría en el primer sueño, o empezando el segundo, si no
hubieran venido a despertarle muy temprano, serían las siete, dos de sus
amigotes. Sin duda ocurren cosas gravísimas. ¿Y sabe usted quiénes son
esos dos que entraron, y, tirándole de una pata, le sacaron de la cama?
Pues yo tampoco lo sé a punto fijo, porque soy poco fuerte en
fisonomías. Uno de ellos me parece que es el Conde de las Navas; el otro
tan pronto me parece Fermín Caballero, como Seoane\ldots{} De que son
pájaros gordos del jacobinismo, no tengo duda\ldots{}

---¿Y a nosotros qué nos importa?

---A usted, hombre feliz por obra y gracia de la Providencia
enmascarada, nada le altera. ¿Ha leído usted \emph{El Español} de
hoy?\ldots{} ¿A que no?\ldots{} ¿A que tampoco ha leído \emph{El
Mensajero} ni \emph{El Eco del Comercio}? En mi cuarto los tengo. Vienen
los tres diarios echando bombas, cada uno según el son a que baila. Yo
me alegro, para que se arme de una vez. Esta visita de los compinches de
Iglesias tan a deshora, significa que anoche hubo gran trapatiesta en la
casa de Tepa, entiéndase \emph{logia}, y en los cafés donde bulle la
patriotería. Parece que las Juntas no quieren disolverse, las de
Andalucía sobre todo, y he aquí al Sr. Mendizábal en un brete, porque
nos ofreció poner fin a esta horrible anarquía, y en los primeros días
creímos que lo lograba. Pero aquí, para que usted se vaya enterando,
tanto puede la envidia de los propios, como la mala voluntad de los
extraños; o en otros términos, que los amigos, o sea el agua mansa, son
más de temer que los enemigos. ¿No lo entiende? Pues quiere decir que
los estatuistas templados caídos del poder con Toreno, se introducen en
los conciliábulos de los patriotas, fingiéndose más exaltados que estos,
para sembrar cizaña, y al propio tiempo los \emph{libres} que aún no
tienen empleo se van a las sacristías del otro bando y atizan candela,
para que los diarios de la \emph{moderación} se desborden y se encienda
más el furor de las Juntas. Estas nos ofrecen un espectáculo delicioso.
Una pide que se restablezca la Constitución del 12; otra que se
modifique el Estatuto, y entre todas arman una infernal algarabía. El
señor Mendizábal pretende gobernar en medio de esta jaula de locos
furiosos. Manda tropas contra las Juntas, y los soldados se pasan a la
patriotería\ldots{} Y los carlistas, en tanto, bañándose en agua rosada,
preparándose para venir hacia acá, porque Córdoba no les ataca mientras
no le manden refuerzos\ldots{} Estamos en una balsa de aceite\ldots{}
hirviendo. ¡Qué gratitud debemos al Señor Omnipotente por habernos hecho
españoles! Porque si nos hubiera hecho ingleses o austríacos o rusos,
ahora estaríamos aburridísimos, privados de admirar esta entretenida
función de fuegos artificiales.

---¿Y esos que están en el cuarto de Iglesias\ldots?

---Son patriotas furibundos\ldots{} de buena fe; de los que creen que
con degollar frailes, azotar monjas y hablar pestes de todos los
ministros, se arregla la nación. Sin quererlo, les preparan la suerte a
los moderados. Algunos creen en Mendizábal, y otros le repudian porque
no va por calles y plazuelas perorando, con un pendón en la mano\ldots{}
A todos tiene que contestar el señor de las largas levitas. Trabajo le
mando\ldots{} Si quiere usted que olfateemos lo que traman los
compinches de Iglesias, vámonos a mi cuarto, donde al paso que usted lee
\emph{El Español} y \emph{El Eco}, yo me daré mis mañas para pescar al
oído alguna palabreja\ldots{} Véngase usted para acá.

Fuéronse de puntillas al cuarto de D. Pedro, y desde él oyeron gran
batahola en el de Iglesias; y no pudiendo este resistir el fuerte
estímulo de su curiosidad, se coló en la caverna de los conjurados,
pretextando recoger un tomo de las \emph{Palabras de un creyente}, de
Lamennais, que había prestado a su amigo. No tardó en volver risueño con
el libro, y con preciosas noticias de la conspiración, que resultaba la
más inocente que en cerebros revolucionarios pudiera caber.

«Nuestro gozo en un pozo, amigo Calpena. No tratan de ahorcar a medio
mundo, ni de sublevar la tropa, ni de meter más fuego a las Juntas. Las
Juntas y toda esa marimorena les importa tanto a esos ángeles de Dios
como las coplas de Calaínos. Lo que les trae tan levantiscos es que las
elecciones para el Estamento están próximas, y ellos, cosa muy natural,
quieren ser procuradores. Mendizábal conferenció anoche con Caballero, y
parece que le asegura la elección por Cuenca. Los otros dos, y alguno
más que vendrá después, andan a la husma de las procuras, y quieren
estar bien con Mendizábal y con el Ministro de la Gobernación, D. Martín
de los Heros. Vea usted el secreto de estos aquelarres misteriosos.»

---¿Será posible, amigo Hillo, que yo, provinciano y desconocedor del
mundo y de Madrid, tenga más malicia, más trastienda que usted, que
lleva ya no sé cuántos años de andar en este terreno? Dígolo porque me
figuro que Iglesias y sus amigotes le han engañado como a un chino. Al
verse sorprendidos por la brusca entrada de usted en el escondrijo, han
variado de conversación.

---Por San Félix de Cantalicio, pienso que está usted en lo
cierto\ldots{} Me han dado el trapo. Soy toro noble.

Aún no había concluido la frase, cuando entró Iglesias resueltamente en
el cuarto de Hillo, y llegándose a D. Fernando con resuelto ademán y
sonrisa un tanto maliciosa, como de hombre muy corrido para quien no hay
nada secreto, le dijo:

«Ya sabemos, amigo Calpena, que ha traído usted de Francia un voluminoso
paquete de papeles para el Sr.~Mendizábal.»

Quedose un tanto suspenso el joven, y no supo qué responder.

\hypertarget{vi}{%
\chapter{VI}\label{vi}}

«Le entregaron a usted ese paquete en Olorón. Lo había traído de Burdeos
una señora\ldots{} No\ldots{} no se ponga usted colorado, después de
haberse puesto pálido. No se trata de ningún delito. Le dan a usted un
encargo, y usted lo cumple puntualmente. No pretendo yo\ldots{} pues no
faltaba más\ldots{} que usted me revele cosas sobre las cuales debe
guardar secreto. No, no, señor. Lo que sí puedo decirle es que el sujeto
que debía recoger ese paquete o caja de manos de usted, para entregarlo
al señor Ministro, ya no vendrá a desempeñar esa comisión, porque anoche
le han preso, y se halla incomunicado en el Saladero.»

Perplejo un buen rato quedó Calpena ante la osada interpelación de
Nicomedes, que con brusquedad tan impertinente quería producir efecto y
ver confirmados sus informes en el rostro del simpático mozo; pero
rehecho este prontamente del estupor, le contestó con tanta dignidad
como cortesía: «Nuestra amistad, señor de Iglesias, que yo estimo mucho,
no es tan antigua que a mí me permita informarle de si traigo o no
encargos para determinadas personas, ni a usted preguntármelo en forma
afirmativa, la cual revela una confianza un poquito prematura. Va usted
demasiado a prisa, amigo D. Nicomedes. Cuatro días hace que nos
conocemos.»

---Sentiría, Sr.~Calpena, que usted interpretase mal lo que acabo de
indicarle---dijo el otro, recogiendo velas.---No pretendo que usted me
revele el secreto de los encarguitos que le han confiado, ni eso a mí me
importa. Creí yo que nuestra amistad, con ser de cuatro días, es ya
bastante firme para que yo pueda tomarme la confianza de prevenirle
contra ciertos peligros\ldots{} Porque usted es un joven tan honrado
como inexperto, y podría, con el candor propio de los pocos años,
prestarse a ciertos mensajes, de cuya gravedad no tiene la menor idea.

---Se me figura, amigo Iglesias, que la calentura patriótica que usted
padece le hace ver peligros y misterios en los actos más sencillos.

---No sabe usted dónde está, y yo tendría mucho gusto, si no se empeña
en creer demasiado fresca nuestra amistad; tendría yo sumo placer, digo,
en iniciarle en la vida política, puesto que a ella piensa, según veo,
dedicarse.

---No he pensado en tal cosa. La vida política no se ha hecho para mí.

---El señor---dijo Hillo con cierta timidez,---es de los que se lo
encuentran todo hecho, y no necesita de que nadie le inicie, pues tiene
mentores y padrinos, en la sombra, que no le permitirían dar un mal
paso.

---Si hace usted caso de este clérigo---dijo Iglesias con
humorismo,---el sotana más honrado del mundo, pero al propio tiempo el
más candoroso, está usted perdido, Calpena. Haga usted caso de mí, y
déjese llevar. En la sombra no hay mentores ni garambainas. Todo eso es
romanticismo de clase averiada\ldots{} Vamos a cuentas. Lo primero,
perdóneme si le hablé con cierta impertinencia del encargo que
trae\ldots{}

---Yo no he traído papeles para el Sr.~Mendizábal---replicó D.
Fernando,---ni me habían de escoger a mí para tales mensajes.

---No abre usted la boca sin que nos dé una nueva prueba de su
inexperiencia candorosa\ldots{} Puesto que aquí todos somos amigos,
déjeme usted que hable y le ponga al tanto de la situación\ldots{} Y
antes me permitirá que le presente a dos amigos, que espero lo serán de
usted en cuanto les conozca.

Cuando esto decía, dejáronse ver en la puerta dos sujetos, que eran los
de la encerrona con Iglesias, ambos como de treinta a cuarenta años, y
al entrar revelaron por su soltura y buenos modos ser de lo más selecto
entre la juventud intelectual de aquellos tiempos. Bien supo Iglesias,
al presentarles, realzar sus nombres: «Mi amigo Joaquín María
López\ldots{} mi amigo Fermín Caballero.»

Era este de color moreno; facciones bastas y rudas, del tipo castellano,
común en campo más que en ciudades; bigote negro con mosca; cabello
encrespado, que parecía un escobillón; complexión dura; el habla ruda y
clásica, de perfectísima construcción castiza. El otro revelaba su
estirpe levantina en la finura del cutis y la viveza del mirar, en la
vehemencia de la expresión, y en la flexibilidad y gracia. Recibiolos
Calpena con franca urbanidad, y se sentaron todos, teniendo uno de ellos
que hacer sofá de la cama de Hillo, y este no cabía en sí de gozo viendo
tan honrada su pobre mansión.

«Trasladamos el \emph{Sublime Taller} desde los alcázares de Iglesias a
las góticas arcadas de Hillo\ldots---dijo con gracia López.---La Iglesia
nos ampara, nos acoge en su santo regazo.»

---La Iglesia---replicó Hillo, sentándose en un cofre,---oye y calla,
mas no otorga. En el regazo de la Iglesia no entran más que los
arrepentidos.

---\emph{Amén}---dijo Caballero,---y expliquemos en pocas palabras la
llaneza con que asaltamos la morada de estos buenos señores.

---El caso es el siguiente\ldots{} Permíteme---indicó Nicomedes, que no
gustaba de que otros dijesen lo que él podía decir.---Sabemos que el
Gobierno por una parte, la Reina por otra, despachan agentes al campo y
corte de Don Carlos, a los cuales encargan que se finjan rabiosos
absolutistas para ganar la confianza de los íntimos del Pretendiente. El
objeto es introducir allí la discordia y acabar con el absolutismo por
su propia descomposición. Al propio tiempo, los facciosos tienen aquí
infinitos emisarios que hacen el propio juego, de lo cual resulta,
señores, un tan espantoso lío, que ni aquí ni allí nos entendemos, y no
sabemos ya cuáles son los adeptos legítimos y cuáles los
apócrifos\ldots{}

---Pero hay otra cosa peor---interrumpió López, que, como buen orador,
gustaba de expresar por sí las ideas de los demás;---hay otra cosa.
Hierven discordias mil en la corte del Pretendiente, por ser muchos los
carlistas de viso que desean la transacción, siempre que el Gobierno
liberal les reconozca grados, emolumentos y honores.

---Andan estos---prosiguió Caballero, que hablaba poco y bien,---en
continuo teje-maneje de Oñate a la Granja y de la Granja a Oñate,
zurciendo voluntades y buscando la reconciliación de antiguos
comilitones, ahora desavenidos; y como, si lograran su objeto, habrían
de sobrevenir grandes males a la Nación, nosotros, que miramos por la
permanencia del sistema representativo, haremos cuanto esté de nuestra
parte porque todas esas artimañas resulten fallidas.

---Y además\ldots{} hay---apuntó Nicomedes,---una tenebrosa y hasta hoy
indescifrable conjura de la infanta Carlota\ldots{}

---Señores---declaró D. Pedro, poniéndose en pie,---la Iglesia, como
dueña del local en el cual, por su tolerancia, que no por su gusto, se
celebra esta nefanda reunión, recomienda a los señores preopinantes que
no hablen de las reales personas.

---Tiene razón nuestro noble castellano---dijo López con sorna.---No
nombraremos a ninguna persona real; pero podemos designar por su nombre
griego al que lo recibió y adoptó conforme a rito, cuando y donde todos
sabemos. Hablaremos, pues, de \emph{Dracón}.

---¡Alto!---gritó Hillo poniéndose en pie,---porque el designado con
notoria irreverencia con ese nombre, que huele a chamusquina masónica,
es S. A. el infante D. Francisco. Al menos yo lo he oído así, y no
permito, señores, no permito\ldots{}

---Bueno, bueno---dijo Caballero:---no lastimemos los sentimientos
religiosos y monárquicos con tanta sinceridad manifestados por este buen
señor. A \emph{Dracón} todos le conocemos, y no hay que hacer misterio
de él ni de su nombre de batalla. Creo que se exagera la importancia del
tal: de mí sé decir que no creo que exista plan ninguno verosímil
fundado en la personalidad del Infante.

---Poco a poco---apuntó Nicomedes.---Fermín, a ti te consta que sí lo
hay.

---No\ldots{} lo que me consta es que algunos cándidos han echado a
volar ese nombre, denigrándolo con la suposición de que teníamos en la
persona que lo lleva un nuevo Pretendiente. Y esto es absurdo; esto no
cabe en cabeza humana, ni aun en la de un español de 1835, que es la
cabeza que nos ofrece la historia como más destornillada.

---Y, sin embargo, hay quien lo dice.

---Y quien lo cree, y lo sostiene como cosa muy práctica.

---Y no falta quien asegure que es la única salvación del país.

---Señores, son muchas salvaciones para un solo país\ldots{} Salvadora
la Reina Cristina, salvador D. Carlos, salvador Mendizábal, y ahora
también D. Francisco nos quiere salvar\ldots{} Vamos, con tantas
salvaciones, España va al abismo.

---Señores, no desvariemos---indicó Hillo.---El señor infante D.
Francisco, que es persona discreta, no ha puesto sus ojos en el
Trono\ldots{} Se contentará por hoy con sentarse en el Estamento de
Próceres.

---Pretensión contraria a las leyes, tras de la cual hemos de ver y
vemos una ambición política muy sospechosa, señores, muy sospechosa.

---No exageremos\ldots{} Cuando más, cuando más, \emph{Dracón} aspira a
la Regencia\ldots{}

---¡Otra te pego!\ldots{}

---Señores conferenciantes---dijo Hillo con festiva severidad,---que no
permito, que no puedo consentir afirmaciones tan contrarias al decoro de
la Real Familia\ldots{} Si siguen sus señorías por ese camino, mandaré
que les lleven al corral.

---¿Somos gallinas?

---Toros de sentido\ldots{} de excesivo sentido, maliciosos, imposibles
para la brega, por lo cual creo que no puede acabar bien la elocuente
corrida que estamos celebrando.

---¡Ja, ja, ja!\ldots{} Muy bien. En fin, concretemos: seamos explícitos
y lacónicos, porque este joven (por Calpena) dirá, y con razón, que le
estamos embromando. ¿Verdad, señor Calpena, que no entiende usted qué
relación puede existir entre su persona y estas cosas desordenadas que
acaba de oír?

---En efecto: no se me alcanza qué concomitancia pueda tener mi humilde
persona con esos agentes reservados, con esas intrigas, con el
Sr.~\emph{Dracón} y demás\ldots{}

---Hemos sabido---dijo Nicomedes con campanuda solemnidad,---que de
Francia se remitió un paquete de interesantes papeles a Madrid\ldots{}
No vaya usted a creer que intentamos sustraer ese tesoro, y
apropiárnoslo por medios contrarios a la hidalguía. En poder de usted se
halla todavía el encargo. La persona que debía recogerlo ha sido presa,
y probablemente no saldrá pronto de la cárcel. Es muy posible que
alguien intente apoderarse del paquete, diciendo a usted que viene de
parte de su legítimo dueño. Yo le suplico, señor D. Fernando, que no lo
suelte, aunque los que vengan a pedirlo le presenten esquela del mismo
Sr.~D. Eugenio Aviraneta, a quien viene dirigido, porque tanto el recado
como la esquela serán falsos de toda falsedad.

---Pues correspondo a su franqueza---dijo D. Fernando, a quien todos
oían con vivísima atención,---que no traigo yo encargo ni cosa alguna
para ese señor que acaba de nombrar; y si algo hay en mi baúl, que me
confiaron en la frontera personas de toda mi confianza, y que no
conspiran ni han conspirado nunca, lo entregaré a quien venga a
reclamarlo, siempre que acredite, por usual conocimiento, ser la persona
a quien viene rotulado.

---Pues aún me resta decir algo para que vean todos mi sinceridad y
nobleza. Antes dije a usted que el paquete venía dirigido a Mendizábal;
pero esto lo hice sin más objeto que desconcertarle a usted, con la idea
de que su turbación le arrastrase a revelarme algo que yo quería saber:
lo que usted trae no viene dirigido a Mendizábal, ni tiene nada que ver
directamente con nuestro célebre gaditano. Pero personas muy altas, muy
altas, fijese bien en lo que afirmo, pudieran tener noticia de que el
señor Calpena es portador de papeles graves, y en este caso no dejarían
de intentar por todos los medios apoderarse de ellos.

---En vez de aumentar la confusión de este excelente joven---indicó
Caballero,---procuremos disiparla, amigo Nicomedes, y al propio tiempo,
convenzámosle de que no pretendemos apoderamos de secretos que no se nos
quieren confiar.

---Justamente---dijo López,---y empecemos por declarar que ignoramos, o
por lo menos, que no sabemos con exactitud qué documentos se han
confiado a su discreción. Puede ser algo que exclusivamente interese a
la Familia Real; puede ser del común interés de los partidos militantes.
Me inclino a creer esto. El propio Aviraneta no sabe lo que es, o no
quiere decírnoslo.

---No lo sabe---afirmó Iglesias.---Así me lo aseguró ayer, y debemos
creerlo.

---\emph{Hame dado en la nariz}---dijo Caballero,---que lo que han
remitido a D. Eugenio es todo el fárrago de papeles concernientes a la
\emph{Confederación isabelina}, de infausta memoria. Él mismo se lo
llevó a Francia no sé con qué objeto, y de allá se lo remiten para que
lo utilice aquí en contra nuestra, y en pro de los Torenos y
Martínez\ldots{} Yo, señores míos, me fío poco de Aviraneta, y no
quisiera que mis amigos tuvieran interés por nada que al infatigable
conspirador se refiera\ldots{} Fíjese usted, Sr.~Calpena, en lo que voy
a decirle, para que no se embrollen sus ideas con la extraordinaria
confusión que ha de resultarle de lo que decimos. Los estatuistas nos
acusan de haber preparado, dispuesto, organizado, en una palabra, el
degüello de los frailes, el asesinato de Canterac y otros abominables
hechos de que usted tendrá conocimiento. Se nos quiere denigrar,
inutilizar para la gobernación del Reino. Si hay responsabilidad, no
pueden ellos eludirla, pues en los terribles días de Julio del año
pasado era Presidente del Consejo el Sr.~Martínez de la Rosa; Ministro
de la Gobernación el Sr.~Moscoso, y Corregidor de Madrid el señor
Marqués de Falces. ¿Sabéis lo que, en mi presunción, contiene la
estafeta que ha traído el Sr.~Calpena? Pues el plan de Constitución que
hicimos Olavarría y yo; la exposición dirigida a S. M. por Flórez
Estrada, condenando el Estatuto; el proyecto de asonada general; el plan
de Ministerio, presidido por Pérez de Castro; los compromisos contraídos
por Palafox y Calvo de Rozas, con el nombre de \emph{trabajos
militares}, y, por último, el informe de la Comisión que nombramos para
proponer al Gobierno el mejor sistema de \emph{extinción de frailes}.
Todo eso y algo más había. Aviraneta, como iniciador de la
\emph{Isabelina}, arrambló con el archivo cuando la persecución de la
policía le obligó a emigrar a Francia. ¿Trataría de hacer algún negocio
con Luis Felipe? ¿Habrá entrado en contubernios con D. Carlos? Yo no lo
sé\ldots{} Ya os he dicho que no me fío de ese hombre, y que de su
refinada astucia y doblez lo temo todo. Vosotros creéis en Aviraneta; yo
no. Para mí es un monstruoso talento, el más sutil y agudo para la
intriga. El año pasado conspiraba o aparentaba conspirar con nosotros.
Este año trabaja secretamente por los enemigos del progreso. Vosotros
creéis en sus alardes de patriotismo revolucionario; yo no. Vosotros
confiáis en su lealtad; yo desconfío hasta de su sombra. Si le ayudáis,
ayudáis al desprestigio de Palafox, de D. Jerónimo Valdés, de San
Miguel, de los patriotas Quiroga y Palarea, de Salustiano, del propio
Mendizábal, pues ya sabéis que D. Juan Álvarez comunicó desde Londres su
propósito de constituir allí un \emph{Círculo isabelino}, y de facilitar
fondos para la causa, y en esfera más modesta ayudáis también a vuestro
propio vilipendio y al mío\ldots{}

---Fermín, Fermín---dijo Iglesias, apretando los puños, encendido el
rostro:---tú siempre pesimista, tú siempre malévolo y suspicaz,
desconfiando de los hombres más adictos a la idea, de los que han sabido
padecer por ella persecuciones horribles.

---Y tú, Nicomedes, siempre iluso y confiado, pobre enfermo de la
\emph{calentura patriótica}, ni aprendes nada de la experiencia, ni
atiendes a las lecciones del tiempo. Tanto a ti, pobre Iglesias, como a
ti, Joaquín, almas crédulas, espíritus generosos, os digo que
desconfiéis de Aviraneta, que no le ayudéis en sus maquinaciones, que le
dejéis solo en la febril inquietud de su conspirar instintivo, genial,
por amor al arte, por ley de su naturaleza.

Y cambiando bruscamente al tono familiar, antes que sus atontados amigos
pudieran replicarle, se levantó y formuló la despedida en estos
términos: «Ya he sermoneado bastante, y ahora me voy, que tengo que
trabajar. Holgazanes, quedaos con Dios.»

---Fermín, aguarda, siéntate\ldots{} que aún tenemos mucho que hablar.

---¡Hablar! La maldita palabra. Es la sarna del país. España llegará al
fin del siglo sin haber hecho nada más que rascarse, es decir,
hablar\ldots{} Quedaos con Dios\ldots{} Y usted, Sr.~de Calpena, al
aceptarme por su amigo, me va a permitir que le dé un consejo. Es usted
muy joven; yo tengo treinta y seis años y alguna experiencia. No haga
caso de estos pobres orates. Si quiere usted seguir el consejo de un
patriota honrado, que no padece la famosa \emph{calentura}, y profesa
sus ideas con fría convicción, no sirva usted de correo a los
conspiradores de oficio. Y pues le han cogido de sorpresa, encargándole
comisiones que no habría aceptado con conocimiento, vénguese por el
método inquisitorial\ldots{} En vez de entregar los papeles al Sr.~de
Aviraneta, arrójelos a las llamas. Ganará usted mucho en tranquilidad de
conciencia.

---¡Quemarlos! ¡Eso no!---gritó Iglesias.

---Créame a mí\ldots{}

---No le crea, no, Fernando. Es de Cuenca, que es como decir leñador y
carbonero\ldots{}

---Carbón, sí; carbón haría yo de todo ese fárrago de sandeces---dijo
Caballero con arrogancia, enarbolando su bastón.---Nuestro pasado
político, amigos revolucionarios, debe ir al fuego\ldots{} Quemad la
broza, que las ideas, no temáis\ldots{} esas no arden.

Y encasquetándose el sombrero, que era de los voluminosos que entonces
se usaban, salió del cuarto y de la casa con resuelto y presuroso andar.

\hypertarget{vii}{%
\chapter{VII}\label{vii}}

Aunque desconcertados por la enérgica manifestación de Caballero, que al
fin hubo de condenar las bajas intrigas, no cejaron Iglesias y López en
su propósito de catequizar al joven Calpena. Aún insistió D. Joaquín en
que entregase el \emph{lío} a D. Eugenio Aviraneta, sin pensar en
hacerlo cisco, como le aconsejara Fermín con implacable rigor; y más
atrevido Iglesias, propuso al joven, no que pusiese en sus manos lo que
era objeto de tantas cavilaciones, sino que permitiera ver su contenido,
prometiendo ambos guardar profundo secreto sobre lo poquito que examinar
pudiesen. Negose resueltamente D. Fernando, y ellos invocaron los
principios liberales que sin duda el joven profesaba; los grandes
intereses del pueblo, al cual todos pertenecían; y añadiendo a los
halagos las promesas, ofrecieron traerle antes de tres días una
credencial de ocho mil reales en cualquier Ministerio, si a satisfacer
su ardiente curiosidad se prestaba. Pero ni las demostraciones de
amistad, ni las ofertas de colocación, quebrantaron la delicada entereza
de D. Fernando, el cual decididamente, con frase categórica y un tanto
áspera, les quitó toda esperanza, alentándole en esto su amigo Hillo con
muecas y manotadas expresivas. Replegáronse de mal talante los patriotas
al cuarto de Iglesias, y lo primero que hizo D. Fernando al entrar en el
suyo fue guardar bajo llave, en los seguros cajones de una cómoda, el
contenido de su baúl, o aquella parte que convenía poner a cubierto de
cualquier sorpresa.

«Hace usted bien---le decía Hillo gozoso,---porque estos \emph{libres},
como ellos se llaman, no se paran en pelillos. Fuera del patriotismo,
son honrados, y por nada del mundo le quitarían a usted un botón ni un
cigarro de papel. Pero en mediando lo que ellos llaman \emph{el interés
de la Confederación} o de la libertad, aunque esta sea tan desacreditada
como la de la imprenta; como se trate de arma política con que puedan
descabellar al contrario y arrastrarle por el redondel, se ciegan, y de
noblotes y decentes se convierten en los primeros badulaques del mundo.»

De acuerdo en esto como en todo, pues los lazos de su amistad se
apretaban más cada hora, salieron a dar un paseo antes de comer.

«¡Qué hermoso apóstrofe el de Caballero!---decía, calle abajo, hacia la
de Alcalá, el buen clérigo Hillo.---Mejor será llamarlo
\emph{conminación} o \emph{deprecación}\ldots»

---Llamémoslo \emph{corrección fraterna}, que así deben nombrarse los
hijos de tal padre. Me ha gustado D. Fermín. ¿Sabe usted que los otros
parecen locos?

---Y no es lo peor que lo parezcan, sino que lo sean, y que nos
comuniquen a nosotros su locura. Yo siento un gran desorden en mi
cabeza.

---Y yo. Le aseguro a usted que me falta poco para ponerme a gritar en
medio de la calle. ¿Con que es verdad que he conspirado sin saberlo?
¿Con que es verdad que traigo papeles que comprometen a la Real
Familia\ldots{} o a los reales masones, o a los isabelinos, o al demonio
coronado? Y ahora consulto yo con usted una sospecha grave: ¿tendrá
alguna relación este enredo con los favores que recibo de mano
desconocida?\ldots{} Esa personalidad misteriosa que en las tinieblas me
protege, ¿tendrá algo que ver con\ldots{} con no sé qué?\ldots{} Yo
desvarío, se embarullan mis ideas. ¿Me encontraré envuelto, sin culpa
ninguna, en alguna endemoniada intriga? Dígame su franca opinión\ldots{}
Usted es hombre de mundo, y conoce esta sociedad y estos manejos de la
política. Yo soy un inocente: vengo de un pueblo fronterizo y de una
ciudad extranjera, donde he vivido amarrado a un bufete de
comerciante\ldots{} Yo no sé nada de esto. Ilumíneme usted; indíqueme si
debo hacer algo, o no hacer nada y dejar correr los
acontecimientos\ldots{}

---Pues, mi amigo D. Fernando, creo, y no hay que asustarse, que se
halla usted metido de hoz y coz en un lío estupendo\ldots{} Dígame ante
todo: ¿es cierto que trae usted esa caja?

---Sí, señor; a usted puedo decírselo. Traigo un paquete bastante pesado
y voluminoso. Me lo dio una señora que en Olorón visitaba mucho a los
hermanos de mi padrino\ldots{} Díjome que se presentaría a recibir el
encargo la persona a quien viene rotulado, y es también una señora, y se
llama Doña Jacoba Zahón.

---Eso de Zahón me huele a masonería. Y la señora que lo entregó a
usted, ¿quién es?

---Allí la llamaban la Marquesa, y decían de ella que politiqueaba, que
sostenía larga correspondencia, y que en Tours y en Burdeos estuvo en
relaciones íntimas con algunos emigrados liberales.

---¡Ah\ldots{} por San Benito de Palermo!\ldots{} Ya veo, ya veo
claro\ldots{} digo, no, no veo más que obscuridades y fantasmas\ldots{}
Señora allá que manda, señora aquí que recibe\ldots{} Aviraneta\ldots{}
La \emph{Confederación isabelina}\ldots{} el degüello de
regulares\ldots{} Mendizábal\ldots{} Usted recibido y aposentado en
Madrid por personas desconocidas que no dan la cara\ldots{} usted
vestido por Utrilla\ldots{} usted obsequiado con billetes de teatro y
con otros regalitos que no habrá querido decirme\ldots{} ¡Ay! D.
Fernando de mi alma, como mi religión me ordena no creer en brujas, y mi
experiencia me permite creer en enjuagues masónicos, yo le veo a usted
tocado de locura, y me vuelvo loco también, porque no entiendo una
palabra de este intrincado negocio.

---¡Y luego decimos que somos clásicos!

---¡Clásicos! Eso quisiéramos. El mundo está tocado de insana
demencia\ldots{} Ya no pasan las cosas como antes, con aquella pausa y
regularidad de otros tiempos; todo está trastornado; reina la sorpresa,
mangonea el acaso, y los acontecimientos se suceden sin ninguna lógica.
Ya no hay reglas, mi querido D. Fernandito. Esto es el caos, la
barbarie, la anarquía de las almas. Corre un viento de desorden, y en la
naturaleza no hay aquella serenidad, aquella calma majestuosa\ldots{}
¿Digo mal?

---Dice usted muy bien. Yo me noto lanzado en este vértigo, en este
espantoso remolino.

---Todo por ese maldito\ldots{} Hasta me repugna pronunciar su nombre.

---Ese maldito\ldots{} ¿qué?

---¿Sabe usted, Fernando Calpena---dijo el clérigo con solemne gravedad,
parándose en firme,---quién tiene la culpa de esta locura que nos saca
de quicio, de esta llamarada que nos abrasa el rostro, de esta comezón
que nos hace bailar la tarántula?

---¿Quién tiene la culpa?\ldots{}

---¡Qué! ¿No lo acierta? Pues tienen la culpa Víctor Hugo y Dumas, esos
dos infames progenitores del romanticismo\ldots{} ¡El romanticismo! Ese
es el remolino, ese es el vértigo, esa es la locura\ldots{}

---D. Pedro---dijo Calpena, sin encontrar pertinente lo que afirmaba su
amigo,---¿qué tiene que ver\ldots? ¡Dumas, Víctor Hugo!\ldots{} son dos
grandes poetas\ldots{}

---Que han desatado las tempestades en nuestra literatura, y tras el
desquiciamiento de la literatura, ha venido el de la política, y luego
el de la vida toda\ldots{} Yo, a esos dos, les mandaría cortar la
cabeza, sin cargo alguno de conciencia, como a malhechores del género
humano, y me quedaría tan fresco\ldots{} ¿No ve usted que ya no hay
orden ni reglas en el curso de los hechos que constituyen la vida? ¿No
ve usted que ya todo es exaltación, misterio, fantasmas, lo desconocido,
lo imponderable?\ldots{} Pues espérese usted un poco, que ya empezarán
los espectros, las tumbas, los cipreses funerarios\ldots{} En fin,
vámonos a comer, que yo, la verdad sobre todo, tengo ya ganas. Y esta
tarde nos iremos a dar un largo paseo por las afueras, para que usted me
cumpla su promesa de contarme algo de su vida, y del cómo y el por qué
de haber venido a este maldito Madrid.

---Volvámonos a casa---dijo Calpena sobresaltado, pues temía un
golpetazo repentino de la suerte, como contrapeso de tantas
venturas,---y veremos cuál es la sorpresa de esta tarde.

---¡Qué!\ldots{} ¿Teme que venga de sopetón la mala?\ldots{} Deseche
usted ese recelo, porque si viniera la mala, caería sobre mí. Quiero
decir que aquí está Pedro Hillo para recogerla, pues yo seré su
pararrayos, Sr.~D. Fernandito. No dude que si salta la chispa caerá
sobre este cura\ldots{} y usted libre, usted siempre feliz\ldots{} Si
no, al tiempo.

Sorpresa hubo, en efecto; mas no desagradable, como Calpena temía. Al
entrar le dio Méndez un paquetito que acababan de traer. Pálido y
ceñudo, el joven no se atrevía a cogerlo. Hízolo Hillo, tomó el peso, y
se echó a reír diciendo: «Que me excomulguen si esto no es dinero
contante y sonante.»

El paquetito era como una carta muy abultada, o como un libro de poco
volumen, esmeradamente envuelto en papel superior, cerrado con lacres.
Estos no tenían sello con letras o escudo. Antes de abrirlo, preguntó D.
Fernando a Méndez quién lo había traído.

«Ha sido el mismo señor, ese que llaman \emph{Edipo.»}

---No puede ser más clásico---observó Don Pedro.---A ver, a ver\ldots{}
abra usted.

---Podría usted haberle dicho que se esperara. Yo le habría
interrogado\ldots{} En fin, veamos qué es esto.

Metiose en su cuarto con Hillo, y en pocos segundos quedó aquel nuevo
enigma descifrado a medias, pues si debajo del envoltorio apareció una
elegantísima y perfumada cartera de piel, con un cartoncillo en el cual
resplandecían ocho medias onzas prendidas con cruce de seda encarnada,
no se encontró papel escrito, ni tarjeta, ni cifra por donde la
procedencia pudiera ser conocida.

«Muy bien---dijo el presbítero restregándose furiosamente las
manos.---Eso no podía faltar\ldots{} Aparece la lógica en medio de este
barullo romántico\ldots{} Le mandan a usted dinero para el bolsillo,
pues un joven vestido por Utrilla, un caballero que ocupará altas
posiciones, que figurará entre los más elegantes de Madrid, no es bien
que ande sin pólvora\ldots{} Ea, no se devane ahora los sesos\ldots{} Ya
parecerá, Señor, ya parecerá el donante. Vámonos al comedor, que con
estas sorpresas se me aguza el apetito.»

Comieron solos, porque Iglesias, convidado por López, se había ido a la
fonda de Genieys; D. Fernando hablaba poco; a Hillo se le despertó la
locuacidad con tanta fuerza como el apetito, y trataba de apartar al
joven Calpena de la sombría cavilación en que había caído\ldots{} «Antes
dije a usted que estábamos locos, y ahora añado que bendita sea la
locura si viene siempre así. Mientras lluevan medias onzas, ora sean
pasta, ora transformadas en cosas de diferente utilidad, no llore usted,
joven. Si luego nos cae alguna rueda de molino, tiempo habrá de
lamentarlo. Y hablo en plural, porque si mi delicadeza no me permite
participar de los beneficios exclusivamente destinados a usted, deseo y
quiero ser partícipe de los males, cuando Dios se fuere servido de
enviarlos. Con que reposemos un rato la comida, y luego nos iremos a
estirar las piernas al Retiro.»

Hiciéronlo así, y descansando de su caminata a la sombra de unos copudos
negrillos, en sitio sosegado, allá por el Baño de la Elefanta, D.
Fernando se franqueó con su amigo, ofreciéndole los datos biográficos
que anhelaba conocer, como clave o guía para descubrir la
\emph{misteriosa mano}.

«Los primeros recuerdos de mi infancia---contestó Calpena,---se refieren
a Vera, y a la casa del cura de aquel pueblo. Pero yo nací y fui
bautizado en Urdax, no constando en la partida más que el nombre de mi
madre, Basilisa Calpena. Ni la conocí nunca, ni he sabido de ella, pues
la mujer que me crió se llamaba Ignacia, natural de Zugarramundi,
habitante en Vera, en una casita próxima a la del cura. No tenía yo dos
años, cuando este me llevó consigo, y ya no me separé de él hasta su
muerte, ocurrida el año 32. Llamábale yo padrino, y él a mí ahijado y a
veces hijo. Era el hombre más excelente que usted puede imaginar, sin
tacha como sacerdote, verdadero pastor de sus feligreses; tan
caritativo, que todo lo suyo era de los pobres; entendido en mil cosas,
principalmente en agricultura, en astronomía empírica y en humanidades;
gran latino, tan modesto en sus hábitos, y tan apegado a la humilde
iglesia en que desempeñaba su ministerio, que rechazó la oferta de una
capellanía de Roncesvalles y del deanato de Pamplona. Para mí, D.
Narciso Vidaurre, que así se llamaba, era la primera persona del mundo,
y en él se condensaron siempre todos mis afectos de familia, pues él era
para mí como padre y maestro. Si no me había dado la vida, me dio la
crianza, la educación, y me enseñó a ser hombre, infundiéndome la
dignidad, la confianza en mí mismo, y preparándome para los mil trabajos
de la vida. Desde niño me enseñó todo lo concerniente, en lo moral y en
lo social, a personas principales\ldots{} quiero decir que me crió para
señor, no para sirviente ni para la vida oscura y zafia del campo.
Aunque no con puntualidad, D. Narciso recibía cantidades para mi
sostenimiento, educación y demás. Él venía unas veces de Madrid, otras
de Burdeos o París. De esto me enteré yo en mi niñez; pero él nunca me
dijo nada, y aunque a veces aludía vagamente a mis padres, dándome a
entender que existían, y que yo podría conocerles andando el tiempo,
jamás me habló concretamente de asunto tan delicado. Sin duda, no se
creía con facultades para hacerme tal revelación; o tal vez aguardaba a
que yo cumpliese determinada edad. No sé, no sé, amigo Hillo\ldots{} Mis
confusiones son ahora las mismas que hace algunos años. Quizás, si mi
padrino viviera, ya habría cesado mi ignorancia de cosa tan importante;
quizás\ldots»

---Permítame\ldots{} Entre paréntesis\ldots---dijo D. Pedro, que ponía
profunda atención en el relato.---Una pregunta: ¿en aquel tiempo recibía
usted también favorcitos misteriosos de la \emph{mano oculta}?

---En tiempo de mi padrino, jamás. En París, una vez sola. Ya llegará
oportunidad de contarlo\ldots{} Seguiré con método.

---Permítame otra pregunta: ¿ese señor murió de repente?

---Sí\ldots{} de un ataque apoplético. No le dio tiempo a nada.

---Claro\ldots{} si hubiese tenido tiempo, lo natural y lógico era
llamarle a usted\ldots{} decirle: «Hijo mío, tal y tal\ldots»

---Su muerte fue para mí un golpe tremendo. Parecíame que se acababa el
mundo, la humanidad; que yo me veía condenado a soledad eterna, a un
desamparo tristísimo\ldots{} Aquel santo hombre era para mí la única y
total familia, el maestro, el amigo, el inspirador de todos mis
pensamientos, guía de todos mis actos\ldots{} Dejome un horrible
vacío\ldots{}

---Dispense\ldots{} Otra pregunta: ¿no tenía el buen D. Narciso, como es
uso y costumbre en la clase de curas, alguna familia de sobrinas,
amas?\ldots{} ¿o es que vivía enteramente solo?

---Tenía una hermana más vieja que él, Doña María del Socorro, que le
llevó tres años por delante en el morir; buena señora, aunque algo
regañona y descontentadiza, y un hermano que no vivía en Vera\ldots{}
Muerta Doña María, siguieron gobernando la casa una sobrina, que al poco
tiempo casó con uno de Fuenterrabía, y dos antiguas criadas de la
familia, que aún sirven al sucesor en el curato, un sobrino segundo,
llamado Avelino, buen muchacho, pero que no es ni la sombra de su
tío\ldots{} No nacerá otro D. Narciso Vidaurre, el santo, el justo, el
sabio, el discreto, el\ldots{}

\hypertarget{viii}{%
\chapter{VIII}\label{viii}}

Nueva interpelación de D. Pedro, que impaciente quería profundizar en el
hermoso asunto, para llegar pronto a la verdad. «Perdóneme otra vez,
Fernandito, si le interrumpo. ¿Ese señor cura no se señaló, como todo el
clero navarro, por la adhesión a las ideas y a la persona de D. Carlos
María Isidro?»

---Verá usted\ldots{} Mi padrino, hombre de acendrada religión,
manifestaba despego a los revolucionarios y jacobinos\ldots{} Del 14 al
20 simpatizó con los realistas, por lo cual le tuvieron entre ojos las
autoridades de los \emph{tres años}. Poco antes de la entrada de
Angulema, tuvimos que salir de Vera y refugiarnos en Cambo. Pero a
principios del 24 ya estaba mi padrino en su parroquia, y entonces le
ofrecieron la canonjía de Pamplona, que rehusó. Desde el 24 hasta la
muerte del Rey, se abstuvo de manifestar con demasiada viveza sus
sentimientos realistas. Debo decir también que el buen señor tenía
relaciones con personas del bando liberal. Era muy amigo del general
Mina\ldots{}

---¡De D. Francisco Espoz y Mina!

---Hacia el 22, comía en la Rectoral siempre que pasaba por Vera\ldots{}
También tenía D. Narciso gran confianza con Eraso, el segundo de
Zumalacárregui, y aun con este, en época anterior al carlismo, cuando
Don Tomás era coronel de ejército. Sí, señor\ldots{} ¡Pues tengo tan
presente a Mina\ldots{} le vi tantas veces en mi casa!

---¿Y con usted se mostraba cariñoso?\ldots{}

---Como que monté a caballo más de una vez en sus rodillas. Me quería
mucho\ldots{} me llamaba \emph{petit caporal} y no sé qué\ldots{} Ahora
que recuerdo: también nos visitó alguna vez el Conde de España.

---¿Y en las rodillas de ese también montaba usted?

---Creo que no. La época es más remota, y apenas me acuerdo.

---¿Y entre tantos generales no iban alguna vez generalas?\ldots{} ¿No
recuerda haber visto en la casa del cura duquesas o princesas\ldots?

---Personas de tanta categoría\ldots{} no sé\ldots{} como no fueran
disfrazadas.

---Adelante. Murió el señor Cura, sin poder decir oste ni moste\ldots{}
y luego\ldots{}

---El hermano de D. Narciso vivía en Urdax, dedicado al tráfico de
maderas. Este señor se encargó de mí. Honrado y cabal, no se parece nada
a su difunto hermano: carece de instrucción, y es seco, adusto, sin
delicadeza. Lo primero que hizo conmigo fue mandarme a Olorón para que
siguiera mis estudios en un colegio. Allí viví unos meses en casa de un
tal Maturana, habilísimo mecánico y armero, algo pariente y amigo íntimo
de los Vidaurres. De pronto recibí órdenes de trasladarme a París a
aprender prácticamente el comercio, pues al comercio quería dedicarme.
Me mandaban acá y allá, sin darme explicaciones, y si alguna observación
hacía yo, me respondían simplemente: «Manda quien manda.»

---Ya me habló usted de su viaje a París para entrar en la casa de Banca
donde conoció a Mendizábal; dígame ahora cómo se le manifestó la
\emph{mano oculta} en aquella ciudad.

---Yo vivía con otro chico guipuzcoano, compañero mío de escritorio, en
una modesta pensión del \emph{faubourg} Poissonière. Un día me encontré
en la mesa de mi cuarto una carta dirigida a mí. Dentro de ella había
dos billetes de la \emph{Banque de France}, que allí circulan como
metálico. Total: doscientos francos, que me vinieron muy bien. No pude
averiguar quién me había llevado la carta: ni en la casa ni en mi
oficina supieron darme ninguna razón. Pero aquella vez el dinero no
venía solo, sino con una cartita muy lacónica en que se me mandaba oír
misa, al día siguiente, a las nueve en punto, en la iglesia de
\emph{Notre Dame des Victoires}. Naturalmente, fui, y nada me sucedió,
es decir, nadie se me acercó a hablarme, como esperábamos mi compañero y
yo, que creímos se trataba de una aventura vulgar.

---Si usted no vio a nadie, sin duda alguien a usted le vería\ldots{}
¿Era ya en el reinado de Luis Felipe?

---Sí, señor. De repente, con la misma brusquedad con que fui enviado a
París, llamáronme a Olorón, y allí estaba cuando se nos presentó
Faustino Vidaurre, al parecer para tratar de negocios\ldots{} Noté yo
que él y Felipe Maturana se decían algo referente a mí, recatándose de
que yo lo entendiera. Una mañana me notificaron que vendría pronto a
Madrid, donde se me daría un destino en las oficinas del Gobierno, con
sueldo bastante para vivir decentemente en esta capital. Yo me alegré,
porque allí no hacía nada, y la holganza monótona de aquel pueblo me
enfadaba, me ponía enfermo\ldots{} Vi los cielos abiertos; me aventuré a
pedir alguna explicación al hermano de mi padrino; pero no me dijo más
que la frase sacramental: «Quien manda, manda.» Y Maturana agregó:
«Llevarás tu viaje pagado, y algo para que puedas vivir un par de meses
en un alojamiento arregladito. Ya puedes empaquetar tu ropa y tus
libros\ldots» Y como yo expresase alguna inquietud acerca de mis
primeros pasos en esta villa, no teniendo aquí conocimientos ni trayendo
carta de recomendación, Faustino me dijo: «Anda, anda, hijo, y no temas
nada, que ya tendrás quien te ampare y mire por ti. Vete descuidado, que
nada te faltará\ldots{} Y no te mandamos tan desprovisto de apoyos y
recomendaciones, pues además de los que allí te saldrán donde y cuando
menos lo pienses, en Madrid tienes a nuestro primo Carlos Maturana,
diamantista que fue de la Real Casa, y hoy comerciante en piedras
preciosas. Ya le hemos escrito para que te preste algún socorro, si por
acaso lo necesitares. Pero no esperes encontrarle en la Corte hasta los
últimos de Septiembre, porque ahora está viajando por el Norte de
Italia, y tardará un mes lo menos en llegar a Madrid. Vive en la plaza
de la Armería junto a Palacio.» Llegó el día de mi partida, y me
despidieron muy conmovidos, como si no pensaran volver a verme. Tanto
Maturana como Faustino y las mujeres de ambos, me dirigieron el último
saludo con una extrañísima gravedad\ldots{} vamos, con algo como
demostración de respeto\ldots{} No sé si me explico\ldots{}

---Comprendido, comprendido\ldots{} Es muy natural\ldots{} ¿Y\ldots?

---Ya, a eso voy. Dos días antes de mi salida de Olorón, se llegó por
allí una señora muy estirada, con muchos moños grises alrededor de la
cabeza, sombrero con cintas y encajes. Hablé con ella dos o tres veces,
asombrándome de su instrucción, de su finura, de su conocimiento de la
política, así francesa como española. La esposa de Maturana, persona
también de excelente educación, francesa, hija de un librero de Foix,
celebraba frecuentes encerronas con la dama desconocida. A esta la
llamaban \emph{Madame Aline}.

---¿Francesa?

---Pues mire usted que no lo sé\ldots{} Habla correctísimamente el
español, aunque con un ligero acento\ldots{} no sé, me pareció catalán.
Pues bien: esta señora fue la que me dio el encargo que tan
soliviantados trae a nuestros patriotas. Tanto ella como Maturana me
encargaron tuviese mucho cuidado de no entregar el paquete más que a la
persona a quien viene dirigido. «Será muy difícil---me dijo madame
Aline,---que haya equivocación ni suplantación, si usted se fija bien en
las señas que le doy. La señora en cuyas manos pondrá usted la cajita es
jorobada.»

---¡Lo ve usted!---exclamó Hillo, dándose un fuerte palmetazo en la
rodilla.---¿Ve usted cómo acertaba yo cuando hablé del torbellino
romántico? En el romanticismo desempeñan siempre un papel culminante los
jorobados, o siquiera cargados de espalda, los tuertos, patizambos, y en
general toda persona que tenga alguna deformidad visible. También
figuran en él los tísicos, los locos y los que padecen ictericia.

---Jorobada---me dijo,---de sesenta años, y algo impedida de la pierna
derecha.

---Bueno, bueno, bueno\ldots{} Lo que digo: en pleno romanticismo. ¿Y
qué nos importa? Mejor, más divertido: no nos faltarán emociones,
sorpresas y\ldots{} corcovas\ldots{} ¡Ay! Fernandito de mi alma, me
equivocaré mucho si de todo esto no resulta una anagnórisis
felicísima\ldots{} Nada, nada, no hay que temer nada malo, sino una
verdadera irrupción de bienes. Yo estoy contento, no sé qué me pasa. El
bien ajeno no me produce envidia, sino una exaltación de cariño y
entusiasmo por la persona favorecida. Así es que estallo de
satisfacción, y me parece que esta noche he de atacar la cena con un
apetito fenomenal. Adelante. ¿Falta algo?

---Sí señor: falta que usted conozca la clase de educación que me dio mi
padrino; los sentimientos con que fortaleció mi conciencia; las ideas
con que fue labrando mi criterio\ldots{} Desde muy niño me acostumbro a
mirar la moral excesivamente severa como base de una vida ejemplar. La
moral rígida, según él, es un deber que impone la fe, y al propio tiempo
una indudable ventaja para la vida. Me enseñó a abominar de la mentira,
siendo en esto tan extremoso, que ni aun me permitía los embustes
inocentes que son el encanto principal de la infancia. De amor al
prójimo, de caridad y abnegación, no hablemos, pues esto, con sólo su
ejemplo, diariamente me lo enseñaba. Ponía un cuidado exquisito en que
yo aprendiese desde muy niño a refrenar los deseos violentos, a no
apetecer cosa alguna con demasiado ardor, a poner freno a las pasiones.
Ya he dicho a usted que era un humanista de primer orden, y clásico
ferviente, resultando armonía perfecta entre su gusto artístico y todos
los actos de la vida, que iban siempre a compás, como sus pensamientos.
De los modernos autores, Moratín era su ídolo. Se carteaba con él y con
el abate Melon, y se sabía de memoria todas las poesías serias y
festivas de D. Leandro, así como sus traducciones de Horacio. ¡Cuántas
veces le oí declamar con grave entonación aquel pasaje!:

\small
\newlength\mlena
\settowidth\mlena{¿De cuál varón o semidiós el canto}
\begin{center}
\parbox{\mlena}{¿De cuál varón o semidiós el canto            \\
                \null \qquad previenes, alma Clío,            \\
                en corva lira o flauta resonante?}            \\
\end{center}
\normalsize

La sátira «¿Quieres casarte, Andrés?» la repetía enterita, sin el menor
tropiezo. Explicándome las bellezas de estas composiciones, me hacía ver
cómo la poesía, para ser de buena ley, debe subordinar la inspiración al
buen gusto y a la regularidad. Mas no quería que fuese yo poeta, y una
vez que me sorprendió haciendo versos, me los puso en solfa, incitándome
a que, en vez de expresar mis pensamientos con música y medida,
cultivara la buena prosa, que, sin duda podía ofrecerme ancho campo al
empleo de la inteligencia, así en la oratoria política, como en la
forense, en la historia, en la filosofía, y en todas las artes
liberales. Por Cicerón tuvo verdadera idolatría, y decía que era lástima
fuese gentil un hombre que expresaba las ideas con tal perfección, dando
al raciocinio la palabra más propia y más enérgica. Repetía la memoria
pasajes del gran orador y filósofo; me los explicaba; me hacía ver su
concisa elocuencia, la propiedad, el empleo exacto de las voces\ldots{}

---Repetiría aquel pasaje: \emph{Nihil agis, nihil moliris, nihil
cogitas}\ldots{}

\emph{---Quod ego, non modo non audiam, sed etiam non videam\ldots{}}

---Ejemplo admirable de lo que llamamos \emph{climax}\ldots{}

---Como usted comprende, me enseñó el latín a machamartillo, porque,
según él, es el latín la madre de todas las enseñanzas, y única escuela
segura del buen gusto. El latín, decía, no sólo hace hombres eruditos,
sino buenos ciudadanos, personas sociables, finas y amenas\ldots{} Por
último, para que usted se haga cargo de cómo formó mi carácter aquel
gran maestro, recordaré las máximas que con tenacidad me iba
claveteando, como si dijéramos, en la cabeza, y así verá el contraste
que forma aquella enseñanza teórica con lo que después me ha traído la
realidad. «Ajusta siempre tus acciones---me decía,---a un plan lógico,
dentro de la más estricta moralidad, y no te separes de él por nada ni
por nadie. Puede que este sistema te ocasione alguna desazón pasajera;
pero a la larga apreciarás y saborearás sus hermosos resultados\ldots{}
No confíes nunca en lo imprevisto; no esperes nada del acaso, y que tu
conducta sea siempre lo que \emph{debe ser}, lo previsto, lo estudiado,
y en modo alguno dependa del \emph{qué será}\ldots{} No aceptes jamás
cosa alguna que no sepas de dónde viene, ni te fíes de prosperidades
fantásticas, que suelen volverse infortunios reales\ldots{} Lábrate la
dicha con tu trabajo, acostúmbrate a que tu bienestar sea obra de ti
mismo, y no esperes nunca favores llovidos del cielo\ldots{} No
contraigas deudas, ni aun por mínima cantidad, y advierte que es
preferible pedir una limosna a cargarte de obligaciones\ldots{} Ama la
regularidad, el orden, pues si no hay arte posible sin reglas, también
está sujeto a cánones invariables el arte de la vida\ldots{} Considera
que lo que no hayas adquirido por ti mismo no es tuyo, sino ajeno, que
si aceptas beneficios que no has ganado con tu esfuerzo, te verás ligado
por la gratitud, y la gratitud puede torcer tu voluntad, y apartarte de
la senda del deber rígido y estrictamente moral\ldots{} En lo tocante a
opiniones políticas, mantente siempre en el fiel de la balanza, y
cualquiera que sea la bandería a que te veas afiliado, no hagas un dogma
cerrado de tus creencias, ni niegues a la creencia de los demás el
respeto que merece\ldots{} Nunca te acalores en la vida pública ni en la
privada; no seas fogoso en tus pasiones, que eso es vicio romántico, de
que debes huir como de la peste; mantente siempre templado, dueño de ti,
sereno y en disposición de sortear las vehemencias ajenas. Así
dominarás, sin ser nunca dominado, porque el fiero se entrega al fin, y
se rinde al flemático\ldots{} En todos los negocios preséntate siempre
de buena fe, situándote en posición derecha, frente a las intenciones
del que ha de tratar contigo\ldots»

---Pues esta máxima---dijo Hillo gozoso,---corresponde a una de las
principales reglas del toreo, que llamamos \emph{situarse en la
rectitud}\ldots{} Adelante.

---Con que ya ve, Sr.~D. Pedro, cómo no corresponde la palpitante
realidad a la norma de conducta que mi preceptor me enseñaba; y aquí me
tiene usted sin voluntad propia, sometido a misteriosas manos que me
gobiernan\ldots{} Lo desconocido me rige, la imprevisión me guía\ldots{}
Estoy amenazado del descrédito de toda la doctrina que aprendí, y no veo
manera de aplicar ninguna regla, porque todas están por el suelo,
pisoteadas por el acaso, a quien pertenezco sin poder evitarlo.

---No es el acaso: es el supremo designio, hijo mío. Pero no te
apures---dijo D. Pedro, empezando a tutearle sin darse cuenta de ello,
por una efusión de cariño que rápidamente invadía su
corazón.---Considera que sobre todas las reglas está la realidad de la
vida, y que no podemos desviar los acontecimientos de su natural curso,
trazado por Dios. Tu padrino debió tener en cuenta el misterio de tu
origen, antes de recomendarte que abominaras de lo desconocido. ¿Por qué
no te reveló lo que sin duda sabía? O es que no sabía nada. De todos
modos, hijo mío, tu existencia se balancea en el misterio, y el misterio
ha de rodearte, y lo imprevisto te rondará por mucho tiempo, pese a toda
la ciencia y a toda la bondad de ese D. Narciso Vidaurre\ldots{} ¿Qué
resulta? Que tu padrino te quiso criar para lo clásico, sin considerar
que eres romántico inconsciente, esto es, que a pesar tuyo el
romanticismo te coge en su remolino furioso\ldots{} Dispénsame que te
tutee: siento hacia ti un profundo afecto. Te miro como un hijo; más
propio será decir como hermano. Quiero compartir tus desventuras\ldots{}
cuando lleguen\ldots{} Seamos románticos; aceptemos la realidad, y pues
esta es ahora tan buena, no le busques tres pies al gato, y date por muy
contento con los bienes que llovidos caen sobre ti. Después vendrá la
\emph{anagnórisis}, y volveremos a lo clásico, al triunfo, a la
apoteosis, que será coronamiento de tu destino. Sí, querido Fernando. Tu
porvenir es hermoso; tú eres lo que no pareces\ldots{} Serás grande,
poderoso\ldots{} Alégrate. Seremos amigos, grandes amigos; seremos
hermanos. Y ahora, chiquillo, pues cae la tarde, vámonos despacito hacia
nuestra vivienda, que la hora de la cena se aproxima, y yo, la verdad,
con todo eso que me has contado, siento que se me avivan de un modo
horroroso las ganitas de comer.

\hypertarget{ix}{%
\chapter{IX}\label{ix}}

Era verdad que D. Pedro se sentía inflamado de un cariño sincero hacia
el joven Calpena, afecto absolutamente desinteresado, pues no se
arrimaba a su amigo con intenciones de parasitismo, viéndole en camino
de doradas grandezas, sino que anhelaba guiarle por los senderos
peligrosos que probablemente se abrirían ante él; aconsejarle,
dirigirle, evitarle todos los escollos, para que gozase libre y
desembarazadamente de los bienes que el cielo le deparaba.

No tardó Utrilla en rematar algunas, si no todas las piezas de ropa de
que había tomado medidas. Dos pantalones, dos chalecos y una levita
fueron entregados a los tres días de la prueba, y la terminación de lo
demás se anunció para la semana próxima. Empezó por fin D. Fernando a
ponerse guapo y elegante, lo que con tal ropa, y los aditamentos de
corbata, calzado, peluquería, etc., era cosa muy fácil en un joven a
quien dotó la Naturaleza de airosa figura, hermoso rostro y modales
finísimos \emph{a nativitate.} Hillo le contemplaba embobado, viendo en
él un perfecto tipo de raza aristocrática. El propio Duque de Osuna, D.
Pedro Téllez Girón, no le aventajara, ni los agregados de la Embajada
inglesa.

Desde que tuvo ropa fue incitado por su amigo a frecuentar los teatros.
Hillo no le acompañaba por causa de su ministerio sacerdotal. Fea cosa
era ir a los Toros; pero más disculpable para un clérigo que el teatro,
por celebrarse las corridas en pleno día y no ser preciso en ellas
descubrirse la cabeza, exponiendo a la befa popular la ungida corona.
Con todo, buenas ganas tenía de colarse una noche en la cazuela,
disfrazado, para ver en el patio a Fernandito, y sorprender el efecto
que causaba en la concurrencia. Contentábase con verle vestirse y
acicalarse, y poner en sus manos el sombrero y bastón cuando salía.
Aunque el niño volviese tarde, D. Pedro no se acostaba hasta que le veía
entrar, y allí eran sus preguntas: «Qué tal, hijo, ¿te has divertido
mucho? ¿Has dado golpe? Apuesto a que todos los lentes, y esos anteojos
que llaman gemelos, se han dirigido a tu gallarda persona».

En el Príncipe daban \emph{Norma}, cantada por la Sra. Oreiro de Lema y
el Sr. Unanúe. En la Cruz, \emph{La joven Reina Cristina de Suecia},
traducida del francés. Así de las obras como de la ejecución, pedía el
clérigo a su amigo noticias prolijas, y el chico se las daba,
advirtiendo la absoluta ignorancia teatral del buen señor, que no había
visto nunca más pieza que \emph{El mágico de Astrakán}, allá en Zamora,
siendo él una criatura.

Menudeaba Calpena sus asistencias al Príncipe y viéndole tan aficionado,
decía D. Pedro: «¡Cómo se conoce que nos salen novias a docenas!\ldots{}
La suerte es que este chico se pasa de prudente y avisado, y no le
atrapará ninguna de esas culebronas que\ldots»

Dígase, para explicar la confusión que seguía presidiendo los destinos
de D. Fernando Calpena, que a fines de Septiembre nadie había ido a
recoger el misterioso encargo traído de Olorón; que una tarde llegó
carta anónima, no llevada por \emph{Edipo}, sino por persona desconocida
que la dejó en la puerta, y que algunas noches, al volver Fernando del
teatro, creía que le seguían dos personas buscándole las vueltas y
espiándole los pasos. La carta no traída dinero: estaba escrita por mano
nada premiosa, menudito el trazo, la gramática bastante correcta, y sólo
contenía lacónicas advertencias y admoniciones cariñosas: «Mira, niño:
los guantes amarillos son de más \emph{distinción} que los
blancos\ldots{} También te digo que no es del mejor tono aplaudir en el
teatro tan estrepitosamente, sobre todo a medianos artistas\ldots{} Por
más que tú creas otra cosa, a juzgar por tu entusiasmo, la Ridaura no
hace nada de particular en su parte de Adalgisa\ldots{} Oye, niño: que
vayas a misa al Carmen Descalzo, a las nueve en punto, y procura no
estar en la iglesia tan distraído. A la iglesia no se va a mirar a las
muchachas, sino a rezar con devoción\ldots---P. D. Cuando se te acabe el
dinero, te pones en misa la corbata escocesa, usando la negra para
anunciar que lo has recibido».

«Observaciones son estas---decía Hillo radiante de
satisfacción,---atinadísimas. Mi leal opinión es que no debes ponerte la
corbata escocesa sino cuando tengas verdadera necesidad de nuevas
remesas de metálico. No hay que abusar, hijo».

La gran sorpresa cayó, como chispa del cielo, una tarde, al volver
Méndez de su oficina. Traía un pliego de oficio dirigido a Calpena, y al
ponerlo en sus manos, le dijo: «Esta comunicación fue entregada al
portero mayor para que indagara las señas. Corrió entre nosotros de mano
en mano, hasta que vi el nombre\ldots{} ¡Qué casualidad! `¡Pero si le
tengo en mi casa!'. Ábralo usted pronto, que, si no me engaño, es
nombramiento».

Calpena se quedó frío de estupor. D. Pedro, como el que sueña despierto,
exclamó: «¡Credencial! Será cuando menos de Administrador de Tercias
Reales, o de Colector del Noveno y Medias Annatas».

Abierto el pliego, resultó contener un nombramiento de Oficial de la
Secretaría de Hacienda, con doce mil reales: firmaba \emph{Mendizábal}.
Un tanto desconcertó a Hillo el ver que la nueva dádiva, parabólicamente
arrojada por la \emph{mano oculta} sobre aquel venturoso mortal, no
correspondía, con ser grande, a las hipérboles que soñara la desbocada
fantasía del clérigo. Pero reflexionando en ello, no tardó en
conformarse y dijo: «Para hacer boca no está mal. Pocos serán los que
empiecen así. Papilla de doce mil reales no se da ni a los hijos de los
Ministros. Y aquí estoy yo, pretendiendo hace catorce años una triste
cátedra con seis mil, sin que hasta la presente\ldots{} Pero no
importa\ldots{} Con que, hijo, alégrate y toca las castañuelas, que por
lo que veo, el mundo es tuyo. Oye: que no pasen dos días sin ir a tomar
posesión y a darle las gracias al señor de Mendizábal».

Ni contento ni triste, sino fluctuando entre sus sombrías inquietudes y
el gozo retozón de su vanidad halagada, Calpena contestó que no pondría
los pies en el Ministerio sin dar antes un paso que su decoro exigía y
su ardiente curiosidad reclamaba. Empleó la mañana siguiente en la
diligencia de buscar al llamado \emph{Edipo}, lo que no le fue difícil
recorriendo oficinas y retenes policiacos; pero el tal no le dio ninguna
luz. No era más que un simple \emph{intromedario}: llevaba los mensajes
sin conocimiento de su procedencia; le llegaban de segunda mano, o sea
por órdenes de su inmediato jefe, el Sr.~D. Manuel de Azara. Sin pérdida
de tiempo echose D. Fernando a buscar a este; solicitó audiencia, que le
fue concedida, después de largos plantones, al anochecer del día
siguiente, y encontrose frente a un hombre extraordinariamente calvo y
con el bigote teñido, que le escuchó benévolo y un tanto malicioso; pero
sin dar lumbres. Aseguró que de la credencial no tenía la menor noticia,
y que de la remesa de encarguitos, así como de la preparación de
aposento, no podía revelar cosa alguna por habérsele impuesto reserva
\emph{bajo pérdida de destino}\ldots{} «Y francamente---dijo al
terminar,---no hay más remedio que defender la plaza como se pueda,
mayormente cuando a uno le tienen entre ojos por ser criado a los pechos
de D. Tadeo Ignacio Gil\ldots{} Gracias que Olózaga me considera y está
contento de mí\ldots{} En una palabra, caballerito, no me pregunte usted
nada, porque no he de responderle. Precisamente el señor Subdelegado me
estima, como he dicho, porque no hay quien me iguale en el don de
silencio. Y si me permite usted darle un consejo, le diré que aprenda
cosa tan fácil, poniéndose a ello, como es el callar. Lo difícil, señor
mío, es callarse cuando a uno le pegan; pero callarse cuando le miman y
regalan\ldots{} ¡qué cosa más fácil! Créame a mí: déjese llevar, déjese
querer\ldots»

No muy satisfecho, aunque resignado con la cómoda filosofía del
polizonte, se volvió a su casa D. Fernando, y antes de poder contar a
Hillo la reciente entrevista, recibieron ambos una nueva sorpresa: carta
del misterioso corresponsal, que decía:

«Tontín, aunque Mendizábal recuerda al jovenzuelo que le sirvió de
amanuense en el hotel \emph{Meurice}, en París, no le hables de tal cosa
cuando le veas, que le verás. No le pidas audiencia para darle las
gracias: él te llamará. Adúlale un poquito, que le gusta, y si
trabajases algún día en su despacho particular, no te muestres cansado,
aunque te tenga diez o doce horas con la pluma en la mano, que le
entusiasman los incansables, como él.

»No faltes el sábado, en el Príncipe, al estreno de \emph{Los hijos de
Eduardo}, traducido de Delavigne por el tuerto Bretón. Dicen que es cosa
buena. Y si repiten el \emph{Don Álvaro}, de Angelito Saavedra, no dejes
de ir a verlo. Ya sé que el viernes pasado estuviste en el cuarto de
Florencio Romea, donde conociste a Ventura de la Vega. Ándate con tiento
en frecuentar cuartos de cómicos: fácilmente pasarás de los cuartos de
ellos a los de ellas\ldots{} y esto no me gusta.

»Con perdón del Sr.~Utrilla, la levita verde no te ha quedado bien. Hace
unas arruguitas en la espalda, que no aumentarán la fama del primer
sastre de Madrid. Que te la vea puesta, y mándasela después para que te
la arregle. De paso te encargas un \emph{surtout} color barquillo, y que
te lo hagan pronto, que las noches ya refrescan; pero no tanto que te
pidan capa\ldots{} Los mejores guantes son los de Dubosc, y las mejores
camisas las de Fernández, calle del Príncipe. El reloj que tienes,
regalo de tu padrino, está pidiendo sucesor. Además de que es feísimo,
se atrasa que es un gusto, y así llegas tarde a todas partes. Ya veremos
de darle jubilación. Pero no lo vendas ni lo des a nadie: guárdalo
siempre como recuerdo de cuando D. Narciso te tiraba de las orejas por
no saber los latinajos.

»Bobillo, no te entretengas más de una hora en el \emph{Café Nuevo}, y
mira con quién te juntas, y a qué tertulias te arrimas. Cuidadito con
Larra, que tiene más talento que pesa; pero es mordaz y malicioso. Si
vuelves al Parnasillo, busca la amistad de Roca de Togores, de Juanito
Pezuela y de Donoso Cortés\ldots{} Con Espronceda y otros tan
arrebatados, \emph{buenos días y buenas noches}, y nada de
intimidades\ldots{} Suscríbete a \emph{La Abeja}, lee \emph{El Español},
y hazle la cruz a \emph{El Eco del Comercio}.

»Adiós. El domingo, a misa de once, en las Niñas de Leganés».

Suspiró Calpena al acabar la lectura, y D. Pedro, echando lumbre por los
ojos, dijo: «Ya no me queda duda de que es una dama. ¡Y qué cariñosa
ternura, qué purísimo y entrañable afecto!\ldots»

---Lo que yo creo---observó el joven,---es que vivo espiado dentro y
fuera de casa, pues la desconocida persona que me escribe sabe todos mis
pasos, observa las arrugas de mi ropa, y se entera de cuándo se me
atrasa el reloj.

---¿Y qué te importa, tontín? ¿Qué mayor dicha para un joven honesto que
tener quien así cariñosamente le vigile, designándole los buenos caminos
y apartándole de los atajos peligrosos? Ahora no hay que pensar sino en
presentarte en el Ministerio, tomar posesión y ponerte al habla con el
grande hombre, con ese gaditano londonense, negociante antes que
político, a quien yo tenía entre ojos; pero me va gustando, ya me va
gustando. Al darte la credencial demuestra que no es rana\ldots{} Ya ha
olido el hombre que tú vas para personaje; que cuando tengas la edad
serás Procurador, Prócer o lo que te dé la real gana, y el muy tuno
quiere atraerte con tiempo, llevarte a su lado, hacerte de su
partido\ldots{}

Meditabundo, Calpena no siguió a D. Pedro en sus apreciaciones
optimistas. Casi toda la noche la pasó en vela, asaltado de una fiebre
inquisitiva, revolviendo en su mente los claros recuerdos de su niñez,
busca por allí, husmea por allá, evocando memorias de rostros, frases o
reticencias de D. Narciso, o de alguien de su familia; mas en ningún
repliegue del pasado vislumbró hilo que le guiara por aquel laberinto en
cuyo seno misterioso se ocultaba la verdad. Tampoco Hillo durmió aquella
noche con el dulce sueño que su pura conciencia ordinariamente le
permitía. Viva excitación cerebral le tuvo en vela, y allí era el
lanzarse a un desenfrenado juego de acertijos, admitiendo y desechando
hipótesis. «Esto no lo hace más que una madre---se decía.---Y que esa
madre es persona de alta posición, no puede menos de admitirse. Bien
claro está: riquezas hay; nobleza también. No me falta más que el nombre
para llegar a la completa solución del enigma. Luego viene el otro
problema: el papá. Por San Dionisio Areopagita, esta sí que es gorda.
¡Dios mío, el padre\ldots! No sé por qué me ha dado en la nariz tufo de
sangre real\ldots{} Sí, sí. Tiene mi Fernandito en toda su persona un
sello de majestad, de grandeza de estirpe, que no deja ninguna duda, no
señor\ldots{} Por la fisonomía, nada saco en limpio\ldots{} Como
narigudo, no lo es; ni tiene el labio inferior echado para
afuera\ldots{} Por tanto, no parece\ldots»

Dormido al fin, soñó con las más estrafalarias anagnórisis que es
posible imaginar, y al amanecer despertó sobresaltado con una idea, que
en su cerebro como ladrón furtivamente se introdujo, hallándose en ese
estado neblinoso que separa el dormir del velar. «Ya, ya lo
acerté---dijo a media voz incorporándose en la cama.---Es\ldots{} de
Napoleón y de\ldots{} No será difícil descubrir una Duquesa o Marquesa
que\ldots»

Media hora después, camino del Carmen Descalzo, donde celebraba, volvía
en sí de aquella aberración, razonando de este modo: «No\ldots{} porque,
bien mirado, no tiene el tipo de los Bonapartes\ldots{} digo, me parece
a mí. Yo no he visto a ningún Bonaparte, como no sea en estampa, porque
a Napoleón I, por más que corrimos tras él los muchachos, el día
siguiente de la batalla de Astorga, no alcanzamos a verle\ldots{} no
vimos más que un bulto\ldots{} el bulto de un jinete, a lo lejos, por el
camino de Otero\ldots{} Al Rey Botellas tampoco le eché la vista
encima\ldots{} Sólo por las pinturas se hace uno cargo de la fisonomía
de aquellos señores\ldots{} No, no, esto es un delirio. Ni aun
quitándole el bigote al niño, y engordándole mentalmente, encontraríamos
el aire de familia\ldots{} ¡Qué demonio!\ldots{} esperemos, y Dios lo
dirá».

\hypertarget{x}{%
\chapter{X}\label{x}}

Uno de los primeros días de Octubre, a los veinte próximamente de su
llegada a la Corte, inauguró Calpena su vida burocrática, presentando su
credencial en la Secretaría de Hacienda (plazuela de Ministerios), y
tomando posesión de su destino. Tocole de jefe de Sección o \emph{Mesa},
un D. Eduardo Oliván e Iznardi (no tenía nada que ver con D. Alejandro
Oliván, entonces redactor de \emph{La Abeja}, ni con D. Ángel Iznardi,
redactor de \emph{El Eco del Comercio}). Hechura de D. Luis López
Ballesteros, respetado por Cea Bermúdez, y por Toreno, bien agarrado en
todos los Gabinetes por sus excelentes relaciones, era un señor bueno
como el pan, sencillo como una codorniz, afable, angosto de cerebro, y
tan ancho de conciencia burocrática, que en ella cabía, y aun sobraba
conciencia, la libertad anchurosísima de sus subordinados. Su llaneza
patriarcal parecía olvidar las jerarquías, alternando amigable y
democráticamente con los inferiores en la tarea deliciosa de leer
\emph{El Español}, \emph{El Eco} y \emph{La Abeja}, fumar cigarrillos,
repetir y comentar todo lo que en Madrid se hablaba de política y
literatura, echando de vez en cuando una plumada a los expedientes, por
vía de distracción, y sin suspender la grata tertulia. Cada cual salía y
entraba en aquella bendita oficina a la hora que mejor le cuadraba. Eran
cinco los funcionarios, con Calpena seis, repartidos en tres mesas, con
la del jefe cuatro, de distinta hechura y edad, si bien todas
representaban una antigüedad venerable. Dígase que la tinta era
excelente, hecha en la casa; las plumas de ave; los tinteros de cobre, y
que sobre las bayetas verdes y los mugrientos hules se extendían los
negros polvos de secar, formando en algunos sitios verdaderos arenales.
Inauguraba el bueno de Oliván su trabajo cortando plumas, en lo que
ponía exquisito cuidado y habilidad, pues su gala era esto y la rúbrica
que echaba en las firmas, no menos rasgueada y pintoresca que la de un
escribano. Mientras duraba el corte hablaba con los madrugadores, o sea
los que recalaban por allí de diez y media a once; les refería
incidentes o sucedidos de su familia, gracias y travesuras de sus niños;
les oía contar algo de Teatro y Toros, alguna mujeril aventura, y así se
pasaba el tiempo hasta las doce, hora en que le traían a Don Eduardo su
almuerzo. Sobre las bayetas arenosas extendía una servilleta, y se comía
su tortilla de patatas y su chuletita de ternera. Salían y entraban los
mozos de café con servicios para el jefe y algunos subalternos, y en
tanto, el que no tomaba café, hacía caricaturas; otro escribía versos, y
el de la última mesa las cartas a su novia. Luego se trabajaba un
poquito, mientras uno leía en voz alta \emph{El Español}, para que los
demás se enterasen. El jefe solía pasarse a la Sección próxima, donde
había otro jefe que \emph{veía largo} en política, y anunciaba con
seguro vaticinio todo lo que iba a pasar. Más tarde descansaban, fumando
un cigarrillo. D. Eduardo recibía cortésmente a las personas que acudían
al despacho de algún asunto, y para hacerles ver la actividad que allí
se desplegaba, les ponía ante los ojos rimeros de papeles que debían
pasar pronto a la Sección correspondiente, y otros rimeros de papeles
que acababan de llegar, después de lo cual les prometía no detener los
expedientes más que el tiempo necesario para \emph{el concienzudo examen
de los mismos}. Luego se limpiaba el sudor de la calva, y contaba a sus
subalternos lo que el otro jefe de Sección le había dicho: que todo iba
muy bien; que la quinta de cien mil hombres daría un resultado
maravilloso, y que no había duda de que Istúriz y Galiano apoyarían
incondicionalmente al Sr.~Mendizábal en el Estamento próximo. No se
podían dar las mismas seguridades de López y Caballero, y Toreno y
Martínez de la Rosa no saldrían de su pasito \emph{moderado}. Había,
pues, situación Mendizábal para un rato, y se verían realizadas las
reformas que el grande hombre había prometido en su famosa exposición a
la Reina. Pero la noticia culminante era que la Milicia urbana se
reorganizaría, tomando el nombre sonoro y magnífico de Guardia Nacional.
«\emph{Todo será a estilo de Francia}---concluía D. Eduardo;---y lo
mejor es que a los milicianos de Madrid y su provincia se nos da
carácter de ejército regular, formando con nosotros una división mandada
por un Jefe superior, y bajo la inspección de un General\ldots{} Por eso
ha dicho San Miguel que seremos el ángel custodio de las instituciones.»

No siempre hablaba de lo mismo, aunque era muy dado a la repetición de
conceptos, vicio que los retóricos llaman \emph{batología}. «¿No saben?
Se suprimen las \emph{cartas de seguridad}, esa rémora, señores, para la
gente honrada que tiene que viajar de un punto a otro. Yo soy partidario
de que se \emph{corten abusos}. Los que han viajado por el extranjero
nos dicen que estamos en el siglo XV, y francamente, yo quiero
pertenecer a \emph{mi siglo}\ldots{} Seamos todos de nuestro siglo,
entrando por el aro de las grandes reformas\ldots{} Otra de las buenas
noticias es que se suprimen \emph{las pruebas de nobleza} para ingresar
en los establecimientos científicos, ora civiles, ora militares\ldots{}
Realmente, semejante ranciedad era un resabio de la Edad Media. Ábrase
la enseñanza para todo el mundo y dese al mérito ancho campo. ¡Abajo la
Edad Media!\ldots{} Créanlo ustedes, en este particular estoy de acuerdo
con Caballero y los de \emph{El Eco}; nada más que en este particular,
pues opino, como él, que la \emph{demo\ldots cracia}, así se dice, la
\emph{democracia} exige que el pueblo se ilustre. Yo soy partidario de
la ilustración del pueblo, como soy partidario de que el pueblo sea
moral, y de que los empleados trabajen\ldots{} Mi sistema es: pocos
empleados, pocos, pero bien pagados.»

Dichas estas cosas, y otras de igual transcendencia y filosofía, el jefe
bromeaba un poco con sus subordinados: con éste por si la novia le daba
calabazas; con aquél por si era alabardero en los teatros; con el otro
por si le sudaban tanto las manos, que toda la arenilla se le quedaba
pegada en ellas, y obligaba a \emph{la casa} a frecuentes reposiciones
de aquel material. Luego les recomendaba benévola y paternalmente que no
dejasen el papelorio esparcido sobre las mesas, y él mismo daba el
ejemplo recogiendo legajos y metiéndolos en una alacena donde tenía
botellas vacías o medio llenas, el \emph{Diccionario geográfico} de
Miñano, confundidos sus tomos con los de novelas y viajes, entre estos
el de \emph{Enrique Walson al país de las Monas}. «Yo soy
partidario---decía,---de que haya orden en las oficinas, para que el
trabajo se haga como Dios manda, y cada cual encuentre lo que necesita
para el pronto despacho de los asuntos\ldots» Con esto se aproximaba la
hora feliz de poner punto en las faenas del día: los sombreros parecían
alegrarse en lo alto de las perchas, viendo próximo el instante de que
sus dueños lo cogieran para echarse a la calle. «Vaya, ya es hora,
ciudadanos---decía D. Eduardo, atusándose los mechones laterales, y
cubriéndose con pausa y solemnidad, como si su calva fuese una cosa
sagrada que reclamaba el respeto de la protección sombreril.---Me parece
que hemos trabajado bastante. Hasta mañana.»

Si la tarde era plácida, se iban de paseo, y si lloviznaba o hacía frío,
al café, donde con charla sabrosa de literatura, de política o de cosas
mundanas, reducían a polvo el tiempo hasta la hora de cenar. Que Calpena
se aburría en la oficina, no hay para qué decirlo. Desde su iniciación
burocrática no había hecho más que extender algunos oficios y copiar dos
o tres estados de recaudaciones. El jefe le consideraba, presumiendo en
él una superioridad aún no bien manifiesta, pero que lo sería pronto; y
los compañeros le mostraron afecto y fraternidad, más admirados que
envidiosos de su buena ropa. Ya era cosa corriente en las oficinas ver
entrar niños bonitos, con sueldos desmesurados, y que no iban más que a
cobrar y a distraerse un rato; hijos o sobrinos de personajes, que de
este modo arrimaban una o más bocas de la familia a las1 ubres del
presupuesto. Los empleados, que lo eran por oficio y medio de vivir, se
habían acostumbrado a la irrupción de señoritos, y alternaban gozosos
con ellos, esperando hacer amistades que en su día valieran para el
ascenso, o para la reposición en caso de cesantía. En la Sección de
Calpena todos los funcionarios eran de peor pelaje que él: alguno pasaba
de los cincuenta años y sólo disfrutaba ocho mil reales, vestía ropa
vuelta del revés y apenas paseaba, por no romper botas; otros
conservaban aún trajes provincianos, estirándolos cuanto podían, y no
faltaba quien vistiese regularmente por el sistema económico de no pagar
al sastre. Sobre todos descollaba Calpena, no sólo por su elegancia y
buena figura, sino por su saber de cosas extranjeras, y su rumbosa
generosidad en el pago de cafés y refrescos después de la oficina. Con
uno de sus colegas, extremeño, envejecido prematuramente y seco como un
esparto, habitante en una casa de huéspedes de ínfima categoría,
parroquiano fósil de diferentes cafés, hizo amistades, seducido por la
sabrosa erudición que ostentaba en cosas y personas de Madrid. Muchas
tardes iba con él al Nuevo, y se le pasaban mansamente las horas
oyéndole contar anécdotas que parecían mentira siendo verdades, y
embustes que resultaban perfecto simulacro de la verdad. Por Serrano
(que así se llamaba) supo Calpena que su jefe, D. Eduardo Oliván, era un
hombre desgraciadísimo en su vida doméstica, aunque no conocía, o
aparentaba no conocer su propia desgracia. La paz que en su hogar
reinaba era la proyección de su mansedumbre, virtud con la cual
adquirido había una triste celebridad. Ponderó Serrano la seductora
hermosura de la mujer del jefe, y algo dijo también de su familia, muy
conocida en Madrid. Se la veía muy a menudo en teatros y paseos,
fingiendo una posición que no tenía, alternando con personas cuya
riqueza consistía en bienes raíces, o en rentas que estaban a la vista
de todo el mundo. Las de aquella buena señora eran un tanto enigmáticas.
«Si quiere usted más detalles, pídaselos al hoy General en Jefe del
ejercito del Norte, D. Luis Fernández de Córdoba. Los sucesores de este
son de menor categoría militar y civil. El último que ha caído en las
redes de nuestra \emph{jefa} es ese capitán de artillería\ldots{}
Escosura, Patricio de la Escosura\ldots{} ¿No le conoce usted? De seguro
que sí. En el Príncipe le tiene usted todas las noches. Es el que
retrató Bretón en el \emph{D. Martín} de la Marcela.»

---No sabía que los tres amantes de Marcela fueran retratos.

---Bien se ve que no está usted aún familiarizado con nuestra
sociedad\ldots{} Pues el \emph{Don Amadeo} es Pezuela, y el \emph{D.
Agapito} el chico de Clemencín.

\hypertarget{xi}{%
\chapter{XI}\label{xi}}

---Una de estas noches, amigo Serrano---dijo D. Fernando,---va usted a
venir conmigo al Príncipe, para que me diga los nombres de todas las
señoras que veamos en los palcos. En el tiempo que llevo aquí, he hecho
algunas amistades, pocas; hace unas noches me llevaron al cuarto de
Florencio Romea; en el teatro he conocido a Ventura de la Vega y a
Mesonero Romanos. El señor a quien debo este conocimiento me le presentó
días pasados en la calle de Alcalá mi compañero de casa D. Nicomedes
Iglesias. ¿Le trata usted?

---¿Cómo no?\ldots{} Iglesias\ldots{} hombre de mucho talento, de gran
porvenir\ldots{}

---Pues me presentó a ese\ldots{} ¿cómo se llama? Alonso\ldots{} Juan
Bautista Alonso, con quien me encontré después una noche en la segunda
fila de lunetas, y charlamos algo de literatura. Por él he conocido a
Vega, he hablado con Larra, y he saludado a Espronceda en el café Nuevo
y en el Parnasillo\ldots{}

---Alonso es poeta y un buen periodista\ldots{} chico que vale. Será
ministro\ldots{} ¿Y no ha querido catequizarle a usted para la sociedad
\emph{Los Numantinos}?

---A mí no\ldots{} Ni yo gusto de meterme en esas cosas, ni la vida
política me seduce.

---A mí\ldots{} sí\ldots{} pero no puedo consagrarme a ella, por\ldots{}

Acometido de una tos violentísima, parecía que se ahogaba. Amoratado y
convulso, faltábale poco para echar los bofes y escupir el alma. «Con
esta maldita tos---dijo cuando se fue sosegando, y se limpiaba de babas,
mocos y lágrimas el encendido rostro,---¿cómo quiere usted que sea uno
político y orador?\ldots{} Mi naturaleza es émula de mi bolsillo en el
agotamiento, en la extenuación\ldots{} No me forjo ilusiones de vivir el
año que viene: estoy tísico pasado.»

Trató de consolarle Calpena, con más lástima que convencimiento, porque
en verdad la flaqueza y el color cadavérico de su amigo invitaban a
entonar el responso. No espantado de la muerte, o echándoselas de
valiente, hablaba Serrano de su próximo fin con entereza estoica un
poquito afectada. Era moda entonces morirse en la flor de la edad,
tomando posturas de fúnebre elegancia. Habíamos convenido en que
seríamos más bellos cuanto más demacrados, y entre las distintas
vanidades de aquel tiempo no era la más floja la de un fallecimiento
poético, seguido de inhumación al pie de un ciprés de verdinegro y
puntiagudo ramaje. «Estos pobres huesos---prosiguió Serrano,---están
pidiendo la mortaja. Le diré a usted, en confianza, que es de tanto
sufrir y de tanto gozar\ldots{} Mi vida, si yo la contara, sería la más
interesante de las novelas. Mis años, por el mucho y precipitado vivir,
parecen siglos\ldots{} ¡Y que llegue uno al borde de la tumba con ocho
mil reales!\ldots{} En fin, doblemos la hoja triste\ldots{} ¿Me decía
usted que desea ir conmigo al teatro para que le dé a conocer a todo el
personal masculino y femenino que veamos en palcos y butacas? No podía
usted encontrar, ni buscándola con candil, persona más para el caso,
porque como de algún tiempo acá no tengo nada que hacer (en la oficina
ya ve lo que trabajamos), me dedico a conocer \emph{de visu} a todo el
mundo y a la averiguación de vidas ajenas\ldots{} Soy un Plutarco para
esto de las vidas, y las hago también paralelas. Sabrá usted los nombres
y las historias, amigo mío, que aquí no hay nadie que no tenga su
historia\ldots{} y las hay de oro. ¡Con decirle a usted que la de
nuestro esclarecido jefe es de las más inocentes\ldots!»

---¡Caramba!

---¿Y lo duda? ¿De qué dehesa viene usted?

---¿Dónde hay más historias, en las clases altas o en las medias?

---En todas; pero las de las altas son más bonitas, más profundamente
depravadas. Yo las conozco al dedillo, y en pocas noches le daré la
instrucción suficiente para que no pase por cándido el día que se
introduzca en la sociedad.

---¿Pero no se exime nadie, galán ni dama, del oprobio de esas
historias? ¡Por Dios, Serrano\ldots!

---Nadie\ldots{} Todo el mundo tiene historia. Por lo común no hay
persona bien vestida que no lleve consigo su misterio: este misterio es
algo que no debe saberse, y, sin embargo, se sabe, porque fíjese
usted\ldots{} Nada es aquí tan público como las cosas secretas\ldots{}
En fin, por tener todo el mundo historia, hasta usted la tiene, usted,
querido Calpena, que acaba de llegar a Madrid; y antes de dar los
primeros pasos en las tablas del teatro social, ya nos indica que trae
buen papel en la comedia.

---¡Yo!---exclamó Calpena palideciendo.---¡Pobre de mí! ¡Si no soy
nadie!

---Los que empiezan no siendo nada, suelen acabar siéndolo todo.

---Bueno. Pues si alrededor mío hay una historia y usted la sabe, amigo
Serrano, ¿tendría inconveniente en contármela?

---Inconveniente, ninguno\ldots{} pero la tos\ldots{} ya ve\ldots{} no
puedo hablar\ldots{} me ahogo\ldots{}

Aguardó Calpena a que el golpe de tos se calmase, y cuando hubo pasado,
aún tuvo que esperar más tiempo, porque el infeliz tísico se quedó un
rato sin respiración, los ojos inyectados, la frente sudorosa, las manos
trémulas\ldots{}

---Pues sí\ldots{} esta maldita tos no me deja vivir\ldots{} Si yo no
tosiera, sería orador, créame usted\ldots{} Pues no hay que tomar a mala
parte esto de las historias. ¡Tan joven y ya protagonista! Si he de ser
franco, no puedo aún decir a usted cosas concretas\ldots{}

---¿Pues no asegura que lo sabe todo?

---Todo no. Es muy pronto todavía, y aún son pocas las personas que se
han fijado en el joven Calpena\ldots{} Lo que yo he oído no es ofensivo
para usted, ni mucho menos.

---Sea lo que quiera, debo saberlo.

---La tos otra vez\ldots{} Me ahogo\ldots{}

---¡Demonio! ¿Por qué no toma usted pastillas? Yo se las traeré de la
botica más próxima.

---No\ldots{} gracias\ldots{} Es inútil. Las he tomado de todas clases,
sin sentir el menor alivio.

---Ya pasa\ldots{} ya puede hablar.

---La verdad, amigo mío, a usted se le tiene en estudio. Sólo he oído
formular preguntas, aventurar alguna hipótesis\ldots{} Conjeturas,
presunciones\ldots{} qué será, qué no será\ldots{}

---¿Nada más que eso? Pues soy, respecto a mí, el primero de los
curiosos investigadores, y yo pregunto también: «¿quién soy?\ldots{}
Calpena ¿quién eres?»

---¿Pero usted no lo sabe?\ldots{}

Comprendiendo que había ido demasiado lejos en la expresión de sus
dudas, D. Fernando se enmendó diciendo: «Sé quién soy; pero en la vida
de todo hombre, por clara que aparezca, hay siempre incógnitas que
resolver.»

---¿De modo que no sabe usted todo lo que le concierne?

---Hombre, todo, todo precisamente, no.

---Pero sí sabrá quién le recomendó para la plaza que hoy ocupa en el
Ministerio.

---Juro a usted que lo ignoro.

---Las recomendaciones toman en este país giros muy extraños, y ofrecen
a veces concomitancias increíbles. A mí, para que me dieran la plaza
mísera que tengo, me recomendó la persona más opuesta a mis ideas, D.
Antonio Zarco del Valle, a quien interesé por el ama de cría de uno de
sus niños. Por un empleado del personal he sabido que en el libro donde
constan los padrinos de cada empleado, figura usted como hechura y
ahijado del propio Mendizábal, lo que nadie extrañará, porque bien
podría el Ministro ser amigo, deudo de su familia de usted.

---No lo es. Ese señor no tiene ningún motivo para interesarse por mí.

---En tal caso habrá recibido cartas expresivas de personas a quienes no
puede negar un favor de esta clase. Por indiscreción de un amigo de la
secretaría particular, puedo\ldots{} no afirmar, ¡cuidado!, sino
sospechar\ldots{} con vehementes indicios de acierto\ldots{}

Sobresaltado y ansioso, aguardaba el otro la terminación del concepto.
Un amago de tos determinó pausa expectante, que a Calpena le pareció un
siglo. Por dicha, no fue más que amago, y Serrano pudo decir claramente:
«Si se empeña usted en oírme lo que sabe\ldots{} ¡vaya si lo
sabe!\ldots{} le diré que debe su plaza a la Duquesa de Berry\ldots»

Pausa\ldots{} Sólo se oía el áspero ronquido que salía del pecho de
Serrano. El estupor de Calpena acabó por resolverse en una risa
nerviosa, que lo mismo podía ser de regocijo que de burla.

«¡La Duquesa de Berry!\ldots{} ¿Está usted loco? ¿La esposa del Príncipe
asesinado a la salida de la Ópera, hijo de Carlos X\ldots?»

---Justo\ldots{} Carolina de Nápoles, hermana de nuestra Reina
Gobernadora Doña María Cristina.

---¿Y esa señora es la que figura como\ldots?

---No figura en el libro de recomendaciones; pero por referencias, por
indicios de secretaría, sé yo\ldots{}

---¡Locura, delirio!---exclamó Calpena levantándose, como hombre que
quiere poner fin por la ausencia a una conversación enfadosa.

---Si usted me probara eso\ldots---indicó Fernando, fingiendo
indiferencia.

---¿Prueba?\ldots{} ¡Oh!\ldots{} Me remito al gran demostrador de
verdades, el tiempo\ldots{}

---Pero ¿cómo es posible\ldots? ¿Qué tiene que ver mi humilde persona
con esa princesa\ldots?

Serrano alzó los hombros, quiso decir algo; pero, ahogándose, no hizo
más que balbucir: «No puedo. La tos, la tos\ldots»

\hypertarget{xii}{%
\chapter{XII}\label{xii}}

La placentera holganza en que vivían los individuos de la sección o mesa
de que era jefe el Sr.~D. Eduardo Oliván e Iznardi tuvo su término, que
si no hay mal que cien años dure, tampoco los bienes suelen ser
duraderos, y el motivo de tan brusca alteración, que produjo enorme
desquiciamiento en la anecdótica parsimonia del jefe, no fue otro que el
haberse manifestado en aquella esfera administrativa el impulso de
actividad que imprimió Mendizábal a los asuntos de su Ministerio, cuando
se desembarazó de las graves cuestiones políticas a que en los primeros
días tuvo que atender. Desempeñando interinamente, además de la cartera
de Hacienda, con la Presidencia, las de Guerra, Marina y Estado, hubo de
promiscuar en el despacho de mil negocios diferentes. Por milagro de
Dios no se volvió loco el bueno de D. Juan Álvarez, que materia ofrecía
cualquiera de aquellas oficinas para trastornar el seso del más pintado
en tiempos tan revueltos. Confiado ya en dominar la espantosa anarquía
de las Juntas que convertían el Reino en una inmensa jaula de locos;
seguro ya del éxito de la quinta de cien mil hombres, arriesgado acto de
Gobierno que revelaba iniciativa poderosa y voluntad de acero, se metió
en su casa propia, Hacienda, y empezó a remover y sacudir, con mano de
atleta, las mohosas inercias de la administración heredada de Fernando
VII. ¡Lástima que no lo hiciera con más pulso, para que las ruinas y los
escombros no embarazaran la obra nueva! Construía con el hacha\ldots{}
Aunque no carecía de habilidad, no pudo evitar el cortarse las manos con
la herramienta que tan presuroso manejaba.

Pues, señor\ldots{} obligado el pobre D. Eduardo a andar de coronilla,
no sabía lo que le pasaba, ni a qué santo encomendarse. En toda su vida
burocrática, que con intercadencias databa de los tiempos de
Ballesteros, no había visto desencadenarse sobre aquella plácida esfera
un ciclón tan duro. No hacía más que ir de una mesa a otra, limpiarse
con fuertes restregones el sudor de la calva, dar resoplidos, subirse el
pantalón, que con tantas ansiedades se le caía. Y una mañana, medio loco
ya, o loco entero, gritaba en medio de la oficina: «Pero este buen señor
nos trata como si fuéramos dependientes de comercio. La dignidad del
funcionario público no consiente estos excesos de trabajo, pues ni
tiempo le dejan a uno para almorzar, ni para dar un \emph{mero} paseo,
ni para encender un mero cigarrillo\ldots{} Cinco intendencias me ha
señalado hoy para el envío de circulares con las instrucciones
reservadas y las nuevas tarifas. Pues para despachar esto, excelentísimo
señor, necesito aumento de personal, necesito catorce oficiales y ocho
auxiliares, y aun así, no podríamos concluirlo dentro de las horas
reglamentarias, que son de diez a cuatro\ldots{} Sería justo además que
al exceso de ocupación correspondiera doble paga mientras durase este
ajetreo. Soy partidario de que a los empleados se les remunere bien,
pues de otro modo la buena administración no es más que un mito, un
verdadero mito.»

Y aquella misma tarde, en el colmo ya del mal humor, que expresaba
alargando los morros, entró en la Sección próxima, diciendo: «Pido al
señor Ministro aumento de personal, ¿y qué hace? Nada: que aún le parece
mucho lo que tengo, y me pide dos chicos que escriban bien y sepan
llevar correspondencia. Estamos lucidos, como hay Dios\ldots{} Ea,
Sr.~Calpena, pase usted a la secretaría particular del señor Ministro; y
usted, Serrano\ldots{} Pero no\ldots{} aguardaremos a ver si se contenta
con uno\ldots{} quédese usted\ldots{} Esto es insufrible. Yo digo que
envidio a los presidiarios\ldots»

Pasó Calpena a donde se le mandaba, y fue introducido en una habitación
pequeña con luces al patio medianero, en la cual había dos mesas y un
solo empleado, viejo, que escribía con la cara tocando al papel. Un
estrecho pasillo comunicaba la tal pieza con el despacho del Ministro.
Allí esperó órdenes. Alzó el viejo la cabeza, y levantándose las
antiparras a la frente, le miró, hizo un saludo monosilábico, volvió a
bajar los vidrios, y dejó nuevamente caer sobre el papel su rostro.
Creeríase que no escribía con la pluma, sino con la nariz\ldots{} Sonó
la campanilla. Levantose el vejete de un brinco, murmurando: «Su
Excelencia llama.» Viéndole desaparecer por el pasillo, advirtió Calpena
que cojeaba. Un instante después volvió con varias cartas en la mano, y
dijo lacónicamente a su compañero: «Que pase usted.»

Grande fue la emoción del joven al atravesar el pasillo, al levantar la
cortina y ver el hueco de la estancia\ldots{} a Mendizábal no le veía.
Quedose en la puerta hasta oír la palabra \emph{adelante}, dicha con
enérgica entonación. Estaba el grande hombre sentado, y se inclinaba
para sacar papeles de la gaveta más baja de su mesa ministerial. Al
incorporarse, presentó a la admiración y al respeto de Calpena su
hermoso busto, el rostro grave de correctísimas facciones, el rizado
cabello, las patillas tan bien encajadas en los cuellos blancos, y estos
en el lioso tafetán de la negra corbata reluciente, las altas solapas de
la levita, y por fin, al ponerse en pie, esta en toda su longitud,
ceñida y al propio tiempo holgada.

Calpena permaneció inmóvil y mudo, estatua de la cortedad respetuosa.
Mendizábal le miró\ldots{} En la extrañísima situación de espíritu en
que el buen chico se encontraba hubo de creer que su jefe le miraba con
picardía. Pero es casi seguro que era pura aprensión; al menos, así lo
creyó después. Contra lo que pensaba, ni le preguntó el Ministro su
nombre, sin duda porque lo sabía, ni sostuvo con él diálogo de
introducción. Entre personaje tan elevado y un pobre subalterno de
ínfima categoría, no podían mediar más palabras que las naturales entre
el señor y el criado que le sirve. Estas fueron corteses, ceñidas al
asunto, y sin fraseología ociosa: «Tiene usted hermosa letra, y buen
criterio para contestar por sí mismo las cartas, con una simple
indicación mía.»

El joven se inclinó. Cuando D. Juan de Dios avanzó hacia él, ostentando
la gallardía total de su persona, su alta estatura, Calpena, que ya
había admirado el busto, admiró también el pantalón, de corte perfecto,
como de sastrería londonense, y el pie pequeño, calzado con zapato bajo
sujeto en el empeine con un lazo de cintas negras.

«Contésteme usted, por de pronto---prosiguió Su Excelencia,---estas tres
cartas. La más urgente y delicada es\ldots»

No encontrando la que llamó delicada y urgente, la buscó en la mesa,
después en el bolsillo interior de la levita, y como allí no pareciera,
manifestó disgusto. «Está bueno. Pues me la he dejado en casa\ldots{}
Pero no importa. Escríbame usted la contestación, que es
sencillísima\ldots{} del tenor siguiente: `Serenísima Señora Duquesa de
Berry. Señora: Tengo el gusto de manifestar a Vuestra Alteza que
obediente a sus ruegos\ldots{} que son órdenes para mí\ldots{}'. Ya
usted comprende\ldots{} una fórmula de gran respeto\ldots{} `que
obediente\ldots{} y tal\ldots{} me he apresurado a complacer, y tal, a
Vuestra Alteza Serenísima en la petición con que se ha dignado
honrarme\ldots{} y tal\ldots{}'. Nada más\ldots{} Ah, sí\ldots{} `Debo
manifestar a Vuestra Alteza Serenísima que el joven\ldots{}'. No, nada
de joven\ldots{} `Que la persona\ldots{} y tal, que se digna
recomendarme es\ldots{}'. No, no\ldots{} `He tomado informes, y puedo
asegurar a Vuestra Alteza que el sujeto, etcétera\ldots{} es digno de la
protección de persona tan elevada\ldots{}'. Así, poco más o menos. Vea
usted cómo sale del paso. Puede tomar nota.»

---No necesito tomar nota. Recuerdo perfectamente las indicaciones de
Vuecencia.

---Mejor. Así me gustan a mí los hombres, vivos de memoria\ldots{} Pues
escríbame la carta al momento y tráigamela para firmarla.

Hizo Calpena la reverencia, se fue a su oficina y mesa, y tanteando la
difícil materia epistolar en un borrador, escribió la carta, esmerándose
en los trazos de su hermosa letra, y la llevó al Ministro. Este había
pasado al salón próximo, donde tenía como unas veinte visitas, y
mientras Calpena esperaba, entró también su compañero, el viejo de las
antiparras, que por primera vez le dirigió la palabra en forma
afectuosa. «Ahora tiene para rato---dijo, refiriéndose al Ministro.---Le
traen loco con esto de las elecciones. Para cada puesto del Estamento
hay setenta candidatos\ldots»

---Ya, ya\ldots{}

---¿Y usted, Sr.~Calpena, se presenta para Procurador?

---¡Yo! ¡Procurador yo!---exclamó Fernando con asombro, casi con miedo.

---¿No? Pues yo no lo he inventado. En la casa se ha dicho\ldots{} y
hasta me parece que oí nombrar la provincia\ldots{}

---Creo que está usted equivocado\ldots{}

---Podrá ser\ldots{} ¡Pero cuando lo dicen por algo será! Vea el
Sr.~Calpena como de mí no se dice nada.

---¿Qué sueldo tiene usted?

---¿Yo? Diez mil, y para eso llevo veintidós años en el ramo. He pasado
por catorce intendencias, he sufrido siete cesantías, y todas las
trifulcas que hemos tenido aquí desde el año 14 me han cogido de medio a
medio. En una me dejaron cojo los liberales, en otra me abrieron la
cabeza los realistas, en esta me apalearon los exaltados, en aquella me
despojaron los apostólicos de todo cuanto tenía. Vive uno por casualidad
en esta tierra, y, sin embargo, la quiere uno\ldots{} pues, como se
quiere a una mala madre\ldots{} Yo soy gaditano, o lo que es lo mismo,
de Chiclana, y por tener algún parentesco lejano con los Méndez y
amistad con los Bertrán de Lis, no me ve usted pidiendo limosna. Soy muy
corto. Aquí sólo hacen carrera los parlanchines, y yo, aunque andaluz,
me callo muy buenas cosas y no tengo el despotrique que ahora se usa.
Sea usted bullanguero, piense como un topo y charle como una cotorra, y
verá cómo se le abren todos los caminos\ldots{} Lo mejor es que siempre
será lo mismo, y no veo yo mejores días para la España. Este grande
hombre, que ha venido como el Mesías, trae mucha sal en la mollera, y el
firme propósito de hacer aquí una regeneración\ldots{} vamos, para que
nos envidien todas las naciones. Pues verá usted cómo no hace nada. ¿Por
qué? Porque no le dejan\ldots{} Ya le están armando la zancadilla. Crea
usted que antes que tenga tiempo de cumplir lo que ha ofrecido, se le
meriendan\ldots{} Ya empiezan a decir si en Palacio gusta o no gusta. Y
es la de siempre: Palacio\ldots{}

En este punto entró Mendizábal acompañado de un sujeto con quien hablaba
vivamente y en tono áspero.

«Esto no puede ser\ldots{} Yo he dicho a todos los Subdelegados que
dejen votar libremente, y que no intervengan en las elecciones. Claro es
que siempre tiene el Gobierno la influencia moral. Pero en Cádiz no
puedo hacer nada. Galiano y el amigo Istúriz son los que manejan el
tinglado de la elección. Por cierto que Istúriz quiere traer algunos que
no conoce nadie. ¿Quién es ese Luis González?»

---Es un chico muy despierto, buen periodista, orador fogoso. No creo
que salga por esta vez.

---Pues si en Cádiz no logra usted meter a su patrocinado, intente algo
en Sevilla. Pero tampoco podrá ser. Ya tengo noticia de los candidatos
probables\ldots{} No les conozco. Hablan con gran encomio de un tal
Cortina\ldots{} Y ese Pacheco, ¿quién es?

---Un escritor notabilísimo: le tengo en mi periódico.

---Bueno, bueno. Tráiganme gente de mérito, segura en sus principios, y
que no se asuste de la libertad\ldots{} Pues decía que procure usted
entenderse con los sevillanos. Yo no puedo hacer nada, amigo mío,
absolutamente nada.

---Mi patrocinado es aquel joven que usted mismo ha elogiado con tanta
justicia, por su actividad, por su inteligencia, en la Secretaría de
Marina.

---Montes de Oca, sí\ldots{} excelente sujeto. Tendría yo mucho gusto en
traerle al Estamento\ldots{} Pero no soy yo quien elige: es el Pueblo.
Vea usted a los gaditanos; entiéndase con Istúriz, que, por lo visto, no
se para en barras, y\ldots{}

Una mirada que dirigió el Ministro a los dos empleados de su secretaría
particular bastó para que estos se retirasen.

«¿Quién es ése\ldots?» preguntó Calpena a su compañero, a lo largo del
pasillo.

---Este es Borrego\ldots{} Andrés Borrego, el que escribe \emph{El
Español}. Dejemos a estos compadres que manipulen a su gusto las nuevas
Cortes, y aguardemos aquí, charlando, a que D. Juan nos llame. Como le
decía a usted\ldots{} ya le están minando el terreno a mi paisano; y
aunque vale mucho, no le salvarán su talento y buena intención, y si le
salvaran, creería yo en lo que no creo: en mi propio nombre.

---¿Cómo se llama usted?

---Me llamo Milagro---dijo el vejete sonriendo,---José del Milagro. Ya
ve usted si es alegórico mi apellido, pues verdaderamente no hay mayor
prodigio que vivir un hombre entre tantas desventuras, cesante cuando no
perseguido, y andando para atrás en mi carrera como los cangrejos, pues
yo empecé a servir con el Sr. Urquijo y el Sr.~Cabarrús\ldots{} Vengo de
Carlos IV, pasando por Pepe Botellas\ldots{} y en los tres
\emph{llamados años}, llegué a tener catorce mil, gracias al Sr.
Garelly. A la muerte del Rey, conseguí por el señor Seoane esta
placita\ldots{} Y usted dirá que el mayor milagro mío es mantener, con
tan poco sueldo, mujer, suegra y cinco criaturas\ldots{} Hay
Providencia. Yo me defiendo con las traducciones; traduciendo a destajo,
visto y calzo a la familia. Y ha de saber usted que entre tantos males,
Dios me ha dado una hija que es un ángel. Diez y seis años cumplirá el
14 de Noviembre. Rafaela se llama: me la sacó de pila mi amigo Rafael
del Riego, hallándose de guarnición en la Isla. Pues la he enseñado el
francés, y me ayuda. Como me estoy quedando ciego del mucho trabajar,
ella sola, solita, se ha traducido más de la mitad del
\emph{Buffon}\ldots{} A más de esto, tengo el recurso de llevar la
correspondencia en algunas casas de comercio, y principalmente en la de
doña Jacoba\ldots{}

Este nombre hirió con súbito rayo la mente de Calpena, y pidiendo más
explicaciones, oyó de boca de Milagro las siguientes: «Doña Jacoba
Zahón, que compra y vende piedras preciosas\ldots{} Calle de
Milaneses\ldots{} Yo le escribo las cartas y le pongo sus cuentas en
orden\ldots»

Campanillazo. Su Excelencia llamaba, y acudieron ambos presurosos. Pidió
las cartas escritas; sonrió; leyó detenidamente la de la Duquesa de
Berry, y sin mirar a Calpena, le dijo: «Está muy bien.» Después,
abrumado de quehaceres, y no sabiendo a cuál acudir primero, dio estas
atropelladas órdenes: «Usted, Milagro, ponga una carta a Alcalá Galiano,
citándole para esta noche aquí\ldots{} Y otra, lo mismo, a Saavedra (D.
Ángel). Usted, Calpena, escriba una a la Duquesa de Almodóvar,
diciéndole que no puedo ir a comer, y tráiganmelas para firmar\ldots{}
¡Ah!, espere usted: otra a Sir George Williers, Embajador de Inglaterra:
Que mis ocupaciones no me permitieron ir anoche a casa de Van-Halen,
como le prometí; que si tiene esta noche libre, se venga por aquí a las
once\ldots{} Usted, Milagro, en una carta breve, cíteme a Olózaga para
las doce, y también a\ldots{} No, no, nadie más.»

En aquel momento anunció el portero: «El Sr.~D. Fernando Muñoz\ldots»

---Que pase inmediatamente\ldots{}

Retiráronse los secretarios, y por el pasillo cuchicheaban:
«Muñoz\ldots{} es la primera vez que viene aquí\ldots{} Muñoz\ldots{}
\emph{el marido del Ama}\ldots»

\hypertarget{xiii}{%
\chapter{XIII}\label{xiii}}

Al quedarse solo, Mendizábal escribió una carta de cuatro pliegos a
Córdoba, General en Jefe del ejército del Norte. Con nerviosa mano, sin
cuidarse de la estructura gramatical, trazaba los conceptos, en algunos
puntos ampulosos, pedestres en otros, fiel imagen de su pensamiento, que
empezaba a ser desordenado y vacilante por el cansancio de la tremenda
lucha. Anhelaba mostrarse amigo del que en su mano tenía la mayor fuerza
existente en España, estar en su gracia, pues tomado el pulso al país y
a la raza, si mucho temía D. Juan del paisanaje de levita y chaqueta,
más temía de la tropa\ldots{} Aunque aplicar quiso toda su atención a la
escritura, no lo lograba: el pensamiento se dividía, fluctuaba, y
dejando a la pluma formular con incorrecta sintaxis los conceptos
epistolares, se escabullía por otros espacios. Trajo el ministro a su
imaginación la historia de los últimos años, desde el 14, y veía las
trifulcas, los sangrientos y bárbaros motines, las sediciones militares,
siniestro brazo de la idea disolvente, ya se llamase liberal, ya
realista\ldots{} Con estas imágenes se confundía en su mente otra, que
como un espectro familiar de continuo se le presentaba. Era su promesa
de terminar la guerra civil en seis meses. ¡Lucido quedaría si no la
cumplía; si el ejército cristino, reforzado pronto con los cien mil
hombres de la quinta, no lograba sofocar la facción y restablecer la
anhelada paz! Su ensueño era Córdoba, el caudillo denodado y
caballeresco, y en medio de aquel trajín electoral, anuncio de las
trapisondas parlamentarias y políticas que habían de sobrevenir con la
apertura de los Estamentos, volvía D. Juan Álvarez sus inquietos ojos al
Norte, mirando a lo que era su temor y su esperanza. Si el General no le
ayudaba, su empresa de salvación nacional fallaría sin remedio. Y para
que Córdoba coadyuvase a la gran obra, era preciso que venciera, o por
lo menos que con rudos achuchones quebrantase a los carlistas; y para
esto era indispensable enviarle recursos en hombres y dinero. La carta,
en su difuso estilo, plagada de noticias de acá y de allá, de
referencias diplomáticas y de rumores de intrigas, vino a parar en
positivas promesas. «Dentro de quince días le mandaré a usted millón y
medio. El mes próximo podré mandarle otro tanto, y si puedo más, más.»
Hablábale de remesas de vestuario y calzado, de arreglo de hospitales.
Exponía también planes estratégicos que a él se le ocurrían. «Respetando
su iniciativa, le diré que si usted lograra ocupar el Baztán con quince
mil hombres, podría atacar a los facciosos por retaguardia\ldots{} Eso
usted verá\ldots»

Concluía ofreciendo remesarle nueve millones antes de tres meses, y
manifestaba viva intranquilidad por la lentitud de las operaciones.
Aplicando a todo su febril genio de travesura y arbitrismo, habría
querido que Córdoba moviese en tres días su grande ejército, que
desalojase a los carlistas de sus formidables posiciones, que los
arrollase, que los deshiciese, dispersando a unos, matando a los más, y
cogiendo prisionero a Don Carlos con toda su trashumante Corte. ¡Qué
hermoso sería esto, y con cuánto desahogo podría dedicarse entonces el
Presidente a la reforma del país, que era su ilusión, su sueño!\ldots{}
Pero ¡ay!, al llegar a este punto, cruelísima duda le asaltaba. Si
Córdoba obtenía una victoria rápida y decisiva, cortándole de una vez a
la hidra todas sus patas y aplastándole la cabeza, Córdoba y no otro
había de emprender y realizar la salvación de la infeliz patria. Buen
tonto sería, juzgando el caso con el criterio genuinamente español, si
siendo él el vencedor guerrero, dejaba a otro la gloria de la campaña
política. Lógico era, no obstante, que el militar allanara el camino, y
que el civil marchase por él desembarazadamente hacia la victoria
política y social. Pero aunque poco ducho aún en artes de gobierno, D.
Juan de Dios conocía la historia, más por lo que había visto que por lo
que había leído, y no ignoraba que, en nuestra tierra de garbanzos y
pronunciamientos, el guerrero victorioso es el único salvador posible en
todos los órdenes.

Terminada la carta, vagó su mente en aquel meditar triste. ¿Quién salva,
quién no salva? ¿Sería un error suyo gravísimo haberse creído capaz de
fundar una nación grande y rica sobre las ruinas de las facciones
deshechas y de las banderías sojuzgadas? De Londres había salido con
esta ilusión; con ella entró en Madrid. Sus entrevistas con la Reina
Gobernadora la confirmaron. El entusiasmo patriótico, la fe en sí mismo
y en la eficacia de sus manejos se avivaron cuando Su Majestad le
encargó del teje-maneje gubernamental. Ya tenía la máquina en su mano.
Ya era dueño de sus iniciativas. ¿No podría desarrollar libremente sus
ideas, aplicar su voluntad potente a la grande obra?

Las cosas, y más que las cosas las personas, enfriaron su entusiasmo al
mes de gobierno. Cierto que le ayudaba la opinión vocinglera; pero las
principales figuras políticas no hacían nada en su favor. Los adictos de
fila pedían destinos y actas, y esperaban que el jefe lo diera todo
hecho. Los contrarios aparentaban una calma prudente, tras de la cual D.
Juan de Dios creía sentir el sordo roer de las conspiraciones. Aún no
había perdido la confianza en sí mismo; seguía creyendo en su papel
providencial; pero ya le anunciaba el corazón que la empresa no era
coser y cantar, y que tendría que tragar mucha quina antes de rematarla
dignamente.

Conferenció con Galiano, a la hora convenida, sobre asuntos electorales;
con Saavedra, sobre la probable benevolencia de los moderados Toreno y
Martínez de la Rosa; con Olózaga, para ver de que las sociedades
secretas hiciesen entender a las Juntas que había llegado la hora de
poner fin a la bullanga, pues en \emph{Palacio} comenzaban los
infalibles síntomas de desconfianza y miedo. De esto le había hablado
aquella misma tarde D. Fernando Muñoz, dándole una prueba de verdadero
aprecio. Y, francamente, no había que esperar ninguna ventaja política,
mientras no se diese a toda la gente de allá, real o morganática, una
plácida confianza y un sueño tranquilo. Con Williers habló de asuntos
diplomáticos y de eso que tiempo ha viene siendo la constante pesadilla
de los pueblos débiles: \emph{la actitud de Inglaterra}. Mendizábal era
muy afecto al leopardo, y esperaba un apoyo más positivo que el de la
prometida legión. El astuto representante de la Gran Bretaña repitió a
nuestro Ministro sus recomendaciones de siempre: refrenar la anarquía,
no temer la libertad practicada dentro de las leyes, poner en funciones
regulares el Parlamento, acudir a la guerra con toda clase de recursos,
y trazar las grandes líneas del porvenir efectuando la venta inmediata
de toda la propiedad territorial de las Órdenes religiosas.

Cerca de la una, Mendizábal se quedó solo; mas no se resolvió a
retirarse a su casa, porque el aposento ministerial le retenía, le
agasajaba; temía dejarse allí las ideas si se iba, y con sus ideas la
ilusión risueña y querida de salvar al país y hacerlo dichoso. No menos
de media hora estuvo paseándose de un ángulo a otro, a la luz ya
mortecina de los quinqués, entre los retratos de personas reales o de
eminencias políticas: la Reina Amelia, clorótica y triste; Fernando,
sanguíneo y echando a borbotones la perfidia por sus ojos de fuego, el
sarcasmo por su belfo labio\ldots{} más allá, personajes de peluca que
habían gobernado la Hacienda y la Marina: Patiño, Ensenada; en un ángulo
Riperdá, con su risa ladina; en otro Macanaz, con su hermosa cabeza
poblada de ricitos.

Cansado de pasearse, Mendizábal sacó de su pupitre varios papeles,
cartas que aún no había leído, de esas cuyo escaso interés se adivina
por el sobrescrito, y que se dejan sin abrir por no desperdigar la
atención; otras de letra bien conocida, que, positivamente, no eran de
asuntos ministeriales, más bien pretensiones ridículas, jaquecas,
extravagancias, anónimos quizás, llenos de injurias repugnantes, o
denunciando algún proyecto terrorífico de las logias masónicas.

Era hombre D. Juan que a lo mejor transportaba toda su atención de lo
grave a lo menudo, como espíritu aventurero, que gozara en suponer la
existencia de cosas grandes, escondidas de un modo carnavalesco detrás
de cualquier insignificancia. Su imaginación le llevaba a la puerilidad.
Creía fácilmente en las posibles emergencias de sucesos importantísimos,
efecto enorme engendrado por la menor cantidad posible de causa. No
estaba exento su espíritu de superstición: esperaba bienes repentinos,
no anunciados por la lógica; temía desventuras abrumadoras, caídas como
el rayo, sin el antecedente natural de errores determinantes.

En aquella hora de calma y soledad, aplicando a los objetos secundarios
más bien la curiosidad que la atención, fijose primero D. Juan en una
cuenta de zapatero; después pasó la vista por un plan en que anónimo
arbitrista ofrecía saldar toda la Deuda de España con una simple
combinación de cifras; leyó en seguida una carta procedente de Londres,
escrita en español de colegio inglés. En la primera carilla, una mano
trémula había trazado quejas melancólicas, reproches agridulces; en la
segunda, se lamentaba de un olvido semejante, de abandono; en la
tercera, formulaba con indecisa escritura una protesta de firme
constancia a prueba de desdenes, y en la última, pedía dinero. En la
postdata suplicaba se le mandase inmediatamente orden contra la casa
\emph{Tal}\ldots{} Esta epístola y los documentos anteriores fueron a
parar, en pedazos, a la cesta de los papeles inútiles. Cogió luego otra
carta, cuyo sobrescrito era un puro adefesio, y abierta, leyó con no
poca dificultad: «Señor D. Juan excelentísimo: Por encargo de la señora
Doña Jacoba Zahón, que permanece enferma en cama, le digo cómo la ropa
de la niña importa mil setecientos y veinte y dos reales efectivos, que
hará el favor de remitir a la mayor brevedad, para atender a las
urgencias. Pues ha de saber que se debe lo del maestro de piano y baile
viceversa, con lo demás que había pendiente del coste del mes pasado
inclusive, y son por junto naturalmente trescientos y doce reales netos,
con lo de medicinas trescientos ochenta y ocho. Doña Jacoba espera le
suministre pronto la suma total de los expresados líquidos reales de
vellón, como débitos naturales, y me encarga conjuntamente le diga que
le besa las manos, y que tendrá el honor de visitarle en cuanto se
alivie de sus reumas achacosos. Dios guarde a usted, excelentísimo, años
muchos, y mande a su servidor, que lo es---Cayetano Lopresti.»

Suspirando fuerte, señal inequívoca de lo desagradable del asunto, cogió
la pizarrita en que anotar solía las obligaciones perentorias del día
siguiente, ya fuesen políticas, ya del orden familiar y privado. Media
pizarra estaba escrita ya con diversos recordatorios de varia
importancia: «circular intendencias\ldots{} ver Argüelles, proyecto
electoral\ldots{} recuento de frailes\ldots{} relaciones de
monjas\ldots{} escribir Duque de Broglie\ldots» Con mano enérgica,
fruncido el ceño, apuntó debajo: «Asunto Negretti\ldots{} \emph{Din.}
\emph{jor.} (que quería decir: mandar dinero a la jorobada).»

Guardó unos pasteles en las gavetas; recogió otros, metiéndoselos en el
bolsillo; tiró de la campanilla. El sonido lejano de esta produjo la
aparición de un portero que surgió de entre los pliegues de la cortina.
«Mi capa\ldots{} el coche» dijo Su Excelencia dando pataditas en la
alfombra, que aún era de verano. Se le habían enfriado los pies,
calzados con zapatito mujeril.

Y con esto se fue Mendizábal a su casa de la calle de San Miguel. Durmió
mal. Volteaba el cuerpo entre las sábanas, y en su cerebro enardecido
por el trabajo se torcían las ideas y se enlazaban como queriendo formar
una trenza: «Ley electoral\ldots{} ¡Pobre Negretti!\ldots{} La
guerra\ldots{} ¡Pero esa niña, esa fastidiosa niña\ldots{} esa guerra,
esa maldita guerra!\ldots»

\hypertarget{xiv}{%
\chapter{XIV}\label{xiv}}

También el bueno de Calpena durmió mal, a causa de los sobresaltos de su
amor propio, que aquella noche, al volver de la oficina, había sufrido
nuevos golpes. La última carta de la \emph{mano oculta} revelaba un
espionaje fastidiosísimo. Era en verdad humillante no poder dar paso
alguno de que no tuviera conocimiento la persona que le protegía. Cierto
que agradecía la protección; pero habríala estimado más, si no
significara para él la pérdida de toda la libertad. Al día siguiente, el
anónimo corresponsal mostró detallado conocimiento de cuanto al señorito
le había ocurrido en la oficina: le reprendió por la compañía del tísico
Serrano; le incitaba a frecuentar menos los cafés y más la sociedad,
pues en aquellos adquiriría hábito de grosería y desparpajo, y
aprendería en ésta la finura y distinción de un perfecto caballero.

«Hijo mío---decíale D. Pedro, resueltamente conforme con las opiniones
de la incógnita,---no te importe esa vigilancia, que puede ser algo
molesta, pero que sin duda te apartará de muchos peligros. Frecuenta la
sociedad, pues ya tienes relaciones que te introduzcan en casas
decentes, donde hallarás exquisito trato, buen comer y placeres
honestos. En fin, te conviene mejorar el terreno. Es la única manera de
irnos librando de este maldito romanticismo que pretende volvernos
locos. No desobedezcas a quien quiere llevarte a la regularidad, a la
buena escuela de tu padrino D. Narciso.»

---Pues le diré a usted con franqueza, mi querido Hillo: la falta de
libertad que me resulta de esta subordinación cargantísima a un poder
misterioso, a un poder benéfico, lo reconozco, pero enteramente
inquisitorial, a estilo veneciano, produce en mí un vivo anhelo de
evadirme de tan enojosa tutela. No sabe usted cuánto deseo hacer algo
que resulte ignorado por mi anónimo gobernante. ¿Por ventura, el
servicio de policía que ha organizado para vigilarme ha de ser tan
perfecto que no pueda yo burlarlo, siquiera para probar la habilidad con
que lo burlo? En la oficina hay ojos que me observan; aquí, en casa, no
digamos; en la calle, en el café, en los teatros, en las casas que
visito, ya sabe usted lo que pasa. No respiro sin que \emph{allí lo
sepan}. Pues yo quisiera respirar a mis anchas, y decir: «te fastidias,
que no lo sabes.»

En el curso de Octubre fue introducido el venturoso mancebo por Mesonero
Romanos en casa del médico Rivas, padre de tres niñas preciosas, muy
saladas: Marianita, Mariquita y Juanita, conocidas en el mundo poético
por \emph{Laura, Silvia y Rosaura}, con que las designaban sus novios o
pretendientes (en aquel tiempo se solían llaman \emph{amantes}), que
eran poetas de lo más granadito entonces. Las chicas, eso sí,
descollaban por su picante belleza, así como por su ingenio; una de
ellas también versificaba, otra pintaba, y las tres hacían en el canto y
baile angélicos primores.

Recibido en palmitas fue Calpena en la casa del ilustre médico, y a la
segunda noche echó de ver que la mayor de las niñas le gustaba
extraordinariamente. A la noche tercera hubo de entender que era
correspondido: a las miradas flamígeras siguió el tiroteo de florecillas
verbales, y alguna breve y ardorosa promesa. Al fin de la semana, ya
corría de sala en sala la opinión de que eran novios. Pero ¡ay!, el
domingo recibió Calpena la carta anónima con el siguiente réspice:
«Niño, me desagradan lo que no puedes figurarte tus revoloteos con la
chica mayor del cirujano Rivas. Simple, ¿en qué estás pensando? ¿Sabes
que haces un papel ridículo? Si estás ciego, caiga de tus ojos la venda.
No digo que Silvia y sus hermanas no sean honestas: lo son. Pero ya en
el nido de sus tiernos corazones ha batido sus alitas otro amor\ldots»

---¡Oh, qué figura tan linda! En el nido de sus tiernos\ldots{}
Adelante. Sigue leyendo.

Y Calpena, dándose a los demonios, continuaba la lectura: «Las tres
tienen sus adoradores. Mesonero es el zagal de la tercera pastorcita, la
linda Rosaura. En los altares de la segunda, Silvia bella, quema el
incienso de su inspiración socarrona Bretón de los Herreros. Y, por
último, escucha y tiembla\ldots{} Ventura de la Vega, tu amigo, ese que
te recita sus versos en el café para que convides a toda la partida, es
el dichoso amante de Laura; la misma noche que os cantó la niña el aria
de \emph{Elisabeth}, del maestro Caraffa, quedó concertado entre Ventura
y los padres encender pronto la antorcha de Himeneo\ldots{} Con que ya
ves\ldots»

---¡Qué elegancia de estilo: \emph{encender la antorcha!}\ldots{}

Concluía la carta con observaciones de otro orden, y la noticia de que
ya se habían dado los pasos para redimirle de la quinta de cien mil
hombres, mediante el pago de cuatro mil reales. En la del siguiente día
se le ordenaba que no volviese a la tertulia del cirujano; que no
pensara más en la bella Laura, y que procurase meter la cabeza, pues
relaciones iba ganando para ello, en casas de más categoría, en los
dorados salones aristocráticos. «Mira, tontín: Roca de Togores, que es
un chico muy introducido, puede llevarte a casa de Campo-Alange, y el
almibarado Clemencín (llamémosle D. Agapito) a casa de Castro-Terreño.»

---Ya ves---decía Hillo cayéndosele la baba,---con qué seguro dedo te
marca tus altos destinos. Pero, tontín, digo yo ahora, ¿cómo has podido
figurarte que te íbamos a permitir entroncar con la hija de un cirujano?
¡D. Fernando Calpena unido en desigual coyunda con una simple Laura, sin
más títulos que los ovillejos que le endilgan poetas chirles!\ldots{}
No, hijo, tú no puedes \emph{encender la antorcha} sino con damas de
otro cuño; y aunque pienso que no habrá en Madrid las hijas de duques o
archiduques que te corresponden, sigue por de pronto el consejo que te
da quien darlo puede, y mete la cabeza en las áureas viviendas de los
Abrantes y Veraguas, de los Oñates y Medinacelis.

Refunfuñando, Fernandito concluía por someterse a todo, y a fines de
Octubre le introdujo un amigo (no se sabe fijamente si fue Ros de Olano
o Miguel de los Santos Álvarez) en las casas de Almodóvar y de
Campo-Alange. En la primera de estas mansiones conoció a una beldad fría
y correcta, hija de un aristócrata, que era al propio tiempo general
poco afortunado, la cual cautivaba a cuantos la veían, no sólo por su
marmórea belleza, exenta, eso sí, de toda gracia, sino por su ingenio.
Educada en Francia, se traía lecturas varias y admiración muy redicha
por Chateaubriand, De Jouy y otros coetáneos, siendo también algo
versada en Racine, Marmontel y Madama Genlis.

Con ella platicaba Calpena: notaba este que su conversación y figura
eran del agrado de la marmórea, de lo cual vino que él también se
sintiese cautivado por la linda estatua, y aun que se lo hiciese
comprender en delicadas perífrasis. La \emph{oculta mano} escribió:
«Bien, bien, caballerito: ese es el camino. Recomiendo, no obstante,
moderación, pausa, fino pulso, y no lanzarse con demasiados ímpetus por
un terreno que, a tus inexpertos ojos, parecerá llano, y no lo es. En él
hay asperezas y obstáculos enormes, que tú no ves, pobre niño. Habrás
notado que nuestra sociedad es la más democrática del mundo, y que en
las casas más linajudas no se niega el pase a ninguna persona bien
vestida. Para recibirle y agasajarle, a nadie se le pregunta quién es,
ni de dónde viene, ni a dónde va. Yo creo que tanta franqueza no conduce
a nada bueno. Por más que sólo sea aparente, esa igualdad significa que
nuestra aristocracia pierde el sentido de su misión y no sabe conservar
el orgullo castizo, el cual sería un baluarte contra las confusiones que
se anuncian, y que traerán un desquiciamiento social. Perdona mi
pedantería.»

---¡Por San Cucufate!, no es pedantería---exclamó D. Pedro
palmoteando,---sino profundísima filosofía de la historia. Sigue.

---«Esa igualdad es un mal síntoma, y nada más por ahora; una forma de
cortesía tolerante\ldots{} En el fondo, en los hechos, no hay tal
igualdad. Por eso, al notar muchos que te aproximas a la marmórea,
empiezan a preguntar: \emph{ese Calpena}, ¿quién es? ¿De dónde ha salido
ese barbilindo?\ldots{} Y ya verás, ya verás cómo empiezan pronto los
desdenes, las envidias\ldots{} Para que nada de esto ocurriese y tus
caminos fuesen llanos, sería preciso que en aquella misma esfera hubiese
personas que evidentemente te protegieran, que respondiendo de ti,
dijesen a quien deben decirlo: \emph{ese pobrete} es digno de la niña, y
cuando sea preciso demostrarlo se demostrará. Si ahora te digo que la
estatua erudita, lectora de Chateaubriand y aun de Destut-Tracy,
heredará tres millones y medio, no lo hago porque veas en la riqueza un
incentivo a tu inclinación, no \emph{Ese Don Nadie} no busca un enlace
de conveniencia, ni necesita los millones ajenos, porque es de los que,
por su gran mérito, pueden permitirse la libertad de ser pobres.»

---¡María Santísima, qué frase!\ldots{} Adelante.

---«De ser pobres\ldots{} Te hablo de la presunta riqueza de la niña de
mármol, para que sepas que tu marcha por ese camino ha de ser muy
disputada. Pero no te acobardes. Sobre que tú no sabes si tendrás aún
medios de apedrear con doblones a los que ahora hablan de tu nulidad y
pobreza, sigue adelante, y no veas en la preciosa damisela más que su
educación cristiana, la hidalguía de su familia y de su nombre, su
honestidad, su talento instruidito, sus condiciones, en fin, de
grandísimo precio, y las virtudes y méritos de sus padres, pues aunque
el pobre General nunca ha sabido mandar cuatro soldados, eso no quita
para que sea excelente persona, muy atenta a sus intereses; y en cuanto
a su madre, bien sabes que no hay en Madrid quien la aventaje en nobleza
y virtudes\ldots{} No escribo más. Me duele la cabeza. ¿Pero qué importa
si el espíritu está gozoso?»

Mucho dio que pensar a Calpena el contenido de esta carta, y tanto se
entusiasmó Don Pedro oyéndola leer, que casi casi se le saltaron las
lágrimas. «¿Ves, ves---le dijo,---cómo yo tenía razón? Y que ha de ser
una mujer de inaudito mérito esa marmórea chica. ¡Vaya que leer a
Destut-Tracy!\ldots{} ¡Y qué guapa será!\ldots{} Hombre de Dios, un día
iremos de paseo al Prado, a ver si la encontramos para que me la
enseñes. Ya me figuro su belleza, su dignidad, su mirar grave, como de
la diosa Minerva, su andar majestuoso. Bien, hijo, bien. Ese es el
camino, ése\ldots{} Y ya sabes, dejaré de ser tu amigo y mentor\ldots{}
si\ldots{} Ya sabes mi tema: hay que \emph{rematar la suerte.»}

En tanto, Calpena continuaba prestando su servicio de secretario
particular del primer Ministro, muy a gusto de este, al parecer, pues
cada día le fiaba epístolas de mayor delicadeza, aun aquellas que
contenían algún secretillo político, o en que desahogaba en la confianza
de un buen amigo el recelo que en él iban despertando las dificultades
de su magna empresa.

Por aquellos días ya no iba Fernandito a los cafés, y esquivaba todo lo
posible la sociedad del tísico Serrano, cuyo pesimismo había llegado a
serle odioso. Dos veces fueron juntos al teatro. Dábale Serrano los
nombres de todas las personas que en palcos y butacas veían, sin que de
esto pudiese sacar ninguna luz el aburrido joven. Y como a cada nombre
que el tísico decía agregaba comentarios injuriosos, pues para él no
había mujer honrada, ni madre que no vendiese a sus hijas, ni esposa que
no imitara la conducta aleve de la señora de Oliván, Calpena no quiso
más tal compañía, ni aquella erudición tan mentirosa como terrible.

Con Milagro, su compañero de secretaría, sí que hizo buenas migas
Calpena, y en los cortos ratos libres platicaban de política o
literatura contemporánea, que el viejo conocía medianamente, o bien de
cosas familiares y domésticas. Todo franqueza y espontaneidad
comunicativa, Milagro contaba los refunfuños y genialidades de su mujer,
las bataholas de sus chiquillos menores, y las gracias habilidosas de
sus dos niñas. «Es ridículo---decía,---que a una persona como usted,
introducida en la mejor sociedad, le invite yo a venir a pasar un rato
en mi humilde casa, donde todo es pobreza\ldots{} también alegría, eso
sí\ldots{} Pero yo creo que habría de gustarle oír tocar el arpa a mi
hija María Luisa, discípula de Fagoaga, gran discípula, para que usted
lo sepa\ldots{} y el instrumento es de lo mejor que ha fabricado D.
Tiburcio Martín, plazuela de Matute\ldots{} Ni le desagradaría a usted
echar un parrafito con mi hija segunda, Rafaela, que sabe francés y me
ayuda a traducir \emph{Mujeres célebres}. Lee todo lo que cae en sus
manos, y ahora está agarrada noche y día a la \emph{Corina} de Madama
Stäel\ldots{} Y en casa puede usted ver a una notabilidad, un chico
poeta de mi pueblo, Chiclana, que aunque soldado de la última quinta,
hace versos como los ángeles; sólo que es tan corto de genio y tan para
poco, que cuesta Dios y ayuda hacerle leer lo que escribe. Se llama
Antonio Gutiérrez, y ha compuesto un dramita que titula \emph{El
Trovador} o cosa así, y en casa nos ha parecido tan bueno, que yo mismo
se lo he llevado a Guzmán para que lo lea, a ver si a él o a Carlos
Latorre les da la ventolera de representarlo. Otro chicarrón va por
allí, Pepe Díaz, que también hipa por la poesía y el teatro. No les
falta más que apoyo, protección, y aquí, ya se sabe, no la hay más que
para los necios enfatuados. Yo les digo: «Hijos míos, no os acobardéis,
que a falta de otros protectores, aquí me tenéis a mí\ldots{} ¡Milagro
será que no os saque adelante Milagro!\ldots{} je, je\ldots»

Cortés y agradecido Calpena, declaró que con mucho gusto aceptaría la
invitación, visitándole una de las noches que tuviera libres. Al mismo
tiempo recordó el conocimiento de Milagro con Doña Jacoba Zahón,
añadiendo que para esta señora había traído de Francia un encargo que
aún se hallaba en su poder. Por voluntad expresa del remitente, no lo
entregaría más que a la misma persona a quien venía destinado, y esta
debía presentarse a recogerlo.

«Seguramente---dijo Milagro,---es una caja de pedrerías\ldots{} ¿Por qué
se asombra usted? La Zahón comercia en diamantes y perlas. La casa es
muy conocida: \emph{Zahón y Negretti}, calle de Milaneses. Hoy, por
muerte de Zahón, se ha quedado al frente la viuda, para quien algunas
noches trabajo, escribiéndole la correspondencia y poniéndole las
cuentas en orden.»

---No puede ser caja de piedras preciosas lo que traje y aún
conservo---observó Calpena,---pues no habían de tardar tanto en recoger
cosa de valor grande. ¿Acaso comercia esa señora en pedrería falsa?

---No, señor\ldots{} Todo lo que compra y vende es de la mejor ley. Si
no ha pasado Doña Jacoba a recoger su encargo, será porque ha estado
enferma, o porque no tiene noticia exacta de la persona que lo ha
traído.

---Debe de tenerla, porque al día siguiente de mi llegada, escribí a
Olorón dando cuenta de mi domicilio. Por cierto que me dijeron que esa
señora es jorobada.

---Cargadita de espaldas\ldots{} Yo le hablaré del caso, y nos iremos a
su casa si ella no puede salir. Verá usted una mujer lista y
estrafalaria, genio desigual, mañas de urraca, agudezas de lince, toda
uñas, toda desconfianza\ldots{}

---Pues yo había creído que el paquete que traigo es de cartas o papeles
políticos. Dígame usted\ldots{} aquí en confianza, ¿esa señora conspira?

---¡Conspirar la Zahón\ldots!---dijo Milagro perplejo.---No\ldots, que
yo sepa, no\ldots{} ¡Conspirar\ldots! Para la Zahón no hay más política
que ganar dinero, engañar a quien puede, y despojar a los infelices que
caen en sus garras.

---Ello será como usted lo dice; pero yo puedo asegurarle que un
compañero mío de hospedaje, que anda en las logias de la casa de Tepa,
supo, a los pocos días de mi llegada a Madrid, que yo había traído ese
encargo, y tanto él como sus amigos López y Caballero creían, y así me
lo dijeron, que el paquete era de papeles políticos y venía destinado al
eterno conspirador D. Eugenio Aviraneta.

---Observe usted, amigo Calpena, que los patriotas, de tanto andar al
obscuro en logias y \emph{sublimes talleres} soterráneos, ven visiones,
y como la policía de aquí vive también palpando tinieblas, entre unos y
otros le arman a usted unos enredos que le vuelven loco. El año del
fusilamiento de Torrijos vine yo de Sevilla a Madrid en galera, y no
acelerada, con mi familia, pasando los mayores trabajos que usted puede
imaginar. Diéronme allí un encargo para la señora de D. Vicente González
Arnao, el amigo de Moratín, la cual era muy obesa y padecía de
estreñimiento. Por esto comprenderá usted que el encargo era una
lavativa, gran pieza, modelo recién enviado de Inglaterra. Pues no puede
usted figurarse la que se armó con el dichoso instrumento, en cuanto me
lo descubrieron los de la policía. No le digo a usted más sino que me
costó la broma cuatro meses de cárcel, y mi mujer y mis hijos no se
murieron de hambre porque les recogió un pariente de Bertrán de
Lis\ldots{}

---¿Y la señora de Arnao\ldots?

---Reventó\ldots{} naturalmente\ldots{} Su muerte debió ser un nuevo
cargo para la Superintendencia de Policía, como verdadero
asesinato\ldots{} político.

Campanillazo\ldots{} Acudió Milagro presuroso al llamamiento del señor
Ministro.

\hypertarget{xv}{%
\chapter{XV}\label{xv}}

A los pocos momentos de quedarse solo Calpena en el despacho, entró
Iglesias por la puerta interior, que comunicaba con la Secretaría. «En
nombrando al ruin de Roma\ldots{} No hace diez minutos, querido
Nicomedes, que le recordábamos a usted.»

---No sería para hablar mal.

---De ningún modo. Al contrario\ldots{}

---Hace un siglo que no nos vemos, amigo Calpena. Ayer y hoy no he
comido en casa. Tenemos usted y yo las horas encontradas, y lo siento,
porque en estas circunstancias me conviene verle a usted con frecuencia.
Por eso he venido.

---Estoy a sus órdenes.

---Ya sé---dijo Nicomedes dejando sobre la mesa su sombrero, que era de
última moda, cilíndrico, enorme, un soberbio tubo de chimenea con alas
planas,---ya sé que el Presidente le quiere a usted mucho\ldots{} Eso se
llama caer de pie. Usted es de los que se lo encuentran todo hecho. Bien
haya quien tiene el padre alcalde\ldots{} Pues yo, contando con su
amabilidad, venía\ldots{}

---Siéntese el buen Iglesias, y dígame en qué puedo servirle.

Sentose Nicomedes, y pasándose la mano por las melenas, que eran largas
y copudas, parecía inquieto, caviloso, extenuado por el insomnio y las
ansiedades de la ambición.

«Quisiera que el simpático Calpena, sin faltar lo más mínimo a la
reserva que le impone su cargo en la Secretaría particular\ldots{}
¡cuidado, que no trato de poner a prueba su discreción\ldots!, pues
quisiera que usted me dijese si ha escrito a D. Juan Álvarez en favor
mío\ldots»

---¿Quién? Supongo que será recomendación para las elecciones.

---Justo. Pues se comprometió a escribir al Presidente, recomendándome
con toda eficacia, imponiéndome más bien, quien menos puede usted
figurarse.

---¿Caballero, Trueba y Cossío?

---Esos son amigos míos, y bastante tienen con manipular su elección, el
uno en Cuenca, el otro en Santander. A mí me habían prometido incluirme
en la candidatura de Murcia. Quiroga me aseguró que allí me votarían
hasta las piedras. Luego resulta que no las piedras, sino los electores,
votan a Escalante. Al fin, me refugié en Villafranca del Bierzo, donde
tengo algunos elementos.

---Por ese lado, Argüelles influye, también D. Martín\ldots{}

---No cuento con esos\ldots{} Ofreció apoyarme\ldots{} vuelvo a decirlo,
quien menos puede sospechar\ldots{} En este juego de la política, los
extremos se tocan. Pues me apadrina D. Francisco Martínez de la Rosa, es
decir, prometió hacerlo\ldots{} en virtud de concesiones mutuas que
acordamos en Tepa, interviniendo por los moderados Ramón Narváez; por
nosotros, mi amigo Palarea.

---Ya\ldots{} comprendo\ldots{} Y usted quiere saber si Martínez de la
Rosa ha escrito\ldots{} Lo ignoro: si algo supiera se lo diría, pues en
ello no veo deslealtad. Por mi mano no ha pasado carta de D. Francisco;
y si D. Juan la ha recibido, habrala contestado por sí propio.

---¿Y su compañero de usted, ese viejo cegato\ldots?

---No sé nada. Es hombre muy reservado.

---Bueno: desde ayer sospecho que esos malditos \emph{anilleros} nos
engañan. Siempre han sido lo mismo. Cuando están fuera del poder, nos
buscan, nos agasajan, se arriman a la \emph{exaltación}\ldots{} Otra
cosa: ¿No recuerda usted si entre las recomendaciones de candidatos, que
hace diariamente este buen señor a Don Martín de los Heros, ha ido mi
nombre?

---Tampoco lo recuerdo.

---Voy creyendo que Heros me engaña también. No puede esperarse otra
cosa de quien no tiene iniciativa ni criterio para nada. Tanto a él como
a Becerra les trata este señor como a criados.

---Pues mire usted---indicó Calpena esforzándose en hacer memoria,---yo
tengo idea de haber visto el nombre de usted en alguna de las cartas que
me ha dado D. Juan para contestarlas\ldots{}

---A ver si recuerda, hombre, a ver si recuerda\ldots---dijo Iglesias
aproximando su silla para poder hablar en voz más queda.---¿Sería en una
carta de D. Fernando Muñoz?\ldots{}

---¿El marido de la Reina? No\ldots{} D. Fernando estuvo aquí una noche,
y habló con el Presidente, lo que no tiene nada de particular, y por eso
puedo decirlo.

---¿Y no ha pasado por aquí una carta de D. Juan Muñoz, Padre jesuita,
hermano de D. Fernando? Me consta que le suplicó se interesase en favor
mío la persona que le salvó la vida en el colegio de San Isidro el día
del degüello, en Julio de 1834.

---Tampoco he visto carta alguna de ese señor jesuita.

---Pues no dudo que su hermano habrá dicho algo a Mendizábal. Sepa usted
que en Palacio, de tiempo en tiempo, echan una mirada a la
\emph{exaltación}, y nos halagan para que no extrememos la guerra.
Decididamente hemos vuelto la espalda al señor \emph{Dracón}, que no nos
sirve para nada. Ya sabe usted que en el actual momento histórico Doña
Carlota y su hermana están a matar.

---No sabía\ldots{} La verdad, me fijo poco en intrigas palatinas. Creo
que mucho de lo que se cuenta es falso, embustes fraguados a gusto del
que los pone en circulación.

---Lo que digo es el Evangelio. Están a matar\ldots{} Nosotros hemos
abandonado a \emph{la} Carlota, y apoyando por el momento a Cristina,
trabajamos en el extranjero para evitar la protección que dan a D.
Carlos los legitimistas y vendeanos. Mendizábal hace la misma política:
no me dirá usted que no escribe cartas a la hermana de estas señoras,
Carolina, Duquesa de Berry.

---Nada sé, amigo mío---declaró Calpena, comprendiendo al fin que debía
refugiarse en la discreción, y evitar revelaciones inconvenientes.

---Pues bien: decidido a minar la tierra para ocupar el lugar que me
corresponde en el Estamento, y viéndome abandonado por algunos amigos,
vendido por otros, por ninguno apoyado resueltamente, he pegado un
brinco horroroso, solicitando el apoyo de un legitimista francés de gran
empuje, para que recabe de la Duquesa de Berry una expresiva
recomendación\ldots{}

---Y ese legitimista es el señor Conde de la Pommeraye, ayudante que fue
del Duque de Angulema. Ha escrito a Mendizábal; pero no hace referencia
a la de Berry, y se limita a dar las gracias por el reconocimiento que
se le ha hecho de varias cruces concedidas el año 23, asunto que quedó
suspenso por error, o por olvido de ciertos trámites\ldots{}

---Me consta que a la de Berry debe el de la Pommeraye que le hayan
reconocido dos cruces pensionadas. Lo sé: es amigo de mi familia. Mi tío
Andrés le salvó la vida en el ataque y toma de Pasajes\ldots{} Por lo
visto, usted no puede o no quiere darme ninguna luz. Cada día me afirmo
más en la idea de que todos me abandonan, de que nadie se interesa por
mí\ldots{} ¡Y esto le pasa al hombre que ha consagrado toda su
inteligencia, su vida toda, a la idea revolucionaria, a la redención de
este pueblo\ldots{} ¡Mátese usted, reviéntese, padezca hambres y
persecuciones por la regeneración de un país, por ennoblecerle, por
desasnarle, por sacarle de las uñas de la feroz tiranía\ldots{} y cuando
cree recibir el premio de su servicio, cuando usted humildemente dice a
ese país: «Dame tu representación, dame tus poderes, pues quiero
desgañitarme en tu defensa,» vese usted desatendido, menospreciado,
tratado como un loco\ldots{} ¡Oh, esto no puede ser, esto clama al
cielo!

Dio un porrazo en la mesa el iracundo Nicomedes, y se levantó,
irguiéndose con fiera majestad y sacudiendo la melena. Quiso calmarle D.
Fernando con frases de esperanza: «No desmaye usted tan pronto. Si no es
ahora, otra vez será.»

---Lo mismo me dijeron en las primeras Cortes del Estatuto\ldots{} No,
no he nacido yo para vestir imágenes\ldots{} ni aun la imagen de la
Libertad. No, ya no espero nada\ldots{} La culpa tiene quien se desvive
por sus ideas, olvidando que ha nacido en la tierra de la
ingratitud\ldots{} Créame usted, los carlistas lo entienden. Van tras de
su objeto espada en mano; persiguen la realidad a sangre y fuego. Esos
no se andan con remilgos, ni fían su éxito a las amistades, ni a los
hinchados discursos, ni a recomendaciones impertinentes. ¡Hierro, y nada
más que hierro!\ldots{} Mientras nosotros no hagamos lo mismo, no iremos
a ninguna parte.

Y cogiendo el enorme sombrero con tanta violencia, que a punto estuvo de
romperle el ala (¡lástima grande, pues lo había comprado aquel día!), se
lo encasquetó sobre la melena, diciendo: «Yo le aseguro a usted, querido
Fernando, que me la pagan\ldots{} ¡vaya si me la pagan!\ldots»

Despidiéndole en la puerta, tuvo Calpena una idea feliz: «¿Por qué no se
decide usted a hablar con el propio Mendizábal? El llanto sobre el
difunto. Pídale usted audiencia ahora mismo.»

---Ya hemos hablado\ldots{} Me recibirá muy atento. A buenas palabras no
le gana nadie. Pero todo se queda en agua de cerrajas\ldots{} Déjele
usted\ldots{} déjele. Fracasará por no rodearse de los verdaderos
patriotas\ldots{} Morirá a manos de los \emph{santones}\ldots{} ¡Que
muera, que se hunda!\ldots{}

En aquel punto entró Milagro con un puñado de cartas, y preguntándole
Calpena si el Presidente estaba solo, dijo que en aquel momento acababan
de entrar D. Agustín Argüelles y D. Ramón de Calatrava.

«Ahí tiene a todo el \emph{santonismo}---dijo Iglesias con
sarcasmo.---Vienen a tomarle medida del féretro\ldots{} y a cortarle los
pies bonitos para que quepa\ldots{} Es muy grandón D. Juan Álvarez
Mendizábal\ldots{} Pero quizás lo que le sobra no es por abajo, sino por
arriba\ldots{} Señores, conservarse.»

No pudieron entretenerse los dos amigos en conversaciones, porque al
punto se enfrascaron en el trabajo, que no era flojo aquel día. Milagro
dio a su compañero algunas cartas, indicándole el sentido de la
contestación, y al instante humilló su flácido rostro, paseando la punta
de la nariz sobre el papel, al propio tiempo que la pluma. Contestó
Calpena varias cartas de pura cortesía, de esas que no dicen nada y
formulan vagas promesas, con arreglo al patrón usual en las secretarías
familiares de los señores Ministros. Toda la tarde se la pasó el de
Hacienda en conciliábulos con prohombres, en firmar asuntos
importantísimos de Deuda, de Aduanas, algunos nombramientos, y en
repasar el proyecto de discurso que había de leer la Reina en la próxima
apertura de los Estamentos. A última hora llamó a Milagro. Dejando a un
lado la política y apartando de sí todo el papelorio que delante tenía,
se dispuso a despachar un asunto privado, que sin duda le causaba
inquietud y fastidio, a juzgar por el tono con que dijo a su
escribiente: «Otra vez esa pejiguera. Oiga, señor Milagro: mañana me
hará usted el mismo favor del mes pasado.»

---A las órdenes de Vuecencia.

---Nada: que esa maldita jorobada, que Dios confunda, ha vuelto a
pedirme dinero. Y no tengo más remedio que mandárselo, aunque voy
pensando que hay en esto mucho de socaliña\ldots{} ¡Pobre Negretti! Como
usted la conoce y trabaja en su casa, me hará el obsequio de llevarle
esta cantidad que me pide\ldots{} Vea usted qué letra y qué
estilo\ldots{} Cuide de hacerle firmar el recibo en la misma forma de la
otra vez\ldots{} «He recibido del Sr.~Tal\ldots{} testamentario del
Sr.~Negretti\ldots{} la cantidad de tal, importe de alimentos y demás
de\ldots»

---Descuide Vuecencia\ldots{}

---Es un asunto que me desagrada, y en la posición que ahora ocupo,
francamente, no me convienen estos tratos, aunque, bien mirada, la cosa
es sencillísima, y nada tiene de particular\ldots{} Usted, como buen
gaditano, conocería al pobre Negretti.

---Sí, señor\ldots{} Tratante en pedrerías y en metales preciosos. Si no
recuerdo mal, era corso.

---No: hijo de padre corso. Oiría usted contar que en uno de sus viajes
a Inglaterra conoció a la Montefiori. ¿Sabe usted quién era? Una mujer
de historia, muy guapa, francesa o italiana, no lo sé a punto fijo, ni
creo que lo supo nadie.

---Algo me contaron\ldots{}

---A lo tonto, a lo tonto, empezando por galanteos de esos que allí,
como en París, son la aventura de un día, o de una semana, sin
consecuencias, Negretti se enamoró perdidamente de aquella
prójima\ldots{} Y a tanto llegó la ceguera del hombre, que se casó con
ella. Crea usted que el día que me lo dijo, por poco le mato. De nada
valieron los consejos, las exhortaciones de sus buenos amigos. Jenaro
sentía el vértigo, y se arrojó a la sima.

---Ya, ya recuerdo la historia\ldots{} Su mujer murió.

---Asesinada, al salir de un baile en Vauxhall, un sitio que hay en
Londres, a donde concurre todo el mujerío\ldots{} ya me entiende
usted\ldots{}

---Comprendo\ldots{} mujeres guapas\ldots{} pues\ldots{} Esa señora dejó
una niña.

---Que recogió Negretti, poniéndola en casa de los Montefioris de
\emph{Halton Garden}, una calle de Londres donde está todo el comercio
de pedrería. A la muerte de Jenaro, la niña, por disposición
testamentaria de este, fue puesta al cuidado del Montefiori de Mallorca,
y luego de Zahón y Negretti.

---Y ha quedado al fin bajo el poder de Doña Jacoba, donde ahora se
halla. La conozco, señor.

---¿Qué tal es la chica? No la he visto desde que tenía tres años.

---Señor---respondió Milagro dando un suspiro,---Aurorita es
preciosa\ldots{}

---Sale a su madre, que era una divinidad---dijo el gran Mendizábal. Y
se le encandilaron los ojos cuando repitió:---Es preciosa la
niña\ldots{}

---Pero muy revoltosa, señor\ldots{} el carácter más desconcertado que
Vuecencia puede imaginar.

---Tiene a quien salir\ldots{} Pues bien, Negretti dejó en mi poder todo
su dinero\ldots{} No crea usted, pasa de un millón de reales lo que
tenía, y con su fortuna me dejó el encargo de atender a la chiquilla
durante su menor edad\ldots{} Ello es enojoso, mayormente hallándose la
joven en poder de los Zahones, de quienes tengo malas noticias.

---Puedo asegurar a Vuecencia que la niña de Negretti está muy mal
educada y tiene los demonios en el cuerpo; pero merece vivir en mejor
compañía, y yo sé que no ha de faltar quien la cuide, con el emolumento
que percibe la urraca de Doña Jacoba.

---Autorizado estoy---indicó D. Juan Álvarez, distrayéndose ya de aquel
asunto y empezando a pensar en cosas de más importancia,---para
confiarla a otras personas de la familia; y si averiguar pudiera dónde
ha ido a parar Ildefonso Negretti, que se estableció en Bayona, también
en joyería, allá por el año 26, de seguro\ldots{} En fin, señor Milagro,
quedamos en que llevará usted a esa señora\ldots{} Vea la nota, y aquí
tiene el dinero\ldots{} Cuidado con el recibo en regla\ldots{} Y pueden
ustedes retirarse\ldots{} Yo me voy también, que hoy ha sido día de
prueba.

\hypertarget{xvi}{%
\chapter{XVI}\label{xvi}}

Acompañado de su amigo y mentor D. Pedro Hillo, fue Calpena a las
últimas funciones de Toros, y a la apertura de los Estamentos, que se
efectuó a mitad de Noviembre con la solemnidad de costumbre, asistiendo
la Reina Gobernadora. En la Plaza admiraron la pericia del afamado
matador Francisco Montes, y el arrojo y gallardía de D. Rafael Pérez de
Guzmán, oficial del Ejército, de la noble casa de Villamanrique, que
había cambiado los laureles militares por las palmas toreras, y la
espada por el estoque. Tomó la alternativa en Madrid en Junio del 31, y
desde entonces fue la más grande notabilidad del arte en aquella década,
después del maestro Montes. Con estos compartía el favor del público
Roque Miranda, muy inferior a Montes y a \emph{D. Rafael} en la suerte
de matar, pero gran banderillero, capaz de poner pares en los cuernos de
la luna.

Ya se comprende que con la compañía de Hillo en el fiero espectáculo
aprendió Calpena, no sólo los terminachos, sino las reglas del toreo,
adquiriendo el placer de la lidia. Algunas tardes convidó también a
Milagro, grande y antiguo aficionado, sólo que la cortedad de su vista
no le permitía enterarse bien de lo que pasaba. Hiciéronse amigos Hillo
y D. José, y su amistad se consolidaba, lo mismo por la comunidad de
afición que por la diferencia de criterio en el juicio de las suertes.
Si coincidiendo, simpatizaban, disputando como energúmenos simpatizaban
y se querían más. Entre los dos sentábase Calpena en el tendido, y a
menudo tenía que intervenir para aplacar sus bulliciosos ardores de
controversia. Era Hillo devotísimo adepto de la escuela rondeña, y el
otro de la sevillana; enaltecía el clérigo el arte propiamente dicho, la
destreza en el engaño, la burla ingeniosa del peligro, la distinción, la
postura, la gallardía de la figura toreril delante de la fiera;
encomiaba Milagro el valor, la brutal acometividad sin remilgos, mirando
más a la eficacia de la suerte que al afán de \emph{pintarla} y hacer
arrumacos. Eran, pues, el uno clásico, romántico el otro. Disputaba
Milagro por temperamento bullanguero y por llevar la contraria. Hillo,
firme en el \emph{dogma rondeño}, lo sostenía con seriedad, digna de una
tesis escolástica. Tan pronto se arrancaba Milagro a sostener que
\emph{D. Rafael} era un chambón, que debía su boga a \emph{ser de la
Grandeza}, como le defendía resueltamente por su coraje ciego y sin
arte. Consideraba a Montes por paisano, pues ambos eran de Chiclana;
pero a lo mejor se complacía en llamarle gandul o \emph{figurero}.

«Pero usted, señor alma de cántaro---le decía Hillo sin poder contener
su enojo,---¿se ha enterado de lo que ha hecho ese tío en el segundo
toro? Sin duda tiene usted telarañas en los ojos cuando no ha visto ese
sublime arte del engaño, cuando no ha visto con qué salero se lo pasó a
la fiera por delante de la cara para componerla, para quitarle los
resabios adquiridos durante la lidia, para igualarle\ldots{} ¿O es que
usted no sabe lo que es igualar al toro?\ldots{} ¿Sabe acaso distinguir
los pases? Para usted es lo mismo el \emph{natural} que el
\emph{redondo}, el \emph{cambiado} y el \emph{de pecho.»}

---Lo que le digo a \emph{zumercé}---afirmó Milagro al concluir la lidia
del tercero,---es que este pase \emph{de pecho} de D. Rafael no lo hace
mejor el Verbo Divino.

---¡Pero si ha sido una gran patochada! ¡Usted no lo entiende! ¡Si no
estaba perfilado D. Rafael cuando se le vino el toro encima, y en vez de
adelantar el brazo de la muleta hacia el terreno de afuera en la
rectitud del toro, lo que hizo fue\ldots!

---Usted sí que no lo entiende. D. Rafael no movió los pies\ldots{}

---¡Pero si parecía un bailarín!

---Le digo a usted que no. Me han salido los dientes viendo matar toros.
D. Rafael se estuvo quieto hasta que llegó la res a jurisdicción.

---¡Pero si no llegó a jurisdicción\ldots! ¡Por San Cornelio, que
no!\ldots{} Y el animal no tomó el engaño; y D. Rafael, con más coraje
que conocimiento, en vez de darle salida por la derecha, se la dio por
la izquierda, y no supo empaparle. Total, que cuando la res dio el
hachazo\ldots{}

---No hubo tal hachazo.

---Le digo a usted que sí\ldots{}

---Pues, hijo, si Su Reverencia entiende de decir misa como de toros,
lucida está la santa Iglesia.

---Quien no entiende una palotada sois vos.

---Paz, paz---les decía Calpena.---No se peleen por un golletazo de más
o de menos. Tan difícil es matar bien un toro como gobernar a un país.
Tanto mérito tiene el que se pone entre los cuernos de una fiera, como
el que se cuadra ante las astas de una nación querenciosa. No
disputemos, y aplaudamos a todos.

Salían tan amigos, y hablando de política, el clérigo y el funcionario
confundían sus respectivos criterios en un escepticismo zumbón. Fueron
también, como se ha dicho, a la apertura de las Cortes, en el Estamento
de Procuradores, que tenía por alojamiento provisional la iglesia de
\emph{Clérigos menores}, (Carrera de San Jerónimo), convertida en
\emph{redondel parlamentario}. Aunque el día no era apacible, la
multitud se agolpaba en las calles por ver a la Reina y su Corte, y por
admirar el lujo de corceles empenachados, los lacayos y cocheros a la
Federica, las carrozas de concha y marfil, y todo el elegante
barroquismo que constituye el ceremonial palatino de calle. La hermosura
de la Reina, su gracia y gentileza eran tales, que ante la realidad se
achicaban las hipérboles que a su paso se oían. Vestía de negro. Su
peinado de tres potencias, con la real diadema y el velo blanco que
graciosamente le caía sobre los hombros; la pedrería que al cuello y
entre los graciosos moños de su pelo ostentaba; la majestad de su
rostro; la sonrisa hechicera con que agraciaba al pueblo dirigiendo sus
miradas a un lado y otro, formaban un conjunto que difícilmente olvidaba
el que una vez tenía la suerte de verlo. Contaba poco más de veintiocho
años, y ya su nombre había fatigado a la Historia, por las
circunstancias de su casamiento, de su corta vida matrimonial, de su
viudez prematura, que puso en sus manos las riendas de una nación
desbocada. Del bien y del mal que hizo se hablará en mejor ocasión. Por
ahora se dice tan sólo que aquel día de Noviembre, camino de la
ceremoniosa apertura, estaba guapísima. Era, sin disputa, la más salada
de las reinas. Su venida fue un feliz suceso para España, y su belleza
el resorte político a que debieron sus principales éxitos la Libertad y
la Monarquía. Su gracia sonriente enloqueció a los españoles; muchos
patriotas furibundos, a quienes las malas artes de Fernando habían hecho
irreconciliables, desarrugaron el ceño. Antes de tener enemigos
encarnizados, tuvo partidarios frenéticos. Difícilmente se encontrará en
la Historia una Reina a la cual se hayan dedicado más versos: verdad que
no todos los que se arrojaban a su paso para alfombrarle el camino eran
inspirados. Lo que llamamos \emph{ángel} teníalo Cristina en mayor grado
que otras prendas eminentes de la realeza, y todos hallaban en ella un
hechizo singular, un don sugestivo que encadenaba los ánimos. Por eso
Quintana, afectando la confusión lírica, le decía:

\small
\newlength\mlenb
\settowidth\mlenb{«¿Quién te dio ese poder?…}
\begin{center}
\parbox{\mlenb}{«¿Quién te dio ese poder?…                    \\
                ¿De quién hubiste                             \\
                La magia celestial?»}                         \\
\end{center}
\normalsize

Y otro no menos famoso poeta, la saludaba de este modo:

\small
\newlength\mlenc
\settowidth\mlenc{\quad «¡Cuán hermosa! ¡Sus ojos celestiales}
\begin{center}
\parbox{\mlenc}{\quad «¡Cuán hermosa! ¡Sus ojos celestiales   \\
                Cuán apacibles miran!                         \\
                Ved en su frente pura                         \\
                La majestad grabada y la dulzura;             \\
                Mirad en su mejilla                           \\
                La rosa del pudor encantadora.                \\
                Al Consorte Real, que en ella adora           \\
                No menos la virtud que la hermosura,          \\
                Ved ¡cuán tierno sonríe                       \\
                Su labio de coral!…»}                         \\
\end{center}
\normalsize

Y fue tal la prodigalidad de epítetos dulzones, \emph{angélica},
\emph{divina}, \emph{divinal}, \emph{dulce}, \emph{amorosa},
\emph{celeste}, etc., que la lengua se nos hizo empalagosa, y de ahí
vino que por devolverle su tonicidad y fuerza la amargaran demasiado los
románticos con sus acíbares, adelfas y cicutas.

En otro orden hubo de manifestarse el mismo fenómeno de reacción. Es
indudable que muchos se fueron al campo realista, no tanto por
convencimiento, como porque estaban hastiados y apestados de tanta
\emph{angélica Isabel}, de tanta \emph{celestial Cristina}; protestaban
de la virilidad contra el feminismo.

Las tres serían cuando entraba la Reina en el Estamento, y si en el
tránsito por las calles y Puerta del Sol los vivas no cesaban, ni las
encantadoras sonrisas de la dama hermosísima, en la casa parlamentaria
los aplausos y vítores fueron delirantes. Aclamando a la Gobernadora, se
rendía tributo a la hermosura y a la ley, a la vida nueva, a la
esperanza de un porvenir dichoso. El símbolo era tan bello, que encendía
el fuego de la fe con más eficacia que las ideas. No es extraño, pues,
que el historiador, o más bien el filósofo de la Historia, se
preguntara: «¿Hasta qué punto y en qué medida influyó en la suerte de
España el dulce mirar de aquella Reina?» Y un faccioso del orden civil,
aficionado a las grandes síntesis, consolaba a D. Carlos, años adelante,
en las soledades de Bourges: «No hay que culpar a nadie, señor, pues así
lo ha dispuesto el que hace las criaturas. Todo habría pasado de
distinta manera, si la augusta cuñada de Vuestra Majestad hubiera sido
bizca.»

Nuestro amigo Calpena, colocado entre los suyos D. Pedro Hillo y D. José
del Milagro, vio desde una tribuna a la hermosa Reina, y la oyó leer el
discurso. Era la primera vez que la veía, y maravillado de tanta
majestad y gentileza, sus ojos no se saciaban de contemplarla. Milagro,
renegando de su menguada vista, no hacía más que preguntar a Hillo: «¿Y
dónde está Argüelles?\ldots{} ¿Y Saavedra?\ldots{} ¿Y los primerizos
Pacheco y Donoso Cortés?» Poco fuerte en el conocimiento de personas,
Hillo las iba señalando a capricho, y a Pita Pizarro le llamaba Conde de
las Navas, y a D. Antonio González le confundía con D. José Landero y
Corchado.

«Ahí tiene usted al Sr.~D. Juan Álvarez y Méndez, tan orgulloso que
parece el czar de Moscovia\ldots---dijo D. Pedro cuando ya se retiraba
Su Majestad.---Con su pelito rizado y su fraque de última moda, es el
más guapo de los que se sientan en el banco negro.

---Ya, ya le veo---manifestó Milagro, que no veía nada.---Está
arrogantísimo mi jefe\ldots{} Ese, ese es el que os ha de poner a todos
las peras a cuarto. Ya veréis cómo las gasta.

---Me parece a mí---dijo Hillo,---que trae buenos planes; pero no el
trasteo que se necesita para ejecutarlos.

---Trasteo le sobra.

---Le falta mano izquierda.

---¡Qué ha de faltarle, hombre!

---No sabe manejar el engaño. Hay aquí ganado de mucho sentido,
voluntarioso, que \emph{hace} por los Ministros, y no para hasta que los
engancha. ¡Pobre D. Juan!\ldots{} Él ha venido por palmas, y le van a
dar\ldots{}

---¿Qué\ldots?, ¿qué le van a dar?\ldots---dijo Milagro, empezando a
amoscarse.

---Nada, hombre: no se sulfure. De toros entiende usted poco; pero de
este tinglado ni una patata.

---Quien no lo entiende es Su Señoría. Me han salido los dientes viendo
Cortes\ldots{}

---¿En dónde, alma de Dios?

---En Cádiz\ldots{} en San Felipe Neri.

---Ese santo no es de mi devoción.

---De la mía sí. En mi iglesia adoramos a los patriotas y abominamos de
la clerigalla.

---Paz, caballeros---dijo Calpena con gracia.---No me riñan aquí o a los
dos les mando a la calle.

---Es broma.

---Jugamos, nos divertimos.

En esto salían ya de la tribuna, y empezaba el penoso descenso entre un
gentío bullicioso, mareante, compuesto en su mayoría de señoras
charlatanas y fastidiosas, a quienes todo el espacio de pasillos y
escaleras les parecía poco para sus faldamentas, chales y cintajos.
Cerca ya de la salida, tropezaron con \emph{Edipo}, el polizonte, y
Calpena, que ya estaba familiarizado con su presencia en calles, cafés y
teatros, le dijo, permitiéndose tutearle: «Sí, aquí estoy\ldots{} No me
escapo, hombre\ldots{} Puedes apuntar, por si no lo sabes, que esta
mañana estuve con Iglesias en el café de Solís, y que hablamos de la
inmortalidad del cangrejo y de la absoluta impertinencia de los
empleados de la policía.»

---No voy contra usted, Sr.~D. Fernandito---replicó el corchete risueño
y humilde.---Viva usted mil años, para que proteja a los pobres el día
que venga alguna tremolina.

---¡Lo que es a ti\ldots! ¿A que no me aciertas dónde estuve hoy cuando
salí del café de Solís?

---En la corbatería de Aguayo.

---¿Y antes de ir al café?

---En la peluquería de Cortina.

---Maldito seas, y quiera Dios que te pase lo que al Rey de teatro que
te ha dado su nombre.

---Era un Rey que padecía de la vista.

---Ciego te vea yo\ldots{} Bueno. Pues si me aciertas a dónde iré esta
tarde, te regalo una docena de puros.

---¿De veras? Pues ya puede ir por ellos. Tráigamelos escogidos, de la
fábrica de Sevilla, de a tres cuartos pieza.

---Antes adivíneme lo que haré esta tarde.

---No necesito adivinarlo, porque lo sé, y usted no.

---¿Y cómo es eso de ignorar a dónde voy, teniendo el propósito de ir a
una parte?

---Muy sencillo. Puede que usted tenga la intención de emplear la tarde
en picos pardos, y puede que haya hablado de eso con Iglesias, que es
muy aficionado a las madamas. Pero aunque el Sr.~D. Fernando tenga esos
planes, no irá a donde piensa, sino a donde yo sé.

---Explícame eso, \emph{Edipo} maldito, o aquí perece un Rey de Tebas.

---Pues\ldots{} esta mañana, mientras el señor andaba de ceca en
meca\ldots{} fue a buscarle a su casa, tres veces, D. Carlos Maturana.
Me le encontré en la calle de Peligros, y me ha dicho que tiene
precisión de cazarle a usted hoy, y que le cazará, aunque sea con
perros.

---¿A mí?\ldots{} ¡Maturana! Sí, sí, es el pariente de mis amigos de
Olorón, a quien me recomendaron. No le he visto aún, porque estaba
ausente de Madrid cuando yo llegué.

---Ayer regresó de sus viajes por Italia y Suiza. Traerá relojes y
abanicos\ldots{} En fin, no sé\ldots{} El motivo de buscarle con tanta
prisa es porque usted trajo un encargo para la Zahón.

---El cual aún está en mi poder, porque esa señora, que me han dicho es
muy cargada de espaldas, no ha ido a recogerlo.

---Pero va de orden suya el Sr.~Maturana, no sólo por el gusto de verle
a usted, sino por llevarle a la calle de Milaneses, donde le espera con
la cajita Doña Jacoba, que no puede salir. Y como el encarguito será de
valor, no tiene el Sr. D. Fernando más remedio que hacer la entrega por
sí mismo, y fastidiarse, y echar la tarde a perros.

---Eso no\ldots{} Con entregar la caja, pedir recibo, tomarlo\ldots{}

---Puede que le entretengan a usted más de lo que piensa las joyas que
hay en la casa.

---No soy aficionado\ldots{}

---Eso se verá\ldots{} cuando lo vea\ldots{} Hay brillantes, perlas,
corales, de los que pintan los poetas\ldots{}

Y sin decir más, dio dos palmadas a Don Fernando, despidiéndose con
palabras de premura: «Con Dios\ldots{} Hago falta dentro\ldots{} Mucha
gente, y alguna no de lo mejor.»

Reuniose Calpena con sus amigos, que en la puerta hablaban con dos
sargentos de la Guardia Real, conocidos de Milagro, y se fueron hacia la
calle de Alcalá, rumbo al Caballero de Gracia.

\hypertarget{xvii}{%
\chapter{XVII}\label{xvii}}

Exactísimos eran los informes de \emph{Edipo}, y cuando llegó D.
Fernando a su casa, díjole la chica de la patrona, al abrirle la puerta,
que un señor que había estado tres veces por la mañana, le aguardaba
sentadito en la sala, al parecer dispuesto a no moverse de allí mientras
no lograra su objeto. Minutos después hallábase Calpena frente a un
sujeto como de sesenta años, acartonado y pequeñito, que llevaba muy
bien su edad; mejor afeitado que vestido, pues su levita era de las
contemporáneas de la paz de Basilea; el pelo entrecano y nada corto, con
ricitos en las sienes, y un mechón largo cayendo hacia el cogote, como
si aún no se hubiese acostumbrado a prescindir del coleto; los ojos
reforzados con antiparras de cristales azules montados en plata; el
perfil volteriano, el habla cascada y lenta.

«¿Con que es usted\ldots? Bien, hijo, bien. Pues me escribió mi sobrino
Felipe; pero hasta ayer no he llegado de mis correrías por el
extranjero\ldots{} Aquí me tiene el Sr.~D. Fernando a su disposición. La
verdad, poco puede hacer por usted este pobre viejo, pues desde que salí
de Palacio\ldots{} ya sabe usted que era yo primer diamantista de Su
Majestad\ldots{} llevo una vida\ldots{} Sentémonos, si usted
quiere\ldots{} Pues perdí aquella plaza, después de treinta años de
honrados servicios\ldots{} y no he tenido más remedio que buscar en el
comercio un modesto pasar\ldots{} Ello fue\ldots{} no sé si estará usted
enterado\ldots{} por malquerencia de esa farolona de \emph{la}
Carlota\ldots{} la mujer del Don Francisco\ldots{} otro que tal\ldots{}
En fin, más vale no hablar\ldots{} Y usted, ¿qué me cuenta? ¿Qué tal le
va por Madrid? ¿Ha conseguido que le coloquen? Ay, señor mío, esto está
perdido con tantas libertades, y la dichosa Pragmática Sanción, que fue
la manzana de la discordia\ldots{} Al Rey le mataron a disgustos, puede
usted creerlo\ldots{} Y a mí\ldots{} toda la inquina que me tomaron fue
por la amistad que me tenía el Príncipe de la Paz primero, y después el
señor Duque de Alagón\ldots{} No sé si sabrá usted que D. Pedro Labrador
me llevó consigo al Congreso de Viena; sí señor\ldots{} Pero estas son
historias marchitas, y usted es joven, vive en lo presente, y le
aburrirá esta manía que tenemos los viejos de revolver la hoja seca del
pasado\ldots{} En fin, vamos al asunto.»

---Ello es que yo---dijo Calpena un tanto impaciente por despachar
pronto,---no he podido entregar\ldots{}

---Ha hecho usted perfectamente. Encargos de cierta naturaleza no deben
entregarse sino en la propia mano de la persona a quien van dirigidos.
La mayor parte del contenido de la cajita que confió a usted
\emph{Aline} es para mí; el resto, para Jacoba. Esta se halla enferma
con un dolor tan fuerte en la cadera, que no puede moverse.

---Iré yo a su casa, si a usted le parece bien.

---Tan bien me parece, que traigo esta comisión, con la cual mato dos
pájaros de un tiro. Cumplo con Felipe, ofreciendo a usted mis servicios,
y cumplo con Jacoba, llevándole el encargo, y el portador y todo, para
que llegue más seguro.

Deseando abreviar, Calpena sacó la cajita, y propuso al Sr.~de Maturana
marchar sin pérdida de tiempo. No deseaba otra cosa el antiguo
diamantista, y se echaron a la calle, no sin que en el portal
recomendase D. Carlos a su acompañante que tuviera mucho cuidado con lo
que llevaba, pues Madrid estaba infestado de rateros, y al menor
descuido le dejarían con las manos limpias. Procuró Calpena
tranquilizarle, y asegurando bien el bulto bajo el brazo derecho, avivó
el paso. Poco hablaron por el camino, y en cinco minutos se plantaron en
la calle de Milaneses. «Amiguito, vaya un paso que tiene usted---dijo el
vejete, fatigadísimo, al entrar en el portal.---Ya se ve\ldots{} un paso
de veinticinco años. Subamos ahora despacito, que por aquí no hay
peligro y no vamos a apagar ningún fuego. Esta maldita escalera no tiene
pasamanos, y usted me ha de permitir que le coja del brazo. Pásmese
usted. En esta casa\ldots»

Se paró en el rellano, donde apenas cabían los dos. La escalera, que
arrancaba casi en la misma puerta de la calle, ascendía obscura,
desigual, angulosa, como los senderos de la traición, y sus escalones
patizambos ofrecían al confiado pie celadas espantosas.

«En esta casa\ldots{} no, en la de al lado, trabajamos juntos, cosa de
un mes, Leandro Moratín y yo. Y enfrente, en el que entonces era número
14 de la manzana 71, tuve yo el gusto de cobrar el primer dinero que
gané en mi vida. Fue por unas arracadas que hicimos para la infanta doña
María Josefa, el año 90\ldots{} Ea, cinco escalones más y llegamos.»

Tiró Maturana de la campanilla, y al poco rato rechinó la tapa de la
mirilla con cruz de hierro. Vio Calpena unos ojos; el viejo no dijo más
que «yo,» después de lo cual empezó a sonar un claqueteo de cerrojos, al
que siguieron vueltas de una llave, luego roce de cadenas, el caer de
una barra, y aun después de todo este estruendo carcelario la puerta
tardó un ratito en abrirse. ¿Era un hombre el que abría, era una mujer?
Fernando no se enteró, porque si el aspecto podía pasar por varonil en
la penumbra del pasillo, femenina era la voz que dijo: «D. Carlos, no le
esperaba tan pronto. La señora duerme, y yo estaba en la cocina
echándome unas piezas a la chaqueta\ldots{} Pasen, pasen. ¿Despierto a
Doña Jacoba?»

---No, déjala que descanse. Aguardaremos. ¿Y Aurorita, qué hace?

Replicó el mancebo (pues hombre era por la facha, aunque la voz de tiple
lo contrario declarase), que la tal Aurorita había salido de paseo con
la señora y niñas de Milagro, y con otras cuyo nombre no recordaba,
hermanas de un sargento de la Guardia Real; y en tanto, abría la puerta
de la sala, que más bien era tienda, por las dos mesas, con trazas de
mostradores, que en ella había, y los armarios de forma pesada y
robusta, cerrados con fuertes herrajes, guardando con avaricia sigilosa
tesoros o secretos. Dos o tres sillones de vaqueta, de un uso secular,
claveteados y lustrosos, y un par de sillas, eran los únicos muebles que
en tan extraña sala brindaban comodidad al visitante. Acomodose Maturana
en un sillón, y Calpena en una silla, dejando al fin sobre la mesa su
enojosa carga, y aguardaron silenciosos, hasta que el diamantista,
sacando su tabaquera de concha, tomó un polvito, después de ofrecer al
joven, que hubo de excusarse graciosamente. La conversación se reanudó
en el mismo punto en que había quedado al subir la escalera. «La buena
señora---dijo Maturana oliendo el rapé con la mayor finura y
encandilando los ojuelos,---se empeñó en que todo había de ser
zafiros\ldots{} y mi padre y mis tíos estuvieron tres meses y medio
buscándolos de gran tamaño\ldots{} Y que escaseaban en aquel tiempo los
zafiros y se pagaban bien, como ahora las esmeraldas.»

---Escasean las esmeraldas\ldots{} ya---dijo Calpena, sólo porque la
cortesía le obligaba a decir algo.

---Se han pagado en los últimos años a doce y catorce duros quilate, las
de buen tamaño\ldots{} ya ve usted. Algo bajaron de precio cuando D.
Pedro de Portugal vendió su soberbia colección, en los apuros de la
Regencia en la Islas Terceras\ldots{} Y a propósito\ldots{} Este
recuerdo de D. Pedro y Doña María de la Gloria (que por cierto ha
recuperado parte de las esmeraldas y aguamarinas de la Corona de
Portugal); este recuerdo, digo, me trae a la memoria al Sr.~de
Mendizábal\ldots{} ¿Es cierto que usted\ldots? Si es impertinente mi
pregunta, no digo nada.

---Hable usted.

---Es que\ldots{} me habían asegurado que es usted el ídolo del señor
Ministro; el niño mimado, vamos\ldots{}

Apresurábase D. Fernando a desmentir tan absurda especie, que no por
primera vez oía, y cuyo origen atribuyó a las hablillas y murmuraciones
oficinescas, cuando sintieron ruido y voces en las habitaciones
inmediatas. Maturana se acercó a la puerta, y entreabriéndola, dijo:
«¿Qué es eso, Lopresti? ¿Se levanta la señora?» Y la voz de tiple
contestó desde dentro: «Allá va\ldots» Momentos después, entraba en la
sala Doña Jacoba Zahón, apoyada por la izquierda en el fámulo, por la
derecha en un grueso bastón, y con difícil paso, marcado por lamentos y
suspiros, llegó hasta soltar sobre un sillón la dolorosa carga de su
cuerpo. Antes de saludar a Calpena, despidió al de la voz aguda con
expresiones displicentes de ama de casa que gasta mal genio: «Entretente
ahora con tus costuras, y olvídate de tus obligaciones, como ayer, que
nos diste de cenar a las nueve de la noche\ldots{} ¡Ah, si yo recobrara
mi salud y pudiera estar en todo, cómo te haría andar derecho!\ldots{}
Anda\ldots{} holgazán, lávame los pañuelos\ldots{} A las seis, el vinito
con la medicina\ldots»

Volvió después su rostro hacia Calpena, y le saludó con graciosa
sonrisa, mostrando al joven su senil y enfermiza hermosura, que
enormemente contrastaba con su desgraciado cuerpo. Ofrecía su cabeza un
exactísimo parecido con la de María Antonieta; mas por el color exangüe
y la extremada delgadez del interesante rostro era la cabeza de la
infeliz Reina después de cortada, tal como nos la ha transmitido la
auténtica mascarilla de cera existente en un célebre Museo. D. Fernando
sintió frío al contemplar aquel rostro tan fino y transparente, de un
perfil distinguidísimo, apagados los ojos, lívido el labio, mostrando
una dentadura en buena conservación. El cabello era gris, y para que
resultara mayor la terrible semejanza con la decapitada Reina, se
sujetaba dentro de una escofieta blanca. El cuerpo no debiera llamarse
feo, sino monstruoso: cada hombro a diferente altura, corvo el espinazo.
Se envolvía en una cachemira muy usada, bajo la cual aparecían la falda
de estameña obscura, y los zapatos de paño, holgadísimos, pertenecientes
sin duda a su difunto esposo. A la cara correspondían las manos, también
de cera, finísimas, bien marcadas las falanges bajo una piel sedosa, las
uñas no muy cortas, pero limpias: lucía en sus dedos una sortija negra,
con un hermosísimo \emph{ópalo de fuego} de gran tamaño.

«Usted me dispensará, Sr.~Calpena---dijo con voz dulce, musical, que
casi daba tonos de italiano al español correctísimo que hablaba,---que
haya tardado tanto en avisarle\ldots{} Que hoy, que mañana\ldots{} Pero
la carta de \emph{Aline} llegó cuando yo me hallaba en lo peor del
ataque. Esta maldita ciática me tenía en un grito. Y el año pasado las
paletillas\ldots{} después todo el esqueleto\ldots{} Ay, si le dijeran a
usted, Sr.~de Calpena, que yo he sido una mujer esbeltísima, se echaría
a reír\ldots{} Vea usted los estragos del reuma en estos pobres
huesos\ldots{} Pues sí, \emph{Aline} me decía\ldots{} Y ayer el amigo
Maturana, al llegar de su viaje, me decía\ldots{} En fin, que celebro
infinito ver a usted en mi casa, y le agradezco la atención de traerme
por su propia mano la caja.»

Por iniciativa de Maturana, se procedió a la apertura del paquete,
rompiendo los hilos que sujetaban el papel que lo envolvía. En tanto
Jacoba continuaba: «Por el amigo Milagro he tenido noticias de usted, y
sé que está en gran predicamento con el Sr.~de Mendizábal\ldots{} No, no
lo niegue. Ya sé que es usted la misma modestia\ldots{} Pues el señor D.
Juan, en la posición que hoy ocupa, no se acordará de mí. ¡Cuántas veces
le vi en mi tienda, calle de la Verónica, esquina a la de la Carne,
donde estuvimos tres años antes de pasar a la calle Ancha! Era entonces
un muchachón de lo más alborotado que puede usted imaginarse, un
busca-ruidos, un métome en todo; ayudaba a los patriotas levantiscos que
armaban un tumulto a cada triquitraque. Bien me acuerdo, bien. Juanito
Álvarez hizo la contrata de víveres el año 23, cuando tuvimos allí
prisionero al Rey. ¡El Rey! ¡Ah!\ldots{} me parece que le estoy viendo,
con su traje de mahón, asomado a los balcones de la Aduana, mirando al
mar con un anteojo muy largo, en espera de barcos franceses o ingleses
que vinieran a liberarle\ldots{} Mendizábal empezaba entonces sus
negocios en gran escala, y, si no recuerdo mal, algo traficó en pedrería
con Londres y Amsterdam. Por si había conspirado o no había conspirado,
le condenaron a muerte, y salió de Cádiz escapado para no volver
más\ldots{} Ya, ya se acordará él de los Zahones, y de los refresquitos
de sangría que le hacíamos en casa, cuando volvía de Rota con Jenaro
Negretti. En Rota tenían ambos sus novias, las de Urtus, dos hermanas
lindísimas. La una murió de calenturas, y la otra casó con un hermano de
este, Cayetano Lopresti, maltés, que está en mi servicio desde el año
25\ldots{} ¡Cómo se pasa el tiempo! ¡Ay, D. Carlos!, ¿qué me dice usted
de este correr de los años? El 23, cuando fue a Cádiz con la Corte,
usaba usted todavía coleta, y los chicos de la calle le hacían
burla\ldots{} ¿se acuerda?»

Más atento a lo que iba sacando del cajoncillo que a las tristes
remembranzas de su amiga, Maturana no contestó. Fijose también Doña
Jacoba en lo que el viejo ponía con religioso respeto sobre la mesa, y
alargó su mano para cogerlo y examinarlo.

«Ya\ldots---dijo,---las peinas que tanto ponderaba \emph{Aline}\ldots{}
El carey es finísimo; los diamantes valen poco\ldots{} Andanada de
veinticinco. Viene bien para completarle a la de Castrojeriz las
arracadas que quiere tomar, rostrillo y cinturón para la Virgen de
Valvanera.»

---¿Tiene bastante ya?---preguntó maquinalmente Maturana, mirando con
lente un joyel montado en plata.

---Tiene\ldots{} ¡Oh, sí!\ldots{} con lo que le vendió la Concha
Rodríguez y este, habrá bastante.

---Si no\ldots{} Yo he traído como unos veinte diamantes de
desecho\ldots{} muy propios para Vírgenes y Niños Jesús\ldots{} Vea
usted, Jacoba, vea qué hallazgo\ldots{}

---¿Qué?\ldots{} ¿qué es eso?

---Esto es un joyel de los que se usaban en los peinados Pompadour,
convertido en alfiler de pecho con poco arte: conozco esta prenda como a
mis propios dedos. No me equivoco, no: es la misma. Esmeralda
\emph{hialina} del Perú, superior, con cerco de brillantes en plata.
Catorce brillantes, dos de ellos de bajo color, y otro con pelo\ldots{}
Es la misma joya, la que perteneció, con otras del propio estilo, a la
Vallabriga, la esposa del Infante D. Luis\ldots{} Todo se vendió en
París el año 8; luego hubo algún descabalo, porque Montefiori cedió en
Metz los pendientes de este mismo juego\ldots{} Juraría que este joyel
lo compró el corredor de \emph{Aline} en Alsacia: los judíos alsacianos
poseían mucha piedra procedente de España, no sólo de la Grandeza, sino
de la de Godoy y Pepita Tudó.

---Es muy lindo\ldots{} Lástima no tener las otras piezas---dijo la
Zahón, examinándolo sin lente, con ojo muy perito.---Esto viene para
usted. Para mí ha de haber un saquito con varias piedras sueltas:
venturinas, turquesas, algunos brillantes\ldots{}

---Aquí lo tiene usted---indicó Maturana, vaciando el saquito en la
palma de su mano.---¡Caramba, qué hermoso brillante!\ldots{} Talla de
Amsterdam, sesenta y cuatro facetas\ldots{} Vea usted qué tabla y qué
culata\ldots{} Este otro amarillea un poco. No daría yo por el quilate
de este ni tampoco cincuenta duros\ldots{} Las turquesas me gustan, y si
usted quiere me quedo con ellas. Tengo yo dos hermanas de estas, tan
hermanas, que no dudo en asegurar que proceden de Venecia, como las
mías, y que pertenecieron a una dama italiana, no me acuerdo el nombre,
de la cual se dijo si tuvo o no tuvo que ver con Massena\ldots{} Estas
\emph{rosas} valen poco\ldots{} Todo es género corriente recogido en el
Bearnés y Languedoc\ldots{}

Pasando de la mano del viejo a la de doña Jacoba, esta lo examinó
fríamente, diciendo: «El brillante bueno no tendrá menos de cinco
quilates y tres cuartos.»

---Lo tomará la de Gravelinas, que ya reúne seis iguales, con el último
que yo le vendí.

---No quiero nada con la Duquesa, que aún me debe la mitad del collar de
perlas. Lo reservo para un parroquiano que sabe apreciar el artículo, y
es caprichoso, espléndido\ldots{}

---Ya sé quién es. Mucho ojo, amiga Jacoba. No cuente usted con las
esplendideces de los que tienen su fortuna en América, en negros y caña
de azúcar. A lo mejor, saldrán estos señores exaltados con la supresión
de la esclavitud, y la plumada de un ministrillo dejará en cueros a más
de cuatro que apalean las onzas\ldots{} Y usted, Sr.~Calpena, ¿se aburre
viéndonos examinar estas baratijas?

---¡Oh!\ldots{} es muy bonito---dijo Fernando;---¡pero cuántos años de
revolver piedras entre los dedos para llegar a adquirir esa práctica,
ese conocimiento\ldots!

---La costumbre\ldots---indicó la Zahón.---Desde muy niña ando yo en
este comercio\ldots{} y créalo usted, si dejara de ver piedras y de
sobarlas y de jugar con ellas, me moriría de fastidio. Ya mis dedos las
conocen solos, y casi no necesito mirarlas para saber lo que valen.

---Yo también, desde que me destetaron, Sr.~D. Fernando, o poco después,
manejo estos pedazos de vidrio.

---Para mí, lo parecen.

---Y lo son: vidrio fabricado por la Naturaleza en el horno de los
siglos\ldots{} ¡Ah!\ldots{} ¡oh!, atención. Aquí viene lo bueno.

Al decir esto, sacaba un objeto estrecho, largo como de una cuarta,
envuelto en finísimas túnicas de papel de seda. Era un abanico, obra
estupenda del arte francés del siglo pasado. Desplegando cuidadosamente
el varillaje de calado nácar, obra de mágicos cinceles, y el país
pintado en cabritilla, ideal escena de marquesas pastoreando en jardín
de amor, entre sátiros, \emph{pierrotes} y caballeros con pelliza,
Maturana lo mostró abierto, sutilmente cogido por el clavillo de oro, a
los asombrados ojos de Doña Jacoba y Calpena, quienes se maravillaron de
obra tan bella y sutil.

«Esta es una de las piezas más admirables que existen en el mundo, en el
ramo de abaniquería---dijo el diamantista, ronco de entusiasmo y del
gozo que le producía el arrobamiento de los dos espectadores.---Fíjense
en esas varillas, que parecen hechura de los ángeles, y no tienen el
menor desperfecto; fíjense en la pintura, en esas caras, en los ropajes
y en el paisaje del fondo\ldots{} observen las ovejitas, que no parece
sino que oye uno sus balidos\ldots{} Pues si notable es esta pieza por
su arte, no lo es menos por su historia, que voy a contar.»

Envolvió de nuevo el abanico en sus fundas finísimas de papel, y
poniéndolo sobre la mesa, protegido por su mano izquierda, se lanzó con
vuelo atrevido a los espacios de la Historia.

\hypertarget{xviii}{%
\chapter{XVIII}\label{xviii}}

«Hiciéronlo Lancret y Lefebvre para la Reina María Leczinska, por
encargo de Su Majestad Luis XV, y naturalmente, apenas concluido, Madame
de Pompadour se dio sus mañas para apropiárselo. En el zócalo de la
columnita que habrán ustedes visto en el país, a la derecha, pusieron
los artistas la divisa de la cortesana, que dice: \emph{virtus in
arduis}. A la muerte de esta señora, pasó el abanico por sucesivas
ventas a la Marquesa de Maurepas, y luego se nos pierde en el laberinto
de la Revolución francesa, hasta que reaparece en Coblentza, donde lo
compra un mercader italiano y lo lleva a Nápoles. Qué vueltas dio por
los aires de mano en mano hasta venir a las del Príncipe de la Paz en
1805, yo no lo sé, ni creo que nadie lo pueda averiguar. Lo que afirmo
es que lo usó Su Majestad la Reina María Luisa. El año 8, por Marzo,
hallándose la Real Familia en Aranjuez, se perdió uno de los diamantes
del clavillo, y por conducto del señor Príncipe de la Paz, vino el
abanico a mis manos para la reparación consiguiente. Entonces ¡ay!, lo
vi por primera vez, y quedé prendado de su mérito. A los pocos días de
tenerlo en mi taller, lo entregué compuesto a Su Alteza; mas la
Providencia no favoreció al pobre abanico, pues antes de que el Príncipe
pudiera devolverlo a la Reina, sobrevinieron los terribles sucesos del
día de San José. A Godoy por poco le matan. Los amotinados saquearon el
Palacio y pegaron fuego a los muebles\ldots{} ¡qué dolor! Era de temer
que el precioso objeto fuese a parar a manos viles, a personas
ignorantes que desconociesen su valor\ldots{} Pues no, señor. A fin del
mismo año de 1808 reaparece en poder del mariscal Soult, hombre
inteligente, soldado artista, que lo estima como merece, y se lo regala
a Napoleón en Enero del año siguiente. Enviado a Josefina con otros
obsequios, esta lo regala a su hija Hortensia, Reina de Holanda, que lo
lució en una ceremonia, a la cual dicen que fue a regañadientes: el
bodorrio del Emperador con la Archiduquesa de Austria. Después de
Waterloo, todo fue peripecias y saltos terribles para el señor abanico,
que tuvo en poco tiempo distintos dueños. Primero, un anticuario
holandés, que lo vende a la princesa Stolbey, fallecida en Baviera el
año 20; segundo, el príncipe Carlos de Baviera, emparentado con Eugenio
Beauharnais; tercero, otro anticuario, de Nancy, que lo lleva a París,
lo hace restaurar, y consigue venderlo a precio exorbitante a un
desconocido, que obsequia con él a Mademoiselle Mars en una
representación de no sé qué tragedia\ldots{} No sé si sabrán ustedes que
la célebre actriz es muy aficionada a los brillantes, y tenía colección
de ellos por valor de ochocientos mil francos; no sé si sabrán también
que el año 27 le hicieron un robo de alhajas, valor de trescientos mil
francos. ¡Pues no ha metido poca bulla ese proceso, que creo no ha
terminado todavía! Parecieron los ladrones; pero las piedras no. Pues
bien: deseando esa señora reponer los brillantes que le quitaron y no
disponiendo de dinero suficiente, hizo varios cambalaches con Bertín y
con los hermanos Rosenthal, sucesores del famoso Bœhmer, y en uno de
estos cambalaches sale otra vez al mercado el famoso abaniquito. Desde
entonces puse yo en él los cinco sentidos, deseoso de comprarlo: ha
pasado por manos de diversos marchantes; fue a tomar aires por Alemania
y Suecia; en cuatro años ha pertenecido a un Poniatowsky, a una gran
Duquesa de Hesse y a un coleccionista que vive en la Selva Negra, el
cual murió el año pasado, y su heredero, que era el santísimo Hospital
de Tréveris, hizo almoneda de todo. Vuelve mi abanico volando al
mercado, y en Lyón se posa en casa de mi amigo Jobard. Trato de cazarle
allí, y Jobard, que es de los que persiguen gangas, me toma a mí por un
inocente y quiere explotarme. Finjo desistir del empeño, y me marcho
tras de otros asuntos; pero sabiendo de buena tinta que el marchante
lionés se tambalea, doy el encargo al amigo Montefiori, de Burdeos, para
que esté a la mira y aproveche la ocasión\ldots{} La ocasión llegó, y
hace tres meses fue adquirida, por cuenta mía, la famosa prenda por la
mitad de lo que le costó al adorador de Mademoiselle Mars\ldots{}

---De lo que usted nos ha contado, por cierto muy bien---dijo Calpena,
que había oído con deleite,---se saca la consecuencia de que hay objetos
inanimados, cuya historia es más interesante que la de muchas personas.

---Eso, admitiendo que sean verdad todas esas traídas y llevadas del
abanico---observó la Zahón, escéptica, desdeñosa, pues no le gustaba que
su colega supiese más que ella en tales materias.---No se fíe, D.
Fernando, que este Maturana le compone su historia a cada pieza que
vende, forma especial suya de hacer el artículo.

---En esto---dijo Maturana riendo,---me ganaba su marido de usted,
Jacoba. Recuerdo que tuvo una pareja de diamantes, que había sido del
Tamerlán, después de Antonio Pérez, y últimamente de Godoy\ldots{} Ya se
sabe: todas las joyas de precio que han salido a la venta del año 8 acá,
se le han colgado al pobre D. Manuel.

---Pues ese abanico---afirmó la Zahón displicente y maligna, entornando
los ojos,---no se vende en España, tal como están hoy las cosas, aunque
lo adornen con más historias que tiene el Cid.

---Este abanico---replicó Maturana, acariciando la joya,---lo vendo yo
en España, y al precio que me dé la gana, señora Doña Jacoba, aunque
usted no quiera\ldots{} ¿Cree usted que voy a ofrecérselo a esos
pelagatos del Estatuto, o a las señoras de los patriotas, que apenas
tienen para poner un cocido?

---Pues a la Grandeza la verá usted completamente acoquinada con estas
revoluciones y estas guerras malditas. ¿Dinero? Poco hay, o es que no
quieren gastarlo. ¿Gusto? Ya sabe usted que aquí no privan más que las
apariencias baratas\ldots{} Vaya, D. Carlos, no ande con misterios, y
díganos que piensa encajarle su abanico a la Reina Gobernadora.

---¡Oh!, no hay otra mujer en el mundo---observó Calpena con
entusiasmo,---que sea digna de tal joya.

---Eso sí\ldots{} Sabe apreciar lo bueno. Pero yo pongo mi cabeza a que
si D. Carlos le propone el abanico, ofrecerá por él una miseria.

---Su Majestad es artista, y además espléndida, generosa\ldots{}

---¡A quién se lo cuenta!\ldots{} ¡Ay, ay! Lo fue, sí, señor---dijo la
Zahón amargando el concepto con quejidos.---Lo fue\ldots{} ¡Dios me
favorezca, ay!\ldots{} pero desde que ha empezado a soltar hijos, se ha
vuelto muy roñosa.

---¡Si no ha tenido más que uno!

---Y lo que ha de venir\ldots{} ¡ay! Está ya de cinco meses,
¡ay!\ldots{} Dos años de casada lleva por lo secreto, según dicen, y al
paso que va, no habrá bastantes rentas para el familión que nos traerá
esa señora\ldots{} ¡Y ese Don Carlos, bobalicón, todavía piensa que le
va a comprar\ldots{} ese juguete!

---Este juguete, y cuanto yo quiera---afirmó el diamantista con
seguridad burlona, casi insolente,---me lo comprará la Reina, y me lo
pagará como a mí me convenga.

---Ciertamente---dijo Fernando.---La Reina está obligada a proteger las
artes\ldots{} y es su deber formar colecciones, que luego pasan a los
Museos.

Era la Zahón envidiosa, y su egoísmo comercial no toleraba que otro del
gremio, aun siendo amigo suyo, hiciese mejor negocio que ella. La
seguridad que mostró Maturana de vender en Palacio con ventajas grandes,
la sacó de quicio; exacerbados sus dolores por la emulación mercantil,
empezó a dar chillidos, y entre ellos iba soltando estas palabras:

«No, no\ldots{} no puede ser\ldots{} Maturana loco\ldots{} Reina no
compra, Reina guarda dinero.»

---Si María Cristina guarda el dinero---afirmó Maturana frío y cruel,
pues cuando se proponía humillar a su rival no conocía la
compasión,---lo sacará de las arcas para dármelo a mí\ldots{} Su
Majestad me comprará todos los objetos y joyas de mérito que yo le
lleve, y a usted no le comprará nada\ldots{} a usted nada\ldots{} a mí
todo.

---Bruto\ldots{} majadero y vanidoso\ldots{} ¡Ay, me muero!\ldots{} Este
dolor para usted\ldots{} para usted debiera ser.

---Gracias\ldots{} no me conviene el artículo.

---¡Vaya con D. Carlos!\ldots{} Ahora sale con que tiene vara alta en
Palacio\ldots{} con que le ha caído en gracia a la Reina\ldots{} ¡Ja,
ja!\ldots{} ¡Ay, ay!\ldots{} Me río llorando, ¡ay de mí! ¡Bien por el
nuevo favorito!

---Favorito soy\ldots{} en mi ramo, se entiende. Y la Reina Gobernadora
me favorece, porque me necesita\ldots{}

---¡Le necesita!\ldots{} Buenos estamos. ¿Cree usted que la Señora
piensa encargarle arreglos y composturas? ¡Si la moda reinante es volver
a lo antiguo!

---La Reina no me ha llamado para ninguna chapuza.

---¿Luego Su Majestad le ha llamado a usted?---preguntó Calpena,
mientras Doña Jacoba, estupefacta, no sabía qué decir.

---Sí, señor, he tenido esa honra. ¿No llamó a Mendizábal para arreglar
la Hacienda y salvar el país? Pues a mí, que en mi ramo soy tanto o más
que Mendizábal en el suyo, me llama también la Corona\ldots{} para fines
no menos altos.

---¿Y qué tiene que ver nuestro ramo, la joyería, con nada de lo que
está pasando en España?

---¿Qué tiene que ver\ldots? Llega un momento, en las peripecias de un
reinado, en que el arte del diamantista puede auxiliar poderosamente a
la Monarquía.

---¡Ay, ay!\ldots{} Este hombre quiere volvernos locos\ldots{} D.
Fernando, no le haga usted caso\ldots{} Se burla de mí, y quiere ponerme
peor haciéndome reír.

---Ríase usted o llore todo lo que quiera.

---No lloro, no, ni me río---indicó la Zahón altanera y burlona.---Estoy
indignada por la falta de respeto con que habla usted de la Reina. ¡Pues
no dice que le ha llamado!

---Seis veces han llegado a mi casa criados palaciegos preguntando
cuándo venía del extranjero el Sr.~Maturana\ldots{} y el Intendente ha
estado a verme hoy\ldots{} No, si no he de decir para qué me quiere Su
Majestad. A su tiempo se sabrá.

---Ya\ldots{} Es que quiere encargar una corona morga\ldots{} nática, o
como se diga, para el Muñoz---dijo la Zahón venenosa, echando por los
ojos toda su envidia, mezclada con su agudo sufrimiento.---Me voy a
poner muy mala\ldots{} Ya lo estoy. Este hombre me irrita\ldots{} Me
cuenta cosas que no me importan\ldots{} Me ahogo\ldots{}
¡Lopresti\ldots{} condenado Lopresti\ldots{} que me muero!\ldots{} ¡La
taza de vino, los polvos, esos polvos\ldots{} Lopresti!

Entró al fin el fámulo, avisado por los gritos de su ama, y le dio a
beber una pócima de vino y caldo, en la cual vertió el contenido de una
papeleta de farmacia.

«¡Qué amargo está!\ldots{} ¡No lo has revuelto, condenado!---dijo la
señora bebiendo a sorbos.---Ahora te traes una luz: ya no se ve\ldots{}
¿Y ha sacado las perlas que vienen para mí, D. Carlos?»

---Aquí están\ldots{} Que traigan luz. Quiero verlas.

Traída la luz, examinó Maturana las perlas, y debió encontrarlas
excelentes, porque al punto formuló esta proposición:

«Al precio que usted sabe, Jacoba, me quedo con ellas\ldots{} Vaya, para
que usted no chille, en esta partida llego hasta los cuarenta y dos por
quilate.»

---Para usted estaban.

---Tiene usted mucho género, Jacoba, género superior, y no sé cómo va a
salir de él.

---Mejor\ldots{} Ea, no empiece a camelarme, que no las cedo.

---¿A ningún precio?

---A ningún precio. Quiero reunir más.

---Y va de historias\ldots{} Estas perlas que le manda a usted
\emph{Aline}, parécenme\ldots{} no puedo asegurarlo\ldots{} pero me da
en la nariz que son las de la Princesa de Beira. Tantas ganas tiene la
buena señora de ser reina, que vende sus perlas para comprar pólvora y
cartuchos.

---Podrá ser\ldots{} A usted le llaman las reinas que gobiernan, y a mí
quizá me llamen\ldots{} y me necesiten\ldots{} las destronadas.

Dijo esto la Zahón sólo con el objeto de poner en confusión a su amigo y
desorientarle. Seguía D. Carlos la broma, sin conseguir sofocar con su
donaire el humorismo maleante de la vieja, cuando esta saltó de
improviso con un recurso que a las mientes le vino en lo mejor de su
charla, y era recurso de ley, fundado en algo verídico, ignorado del
astuto D. Carlos.

«Amigo Maturana, no le he dicho lo mejor: me ha escrito
Mendizábal\ldots{} ¡Vaya una cara que pone usted!\ldots{} Sí, señor, me
carteo con el Ministro. Y si no lo cree, aquí está su secretario
particular, que no me dejará por mentirosa\ldots»

---No sé\ldots---balbució Calpena.---Sin duda es cierto\ldots{} Creo
haber oído algo al amigo Milagro.

---A Su Excelencia le da por las botonaduras llamativas---dijo Maturana
mirando fijamente a su colega, no sin malicia.---Pero ya caigo: si el
Ministro se cartea con usted, será porque quiere consultarla sobre ese
plan de vender los bienes de los frailes.

Y volviéndose hacia Calpena, le preguntó: «Joven, ¿y será cierto que
vende también las alhajas de los santos, y la plata y oro de las
catedrales?\ldots{} Porque con tal medida, si a ella se resuelve, sí que
podría sacar de apuros a la Tesorería.»

---No he oído nada de eso---replicó D. Fernando.---Parece que se
venderán todos los bienes raíces del Clero, y además las campanas.

---Que son los bienes aéreos\ldots{} ¡Buena se va a armar! ¡Será sonada!
Créame usted, Jacoba: si no trasladamos nuestro negocio al extranjero,
estamos perdidos.

---Yo no: con el arreglo que nos hará ese señor Ministro, verá usted
prosperar la nación. Usted no es partidario de Mendizábal.

---Yo creo que vale\ldots{} sí vale. Pero fracasará.

---Dios quiera que no\ldots{} Voy a entrar en negociaciones con él para
un asunto\ldots{} Y el Sr.~Calpena, que, según nos dijeron, es el amigo
íntimo del gran Ministro, ¿me hará el favor de interceder por mí?

---¿Negocitos con Mendizábal?---murmuró D. Carlos.

---Señor mío, si a usted le necesitan las reinas, a mí me necesitan los
Ministros, que en realidad son los que gobiernan\ldots{} Sr.~Calpena,
usted es muy amable, y tomará mi asunto con interés.

Excusose el joven con finura y modestia, alegando que no tenía amistad
con el Ministro, ni podía permitirse recomendarle asuntos de ninguna
clase; mas no se dio por convencida la Zahón, y elogiando la delicadeza
del joven, y echándole mucho incienso dijo: «Es natural que usted se
exprese de ese modo. Pero yo sé que D. Juan Álvarez le quiere a usted
mucho y le protege, y le hará procurador\ldots{} Los motivos de esta
protección quizás usted mismo no los sepa\ldots{} Yo tampoco; la verdad,
no sé nada: sólo sé que\ldots{} En fin, \emph{Aline} me ha dicho que es
usted un joven de gran mérito\ldots{} No hay que ruborizarse\ldots{} Por
todas esas razones, y otras que callo, yo quisiera, Sr.~D. Fernando, que
esta noche cenara usted con nosotros\ldots»

Antes que el invitado pudiese formular sus excusas, se metió por medio
D. Carlos, diciendo muy gozoso: «Aceptará, ya lo creo, y yo también.
Quiero decir, que si el señor cena con ustedes, me convido\ldots»

---Lo siento mucho---dijo Calpena.---Otra noche, señora mía, tendré
mucho gusto\ldots{} Esta noche no puedo\ldots{} créame usted que no
puedo.

---Ya se ve\ldots{} Es verdadero sacrificio sentarse a nuestra pobre
mesa, acostumbrado usted a los convites de las grandes casas.

---No nos tratarán mal aquí, Sr.~D. Fernando---dijo D. Carlos;---y si
Lopresti tuviera tiempo de poner esta noche el pescado en tomatada
maltesa\ldots{}

---Hay tiempo\ldots{} ¡Lopresti!

Repetía sus excusas D. Fernando, cuando llamaron a la puerta. El maltés
acudió. Eran campanillazos, golpes repetidos, dados al parecer con el
puño de un bastón, y luego voces femeninas, la del sirviente y la de
otra persona, riñendo, disputando. «Es ese torbellino---dijo Doña
Jacoba.---Aura, hija mía, ¿por qué alborotas? Mira que hay
visita\ldots{} pasa\ldots{} ven.»

\hypertarget{xix}{%
\chapter{XIX}\label{xix}}

En el mismo instante vio D. Fernando, en el hueco de la puerta, una
mujer, una joven, que más que persona humana le pareció divinidad bajada
del cielo. ¿La había visto antes alguna vez? Creía que sí, creía que no.
¿Y cómo había vivido tanto tiempo sin verla? ¿Y qué habría sido de él,
si por torpeza de su destino no la hubiese visto cuando la veía? Esto
pensaba en la perplejidad casi estúpida de que fue acometido su espíritu
ante aquella visión celeste. La que respondía por Aura se quedó también
suspensa, y pensaba que no veía por primera vez al sujeto, cuyo nombre
pronunció la Zahón presentándole.

«Vete adentro: deja la mantilla; deja la sombrilla con que has apaleado
al pobre Lopresti, y vuélvete acá\ldots---le dijo la señora.---No hagas
la de otras veces, que tengo que ir a buscarte. Ya ves que no puedo
moverme.»

Fuese la joven, y tal era su turbación, que ni acertó a saludar con una
ligera inclinación de cabeza a la persona que acababa de serle
presentada. «¡Qué estúpida soy---se decía, corriendo hacia su
cuarto,---y qué grosera y qué desmañada! No he sabido saludarle\ldots{}
Verdad que él no me saludó tampoco, y se quedó como un santirulico que
está en oración\ldots{} ¿Cómo ha dicho Jacoba que se llama? Pues ya no
me acuerdo\ldots{} Yo le conozco\ldots{} No, no le he visto nunca: no
hay más sino que yo sabía que le vería pronto\ldots{} ¡Y ahora qué
vergüenza me da de volver!\ldots{} No vuelvo\ldots{} ¡Pero si tardo, y
el hombre se cansa, y se va, y no vuelve más, y no le encuentro en
ninguna parte\ldots!»

En tanto Calpena, mal repuesto de su trastorno, apenas podía enterarse
de lo que Maturana y la Zahón le decían. Miraba para dentro de sí: en su
mente había quedado impresa la imagen fugitiva\ldots{} ¡Qué ojos, qué
boca, qué talle! Quería recordar pormenores; cómo eran estas o aquellas
facciones, y no podía. La imagen se borraba con el análisis; llegó un
instante en que sólo quedaba de ella una vaguedad, un rastro, algo como
una herida, o como una sombra que doliera. Pero de improviso volvió a
presentarse ante los turbados ojos de Calpena, no precedida de ningún
rumor de pasos ni de voz alguna. Entró como fantasma, trayendo consigo
una luz ideal, y para mayor asombro y arrobamiento de D. Fernando, se
presentaba risueña, mostrando unos dientes dignos de morder un cachete
al Padre Eterno. Así lo pensó Calpena, que también se sonrió al verla, y
salió como a recibirla, brindándole un asiento\ldots{}

«No me siento; gracias»---dijo Aura, y pasó\ldots{} Fue a recoger algo
al otro lado de la pieza. Cuando regresaba con una cestilla de labores,
recibió de lleno el galán todo el brillo, toda la expresión, toda la
intensísima divinidad de los ojos negros de la damisela. El infeliz no
dijo nada, miró a la mesa, y cogiendo la silla que cerca tenía, dio un
golpecito en el suelo, diciendo o pensando así: «¡Qué rayo de
Dios!\ldots{} Tempestad, locura\ldots{} Si esta mujer no me quiere, me
mato\ldots{} vaya si me mato. No puedo vivir.»

---Aura---dijo Doña Jacoba dándole un manojo de llaves.---Saca de aquel
armario la cajita de perlas, y dásela a D. Carlos para que me haga el
apartado\ldots{}

Y mientras Aura traía las perlas, Calpena se decía: «Esto es sueño. Tal
mujer no existe. Es la que traigo en mi imaginación desde qué sé yo
cuándo\ldots{} Lo que ahora me pasa es como el morir, como el nacer. No
sé si muero o nazco\ldots{} ¡Vaya una mano! Si me diera una bofetada,
vería yo a Dios en su trono\ldots{} ¡Y qué cuerpo, qué flexibilidad, qué
gallardía! Ese traje que antes me pareció verde, ahora es azul,
obscurito como un cielo sin luna, y esas motitas son como estrellas, que
en los pliegues se esconden, se apagan\ldots{} El espacio entre el borde
del vestido y el suelo parece, cuando anda, un espacio que ríe, una boca
que habla\ldots{} No sé\ldots{} estoy loco\ldots{} Si la jorobada no
repite su invitación, me convido yo mismo. Si me apalean para que me
vaya, no me voy.»

---Oye, mujer---dijo Doña Jacoba poniendo las perlas sobre un tablero
con bordes y forrado de bayeta, previamente colocado ante sí por D.
Carlos,---¿cómo es que no subieron tus amigas las de Milagro?

---Me dejaron en la puerta. Era tarde, y como las de Fonsagrada tenían
prisa\ldots{}

---¿Iban con ellas los dos chicos de la Guardia Real?

---Sí\ldots{} y también tenían prisa. Les han mandado recogerse temprano
en el cuartel. Parece que hay run-run de revolución.

---Todos los días dicen lo mismo, y nunca pasa nada. ¿No sabes, Aura? He
invitado a cenar a este Sr.~Calpena, y no quiere, digo, no puede\ldots{}
Convéncele tú.

---¿Y qué caso ha de hacer de mí?---dijo Aura queriendo mirarle y sin
poder levantar los ojos.---Estará invitado en otra parte\ldots{}
comprometido en casas ricas\ldots{}

---Si mil compromisos tuviera---manifestó Calpena haciendo por tragarse
el nudo que tenía en la garganta,---los dejaría todos por la
satisfacción, por el honor, por el placer de pasar algunas horas en tan
amable compañía.

---Gracias---dijo Aura, echándole toda la mirada y clavándosela con
ímpetu, hasta con ensañamiento.

Y la voz de Aura al decir \emph{gracias}, o al decir otra cosa
cualquiera, se le metía a Fernando dentro del sentido como una lanceta,
y le inoculaba un goce inefable, una turbación honda, ganas de dar
gritos y de tirarse al suelo\ldots{} «¿En qué
consistirá---pensaba,---que me parece que la he conocido toda mi vida?
Si me equivoco respecto a esta mujer; si no es la que yo soñé, la que ha
venido al mundo para mí, que me parta un rayo, o que me asesinen esta
noche al volver de una esquina. ¡Esta mujer para otro! No puede
ser\ldots{} Quien me lo diga miente\ldots{} y si yo lo dudara o lo
temiera, estaría loco.»

Mientras doña Jacoba daba órdenes a Lopresti, Aura y Fernando cambiaron
palabras insignificantes, sentados uno frente a otro, en el lado de la
mesa o mostrador opuesto al que ocupaba D. Carlos. Entre este y la
pareja estaba la luz, con enorme pantalla verde.

«¿También usted, señorita, entiende de pedrerías, y sabe distinguir los
brillantes legítimos de los falsos?»

---No sé nada\ldots{} Para mí como si fueran cuentas de vidrio. No
entiendo nada de esto. Y usted, ¿sabe\ldots?

---Yo no\ldots---dijo Calpena sintiendo un impulso violentísimo de
manifestarse.---No sé más sino que\ldots{} No crea usted que voy a
llamarla piedra preciosa, diamante, perla o cosa tal\ldots{} Eso es no
decir nada. Lo que digo\ldots{} Digo que cuando la vi a usted
entrar\ldots{} creí que no era usted persona de este mundo.

---¿Pues de qué mundo?

---Del otro, del Cielo\ldots{}

---¿Pero usted cree que si yo hubiera estado en el Cielo iba a dejarme
caer aquí? ¡Qué tontería!

---No haga usted caso---dijo la Zahón.---Esta niña es una revoltosa sin
juicio. Ya es tiempo de que vaya sentando la cabeza.

---Soy muy mal criada---afirmó Aura con graciosa ingenuidad, sin el
menor dejo de falsa modestia.---Vamos, que no tengo educación\ldots{} No
he tenido quien me eduque ni quien me enseñe nada\ldots{} Y ahora trato
de educarme yo misma; pero, la verdad, no sé por dónde empezar.

---¡Qué deliciosa modestia!

---¡Modesta yo! No, señor: ya verá usted cómo no lo soy. Algún mérito me
parece a mí que tengo, y como lo sé, lo digo.

---La sinceridad es la primera de las virtudes---afirmó Calpena
fascinado por los ojos negros de Aura, que no podían ser contemplados de
cerca. La ardiente admiración del joven veía en ellos tan pronto una
inmensidad de dulzura que atraía, como una inmensidad de peligro que
rechazaba. Dulzura o peligro, el hombre sentía un irresistible impulso
de comérselos, de apropiarse toda su luz, toda su pasión. ¡Y qué
perfecta armonía entre los ojos y lo demás del rostro, en el cual sólo
se veían perfecciones! El color era moreno suave, blancura encendida más
bien, como si en sus mejillas se reflejasen llamaradas lejanas\ldots{}
La frente dominaba tan hermoso conjunto con su pureza de alabastro
caldeado.

«Déjeme usted que admire---dijo Calpena en tono y actitud de
devoción,---esas cejas divinas, esas pestañas que hablan y esos labios
que miran\ldots{} No sé lo que digo.»

---Diga usted de una vez que soy muy bella\ldots{} ¿Por qué no se ha de
decir lo que es verdad? Ya ve usted cómo no conozco la modestia. El ser
bonita no tiene ningún mérito, porque así ha nacido una\ldots{}

---Aura, por Dios, no tontees\ldots---indicó Doña Jacoba levantándose
con gran esfuerzo.---Voy a ver qué hace ese pelmazo.

---¿Quieres que vaya contigo?

---No, hija: quédate aquí acompañando a estos señores\ldots{} Puedo
andar sola.

Ponía D. Carlos toda su atención en las perlas que examinaba
cuidadosamente, y luego las distribuía entres grupos. Aura y Fernando se
creían solos.

«¿Qué?---dijo ella viendo al galán suspenso y como asustado;---¿se
enfada usted porque yo misma me alabo y digo que soy hermosa?»

---No; la sinceridad\ldots{} Todo en usted es extraordinario, inaudito,
sin igual.

---No me haga usted caso. Soy muy mal educada\ldots{} La buena educación
pide que cuando una se siente discreta diga: «soy tonta,» y que cuando
somos bonitas, sostengamos que no valemos nada.

---No es eso buena educación: es gazmoñería, y falsa humildad, máscara
de la soberbia.

---A mí me han hecho creer que la verdadera finura consiste en rebajarse
y elogiar a los demás.

---¿Aunque no se sienta el elogio?

---¡Ah!, no: eso sí que no puedo hacerlo yo. Por nada del mundo le diría
yo a usted, por ejemplo, que me agrada, si no lo sintiera.

---Luego usted me dice que no le soy desagradable.

---Yo no pensaba decírselo\ldots{} Si lo he dicho sin querer, dicho se
queda.

Se le encendieron las mejillas, y después de una pausa, en que Fernando,
absorto, no sabía qué expresar, rectificó la joven su atrevido concepto:
«La culpa tiene usted por hacerme caso y darme conversación. Se me
escapan las tonterías cuando menos lo pienso. Bien dice Jacoba que no
tengo vergüenza\ldots»

---Eso no es verdad.

---Quiero decir que soy muy descarada\ldots{} Y no sabe usted los
disgustos que he tenido en Madrid por esta mala costumbre mía de decir
todo lo que siento. Mis amigas me critican, y algunas se han negado a
salir de paseo conmigo. Otras, en cuanto me han oído hablar dos veces,
se han resistido a recibirme en su casa. Vamos, que me tienen por una
salvaje, y lo soy, aunque lo disimulo vistiéndome, ya usted ve, como las
mujeres civilizadas\ldots{} Eso lo sabe una sin que se lo
enseñen\ldots{} Pero\ldots{} mire usted qué cosas tan raras me pasan a
mí: esta noche es la primera vez que siento pena de ser como soy. Al
decirle lo que le dije, ¡me subió un calor a la cara\ldots! Me figuré
que usted se enfadaba conmigo, que me iba a querer mal por mi
desvergüenza\ldots{}

---No, no, eso no. Es sinceridad, y yo la admiro y la aplaudo\ldots{}
¿Pero por qué no hemos de ser todos así? ¿Qué educación es esta que nos
impone la mentira en todos los actos?

---Pues ahora me confunde usted más---dijo Aura con una ingenuidad y una
sencillez que acabaron de enloquecer a Calpena.---Porque yo empezaba a
querer educarme procurando hacerme la vergonzosa, y usted sale ahora
diciéndome que cuanto más desvergonzada mejor.

---No, cuanto más sincera\ldots{} Lo que usted debe hacer es no
empeñarse en cosa tan difícil como la educación por sí misma. No
acertaría usted. Lo mejor es que confíe ese cuidado a otra persona: a
mí, por ejemplo.

---¿Pero cómo me va usted a educar, si no está siempre conmigo?

---¡Oh!\ldots{} eso se arreglaría de un modo muy fácil\ldots{}

---¿Cómo?

---Estando\ldots{}

---¿Siempre conmigo? Pues le juro a usted que no me disgustaría. En
decir esto no veo yo que haya maldad.

---Ninguna\ldots{}

Al llegar a este punto, miráronse los dos largo rato sin pronunciar
palabra. ¿Les estorbaba el viejo diamantista, aunque sólo en presencia
corporal, por tener todo su espíritu aplicado al examen y selección de
perlas? Calpena, perdidamente enamorado de aquella mujer con súbito
incendio pavoroso, pensaba en el singular caso, en la inaudita sorpresa
que le ofrecía su destino. Era en verdad estupendo que siendo él un
misterio vivo, y encontrándose en el mundo, en su florida edad, rodeado
de sombras, le saliese al paso, en aquella ocasión suprema de su amor
primero (el cual, por la fuerza con que venía, debía de ser único), un
enigma tan extraño como el suyo propio. «Ya sospechaba yo---se
dijo,---la existencia de esta mujer tan hechicera y seductora; ya me
anunciaba el corazón que en nuestras sociedades puede encontrarse un ser
tan bello, tan ingenuo, en toda la hermosura libre y silvestre de quien
no ha pasado por los absurdos tamices de la educación corriente. Esta
mujer superior, este admirable pedazo de la Divinidad, aunque sin
pulimento, para mí estaba guardada; para mí, que he venido al mundo en
algún torbellino de las pasiones humanas, y tengo por ley de mi destino
la misión ¿por qué no ha de ser misión?, de venir a chocar con otro
misterio como el mío, con otro enigma, y fundirnos misterio con
misterio, y\ldots» De buena gana habría roto el silencio soltándole
estas preguntas, expresión de la ansiedad de un amor investigador,
receloso, policiaco: «¿Quién eres tú?\ldots{} ¿De dónde has salido
tú?\ldots{} ¿Quiénes son tus padres?\ldots{} ¿Por qué estás en esta
casa?»

El silencio fue interrumpido por Maturana, que, mostrando entre sus
dedos una gruesa y hermosa perla, se volvió a los que ya es forzoso
llamar amantes, y en tono grave les dijo: «¡Qué hermosura, qué redondez,
qué oriente!\ldots{} ¡Y que este prodigio de la Naturaleza haya salido
de los profundos abismos de la mar!\ldots{} ¡Y que esto sea, como dicen,
una enfermedad de la ostra\ldots{} un tumor, según otros, producto de la
baba con que el pobre animal se cura de los golpes que le dan los
crustáceos! ¡Y cosa de tanto valor no es, en su origen, más que una
baba!\ldots{} ¡Misterios de la vida, del tiempo!\ldots»

\hypertarget{xx}{%
\chapter{XX}\label{xx}}

No se manifestaba en la mesa la sordidez de Jacoba Zahón, como
vulgarmente creían vecinos chismosos, y amigos desconocedores de las
interioridades de la casa. Del trato comercial procedía su fama de
avaricia, y cuanto se dijese en este terreno era poco, pues no ha venido
al mundo persona que con más cruel ahínco defendiera el ochavo. Los del
gremio la temían; gimieron siempre los parroquianos entre sus uñas
rapaces; en tratándose de negocio pingüe, no reparaba en medios, ni
había para ella compañerismo, ni delicadeza, ni caridad. Reproducíanse
en ella todas las cualidades de su marido, Bartolomé Zahón, a quien
llegó a sobrepujar en la frialdad de cálculo, en la codicia desmedida y
en la dureza de las condiciones de venta o empeño, aprovechando siempre,
sin miramiento alguno, las ocasiones ventajosas. No perdonaba; hacía
cumplir los contratos, implacable sacerdotisa de la letra, y al propio
tiempo los cumplía fielmente por su parte. Jamás la cogió nadie en
renuncio legal; jamás tuvo que ver con la justicia humana. Vivía, pues,
dentro de la estricta honradez social, del respeto de las leyes y
costumbres. No tomó nunca nada que en rigor de derecho no fuera suyo, ni
dio a nadie parte mínima de su legal pertenencia. Con tal modo de ser,
se fue labrando su fama de miseria, fundadísima en todo, menos en los
cuentos que corrían acerca de la mala vida que se daba. Como en su casa
entraban pocas personas, y las amistades y relaciones no pasaban de un
círculo estrecho, pocos sabían que la mesa de Jacoba no era escasa, que
a veces era espléndida, y que si ocurría tener que obsequiar a alguien,
lo hacía con decente abundancia y hasta con ostentación. Así queda
explicado que la cena de aquella célebre noche fuera excelente, y que
Calpena la encontrase muy superior a lo que había imaginado. Añádase que
Lopresti era un hábil cocinero, que guisaba a la italiana y a la
francesa, y poseía el secreto de algunos platos sabrosísimos a estilo de
La Valette y de Cagliari.

Por milagro de Dios, Jacoba se sintió, después de anochecer, muy
mejorada de los horrendos dolores que le habían retorcido el cuerpo, y
gozosa, renqueando de aquí para allí con el apoyo de su bastón, iba del
comedor a la cocina, o al revés; sacaba de los armarios una mantelería
riquísima (que había ido a parar allí sabe Dios cómo); exhumaba vajilla
fina, alguna hermosa pieza de plata repujada, y en fin, lo disponía todo
para lucimiento de su casa y satisfacción de su amor propio. Digase
también que Jacoba Zahón, fuera de los asuntos mercantiles, era bastante
agradable, de mucho mundo, conocedora de los usos que constituyen la
etiqueta, de hablar ameno y correctísimo. Pero estas cualidades, junto
al mostrador, trocábanse en una ferocidad egoísta que ponía los pelos de
punta al infeliz que trataba con ella. En esto seguía las tradiciones de
su familia: no hacía más que manifestarse en toda la plenitud de su ser,
heredado de otros seres, consecuente con lo que los Zahones llevaron
siempre en la masa de la sangre. Malta en tiempos remotos; después
Mallorca, Gibraltar, Sevilla, y desde mediados del siglo pasado, Cádiz,
Córdoba y Madrid, fueron campo donde esta planta Zahónica creció con
varia lozanía. Algunos se enriquecieron; otros trabajaron con mediano
fruto, y los últimos tuvieron no pocos reveses, que remedió el tino
económico de Bartolomé Zahón, y las dotes rapaces de su mujer. En la
época en que encontramos a esta señora, toda estevadita, patizamba, y
hecha una calamidad, la casa no era más que sucursal de la establecida
recientemente en Córdoba por Laureano Zahón, hijo único de Doña Jacoba y
su heredero. En Córdoba se había montado un taller, y allí se acumulaba
la pedrería más usual conforme a las exigencias de una industria y
comercio bastante activos. En Madrid sólo quedaba la compra y venta, la
red tendida para recoger gangas, todo el género vagabundo que siempre
fluctúa en grandes poblaciones; quedaban también valiosos préstamos con
prenda, que Doña Jacoba sabía hacer como nadie, a cencerros tapados, sin
pagar contribución de prestamista.

Por causa de los achaques de su madre, el Zahón de Córdoba tiraba a
suprimir completamente la casa de Madrid, llevándose todo allá, y así lo
había convenido con Doña Jacoba; pero dificultaba la traslación la plaga
de bandidos y ladrones que había por entonces en Sierra Morena, sin que
justicia, ni policía, ni aun el ejército pudiesen con ellos. El envío de
alhajas se hacía muy lentamente, aprovechando coyunturas favorables que
no se presentaban todos los días. Además, Doña Jacoba, por ley de
inercia, lo dificultaba también. El hábito de traficar, de allegar
dinero, podía más que todos los planos dictados por la razón: sin darse
cuenta de ello, dilataba las remesas, y cuando se proponía no hacer más
negocios, se le entraban por la puerta gangas increíbles\ldots{} En fin,
que la codicia y la costumbre daban un carácter de sólida petrificación
al establecimiento de la calle de Milaneses.

De las relaciones de la Zahón con Maturana conviene dar alguna noticia.
Ya se ha visto que era D. Carlos el primer perito y tasador de pedrerías
que por aquel tiempo había en España. Criado en los talleres del gran
Martínez, y trabajando de continuo para Palacio y la Grandeza, su
práctica era al fin tan notoria como había sido su habilidad. Sus viajes
frecuentes le afinaron el gusto; el trato mercantil y el roce social
hicieron de él un hombre en quien la urbanidad no desmerecía de la
inteligencia. Exonerado de su cargo de diamantista de Palacio, a la
vuelta del Rey, sin otro motivo aparente que la protección que le
dispensara el Príncipe de la Paz, hubo de lanzarse al comercio con buena
suerte: del 15 al 35 habla reunido un buen capital. No tenía taller, ni
tienda, ni le hacían falta para nada, pues procuraba colocar prontamente
el género, y remitía sus dineros a París, a la casa del Sr.~Aguado,
Marqués de las Marismas, de su absoluta confianza.

En tiempos bastante lejanos, cuando a Jacoba no le habían salido las
corcovas que agobiaban su cuerpo y afligían su existencia, y cuando
Maturana, aunque de cuerpo chico, era un hombre de alientos, no exento
de gracia, corrieron voces de si se entendía o no se entendía con la
mujer de Bartolomé Zahón; pero todo ello fue malicia, malquerencia de
compañeros envidiosos. Siempre entró D. Carlos en casa de sus amigos con
la mayor limpieza de intenciones, y si allí permanecía largo tiempo, era
por menesteres periciales y mercantiles. Vivía el diamantista
honradamente con su mujer, que nunca salió de Madrid, y tenía dos hijas,
casada la una con un teniente de la Guardia, y otra con un capitán de
lanceros.

Mirábale siempre Jacoba como un buen amigo, con quien se asociaba en
cualquier negocio que uno solo no pudiera emprender. La opinión de
Maturana en asuntos de pedrería era para ella cosa sagrada, y la
confianza entre los dos, comercialmente hablando, no se alteró jamás.
Verdad que Jacoba, como hembra envidiosa, de un egoísmo implacable, no
podía ocultar su rabia cuando Maturana hacía un buen negocio en que ella
no llevara parte, y le contradecía, le hostilizaba por todos los medios,
vengándose de su suerte con burlas y recriminaciones. Pero esto no
estorbaba para la confianza, que era incondicional, absoluta. La Zahón
le entregaba sin ningún recelo sus llaves; y él, en justa
correspondencia de esta fe ciega, le dejaba en depósito, cuando se iba
al extranjero, cosas de grandísimo valor. En suma, socios alguna vez,
rivales otras, amigos siempre.

Sentáronse a la mesa las dos damas y sus dos invitados a punto de las
nueve. Todo estaba muy bien dispuesto, aunque con un poquito de
precipitación. Pudo admirar Calpena piezas hermosísimas de porcelana y
de plata antigua; todo era heterogéneo, revelando, más que la casa del
rico, la del comerciante o el coleccionista. Uno de los candelabros de
dos velas con guardabrisas, era evidentemente de iglesia, y había
servido en mejores días para alumbrar el Santísimo; el otro de estrado
de casa grande; y por este estilo variaban las formas y abolengo de
cuanto allí se ostentaba. De lo que cenaron, nada había que decir, como
no fuera para elogiarlo sin reservas. Todo era bueno, con tendencias a
la condimentación italiana, y revelaba la mano culinaria del atiplado
maltés. La mujer, vecina del tercero, que servía, hízolo con destreza, y
Jacoba no tuvo que reprenderla más que dos veces\ldots{} por no perder
la costumbre.

Obtenida venia de sus huéspedes para no cambiar de vestido, la Zahón
ostentaba en la cabecera de la mesa su cara austriaca, su escofieta, sus
jorobas y los trapos con que las envolvía. A su derecha se sentaba Don
Fernando, a su izquierda Maturana, Aura enfrente. No apartaba los ojos,
y menos el pensamiento, de la hermosa doncella el enamorado Calpena, y
pudo observar que en el comer no revelaba salvajismo ni desconocimiento
de los hábitos sociales, sino todo lo contrario: «Ella será salvaje en
sus afectos, de inteligencia inculta; pero en sociedad sabe lo
suficiente para dar relieve a sus extraordinarias gracias
naturales\ldots{} ¡Qué mujer, Dios mío! ¿Pero de dónde ha salido este
sol que viene a alumbrar mi vida?\ldots{} Ahora veo cuanto hay en el
Universo\ldots{} antes creía ver, y no veía nada.»

Entabló Maturana la conversación hablando de perlas. «Ya le dejo a usted
los tres apartados, a saber: primera calidad, en \emph{elencos} y
\emph{avemarías}; segunda calidad, en aljófares, \emph{timpanías} y
\emph{berruecos}, y, por último, género \emph{muerto}. Otro día que
venga yo a buena hora pesaremos todo lo selecto, formando igualdades. En
el primer apartado tiene usted un par de perlas de perfecta redondez y
oriente superior, que juntas no pesan menos de 27 quilates. Sé quién
daría por ellas 350 duros. Las \emph{muertas}, si usted quiere, me las
llevaré a París, donde conozco un platero que ha descubierto la manera
de devolverles la irisación por una \emph{alquimia secreta}, en la cual
entran, según dicen, 83 drogas. Entre las \emph{avemarías} de segunda,
veo una tandita de iguales, lindísimas, que, si no estoy equivocado, son
las del medio collar que le cedió a usted Negretti, el papá de
Aurorita.»

De esto tomó pie D. Fernando para llevar la conversación a la familia de
Aura, anhelando explorar aquel interesante mundo desconocido. Algo
descubrió de lo que deseaba, y otras cosas quedaron en el misterio. Con
mucha gracia describió la joven algunos pasajes de su infancia; y
respecto a su nacionalidad, que fue motivo en la mesa de grandes
controversias, dijo lo siguiente: «Verá usted, D. Fernando, el surtido
de sangres que llevo en mis venas. Mi padre era hijo de un corso y de
una española, la cual, mi abuela, era hija de portugués, y catalana.
¿Qué tal? Pues voy ahora con mi madre. Verá usted qué lío. Mi madre era
hija de un francés y de una griega, y no había nacido en ningún país,
sino en medio de la mar, viniendo sus padres de Salónica, donde tenían
comercio de oro y plata. Yo nací en un pueblo cerca de Londres, que lo
llaman Rochester, y a los tres años me llevaron a Mallorca. De niña
hablaba inglés; pero luego se me olvidó, y sólo recuerdo algunas
palabras. De Mallorca pasé a La Valette, en Malta, donde hablé italiano,
y volví a saber un, poquito de inglés. A los diez años, vuelta a
Mallorca, después a Cádiz, y de Cádiz a Madrid, donde me parece que
estoy ahora, aunque no lo aseguro: tengo mis dudas de que esté yo ahora
donde ustedes me ven\ldots{} si es que me ven, que también lo
dudo\ldots{}

---No le haga usted caso, señor Calpena---indicó la Zahón
benévola.---Todo el día la tiene usted pensando y diciendo estas
extravagancias. Es un genio inflamado, y tan desigual, que si le da por
reír y alegrarse, nos atruena la casa con sus gorjeos; y si le da por
las tristezas y por lo fúnebre, nos pone a todos con el corazón en un
puño. Trabaja como nadie, y hace mil primores cuando le da la ventolera;
y cuando se pone a ser holgazana, no hay quien la aventaje. No es
constante más que en dos cosas: limpieza, así de su persona como de
cuanto cae bajo su mano, y caridad. No deje usted en su poder cosa de
valor, porque, de seguro, se la da al primero que se la pide\ldots{}
hablo de cosas metálicas o comestibles, ¿me entiende usted?

---Sí, señora: entiendo perfectamente.

---Oiga usted más: rarísima vez coge en su mano un libro aunque aquí no
faltan\ldots{} La hemos puesto maestro de piano y canto, y de baile.
¿Querrá usted creer que toca lindamente y que baila con toda la gracia
de Dios?

---Lo creeré si nos da esta noche una muestra de sus habilidades, en el
piano y canto sobre todo, pues la danza es más bien para lucida en
sociedad.

---¿Y si no, no lo cree? Pues no toco---dijo Aura.---Tiene que creerlo
antes. En estas cosas en necesaria la fe.

---Bueno, pues la tengo\ldots{} Sin oírla cantar, ya estoy proclamando
que se deja usted tamañita a la Todi.

---Eso es burla. No tanto, señor mío. Pero no vaya a creer que salgo
ahora con modestias ridículas. Sepa usted que canto muy bien. Digo, muy
bien no; me quedo en el bien a secas. Ni me quito ni me pongo
nada\ldots{} Pero no cantaré esta noche\ldots{} digo, sí cantaré, con
tal que D. Carlos me prometa no dormirse.

---Lo prometo\ldots---dijo Maturana,---sin responder, hija mía, sin
responder de nada.

---Yo emprendería la completa educación de Aura---dijo Jacoba, que no
sabía cómo llegar al asunto que era su objeto principal aquella
noche,---si me dieran medios suficientes para ello. Y no es que la niña
carezca de patrimonio, pues lo tiene sobrado: sólo que está en manos que
lo escatiman, que lo tasan en demasía, como si desconfiaran de
mí\ldots{} Sr.~D. Fernando, yo espero de usted un favor muy señalado. Me
consta su amistad con nuestro gran Ministro, el Sr.~Mendizábal; sé que
Su Excelencia\ldots{}

---Señora, ya dije\ldots---interrumpió D. Fernando lleno de
confusión.---El señor Ministro me trata como a todos sus subordinados,
con cortesía\ldots{} y nada más.

---A un lado las modestias, caballerito---añadió la diamantista,---y no
me salga usted con negativas, que sólo sirven para demostrarnos su
delicadeza\ldots{} Pues sí señor: espero de usted una prueba de amistad
hacia mí y de interés por Aura. ¿No adivina lo que quiero? Que usted me
ponga en comunicación con su jefe, y si es posible, y quiere extremar el
favor, que antes de llevarme a la audiencia, le hable de mí, pues me
figuro que el Sr.~Mendizábal tiene de esta servidora una idea
equivocada. Sin duda le han llevado algún cuento\ldots{} En fin, yo
quiero ver a Su Excelencia, deseo hablarle, y que usted tome mi empeño
como cosa propia\ldots{}

Interesado en el asunto, por tratarse de la mujer que le fascinaba,
Calpena quiso saber más, y descubrir qué relación podía existir entre la
hermosa hija de Negretti, nieta de tan distintos abuelos, y el gran
Mendizábal, relación cuyo simple anuncio le sorprendía y anonadaba. ¿Qué
era, Santo Dios? Sólo por tirarle de la lengua a la Zahón y adquirir
mayor conocimiento, cedió en aquel punto de sus supuestas confianzas con
el Ministro, y ni afirmaba ni negaba, dando a entender que favorecería
las pretensiones de la jorobada, siempre que se le diese alguna
explicación de ellas. Por este medio sutil pudo averiguar que D. Juan
Álvarez era testamentario de Jenaro Negretti y depositario de su
fortuna, con algo más de lo que referido queda.

No se paraba en barras la codiciosa diamantista, y desde que Mendizábal
vino a España y se puso a ministro, acarició la idea de que debía
transferirle a ella las facultades que le otorgaba el testamento de
Negretti. ¡Cosa más natural! Pues ¿cómo podía administrar holgadamente
los bienes de la niña, un hombre abrumado de quehaceres políticos, con
tantas cosas dentro de la cabeza? ¡Que la Hacienda, que el empréstito,
que las juntas, que el Estatuto, que los frailes\ldots! Imposible
atender a todo, Señor. De su peso se caía que debía entenderse con la
Zahón, y pedirle por favor que se encargarse de la tutela y gobierno de
bienes de Aurora Negretti, pues algo habría en el testamento que tal
abrogación consintiera. No se le apartaba del magín esta temeraria idea,
y si el horrible acceso reumático que en aquellos meses sufría no la
imposibilitara totalmente, ya se habría presentado a D. Juan de Dios, a
fin de proponerle lo que para él era un alivio y para ella una carga muy
de su gusto. Bien clara está la razón de que, suponiendo al Don Fernando
cordialmente ligado a Su Excelencia, le recibiera con finuras y
agasajos, y echara la casa por la ventana en aquel desusado convite.

En los postres sirvieron \emph{curaçao}, que era quizás la única pasión
o debilidad del viejo Maturana. Aquel dulce licor le hacía desmentir muy
de tarde en tarde sus hábitos de formalidad y grave continencia. Siempre
que allí comía o cenaba, Jacoba, por hacerle rabiar, aseguraba no tener
\emph{curaçao}; por fin, después de mucho trasteo, hacía traer la bebida
y le daba un poquito, cuatro lágrimas, y así se divertía con él,
vengándose de alguna trastadilla que en los negocios le había jugado.
Pero aquella noche, antes de que la señora empezase el sainete, le
convidó Aura, y sacando del aparador la botella, le sirvió cuanto él
quiso, y después a Fernando. Mientras D. Carlos paladeaba con embeleso
los primeros sorbitos y Jacoba le afeaba su vicio con afectado enojo,
Calpena charló brevemente con Aura, cuando esta a su asiento volvía.
Doña Jacoba no reparaba en ello, o se hacía la distraída, que también
pudo ser, y Maturana se halló bien pronto bajo la influencia
embelesadora del rico néctar.

«¿Y qué?, ¿canta usted o no?»

---No\ldots{} me temo que D. Carlos no se duerma si canto. Pero si usted
se empeña en ello\ldots{}

---Deseo que usted cante\ldots{} Si hablando es su voz tan divina, ¿qué
será\ldots?

---¿Cantando? Pues más divina todavía\ldots{} Bueno; pero conste que, si
usted me manda cantar, hace una gran tontería.

---¿Qué está usted diciendo?

---Que hay otra cosa mejor que el canto mío.

---¿Qué\ldots?, ¡por Dios!

---Hablar\ldots{} que hablemos.

---Chist\ldots{} silencio.

\hypertarget{xxi}{%
\chapter{XXI}\label{xxi}}

Entró en aquel punto Milagro, que venía sin más objeto que hacer
asientos de facturas atrasadas, y se asombró no poco de ver aquel
aparato de festín, y a Calpena en la mesa. Pero como en aquella casa
todo era raro, y pasaban las cosas en contra de lo usual y corriente, se
guardó su sorpresa y no dijo nada. Pareció que a Fernando contrariaba la
importuna visita de su compañero de oficina; pero Aura, más lista que la
pólvora, se apresuró a tranquilizarle, diciéndole: «Este infeliz es lo
mismo que nadie, y además, también se pirra por el \emph{curaçao}. Le
ofreceré una copita, ¿sí?»

En esto propuso la señora pasar a la sala, y allá se fueron todos con la
botella por delante. Poseídos Aura y Calpena de una audacia loca, cuyo
móvil psicológico no se explicaban ni había para qué, se arrimaron al
extremo de uno de los mostradores, en el sitio menos alumbrado por la
lámpara, y a la mayor distancia posible de los bebedores de
\emph{curaçao}. Doña Jacoba hizo plantar su sillón junto a estos, sin
perder de vista a la juventud, con quien desde su asiento a ratos
hablaba, y ordenó a Lopresti que pusiese luz en el gabinete próximo, y
velas en el piano, abriendo de par en par la comunicación de esta pieza,
la única bonita de la casa, con la sala o tienda. Milagro y Maturana
rompieron, con los primeros tragos, a hablar de política, metiendo en
ella su cucharada la Zahón, con ardientes alabanzas del primer Ministro,
salvador del desdichado Reino, remedio de todos nuestros males. Y
conforme aumentaban las ingestiones de bebida, la imaginación de
Maturana se lanzaba intrépida al simbolismo: «Reina Cristina es la
\emph{Peregrina} entre las perlas, y Méndez el \emph{Gran Mogol} entre
los diamantes. Carlos V es el diamante falso, el strass\ldots{} tras,
tras\ldots{} Jacoba el \emph{Ojo de Gato}, tallado en
\emph{cabujón}\ldots{} y tú, Milagro, eres la \emph{Montaña de
Luz}\ldots{} sólo que todavía no te han tallado, hijo\ldots{} estás en
bruto\ldots»

Con sólo probar el delicioso licor, se le quitaban al buen Milagro diez
años de vida; y a medida que iba apurando el vasito, presentaba síntomas
diversos de exaltación cerebral. Al tercer trago le atacaba
infaliblemente una sensibilidad lacrimosa, con recuerdos tiernísimos de
su familia e invocaciones a la santa pobreza, a la caridad sublime, a
los más altos y puros ideales. Hacia el cuarto o quinto sorbo se le
iniciaba la tendencia a expresarse en forma poética, reverdeciendo las
aficiones de su edad juvenil, en la cual más le gustaba hacer versos que
comer, y era un adepto fidelísimo de la retórica que entonces se
gastaba. «¡Ah!---decía con trémula voz, mirando al vaso:---¡la
Reina\ldots{} angélica Cristina, pía matrona!\ldots{} Desde que vino de
Parténope, vimos abierto el Empíreo los buenos españoles\ldots{} Cuando
contemplo este doméstico regocijo\ldots{} ¡ah!, viene a mi mente la
imagen de mis pobres niños, de mi dulce esposa, alma virtud\ldots{} ¿Qué
será de vosotros, \emph{oh dulces exuviæ}, el día en que fiera Parca me
corte el hilo?\ldots{} Mendizábal tonante, aplaca el furor de
Mavorte\ldots{} La oliva sucede al laurel\ldots{} somos felices\ldots{}
Vuelve el reino de Ceres prolífica\ldots{} Comeréis, hijos míos, blancos
panes y bizcochos duros\ldots»

Doña Jacoba, sin catarlo, era atacada de somnolencia, que procuraba
vencer. En tanto, recogía cuidadosa la caja de las perlas, acomodando en
ella los paquetitos que contenían las divisiones hechas por Maturana.
Esto no le estorbaba para dirigir a la gallarda pareja estas
insinuaciones: «Sr.~Calpena, cuéntenos usted algo de política\ldots{}
Aura, ¿por qué no cantas?»

Aprovechaban ellos las distracciones y cabezadas de la señora para
entregarse con efusión al ardiente coloquio que enlazaba sus almas, en
cláusulas cortas, balbucientes: «¿Me había usted visto alguna vez?»

---No, no\ldots{} La impresión de usted en mi espíritu es antigua, eso
sí\ldots{} Cuando la vi entrar por esa puerta, creí recobrar algo que se
me había perdido\ldots{}

---¡Qué cosa más rara!\ldots{} Esta noche, cuando subía yo la escalera,
sentí miedo, alegría y qué sé yo qué\ldots{} No podía respirar\ldots{}
por poco me caigo.

---¿Y por qué pegaba usted a Lopresti?

---Es juego. Suelo darle así, con la sombrilla. A él le gusta, y conozco
yo que está de mal humor cuando no le pego. Es un perro fiel, y me
quiere con delirio. Esta tarde, al entrar, me dijo: «La está esperando a
usted un caballero muy guapo, de parte de su tío el Sr.~Mendizábal.» Ya
ve usted cuánto desatino. Me eché a reír\ldots{} y le casqué más fuerte
que otros días. ¿Oye usted? Jacoba me dice que cante\ldots{} ¿Qué debo
hacer?

---Obedecerla, creo yo.

---Lo que agrade a usted haré, y nada más. ¡Qué extraño es lo que me
pasa! Hasta esta noche me ha costado siempre mucho trabajo someterme a
la voluntad de los demás. He sido voluntariosa, díscola, rebelde\ldots{}
Pues ahora creo que si alguien me pegase, me alegraría, y mi mayor gusto
sería obedecer, ser mandada.

---¿Y si yo me tomase la libertad de decirle: «Aura, haga usted esto;
Aura, sería yo muy feliz si usted\ldots»

---¿Si yo qué\ldots? Había de mandarme cosas buenas, las que ahora me
parecen buenas\ldots{} Y también, también yo mandaría un poquito, que es
muy grato para una mujer verse obedecida. Obediencia y mandato, pienso
yo que deben ir juntos.

---Servidumbre y tiranía en una sola persona, en dos quiero
decir---indicó Calpena enteramente trastornado.---El amor nos hace
dueños y esclavos de la persona amada\ldots{} Aura, esta noche, después
que yo me retire\ldots{} y mañana, mañana, ¿se acordará usted de mí?

---Se lo diré cuando vuelva.

---Según eso, ¿he de volver?\ldots{}

Al llegar aquí sintió Calpena que se ponía tonto. A su primera audacia
sucedió una timidez aplanante, y no encontraba fórmula adecuada para la
expresión de sus afectos. Pero de súbito, en la tremenda revolución de
su alma, vino el golpe de osadía, y poco faltó para que diese un grito,
dejando salir, sin ningún recato ni miramiento, las llamaradas que le
abrasaban. Con su mirar frío le contuvo la Zahón\ldots{} Poco después le
hizo Aura una pregunta insignificante: «¿Cómo es su segundo apellido?» Y
él replicó: «Igual que el primero\ldots{} Aura, nos conviene que usted
cante un poquito, y es de todo punto indispensable que, cuando usted
pase al gabinete ese del piano, pase yo también y estos se queden aquí.»

Pronto lo arregló Aura dirigiéndose a la próxima estancia y ordenando a
Fernando, desde la puerta, que tuviese la bondad de \emph{volverle la
hoja}, pues no daba pie con bola sin mirar al papel\ldots{} Y ya están
allá; ya desliza Aura sus lindísimos dedos sobre las teclas; él a su
lado, sin entender la escritura musical, hace como que atiende al papel,
mira embelesado a la divina cantora, y más embelesado aún, o
transportado al séptimo cielo, la oye. Canta ella el aria de
\emph{Semíramis}, \emph{Bel raggio lusinghier}, y después una canzoneta
napolitana.

Duda Calpena si vive o muere, si duerme o vela. La voz de Aura le
penetra en el sentido como un himno de deidades lejanas, desconocidas,
apenas visibles en su envoltura de blancos cendales. A ratos siente como
un súbito rayo que le hiere, que le destroza, que le arrojaría exánime
al suelo, si un poderoso estímulo de su voluntad no le contuviera. Desea
que calle Aura; desea cogerla y llevársela consigo en aquel mismo
instante, como el hecho más natural del mundo. A su timidez sucede una
arrogancia que nada respeta, una prepotencia que todo lo allana. Se
siente capaz de saltar por encima de los obstáculos más imponentes, y de
atravesar con su hermosa conquista por entre las multitudes, que a sus
ojos se empequeñecen ya, y sólo se compone de figurillas despreciables,
microscópicas\ldots{} Aura sola es toda la vida, Aura toda la ley, Aura
el Universo físico y moral, Aura cuanto existe de Dios abajo.

En uno de los que podríamos llamar entreactos, el ardoroso galán,
revolviendo papeles de música, como para escoger, le dijo: «Aura, cuando
entraste esta noche y nos vimos, ¿no comprendiste que te adoraba?»
Acalorada por la turbación que al rostro en centellas le subía, Aura se
abanicó con una pieza de música. No se hizo cargo el joven de que la
había tuteado, y ella, sin parar mientes en la forma familiar usada por
primera vez, pasó maquinalmente sus dedos por las teclas. «El piano me
responde por ti, Aura---prosiguió D. Fernando;---el piano me dice que tú
también me quieres, que no me dejarás morir de desesperación\ldots{} Un
instante ha bastado para hacerme pasar de una vida a otra vida, de la
vida muerta a la vida viva\ldots{} Si es verdad esto que pienso, no
necesitas decírmelo. Me lo confirmarás callando\ldots»

---Si callo, y tú lo dices todo\ldots{} verá Jacoba que\ldots{} que tú
me quieres, que me estás enamorando; y si hemos de hacerle creer que yo
no te quiero, porque así nos convenga\ldots{} mejor será, tontín, que
hable, y que me ría ¿sí?\ldots{} como hacen las muchachas que
coquetean\ldots{}

---Conviene que cantes otro poquito\ldots{} Dos palabras antes del
canto: Hagamos de nuestros corazones un mundo aparte, sólo para
nosotros\ldots{}

---Mundo aparte\ldots---murmuró Aura con firme acento, arrojando sobre
los ojos de su amante toda la luz y el fuego de los suyos.---En un
momento hago yo toditos los mundos que quiera.

---Aura, no hables más o me muero\ldots---dijo Calpena casi delirante,
violentándose para no gritar,---y si no me muero, te arrebato ahora
mismo de esta casa y te llevo a la mía\ldots{} Canta por Dios, canta un
poquito.

---Y tú te callas\ldots{} Después hablaremos.

---Un momento\ldots{} ¿Dónde, cómo?

---Luego te lo diré\ldots{} Silencio ahora.

Mientras cantaba con sublime expresión un trozo de la \emph{Medea} de
Cherubini, Jacoba y sus dos amigos, en la otra estancia, hablaban con
elogio del joven Calpena. Propiamente, la Zahón lo decía todo, y ellos,
bajo la influencia del dulce elixir que alegraba sus gastados cerebros,
apoyaban con fáciles exclamaciones y con expansivos movimientos de
cabeza las palabras de la diamantista. Maturana se había encerrado en
los monosílabos; Milagro, por el contrario, se lanzaba a la verbosidad
más desenvuelta; Doña Jacoba tuvo que cogerle por un brazo, obligándole
a recobrar su asiento a contestar formalmente a lo que tres o cuatro
veces le había preguntado sin obtener respuesta. «No vuelvo a admitirle
a usted en mi casa---le dijo,---si no me contesta con claridad. A ver:
si usted lo sabe, me lo tiene que decir\ldots{} No valen misterios
conmigo.»

---Señora mía---respondió D. José plantándose la mano abierta sobre el
pecho.---Por el nombre que llevo, nombre ilustre si los hay; por la
salud de mis hijos, por el amor purísimo de mi esposa, digo y juro que
este mozo gallardo es hijo del mismísimo D. Juan Álvarez Mendizábal, mi
augusto jefe.

---Me lo figuraba---dijo Doña Jacoba con mirada resplandeciente.---Pero
me falta saber otra cosa\ldots{} ¿Y la madre?\ldots{} ¿quién es la
madre?

---¡La madre!\ldots{} ¡la madre!\ldots---murmuró Milagro como en grande
confusión, pasándose la mano por el cráneo.

---Sí, hombre\ldots{} ¿quién es la madre?

---¡La mamá!\ldots{} ¡Ah!, ya recuerdo\ldots{} Con el maldito néctar se
le va a uno la memoria\ldots{} Pues la madre\ldots{} silencio, que no
nos oiga nadie\ldots{} es\ldots{} ¡una reina!

---¡Una reina!---exclamó D. Carlos con espantados ojos.

---Chitón\ldots{} Es un secreto\ldots{} Y créanme a mí\ldots{} peligran
las cabezas de los insensatos que lo divulguen\ldots---dijo Milagro
puesto en pie, aplicando su dedo índice a los morros alargados.---¡Una
reina!\ldots{} Chist\ldots{} Aunque me amenacen de muerte, no saldrá de
mi humilde labio el nombre del Reino en que reside la señora reina
que\ldots{}

\hypertarget{xxii}{%
\chapter{XXII}\label{xxii}}

Todos los biógrafos del insigne Milagro están acordes en afirmar que al
salir este de casa de la Zahón para dirigirse con inseguro paso a la
suya, quitose el sombrero y con él se abanicó, ávido de frescura y de
bañar en aire limpio sus sienes abrasadas, su cráneo sudoroso. Y añaden
que con el aire y el ejercicio se le aclararon de tal modo las
entendederas, que al atravesar la plazuela de Provincia, camino de la
Concepción Jerónima, donde vivía, empezó a sentir en su conciencia la
garrafal tontería que a propósito del señorito Calpena se había dejado
decir, bajo la acción tóxica del nunca bastante maldecido
\emph{curaçao}\ldots{} «¿Pero he dicho yo esa barbaridad,
Señor?---pensaba, parándose y mirando al cielo.---¿Lo habré
soñado?\ldots{} No, no; lo he dicho\ldots{} aún me parece que estoy
oyendo cuando solté el trueno gordo, cuando afirmé que
Mendizábal\ldots{} ¡Jesús!\ldots{} y nada menos que una reina\ldots{}
Vamos, que me daría una tremenda bofetada en castigo de tanta necedad,
de tanta estupidez\ldots{} ¡Una reina\ldots{} Mendizábal!\ldots{}
¡Válgame Jesús bendito! ¡Que un hombre formal como tú, oh Milagro, haya
repetido, dándolo por cosa verídica, esos ridículos dicharachos con que
se mata el tiempo en las oficinas!\ldots{} Pues digo, si el señor
Ministro se entera de que yo\ldots{} ¡Válgame mi santo Patriarca\ldots!»
Al pensar esto, se le erizaron sobre el cráneo los escasos cabellos que
poseía\ldots{} Consternado, intentó volver a la calle de Milaneses para
desdecirse de todos aquellos embustes que no eran más que cháchara
insubstancial de gente ociosa y frívola; pero no se determinó a desandar
el camino, juzgando muy oportunamente que \emph{peor era meneallo}.
Siguió, pues, hacia su vivienda, haciendo propósito de rectificar
serenamente, en noches sucesivas, los groseros dislates de aquella
noche, y se recogió taciturno, caviloso. Su mujer le sintió desvelado,
dando suspiros y pronunciando monosílabos con que a sí propio se ponía
de oro y azul. ¡Infeliz Milagro!

Embebidos en su amorosa charla, los amantes no repararon en la salida de
D. José, que les dijo «¡adiós!» desde la puerta del gabinete; ni se
cuidaban de ser vistos u oídos por Doña Jacoba, que hablando permanecía
con el diamantista, entre cabezadas. Habían alzado, sin darse de ello
cuenta, una valla anchísima entre su pasión y el mundo, y nada temían;
la pasión crecía por momentos, como una enfermedad fulminante, y a las
pocas horas de iniciada, ya no cabía dentro de la reducida esfera del
secreto: se salía, se ensanchaba, quería ser patente a los ojos
extraños, o por lo menos no temía ser lo bastante poderosa en sí para
afrontar la opinión y cuantos obstáculos esta le ofreciera. Mejor que el
narrador lo expresaban ellos mismos: «Antes de verte, antes de esta
noche bonita---decía Aura,---yo, sin saber por qué, tenía la seguridad
de que no estaba sola en el mundo. Cuando te vi, se me quitó de encima
del alma el peso terrible de mi soledad.» Y él: «¡De ayer a hoy, qué
abismo! Ayer iba tras de tu sombra; hoy te poseo\ldots{} Había de
llegar, puesto que hay Dios, este divino abrazo de nuestras almas.» Y
por aquí seguían, en un vértigo de fogoso idealismo, locos, ávidos de
amplificar cada concepto con otro más apasionado y sutil.

Viendo que Maturana se ponía en pie, Calpena hizo lo mismo, y dijo a su
amante, consternado: «Horror de los horrores. D. Carlos se despide.
También yo tendré que retirarme\ldots»

---Mañana volveremos a vemos\ldots{} lo más temprano posible.

---¡Mañana!, es muy lejano eso\ldots{}

La mujer, en lances de pasión, posee más iniciativa y más arbitrios que
el hombre. En voz muy baja propuso Aura algo que Calpena oyó con
alegría. Cuchichearon\ldots{} Despidiéronse luego en alta voz. Al poco
rato, Doña Jacoba le daba al Sr.~D. Fernando la venia para retirarse, y
con afectuosos apretones de manos le ofrecía su casa, y le rogaba que
viniese a honrarla con toda la frecuencia que le permitieran sus
obligaciones al lado del señor Ministro. Juntos salieron el joven y
Maturana; separáronse en la esquina de la calle de Santiago; vivía el
diamantista en una de las casitas del Patrimonio, plaza de la Armería,
junto a la casa de Pajes.

Consta en las monografías del buen Maturana que en el trayecto hasta su
domicilio se agarró más de una vez a las paredes para no medir el suelo;
y algún biógrafo añade que hubo de subir a gatas la corta escalera de su
casa, y que se acostó al instante, muy arrepentido de sus recientes
abusivas relaciones con el \emph{curaçao}. «No está bien, no está
bien---decía, desnudándose al revés, quitándose las botas antes que el
sombrero, y las medias antes que la corbata.---Un artífice, un tasador
no debe\ldots{} no, señor\ldots{} Es muy expuesto\ldots» Felizmente, era
en él añeja costumbre no aceptar invitación o cena o merienda cuando
llevaba en su cartera piedras de valor. Aquella noche no llevaba nada.
Tardó en dormirse, y daba vueltas en su abrasado cerebro a las ideas
sugeridas por Milagro: «¡Vaya con D. Juan Álvarez!\ldots{} No hay grande
hombre que no tenga sus enredos\ldots{} Ya, ya se ve claro por qué
arrambla todos los bienes del clero, que no es flojo botín.
Naturalmente, ese dineral lo quiere para sí. Parece tonto, y pide para
las ánimas\ldots{} ¡Tremendas hormigas nos trae Dios acá! Bueno, hombre,
bueno: cójase usted media España, y constituya un reino para el niño,
para ese hijo de reina\ldots{} Y ya veo a dónde va a parar con eso de
coger todas las campanas de las iglesias y monasterios. Hará un palacio
de bronce, todo de bronce, en el que las pisadas de los que entran y
salen suenen como campanadas\ldots{} ¡Ji, ji!\ldots{} ¡Qué
extraño!\ldots{} el palacio del sonido\ldots{} tin, tan\ldots{} Otra: lo
mejor sería que afanase las innumerables alhajas de las Santísimas
Vírgenes y toda la plata y oro de las reverendas catedrales, echándolo
al mercado\ldots{} ¡Por Belcebú, qué negocio, qué pujas!\ldots{} No
quiero pensarlo. De Londres, de Amsterdam y de Francfort vendrá la nube
de marchantes\ldots{} Mucho ojo, Maturana\ldots{} ¡Por San Carojulián
bendito, no te descuides!\ldots{} Y tiene que venir, tiene que sacarse a
subasta. Porque todo, digo yo, no ha de ser para el niño\ldots»

El niño, el hijo de la reina, se paseaba en la inmediata calle de
Santiago. Aura le había dicho: «Mi habitación corresponde al último de
los tres balcones por la otra calle. Cuando Jacoba duerma, me asomaré.»
El hombre hacía su centinela entre las esquinas del Bonetillo y de Mesón
de Paños, temeroso de perder, si se alejaba, el sublime momento en que
su amada en el balcón apareciese. La noche era obscura; dieron las doce
en el reloj de Palacio; no se veía por allí más gente que las pocas
mujeres que entraban por el Bonetillo y se deslizaban calle abajo, y
algún hombre que en la misma dirección iba, o hacia las tabernas de la
plaza de Herradores. El sereno se hacía presente por la luz de su
farolillo, allá junto a los altos muros de San Felipe Neri.

Media hora pasó Calpena en gran ansiedad, recelando que Doña Jacoba,
enterada del propósito de los amantes, lo estorbase encerrando a la dama
o conminándola con algún castigo. Paseo arriba, paseo abajo, sin quitar
ojo del balcón, pensaba en aquella su mudanza súbita, tan semejante a la
explosión de un volcán. Toda su vida era nueva; todas sus ideas habían
cambiado, dispersándose las de ayer y entrando con empuje dominante las
de hoy. Ningún sentimiento de los de ayer, refiérase a la política, a
los amigos, a la sociedad, en él persistía. De aquel espacio luminoso,
donde flotaba la ideal imagen de Aura, venían nuevos conceptos de todas
las cosas. Impaciente por la tardanza de ella, ni por un momento pensó
que pudiera burlarle: tenía confianza absoluta en su firmeza y lealtad.
Tampoco le amargó la sospecha de que Aura hubiese conocido el amor antes
de conocerle a él. Era \emph{mujer nueva}, como la esposa de Adán. Dios
les había criado destinándoles el uno al otro, y no estaba en el orden
del universo que hubiesen precedido al feliz hallazgo otros encuentros,
ni aun siquiera fortuitos y sin importancia. Tal era su ardor ciego y
entusiasta, tal su fe en aquella felicísima obra de integración,
dispuesta por el destino de ambos.

Al fin\ldots{} oyó ruido en el balcón, y apareciose en él una forma
blanca. Era principal el cuarto, y la distancia entre el balcón y la
calle como de cuatro varas. Arrimose el galán a la pared, y Aura echaba
medio cuerpo fuera del antepecho, doblándose como un junco, para que el
espacio entre las enamoradas voces fuese lo más corto posible. Explicó
primero su tardanza, motivada por lo que Jacoba tardara en dormirse, a
causa de sus dolores, siendo preciso darle friegas y ponerle bayetas
calientes. Ya parecía dormida, y Lopresti, fiel esclavo, quedaba
encargado de la centinela, para avisar en caso de que la enferma
remusgara. Recayó luego la conversación en un punto interesantísimo:
«¿Tú quién eres? Conozco en ti al hombre que quiero, y me basta. Pero
deseo saber quién eres para los demás. Lo mismo me da que seas noble,
que seas plebeyo, que seas mucho, que no seas nada, pues siendo para mí
el único, me basta\ldots{} ¿Te enteras bien de lo que te pregunto?»

---Sí, vida y gloria mía\ldots{} Yo no soy nadie. Ignoro quiénes son mis
padres. Vivo de la protección misteriosa de una persona desconocida, por
quien estoy en Madrid, por quien disfruto ese destinillo, y no sé más.
¿Verdad que es raro?

Contó en seguida concisamente su vida toda: su crianza en Vera, lo del
padrino, la estancia en París, la traslación a Madrid y todo lo demás
que ya se sabe, poniendo en su relato tal sinceridad y sencillez, que
Aura se embelesaba oyéndole; y si no estuviera enamorada hasta la
médula, es de creer que sólo con aquella historia tan poética y linda se
prendaría locamente del pobre desheredado. Refirió ella que no había
conocido a su padre ni a su madre: habíanla criado parientes egoístas
que jamás la demostraron vivo afecto. Creíase sola en el mundo, hasta
que Dios le deparó el compañero de su existencia, su salvador, su
\emph{única familia}. ¡Qué hermosura ser los dos solos en sí,
reconocerse en medio de los espacios de la vida, como pajarito y
pajarita que se encuentran en la espesura de la selva, y, saludándose
con sus piquitos, se unen para siempre! No faltaba sino que se
declararan libres, sin más obligaciones que las que cada uno para con el
otro había contraído, por vía de unión divina, como si Dios les echara
un lazo y les dijera lo que dicen los curas cuando casan. De pronto,
Aura tuvo una idea, y la expresó al instante con infantil candidez: «¿No
sabes?\ldots{} Como aún no hemos tenido tiempo de decirnos todas las
cosas, no te has enterado de que yo soy rica. Sí, hijo, sí. ¿Pensabas
que éramos nosotros unos pobrecitos, dejados de la mano de Dios? Mi
padre, Jenaro Negretti, dejó mucho dinero. Lo tiene guardado el Sr.~de
Mendizábal, que es quien le da a Jacoba para mis gastos\ldots{} Con que
ya ves. No hay que apurarse\ldots{} Estamos en grande, y seremos los
reyes del mundo.»

---Pues yo---dijo el amante con tristeza,---soy pobre: nada tengo; pero
no me faltan alientos, ni tampoco, creo yo, disposiciones para
trabajar\ldots{} También te digo una cosa, Aura: bien podría suceder que
de la noche a la mañana recibiera yo, como caída del cielo, una fortuna
grande\ldots{} Se han dado casos: yo he leído de algunos casos\ldots{}

---Pues si sale lo que esperas, ¡oh Dios mío, cuánta felicidad!\ldots{}
Eso sería lo más lindo del mundo. Resultaríamos en posesión de unos
dinerales que no nos harían maldita falta\ldots{} Si quieres que te diga
la verdad, a mí no me hace dichosa el dinero, ni creo que sirvan las
riquezas más que para disgustos. Con poseerte a ti me basta; y si mañana
viniera el señor Mendizábal y me dijera: «niña, no tienes ni un
maravedí,» yo me quedaría tan fresca. ¿Y tú?

---Pienso como tú piensas, y siento todo lo que tú sientes\ldots{} Quien
nos ha puesto hoy el uno junto al otro, se cuidaría de darnos lo
necesario, si por nuestra parte no lo tuviéramos. Es hermosísimo, sí,
lanzarse a la vida sin más alas que las inmensas del amor. Somos
jóvenes, nos adoramos\ldots{} Esto es la suma dicha. ¡Qué bueno es
Dios!, ¡y la Naturaleza qué hermosa!, ¡y nosotros, qué bien hicimos en
nacer!\ldots{} Si tú o yo nos hubiéramos quedado por allá, ¡qué insigne
tontería habríamos hecho!

---Es verdad; porque no naciendo, ¿cómo podría yo quererte con toda mi
alma?

---Oye otra cosa, vida mía\ldots{} Si te parece, nos casaremos pronto,
muy pronto.

---Sí, sí---dijo Aura con tan vivo movimiento de inclinación, que
pareció querer arrojarse a la calle.---¿Cuándo?

---Pronto. Mañana\ldots{}

---¿Mañana?\ldots{} ¿Y hoy por qué no?\ldots{} ¡Pero qué tonta soy! Eso
no puede determinarse así en días, en horas. Tengamos paciencia y
formalidad. Lo que acabo de decir es muy desvergonzado. ¿Me lo perdonas?

---Pues si el \emph{hoy} te parece demasiado presuroso, diré:
\emph{ahora mismo}.

---Quita allá, hombre\ldots{} ¿Acaso el casarse es cosa de un soplo? No,
niño mío, no seas tan arrebatado. Ten juicio. Pues apenas hay que
preparar cosas: ropa, papeles, y, ante todo, casa.

---¡Casa! Tenemos el mundo por nuestro\ldots{} Dime---añadió el galán,
casi loco ya, señalando hacia la bóveda celeste,---¿te gusta ese techo?

---Es precioso\ldots{} Pero ahora, desde que te quiero, todo me parece
cielo, y la obscuridad claridad, y la noche tan bonita como el día, casi
más, y Jacoba me parece amable, y todas las personas muy buenas\ldots{}
Pero tengamos calma, y esperemos.

---Sí, esperemos. ¿Qué nos importa retrasar la felicidad, si la tenemos
segura, si es nuestra ya?

Asaltado de una idea triste, cosa natural en aquella irradiación de
ventura, Calpena no vaciló en expresarla: «Dime, amor mío, si Jacoba,
que me parece persona egoísta\ldots{} no sé en qué me fundo; pero me lo
parece\ldots»

---Y lo es: tú tienes mucho talento y todo lo aciertas. Sigue.

---Pues si Jacoba, y lo mismo podría decir de otro cualquier pariente
tuyo, se opusiese, por móviles de interés, a que nosotros nos amáramos:
no, no, a eso no pueden oponerse\ldots{} quiero decir, que se opongan a
que nos casemos\ldots{}

---Eso no puede ser\ldots{} porque nosotros saltaríamos por encima de
todas sus artimañas, y pisoteándoles nos juntaríamos y nos casaríamos,
¿sí?

---Pero suponte tú que contra toda nuestra buena voluntad y contra las
energías de nuestra pasión, lograran separarnos, imposibilitarnos
materialmente de\ldots{}

---No, no puede ser, no será---dijo la enamorada con expresión de
voluntad tenacísima.---¡Pues si Jacoba fuera tan mala que\ldots! No, no
quiero pensarlo.

---¿Qué harías?

Aura se irguió, y apretando en su nervioso puño, con fuerza de mujer
furiosa, el hierro del balcón, dijo: «¡La mataría!»

---No, no tendrías que tomarte ese trabajo, mi bien, mi vida, mi
encanto, porque antes la habría matado yo.

---Y luego iríamos juntos al presidio, ¿sí?

---No pensemos en eso, que no ha de suceder. Yo digo: ¡qué más querrá
Jacoba\ldots!

---Claro: ¡qué más querrá ella! No te creas, Jacoba es buena, siempre
que no la arrastra a la maldad la infame codicia. Por un brillante de
buenas aguas, o por una docena de turquesas de \emph{roca vieja}, sería
capaz de sacrificar a su padre.

A todas estas se les iba pasando la noche. Las primeras claridades del
alba trajeron a la calle alguna gente de los mercados próximos, y el
sereno pasó varias veces, dirigiendo a Calpena miradas recelosas. Aquí y
allá sonaban porrazos; los gallos del comercio de aves en la calle de la
Caza cantaban anunciando el día. Sobre esto llamó Calpena la atención de
Aura, indicándole con pena que ya era hora de retirarse.

«¿Qué prisa todavía?\ldots{} Esos pobres gallos enjaulados están tan
aburridos por la falta de libertad, que anuncian la aurora antes de
tiempo.»

---Ya es de día\ldots{} ¿No lo ves?

---¿Y qué? Mejor. Así podremos vernos las caras.

De improviso se abrió una de las puertas del piso bajo de la casa, y
Calpena se vio sorprendido por un mozo, soñoliento, que salía con una
escoba. Luego se abrieron dos puertas más: una cacharrería y un despacho
de huevos. Imposible seguir más tiempo allí. Los hados fieros ordenaban
la suspensión del coloquio dulcísimo, y que los amantes guardasen la ley
del recato ante el público, pues cada cosa tiene su ocasión y lugar
propios. ¡Bonita idea tendría de la señorita de Negretti el vecindario
de Milaneses si la veía colgada al balcón, al amanecer de Dios,
picoteando con su novio! Antes que ella comprendió él la inconveniencia
de prolongar la alborada de amor, y así se lo dijo. Convenidos el cómo y
cuándo de verse en el curso del día, Calpena se arrancó con esfuerzo del
celestial muro. El día se recreaba iluminando con sus primeras
claridades la ideal belleza de Aura, quien no se apartó del balcón hasta
que hubo recibido el último saludo de D. Fernando. Se fue y volvió el
galán como unas tres o cuatro veces, jugando al escondite en la esquina
de la calle Mayor, hasta que al fin, siendo preciso poner término al
juego\ldots{} se arrancó de veras.

\hypertarget{xxiii}{%
\chapter{XXIII}\label{xxiii}}

Más que inquieto, lleno de zozobra por la desusada tardanza de
Fernandito, le esperó levantado su amigo D. Pedro, y al verle entrar,
conoció por su rostro encendido, por el febril centelleo de su mirada,
que algo muy grave le había ocurrido aquella noche. Interrogole
dulcemente, y no obtuvo respuesta categórica.

«Luego me lo contarás---dijo Hillo,---que ya es hora de que me vaya a
decir mi misa. Me has tenido toda la noche en vela. Como no es tu
costumbre trasnochar, me alarmé. ¿Has estado en alguna logia? ¿Se trata
de algún mal paso, de algún lance?\ldots{} Pero no quiero molestarte
ahora. No me cuentes nada, y descansa, pobrecito, que estarás muerto de
sueño. Yo me voy al Carmen\ldots{} Duerme todo el día si quieres, y a la
tardecita me contarás\ldots»

Se fue D. Pedro a celebrar, y al regreso de la iglesia, Calpena dormía.
Acercose a su lecho el presbítero, y le vio dormidito como un ángel, con
ese leve sonreír que indica un venturoso sueño. A la hora de comer quiso
Doña Cayetana despertarle; pero se opuso Hillo diciendo: «No, no, pobre
hijo; dejarle que duerma: sabe Dios lo molido y ajetreado que estará ese
bendito cuerpo. Guárdesele la comida.» Salió después a una diligencia
que le entretuvo dos horas, y al volver a casa díjole Delfinita que D.
Fernando había comido presuroso y sin enterarse de lo que metía por la
boca; que no respondía a lo que se le preguntaba, como si se hubiese
dejado en otra parte el pensamiento y la palabra. Y lo más singular fue
que, sin probar el postre, que era miel de la Alcarria y queso de
Villalón, había cogido el sombrero y echádose a la calle con tanta prisa
como si le llamaran a apagar un fuego. ¡Cosa más rara! Indudablemente
ocurrían sucesos inauditos. ¿Sería, por fin, la estupenda anagnórisis
que Hillo por momentos esperaba? Entregándose a sutiles cavilaciones y
al trabajo de adivinar, esperó el clérigo la vuelta de su amigo; pero
tuvo el acierto de esperarle sentado, porque Calpena no entró en casa
hasta la mañana del siguiente día.

Ya no pudo Hillo aguantar más los ardientes picores de la curiosidad, y
tomando una actitud serena, le dijo: «Hoy sí que no te me escapas sin
contármelo todo.» Calpena, confuso, no sabía por dónde empezar. Hillo
cortó la solemne pausa diciendo \emph{¡Habla!,} con el acento con que
esta palabra se pronuncia en las tragedias de secano.

«Pues\ldots{} nada.»

---¿Cómo nada? ¿Es acaso alguna intriga política?

---No señor.

---Pues yo sé que en el Ministerio no se vela\ldots{} Vamos, será
cuestión de amoríos\ldots{}

---Tampoco; porque los amoríos son cosa frívola y pasajera, y esto no.

---Amor entonces---dijo Hillo con benevolencia, y terminó la expresión
de su idea con una nota humorística:---¿Con que amor tenemos? Bueno: con
tal que sea clásico\ldots{}

---¿Y qué entiende usted por amor clásico?

---El que se contiene dentro de los límites de la conveniencia y de la
regularidad; el que no es motivo de escándalo, sino ejemplo de buenas
costumbres; el que no es furor insano, sino afecto plácido y limpio; el
que tiene por norte la familia y por cebo una relación casta, con el
consentimiento de los padres\ldots{}

---Yo no tengo padres.

---Di que no los conoces. Mientras te llega la anagnórisis, tu padre soy
yo: yo miro por ti, y te guío en el camino de la vida.

---Me temo, querido Hillo, que después del paso que he dado, tenga yo
que arreglármelas solo para seguir andando\ldots{} En fin, puesto que
usted habla de amor clásico, diré a usted que el mío, como águila a
quien quisieran encerrar dentro de un huevo de paloma, ha roto los
moldes, ha roto el viejo y podrido cascarón del clasicismo.

---No te conozco---dijo D. Pedro con sobresalto.---¿Eres tú el joven
Calpena?

---No señor\ldots{} El joven Calpena que usted conoció, se ha
transformado radicalmente en días, en horas. Cuando menos uno lo piensa,
sobreviene la crisis capital de la vida\ldots{}

---Hombre, eso es gravísimo. ¿Y quién es ella? ¿Acaso la niña que
llamamos marmórea?\ldots{} ¿Dices que no? ¿Pues de quién se trata? ¿No
puedo saberlo? Sea quien fuere podré darte una opinión franca, un buen
consejo.

---Me hallo en una situación tal, que toda opinión que no sea la mía me
hará el efecto de una enemistad irreconciliable; y en cuanto a los
consejos, debe usted esperar a que yo se los pida.

---Arrogantillo estás. Por lo que dices, voy entendiendo que tus amores
son de esos que llaman, que llaman\ldots{} no sé\ldots{} esta clase de
bregas son para mí desconocidas. Pero ello debe de ser cosa vergonzosa,
una pasión de estas que nos ha traído el romanticismo, y que suelen
acabar con \emph{descabello} de media humanidad.

Interrumpió el diálogo la llegada de una carta. Era de la \emph{mano
oculta}, que no había escrito en toda la semana. A Fernando le dio un
vuelco el corazón, y barruntando que el contenido de la epístola heriría
su vidriosa sensibilidad, rogó al clérigo que la leyese. Él oiría,
procurando enterarse, pues su espíritu, en aquellos días de ansias y
delirio, no acudía fácilmente al reclamo de la realidad próxima. Después
de suspirar fuerte, D. Pedro leyó:

«¿Con que tenemos al niño enamorado? Ya me esperaba yo ese sarampión,
que rara vez falla a los veintidós años. Paciencia, y pues no hay más
remedio que pasarlo, no lo combatamos, y pónganse los medios para que
brote bien\ldots{} Tontín, se te tolera esa pasioncilla juvenil, que es
el paso de la adolescencia a la madurez de la vida. Los hombres
conceptúan eso necesario, inevitable; tales turbonadas, dicen, son
necesarias, hasta convenientes. Sea: con pena lo admito, y te suplico
que acabes cuanto antes, no sea que la enfermedad se meta demasiado en
lo hondo. No tengo tranquilidad hasta que sepa el radical fin de esa
novelesca aventurilla, y no dudes que he de saberlo, como supe lo del
banquete que te dio la Zahón, como tengo noticias del desenfado con que
te pones a pelar la pava con la chiquilla de Negretti. También sé que es
muy linda. No te acusaré de mal gusto, no; y como te tengo por hombre
perspicaz y conocedor del género, presumo que en tus largos plantones al
pie del balcón habrás tenido tiempo de comprender que la niña es
diamante falso. ¡Ah, tontín!, la pedrería fina es muy escasa, y no se
encuentra en la primera cena a que nos convidan\ldots»

Al llegar a esto, Calpena no pudo contener el dolor, la ira que estas
apreciaciones le produjeron, y estalló diciendo: «Eso es sencillamente
infame\ldots{} Dígalo quien lo dijere, es inicuo, ultrajante. No debo
hacer caso de la opinión de persona anónima, que no puede sentir la
verdad, como la siento yo\ldots{} Y juro que no habrá voluntad que me
tuerza, ni razón humana que me persuada de que esto no es para mí el
supremo bien, el único bien posible.»

---Espérate un poquito y déjame acabar. Sigo: «Como para estas
aventurillas, que mejor será llamar calaveradas, se necesita dinero, te
mandaré mañana seis onzas. Más, mucho más recibirás; pero entiende que
este dinerito no debe servir para prolongar la enfermedad, sino para
ponerle término\ldots{} Y no te digo más por hoy.»

«¡No puedo, no puedo---exclamó Calpena dando vueltas por la habitación
como un loco,---sufrir por más tiempo esta tutela anónima!\ldots{} Y
estas burlas, este desconocimiento de la verdad, me lastiman, me hieren
más que si me asestaran cien puñaladas\ldots{} ¡Oh, cuánto diera yo por
conocer a la persona que me escribe, y poder decirle lo que
siento\ldots! No, no dudo que esa persona se interesa por mí, que me
ama. También la quiero yo sin conocerla. Pues bien: yo la
convencería\ldots{} ¿Cómo no había de convencerla, si yo lo estoy
firmemente, si llevo dentro de mi alma, no sólo todo el amor, sino toda
la lógica del mundo?\ldots»

---Hijo mío---le dijo Hillo con expresivo afecto,---lo que la señora
incógnita te escribe es el puro Evangelio. Considera tú ese amor como
una aventurilla pasajera\ldots{} cosas de muchachos, ejercicio
vital\ldots{} y\ldots{} dale ya puntillazo\ldots{}

Le miró Calpena, plantándose ante él desdeñoso, altanero, y con grave
entereza contestó:

«Soy un hombre; tengo un alma que es mía, una inteligencia que me
pertenece, y con ellas siento y juzgo lo que me incumbe. Ni de usted ni
de esa desconocida persona admito lecciones, ni soy un niño para
recibirlas en esa forma. Quien nunca ha tenido familia, bien puede
declararse independiente como lo hago yo ahora. La soledad en que he
vivido me ha enseñado a gobernarme por mí mismo. Soy libre, Sr.~D.
Pedro; a nadie me someto. Los que me protegen por motivos que aún están
rodeados de obscuridad, que den la cara, y entonces hablaremos. Si
conseguimos entendemos, bien, y si no, lo mismo. No altero mis
propósitos, no me someto, no me rindo.»

Sin dejar de admirar esta noble gallardía, trató Hillo de reducirle a la
obediencia ciega de la \emph{deidad velada}, pues así también solía
llamarla, no sabiendo qué nombre darle, y el primer argumento que empleó
fue que le convenía dicha sumisión para no comprometer su brillante
porvenir.

Echándose a reír, le contestó D. Fernando que él no contaba con más
porvenir que el que por sí mismo se labrase, pues todo lo demás era
fantasmagorías y sueños; y en último caso, que no sacrificaría a ninguna
consideración, ni a interés alguno por grande que fuese, la pasión que
colmaba todos los anhelos de su existencia. Y como Don Pedro insistiese
en que la aventura no merecía nombre de pasión seria, y que debía
ponerle punto final, replicole el joven con flema: «No puede ser, mi
querido Hillo. En esto he querido aplicarme fielmente el precepto
fundamental de su filosofía práctica\ldots{} Para que no diga usted que
fracaso como todos los españoles que emprenden algo, me propongo
\emph{rematar la suerte.»}

---¡Ah!, pillo\ldots{} ¿De modo que te casas\ldots?

---Tal creo\ldots{} Esto no es aventura\ldots{} para que vaya usted
enterándose.

---Estás perdido, perdido sin remedio\ldots{} Un joven llamado a\ldots{}
qué sé yo\ldots{} llamado a grandes destinos\ldots{} ¡Por Dios,
Fernandito de mi vida, mira bien lo que haces!\ldots{} Y a mí que me
parecían poco para ti todas las duquesas y princesas que andan por esas
cortes.

---Yo soy pueblo, pueblo nací y pueblo me encuentro ahora. ¡Ay!, amigo
Hillo, me acuerdo de mi cuna. Era de mimbres, y estaba rota y medio
deshecha. Yo ensanchaba los agujeros con mis manecitas, y me echaba
fuera para jugar con un perro y dos cabras que había en la pobrísima
estancia donde me criaron\ldots{} ¡Y ahora me habla usted de duquesas y
princesas! A usted le ciega, o más bien le enloquece su bondad\ldots{}
Yo no soy lo que era. He dado un gran vuelco: mis ideas son otras. No
tengo ya más que una ambición, y a satisfacerla se encaminan todas las
potencias de mi alma. Me crió aquel bendito en la templanza, en la
regularidad, en el justo medio de todas las cosas. Pues ya no quiero
justo medio; ya me solicitan las situaciones extremadas\ldots{} Quiero
exceso de vida, energías poderosas, mucho gozar o mucho sufrir, luchar,
hacer cara a los grandes desastres si vienen, hartarme de felicidad si
Dios me la depara. No quiero andar por caminos trazados, ni que me
cuenten los pasos que doy, ni que me lleven con andadores, ni que me
muevan con hilitos, como si fuera yo figura de titiritero. No, no: de un
salto me he echado fuera del retablo, y entro en el mundo yo solo. El
mundo es grande. Un sentimiento, grande también, llevo yo conmigo. ¿Hay
espacio? Sí. ¿Tengo yo alas? Sí. Pues a volar.

Y cogiendo el sombrero, se fue a la calle, sin añadir una palabra,
dejando a su excelente amigo todo confuso y turulato, con las manos en
la cabeza, desahogando con patéticas exclamaciones la turbación de su
espíritu: «¡Señor, devuelve el seso a este noble chico, digno de mejor
suerte\ldots{} le he tomado tanto cariño, que sus asuntos me interesan
más que los propios!\ldots{} ¡Señor, descúbreme el misterio de Calpena;
dame a conocer la \emph{mascarita} esa que le protege y le dirige! Que
yo la descubra, para llegarme a esa divina tutora y decirle que se
declare, que se quite la careta, único medio de que nuestro Fernandito
entre en razón. \emph{Tutora} he dicho, pero mejor será decir
madre\ldots{} En su estilo se ve la delicadeza, la gracia, y un cariño
intensísimo. Es madre, y además dama ilustre. Su estilo lo revela, esa
discreción de alto tono, esa exquisita habilidad para ocultarse\ldots{}
¡Dios mío, santo Apóstol bendito mi patrono, santa Virgen, y vosotros,
santos, santos todos de la Corte Celestial, despejadme esa incógnita,
pues creo que entre ella y yo, puestos al habla, salvaríamos a este
alucinado chico de la perdición, de la ignominia, de la muerte!»

Su generoso anhelo sugirió al buen presbítero una idea, un plan, y
propósito firmísimo de empezar a realizarlo aquella misma tarde. «Voy a
minar la tierra para \emph{desvelar} a esa \emph{velada}. Dios me abrirá
camino; Dios iluminará las obscuridades que encontraré en los comienzos
de mi trabajo. A esta investigación consagraré mi tiempo, pues ya no me
importa que me den ni que me quiten la cátedra que me
corresponde\ldots{} Y ahora digo yo: ¿por dónde empiezo?\ldots{} A ver,
Pedro, discurre un poco, \emph{afina la suerte}\ldots{} Por de pronto,
si a ese loquinario le da la ventolera de desdeñar las cartas de su
protectora, yo las recogeré cuando vengan, las leeré y las tendré bien
guardaditas hasta que a él se le caiga de los ojos la venda. Y si envía
dinero, como anuncia, yo lo guardaré también para írselo dando conforme
a sus necesidades, que ahora presumo han de ser muchas\ldots{} Esto lo
primero; después\ldots»

Dándose un golpe en la frente, lanzó una exclamación de alegría:
«\emph{Eureka}, ya sé cuál es el primer paso que tengo que dar: ir a la
casa de esa mozuela de quien se ha enamorado, y verla y hablar con su
familia, para lo cual me valdré o del compañero de oficina de Calpena,
Sr.~Milagro, o del Sr.~Maturana, el diamantista que vino a buscarle y se
le llevó, con la cajita de Olorón bajo el brazo, en aquel aciago
día\ldots{} Perfectamente: ya tengo mi base de operaciones\ldots{} Luego
trataré de averiguar por qué medios, por qué espionaje pasan a
conocimiento de la \emph{velada} todos los actos de Fernandito, cuantos
pasos da en este Madrid tan grande. Pondreme, pues, en relación con los
acechadores o centinelas que tiene esa señora. Sepa ella que yo quiero
ser también su misterioso vigía, y que ninguno habrá más diligente ni
más desinteresado que yo\ldots{} Procuraré además el trato y
conocimiento de todos los amigos de Calpena: ese empleado tísico, ese
Larra, ese Ros de Olano, ese Pezuela, ese Veguita\ldots{} Ellos quizás
me den alguna luz\ldots{} Y si pudiera colarme en los dorados palacios
donde el señorito fue introducido no hace mucho, también me
colaría\ldots{} sí señor\ldots{} dispuesto estoy a todo, hasta a
disfrazarme\ldots{} Sí, sí, Sr.~D. Fernando Calpena: usted no se ríe de
mí; usted no se emancipa, no, mientras esté aquí su viejo amigo, este
pobre clérigo, que beberá los vientos por evitar que un mozo de tales
prendas, que evidentemente lleva sangre de reyes\ldots{} ¡lo dicho,
dicho!\ldots{} sangre de reyes, caiga en los abismos del amor enfermizo
y de la calentura romántica.»

\hypertarget{xxiv}{%
\chapter{XXIV}\label{xxiv}}

No constan los días que empleó el buen Hillo en su investigación
preliminar; sólo se sabe que no fueron pocos, y que al cabo de una
semana conocía algo y aun algos de la familia Zahón, y había hablado
largamente con Milagro y Maturana, los cuales, lejos de aclarar el
enigma principal, lo que hicieron fue añadirle nuevas
obscuridades\ldots{} Sin desmayar ni un punto en sus tareas policiacas,
trató de hacer cantar a Méndez; mas toda tentativa cerca del estirado
patrón resultó inútil, bien porque nada de lo substancial sabía, bien
porque quisiera echárselas de discreto, contraviniendo el tradicional
tipo de los pupileros y fondistas. Cuando se veía el hombre muy
estrechado por la apremiante argumentación de D. Pedro, no se le ocurría
más que remitirle a \emph{Edipo} y al Sr.~de Azara. Salía D. Pedro al
ojeo del polizonte, conseguía echarle la zarpa, le interrogaba, y el feo
\emph{Edipo} le decía: «Sr.~de Hillo, estoy muy a gusto en mi
\emph{colocación} y no quiero perderla. Tengo seis criaturas, que son,
vamos al decir, seis candados que cierran mi boca. Si por contestar a
sus preguntas me dejan cesante, no será usted quien me coloque. Con que
déjeme en paz y llame a otra puerta.» Y D. Manuel de Azara, hombre más
avinagrado y de mejores despachaderas que Dios ha echado al mundo, le
recibía, después de plantones de tres horas, para decirle que se metiera
en sus asuntos y dejara los ajenos. Ni un indicio, ni una ráfaga de luz,
ni un vocablo indiscreto.

Acudió después mi hombre al tísico Serrano, que llenándole la cabeza de
mentiras y encaminándole por una pista falsa, le hizo perder el tiempo y
la paciencia; y tantea aquí, tantea allá, se refugió en la amistad y en
los grandes conocimientos sociales de su compañero de casa, Nicomedes
Iglesias. Si al principio pareció que el politicastro tomaba el asunto
con interés, pronto dejó de hacerlo; tan sorbido le tenían el seso los
negocios políticos, el interés de las sesiones y el periodiquillo que
había fundado en unión de su amigote reciente, Luis González o Luis
Brabo, que de ambos modos respondía, en el cual papelejo apoyaban al
grupito de oposición parlamentaria que formaron en \emph{Procuradores}
Caballero, López y el Conde de las Navas. Si el hombre no estaba
demente, le faltaba poco; su cortante lengua no desmayaba un instante
durante el día, ni su enconada pluma por la noche. Competía con él en
acrimonia y acometividad el tal Brabo, andaluz, delgadito, aguileño, más
vivo que la pólvora, cortado para la política del ruido y para
soliviantar con gracia a las multitudes. Meses después, Brabo escribía
en papeles moderados; Iglesias extremaba sus ideas revolucionarias en
los del bando liberal; su consecuencia, que era una forma de su orgullo,
le valía persecuciones y desdenes. Pero en Diciembre del 35 todavía se
le contaba entre los hombres de porvenir, aunque su irritación por no
haber entrado en el Estamento le creaba enemigos, alejándole de la meta
de su ambición.

Mientras Hillo con tan poca fortuna emprendía la reconquista de Calpena,
este se transformaba, haciéndose huraño, apartándose de sus primeras
amistades para contraer otras nuevas con personas bien distintas de los
literatos del Parnasillo y de los concurrentes a tertulias de tono.
Abandonó en absoluto la sociedad elegante, y no volvió a parecer por la
casa aristocrática, donde se entristecían por su ausencia las bellezas
más o menos marmóreas. Cultivaba la amistad de los oficiales de la
Guardia y de Infantería, yernos de Maturana, y conoció a los de
Fonsagrada, la familia que más trato tenía con la Zahón. Algunas tardes
paseaba con el soldadito chiclanero y poeta, amigo de Milagro, Antonio
García, autor imberbe de un drama caballeresco que tenían en su poder
los cómicos del Príncipe.

Contra lo que Fernando temía, Doña Jacoba no se opuso a sus amores con
Aura; casi los alentaba y protegía, pero encerrándolos dentro de la
esfera de castas relaciones con buen fin, y sometiendo la fogosa pasión
de ambos amantes a las reglas caseras que para tales casos se usan, y
que en aquel tiempo eran de una simplicidad enfadosa. Hacía esto la
Zahón, más que por sentimiento, por cálculo, mirando a su propio interés
antes que al de la joven puesta a su custodia. Era ante todo traficante,
se había criado en el compra y vende; todas sus canas, que eran muchas,
y las jorobas que en su esqueleto se formaban, le habían salido en el
continuo y anheloso estudio de la ganancia fácil. Por lo demás, su moral
era tan ancha como las mangas del vestido que el reuma le obligaba a
usar, y sus creencias religiosas, tibias como las aguas con que se
lavaba. La moral de los contratos de cosas, interpretada a su manera,
érale muy conocida y familiar; la otra, la tocante al honor y al recato,
sólo existía en su conciencia con formas desleídas.

Sujetó, pues, a los amantes a un régimen de apariencias estrictamente
morales, prohibiendo en absoluto las entrevistas de calle y balcón, y
permitiéndoles hablarse a horas fijas en su casa y en su presencia. Con
esto cumplía, y sentaba sobre bases decorosas su bien planeado negocio.
Muy mal sabían a Fernando y a su dama esta reglamentación de colegio y
este régimen de insulso noviazgo, aplicado a una pasión tan flamígera;
pero lo soportaban en espera de los arranques de su albedrío, planeando
también algo, que muy calladito tenían, y desquitándose por el pronto
con el carteo constante y clandestino de que era mediador el cuitado
Lopresti. Con los Fonsagradas se les permitía salir alguna vez de paseo,
bien vigiladitos, no pudiendo campar libremente ni a la ida ni a la
vuelta, ni extraviarse en las arboledas de la Florida, ni jugar a la
gallina ciega. Estaba, pues, Calpena hecho un novio \emph{clásico},
contra lo que su temperamento y sus altas ideas le dictaban; pero se
sometía o afectaba someterse, con la esperanza de que no había de durar
mucho la insípida comedia. Por aquellos días iba al Ministerio nada más
que el tiempo preciso para no caer en falta, y a veces dejaba de asistir
pretextando enfermedades. Rara vez le llamaba ya el Ministro a su
despacho para encargarle contestaciones de cartas. Hacíalo siempre dando
las instrucciones a Milagro, el cual repartía la tarea y vigilaba la de
su compañero, llevándolo todo a la firma.

Hacia el 20 de Diciembre, poco antes de la célebre discusión del
\emph{voto de confianza}, en días en que Mendizábal estaba gozoso, como
hombre que vislumbra el éxito y ve próxima la realización de sus ideas,
llamó a Milagro y le hizo sentar frente a sí en la mesa de su despacho.
Habíale tomado afición por la donosa vaguedad que sabía emplear en la
redacción de cartas de pura fórmula, en que no se dice nada, y por el
estilo cortesano y elegante en que envolvía el \emph{perdone usted por
Dios}, receta contra los pedigüeños de gollerías.

«Ante todo---dijo Mendizábal con aquella presteza nerviosa que ponía en
su trabajo,---póngame usted ahora mismo, pero ahora mismo, una carta a
D. Martín, diciéndole que detenga el nombramiento de Catedrático de
Retórica de un clérigo que se D. Pedro Hillo, en favor del cual le
escribimos no sé cuándo\ldots»

---Anteayer.

---Me había recomendado a este sujeto Musso y Valiente, si no recuerdo
mal.

---Sí, señor; y antes D. Manuel José Quintana\ldots{}

---Y creo que también Juan Nicasio Gallego\ldots{} en fin, medio mundo.
Tanto me han mareado, que me decidí a recomendarle a Heros. Pero después
he sabido algo que me pone en guardia\ldots{} Francamente, yo hago todo
el bien que puedo; pero en este puesto, y rodeado de dificultades, no
creo estar en el caso de favorecer a mis enemigos. Dígame, ¿conoce usted
a ese Hillo?

---Sí, señor: vive con mi compañero de oficina, Calpena, y hemos ido
juntos al café y a los Toros. Es muy entendido en tauromaquia.

---¡Qué atrocidad!\ldots{} cura, torero y retórico\ldots{} No he visto
jamás una ensalada semejante\ldots{} Ello es que ese sujeto ha dado en
perseguirme\ldots{} Aquí viene todos los días a pedirme audiencia. Como
ahora no estoy para perder el tiempo, no se la he concedido. Pero el
hombre ha dado en acecharme cuando entro en mi casa y cuando salgo.
Todas las mañanas tira de la campanilla tres o cuatro veces. En la
escalera, hoy, bajando yo con Cano Manuel y con Olózaga, me le
encontré\ldots{} Demudado, la voz temblona, me habló\ldots{} La verdad,
no me enteré bien de lo que dijo\ldots{} Que no quería hablarme de la
cátedra\ldots{} que se había hecho campeón de una causa de moralidad, de
justicia\ldots{} que era preciso descorrer el velo\ldots{} Esto del velo
lo repitió no sé cuántas veces\ldots{} En fin, me dio lástima. Paréceme
que el tal presbítero no tiene la cabeza buena. Yo me zafé como pude, y
luego me dijo Olózaga: «¿Sabe usted, D. Juan, que este pajarraco de
sotana es de los que hacen correr por ahí historias denigrantes en que
mezclan, sin ningún miramiento y quizás con aviesa intención, el nombre
de usted?\ldots---¿Qué me cuenta, Salustiano? ¡Mi nombre\ldots!---Sí,
señor: quieren minarle a usted el terreno, echando a volar especies
absurdas, actos o relaciones de la vida privada.»

Al oír esto, palideció el buen Milagro, y contestando a su jefe con un
monosílabo que expresaba tanta sorpresa como indignación, hizo solemne
voto mental de no volver a probar el\emph{curaçao} en lo que le quedara
de vida.

«No es la primera vez---continuó Su Excelencia,---que llegan a mí
rumores de esta naturaleza, unos verdaderos, referentes a los hechos y
casos que no tienen nada de ignominiosos, otros absurdos y sin ningún
fundamento, y otros van derechos contra mi reputación, contra mi
prestigio. Nada de esto me sorprende ni me arredra: sé que en mi
posición, y entre españoles, no puedo esperar más que una guerra en la
cual se emplean todas las armas, sin desdeñar las más viles. Con que ya
sabe usted: lo primero me escribe esa carta. Que detenga el nombramiento
para la cátedra de Alcalá. Ese Sr.~Hillo tiene todas las trazas de un
perturbado.»

---No creo tal, señor---dijo Milagro.---Quizás oiría el Sr.~Hillo algún
disparate de esos que hace correr la gente mal intencionada, y el pobre
señor lo habrá repetido\ldots{} Y también puede ser que soltara la
especie hallándose en ese estado de atontamiento que produce el\ldots{}
la\ldots{}

---Pero qué\ldots{} ¿es bebedor?

---No sé\ldots{} creo que\ldots{} Una noche, estando varios amigos en el
café con Maturana, el diamantista, este pidió \emph{curaçao} y quiso que
yo le acompañara; pero como no pruebo nunca ninguna clase de bebida, me
resistí, dándole las gracias. Hillo bebió y se puso perdido. Salió
diciendo cada desatino\ldots{} ¡Pero después, cuando el aire de la calle
le serenó, se desdijo de todo, y hasta lloraba el pobre recordando las
borricadas que habían salido de su boca! No es mal hombre: el
Sr.~Olózaga me dispense; que si algo contra la respetabilidad de
Vuecencia ha dicho ese clérigo, no ha sido con mala idea\ldots{}

---Bueno---dijo Mendizábal, cuya atención, queriendo abarcar mucho de
una vez, se detenía poco en un asunto.---Escríbame usted la carta a
Argüelles, incluyendo esta minuta de los principales puntos de Hacienda
que debe tener presentes al defender el \emph{voto de confianza}. Luego
carta citando a Istúriz y a D. Antonio González, para que nos pongamos
de acuerdo sobre el orden y método de discusión\ldots{}

Despedido el secretario familiar, entraron los que iban a la firma, y Su
Excelencia trabajó con ellos el resto de la tarde. Dos días después
empezó en el Estatuto la gran tremolina parlamentaria del \emph{voto de
confianza}, en que Mendizábal, blasonando de atrevido gobernante, pidió
a los Estamentos poder y autoridad para disponer de las rentas públicas,
con el desembarazo que exigían las críticas circunstancias por que
atravesaba la Nación.

Ya en aquellos debates empezó a torcerse la buena estrella del
reformador, que hasta entonces no había visto más que satisfacciones,
bienandanzas y popularidad. Los patriotas extremaron su oposición; los
llamados \emph{moderados} llenaban sus discursos de reticencias
maliciosas, chispazos que levantaban llamaradas y humareda en la opinión
neutral; y los amigos de Mendizábal, que hasta entonces le habían
defendido con ardor, empezaban a sentir ese frío triste, que es síntoma
de ver con malos ojos el bien ajeno. Algunos continuaban apoyándole,
porque estaban ligados por la gratitud; otros hacían de ésta tabla rasa,
y empezaban a mostrarse temerosos de que D. Juan de Dios realizase lo
que había ofrecido. Entre políticos, el fracaso de los grandes halaga a
los pequeños. La masa total no se entusiasma con el éxito si este lo
representa un hombre. La vulgaridad colectiva tiende siempre a conservar
el nivel.

Empezaron, pues, las inquietudes, las comezones, las ganitas de jarana,
y la curiosidad sabrosa de ver al jefe embarullado y sin saber por dónde
salir. Claro que los más votaban como carneros; pero otros se hicieron
los bobos, afectando escrúpulos de rigidez constitucional. A estos
llamaban \emph{santones}.

\hypertarget{xxv}{%
\chapter{XXV}\label{xxv}}

Aburrido y desalentado, vio D. Pedro Hillo entrar el año 36, a quien,
desde el primer día de su Enero, diputó tan calamitoso y funesto como su
antecesor el maldito 35, que todo se pasó en guerras, disturbios y
trapisondas. Nada había podido adelantar en la noble misión que se había
impuesto, y el problema que desentrañar quería presentábasele cada día
más obscuro y embrollado. Para colmo de amargura, Calpena no le refería
cosa alguna de su vida y planes; apenas pasaba con él breves ratos a las
horas de comida y cena, y luego a sumergirse volvía en la tenebrosa
cisterna del vicio y la deshonra, pues no otra cosa significaba para D.
Pedro la casa de la Zahón. Para mayor desdicha, tuvo el buen presbítero
el disgusto de saber, por un amigo de lo \emph{Interior}, que hallándose
extendido su nombramiento para la cátedra, Don Martín de los Heros le
había dado carpetazo por indicación del Presidente del Consejo. Esto le
llevó a una tristeza profunda, y no veía más que ocultos enemigos y
persecuciones misteriosas\ldots{} ¡Misterio por todas partes,
romanticismo y sombras espectrales! Lo único que alegraba su espíritu
era las cartas de la incógnita que, autorizado por Calpena, leía y
guardaba. En todas ellas latía la tristeza y el intenso cariño de quien
las redactaba. Véase un ejemplo: «Aunque diariamente recibo pruebas del
olvido en que me tienes, no puedo acostumbrarme a tu desobediencia. Te
mandé que fueras a la misa de once en el Carmen, y no fuiste ni a esa ni
a ninguna, pasándote toda la mañana en casa de la diamantista. Te
encargué la asistencia al Estamento para que oyeras y gozaras la
discusión del \emph{voto de confianza}, y tampoco pareciste por allí. Ni
en el \emph{Casón} de los Próceres se te ha visto tampoco, por más que
te recomiendo concurrir a menudo, para que habitúes el oído a las buenas
formas oratorias, para que tomes gusto a la política seria y veas de
cerca a los hombres eminentes que han de gobernarnos ahora y después,
los cuales serán malos, si quieres, pero con ellos tenemos que apencar,
porque no hay otros.»

»Te veo adquiriendo hábitos groseros: te has hecho huraño,
desagradecido, siempre devorado por insana inquietud, presuroso en todas
partes; te veo encenagado en una pasión loca, impropia de toda persona
regular; no haces caso de nada, no miras a tu porvenir, no correspondes
a la ternura de quien por ti se interesa y quiere dirigirte, sin que
mueva tu voluntad el considerar lo que esta protección reservada cuesta
y supone, ni las amarguras y sufrimientos que hay bajo de ella.»

Al terminar este pasaje, tuvo Hillo que suspender la lectura para
limpiarse los lagrimones que por sus mejillas resbalaban. Luego siguió
leyendo: «Y no paran aquí los estragos de tu devaneo amoroso, pues no
sólo te muestras ingrato conmigo, sino con ese buen sacerdote, tu
compañero de casa, que tanto interés demuestra por ti. Le desdeñas,
evitas su compañía porque quiere apartarte, como yo, del despeñadero a
que corres. Has delegado en él la lectura de mis cartas y la custodia de
tu dinero, prueba de confianza que me agradaría si no significara
indolencia y criminal olvido de tus obligaciones. El pobre Sr.~de Hillo,
por salvarte y correr tras de tus errores, ganoso de corregirlos, ha
dado un mal paso. De los males que se le ocasionen eres tú responsable.
Verdad que en su generoso afán, incurrió el cleriguito en la tontería de
pretender descubrirme y desenmascararme, y esto forzosamente había de
producirle algún desavío, porque nosotras las esfinges solemos dar un
zarpazo al que intenta descifrar el enigma que encerramos. Buscando
indicios aquí y allá, interrogando a gentes diversas, el Sr.~D. Pedro ha
oído enormes disparates, y cometido después la grave indiscreción de
repetirlos. Algunas de las absurdas hablillas que tu amigo recogió en
los cafés o en medio de la calle, afectaban al señor Presidente del
Consejo, y eran escandalosa infracción del respeto que se debe a la vida
privada. Alguien se enteró de ello, y fue con el cuento al Sr.~D. Juan
de Dios (a quien solemos llamar \emph{Juan y Medio} por su gigantesca
estatura), y he aquí que el grande hombre se amosca, demostrando cierta
pequeñez de espíritu, pues lo que de él dijo nuestro capellán no merecía
más que olvido y menosprecio: tan necia y ridícula era la invención.
¡Pobre Hillo! Acordado ya su nombramiento para la cátedra que pretende,
el Sr.~Mendizábal ordenó que se anulara. Paréceme este rigor poco digno
de un hombre que se nos ha venido acá con la pretensión de traernos el
reinado de la libertad, de la justicia y del orden social, y así pienso
decírselo. Perdóneme el Sr.~D. \emph{Juan y Medio}; pero me parece que
ha obrado como un \emph{santón} cualquiera, de esos que ahora le están
armando la zancadilla. El motivo de estas pequeñeces es que el grande
hombre considera la popularidad como el principal fundamento de su
fuerza, y le saca de quicio todo lo que puede mermar o poner en peligro
ese fantástico y vano poder. ¡Qué error! Fíjate en esto para que vayas
aprendiendo. La fuerza la da el buen gobernar, el cumplimiento de lo que
se ha ofrecido, la energía, la rectitud; de todo esto sale al fin el
aura popular. Pero pretender el calor de la opinión cuando no se hace
nada, o se hacen las cosas a medias, es grande ceguedad. De este mal
mueren todos nuestros políticos\ldots{} La confianza en un prestigio
ilusorio perderá a este buen señor, que podría indudablemente regenerar
el país si se cuidara menos de aspirar el incienso que le echan sus
aduladores y paniaguados. Buenas ideas trae, grandiosos planes ha
concebido; pero difícilmente logrará realizarlos, porque, como dice tu
amigo, no sabrá \emph{rematar la suerte.»}

Sonriendo pensativo, guardó la carta Don Pedro en la gaveta donde
metódicamente las iba poniendo, para dar cuenta a Calpena como
secretario fiel. Desconcertado por su fracaso, permaneció unos días en
situación expectante, soñando con inesperadas sorpresas de la
Providencia Divina, hasta que llegó otra carta de la incógnita, con la
particularidad de que no iba dirigida a Fernando, sino a él, al propio
D. Pedro Hillo, presbítero. Con vivísima emoción se encerró en su
cuarto, recatando el papel cual tímido enamorado que recibe la primera
esquela de la niña que adora, y leyó lo siguiente: «Sr.~de Hillo: Me
dirijo a usted como al único leal amigo del descarriado Fernando, para
suplicarle con efusión del alma que, mientras yo trato de cortar el
vuelo de esa criatura por los espacios tempestuosos del romanticismo,
intente usted poner estorbos a su temeraria iniciativa, y desbaratar sus
planes, aunque para ello tenga que valerse de las artes del disimulo, y
poner en juego resortes que, si bien algo violentos, no son ilícitos
tratándose de tan generoso y noble fin. Indudablemente, Fernandito y su
desatinada novia traman alguna travesura, que me temo sea de gravísimas
consecuencias. Sé que ese insensato ha comprado armas: dos pistolas,
espada, navajas grandísimas. Me permito encargar a usted que si el chico
ha llevado las armas a su casa, procure quitárselas sin miramiento
alguno, y esconderlas donde no las pueda recobrar; le recomiendo además
que le prive de dinero, dejándole sólo lo más preciso. Todo lo que
enviaré estos días, en la forma acostumbrada, hágame el favor de
recogerlo sin darle de ello noticia, y resérvelo para los gastos que
ocasionen las diligencias que hará usted, conforme yo le vaya indicando,
a medida que reciba más noticias de lo que traman esos pillos.

»Igualmente le invito, afrontando las objeciones que ha de hacerme su
delicadeza, a emplear en sus atenciones propias la parte que estime
conveniente del dinero de Fernando. No me venga usted con remilgos. Le
nombro capellán, o si se quiere, ayo de ese inexperto joven, y es muy
justo que perciba los emolumentos que de ley le corresponden. Déjese
usted de cátedras y de más correrías por los Ministerios pretendiendo
una plaza que ya no le hace falta para nada. Me figuro que sus posibles
se van agotando con tan ineficaz y largo pretender, y espero que sin
reparo alguno acepte usted lo que con todo el respeto debido le ofrezco.
¿Qué sería de usted si no aceptara? ¿De qué vivirá si, como es muy
probable, no le dan la dichosa cátedra? Usted no es hombre capaz de
hacer el parásito; usted no se humillará a postulaciones impropias de su
severa dignidad. ¿Qué remedio tiene mi buen cleriguito más que dejarse
querer, y admitir lo que nunca será proporcionado al gran servicio que
prestará a ese pobre niño? Además, ni tiene usted carácter para instruir
muchachos, ni podrá nunca acomodar su condición amable a tan ingrata
tarea. Si me promete no enfadarse, le diré una cosa: no está mi señor D.
Pedro muy versado en letras humanas, y apenas conserva en la memoria
unas cuantas reglas de retórica anticuada y fiambre, y ejemplos sueltos
de prosa y poesía, que ya están mandados recoger. ¿Ni cómo podía ser de
otro modo, si usted no coge un libro a ninguna hora del día, y no hace
más que hablar de política y toreo, y bromear con Nicomedes? El baúl de
libros que trajo de Zamora, lo tiene usted lleno de polvo y telarañas.
No ha sacado usted más que un par de cuadernos del \emph{Almacén de
frutos literarios,} de Burgos, y el primer tomo (A B) del
\emph{Diccionario de Autoridades}\ldots{} pero no lo sacó para leerlo,
sino para recalzar el colchón de su cama que se le hundía por los
pies\ldots{} Quedamos en que no más retórica, no más echar los bofes
detrás de una cátedra que desempeñará mejor otro cualquiera. Desde hoy
se consagra usted a Fernando, a salvarle del deshonor, a traerle al
camino de la honestidad, de la obediencia a los superiores. Es usted,
con menos humanidades, pero no con menor abnegación y cariño, el sucesor
del benditísimo párroco de Vera, D. Narciso Vidaurre. No me replique,
Sr. Hillo, ni me ponga esa cara compungida. Cállese usted y obedezca.»

Mediano rato estuvo D. Pedro sobrecogido de la fuerte emoción, que hubo
de manifestarse en lágrimas y suspiros. Estimando la confianza que en él
ponía la divina incógnita, más que la oferta de recursos materiales,
decidió aceptar oficialmente el cargo que ya por su voluntad oficiosa
desempeñaba, y consideró que rechazar el estipendio sería insigne
ingratitud y gazmoñería. Era una salvación milagrosa, pues ya se le
acababan a toda prisa los dineros, sin que de ninguna parte pudieran
venirle rentas ni gajes, como no fuesen los de la misa que diariamente
celebraba. Precisamente había pensado días antes que si no malbarataba
todos sus libros, no tendría con qué pagar la casa.

Contento y animoso, sintiendo duplicado el interés por Fernandito y el
respeto y admiración de la oculta deidad, dedicó toda su energía a
desempeñar la misión que aquella con suprema autoridad le había
conferido. Registrado el cuarto de Calpena, no encontraron armas.
Recelando que las tuviera en la cómoda guardadas con llave, pensó en
proveerse de ganzúa para sustraerlas, pues la incógnita le había mandado
que no se parase en pelillos. Pero en esto llegó nueva carta, que decía:

«No busque más las armas, señor presbítero, porque las tiene en casa de
un amigote con quien ahora intima mucho: Patricio de la Escosura, el
artillerito ese a quien suponen, y debemos creerlo, la última mosca
cogida en las redes de esa araña de la Oliván. Escosura y otro joven
llamado Miguel de los Santos (no me acuerdo del apellido), son ahora los
inseparables de Fernando: me figuro que este último le acompañará alguna
vez a casa de la Zahón. Según mis noticias, es un truhán de primera, que
de todo saca partido para divertirse. Vive en la calle de la Gorguera.
Suele andar con uno de los chicos de Madrazo, Perico, a quien apenas
apunta el bozo, pero que ya es poeta y prosista. Todos estos niños y
otros se traen unas ideas sentimentales que creo yo harán más estragos
que los devaneos fúnebres, incendiarios y sanguinolentos del
romanticismo. Busque a ese Miguelito de los Santos y hágase su amigo.

»Y vamos a lo principal. Esté usted preparado para un viaje, ¡oh
pacientísimo señor D. Pedro!, y perdone que le haga andar de coronilla.
Dentro de unos días, quizás mañana o pasado, será Fernando trasladado a
una Intendencia de provincia, probablemente a Cádiz o Barcelona, lejos,
lejos. Se le destina a las nuevas oficinas que se crean para la
redención de censos y la venta de bienes del clero. No creo que se
rebele contra las órdenes del Ministro, negándose a salir. Si así lo
hiciera, será preciso recurrir a otros medios. Pero no es probable que
llegue a tanto su rebeldía\ldots{} Oiga usted lo que tiene que hacer. En
cuanto él reciba su nuevo nombramiento, que irá acompañado de una orden
para salir en posta, usted le incita a no dilatar la partida, le dispone
coche, se brinda a acompañarle, le dice que volverán pronto; pero la
vuelta de ustedes será la del humo; y una vez allá, trínquemele bien. Si
logramos apartarle de su infierno siquiera cuatro o cinco meses, estamos
salvados, mi buen amigo y \emph{coadjutor}.

»Otra cosa tengo que advertirle. Debe usted, desde que disponga el
viaje, abandonar el traje eclesiástico y vestirse de corto. Hasta creo
que le sentará bien la ropa de \emph{hombre}, digo, \emph{de
paisano}\ldots{} tampoco es esto; vamos, de seglar. Como los vientos que
hoy corren en España no son muy favorables a las personas eclesiásticas,
por la guerra que estas hacen al Gobierno, unos con las armas en la
mano, otros con sermones y escritos virulentos, no le conviene a nuestro
cleriguito echarse con sotana y balandrán por esos mundos. Tenga
presente que dentro de quince días, lo más, saldrá el decreto en que se
ordena limpiar a los frailes el comedero, y ya verá usted la tremolina
que se arma\ldots{} Con que cuidado: fíjese bien en lo que me permito
indicarle, y procure cumplirlo, sin nuevos intentos de descubrirme,
porque si llega a mis oídos el \emph{mascarita te conozco}, no hemos
hecho nada. Yo me quedo donde estoy; Fernando, en su laberinto de
perdición, y usted en su páramo de cazador de cátedras. Adiós.»

\hypertarget{xxvi}{%
\chapter{XXVI}\label{xxvi}}

Jurando \emph{in mente} hacer todo lo que le mandaba la que tenía ya por
autoridad suprema y tirana indiscutible, se fue Hillo al Estamento de
Procuradores, donde le había citado Iglesias para presentarle a D.
Agustín Argüelles. Habían concertado destruir, por mediación del que
llamaban \emph{Divino}, la mala impresión de Mendizábal con respecto a
Don Pedro, haciéndole ver que ni era loco ni había sido difamador de Su
Excelencia, pues si bien dijo en cierta desgraciada ocasión cuatro
palabrejas inconvenientes, hízolo con el noble fin de condenarlas. Menos
le importaba la cátedra, con importarle mucho, que la opinión que el
señor Ministro formase de él; y hasta que no lograse rectificar aquel
temerario juicio, no tenía tranquilidad. Mas desde el momento en que
aceptaba el cargo que la divinidad incógnita le había conferido, ya la
suspirada cátedra y los Ministros que la concedían, y todo el Gobierno,
y lo que Mendizábal pensara de clérigos locos o calumniadores, le
importaba un bledo. Iba, pues, con ánimo de decir a Iglesias: «Amigo
mío, no haga usted nada, ni se tome el trabajo de presentarme a estos
señores, pues renuncio a la \emph{mano de doña Leonor}, y es muy
probable que me vaya a mi pueblo, a cavar.»

En los pasillos del Estamento había tanta gente, que le fue muy difícil
cazar a Nicomedes. La sesión era interesantísima: se discutía el
\emph{voto de confianza}. Anduvo de aquí para allá, saludando a los que
encontró conocidos, y uno de estos le dijo que Iglesias estaba en la
tribuna oyendo hablar a Toreno. Hablaría después Mendizábal, y se
procedería inmediatamente a la votación. Arrimose Hillo a una de las
puertas laterales, donde había una gran masa de intrusos aplicando la
oreja al rumor oratorio, y oyó algunas palabras del Conde, pocas y
desvanecidas por la distancia. El local era malísimo: el salón de
sesiones una iglesia secularizada. Para formar los pasillos circundantes
se habían derribado tabiques de la sacristía, aprovechando con fáciles
chapuzas la parte de capillas y salas interiores que destruyó el
incendio de 1823. Buscó Hillo mejor sitio de escucha por otro lado, y al
fin, agazapándose en un rincón de lo que fue camarín de la Virgen, y que
caía detrás de la Presidencia, pudo ver y oír algo. Por entre una
crestería de cabezas distinguió a lo lejos la del Sr.~Mendizábal y parte
de su busto. Acababa de levantarse, y hablaba premioso, mirando, ya al
pupitre, ya a los \emph{señores de enfrente}. Por su gigantesca estatura
descollaba D. Juan entre aquel cúmulo de hombres chicos y medianos. A su
corpulencia no correspondía su voz, parda y cavernosa, ni menos su
oratoria, que en las cuestiones de Hacienda era muy árida, y en las
políticas elevábase tan sólo por la energía que le prestaba su
convicción y los tonos dulces que le daba la sinceridad. Estirando mucho
el pescuezo por entre brazos y cabezas de curiosos que bloqueaban la
puerta, pudo pescar Hillo alguna que otra frase: «\ldots Pues habiendo
tenido la suerte de negociar un empréstito para una nación vecina a 74
por 100, cuando Don Miguel\ldots» Y después: «Se ha dicho aquí si el
Gobierno, en virtud del artículo 3.º\ldots» Siguió un concepto
ininteligible, y luego: «Pero, señores, un Gobierno que no quiere apelar
a poner una contribución extraordinaria, ¿cómo es posible que\ldots?»
Retirose Don Pedro aburridísimo, viendo que nada en limpio sacaba, y
esperó paseándose, leyendo la orden del día puesta en una tablilla, o
los partes de la guerra, que siempre decían lo mismo. Por fin, comenzada
la votación, los parroquianos de tribunas descendían a los salones bajos
y pasillos. Los Procuradores, conforme votaban, iban apareciendo por las
puertas del salón de sesiones, y el tumulto crecía, la atmósfera era
espesa y cálida, y el ruido bastante a marear la cabeza más firme.

Apareciósele Nicomedes, sofocadísimo, echando lumbre por los ojos, entre
un pelotón de periodistas, y desde lejos le intimó en esta forma: «¡Eh,
clérigo\ldots{} en qué mal día viene! Imposible hacer nada hoy. Ya ve
\emph{su merced} el jaleo que hay aquí.» En pocas palabras le informó D.
Pedro de que no venía más que a retirar todo lo actuado, y a manifestar
a su amigo que ya no quería más recomendaciones ni molestar a nadie. Sin
hacer caso de lo que decía el presbítero, prorrumpió Iglesias en
ruidosas exclamaciones, a las que siguieron cláusulas narrativas, en
pintoresco y familiar lenguaje: «¡Válgame Dios, qué discurso nos ha
largado el camello! Lo que me hace más gracia es el tonillo sentencioso
que toma para decir las mayores simplezas.»

Apretose el corrillo alrededor de Iglesias (metiéndose en él D. Pedro
con empuje de codos), y uno de los jovenzuelos más avispados que en el
cotarro bullían, se echó a reír diciendo: «¿Pero ustedes le oyeron los
latines con que hoy nos ha obsequiado?\ldots{} \emph{Mutatas
mutandas.}.. Es divino este señor.»

---Él no sabrá de \emph{citas históricas}, como dijo ayer\ldots{} pero
lo que es gramática\ldots{}

---Esto del \emph{voto de confianza}---manifestó con saña
Nicomedes,---resulta lo que digo en mi artículo de esta mañana: un
\emph{cubilete de charlatán}.

---Como que todo esto no es más que un tapujo de los agios y embrollos
que este \emph{D. Juan y Medio} se trae.

---Bueno es el mundo, bueno, bueno, bueno---dijo uno de los presentes,
mozo espigadillo, de grandísimos ojos negros, que relampagueaban en su
rostro expresivo, con una seriedad que por ser tan seria resultaba
extraordinariamente burlona.

---Eso mismo digo yo---indicó Hillo tímidamente.---Bueno, bueno,
superior.

---Mi queridísimo amigo Miguel Álvarez---dijo Iglesias, presentándole.

Diéronse las manos, y D. Pedro se mostró muy afectuoso, pues aquel
encuentro y presentación colmaban sus deseos, y se permitió decir al
joven Álvarez que ya le conocía de nombre por sus galanas poesías, por
sus artículos y discursos\ldots{}

«Discursos no---replicó el otro con gravedad socarrona,---porque todavía
no los he pronunciado. Los tengo, sí, aquí en mi mente, y no los cambio
por los de Cicerón. Pero todavía están inéditos, Padre\ldots{} Yo
también tenía vivos deseos de conocerle a usted personalmente\ldots{}
que de fama ¿quién no le conoce? Mi amigo Fernando Calpena me ha hablado
mucho de usted\ldots{} Sé que es un profundo humanista, y que distrae
sus ocios en la afición taurina\ldots{} Yo soy amantísimo de los Toros.»

---Lo que tú eres, bien lo veo---dijo Hillo para su sotana:---un guasón
de primera.

Y siguieron charlando, mientras Iglesias, con hueca voz ponderativa,
encomiaba el discurso pronunciado en la primera parte de la sesión por
D. Agustín Argüelles, a quien se seguía llamando \emph{el Divino}, si
bien no aplicaban todos este lisonjero mote en sentido recto. «¡Señores,
vaya un discurso el de Don Agustín! Es de los mejores, de los más
elocuentes que ha pronunciado en su larga vida parlamentaria. Si el
\emph{camello} hablara así, ¿quién le aguantaba?»

Y deteniendo a un joven espigado, pulcro, bien afeitadito, vestido con
esmero y elegancia, que de un cercano grupo se desprendía, le dijo:
«Querido Juan, ven acá. ¿Qué te ha parecido el discurso de la
\emph{divinidad}?»

---Verdadera divinidad tutelar es D. Agustín para ese buen señor. ¿Qué
sería de Mendizábal sin esta defensa, sin este escudo, sin esta
protección?

---Sería lo que la yedra, cuando muere el tronco del olmo a que se
agarra---dijo uno de los que se adherían a Iglesias.---A ver, Sr.~D.
Juan Donoso, usted que lo entiende, ¿qué opinión ha formado del discurso
de Don Agustín?

---Admirable como forma---declaró con aire de suficiencia el que
llamaban Donoso, joven extremeño que iba para notabilidad literaria y
política,---poco sólido como aparato dialéctico. Me recuerda la oración
\emph{Pro lege manilia}. Fáltale la primera condición de toda pieza
oratoria, el convencimiento. Se ve que no cree en la leyenda de este
buen señor, ni en sus planes, ni ve nada dentro del artificio del
\emph{voto de confianza}. Le defiende porque no es decoroso despedirle
cuando hace tan poco tiempo que nos le han traído con tanta parambomba.
Para mí esto es claro. El generoso D. Agustín, empleando excesivamente
la argumentación \emph{extra causam}, ha sabido cubrir con la púrpura de
su elocuencia esta olla vacía\ldots{}

Alejose llamado desde el cercano grupo, y dejó el puesto a otro de los
amigos de Iglesias, al inquieto y vivaracho González, el cual, antes de
que le preguntaran, se metió en el corrillo diciendo: «Caballeros, para
mí, este buen D. Agustín chochea\ldots»

Prodújose después de esto un silencio repentino, porque apareció el
propio Argüelles, viniendo del salón hacia la sala donde despachaban y
recibían los Ministros (que era parte del refectorio del transformado
convento; en la otra parte se reunían las juntas de comisiones). Pero
acosado por los felicitantes y aduladores, el buen señor no podía dar un
paso. «Bien, D. Agustín, sublime\ldots{} Como siempre, el Demóstenes
español.» Y él, con bondades y modestias, de esas que se usan en la
política, desplegando todo aquel sonreír dulce y un poquito clerical,
que caracterizaba su rostro austero, respondía: «He salido del paso como
he podido\ldots{} No tenía más remedio que defender el \emph{voto de
confianza}, que es un resorte político y parlamentario muy recomendable
en ocasiones como la presente\ldots{} No sé de qué se maravillan estos
señores moderados; si en el Parlamento inglés estamos viendo todos los
días esta clase de concesiones amplias a la iniciativa
gubernamental\ldots{} Creo haber puesto la cuestión en su verdadero
terreno\ldots{} Ya se le habrá pasado el susto al pobre
Mendizábal\ldots»

---Sr.~D. Agustín---le dijo Iglesias con toda la franqueza compatible
con el respeto,---es usted el hombre de más abnegación que existe en el
mundo. Yo creí que ciertas virtudes eran incompatibles con la política;
pero ya veo que no, ya veo que no.

---¿Por qué dice usted eso?---preguntó el \emph{Patriarca de la
libertad}, más risueño que sorprendido.---He cumplido con mi
deber\ldots{} Están ustedes soñando si creen\ldots{}

---No les ha parecido ésta buena ocasión para derribar el falso ídolo.

---Aquí no somos idólatras, amigo Iglesias: aquí no hay más que hombres
de buena voluntad que trabajan por la libertad y el bien del país, cada
cual según lo que puede y sabe\ldots{}

Y acosado por la turba de felicitantes, siguió de grupo en grupo,
perdiéndose entre el gentío. Trueba y Cossío, secretario de la Cámara,
pasó saludando risueño; mas no quiso dar su opinión. En un grupo de
ministeriales, de los empedernidos, claveteados de optimismo, decían:
«Argüelles, haciendo equilibrios; Toreno velado, avieso, dejando
traslucir, hoy más que nunca, su mala intención; Mendizábal admirable,
diciendo claramente lo que debe decir y callándose lo que le conviene
reservar.»

---Esta es la verdadera elocuencia parlamentaria, a la inglesa\ldots{}
Lo que yo digo: el Parlamento no es una academia. Aquí se viene a
ilustrar las cuestiones.

Y más allá: «Esto es una farsa. Lo que se quiere es desacreditar la
representación nacional\ldots{} poner en un conflicto a la Corona\ldots»

---Y el desquiciarlo y revolverlo todo, ya está visto, para traernos el
reinado de la plebe\ldots{}

---Que sigan así las cosas, y pronto tendremos que no hay más que dos
partidos: la camisa sucia y la camisa limpia.

---Se ve venir el imperio de las chaquetas. Las levitas van a menos.

---No así las de \emph{D. Juan y Medio}, que cada día son más largas.

Salió al fin del tumulto D. Pedro acompañando al joven Álvarez, y como
este dijera que iba al café del Príncipe, vulgo Parnasillo, se pegó a
él, pretextando quehaceres en la misma calle, con la plausible intención
de sonsacarle lo que supiera referente a Fernando. En la Carrera
encontraron a Pepe Díaz, y estando con él de conversación, llegaron por
la calle del Lobo otros dos, que Hillo no conocía. Eran Segovia y Juan
Bautista Alonso, que traía bajo el brazo un rimero de poesías. Nada más
frecuente entonces que ver a los mozalbetes por la calle cargados de
paquetes de versos, como si vinieran de compras.

«Oye, tú---dijo Segovia a Miguel de los Santos cogiéndole de las
solapas,---he visto a ese chico que me recomendaste, ese Eugenio\ldots»

---Hombre, sí\ldots{} excelente chico. ¡Qué simpático, qué modesto! Por
cierto que no acabo de aprender su nombre.

---Ni yo. Espérate a ver si me acuerdo\ldots{}

---Yo me acuerdo, yo---dijo Díaz rascándose la frente.---Un apellido
endemoniado\ldots, así como\ldots{}

---Es hijo de un alemán---indicó Alonso.---Le conozco, sí\ldots{} Su
padre le ha hecho un flaco servicio llamándose como se llama.

---Ya me acuerdo\ldots{} \emph{Arzen}\ldots{} \emph{Arzin}\ldots{}

\emph{---Arzembuch,} escrito con \emph{H} y con \emph{n.}

---Justo, así es---añadió Segovia.---Pues, como te digo, el pobre
muchacho no sabía qué hacer conmigo. Me llevó a su casa y me enseñó una
obra\ldots{} ¡Vaya una obra!

---¿En prosa o en verso?

---¿Pero qué dices ahí?\ldots{} ¡Si era una mesa!

---¡Una mesa! Verdad que es carpintero antes que poeta.

---Si a la caoba llamas tú poesía, la mesa es una obra en verso.

---¿Y esa mesa no tenía cajón?

---Hombre, sí; y del cajón sacó cuatro tragedias y dos comedias del
teatro antiguo barnizadas por él\ldots{} \emph{Los empeños de un acaso}
y \emph{La confusión de un jardín}.

---Ya caigo---dijo Alonso:---es el autor de aquella famosa
\emph{Restauración de Madrid} silbada horrorosamente en la Cruz hace dos
o tres años.

---¡Pobre Eugenio!---exclamó Díaz,---es tan tímido, tan para poco, que
no saldrá adelante, valiendo mucho y sabiendo lo que sabe.

---Pues veréis: entre las tragedias que sacó del cajón de la mesa, había
un drama, los dos primeros actos de un drama\ldots{}

\emph{---Los amantes de Teruel}\ldots{} ¿te los leyó?

---Empezaba yo a leer, cuando entró ese loquinario, ese Calpena,
y\ldots{} Él fue quien leyó, ¡pero con una entonación, chico\ldots!,
vamos, tan bien leía, que si nos encantó la obra, no nos maravilló menos
el intérprete.

---Ya le he dicho---indicó Alonso,---que debe dedicarse al teatro, a la
escena. Sería un gran actor.

---¿Y dónde dejasteis a Calpena?---preguntó Álvarez.

---Con Eugenio ha ido al Príncipe, a ver el ensayo del \emph{Antony}.

---Pues allá me voy\ldots{} ¿Vamos?

Excusáronse Alonso y Díaz por tener quehaceres, que debían de ser
poéticos; pero Segovia se agarró del brazo de Álvarez, con ánimo de
acompañarle. Calle abajo se fueron dos, y los otros, con el pegadizo D.
Pedro, se metieron por la del Lobo. Por cierto que el buen presbítero,
ya en la pista de su D. Fernando, si por una parte se hallaba satisfecho
de haber encontrado en Miguel de los Santos un diligente y afectuoso
auxiliar de su campaña, por otra se sentía contrariado de tener que
abandonar el campo, cuando tan favorables circunstancias aquella tarde
le ofrecía el acaso, o la Divina Providencia. Al despedirse de Álvarez
en la puerta del teatro por la calle del Lobo, le dijo apenadísimo: «No
saben cuánto siento no poder colarme con ustedes en el ensayo. Me gusta
extraordinariamente ver ensayar\ldots{} ¿Pero cómo entro vestido de
cura? No puede ser. Otra vez será.»

Y se fue triste y cabizbajo, diciendo a las baldosas de la calle: «Razón
tiene la señora incógnita al recomendarme que para andar en estos trotes
me vista de seglar\ldots{} No más hábitos. Por San Juan Capistrano,
mañana mismo los ahorco.»

\hypertarget{xxvii}{%
\chapter{XXVII}\label{xxvii}}

Salió D. Fernando Calpena del ensayo de \emph{Antony} con un grave
aumento de la locura que ya por sus exaltados amores padecía, y al
despedirse de su amigo Juan Eugenio en la esquina de la calle de las
Huertas, le dijo que ni se había escrito ni se volvería a escribir un
drama tan excelente, verdadero Evangelio de los desheredados a quienes
oprime la balumba del artificio social. El carpintero-poeta, cuya mente
conservaba un excelso reposo, no expresó nada en contra de tan radical
opinión; pero algo tenía que decir, sin duda, sólo que se lo reservaba
para más adelante, cuando los años y la experiencia le dieran la
autoridad de que entonces carecía. No hizo más que mirar a su amigo con
aquella expresión de intensísima agudeza que conservó hasta su vejez, y
apretarle las manos. Al separarse le dijo: «Tendré copiado el acto
tercero el sábado, y en seguida podrás leerlo. Aparece Isabel en la
primera escena, vestida para la boda\ldots{} luego entra D.
Rodrigo\ldots{} En fin, ya lo verás. Adiós.» Y echó a correr hacia su
casa, con pasito corto y vivaracho. Era pequeñín, todo nervios, con una
cara ratonil, graciosa y llena de inteligencia, unos ojuelos que
despedían lumbre, y una boca como la de los ángeles feos, que también
los hay, según dicen. Calpena le miró alejarse, y, melancólico se decía:
«¿Por qué Dios no me dio a mí su talento?\ldots{} Bien podía habérmelo
dado, sin quitárselo a él\ldots{} bien podía\ldots»

La transformación moral del enamorado joven se traslucía claramente en
lo físico: había enflaquecido; sus ojos, que antes eran hermosos y
alegres, brillaban después de la crisis con mayor hermosura, y su
alegría era extraña combinación de zozobra y delirio. Hablaba con más
viveza, amontonando ideas sobre ideas, empleando con frecuencia imágenes
felices. Vestía con elegante descuido, olvidado ya del atildamiento
presuntuoso que hacía de él un perfecto \emph{estatuista} en capullo.
Dejaba crecer la negra melena y la mantenía crespa, indómita, dando a
los rizos y mechones libertad para estirarse o encogerse como quisieran.
Había llegado a adquirir, con estas y otras costumbres nuevas, un sello
propio, personal, que le distinguía y señalaba entre sus amigos. Estos
eran cada día en mayor número desde que se lanzó a la independencia, y
los tomaba conforme le iban saliendo, aristócratas o plebeyos: se
mezclaba en la turbamulta humana con indecible gozo, ávido de vivir, de
ver, de apreciar y discernir, de ejercitar, en fin, toda la energía
intelectual y moral que a raudales brotaba de todas las honduras de su
alma renovada.

Hizo en aquellos días conocimiento con los Madrazos, Federico y Perico,
el uno precoz artista, el otro escritor y poeta, ambos excelentes
muchachos, entusiastas, locos por el arte y la belleza; con Ochoa,
inseparable de aquellos y co-fundador de \emph{El Artista}, para el cual
unos escribían y otros dibujaban; con Villalta, con Trueba y Cossío,
político audacísimo al par que escritor bilingüe, pues lo mismo escribía
en inglés que en español; con Dionisio Alcalá Galiano, hijo de D.
Antonio, uno de los jóvenes más despiertos y más inteligentes de aquel
tiempo; con Revilla, Gonzalo Morón, Larrañaga y otros que en la
literatura, en la crítica y en la política empezaban a bullir; con ambos
Escosuras, con ambos Romeas, con Guzmán y Latorre; y al propio tiempo
intimó más con Espronceda, Mesonero, Roca de Togores, Ventura, y otros
que ya conocía. Aquella juventud, en medio de la generación turbulenta,
camorrista y sanguinaria a que pertenecía, era como un rosal cuajado de
flores en medio de un campo de cardos borriqueros, la esperanza en medio
de la desesperación, la belleza y los aromas haciendo tolerable la
fealdad maloliente de la España de 1836.

Más firme cada día en la fe de sus amores, veía Calpena en Aura algo más
que una mujer bella, veía la mujer misma, con todas las cualidades
propias del sexo en grado superior. Por perfecta la tenía desde la punta
del pie a la última mata del cabello; perfecta era también en su
inteligencia, que exhalaba rayos; en su voluntad ardorosa, rebelde a los
términos medios; en sus caprichos, que escondían una profunda
psicología; en todo, Señor, en todo, pues si Aura reía, toda la
Naturaleza se alegraba con ella, y si lloraba, Cielo y Tierra se cubrían
de tristeza.

Pues, señor: bastantes días habían pasado desde el ensayo del
\emph{Antony}; bastantes, sí, porque ya se había estrenado el
revolucionario drama de Dumas, cuando ocurrió lo que ahora se referirá.
Ello fue al principiar Febrero, pasadas las tremolinas parlamentarias de
fin de Enero, cuando se discutió la ley electoral y derrotaron al
Gobierno, y el señor de Mendizábal, entre la espada y la pared, no tuvo
más remedio que disolver los Estamentos y convocar nuevas Cortes. Y como
el diablo, cuando no tiene que hacer, se entretiene en coger moscas, D.
Juan de Dios, libre de la fatiga del Parlamento, que tan agobiado le
traía, se dedicó a remover el personal de su Ministerio: todo era
traslaciones, cesantías, empleados que venían no se sabe de dónde; otros
que se iban a sus casas a \emph{mascar el vacío}, como dijo un cesante
de aquel tiempo\ldots{} En fin, que una tarde, hallándose Calpena en su
oficina aburridísimo, esperando ansioso la hora, antes que esta llegó un
antipático, maldecido papel\ldots{} ¡Ay!, era nada menos que su
traslación a Cádiz, a las secciones recientemente creadas para la
Liquidación de Créditos. El efecto que esto le hizo fue deplorable: vio
en ello la malquerencia de un oculto enemigo, y echaba pestes contra los
malos Gobiernos y contra el propio D. Juan de Dios, a quien desde aquel
día retiró su admiración y cariño.

En aquel estado de amargura y rabia le encontró Hillo una mañana, cuando
de vuelta de misa disponíase a endilgar la ropa \emph{corta} para
echarse a la calle.

«¡Pero, chico---le dijo,---si estás de enhorabuena! Vas a Cádiz, la
\emph{cuna de nuestras libertades}, como decís los patriotas, y allí
vivirás como un príncipe, y harás conquistas, y beberás la rica
manzanilla, y tienes ancho campo para conspirar con los Riegos de ogaño
por la Constitución del 12.»

---Ni usted sabe lo que se dice, ni yo voy a Cádiz---replicó Fernando de
malísimo talante.---Pensaré de hoy a mañana lo que debo hacer, y se lo
diré a usted\ldots{} Veo la mano, sí; veo la mano que en las tinieblas
me ha descargado este golpe de maza\ldots{} Pero no caeré, no: si creen
que voy a desplomarme, a rendirme y a pedir perdón, se equivocan. Abur.

Se marchó con esta seca despedida, y Don Pedro no volvió a verle hasta
el día siguiente. No pocas noches dormía fuera de casa. Leyendo dramas o
charlando de literatura en casa de algún amigo, se le pasaban las horas
insensiblemente, y sorprendido por la aurora en esta febril tarea, se
quedaba dormidito en un sofá o en el santo suelo, ya en el hospedaje de
Álvarez, ya en el de Pepe Díaz. También D. Pedro andaba un poco salido:
entre diez y once de la mañana se vestía de paisano y se lanzaba a
divagar callejero; por tarde y noche frecuentaba los cafés, y hacía en
unos y otros diversas amistades. En el de Solís encontró a Calpena con
un chicarrón que iba cargado de dramas: le vio desde lejos, se acercó en
el momento en que salía, le fue siguiendo, y, por fin, le dio alcance en
la calle del Turco.

«Voy contigo---le dijo poniendo en práctica las instrucciones
últimamente recibidas.---Tenemos que hablar. ¿No sabes lo que ocurre?
Pues que mañana nos largamos.»

---¿A dónde, mi reverendo amigo y capellán?

---A Cádiz: tengo yo también allí un asuntillo. ¡Qué oportunidad!, me
acompañas y te acompaño.

---Irá usted solo. Mejor va uno solo que mal acompañado. Yo, Sr.~D.
Pedro Hillo, no salgo de Madrid\ldots{} Y no me ponga usted la cara
fosca y patibularia, porque como no es usted mi padre, ni mi tío, ni
menos mi abuelo, y tan sólo es un amigo muy apreciable, yo no estoy en
el caso de que usted me riña.

---Hombre, reñirte no---repuso Hillo con mansedumbre.---Somos tan sólo
amigos, dices bien, y ninguna autoridad tengo sobre ti, como no sea la
que me dan los años. ¡Triste autoridad!\ldots{} Bueno, bueno: no quieres
ir a Cádiz. \emph{Ergo}, ¿renuncias a tu destino?

---Renuncio, sin \emph{ergo}; presento la dimisión\ldots{} le digo al
Sr.~Mendizábal que vaya él si quiere\ldots{}

---Pues, hijo, siento hacerte una observación que te va a saber muy
mal\ldots{} pero qué remedio, es mi deber hacértela, para que medites el
caso, y resuelvas tu libérrima voluntad\ldots{} Ya leo en tu cara que lo
has adivinado. Palideces\ldots{}

---Palidezco de verle a usted tan meticuloso, empleando rodeos y
perífrasis para decirme algo que podrá ser amargo y triste, pero que no
me anonada, no señor, no me anonada\ldots{}

---¿Sabes\ldots?

---Y si no sé, sospecho\ldots{} Vaya, suélteme usted pronto el rayo.

El bigardón que llevaba a cuestas mediano fardo de dramas y tragedias en
cuatro y cinco actos, con prólogo y epílogo, comprendiendo que trataban
de asunto delicado, se largó, dejándoles en su grave contienda en medio
de la calle.

«Pues lo que debía suceder ha sucedido. La deidad próvida, la dulce
enmascarada, nuestra grande amiga, nuestra\ldots»

---Hombre, acabe usted de una vez. Total, que se ha incomodado porque no
quiero ir a Cádiz. ¿Y cómo sabe mi resolución?

---No la sabe, la teme, y dice en su última carta que si no vas no
cuentes más con ella.

---Creo---dijo Calpena con gravedad,---que no falto a la gratitud
respondiendo que no acepto la protección en esa forma despótica,
altanera. Se obedece ciegamente a una madre, a un padre, aun cuando la
obediencia nos destroce el corazón; pero ¿quién puede exigir que
sacrifiquemos la libertad, dignidad, vida, a los caprichos de un
fantasma? ¿Que no es fantasma dice usted? Pues que se quite la gasa, el
capuchón\ldots{} Abandonado estuve, abandonado estoy\ldots{} ¿Qué me ha
dado el fantasma? ¿Me ha dado un nombre? ¿Me ha dado algo más que
algunos trajes y algún dinero? ¡Y a cambio de estos beneficios, pide que
me convierta en un párvulo sin voluntad, sin iniciativa para nada! Amigo
Hillo, antes que el bienestar adquirido con una pasividad humillante,
pueril, ridícula, quiero una pobreza con dignidad\ldots{} No, no entra
en mis ideas vivir de lo que se me arroja en mitad de la calle; soy
joven, no me falta inteligencia: quiero vivir por mí y para mí\ldots{}

---Todo eso está muy bien---dijo el clérigo.---Quieres trabajar, lucir
tus facultades. ¡Magnífico! Pero, tonto, si con la protección del
fantasma lo harás mejor que solo y abandonado. ¿A qué luchar
desesperadamente para sucumbir\ldots? En cambio, con la base de tu
destinito\ldots{}

---No sea usted inocente, D. Pedro. ¡El destinito!, ¡vivir amarrado al
pesebre de la administración! ¿Pero no comprende usted que el que una
vez prueba las facilidades de ese pesebre, ya está enviciado para toda
la vida, ya no se pertenece, ya es una máquina que los ministros paran o
echan a andar, según les acomoda? No, no me digan que sea
máquina\ldots{} En los empleos tiene usted la explicación de la inercia
nacional, de esta parálisis, que se traduce luego en ignorancia, en
envidia, en pobreza\ldots{}

---Muy bonito como teoría\ldots{} pero\ldots{}

---De esto hablamos anoche largamente Larra y yo, y renegamos de los
empleos, que son como el opio o el hastchís2 para esta nación viciosa,
indolente. Por mi parte, digo que antes comerán en un mismo plato
constitucionales y facciosos, antes se volverán chaquetas las levitas de
D. Juan Álvarez, que yo resignarme a ser toda mi vida funcionario
público.

---Has empleado lindamente la figura que llamamos \emph{imposible o
adynaton}.

---Déjese ya de retóricas, D. Pedro. ¿Cree usted que están los tiempos
para retóricas? Eso pasó. Aquí vendrá un desquiciamiento si no vienen
nuevas ideas, aire nuevo, a regenerarnos\ldots{}

Y abriendo los brazos en plena calle, parados uno frente a otro, dijo a
su amigo: «Déjeme usted ser libre, déjeme usted probar mis
fuerzas\ldots{} No quiero protección anónima. Si conoce usted a la
divinidad encapuchada, dígale que quiero pertenecerme, pensar por mí
mismo y poner en ejecución lo que pienso\ldots{} ¿Que me estrello?,
bueno: Pues estrellado y con media vida, podré decir: `¡Viva la
independencia! ¡Viva la dignidad humana!'.»

\hypertarget{xxviii}{%
\chapter{XXVIII}\label{xxviii}}

Separáronse. A los pocos días se despidió Calpena de la casa de Méndez,
porque en su nueva vida independiente, abandonado de la invisible
protección, necesitaba aposentarse con mayor economía. Tanto Méndez como
su hija y esposa con lágrimas en los ojos viéronle salir, y le abrumaron
con amabilidades quejumbrosas, mostrando lástima de su partida, por un
\emph{punto de quijotismo}, como decía el patrón, el cual añadió a esta
frase sanos consejos y exhortaciones atinadísimas. «¡Vaya que dejar un
empleo tan bueno por no ir a Cádiz!»---clamaba Doña Cayetana,
oprimiéndose el pecho, que rebotaba contra la garganta. «Y ¿por qué no
han de dejarle aquí?---decía Delfinita bizcando más el ojo.---También es
tema querer echarle de Madrid\ldots{} Todo por una mala novia\ldots»

En fin, que el hombre se fue. Hillo no se hallaba en casa cuando estas
patéticas escenas ocurrían. Y por cierto que andaba el tal curita hecho
un paseante en corte, vestidito de seglar, con bastón y sombrero de
copa, todo el santo día de mazo en calabazo, y no ciertamente en las
mejores compañías. Muchos, ignorantes de los móviles de su conducta, le
tenían por echado a perder; otros sospechaban que los jacobinos y
masones le habían seducido, atrayéndole a sus conciliábulos obscuros. Su
buen nombre eclesiástico no ganaba nada con esto; pero a él le importaba
ya una higa la opinión clerical, y todo lo que no fuera el honrado
objeto de sus trabajos y pesquisas.

Como Calpena no ocultaba su domicilio, calle de las Urosas, allá se iba
D. Pedro a diferentes horas, sin dar a sus visitas apariencias de
persecución o de fisgoneo policiaco. Siempre buscaba un pretexto,
comúnmente literario, y hasta llegó a fingir que escribía un
\emph{Florilegio de refranes}, y que necesitaba compulsar textos muertos
y vivos. Igualmente iba en busca de Miguel de los Santos; pero siempre
con mala suerte: no se podía hacer carrera de aquel chico, dotado de
excelsas cualidades, que desvirtuaba con su pereza. «Miguelito---le
decía Hillo, que al poco tiempo de amistad ya le tuteaba,---tú vales
mucho y no serás nunca nada.» Acontecía no pocas veces que iba a
buscarle a las nueve de la mañana y le encontraba en el primer sueño.
Algunos días tomaba el desayuno a las cinco de la tarde. Con semejante
vida, ¿qué había de hacer el hombre, ni de qué le valía su grande
ingenio? No concluyó jamás nada de lo que empezaba. De sus propias obras
se aburría, a fuerza de admirar las ajenas; amaba a sus amigos
entrañablemente; de sí mismo no hacía ningún caso.

Lo que a Hillo mayormente le incomodaba era no encontrar en él eficaz
ayuda para traer a Fernando al buen camino, y siempre que de esto le
hablaba, salía el bueno de Miguelito con unas filosofías que dejaban
helado al pobre D. Pedro. Quería este aplicar a todo los principios que
establecen el gobierno de los individuos por la familia, y de la familia
por el Estado, organizando una especie de colegio universal, y Álvarez
profesaba un donoso fatalismo con profundas raíces en su mente. Sacaba
de quicio al buen capellán el humorismo con que Miguel de los Santos
trataba las cosas más graves; aquella pachorra, aquel mirar tierno con
que afirmaba el imperio absoluto, soberano, de la fatalidad. Todo pasa
como debe pasar, y es inútil y ridículo pretender desviar personas y
cosas del camino que les imprime la escondida fuerza que todo lo
gobierna. De esto resulta que no debemos tomar a pechos ningún humano
incidente. Desgracia y ventura no son más que términos de relación,
convencionalismos. Así como no podemos influir en los fenómenos
meteorológicos, nos está vedado el oponernos al fenómeno histórico,
afecte a las naciones, afecte a los individuos\ldots{} Lo único que sacó
en limpio D. Pedro fue alguna que otra noticia íntima referente a los
amores de Calpena. La Zahón, que ya venía algo esquinada, sin que se
sepa por qué, vio con malos ojos la renuncia que hizo Fernando de su
destino: si primero le había tenido por príncipe con disfraz, luego le
tuvo por un ladino pelagatos, que husmeaba la dote de Aura; y deseando
poner punto en tales relaciones, empezó por limitar las entrevistas de
los novios y dificultar el carteo. De todo esto resultaba la espantosa
murria de Calpena en aquellos días. Su exaltada mente le sugería sin
duda proyectos audaces, caballerescos, traduciendo a la realidad el
peregrino enredo de los dramas románticos. «¿Querrá usted creer---dijo
Álvarez,---que a nuestro amigo se le ha ocurrido aplicar al caso de la
calle de Milaneses el procedimiento del narcótico? Sí\ldots{} dar a la
señorita un bebedizo para que se quede tiesa y fría, simulando la
muerte\ldots{} Vamos, como en \emph{Romeo y Julieta} y en \emph{Catalina
Howard}, y luego cargar con la difunta, que no es difunta más que de
mentirijillas, y\ldots{} ya supondrá usted lo demás. De las distintas
clases de raptos, pienso que no se le ha quedado ninguna por
estudiar\ldots{} y ya verá usted cómo sale por algún registro
inesperado, teatral, y a todos nos deja con la boca abierta.»

Y mientras Miguelito poníale ante los ojos estas probables contingencias
de trágicos lances, la invisible tutora le empujaba cada día con más
apremio hacia el remolino que la voluntad y la pasión de Calpena iban
formando. En una de las últimas comunicaciones de la \emph{velada}, le
decía entre otras cosas: «Por Dios, no olvide usted lo que tanto le he
recomendado: que le siga a esa zahúrda donde vive, que procure por
cualquier treta ingeniosa introducirse en ella. Cuide usted de que nadie
le falte, pues su abandono no es más que aparente. Sin que él pueda
sospecharlo, páguele usted su hospedaje, y encargue a los dueños de la
casa que finjan el mal humor de todo patrón que no cobra\ldots{} Y otra
cosa espero de su hidalga cooperación. Sé que se junta de noche con los
patriotas exaltados, que asiste a sus nefandas logias y a sus ritos
extravagantes. Sin duda, al verse solo y perdido, trata de reformar el
mundo, armándonos aquí otra revolución como la francesa, con su
convención, guillotina y todo\ldots{} Pues es preciso, mi querido amigo
y capellán, que usted se meta también en esas logias y cavernas
endemoniadas. ¿Qué le importa a usted, si su masonismo es fingido y
conserva en su conciencia el amor de la verdad y el desprecio de tales
majaderías? Métase usted en la boca del lobo, sin rebozo alguno ni temor
de que le crean jacobino. Nada debe usted recelar, pues aquí estoy yo
para sacarle de cualquier mal paso. Adelante, y no vacile en hacernos
esta grande y noble caridad. A nadie tiene usted que dar cuenta más que
a Dios y a mí, y Dios sabe la rectitud con que procede mi buen capellán,
penetrando en los antros donde se forjan las revoluciones y el ateísmo.
De allí saldrá usted como entre, y si consigue sacarme de ese y otros
peores infiernos a esa querida alma extraviada, tendrá usted dos
recompensas: la temporal y la eterna.»

«Bueno, señor, bueno,» murmuraba D. Pedro, cayendo en profundas
meditaciones. Y al día siguiente le decía la incógnita: «No sólo le
seguirá usted a todos los sitios a donde le lleve su reciente amistad
con los patriotas furibundos, sino que debe penetrar en casa de la
Zahón. Dos días llevo pensando en el medio mejor para realizar este
metimiento, y creo haber encontrado uno magnífico, superior. Verá usted:
la Zahón es socia, compinche o comadre de Maturana, el diamantista.
Maturana, corredor y traficante de alhajas y obras de arte en toda
Europa, gran perito, gran joyero, gran chalán, posee un abanico
magnífico, que ha pertenecido a la Pompadour, a la Emperatriz Josefina,
a Pepita Tudó, a la Reina Hortensia, a Mademoiselle Mars y a otras
personas que no han adquirido celebridad. Es pieza de gran valor
histórico y artístico, y con él pensó Maturana hacer un buen negocio,
ofreciendo su compra a la Reina Cristina. Pero Su Majestad, que ahora
está por lo positivo y prefiere emplear su dinero en salinas, en minas,
en empresas de utilidad, le ha ofrecido muy poco dinero, con lo cual ha
estado el hombre fuera de sí, tirándose de los pelos. Por fin, creo que
se entendió con la Zahón: han hecho un cambalache, dándole él su abanico
a cambio de una colección de perlas. Hállase, pues, hoy la hermosa obra
de arte en manos de la jorobada. Nada tiene de particular que el Sr.~de
Hillo, variándose el nombre y fingiendo el empaque de un señor
aficionado a lo antiguo, se presente en la joyería de la calle de
Milaneses, y pida que se le muestre el abanico para comprarlo. Usted lo
ve, lo examina por un lado y otro, mira bien el país, el varillaje, el
clavillo, diciendo algo que revele al conocedor de estas cosas; elogia
la perfección del trabajo de Lefebvre y el mérito de Lancret, pintor de
la cabritilla\ldots»

Traía después de esto la carta una prolija descripción del país, dando
noticia de todas las figuras, de sus trajes, etc., y concluía: «Para que
no se maraville mi Sr.~D. Pedro de que tan bien conozca yo el abanico,
le diré que lo he tenido en mi mano más de una vez, y lo he mirado y
remirado\ldots{} Vaya, lo diré todo: esa artística joya ha sido mía. La
poseí dos años, sin que nadie lo supiera\ldots{} Es decir, alguien lo
supo; pero no Maturana\ldots{} Una vez que usted la vea bien, pide
precio, y cualquiera que sea, se descuelga con la muletilla de que le
parece caro, y ofrece pensarlo. Después se hace mostrar perlas y
diamantes, lo ve todo, y se retira diciendo que volverá. Al día
siguiente vuelve, y manifiesta resueltamente y sin rodeos a la Zahón que
le compra el abanico al precio propuesto, siempre que ella se comprometa
a romper de una manera radical las relaciones de Fernando con la
chiquilla de Negretti. Esta manera radical no puede ser otra que sacar
de Madrid a esa loquinaria y llevársela a Córdoba o Cádiz, donde también
tienen casa de comercio; pero de tal modo y con tal sigilo efectuada la
salida, que no pueda Fernando saber a dónde se la llevan, ni, por tanto,
pensar en seguirla. ¿Qué le parece, mi bondadoso capellán, este
pensamiento mío? Si lo estima feliz, mañana, cuando salga la primera vez
de su casa, sobre las diez, póngase el sombrero bien terciado al lado
derecho, de modo que le caiga sobre la ceja\ldots{} Si lo encuentra mal,
colóquese el susodicho aparato tapa-cabezas en forma rectilínea, bien
aplomado, el ala todo lo horizontal que sea posible.»

Salió Hillo al siguiente día con el sombrero bien derecho. Conceptuaba
peligroso y contraproducente el recurso del abanico para avistarse con
la Zahón; discurría que siendo ésta mujer avariciosa, y además muy
ladina, si se le ofrecía dinero por el quebrantamiento de relaciones,
vería en esta oferta el reclamo de gentes poderosas. Era, pues, lógico
que, encendida su ambición, pensara en afianzar las relaciones de los
dos amantes antes que en destruirlas, o bien pediría más, mucho más que
el precio relativamente corto del histórico abanico. «Por esta vez---se
decía Hillo,---no ha sido usted, mi señora incógnita, tan lista y
perspicaz como de costumbre; y permítame que se lo exprese con el
pensamiento, ya que de otro modo no pueda expresárselo\ldots{} ¡En buena
nos metíamos si esa mercachifle astuta llegara a entender que es
Fernandito en el orden social persona muy distinta de lo que parece!
Déjeme usted a mí, señora invisible, que ya me arreglaré yo para llegar
al fin que todos deseamos.»

En efecto, tomadas de un platero de la Concepción Jerónima, amigo suyo,
dos lecciones de arte del diamantista, y aprendidos cuatro terminachos,
se fue a casa de la Zahón, y trató con ella, arrancándose a comprarle
unos aljófares y media docena de rosas, todo ello de poco valor. En su
segunda visita le habló del asunto con habilidad, enjaretando embustes
muy sutiles, para llevar al ánimo de la corcovada sentimientos
contrarios a los fines de Calpena. Harta ya Jacoba de un noviazgo que
ninguna ventaja le traía, acabó de abominar de él con las tremendas
cosas que D. Pedro le dijo, y se propuso tomar sin pérdida de momento
las medidas necesarias para mandar a paseo al joven romántico, y
quitarle de la cabeza a la niña su desatinada pasión. Todo lo temía ya.
Calpena, si le dejaban, consumaría el rapto de su Julieta con todo el
salero, con toda la audacia de que ofrecían ejemplos mil las obras
poéticas de aquel tiempo. Urgían, pues, resoluciones eficaces,
perentorias; despedir a D. Fernando y empaquetar a la chiquilla para
Córdoba.

Un poquitín alborotada quedó la conciencia del buen presbítero después
de su última conferencia con Jacoba, porque, en verdad, las atrocidades
que allí soltó traspasaban quizás la medida de la intriga inocente.
«¡Qué pensaría Fernando de mí---se dijo, andando taciturno hacia su
casa,---si supiera que le he presentado como un desalmado
hipócrita\ldots{} si supiera ¡ay!, que le supuse en connivencia con Luis
Candelas, y otros eminentísimos ladrones!\ldots{} Pero la buena voluntad
me absuelve de esta triquiñuela, y Dios, que ve los corazones, sabe que
en el mío no hay más que amor al bien, deseo de impedir el extravío de
un ilustre caballero, llamado a los grandes destinos\ldots{} Creo que no
sólo Dios, sino el mismo Fernando me absolverá cuando le haya pasado
esta calentura\ldots{} ¡Ah, y entonces los dos nos reiremos de los
disparates, de las abominaciones que dije!\ldots»

Y a la mañana siguiente le escribía la \emph{velada}: «Antes de
enterarme, por lo que me manifestó quien pudo observarlo, de la postura
recta de su sombrero, Sr. de Hillo, señal de su desconformidad con lo
que le propuse, ya había yo reconocido que anduve muy descaminada en
aquel plan de comprar con el precio del abanico la liberación de
Fernando. ¡Qué despropósito! ¡Cuánto me alegro de que usted opinara de
distinto modo, según declaró su \emph{góndola!\ldots{}} Es que con el
cavilar continuo, mi cabeza se pone a veces perdida, señor capellán, y
si \emph{dormitó el buen Homero}, como dicen ustedes los retóricos, ¿qué
extraño es que no sólo dormite yo, sino que sueñe disparates? Despejada
mi razón, he visto claro que si la diamantista huele dinero, estamos
perdidos. Usted seguramente habrá imaginado y puesto en ejecución otros
artificios por llegar al fin que anhelamos. Eso no quita que yo desee
adquirir el abanico, y lo adquiriré, Dios mediante, cuando salgamos de
este atolladero. No quiero que aquella preciosidad, que ya estuvo en mis
manos, vaya a parar a otras, ni aun a las de la misma Reina. En este
anhelo mío se manifiesta la mujer más de lo que yo quisiera, y quizás me
vea usted frívola, caprichosa\ldots{} Perdóneme, y cierro este
paréntesis para decirle que no desmaye, que veo cercano el peligro. Si
Fernando consigue apoderarse de Aura y desaparece, cualquiera les coge
después\ldots{} ¡Y si contrariados en sus amores, enloquecidos por la
pasión, resuelven suicidarse juntos\ldots! ¡Dios mío, qué horror! Crea
usted que esta idea me persigue desde anoche\ldots{} No duermo nada
pensando en los distintos procedimientos de matarse que inventa el
romanticismo, y que los malditos poetas han puesto de moda,
infundiéndolos a la juventud exaltada, con el continuo ejemplo de dramas
y novelas\ldots{} Estemos alerta\ldots{} y si hay vislumbres de suicidio
mutuo, entonces, ¡ah!, entonces no hay más remedio que transigir\ldots{}
Todo, todo, antes que ver morir a Fernando\ldots{} Eso no, eso
no\ldots{} repito que eso no\ldots{} Concluyo, mi señor capellán,
advirtiéndole que en la logia de la plazuela del Carmen andan ahora en
grandes peloteras. \emph{Los libres} se desatan, y en su delirio, en la
fiebre del motín y de la bullanga, ayudan a los estatuistas a derribar a
Mendizábal\ldots{} Los de la \emph{moderación}, que se traen ahora un
cierto tacto de codos con el absolutismo, se proponen no dar tiempo a
\emph{Don Juan y Medio} para la realización de su plan de reformas.
Tiran a impedir que decrete la supresión de monacales y la venta de sus
bienes, porque calculan que con los recursos de la enajenación se haría
fuerte el hombre, rodeándose de un baluarte de plata y oro\ldots{} ¡Y
esos badulaques, esos patriotas exaltados no ven que son instrumento de
los que abominan de la Libertad! ¡Siempre lo mismo!\ldots{} Con que ya
sabe: métase allá, y no vacile en ponerse al lado de los que alboroten
en pro de Mendizábal. No nos conviene que caiga tan pronto D. Juan: lo
necesitaremos más adelante, quizás muy pronto. Adiós, señor capellán; en
sus oraciones no deje de encomendarme a Dios.»

\hypertarget{xxix}{%
\chapter{XXIX}\label{xxix}}

Según atestiguan personas coetáneas de la Zahón, tanto se afectó esta
con las inquietudes y cavilaciones de aquellos días, que se le
disminuyeron las jorobas, y la exaltación de su espíritu fue parte a
mermar las graves pesadumbres de su cuerpo. Pero como otros autores
afirman lo contrario, manifestando que las corcovas, y con ellas el
dolor y tirantez de músculos, aumentaron horrorosamente, el narrador de
estos sucesos cree obrar con prudencia quedándose en el justo medio
entre tan opuestas aseveraciones, y así declara y establece que las
protuberancias, los sufrimientos físicos y morales y el avinagrado genio
de Jacoba Zahón, eran los mismos que en los días aquellos del convite
que abrió a Calpena las puertas de la casa.

Un día entero estuvo la diamantista rumiando una solución pronta y
eficaz: escribió a su hijo, residente en Córdoba, ordenándole que
viniese en su ayuda. Era urgente apartar de la familia al exaltado
joven, a quien recibió y agasajó suponiendo en él secretos enlaces con
damas poderosas y con ministros y personajes de gran viso. ¡Buen chasco
le había dado el tal Fernandito, que resultaba un triste y desamparado
poeta, uno de tantos pelagatos del romanticismo, sin más fortuna que su
melena y su enfática misantropía! Y lo mismo pensaba seguramente el
Sr.~de Mendizábal, que habiendole sin duda colocado por intrigas de las
logias, acababa de ponerle de patitas en la calle. Vivía el tal
miserablemente en un cuchitril de la calle de las Urosas, entre ratones,
poetas, comicastros, y quizás mujeres de mala estofa, y todo en él, su
traza y su fraseología, revelaba un presumido sin substancia, abandonado
de Dios y de los hombres. ¡Fuera, pues; fuera D. Fernando\ldots{} que no
era bien comprometer el grandioso porvenir de la niña, ni arrojar a
puercos las margaritas de la herencia de Negretti! Maturana, y otras
personas a quien consultó, opinaban del propio modo. ¡Fuera niños
románticos, que no traían consigo más que desvaríos, barullo, hambre!

Aunque hacía días que la Zahón se esmeraba en manifestar al joven, ya
con miradas desapacibles, ya con palabras ásperas, el desprecio que
hacia él sentía, no le pareció bastante decisiva esta forma de romper
amistades, y una tarde le espetó, con seca y rotunda frase, la orden de
poner fin al visiteo: «La familia meditaba otros planes con respecto a
Aurora; la familia tenía sobre sí la responsabilidad del porvenir de la
huérfana de Negretti; la familia no necesitaba explicar a nadie el
motivo de sus resoluciones; la familia\ldots»

---La familia de Aura soy yo---dijo Fernando con noble ademán y firme
convicción; y dicho esto se marchó altanero, no ciertamente como salen
los que no piensan volver.

Pero a Jacoba se le figuró, en su desconocimiento de las humanas
pasiones, que Fernando salía de su casa corrido, como si todas aquellas
razones de la familia (y vuelta con la familia) hubieran convencido al
romántico de la vanidad de sus pretensiones. Creyéndose, pues,
victoriosa, ya no le faltaba más que llamar a la tontuela y echarle la
rociada que preparado había para aterrarla y reducirla: «Aura, ven acá,
Aura: ¿en dónde te metes que no acudes cuando te llamo? Ves que estoy
baldadita, que no puedo moverme, y no vienes\ldots»

Por fin apareció en la puerta, como alma del otro mundo, vaga en la
forma, insensible el paso, la imagen de Aura, toda palidez en el rostro,
en los ojos toda fuego, el pelo sencillamente recogido más que peinado;
y antes que hablase la jorobada, le dijo con voz que parecía salir de
algún hueco misterioso bajo el suelo de la habitación: «Mi familia es
él\ldots»

---¿Has oído lo que le dije, niña?

---Mi familia es él\ldots{} yo no tengo más familia que él.

---Vete a tu cuarto, simple, y a la noche hablaremos, que ahora espero
visita y no me conviene incomodarme\ldots{} Si quieres tocar y cantar,
puedes hacerlo; pero cierra la puerta.

Desapareció Aura, y al poco rato llenaba toda la casa su voz tiernísima
cantando \emph{Assisa al pie d'un salice}. Entraron dos marchantes, y
allá se entretuvieron largo rato con Doña Jacoba examinando piedras,
dándose recíprocamente la jaqueca con el regateo de quilates y precios.
Fuéronse ya muy tarde, llevándose aljófar, media docena de esmeraldas de
las llamadas \emph{aguamarinas}, y aflojaron dinero: oro, plata.
Arrastrando su cuerpo, más bien que llevada por él, llegose la Zahón a
los armarios, guardó preciosos objetos, estuvo mediano rato dando
vueltas y más vueltas de llaves, y con la misma lentitud pudo ganar el
sillón, donde se apoltronó, hasta que Lopresti fue a anunciarle la cena.
En el comedor la aguardaba una sopita de sémola y un plato de pescado
frito. Viendo que Aura no acudía a la cena y que su cubierto continuaba
baldío, la señora dijo al maltés: «¿Y la niña?\ldots{} Ya: ¿no quiere
cenar \emph{su alteza}?\ldots{} Pues déjala, no la llames otra vez. Que
coma música\ldots{} Me importan poco sus rabietas\ldots{} Era ya loca, y
el maldito romanticismo me la ha trastornado más de lo que estaba.
¡Grande error ha sido! Pero se irá curando\ldots{} ¡Qué remedio tiene
más que someterse!\ldots{} Con ayuda del tiempo y de la ausencia, me
prometo ponerla como un guante. No me dé Dios más trabajo que
este\ldots»

A poco de cenar la llamó. Continuaba la joven en el mismo desgaire, mal
peinada, mal vestida, con un lindísimo \emph{deshabillé} que marcaba sus
incomparables líneas corporales, hermosísima, toda blancura en traje,
cara y manos, toda tinieblas en el pelo y en los ojos\ldots{} el andar
ligero, la mirada grave, pasiva, calmosa, fría como una espada cuando la
clavaba en la Zahón.

«Siéntate a mi lado, hija mía---le dijo la corcovada, arrimando la silla
más próxima,---y óyeme\ldots{} ¿Qué? ¿No me has oído?\ldots{} ¿Por qué
estás ahí parada, inmóvil\ldots? ¿Cómo quieres que hablemos con la mesa
de por medio? Acércate más\ldots{} Bueno hija, te empeñas en hacer la
fantasma y nada tengo que decirte. Tú te cansarás\ldots{} De verte así,
tan callada, me entra sueño\ldots{} y sueño me da también esa quietud
con que me miras\ldots{} En fin, si no quieres hablar, tendrás que
oírme, porque no dormiría yo tranquila esta noche si no te dijese que
ese falso duque y trovador de filfa no entra más en mi casa. Nos hemos
equivocado, hemos estado en Babia. Acabarás por convencerte de dos
cosas; digo, de tres; de tres, hija mía. La primera es que nada de lo
que yo disponga puede ser contrario a tu felicidad: con razón se ha
dicho `quien bien te quiere\ldots{} etcétera'. La segunda, que te
conviene, por tu salud corporal y del alma\ldots{} te conviene, repito,
tomar aires, salir de Madrid\ldots{} y para esto, niña, para llevarte y
cuidar de ti, viene mi hijo\ldots{} le espero mañana\ldots{} Y la
tercera cosa es que encontrarás, no a docenas, sino a miles, galanes de
más mérito y de más enjundia que ese tontaina de Fernandito, que no es
más que un pobre pájaro aburrido, tan vacío de mollera como de
bolsa\ldots{} ¿No respondes? ¿Te vas convenciendo?\ldots{} Parece que te
has vuelto tonta\ldots{} Aura, por Dios, da sueño mirarte\ldots»

Sin responder nada, Aura se fue con lento paso, y Jacoba permaneció un
instante con los ojos fijos en la puerta por donde se había ido. Puso
atención después, aplicando la oreja\ldots{} pero nada oyó: ni ruido de
pisadas, ni llanto, ni voz alguna.

«Cayetano---dijo después la señora, apartando de Aura su
atención,---tráeme eso, y acerca más la luz.»

Púsole delante Lopresti el tintero de cobre con polvorera y la negra
carpeta sebosa donde la señora escribía. De ella sacó la jorobada un
pliego de buen papel, escrito ya en dos y media de sus carillas, y
aproximado el quinqué y bien atizada la mecha, continuó su obra
interrumpida, trazando con lentitud y vacilante pulso los caracteres,
hasta que llegó al fin, y puso la firma y rúbrica. Leyó cuidadosamente
toda la carta, salpicando las comas donde le parecía, arreglando algún
trazo de letra torcido, o haciendo leves enmiendas que no afearan la
escritura, y bien regado el papel de polvos abundantes, se entretuvo en
doblarlo y cerrarlo con prolijo esmero, y extendió al fin despacio,
letra por letra, el sobrescrito: \emph{Excelentísimo Sr.~D. Juan Álvarez
de Mendizábal, Ministro}.

Muy satisfecha debió de quedar de su obra, porque sus ojos se animaban,
sus labios se movían, hablando para sí, silenciosos, y acariciaba la
carta entre sus finísimos y blancos dedos\ldots{} Pasado un rato de
meditación, intentó ponerse en movimiento. «Lopresti, ven, que no puedo
levantarme, ¡ay, ay, ay! Cógeme por la cintura\ldots{} con
cuidadito\ldots{} ¿Y esa?»

---En su cuarto\ldots{}

---Déjala\ldots{} Se pasará toda la noche lloriqueando, y mañana estará
más tranquila\ldots{} Que llueva, que llueva\ldots{} para que el alma se
descargue de nubarrones\ldots{} Vete a ver si duerme.

---Me parece que sí\ldots{} No siento nada---dijo el maltés, volviendo
de su inspección, que sólo duró un par de minutos.

---Pues vamos\ldots{} sostenme bien, que me caigo\ldots{} ¿Has cerrado
todo\ldots{} has apagado la lumbre?\ldots{} En seguida que yo me
acueste\ldots{} ya sabes, te traes aquí una manta, y te acuestas en el
sofá de paja, para que estés toda la noche al cuidado. Deja encendida la
luz\ldots{} Como tienes el sueño ligero, no se moverá un ratón en la
casa sin que tú lo sientas\ldots{} Clavadas como están las maderas de
todos los balcones, me parece que tenemos completa seguridad\ldots{} Yo
me caigo de sueño\ldots{}

Dejola el buen Cayetano en su alcoba, donde se acostó vestida, bien
cubierta de mantas. Una candelilla de aceite dentro de un vaso le daba
la claridad suficiente para no estar en tinieblas. Entre la lana
obscura, lucía el lívido rostro de María Antonieta guillotinada, y no
viéndose configuración de cuerpo, sino un informe bulto, podía creerse
que Doña Jacoba no era más que una cabeza colocada al azar sobre
montones de trapos.

Transcurrió más de una hora sin que Lopresti, tumbado en el sofá del
comedor, conforme a las órdenes de su señora, observase novedad en la
casa, ni oyese ruido alguno. Los de la calle, sonar de relojes
distantes, pasos de transeúntes, rumor de alguna pendencia, rodar de
carros, quedábanse fuera, y no había para qué poner atención en ellos. A
las once y media comenzó el roncar suave de la Zahón, que luego fue en
aumento, con notas aflautadas y acordes graves, que infundirían pavor a
quien no estuviese acostumbrado a oírlos. Lopresti se adormiló un rato,
al son de aquella tan conocida música; pero le despertó algo que no era
ruido\ldots{} un presentimiento no más, tal vez una idea.

Dudó un momento si le engañaban sus ojos, o si era, en efecto, la propia
persona de Aura aquella imagen que vela, avanzando cautelosa,
deslizándose ante la pared del comedor como proyección de linterna
mágica. La mesa interpuesta impedíale ver la mitad inferior de la
figura\ldots{} Traía una luz en la mano izquierda, y con la otra
apretaba contra el pecho un objeto que no se distinguía
fácilmente\ldots{} ¡Vaya si era Aura! ¿Pues quién podía ser más que
ella? «Esta madamita está loca o es sonámbula,» pensó el maltés. Pero
esta última presunción no se confirmó, porque la joven fijó en Lopresti
su ardiente mirada, y luego se fue hacia él indecisa, andando y
deteniéndose por segundos. A medida que se acercaba, iba perdiendo aquel
aspecto de \emph{Lady Macbeth} con que se apareció a los encandilados
ojos del fámulo. Dejó sobre la mesa la luz que traía, y miró espantada a
la puerta por donde los furibundos ronquidos de la Zahón llegaban al
comedor. Eran el propio ser de la diamantista manifiesto en el sonido.

Lo primero que hizo Lopresti al tener a la señora al alcance de sus
manos fue tratar de quitarle de la mano derecha un largo y afilado
cuchillo que con ella vigorosamente empuñaba: era el cuchillo de la
cocina. «Déjame, déjame, Cayetano\ldots---dijo Aura con voz ahogada,
defendiendo el arma con toda la fuerza que desplegar podía.---Esta noche
la mato, la mato\ldots{} Déjame.»

Al pronunciar el último \emph{déjame}, ya Lopresti le había quitado el
cuchillo. Aura se sentó, y poniendo los codos sobre la mesa y la cara
entre las palmas de las manos, rompió a llorar.

«Eso de matar es cosa mala, señora doña Aurorita; cosa mala casi
siempre, y, en todo caso, no es obra para mujeres.»

---Sí que la mato---reiteró Aura, pasando bruscamente de la sensibilidad
al insano furor homicida.---Dame el cuchillo, Cayetano; dámelo, y
verás\ldots{} ¿Para qué vive ese monstruo, ni qué falta hace en el
mundo? Es un bien que yo le quite la vida, que para nada sirve. ¿No
quiere ella matarme a mí? Pues véala yo muerta antes de morirme.

---No, no---dijo Lopresti escondiendo el cuchillo:---el matar es cosa
fea y sucia. Se manchará de sangre la señorita, y esas manchas de sangre
no se las quitará nunca, por más que se lave\ldots{}

Vuelta a la llorera y a la aflicción intensísima. «Mira tú, Cayetano:
cuando hice intención de matarla y fui por el cuchillo, estaba yo tan
decidida, que ya me parecía ver a Jacoba delante de mí,
expirando\ldots{} sin derramar sangre, porque no la tiene\ldots{} Yo la
mataba de un golpe, así\ldots{} y le decía: `Villana mujer, ¿por qué
quieres asesinar mi alma, matarnos a los dos de pena, de desesperación?
Pues muérete ahora rabiando, y vete a donde puedas desplegar toda tu
infamia, toda tu avaricia, toda tu maldad hipócrita: al
Infierno\ldots{}'.»

Al decir esto, Aura apretaba los dientes; sus ojos despedían llamas, y
accionaba fieramente con el puño cerrado. Los ronquidos de Jacoba eran
en aquel instante de una intensidad aterradora.

«Y al entrar aquí---prosiguió la Negretti,---pensaba yo que me sería muy
difícil rematarla\ldots{} ¿Quién hace pasar de la vida a la muerte todo
aquel cuerpo lleno de jorobas? Sería preciso un hacha, ¿verdad,
Cayetano\ldots? Porque nada adelantábamos con querer darle en el
corazón, porque no lo tiene\ldots{} Sólo conseguiría yo matarle una o
dos jorobas\ldots{} ¡y ella siempre viva!\ldots{} Es muy grande esa
mujer\ldots{} Hay en ella mujeres muchas, una dentro de otra, y todas
malas, muy malas, a cual peor\ldots{} Matas una, y siempre queda mujer,
o demonio, para martirizarme y volverme loca\ldots{} Sí, sí, tienes
razón: mejor es que no la mate\ldots{} ¿A qué, si ha de morirse
pronto?\ldots{} Le haremos un buen entierro, Cayetano, y le meteremos en
la caja todos sus diamantes, perlas y rubíes para que se vaya contenta.»

---Eso no, carambito\ldots{} Quédense las piedras acá\ldots{} En la otra
vida no sirven más que para hacer peso en el que las lleva y no dejar
que se salve\ldots{}

---Esta no se salva ni con peso ni sin él\ldots{} En el Infierno le
recamarán el cuerpo con carbones encendidos, y le darán a comer
esmeraldas fundidas, calentitas, y por cada ojo le meterán brillantes
tallados en pico\ldots{}

Con esto se iba tranquilizando la pobre Aura, y empezaba a sentir
calmado el horrendo desvarío, repercusión insana del amor en su caldeado
cerebro. Pasábase la mano por su frente ardorosa y por toda la cabeza,
sentándose el pelo, y con aquellos pases diríase que se suavizaba su
furia y se dispersaban las ideas de exterminio.

«¿Pero quién es esta mujer maldita---dijo en tono más humano,---para
querer tiranizarme a mí, para imponerme su voluntad? ¡Si yo no tengo por
qué obedecerla, si no es madre, ni tía siquiera, ni nada! Bueno que su
marido, si viviera, me mandase\ldots{} Pero esta, este galápago
codicioso, ¿por qué se mete a decidir de mi suerte? ¿Qué razón hay para
que no la decida yo misma?\ldots{} ¡Ah, qué desgraciada soy, y qué bien
haría Dios en quitarme la vida esta noche, a mí y a Fernando juntos,
pues ni morirme\ldots{} mira tú, ni morirme quiero sin él!\ldots»

Rompió en lágrimas amargas, y Lopresti, en el colmo de la compasión, no
acertaba a darle consuelo. «Sí, sí---dijo Aura bebiéndose su
llanto,---mañana moriremos los dos\ldots{} Lo hemos decidido y lo
haremos\ldots{} Cuando es imposible la vida juntos, el morir unidos es
un bien, un gozo\ldots{} Nuestras almas subirán abrazadas al cielo, y
abrazadas estarán por toda la eternidad\ldots{} Mañana, mañana mismo; ni
un día más\ldots»

---¡Morirse, matarse\ldots{} cosa fea!---exclamó el maltés con el más
agudo registro de su voz mujeril.---Mala es esta vida; pero\ldots{} ¿y
si la otra es peor? Nadie ha vuelto para decirlo\ldots{} Verdaderamente
que hay vidas aquí tan arrastradas, que le dan a uno ganas de
arrojárselas a la muerte\ldots{} Pero usted, señorita Aura, y el Sr.~D.
Fernando, no están de muerte\ldots{} todavía\ldots{} ¡Pues si yo fuera
él, si yo fuera usted, cualquier día me mataba! ¡Él tan guapo, usted tan
hermosa\ldots! ¡Ay, quién fuera ustedes!\ldots{}

Y pasando de la compasión de sí mismo a la suprema piedad por los dos
amantes, arrimó más su silla a la de Aurora, bajó la voz todo lo que
permitía el estruendo de los ronquidos del ama, y dijo: «A la niña le
pasan estas amarguras porque quiere. Cierto que Doña Jacoba no debe
imperar en usted. Manda porque la dejan. La autoridad no la tiene ella,
la tiene otro que está más arriba, mucho más arriba\ldots{} En fin, mi
Doña Aurorita saldría del despotismo de este coco si hiciera caso de
mí\ldots{} Usted no discurre, señorita; yo sí\ldots{} Usted no tiene más
que amor, amor y venga más amor, y yo calculo\ldots»

---¿Qué calculas tú?\ldots{} ¿Piensas lo que a mí pueda
interesarme?---preguntó Aurora tardando mucho en comprender la idea del
maltés.

---Ayer tarde, cuando usted se emperró a llorar, después de lo que la
señora le dijo, yo, desde aquel rincón, le hacía a usted señas para que
no se apurase\ldots{} y tuviera calma y hablara conmigo. Yo
calculaba\ldots{} Porque no ha de ser todo amor\ldots{} es preciso
cálculo, señorita, cálculo.

---Que me muera ahora mismo si te entiendo.

---Quise entrar en su cuarto con el aquel de llevarle una taza de tila;
pero la niña se había encerrado por dentro, y, naturalmente, no
entré\ldots{} Pues si me hubiera dejado entrar, le hubiera dicho lo que
yo calculaba, lo que voy a decirle ahora para que se sosiegue y tenga
esperanza de salvación\ldots{} ¡Qué!

¿Por qué me come con los ojos?\ldots{} Ahora se lo digo; pero prométame
antes hacer lo que yo aconseje\ldots{}

Diciendo esto, le acercaba el tintero y le ponía delante la carpeta en
que había escrito la Zahón: «Tonto, más que tonto. ¿Me mandas que le
escriba? Si ya lo hice esta tarde, diciéndole que sí, que nos
mataríamos, que preparase todo\ldots{} ¿No llevaste la carta?»

---Chitón\ldots{} aquí no se habla\ldots{} Ha prometido la señorita
hacer lo que yo mande. En guardia. Aquí tiene papel, pluma\ldots{}
Cójala y escriba lo que yo le diga.

---¿Pero a quién?\ldots{}

---Ponga\ldots{} clarito\ldots{} con buena letra: \emph{Señor D. Juan
Álvarez Mendizábal.}..

Absorta le miró Aura, posesionándose en un instante de las ideas que
bullían en el cerebro del maltés, y lanzó una exclamación de gozo, como
el que, perdido en tenebrosa noche, ve de súbito la luz que ha de
guiarle.

«¡Qué gran idea, Cayetano!\ldots{} ¡qué gran idea! ¿Lo has cavilado
tú?\ldots{} ¿Por qué no me lo habías dicho?»

---Si los enamorados, en vez de pensar en la muerte, calcularan\ldots{}
Pero ¿qué han de calcular, si están locos?\ldots{}

---Es verdad. ¡Qué gran idea! ¡Dios mío, qué alegría, qué
esperanza!\ldots{} ¿A quién he de pedir amparo más que al grande amigo
de mi padre\ldots{} al que\ldots?

---Doña Jacoba le ha escrito también esta noche.

---¿Qué me cuentas?

---No importa. Puede que el \emph{Excelentísimo} reciba la carta de
usted antes que la de ella. Eso es cosa mía. El \emph{coco} manda su
carta por Milagro. La de la señorita la mandaré yo por Méndez, mi amigo
Méndez, portero en Hacienda. Vamos, vamos, no perder tiempo.

---¿Y qué le digo?\ldots{} Cayetano, yo que acabo de estar loca, que
casi lo estoy todavía, no acierto a discurrir nada.

---Ponga\ldots{} \emph{Señor}, o \emph{Excelentísimo señor}: \emph{soy
la hija de Jenaro Negretti}\ldots{} Así, empezar con un golpe bueno:
\emph{soy la hija de Negretti, y\ldots{}}

---Y\ldots{}

---Y\ldots{} ahora vaya poniendo todito lo que le pasa.

Meditó la huérfana un rato, mordiendo las barbas de la pluma, y no tardó
en sentir la inundación de ideas en su cerebro, de que eran señal segura
la coloración de sus mejillas y el júbilo que flameaba en sus
hermosísimos ojos\ldots{}

«Ya, ya\ldots{} No necesitas dictarme, Cayetano. Ya calculo\ldots{} ya
sé lo que tengo que decir.»

Y escribió con más inspiración que soltura, sin quitar los ojos del
papel, haciendo con sus labios unos hociquitos muy monos.

\hypertarget{xxx}{%
\chapter{XXX}\label{xxx}}

No se abatía con los reveses el animoso espíritu de D. Juan Álvarez, ni
por un tropiezo parlamentario, o por la defección de media docena de
amigos a quienes tuvo por incondicionales, dejaba de creer que su buena
estrella triunfaría de todo, llevándole al cumplimiento de las promesas
hechas a la Nación. La confianza en sí mismo no le abandonaba nunca.
Formábanla el conocimiento de las energías que atesoraba su voluntad, y
los recuerdos de sus éxitos anteriores, todo ello amalgamado con un
poquito de soberbia. En su gigantesca estatura, que dominaba los
cuerpecillos de sus compañeros de Estatuto, como el alto ciprés a los
helechos humildes, veía un simbolismo de la supremacía de su voluntad.
Fe ciega tenía en su entendimiento, más fecundo en recursos sagaces, en
mañosos ardides que en concepciones hondas. Verdad que la política de
entonces, como la de ahora, no era terreno propio para lucir las
supremas dotes de la inteligencia: era un arte de triquiñuelas y de
marrullerías. En la oposición sí desplegaban los políticos una ideación
fastuosa, con carácter teórico, que deslumbraba a los papanatas del
partido y a la parte de opinión neutral que toma en serio las batallas
oratorias, comúnmente sin sacar nada en limpio de ellas; pero gobernando
no eran más que unos pobres caciques, unos manipuladores más o menos
hábiles del teclado de la cosa pública, en pro de intereses siempre
inferiores a los supremos de la Nación.

Cierto que Mendizábal tuvo alguna idea grande, y que su ambición, en vez
de limitarse, como la de otros, a prolongar todo lo posible las
maniobras caciquiles, picaba en los altos fines nacionales; pero no le
asistió la inteligencia en proporción de la magnitud de su deseo. Buena
es la fecundidad en arbitrios, buenos el ingenio y la travesura; pero el
perfecto hombre de Estado, rara avis, debe unir a tales dotes otras de
carácter sintético. La vista de Mendizábal solía percibir los remotos
ideales; pero no discernía bien el camino para llegar a ellos, no poseía
la completa y audaz visión del hombre de Estado, el cual necesita saber
mirar, sin cegarse, lo mismo al sol que al polvo.

Las trapatiestas parlamentarias de la ley electoral, que terminaron con
la derrota de D. Juan de Dios, y el compromiso de proponer a la Reina la
disolución de los Estamentos, quebrantaron los ánimos del primer
Ministro. Verdad que la batalla había sido ruda. La cuestión electoral
fue entregada sin detenido estudio a las iniciativas de una ponencia,
compuesta de cinco procuradores mal elegidos. Todo era desconcierto,
imprevisión, ignorancia de los métodos de gobernar. Salió, pues un
grande cien-pies, que veían con gozo los moderados. En el partido de
Mendizábal no faltaba gente práctica; pero no supo o no quiso prestarle
ayuda, ilustrándole en el procedimiento parlamentario para sacar
adelante las leyes, y el hombre pasó las de Caín en una mortal semana de
estériles y rencorosos debates. Sobre si la elección debía ser directa o
indirecta, por provincias o por distritos, sobre si se daría o no voto a
las capacidades, estuvieron aquellos hombres, como locos, agotando toda
la retórica insubstancial que viene siendo la función abusiva de los
cerebros políticos, y ha concluido por esterilizarlos.

No tuvo más remedio el Jefe del Gabinete, al término de esta desdichada
campaña, que disolver los Estamentos. La Reina no le puso obstáculos, y
Próceres y Procuradores fueron mandados a sus casas. En la brega perdió
\emph{D. Juan y Medio} la amistad de sus dos más ardientes defensores,
Istúriz y Alcalá Galiano, en quienes ya, desde Diciembre, se columbraban
las ganitas de formar rancho aparte; juego escénico que ha llegado a
constituir el resorte más rutinario y más amanerado de nuestra
fastidiosa comedia política. Aunque a Mendizábal le llegó al alma esta
defección, no por eso se acobardó, y aún soñaba con que el nuevo
Estamento le proporcionara medios eficaces de realizar sus grandes
propósitos. Pero si no desmayaba en sus alientos y ambiciones,
físicamente se sentía fatigado, pues la tarea de los últimos días de
Enero y de los comienzos de Febrero fue para rendir a un gigante. Bien
se le traslucía el cansancio en la palidez del rostro, y también en la
inclinación de su cuerpo, ya no tan espigado como cuando nos vino de
Inglaterra radiante de esperanzas. El buen señor propendía más a la
meditación; gustaba de la soledad, donde pudiese ahondar en los graves
problemas que la realización le ofrecía; mostraba menos confianza en las
personas circunstantes, y un poquito de asco de la adulación, de aquel
incienso continuo con que algunos se recomendaban a su benevolencia. En
tal situación moral y física le encontramos una noche en su despacho, a
hora muy alta de la noche, engolfado en diversos asuntos apremiantes,
queriendo resolverlos todos, y aplicando desordenadamente su atención a
este y al otro con voluble inquietud. Había comido en casa de Seoane,
retirándose después a su Ministerio con varios amigos, a quienes
despidió para poder trabajar. Deslizábase el tiempo entre la actividad
febril y súbitas caídas en la sima de la meditación. Escribía, soltaba
la pluma, revolvía papeles. Su pensamiento iba de un asunto a otro,
ondulante, vagabundo, como mariposa que no sabe en qué flor quedarse. A
lo mejor se posaba en una idea y en ella permanecía, perdiéndose en un
discurrir opaco, dulce imaginar que casi tocaba en la somnolencia.

«Este Córdoba\ldots{} este Córdoba\ldots---decía entre dientes
escribiendo al General en jefe del ejército del Norte.---¿Será cierto
que es la clave de la situación? ¿Será cierto que vivimos en el Gobierno
porque nos tolera, y que moriremos cuando se canse de vernos vivos?» Y
luego escribía, interrumpiéndose a menudo para pensar los conceptos,
cosa nueva en él, pues comúnmente enjaretaba un largo escrito, como el
buen nadador que aguanta mucho tiempo en las profundidades sin tomar
aliento. Antes de terminar la carta al General, la dejó para leer
párrafos de otras ya leídas, que quería recordar\ldots{} Y de pronto
contemplaba con vago mirar un montoncito de cartas que aún no habían
sido abiertas: las removía, se fijaba en los sobrescritos\ldots{}
Apareció de pronto un portero con dos más, y al poco rato volvió con
otra carta que dejó sobre la mesa, sin que el señor Ministro se dignara
mirarla.

Cerrando por fin los pliegos para Córdoba, cayó la mente de D. Juan en
un sombrío bache de ideas que le tuvieron suspenso, fija la vista en los
diferentes papeles que en la mesa había, sin ver nada. He aquí lo que
pensaba: «Olózaga acaba de decírmelo, y no me decido a creerlo\ldots{}
En Palacio están hartos de mí\ldots{} estoy caído ya\ldots{} Gobierno
aún porque no han encontrado el modo, decoroso para ellos, de ponerme en
la calle\ldots{} Esto no puede ser. Olózaga es muy mal pensado, y tiene
en la masa de la sangre el odio a los Borbones\ldots{} La Reina me ha
recibido hoy con visibles muestras de aprecio\ldots{} ¿Pero quién se
fía\ldots? Será o no será sincera\ldots{} ¡Dichosos reyes!\ldots{} y
nosotros medio locos aquí por defenderles, por sostenerles en el trono;
nosotros muriendo para que ellos vivan\ldots{} No, no es verdad que esté
acordada mi caída, ni mi sustitución por Córdoba o Martínez de la Rosa.
Creo en la lealtad de Córdoba\ldots{} que en su última carta,
concretándose a cosas militares, nada me dice de política\ldots{} En
Martínez lo creo\ldots{} de Toreno todo lo temo; los fabricantes del
Estatuto se mueren de tristeza lejos del poder\ldots{} Los señoritos
esos de la \emph{suprema inteligencia} no acaban de persuadirse de que
el país no existe exclusivamente para ellos\ldots{} El país,
\emph{señores del Anillo}, no es un fraque hecho a vuestra
medida\ldots{} el país\ldots» Estimulado al trabajo por un aguijonazo de
su voluntad, pasó la vista por otra carta, y quiso contestarla; pero no
tardó en distraerse de nuevo, pensando: «Debe de estar en lo cierto
Olózaga\ldots{} Como que me lo ha dicho también Seoane\ldots{} El Sr.~D.
Fernando Muñoz, a quien Romero Alpuente llama con mucha gracia
\emph{Fernando Octavo}, no se recata para hablar pestes de mí: me llama
\emph{déspota}, y a Castroterreño le dijo que yo soy un
\emph{Calígula}\ldots{} ¡Calígula!\ldots{} Este buen señor sabe menos de
historia que yo. ¡Llamarme Calígula porque me apoyo en la voluntad del
pueblo, porque me inflama el amor del pueblo, porque con y para el
pueblo me propongo llevar hasta el fin mis planes\ldots! Aguárdese usted
un poco, Sr.~Muñoz, buen caballero y amigo mío. Gusta usted, según
dicen, de acercarse a los corrillos de las tertulias aristocráticas y
palatinas, y aplicar el oído y enterarse de lo que charlan, para dar
traslado \emph{al Ama}, como usted dice\ldots{} Pues lléguese usted aquí
y óigame esto que el \emph{Ama} debe saber\ldots{} Juan Álvarez
Mendizábal ha caído en desgracia porque no quiere la cooperación
francesa para terminar la guerra, porque no accede ni accederá a que
\emph{Palacio} nos traiga acá otro Duque de Angulema, que es lo que allí
pretenden\ldots» Rápidamente giraba de un punto a otro su
pensamiento\ldots{} La memoria le punzaba, haciendo dar a su atención un
salto atrás. «Se me olvidó decir a Córdoba que no deje de poner diez mil
bayonetas en el Baztán\ldots{} explicarle los motivos por que prefiero
la intervención inglesa a la francesa\ldots» Y no tardó en enlazar esta
idea con otra: «Williers me apoya, Williers no me falta. Bien claro me
lo dijo anoche, añadiendo que no recele de Córdoba. Él y Córdoba son uña
y carne. Se escriben todos los días\ldots{} Pero me decía en París mi
amigo Maury, el poeta, que no me fíe nunca de los diplomáticos. Esta
noche, charlando en casa de Seoane, dijo aquel joven, secretario que fue
de Ofalia, no recuerdo su nombre\ldots{} dijo que Williers juega con dos
cartas\ldots{} Yo no hice caso\ldots{} Confío en Williers. Su apoyo es
sincero. ¡Que no tenga uno, en esta posición, un lente milagroso para
ver las almas, para ver el pensamiento de los que nos hablan!»

Y divagando siempre, encontrose frente \emph{al Ama}, y le dijo: «Señora
Ama, para que Vuestra Majestad se ahorre el pretexto de que no hago
nada, voy a demostrar ahora que no quiero que la posteridad ignore quién
ha sido Mendizábal\ldots{} Todo lo paso, menos que los niños de las
escuelas, dentro de cincuenta años, pregunten: «¿Quién fue ese
Mendizábal?\ldots» Buscó en la mesa un papel que le habían traído poco
antes para que lo examinara, por si deseaba corregir algo en él, y no
hallándolo tan fácilmente como creía, se impacientó. «\ldots Es mucho
cuento\ldots{} ¡Si lo tuve en mi mano hace dos minutos\ldots! ¡Ah, no me
negará la señora Reina que está influida por el Embajador de
Francia\ldots! Menudean las cartas del hijo de \emph{Igualdad}\ldots{}
¡Francia, Francia! De allí ha venido siempre la perdición de nuestros
Reyes borbónicos\ldots{} ¡Francia\ldots! ¿Pero dónde lo he puesto,
Señor\ldots?, y de los de acá, Martínez es el inspirador de Vuestra
Majestad. Reconozco realmente que Martínez es un hombre honrado\ldots{}
pero\ldots{} padre del Estatuto, le molesta que mi personalidad anule su
personalidad\ldots{} Yo no he fabricado Estatutos, pero sé hacer
países\ldots{} yo no soy poeta; pero soy hacendista, y en este momento
voy a cantar una oda, que no le cabe en la cabeza al
Sr.~Martínez\ldots{} porque yo, Sr.~Martínez, no sabré latín, pero
sé\ldots{} ¡Ah!, aquí está\ldots{} ¿Pero dónde te habías metido, papel?
¿Quién te puso en este montoncito de las cartas de mujeres?\ldots»

Fijó su atención en el largo escrito, y leyó cuidadosamente, recreándose
en cada párrafo, en cada palabra, en cada letra. El preámbulo era frío,
despiadado, cruel. El artículo 1.º, semejante a una inmensa hoz, decía
con aterrador laconismo: «Quedan suprimidos todos los Monasterios,
Conventos, Colegios, Congregaciones y demás casas de Comunidad o de
instituto religioso de varones, inclusas las de clérigos regulares y las
de las cuatro ordenes militares existentes en la Península, islas
adyacentes y posesiones de España en África\ldots»

Continuando la detenida lectura, algo hubo de encontrar en el artículo
5.º que no le gustaba. Trazó la enmienda entre líneas, y después de
borrar y escribir de nuevo al margen, tiró de la campanilla. A poco de
penetrar el portero y de recibir una breve orden del Ministro,
presentose un señor de mezquina estatura, con anteojos de oro sobre el
huesudo caballete de su nariz de trompa; traía en la mano un papel
semejante al que D. Juan de Dios acababa de leer.

«Mire usted, Sánchez---le dijo el Ministro dándole el decreto,---hay que
modificar la disposición referente a los conventos de monjas que deben
quedar. No están claras las atribuciones de las Juntas que han de
determinar el número de religiosas\ldots{} Prevengamos las malas
interpretaciones, los abusos. Vea usted cómo he redactado el párrafo
segundo del artículo 5.º\ldots{} Ponerlo todo en limpio y que lo vea
Argüelles\ldots{} Ese otro decreto (el que Sánchez le traía recién
copiado), no necesita más enmienda. Perfectamente claro y preciso\ldots»
Recreose también en su texto, fríamente ejecutivo, revolucionario. Como
quien no rompe un plato, el artículo 1.º decía: «Quedan declarados en
venta, desde ahora, todos los bienes raíces de cualquier clase que
hubiesen pertenecido a las Comunidades y Corporaciones religiosas
extinguidas, y los demás que hayan sido adjudicados a la nación por
cualquier título o motivo, y también los que en adelante lo fueren,
desde el acto de su adjudicación.»

«¿No tenemos ya nada que corregir aquí?»---preguntó el de la aventajada
nariz.

---Absolutamente nada.

---¿De modo que\ldots?

---A la \emph{Gaceta} con él\ldots{}

---¡A la \emph{Gaceta}!---replicó el funcionario, recogiendo de manos de
su jefe el terrible documento.

---Daremos el otro dentro de unos días\ldots{} Me lo trae usted mañana,
puesto en limpio\ldots{} Y ahora\ldots{} Media noche ya\ldots{} pueden
ustedes retirarse\ldots{} Yo me quedaré un rato más examinando esta
correspondencia\ldots{} Que se aguarde Milagro.

Volvió a quedarse solo; y tan grande excitación sentía, que tuvo que
espaciar sus ideas y sacudir sus nervios, paseándose de largo a largo en
la vasta pieza. «¡Para que digan que no hago nada!\ldots{} ¡Qué
revolución, qué colosal sacudimiento!\ldots{} Entrego a la clase
media\ldots{} \emph{cuatro mil millones}\ldots{} ¿qué digo?, más, mucho
más.» Volvió a la mesa, y rápidamente trazó algunos números\ldots{}
«\emph{Seis, siete mil millones}, y aún me quedo corto\ldots» Mirando al
espacio, quedose como en un embeleso dulce o embriaguez
financiera\ldots{} Su mente se lanzaba a las presunciones del porvenir,
nadando en un océano tan revuelto como profundo, con olas de cifras cada
vez más hinchadas\ldots{}

\hypertarget{xxxi}{%
\chapter{XXXI}\label{xxxi}}

Otra vez en su mesa el Sr.~D. Juan, incansable, desvelado\ldots{}
Adquirida la costumbre de trasnochar, no le apuntaba el sueño hasta la
madrugada. En las altas horas de la noche sentía sus facultades más
claras, su ingenio más agudo, y extraordinariamente aumentada su
fecundidad de recursos expeditivos, de mañosas tretas, para escamotear
las dificultades antes que para vencerlas.

«Que venga Milagro;» y al punto se presentó el buen D. José con varias
cartas a la firma. Firmó Mendizábal, y entregó cuatro más que requerían
contestación. Eran todas referentes a negocios electorales. Este pedía
la procuración para sí; aquel para su pariente o amigo. Quién solicitaba
humildemente; quién reclamaba con soberbia mal envuelta en cortesía,
alegando servicios a la Libertad y una larga historia bullanguera. A
unos se les contestaba con el \emph{perdone, hermano}; a otros se
ofrecían esperanzas bien rebozaditas, y ciertos y determinados nombres
sacaban tajada, seguridades de éxito.

«Oiga usted, Milagro---dijo Su Excelencia cuando ya el funcionario se
retiraba,---hágame el favor de manifestar a su amiga de usted, a esa
cansada Zahón, que no puede ser y que no puede ser\ldots{} En una larga
carta muy difusa, que no he podido leer entera\ldots{} me pide un
desatino tal, que le contestaría con un puntapié si estuviera yo en otra
posición\ldots{} Pero diga usted, ¿es loca esa mujer?»

---Me parece que sí\ldots{} Abusa horrorosamente del \emph{curaçao}.

---Ya\ldots{} Pues le dice usted que no me maree más\ldots{} No le
contesto por escrito porque tendría que tratarla con dureza\ldots{} y
puede añadir que ya sé el paradero del tío de Aurorita, Ildefonso
Negretti, y que le escribiré un día de estos para que venga a hacerse
cargo de su sobrina. No quiero que esa pobre niña permanezca más tiempo
en poder de la Zahón\ldots{} ¿Y qué?\ldots{} No sé quién me ha dicho que
es hermosa.

---Hermosa es poco decir; es divina, señor\ldots{} pero tan romántica,
que no hay quien pueda con ella. Mejor estará con su tío que con Doña
Jacoba.

Otra vez solo, engolfado el pensamiento en el maremágnum político:
«Traeré un Estamento a mi gusto\ldots{} La ingratitud de Galiano, la
envidia de Istúriz no prevalecerán\ldots{} Yo no miro más que a la
libertad, que deseo afianzar; a la guerra, que quiero concluir a todo
trance; al país, a esta infeliz patria devorada por las malas pasiones,
por tantos odios\ldots{} pobre, sumida en la ignorancia\ldots{} ¡Triste
herencia la del tal D. Fernando VII! Si este señor hubiera sido de otra
condición, ¡qué bien estaríamos!\ldots{} Quizás podría yo ahora
desarrollar tranquilamente mi pensamiento, madurarlo bien\ldots{} Con
estas prisas, allá va todo como Dios quiere\ldots{} ¡Qué lástima, Señor,
qué lástima!\ldots{} Porque tiene razón Caballero. ¡Cuánto mejor, en
política y economía, repartir al pueblo esta masa de bienes en vez de
sacarlos al mercado! ¿La parte de deuda que se amortiza vale más o vale
menos que los intereses territoriales que podrían crearse con ese
reparto, hecho juiciosamente? ¿Es preferible el crédito circunstancial,
para encontrar quien preste, a las ventajas futuras de la buena
distribución del terreno?\ldots{} ¿Y qué decir de los abusos que en las
subastas pueden cometerse?\ldots{} Resultará que los caciques de los
pueblos, la clase bursátil, los que poseen ya una mediana fortuna,
adquirirán bienes considerables pagándolos a largos plazos con el mismo
producto de las tierras\ldots{} Y en tanto el pueblo agricultor y
laborioso no podrá adquirir propiedad\ldots{} ¡Si lo he pensado, Señor,
si lo he pensado!\ldots{} ¡Pero no le dan a uno tiempo para
nada!\ldots{} ¡Esta política, esta vida\ldots! No es posible, no es
posible. Que venga aquí el \emph{Sursum corda}, y se volverá para
arriba, para el Cielo, sin haber hecho nada. ¡Vivir al día, defenderse
hoy de las asechanzas de mañana, temblando siempre, sin hora
segura\ldots{} y tener que sufrir una descarga cerrada de
discursos\ldots! ¡Las dichosas polémicas, los malditos abogados\ldots! Y
menos mal si uno contara con tener bien cubiertas las espaldas\ldots{}
¡Pero si \emph{Palacio} le pone a usted en la calle el mejor día, como a
un criado\ldots! ¡Ah! Con esta inseguridad, con esta zozobra, ¿qué
planes, ni qué reformas, ni qué soluciones grandes son posibles? Esto es
un vértigo, dar quiebros al enemigo, agarrar el poder con las dos manos,
sujetarlo además con los dientes para que los de allá no nos lo
quiten\ldots{} No puede ser, no puede ser\ldots{} Pero Mendizábal no se
va sin realizar algo, ya que no toda la grande obra, y le dice al país:
te he quitado \emph{treinta y seis mil frailes y diez y siete mil
monjas}; te doy \emph{cuatro mil millones, seis mil}, para que empieces
a formar un conglomerado social fuerte y poderoso\ldots{} De mogollón lo
hago\ldots{} No me dan tiempo para más. Luego, Dios dirá\ldots»

Cambio repentino de ideas: «Se me olvidaba\ldots{} Tengo que decir a
Córdoba que irá la remesa de zapatos la semana que viene\ldots{} y dos
millones en metálico. Lo apuntaré en la pizarra, para que no se escape
de la memoria\ldots{} ¡Ya se ve\ldots{} con tal diversidad de
asuntos!\ldots{} ¡Pero este Córdoba!\ldots{} El eterno enigma: si la
Reina le llama para que forme Ministerio, como cuentan por ahí, tratará
de enjaretar una situación mixta, combinando las fuerzas moderadas con
las liberales\ldots{} En este caso, yo le ayudaría\ldots{} ¡Pero si no
puede ser; si es todo un puro embuste de los periódicos, y de esa
turbamulta de desocupados que hormiguean en este pueblo chismoso y
novelero! Córdoba me dice que no se cuente con él para nada que sea
política\ldots{} Y en su alocución al Ejército, bien claro lo
expresa\ldots{} Va uno haciéndose, insensiblemente, a no creer nada, a
considerar toda palabra de hombre\ldots{} o mujer, como un ruido del
viento, como el gotear de la lluvia\ldots{} Veremos grandes cosas. El
nuevo Estamento nos traerá batallas formidables. ¡Hablar, hablar y
siempre hablar! Señor, en aquel Parlamento inglés es otra cosa: discuten
y votan el mensaje en un día. Son mal mirados los oradores galanos que
van a lucirse, y los abogados indigestos y sofísticos\ldots{} Debo decir
también a Córdoba que corre una especie saladísima: los Grandes de
España le proponen para formar Gabinete\ldots{} ¿Quién meterá a los
Grandes en camisa de once varas?\ldots{} ¡Ah! También le contaré lo que
anda diciendo por ahí \emph{D. Fernando octavo}\ldots{} que la Corte se
trasladará a Burgos, para estar más cerca del Ejército\ldots{} ¡Qué
tontería!\ldots{} No creo que el \emph{Ama} participe del cerval miedo
de sus cortesanos.» (Nuevo trazado taquigráfico en la pizarra).

Puso la mano sobre un montoncillo de cartas, algunas de las cuales aún
no estaban abiertas. Diríase que una de ellas se pegó a sus dedos. La
cogió maquinalmente, y empezó a leer por el medio: «¡Bueno está!\ldots{}
(\emph{Soltando la carta con desdén}.) Las Navas se me incomoda. Otro
que se tuerce\ldots{} ¡Como si yo pudiese hacer Procuradores a todos los
amigos de mis amigos\ldots! Y aquí otra y otra carta pidiéndome
destinos, contadurías, administraciones, secretarías, intendencias,
y\ldots{} ¿Pero de dónde, señores y amigos, de dónde voy yo a sacar
tantas plazas?\ldots{} ¿Y este que se me atufa porque no le he dado
privilegio en el asunto de las campanas?\ldots{} No faltaba más.
Bastante tengo con los azogues, que me darán no poca guerra cuando se
abra el Estamento\ldots{} ¡Dichosas campanas, azogues malditos!\ldots{}
Pero estos señores no ven en el Estado más que una vaca muy gorda y muy
lechera, a cuyas ubres es ley que se agarren todos los ambiciosos, todos
los glotones, todos los hambrientos\ldots{} ¿A ver esta otra carta? Ya
conozco la letra\ldots{} ¡Pobre Duquesa de Berry! También esta se ha
echado marido morganático, y hoy es Condesa de Lucchesi Pella. Por andar
menos lista que otras, ha perdido la tutela del chiquillo\ldots{} el
Delfín\ldots{} A ver qué me cuenta. (\emph{Lee por el final}.) Lo de
siempre: sus hermanas no le hacen caso\ldots{} la vituperan por la
campaña desastrosa de la Vendée\ldots{} (\emph{Se ríe}.) Y no le
perdonarán, no, el famoso episodio de la chimenea\ldots{} (\emph{Leyendo
por el centro}.) Me da las gracias por haber admitido en el Ejército
español al hermano de su esposo, el oficial napolitano Lucchesi, que
recomendé a Córdoba\ldots{} ¿Y qué más? Vaya, vaya con las princesas
destronadas\ldots{} parece que les hizo la boca un fraile. Ahora pide
que admitamos a otro hermanito, subteniente\ldots{} ¿Por qué no les
coloca en las tropas carlistas? ¡Ah, es que allí las pagas son en papel,
en ilusiones!\ldots{} Verdad que las pagas de acá\ldots{} también andan
como Dios quiere.»

Puesta a un lado la carta, trazó con rápida mano nuevas apuntaciones en
la pizarrita, y luego extendió las demás epístolas sobre la mesa
formando abanico\ldots{} Entre los sobrescritos, de muy diversa
escritura, vio uno que no se le despintaba. Sonriendo se dijo: «Quien no
te conoce, que te lea,» y la sacó del semicírculo con ánimo de someterla
a cuarentena rigurosa. «Pues sí, debo leerla---pensó variando
inmediatamente de propósito, en la versatilidad de su espíritu
inquieto;---veamos qué cuenta.» Era una de tantas comunicaciones de los
secretos agentes que el Gobierno tenía en la frontera. Diariamente
llegaban dos o tres por diferentes conductos, y la que a la sazón leía
Su Excelencia era remitida por una tal \emph{Madame Aline}, de fantasía
tan novelesca y de tan extremado celo en el desempeño de su misión, que
cuando no había sucesos graves que referir, los sacaba de su cabeza; y
si escaseaban las maquinaciones, o no sabía la verdad de ellas, ponía en
el telar los productos más inspirados de su numen. Engañado varias veces
por los cuentos de esta poetisa del espionaje, Mendizábal le había
tomado ojeriza, y aguardaba coyuntura para suspenderla del cargo; si ya
no lo había hecho era por consideración a nuestro Embajador en París,
que aún creía en ella y se fiaba de sus embustes.

«Ya te veo. (\emph{leyendo}.) La historia de siempre\ldots{} Que los
carlistas han recibido proposiciones de la Reina\ldots{} Que han llegado
a Oñate dos clérigos emisarios de \emph{Palacio}\ldots{} los cuales se
entienden con otro clérigo de Madrid para poner en autos a Doña Cristina
de los deseos y opiniones de D. Carlos\ldots{} Que los agentes de
Aviraneta en Olorón han entrado también en negociaciones con los
facciosos, ofreciéndoles un levantamiento en Madrid. Que al propio
tiempo los realistas franceses se proponen armarla, si Thiers se
decidiera al fin por la intervención. Que la frontera está infestada de
frailes trashumantes y perdidizos, que huyen de las degollinas de
Zaragoza, y muchos de ellos, transfigurados de la noche a la mañana, se
afilian en el ejército de Gómez o de Villarreal\ldots{} Que Zaratiegui y
otros andan a la greña con los palaciegos y toda la \emph{ojalatería} de
Oñate, y que de tantos piques y desazones tiene la culpa el carácter
despótico y entrometido de la Princesa de Beira, que de continuo pasa y
repasa la frontera, acompañada de \emph{Monsieur} Saint-Silvain, o sola,
con dos pastores: las autoridades francesas no la molestan\ldots{} Que
D. Carlos se propone formar Corte y Ministerio de verdad, y que para
presidir el Gabinete faccioso ha venido de Londres D. Juan Bautista
Erro. Por el Ministerio de Gracia y Justicia andan a la greña el Obispo
de León y Don Wenceslao Sierra\ldots{} El confesor del Rey, D. Juan
Echevarría, gobierna interinamente el ramo de Guerra. En medio de este
grande aparato político, en la Corte apenas tienen qué comer. D. Carlos
y sus allegados van viviendo con castañas y leche\ldots{} Las borrajas
son el plato de cada día, y el cocinero de Palacio discurre los
diferentes modos de poner las alubias\ldots{} Por referencia de un ayuda
de cámara del Rey, que despidieron por haberle pegado una tremenda
bofetada al gentil-hombre de servicio, sabe la manifestante que D.
Carlos se casará en secreto con la Princesa de Beira\ldots{} Esta había
comprado en Olorón varios objetos de bisutería falsos para su dueño y
señor, y había vendido dos docenas de perlas magníficas, para adquirir
con el producto de ellas fusiles\ldots{} También gestionaba que le
vendieran dos obuses, ofreciendo unas arracadas que posee\ldots{} La
comunicante las ha visto, y no duda que Su Alteza encontrará quien por
ellas le facilite un par de cañones\ldots{} Que los realistas habían
logrado entenderse con Aviraneta, ofreciéndole la Superintendencia de
policía para cuando triunfara D. Carlos\ldots{} y que últimamente se le
habían enviado desde Francia papeles que comprometían al Sr.~Mendizábal,
y al Sr.~Caballero, y al señor Duque de Zaragoza, documentos que se
publicarían en \emph{El Jorobado} para armar gran escándalo\ldots{}

Aturdido ya, la cabeza mareada con este aluvión de noticias, que no eran
en su mayor parte más que repetición de anteriores informes, D. Juan
echó a un lado la carta sin acabar de leerla. Por natural encadenamiento
de ideas, la mención de \emph{El Jorobado}, papel violentísimo, le llevó
a pensar en \emph{El Mensajero}, que también había comenzado a atacarle,
y en \emph{El Eco del Comercio}, que ya cerdeaba\ldots{} «No es bueno
que la prensa abuse de la libertad---se dijo mal humorado.---A bien que
con \emph{El Liberal}, que fundaremos nosotros, zurraremos de firme a
los que se vengan con injurias y enredos\ldots{} ¡Lástima que no
encontremos muchachos despabilados de estos que salen ahora con la
fiebre del romanticismo!\ldots{} Me dice Palarea que casi todos los que
valen están ya colocados en papeles enemigos\ldots{} ¡Colocados!\ldots{}
me río yo de esto. Ya vendrán, ya vendrán al reclamo\ldots»

Apuntó algo en su pizarra, pertinente a prensa y al nuevo periódico, y
fijándose en otra carta, cuya letra menudita y elegante conocía, la leyó
al punto: «Pepe no escribe a usted porque está consagrado hoy en cuerpo
y alma a la limpieza de sus panoplias y a la colocación de las espadas
del siglo XVII, que ayer adquirió. A su gloriosa ferretería se han
añadido unas espuelas, que diz pertenecieron a Íñigo Arista; el almirez
que a Doña Blanca de Borbón le servía para llamar a sus servidores en la
torre de Sigüenza, y otras quincallas magníficas\ldots{} En nombre de
Pepe, y en el mío, le invito a usted a comer, mañana viernes. Por Dios,
no falte, mi buen Don Juan, que tenemos mucho que hablar, y he de
contarle cosas muy tristes, ¡ay!\ldots{} Si le sobran a usted campanas,
mande hacer rogativas porque recobre el juicio su consecuente
amiga---\emph{Pilar}.»

«¡Pobrecilla\ldots---pensó el grande hombre, soltando la carta,---sí que
es desgraciada!\ldots{} ¡Qué mundo, qué cosas!\ldots» Y con mental
propósito de aceptar el grato convite, pasó a otro asunto\ldots{} algo
de elecciones, de una probable conferencia con Williers. Mas no tardó en
distraerle otro sobrescrito que en la rueda de cartas lucía con gruesos
y algo torcidos caracteres. Dijérase que aquella desconocida escritura
le miraba y atraerle quería, pues los ojos de D. Juan se habían como
enganchado varias veces en sus letras. Habíalas visto ya y hecho
intención de abrir y leer\ldots{} Por fin, salpicado de curiosidad, se
apresuró a satisfacerla. La carta, después del nombre y la fórmula de
respeto, empezaba con esta frase: «Soy la hija de Jenaro Negretti\ldots»
Era bastante larga. Leídos los dos primeros párrafos, no encontró, sin
duda, el Ministro interés bastante intenso en la lectura, y su mente
fugaz corrió otra vez hacia la idea política. «¡Ah, me olvidaba\ldots{}
(\emph{Modulando entre dientes}.), de la ley de mayorazgos! ¡Qué cabeza
la mía! Prometió Argüelles traérmela hoy, y yo, tan torpe, que no se lo
recordé esta tarde\ldots{} (\emph{Rápida anotación en la pizarra}.)
Mañana me explicará D. Agustín su protección a la revista \emph{El
Mensajero}, que publica contra mí artículos que se atribuyen a
Galiano\ldots{} ¡Qué amigos, Señor!\ldots{} He de procurar atraer para
el nuevo periódico, a las primeras plumas\ldots{} Ese Espronceda, ese
Larra\ldots{} Todos ellos, según dicen, viven miserablemente. Pues demos
a Espronceda y a otros poetas destinos adecuados a su mérito: las
secretarías de las subdelegaciones, plazas en las Bibliotecas, si queda
alguna\ldots{} Dígase lo que se quiera, la prensa no vive sólo de
libertad\ldots» Cayó en profunda meditación, cogiéndose la barbilla con
las puntas de los dedos. Dio después un palmetazo sobre la mesa, y
formuló en su mente graves acusaciones contra sí mismo: «Hubiera yo
podido impedir los sangrientos sucesos de Barcelona, que me han
perjudicado enormemente\ldots{} ¿En qué estabas pensando, Juan, cuando
le diste al D. Eugenio Aviraneta la carta para el general Mina? Tenemos
cuartos de hora funestísimos, mortales\ldots{} En un instante se
compromete una posición; una idea mala y extraviada esteriliza miles de
ideas grandiosas, fecundas\ldots» Se pasó la mano por la frente. Su
cansancio era muy grande. Pensó en los pobres empleados que por la
índole de su cargo tenían que permanecer en las oficinas a horas tan
absurdas, mientras el Ministro no se retirase.

Campanillazo\ldots{} «Que venga el Sr.~Milagro. Mi capa, el coche\ldots»

Cayéndose de sueño, recibió Milagro las últimas órdenes de Su Excelencia
para el siguiente día. «Estas cartas me las contestará usted a primera
hora; las demás no son tan urgentes. Es muy tarde. Estarán ustedes
rendidos. Hasta mañana\ldots{} ¡Ah! Milagro, un momento: no me olvide lo
de la Zahón\ldots{} Que no puede ser\ldots{} que\ldots{} En fin, mejor
será ponerle una carta. Recuérdemelo usted mañana.»

Y por engarce de ideas, ya cuando el portero le estaba poniendo la capa,
volvió presuroso hacia la mesa por recoger algo que quería llevarse a su
casa. «Soy la hija de Jenaro Negretti\ldots» Este párrafo inicial de la
dolorida carta le andaba por el cerebro, disputando el sitio a
pensamientos de mayor bulto y gravedad. Fuese a su casa el grande
hombre, soñoliento ya, revolviendo todo el fárrago de aquella noche:
Córdoba\ldots{} Galiano\ldots{} Palacio\ldots{} Ley de
Mayorazgos\ldots{} campanas\ldots{} Aviraneta\ldots{} prensa\ldots{}
frailes\ldots{} chiquilla de Negretti\ldots{}

\hypertarget{xxxii}{%
\chapter{XXXII}\label{xxxii}}

La desconsoladora respuesta que dio el señor Ministro a la carta de la
codiciosa diamantista puso a esta en formidable, épica irritación. En
tres días no le sacaron del cuerpo más que palabras airadas y
monosílabos rencorosos; en sus manos escribió, con sus propias uñas,
cifra lastimosa del despecho que la dominaba, y los marchantes o
compradores que por allí asomaron salieron o desollados vivos o
llamándose a engaño, con pocas ganas de volver. En la comida decretó
parvedades de la escuela del licenciado Cabra; y tales fueron, que
Aurora y Lopresti se habrían quedado en los huesos si no tuvieran la
precaución de reservar en sus respectivos escondrijos pedazos de pan y
otras cosillas de comer. Sentía la maldita Zahón odio a toda criatura
humana, y a las que más próximas tenía, hacíalas responsables de la
bofetada que le diera el ministrillo gaditano, aquel que conoció con
manguitos y la pluma en la oreja, \emph{en la casa de los Méndez}, allá
por los años 97 y 98 del siglo pasado. Porque el hombre de las levitas,
el verdugo de frailes y monjas, el secuestrador de campanas, no se
contentaba con tomar a chacota la proposición de constituirse en
administradora de la huérfana de Negretti (con lo cual aliviaba al señor
Ministro de sus cuidados), sino que la relevaba ignominiosamente del
cargo honrosísimo de custodiar y dar alimento y educación a la niña,
confiriendo estas funciones a Ildefonso Negretti, hermano de Jenaro.

No obstante su fiereza y despecho, pasados tres días de crisis, juzgó
prudente disimular la grave herida de su amor propio, y astuta y
cautelosa reservó de la familia y de los amigos la dura respuesta de D.
Juan Álvarez. Ni se le pasaba por la imaginación oponer resistencia a
las disposiciones de este, pues su naturaleza medrosa, calculista, alma
de mercader en pedrería, repugnaba el giro dramático en los actos de la
vida y todo lo que fuese ruidoso y violento. Encerrose, pues, en una
resignación torva, como gato a quien le han cortado las uñas; esperó los
acontecimientos envolviéndose en sus corcovas con cierta dignidad,
quejándose del reuma con más fuertes alaridos, elevando el precio del
quilate en los brillantes de talla superior, y extremando los rigores
con que celaba a la doncella puesta a su cuidado.

Aumentó su tristeza en aquellos días la demora de su hijo Laureano
Zahón. Había salido éste de Córdoba hacia Sierra Morena; pero tales
historias en el camino le contaron de los bandidos que la infestaban,
que tomó ascos al paso de Despeñaperros y se volvió para su casa, con
idea de esperar a que saliese tropa para venir con ella. Tal
contrariedad no tuvo poca parte en la prudencia que desplegó la Zahón
después de su fracaso. Con Aura era toda sequedad y desabrimiento; no le
permitía apartarse de su lado y de su vista; no creyendo bien guardada
la casa con la fidelidad de Lopresti, se procuró dos cancerberos más:
una tal Verónica, asistenta para centinela de día, y para vigilante
nocturno, Severo Meca, dependiente de Maturana, hombre a prueba de
sobornos, incorruptible, probado en veinte años de manejo de alhajas.
Con tal guardia, y el examen y reparación que mandó hacer de todas las
llaves, cerrojos y cerraduras, se creía libre de un atropello.

Inopinadamente se presentó Hillo a comprar otra partida de aljófar, que
regateó, poniéndose muy pesado, para encubrir con el negocio su
espionaje, y haciéndose mostrar el abanico, pidió precio, que la Zahón
fijó en setecientos y cincuenta duros, ni un maravedí menos. No le fue
difícil al presbítero llevar la conversación comercial al terreno
doméstico, y se enteró de la situación, por referencia espontánea de la
despechada Doña Jacoba. «No sabe usted bien---decía, poniendo los ojos
en blanco,---cuánto me agrada la resolución del \emph{caballero ese de
las campanas}, que por lo visto tiene tiempo sobrado para atender a
todo. Él sabrá lo que hace. No estoy yo para cuidar niñas, y menos a
esta diablesca dislocada, sin respeto a nadie, ni a mí misma. Mentira me
parece que ha de venir su tío y ha de quitarme este cuidado, pues aunque
tengo costumbre de guardar cosas de precio y de asegurarlas contra
ladrones, no sé cómo se custodian estas joyas que andan y enredan, que
discurren todo lo malo; joyas que es forzoso clavar en los estuches para
que no se escapen de ellos\ldots{} También le digo a usted, Sr.~de
Timoneda (con este falso nombre había ocultado Hillo su personalidad),
que si deseo perderla de vista, no deseo menos conservarla, mientras
esté aquí, libre de todo detrimento. Quiero que su nuevo guardián la
reciba en situación de honestidad material, aunque mentalmente la haya
perdido. Cuando esté fuera de mi casa, que haga lo que quiera, que se
deshonre; pero aquí no\ldots{} Esto es un sagrario, Sr.~de Timoneda;
aquí viven y han vivido siempre el recato, la virtud. De esta casa, no
ha salido jamás una piedra falsa\ldots{} ¿Cómo había yo de consentir que
ahora saliera?»

Alabó mucho el disfrazado clérigo estos alardes, y se permitió aconsejar
a Jacoba que, lejos de estorbar, favoreciese el traspaso de aquella
joven al tío carnal, pues la tal niña le daría disgustos muy gordos si
no la echaban pronto de Madrid. Y añadió a esto tales observaciones y
noticias, que la jorobada, fácil al miedo, no necesitó más para verse
rodeada de catástrofes. Dos veces más, en diferentes días, volvió D.
Pedro, regateando el abanico y haciéndose mostrar unos topacios, que no
compró; y con esto finalizaron sus averiguaciones en la caverna de la
Zahón, pues ya había adquirido los datos y conocimientos más
importantes: Aura delirante de amor; extremadas las precauciones para
evitar que se vieran los amantes, y, por fin, próximo el arribo del tío
carnal para cargar con la romántica niña y llevársela a los quintos
infiernos. Cuando esto fuera un hecho positivo, sólo restaba impedir que
Calpena descubriese a dónde había ido a parar la cabra loca; y
establecida la radical separación, no era ya difícil traer al buen
camino al descarriado joven. A este le visitaba diariamente, guardándose
bien de contarle sus tratos y contubernios con la diamantista; lo que no
impidió que Calpena los supiera por aviso de Aura, atisbadora
infatigable de quién entraba y salía en la casa.

No pareciéndole aún bastante inquisitorial la incomunicación entre los
tórtolos, sometió Jacoba a escrupuloso registro al menguado Lopresti,
guardando bajo llave papeles, pluma y tinta: por su gusto habría borrado
de las costumbres humanas, como ocasionado a la desobediencia, el arte
de la escritura. No creyendo eficaces estos rigores, y desconfiada del
maltés, determinó asimismo la señora que no pusiera los pies en la calle
mientras tal situación durase, y los recados los hacía Meca, el bárbaro
y frío Meca, incapaz de aliviar una pena de amor, aunque le dieran un
brillante de talla superior por cada lágrima que evitase. Ya se sabrá la
causa de esta insensibilidad. El último mensaje que llevar pudo Lopresti
a los portales de Santa Cruz, donde Calpena aguardaba la cartita, fue
verbal y nada satisfactorio: «Sr.~D. Fernando---le dijo, afilando la voz
más que de costumbre por la fuerza de su congoja,---ni traigo carta, ni
la traeré más: válgame la Virgen. Estamos dejados de la mano de Dios. La
señora me ha registrado al salir, todo, señor, como si fuera yo una
mujer\ldots{} ¡Qué vergüenza me ha hecho pasar, ay! Y no es lo peor que
me meta las manos por entre la ropa, haciéndome cosquillas, sino que ya
no me deja salir de casa. ¡Preso yo también, sin comerlo ni
beberlo!\ldots{} preso de desconfianza, porque hago este favor a dos que
se quieren\ldots{} Es mi gusto, señor; es mi único gusto servir a los
amantes finos\ldots{} Salgo esta tarde porque voy por la medicina, aquí,
calle Imperial\ldots{} ¡Ay! Dios mío, que no se le volviera
solimán\ldots{} y ya me despido de la bendita calle, porque desde esta
noche hace los recados ese Meca, montador que fue de la familia,
montador de piedras finas, y hoy vive de la tasa y fiel
contraste\ldots{} Pues verá: la señorita, que, como enamorada, discurre
más que cien doctores, me encarga diga a usted que esta noche le
escribirá. Tiene papel y lápiz, que le he dado yo\ldots{} Para mandar a
su amador la carta ha inventado una graciosa treta\ldots{} Ahora tenemos
allí todas las noches a D. José del Milagro. Entra\ldots{} deja su
sombrero en la percha\ldots{} En el forro del sombrero pondremos el
papelito. ¿Qué le parece? Lo que no inventa el amor, ni Dios lo
inventa\ldots{} Pues lo que falta es que usted se haga el encontradizo
con Milagro, cuando este salga de casa; que le convide; que le
entretenga hasta sacarle el embuchado; que mañana le vuelva a convidar y
a entretenerle para que lleve la respuesta del mismo modo, y
arreglárselas como pueda para seguir trayendo y llevando papeles
ensombrerados cada lunes y cada martes\ldots{} Con que ya lo sabe.
Prevenido, señor\ldots{} ¡Ojo al casquete!\ldots{} Adiós, D. Fernandito
de mi alma; no puedo entretenerme más\ldots{} Si tardo, me mata.»

Véase aquí cómo fue conductor inocente de la amorosa correspondencia el
tubo grasiento y anticuado que cubría la venerable cabeza del buen
Milagro. No le fue difícil a Calpena echarle la zarpa, acechándole a la
salida de Milaneses, y le convidó a cenar (felizmente, por ser domingo,
no tenía que ir a la Secretaría de Hacienda), y hablaron cuanto les dio
la gana. Concluyó Fernando por fingirse delicado de salud, y suplicar a
su amigo que le hiciese diariamente compañía en los ratos libres, pues
de ello recibiría gran consuelo. Hubo de manifestar sentimientos
contrarios a los que llenaban su alma; hizo el papel de que le pesaba
haber abandonado su destino; mostrose arrepentido de sus amores, sobre
los que hacía recaer toda la culpa de tantos infortunios, y pedía
consejo a su buen amigo sobre la conducta más propia y eficaz para
volver a la gracia de Su Excelencia. Con gran júbilo le oyó Milagro, que
de veras le apreciaba, y prometió visitarle en el rato libre, entre la
contabilidad de la Zahón y el trabajo nocturno de la oficina.

Con tal ardid tuvo Calpena carta fresca todas las noches. No eran
palabras amorosas lo que Milagro llevaba y traía en su sombrero; era
fuego, llamas cogidas a puñados del mismo sol. Véase la muestra:

«\emph{De Fernando a Aura}.---Si hallamos libre el camino del cielo, al
cielo. Si no hay otro camino que el del abismo, al abismo\ldots{} Todo
antes que arrastrar esta oprobiosa cadena del presidio social; todo
antes que sufrir el ultrajante despotismo de los cabos de vara que, con
el nombre de autoridades, civil, doméstica y política, cobran el barato
en este patio inmundo. Huyamos de ellos. Busquemos el aire libre, lejos
del aliento infecto de los cabos de vara. Sobre todas las leyes,
prevalece el amor, ley suprema, porque él es la creación, el principio
de las cosas.»

«\emph{De Aura a Fernando}.---Cariño, ¿verdad que me sacarás pronto de
este encierro? Con esta esperanza vivo. Cuento las horas que me faltan
para el momento dichoso en que dejaré de ver el rostro patibulario de
Jacoba Zahón. ¿Cómo no odiarla, si me priva de verte? Si ella me
asesina, ¿cómo no desear que se la trague el infierno, como se tragó
Jonás a la ballena?\ldots{} digo, no: fue la ballena quien se tragó a
Jonás, y no pudo digerirlo. Tampoco el infierno digerirá a Jacoba, y
tendría que vomitarla con todas sus preciosas\ldots{} Es la una de la
noche: la bestia monstruosa duerme; yo velo. El amor siempre alerta.
¿Cuándo nos echamos a volar? Quiero ser pájaro y mirar desde lo alto de
una ramita a estos pobres caracoles, que nos quieren llevar a su
paso\ldots{} Una de estas noches mi desesperación me inspiró la idea de
matar a Jacoba\ldots{} Estuve loca un ratito\ldots{} ¿Verdad que me
librarás pronto? ¿Verdad que si no nos dejan vivir nos mataremos? Sin
ti, no quiero la vida ni la muerte. ¿Qué sería de mí solita dentro de la
sepultura?\ldots{} Voy a decirte una cosa que no sabes\ldots{} Te
adoro\ldots{} Tonto, no te rías\ldots{} Me estoy muriendo por
vivir\ldots»

«\emph{De él a ella}.---Duerme tranquila; yo velaré, velaré siempre. El
sueño no quiere amistades conmigo. Si tu cárcel fuera de diamantes y la
custodiaran todos los ejércitos del mundo, de ella te sacaría yo\ldots{}
Si Jacoba fuera la hidra de seis cabezas, yo se las cortaría
todas\ldots{} Nunca me tuve por héroe. Ahora lo seré, porque te amo. El
amor me hace indómito; el amor me hace invulnerable. Si fuese preciso ir
hasta el crimen, hasta el crimen iré\ldots{} Ser tú mía, ser yo tuyo, es
hablar con vaguedad: somos un solo ser\ldots{} ¿No sientes un solo ser
en nosotros? No estamos separados, sino divididos; cada mitad en
diferente esclavitud. Pronto estará todo el ser integrado en la
libertad. Pronto te fijaré el día y hora en que debe terminar esta doble
agonía. Será sin bullicio, sin aparato; será la suma sencillez\ldots{}
No puedo más. Bendiga Dios el divino fieltro en que irá esta carta.
Adiós.»

«\emph{De ella a él}.---Poquito me faltó para besar el fieltro sublime
cuando de él saqué la luz de mi vida. Pero no lo besé\ldots{} No hice
más que acariciarlo\ldots{} Pronto, sí, mi bien, que sea pronto. Estoy
alegre, porque tú me lo mandas. Jacoba despide de sus ojos un veneno
verde, como el rayo de las esmeraldas. Pero ya no le tengo miedo: confío
en mi caballero, a quien amo, a quien pertenezco por toda esta vida
fugaz y por la eterna\ldots»

En este tono se escribían siempre. Arrebatado el espíritu de Calpena a
las altas cimas de la idealidad, no conocía freno. Tan profunda era su
transformación, que hasta se olvidaba de cómo fue, y de lo que había
sentido y pensado bajo la férula del buen D. Narciso Vidaurre. Aquella
serenidad del alma, aquel justo medio en que blandamente se mecía su
voluntad, ¿dónde estaba? ¿Dónde la placidez clásica, el amor de las
reglas, el gusto de lo incoloro, del vivir cómodo y bien repartido en
casillas metódicas? Todo aquel mundo blancucho y opalino se había
resuelto en un orden de sentimientos y de ideas que le asemejaba al
famoso héroe de Dumas, Antony. Como este, se había erigido en
desheredado, y con los fueros de tal, en aborrecedor de toda la
sociedad; como este, no vivía más que para un amor frenético, dispuesto
a consumar, por la satisfacción de sus anhelos, las violencias y
tropelías más abominables.

\hypertarget{xxxiii}{%
\chapter{XXXIII}\label{xxxiii}}

¡Quién le había de decir a Fernando Calpena, cuando con un amigo vio
representar el \emph{Antony} en la \emph{Porte Saint-Martin}, que aquel
drama, que entonces le pareció afectado, mentiroso, uno de tantos
artificios con que los dramaturgos amañados satisfacen el
convencionalismo teatral, había de ajustarse, traducido al castellano, a
la realidad de su pensamiento! El drama de Dumas, y el de Calpena, drama
real, no se parecían en el asunto, aunque sí mucho en la enfática
desesperación del héroe, no bien motivada, y en el ardor de su lenguaje.
El odio a la sociedad no era en él más que una repercusión hueca del
criollo de Dumas. En política había extremado bruscamente sus opiniones,
simpatizando con los revolucionarios más ciegos y brutales. Para D.
Fernando no tenían derecho a la permanencia ni el Gobierno aquel, ni
otro semejante, ni el Trono mismo. La Familia Real, de cuyo seno había
nacido una espantosa guerra, que llevaba trazas de no concluir nunca,
tampoco debía continuar ligada a la suerte del país. Las disensiones
entre los hijos de Carlos IV habían convertido a España en una inmensa
jaula de locos furiosos. Por averiguar si debía reinar hembra o varón,
se vertían ríos de sangre\ldots{} Y no pareciéndoles bastante sangría a
nuestros prohombres, todavía andaban a trastazos por si repartían las
mercedes del presupuesto los negros o los blancos, los amarillos o los
rojos. El propio Mendizábal, a quien siempre vio Calpena descollando
sobre la turbamulta política, se había empequeñecido a sus ojos: ya no
era el grande hombre que debía salvar y refundir la nación. Malogrados
sus propósitos por falta de constancia o malicia para llevarlos a la
realidad, resultaba perfectamente sentencioso y oportuno aplicado a él,
como a todos los del oficio, el dicho de Hillo: \emph{No remata la
suerte}.

Por otra parte, si el conocimiento de las conexiones jurídicas de
Mendizábal con Aura le indujo a mirar al ilustre gaditano con simpatía,
cuando supo que a la carta de la joven había respondido verbalmente, por
mediación de Milagro, sin darle más consuelo de su esclavitud que la
promesa de mudarla de cárcel, sacándola de las cadenas de Zahón para
ponerla en las de Negretti, la simpatía hubo de trocarse en ojeriza y
mala voluntad. Hallándose obligado a mirar por la huérfana, debió D.
Juan atender en otra forma a su angustiosa solicitud. Ni de tutor ni de
caballero era esta fría respuesta: «Diga usted a esa señorita que estoy
atareadísimo y no puedo ocuparme de ella todo lo que quisiera. He
escrito a Ildefonso Negretti para que vengan a recogerla. Yo hablaré con
él y le recomendaré que la cuide mucho y procure perfeccionar su
educación.»

«Pues yo le aseguro a usted, Sr.~D. Juan Álvarez---decía Calpena
\emph{in mente}, paseándose solo por las calles,---que cuando venga el
tan cacareado tío carnal para hacerse cargo de mi Aura, no la
encontrará. Aura me pertenece, y todos los Negrettis del mundo,
auxiliados por todos los Álvarez gaditanos, que no saben rematar la
suerte, no me la quitarán. Ahora veremos quién puede más: si Vuecencia
con sus altanerías de Ministro y jefe de partido, o yo solito, inerme,
sin más fuerza que la que me da la ley de amor\ldots{} Ley es esta que
no entiende ningún político, ni Vuecencia tampoco\ldots{} Creerá que es
como la Ley de amortización de la Deuda, o la de Redención de censos,
imposiciones y cargas\ldots{} Y no necesito extremar las conjeturas,
señor D. \emph{Juan y Medio}, para ver segunda intención en su proyecto
de poner a la huérfana en manos de un Negretti, que seguramente será
sumiso ejecutor de los deseos de un amigo poderoso. ¿Tendremos aquí una
comedia en que le toque a Vuecencia el papel de tutor, de ese anciano
verde, siempre chasqueado? ¿Le seducen a Su Excelencia los viejos de
Moratín? Pues tampoco ha de valerle el hacer el D. Diego, aun cuando
tomara las precauciones para asegurar un desenlace contrario al de
\emph{El sí de las niñas}, porque aquí estoy yo para llevar las cosas a
su término natural. Y si para esto tuviera yo que pegarle a Vuecencia un
tiro, se lo pegaría, como a Negretti, si este me contrariara con
malevolencia\ldots{} Por mi Aura, voy yo a las grandes y nobles
virtudes, como a las más negras demostraciones de la maldad; por mi
Aura, escalo yo el cielo o me precipito en los abismos. Nada tiene valor
para mí; cuanto hay en el universo se cifra en ella. Póngame usted entre
Aura y mi voluntad todas las llamadas leyes morales y sociales, y salto
por encima de ellas; y si quieren que pase sin saltar, pasaré, y pisaré,
y si pongo el pie sobre alguien que reviente con mi peso, quéjese al
diablo, porque Dios no ha de oírle.»

Entró en casa de Hillo, con quien hablar quería. D. Pedro le esperaba:
encerráronse en el cuarto de este. «Tu puntualidad en acudir a la cita
me demuestra que el caso es urgente. Necesitas dinero: ayer no pude
dártelo; hoy te lo daré, pero no sin condiciones.»

Adivinando las terribles condiciones que su amigo, cruel usurero en
aquel caso, le impondría, Calpena sintió frío glacial en el corazón, y
en la boca todo el acíbar que suele ser producto natural de la carencia
de dinero. «Te daré lo que necesites---prosiguió Hillo con severidad
noble;---pero has de darme garantías, seguridades de que ha de ser
empleado dignamente. Esas órdenes tengo.»

---Pero usted---dijo Calpena con voz cavernosa,---entiende por empleo
digno lo que para mí es el fin más alto que se puede imaginar. No nos
entendemos.

---No nos entendemos\ldots{} Yo tengo órdenes que he de cumplir
estrictamente. Para lanzarte sin freno a la perdición, necesitas oro. Es
natural: sin dinero no se puede realizar el bien\ldots{} ni el mal. Para
el bien tendrás lo que quieras, Fernando: demuéstrame que quieres el
bien, abandona tus locos devaneos, y partiendo los dos de Madrid esta
misma noche\ldots{}

Calpena se levantó del asiento sin decir más que: «Guarde usted su
dinero\ldots{} Me voy.»

---Oye\ldots{} no seas tan vivo de genio. No hago más que cumplir las
órdenes que recibo\ldots{} Muy dañado estás, hijo mío, cuando así me
vuelves la espalda; a mí, que te quiero como a un hermano\ldots{} No, no
eres digno de esta hermosa fraternidad, ni tampoco, lo digo muy alto, ni
tampoco eres digno de la piedad suprema, del cariño lejano, escondido,
para que sea más bello, de la persona que\ldots{}

Ahogado por la emoción, Hillo no pudo continuar, y se llevó ambas manos
a los ojos\ldots{}

«Para que yo venere a esa persona como ella se merece sin duda---dijo
Calpena en grave desconcierto,---es preciso que\ldots{} se necesita
que\ldots{} Yo la adoraré si la conozco, lo primero\ldots{} Encubierta,
y oponiéndose a la felicidad de mi vida, no puedo, no puedo quererla.»

Hillo le cogió de una mano, no secas aún sus lágrimas, y en grave tono
le dijo: «Te doy mi palabra de que si haces lo que dije\ldots{}
Renunciar radicalmente a ese devaneo, impropio de tu condición, y partir
conmigo de Madrid esta misma noche sin ver a nadie\ldots{} la deidad
invisible dejará de serlo\ldots{} así lo declara y promete en su última
carta\ldots{} Se nos revelará\ldots{} pero es condición previa que
tú\ldots{} ya sabes\ldots»

El rostro de Calpena se volvió de mármol; sus manos quedáronse heladas;
sus miradas perdieron toda luz. Miró al clérigo con estupidez; hízole
repetir la proposición. Repetida por Hillo, este añadió hasta tres
veces: «¿Te conviene el trato?»

De súbito fue acometido Fernando de un frenesí nervioso; cayó en un
sillón, mordiose los puños, contrajo todo su cuerpo, y clavando las uñas
en el brazo del sillón, prorrumpió en gritos dolorosos: «No
quiero\ldots{} no quiero\ldots{} Me ofrecen un nombre a cambio de la
vida. No, no\ldots{} No me hacen falta parientes; no necesito
familia\ldots{} Que se vayan, que me dejen. Solo viví, solo
estoy\ldots{} solo moriré\ldots{} moriremos\ldots{} ¡No quiero, no
quiero!»

Cogida en las convulsas manos la cabeza, como si quisiera arrancársela,
no dijo una palabra más. D. Pedro no le veía el rostro.

«Serénate---le dijo, tocando suavemente sus cabellos, cuyos rizos
desordenados por entre los dedos salían.---Te doy tiempo para pensarlo.
La cosa es grave\ldots{} no te precipites a resolver, así\ldots{}
airadamente.»

---¡Si está resuelto---dijo el desesperado joven, incorporándose,---si
no puede ser!\ldots{} ¡Si es como si me mataran!\ldots{} Y francamente,
no me dejo matar\ldots{} no me conviene morir todavía.

Y puesto en pie, cogió el sombrero con gallardo ademán, mostrando en
acto tan sencillo la firmeza de su resolución. Las últimas palabras de
aquella breve conferencia fueron: «Me equivoqué al pensar que usted
podía darme\ldots{} eso. Error grave fue pedirlo. ¡Qué bochorno!\ldots{}
¡pedir lo que no es nuestro, lo que me darían, no por favorecerme, sino
por comprarme! Dígale usted a quien sea, que no me vendo. El alma no se
vende. ¿Por qué no la adquirió, en tiempos en que fácilmente pudo
hacerlo? ¡Y ahora quiere quitármela, comprármela\ldots! Aunque yo
quisiera venderme, amigo Hillo, no podría\ldots{} no me
pertenezco\ldots{} Y para concluir, guárdese usted su dinero, o
devuélvalo a quien se lo ha dado. Para mí no ha de ser. Lo que yo
necesito con urgencia, lo buscaré como pueda.»

---Aguárdate\ldots{} hablemos otro poco.

---Usted puede perder el tiempo, yo no\ldots{} Es inútil\ldots{} Si
cierra la puerta me descolgaré por el balcón\ldots{} Quédese con
Dios\ldots{} No intente seguirme\ldots{} corro yo más que usted. Adiós.

Y con la presteza que estas palabras indicaban salió de la casa, dejando
a Hillo confuso y atribulado. Hubo de pasar un mediano rato antes que el
buen clérigo pudiera sacar del desorden de su mente una idea clara y ver
el derrotero más conveniente. «No me queda duda, va a la
desesperación\ldots{} Loco de amor y sin dinero, algo hará que nos dé
mucho que sentir\ldots{} ¿Iré tras él? ¿Pero quién le caza? No, no,
Pedro Hillo\ldots{} no te metas en cacerías peligrosas. Yo cumplo dando
la voz de alarma, como me ordenan. Ha llegado el momento crítico, el
momento del peligro supremo, que obliga a emplear el recurso final, lo
que los médicos llaman el remedio heroico. Me han mandado que avise
cuando estalle la crisis de locura, y aviso\ldots{} Pedro Hillo cumple
siempre con su deber; es hombre que sabe rematar la suerte.»

Escribió una breve carta, y al punto salió para entregarla al
\emph{Sr.~Edipo}, que en determinada calle estaba de servicio. Hecho
esto, se fue al club de la casa de \emph{Tepa}, donde había quedado
pendiente de la noche anterior una furiosa disputa, cuyo desenlace
quería conocer. Allá fue a parar también Calpena, sin más objeto que
matar el tiempo hasta media noche, y ver a un amigo que le había
ofrecido facilitarle algún dinero. Ya se comprende que este amigo no era
poeta.

Por obra y gracia de la armonía resultante entre la exaltación de su
espíritu y la atmósfera jacobina que en \emph{Tepa} reinaba aquella
noche, Calpena se lanzó, sin proponérselo, a la oratoria furibunda,
notas estridentes de rabia política con juicios abominables de cosas y
personas. Sus palabras eran materia inflamable arrojadas varonilmente en
aquel rescoldo de pasiones. De una parte le aplaudían con rabia; de otra
le vituperaban. Entre D. Pedro Hillo y otro señor tuvieron que cogerle
por un brazo y bajarle casi a rastras de la tribuna. Parecía loco
furioso, y su rostro echaba llamas. Después, entre el tumulto que en
torno del joven se formó, Hillo le perdió de vista. Cuatro amigos le
sacaron a la calle para que con el fresco de la noche se le despejara la
cabeza. Fueron a un café, pasearon hasta las doce, hora en que Fernando
se encaminó a su casa con el amigo que le había facilitado la cuarta
parte del dinero que creía necesitar.

Solo al fin en su cuarto y no teniendo nada que hacer, sentose en la
cama y se zambulló en el mar sin fondo de sus pensamientos. «Con poco
dinero, pero con dinero al fin, mañana será. No varío mi plan, ni tengo
que modificar las instrucciones que Aura habrá recogido esta noche en el
sombrero de Milagro. ¡Mañana\ldots! Y a pedir de boca saldrá, pues
previsto está todo, y bien determinada la manera de sortear cualquier
peligro\ldots{} Mañana, en pleno día, cuando menos lo pienses, cuando
nada temas, maldita Jacoba, soltarás tu presa\ldots{} Y viviremos los
que debemos vivir, y rabiarán los que deban rabiar\ldots{} y el que
quiera reventar de ira, que reviente\ldots{} Mi gusto es pisotear a la
Zahón; al Sr.~Mendizábal no\ldots{} está próximo a una caída
ignominiosa. En Palacio le tienen ya bien preparada la zancadilla con
Istúriz y Saavedra\ldots{} ¡Los dichosos políticos! No vendría mal una
degollina de próceres y patriotas, como la que se ha hecho de
frailes\ldots{} Pues sí, Sr.~de Mendizábal, bastante tiene Vuecencia con
la que le están armando. Hillo diría que ya se oye el cencerro del
cabestro que viene para conducirle al corral. Y Vuecencia matará los
ocios del corral con la educación de doncellas\ldots{} A Hillo no le
deseo mal alguno\ldots{} ¡Ojalá le hicieran obispo! Bien se lo merece el
pobre por su mansedumbre y buenas intenciones\ldots{} Y en cuanto a
Milagro, nuestra gratitud no se contenta con menos que con nombrarle
Ministro de Hacienda\ldots{} Y a Lopresti, ¿cómo le recompensaremos sus
servicios?\ldots{} Es facilísimo: pinche mayor de Palacio, y además
director de la Real Capilla; cocinero y tiple de S. M\ldots{} De todos
nos despedimos, porque espero que no hemos de tener el gusto de ver
rostros conocidos en mucho tiempo\ldots{} Y que nos persigan, que nos
busquen, que nos cojan ahora\ldots{} El vuelo será alto\ldots{} y luego,
nuestra cueva de amor tan profunda, que a ella no llegará ni la mirada
de cernícalo de la Zahón, ni el olfato de \emph{Edipo}\ldots»

Por este derrumbadero vertiginoso iban sus pensamientos, cuando llamaron
con fuerte campanillazo y golpes a la puerta de la casa. Sorprendido del
ruido, y alarmado también, pues en su estado nervioso el vuelo de una
mosca le hacía estremecer, salió Calpena a punto que alguien abría; y
vio que avanzaban hacia la puerta de su habitación dos hombres de mala
facha, los cuales con formas rudas y descorteses, previa indagación de
la personalidad, le ordenaron que se dispusiese a salir en su grata
compañía. «¿Pero a dónde?\ldots»

---A la cárcel---dijo el más feo y bruto de la pareja, a punto que
comparecían otros dos, de uniforme, pues eran salvaguardias de la
Subdelegación.

Lo primero que se le ocurrió a Calpena fue coger una silla, con intento
de estrellarla sobre la cabeza del más próximo. Pero pronto se
abalanzaron los esbirros a trincarle del brazo, y privado de todo
movimiento, no tuvo más remedio que entregarse, maldiciendo con terrible
exclamación su fiero destino. Salieron en paños menores los patrones y
algunos huéspedes a lamentar el triste suceso; y mientras uno se
indignaba, y le consolaba otro con frase vulgar, asegurando que todo era
equivocación, los polizontes registraban la cómoda y mesa, para llevarse
cuantos papeles encontraran pertenecientes al presunto criminal
político.

Bajando entre tales sayones, taciturno, mas no resignado, devorando la
angustia y terror de su alma, D. Fernando empezó a ver claro en aquella
inopinada prisión, y se dijo: «Es ella, es la \emph{mano oculta} quien
me lleva a la cárcel.»

De la calle de las Urosas al Saladero había mucho que andar. Por el
camino vio dos traíllas de presos. Sin duda, el medroso Gobierno,
acosado de conspiradores, viendo por todas partes misteriosos enemigos
que le acechaban en la obscuridad de las logias, o le provocaban en el
público escándalo de los cafés, había mandado echar la red. Cuando
metieron al desdichado Calpena en el patio donde debía empezar la
expiación de sus nefandos delitos, ya había llegado la primera cuerda,
en la cual vio personas de aspecto decente. Al poco rato entraron dos
racimos más, ¿y cuál no sería la sorpresa de D. Fernando al vislumbrar
en uno de ellos nada menos que la venerada, inofensiva persona de D.
Pedro Hillo?

En cuanto pudieron reconocerse, a la luz de los farolillos que
alumbraban los tristes grupos, corrieron el uno hacia el otro y se
dieron los brazos.

«\emph{Tu quoque}\ldots{} ¡También usted, D. Pedro!» dijo Calpena con el
gozo amargo de la venganza.

---También---replicó Hillo con voz opaca, casi lloroso.---Y verdad que
por más que me devano los sesos, no acierto a explicarme\ldots{} De la
cama me sacaron estos verdugos. Comprendo que a ti\ldots{} ¡A ti
sí!\ldots{} Era necesidad ponerte a la sombra.

---Yo no conspiro.

---Conspiras contra ti mismo. Yo, ni contra mí ni contra nadie\ldots{}
No he hecho más que hablar mal de Mendizábal\ldots{} y eso no mucho.

---No es Mendizábal, no, quien ha tenido la humorada de juntarnos aquí:
es la \emph{mano oculta}\ldots{} ¿Tan candoroso es mi buen clérigo que
no lo ve?

---¡Fernando!

---¡La invisible deidad, la tutelar, la próvida mascarita!\ldots{} ¡Ah!,
no se quiere que el niño esté solo\ldots{} Se teme su desesperación, se
teme su rabia\ldots{}

Enorme distensión de músculos en ojos y boca declaraba el estupor del
buen presbítero.

«No está mal esto. ¿Verdad que no está mal?\ldots{} Para que diga usted
ahora que no \emph{remata}\ldots»

---¡Vaya si \emph{remata\ldots{}}!

\flushright{Santander (San Quintín), Agosto-Septiembre de 1898.}

~

\bigskip
\bigskip
\begin{center}
\textsc{fin de mendizábal}
\end{center}

\end{document}
