\PassOptionsToPackage{unicode=true}{hyperref} % options for packages loaded elsewhere
\PassOptionsToPackage{hyphens}{url}
%
\documentclass[oneside,14pt,spanish,]{extbook} % cjns1989 - 27112019 - added the oneside option: so that the text jumps left & right when reading on a tablet/ereader
\usepackage{lmodern}
\usepackage{amssymb,amsmath}
\usepackage{ifxetex,ifluatex}
\usepackage{fixltx2e} % provides \textsubscript
\ifnum 0\ifxetex 1\fi\ifluatex 1\fi=0 % if pdftex
  \usepackage[T1]{fontenc}
  \usepackage[utf8]{inputenc}
  \usepackage{textcomp} % provides euro and other symbols
\else % if luatex or xelatex
  \usepackage{unicode-math}
  \defaultfontfeatures{Ligatures=TeX,Scale=MatchLowercase}
%   \setmainfont[]{EBGaramond-Regular}
    \setmainfont[Numbers={OldStyle,Proportional}]{EBGaramond-Regular}      % cjns1989 - 20191129 - old style numbers 
\fi
% use upquote if available, for straight quotes in verbatim environments
\IfFileExists{upquote.sty}{\usepackage{upquote}}{}
% use microtype if available
\IfFileExists{microtype.sty}{%
\usepackage[]{microtype}
\UseMicrotypeSet[protrusion]{basicmath} % disable protrusion for tt fonts
}{}
\usepackage{hyperref}
\hypersetup{
            pdftitle={LA CAMPAÑA DEL MAESTRAZGO},
            pdfauthor={Benito Pérez Galdós},
            pdfborder={0 0 0},
            breaklinks=true}
\urlstyle{same}  % don't use monospace font for urls
\usepackage[papersize={4.80 in, 6.40  in},left=.5 in,right=.5 in]{geometry}
\setlength{\emergencystretch}{3em}  % prevent overfull lines
\providecommand{\tightlist}{%
  \setlength{\itemsep}{0pt}\setlength{\parskip}{0pt}}
\setcounter{secnumdepth}{0}

% set default figure placement to htbp
\makeatletter
\def\fps@figure{htbp}
\makeatother

\usepackage{ragged2e}
\usepackage{epigraph}
\renewcommand{\textflush}{flushepinormal}

\usepackage{indentfirst}

\usepackage{fancyhdr}
\pagestyle{fancy}
\fancyhf{}
\fancyhead[R]{\thepage}
\renewcommand{\headrulewidth}{0pt}
\usepackage{quoting}
\usepackage{ragged2e}

\newlength\mylen
\settowidth\mylen{...................}

\usepackage{stackengine}
\usepackage{graphicx}
\def\asterism{\par\vspace{1em}{\centering\scalebox{.9}{%
  \stackon[-0.6pt]{\bfseries*~*}{\bfseries*}}\par}\vspace{.8em}\par}

 \usepackage{titlesec}
 \titleformat{\chapter}[display]
  {\normalfont\bfseries\filcenter}{}{0pt}{\Large}
 \titleformat{\section}[display]
  {\normalfont\bfseries\filcenter}{}{0pt}{\Large}
 \titleformat{\subsection}[display]
  {\normalfont\bfseries\filcenter}{}{0pt}{\Large}

\setcounter{secnumdepth}{1}
\ifnum 0\ifxetex 1\fi\ifluatex 1\fi=0 % if pdftex
  \usepackage[shorthands=off,main=spanish]{babel}
\else
  % load polyglossia as late as possible as it *could* call bidi if RTL lang (e.g. Hebrew or Arabic)
%   \usepackage{polyglossia}
%   \setmainlanguage[]{spanish}
%   \usepackage[french]{babel} % cjns1989 - 1.43 version of polyglossia on this system does not allow disabling the autospacing feature
\fi

\title{LA CAMPAÑA DEL MAESTRAZGO}
\author{Benito Pérez Galdós}
\date{}

\begin{document}
\maketitle

\hypertarget{i}{%
\chapter{I}\label{i}}

En la derecha margen del Ebro y a cinco leguas de la por tantos títulos
esclarecida Zaragoza, existe la villa de \emph{Julióbriga}, fundación de
romanos, según dicen libros y rezan lápidas desenterradas, la cual, en
tiempos remotos, mudó aquel nombre sonoro por el de \emph{Fuentes de
Ebro}, con que la designaron cien generaciones aragonesas. No por los
hechos históricos que ilustran esta villa (pues en lo antiguo dicen que
fue \emph{lugar de moros}, y algún chinazo le tocó en la guerra de la
Independencia y en los dos inmortales \emph{sitios}); no por la
fertilidad de su término, regado por el Canal Imperial; no por las
estameñas que fabrican sus tejedores, ni por las excelentes lechugas que
crían sus huertas, ni tampoco por su gótica iglesia parroquial, donde
yacen, en desmoronados sepulcros, multitud de Condes de Fuentes que
rabiaron o hicieron rabiar al pueblo, aparece este en la primera página
de la presente relación, sino por la fama del \emph{parador de
Viscarrués}, situado en la plaza junto a la llamada \emph{casa del Rey},
el cual gozaba de gran crédito y favor entre los arrieros y trajinantes
que comunicaban a Zaragoza con el Reino de Valencia. Asimismo confluían
allí los trayectos peoniles y carromateros de la parte de Alcañiz, del
Maestrazgo y Vinaroz, de la tierra baja de Teruel, Híjar y la cuenca del
río Martín. Los barqueros del Canal Imperial, así como todo el personal
de fontanería, eran también fieles parroquianos de Viscarrués, el cual
daba excelente trato a las caballerías primero, a las personas después,
y poseía un amplio local con cuadras extensas, donde podían acomodarse,
entre animales y arrieros, como unos treinta pares. En el piso alto no
faltaban aposentos \emph{para señores}, algunos hasta con camas, otros
bien acondicionados de mullidos jergones. Era la cocina monumental, con
el hogar guarnecido de poyos, y por uno y otro lado mesas largas, donde
podían tomar el pienso hasta veinte parroquianos. Servía Viscarrués un
Cariñena superior, sin competencia en cuatro leguas a la redonda, y para
todo pasto un tintillo de Contamina que en lo de alegrar corazones y
cabezas parecía hermano de la jota. Uno y otra procedían de la misma
cepa.

Los más de los días Viscarrués y su familia no tenían manos para servir
a la mucha y diversa gente que en el parador se juntaba. Uno de los
criados, llamado \emph{Guasa} (verdadero apellido, no apodo), natural de
Jaca, y más vivo que el azogue, hacía milagros de ubicuidad y
diligencia. Pero llegó un día; mejor dicho, llegaron tres días, en que
ni el ventero con sus hijas y su mujer, ni Guasa con toda su agilidad
ratonil, pudieron atender al golpe de personas y acémilas que se
metieron por aquellas puertas con hambre y sed, pidiendo vino, cebada,
carne y un montón de paja para dormir. Furioso Viscarrués por no
disponer de cuádruple local, se tiraba de los pelos, y su mujer del
moño; Guasa andaba de coronilla; la \emph{parroquia} se impacientaba;
todos pedían a un tiempo su remedio. Con gran trabajo y a puñados les
iban acomodando aquí y allí, metiendo ocho en cada cuarto de arriba,
estibando a otros en las cuadras, por grupos, por series, por manadas: y
para dar de comer se ponían los platos en el suelo, por no haber ya
mesas, los jarros de vino pasaban de boca en boca, sin vasos; los
guisados iban a la rueda en grandes fuentes, chorreando salsa, y no se
oían más que voces airadas del que pedía su parte, del que, no contento
con la primera ración, pedía la segunda. Aquí esgrimían cucharas, allá
repicaban en los vasos con toque de cuchillos. El vino abundante suplía
las escaseces del comer, y si en una parte echaban maldiciones a
Viscarrués, en otra le vitoreaban como al primer posadero del mundo.
«Hay que dispensar en días como este,» decía él, rascándose la cabeza,
luego los brazos, levantándose después la faja que se le caía. A Guasa
colmábanle de injurias, que le excitaban a un enojo risueño; y era tal
su sofocación, que regaba con honrado sudor los manjares que servía.

Fue a causa de tan desmedida aglomeración la coincidencia de dos
caravanas de pasajeros, la una que venía de Oriente huyendo de la
guerra, la otra de Occidente que hacia la guerra iba. Componían la
primera familias neutrales o que querían serlo, algunos lisiados y
enfermos; la segunda constaba, principalmente, de la oficialidad y
clases de una columna enviada del Norte para incorporarse a la brigada
de Borso di Carminati. La guerra mata y resucita; destruye y crea. La
sangre que no se derrama en los combates, circula con más vigor, y nutre
partes desmedradas del organismo social, mientras otras perecen.
Viscarrués, que se estableció sin un cuarto en 1830, se retiró el 46 con
el riñón bien cubierto. Sus hijos siguieron carrera en Zaragoza.
Traspasado el parador a Guasa, este se hizo también rico, y en 1860
poseía casas en la Almunia, un café en Cariñena, y suyos eran los coches
de la estación de Calatayud, y los que hacían el servicio a Paracuellos
de Jiloca. Volviendo a lo que se refiere, debe decirse que aquel tumulto
del parador de Fuentes de Ebro pertenece a las cronologías del año 37,
que hasta en los mesones había de ser año de confusión y trapisondas: el
mes era \emph{Febrerillo loco}. Un solo dato pudo arrancar el
historiógrafo a la empedernida memoria de Mateo Guasa: era que aquel día
fue el primero del año en que se agregaron al cocido las habas verdes.

Y que estaban muy buenas, como declararon todos, con excepción de una
señora, ribereña de Navarra, que sostuvo la superioridad de las habas de
tierra de Cintruénigo\ldots{} A esto observó uno, después de empinar el
codo, que mejor que las habas le sabían a él las hembras de la Ribera, y
buena muestra del género era \emph{lo presente}, cuya gentileza y
hermosura a todos cautivaban\ldots{} Replicó ella con donaire que no era
ensalada más que para un solo y único dueño, el cual no admitía bromas.
Pronto se corrió entre los individuos de aquel jovial grupo que la tal
moza era casada, y que iba a la guerra con su marido, sargento
recientemente ascendido a alférez, el cual se alojaba también allí, y
había salido a ocupaciones del servicio. Entrando en conversación la
hermosa mujer, en quien habrá reconocido el lector a Salomé Ulibarri,
les dio cuenta, con abundosa y pintoresca verbosidad, de los prodigios
de Luchana y Banderas, y de las proezas que allí había realizado
Baldomero Galán, su esposo, secundando las disposiciones del otro
Baldomero. El empleo de alférez era recompensa mezquina para servicios
tan eminentes\ldots{} Despertada en el auditorio la curiosidad, se
prolongó el relato de lo de Bilbao bastante tiempo, tan gustosos ellos
de oír a la historiadora, como esta de pregonar tan lucidas hazañas.
Emprendieron después los otros historia fresca de lo del Maestrazgo, que
habían visto; pero a lo mejor de ella, solicitada de otra parte la
atención de Saloma, se apartó de la mesa. Mirando casualmente hacia la
escalera del parador, vio que por ella descendía un caballero anciano en
compañía de dos mozos, al parecer de su servidumbre, el cual, renegando
con agrias voces de no encontrar alojamiento adecuado a su categoría,
avanzó hacia la calle cogido al brazo de un criado. Tanto el fláccido
rostro del noble señor, como su desmayado cuerpo y su deslucida y
polvorienta ropa, declaraban el cansancio de un largo camino. Fue tras
él Saloma, y viéndole parado en medio del portal, se le puso delante en
actitud de quien intenta dar una sorpresa; mas no hizo el buen señor
ademán de conocerla. Impaciente y desconcertada la moza, además movida
de grande compasión hacia el caballero, le tocó suavemente en el brazo,
diciéndole: «¿Pero es posible que no me conozca o no quiera conocerme el
Sr.~D. Beltrán de Urdaneta?»

«¡Saloma\ldots{} hija\ldots{} chica!---exclamó el prócer abriendo los
brazos.---¿Tú por aquí?\ldots{} \emph{Maña}, te he conocido por la
voz\ldots{} ¿No sabes? ¡Ay, me estoy quedando ciego!\ldots{} Salgamos un
poquito afuera, para que con la luz de la calle pueda ver tu hermosura.

---¿Pero a dónde va por aquí tan descarriadico, señor?

---Hija\ldots{} es largo de contar---replicó Don Beltrán, sacando un
pesado suspiro de las honduras de su pecho.---Me muero de fatiga, de
hambre\ldots{} y ese bruto de posadero no quiere alojarme\ldots{} No
puedo ya con mi cuerpo\ldots{} ni con mi alma.

---Todo lo de arriba está lleno\ldots{} En cada aposento siete
personas\ldots{} como sardinas. Tampoco yo tengo cuarto.

---Déjame, déjame que te mire\ldots---dijo el prócer acercando su rostro
al de ella, embobado, sobreponiendo su afición estética a las tristezas
del desamparo en que se veía.---Sí, sí: te reconozco\ldots{} ¡Qué linda
eres! Si no fuera sacrilegio suponer que Dios se equivoca, le
preguntaría por qué no te hizo nacer en posición elevada. Habrías sido
una gran mujer, una gran dama, una\ldots»

Más atenta a proporcionar al noble señor el reparo que necesitaba que a
sus delicados galanteos, le dijo que urgía disponerle al instante la
mejor comida que se pudiese. Enganchándole del brazo, le condujo hacia
la cocina, dando voces al paso, en requerimiento de Guasa y de los demás
servidores de la posada. «¡Qué desconsiderados sois!---dijo al propio
Viscarrués.---¿Pero no conocéis al señor, el primer noble de Aragón? No
sabéis tratar más que con animales.» Disculpose el ventero, alegando que
no había conocido al Sr.~D. Beltrán, y se apresuraron amo y criados a
ofrecerle cuanto tenían. A ratos ayudando a servirle, a ratos sentada
frente a él viéndole comer y beber con gana, nuevamente le interrogó
Saloma sobre su viaje, movida no tan sólo de la mujeril curiosidad, sino
del interés afectuoso y desinteresado que el ilustre viejo le inspiraba.
«¿Va el señor a Zaragoza, o viene de allí?

---Vengo, hija, vengo\ldots{} He salido de Cintruénigo con ánimo de no
volver más allá. Un rapto de cólera, de orgullo, de dignidad más
bien\ldots{} Yo soy así: no tolero que nadie me humille; y las
impertinencias y groserías de Rodrigo y de \emph{Doña Urraca} han sido
tales, que no he tenido alma para tolerarlas más tiempo. Salí del
caserón de Idiáquez como un colegial que se escapa. A la falta de
libertad, al despotismo de \emph{Doña Urraca} y de su hijo, prefiero la
vagancia, la miseria, la muerte misma\ldots{} No más, no más\ldots{}

---Supe que el señor había ido a Medina de Pomar.

---Y no encontré ¡ay de mí! la acogida que esperaba\ldots{} Ya no hay
hijos, quiero decir, hijos buenos. Esa raza concluyó. Con estas malditas
guerras entre hermanos, parece que ha venido al suelo toda ley de
humanidad, y hasta los sagrados fueros del parentesco y de la
sangre\ldots{} Al hablar de estas cosas, se me atraviesa aquí en el
pecho un bulto, una cosa dura y lacerante que no me deja comer ni
respirar\ldots{} Espérate a que pase\ldots{} Ya pasa\ldots{} Te contaré
en dos palabras que al volver de Mena, donde, lo repito, encontré más
egoísmo que piedad, desconsideraciones que me han llegado al alma,
recibiéronme los Idiáquez de un modo muy desapacible. Los morros de
\emph{Doña Urraca} se extendían cuarta y media fuera de las líneas
borriquiles de su rostro, y mi esclarecido nieto no hacía más que
contrariar mis hábitos y rodearme de estrecheces indecorosas. ¿La causa
de esto? Es muy sencilla. Sabrás que entre mi nuera y Doña María Tirgo
habían concertado la boda de Rodrigo con una rica heredera de La
Guardia. Celebráronse vistas. No sé lo que pasó, pues yo me hallaba en
Mena; sólo supe, antes de salir de allí, que de improviso y con algo de
estruendo se vino a tierra todo aquel tinglado de la boda.

---¿Y le echaron al señor la culpa?

---Naturalmente: yo soy el gato, el niño enredador causante de todas las
roturas de platos y demás averías que ocurren en la casa. No hay quien
le quite de la cabeza a Juana Teresa que por intrigas mías se deshizo el
bodorrio. Y yo te aseguro que no he tenido arte ni parte en ello.
Declaro ingenuamente, eso sí, que me alegré y me alegro del percance,
festejándolo como justicia de Dios y castigo de la conducta inhumana de
los Idiáquez con este pobre viejo. Pero nada más, nada más\ldots{}
Cansado al fin de la reglamentación de colegio a que pretendían
sujetarme, me vi en el duro caso de preferir la miseria a la esclavitud,
y la libertad al vivir triste, al régimen conventual de la casa de
Cintruénigo. La imagen de \emph{Doña Urraca} se me ha hecho tan odiosa,
que por no verla me iría descalzo y pidiendo limosna a la más lejana
región del mundo. Créelo, chica. Soy noble: no tolero la humillación. En
cualquier estado sabré conservar mi dignidad.»

Con pena y lástima muy vivas oyó Saloma el relato de D. Beltrán, no
atreviéndose a contradecirle ni a proponerle la vuelta al hogar
abandonado, porque el respeto a tan gran caballero y a su desgracia la
cohibía. Atenta al alivio de su necesidad, le dijo que pues era
totalmente imposible recabar de Viscarrués un buen aposento, no había
más remedio que acomodar al señor en la cuadra. Ella respondía de
arreglarle en aquel humilde lugar un lecho abrigado y cómodo, combinando
los haces de paja y las buenas mantas que ella traía, de tal modo que no
echara de menos los infames camastros de la posada. Accedió a esto D.
Beltrán con expresiones de gratitud, muy conmovido, sonándose fuerte, y
añadió que pues Jesucristo Nuestro Señor nos había dado ejemplo de
humildad naciendo en un pesebre, bien podía sin desdoro un noble, que
nada tenía de divino, dormir y hasta terminar su existencia en montones
de paja, al abrigo de gentes sencillas y de rústicos animales.

\hypertarget{ii}{%
\chapter{II}\label{ii}}

«Ya sé---dijo después el prócer a la guapa moza, plegando los ojos para
verla mejor,---que al fin te has casado con Baldomero. No ha sido poca
suerte para ese bruto. ¡Vaya una hembra que se lleva!

---Sí, señor\ldots{} ¿Pero usía no sabe que es alférez?

---¡Qué me dices!\ldots{} ¡Alférez! ¡Hola, hola!\ldots{} ¡Todo un
oficial del ejército! Siempre fue arrojadísimo, con una cabeza más dura
que el mármol, y un corazón insensible al miedo\ldots{} Vaya: ¿y está
aquí, en la columna que ha llegado del Norte?

---¡Y que no se alegrará poco de ver a Vuecencia! No tardará en venir.»

A uno de los mozos de Urdaneta, que en otra mesa comían, ordenó Saloma
que saliese a buscar a Galán por las calles del pueblo, y a darle
conocimiento de la presencia de su antiguo amo. Nacido en Fuenmayor y
recriado en Cintruénigo, Baldomero había servido a D. Beltrán antes de
entrar en el militar servicio. Seis años comió el pan de Idiáquez y
Urdaneta, ya en el empleo de ayuda de cámara, ya en el ejercicio de
montería, o en otros menesteres de la casa. Bien quisto de sus amos,
dejó en la familia memoria de leal y honrado, aunque muy duro de
mollera. Andando el tiempo, ya soldado distinguido, sargento después,
siempre que su batallón pasaba por Cintruénigo, visitaba a los señores.
Allí conoció a Saloma, que, rodando de aquí para allá con borrascosa y
turbada vida, después del fusilamiento de su padre en Miranda de Arga,
fue a parar a casa de una tía materna, que tenía en arrendamiento
tierras de Idiáquez y vivía en una torre próxima al palacio señorial.
Toda esta parte de la historia de Galán y Saloma es algo obscura, y no
ofrece bastante interés para que se emprendan, por esclarecerla,
investigaciones muy minuciosas.

Volviendo al relato, se dirá que D. Beltrán manifestó a su amiga que no
iba, no, a la ventura por aquellos derroteros, pues le guiaba un fin
concerniente a sus intereses y al remedio inmediato de su actual
posición lastimosa. «Ya te lo explicaré cuando esté más
sosegado---agregó recobrando algo de su animación,---pues supongo que
iremos juntos largo trecho. Por de pronto, sólo te digo que salí de
Cintruénigo con recursos muy inferiores a lo que exige mi categoría, que
tendré que resignarme a ciertas privaciones\ldots{} Mi principal
inquietud es que me corten el paso las tropas de Cabrera o las partidas
que, sueltas y desmandadas infestan toda la tierra de Teruel. Otro temor
me quita el sueño, y es que los dos únicos chicos que he podido traerme,
Tomé, el de \emph{la Chata}, y Francisquillo Maestre, no puedan seguir
en mi compañía más allá de Híjar, por el peligro de que les coja la
facción. Tú les conoces: dos chicarrones de diez y nueve años, que no
manejarían mal el chopo, y de uno de ellos sospecho que lo cogería de
buena gana, por dar gusto al dedo. En fin, si les pierdo, ya sea por
medrosos, ya por atrevidos, tendré que ir solo, encomendándome a Dios y
a la Virgen, pues no puedo abandonar mi empresa, única solución decorosa
para los pocos días que me restan de vida.»

En esto entró Baldomero, que derechamente, morrión en mano, se fue a
besar la de D. Beltrán, y poco le faltó para hincar una rodilla en
tierra. Sincero, nacido del corazón era su acatamiento, pues amaba al
anciano; y cuando este abrió sus brazos para expresarle con un buen
apretón su enhorabuena y el regocijo de verle oficial, Galán hizo
pucheros, y algunas lágrimas bajaron a humedecer su bigote de moco,
imitación del de Espartero.

«Bien, hijo, bien, adelante\ldots{} Capitán, será ya como tenerlo en la
mano. Date prisa a ganar empleos, porque antes de morirme quisiera ver a
Saloma hecha una señora coronela.»

Era Baldomero Galán un mocetón en quien la estampa no desmerecía del
apellido, alto, garboso, mejor formado de cuerpo que de facciones, pues
su nariz excedía un tanto de la medida proporcional, y sus ojos,
hermosos y grandes, bizcaban un poco, resultando una desmedida fiereza
de expresión. Indomable en la guerra, fiel a sus deberes cual ninguno,
pronto a dar la vida cien veces por el honor de su bandera, en la vida
doméstica era un angelón, y su esposa no tenía que hacer el menor
esfuerzo para dominarle. Hízole sentar D. Beltrán a su izquierda: le
sirvió vino, después de obsequiarle con un puro. Fumando los dos, el
pobre viejo, gozoso de tener a quien contar sus infortunios, hizo
segunda edición de lo que ya había referido a Saloma, recargando
amargura en las acusaciones contra su nieto y nuera. Suspiraba Galán al
compás de los suspiros de su antiguo señor; y no acertando con la mejor
fórmula de consuelo, se ofreció a prestarle en su viaje toda la ayuda
que el servicio le permitiera. «Tanto Saloma como yo, Sr.~D. Beltrán,
estamos a la disposición de usía para lo que guste mandarnos, y le
cuidaremos y asistiremos como a un padre.» Urdaneta le apeó el
tratamiento, pues del chicarrón que tuvo a su servicio al señor alférez
que delante veía, había distancia social muy grande: agradeciendo al
matrimonio sus ofrecimientos, manifestó que deseaba recogerse.
«Véase---dijo a Galán, mientras corría Saloma en busca de las
mantas,---cómo Dios no abandona a los buenos. Solo y triste venía yo por
esos caminos, agobiado del peso de mis desdichas, afligido al propio
tiempo por mi ceguera que crece de día en día, y cuando menos lo
esperaba, me salen al encuentro dos amigos cariñosos, dos almas
caritativas que me consuelan, que me alientan\ldots{} ¡Qué hermoso es
encontrar en nuestro camino la gratitud! Tú y tu mujer me debéis algunos
beneficios; también los prodigué yo al buen Adrián Ulibarri, padre de
Saloma, y ahora me veo recompensado por vosotros\ldots{} ¡Ah! si me
pierdo, que me busquen entre los humildes, que son siempre los
agradecidos y generosos.»

Irguiéndose, como si al restaurar las fuerzas de su cuerpo recobrase
también vigor y esperanza su espíritu, emprendió, asido del brazo de
Galán, el camino de la cuadra. Parándose a cada instante, decía: «No,
no: Urdaneta no puede ni debe terminar sus días en la humillación. Oye,
\emph{Mero}: ¿será fácil penetrar en tierra de Teruel hasta Mora de
Rubielos, siquiera hasta los montes de Gúdar?

---Señor, \emph{las hordas de Cabrera} son dueñas de casi todo el
país---replicó Galán, que hablando de guerra solía emplear las fórmulas
usuales de la prensa patriótica, de las proclamas y órdenes generales en
campaña;---y mientras no consigamos limpiar de \emph{enemigos
fratricidas} todo el territorio de esta Comandancia general, no le
aconsejo a nadie que penetre, señor\ldots{} a menos que lleve un
salvoconducto en regla, expedido por el \emph{obcecado Pretendiente}.

---Ya, ya lo pensaremos, pues entre los cabecillas facciosos no me
faltan amigos.»

En esto, Saloma escogía el rincón más abrigado de la cuadra, el mejor
defendido contra las corrientes de aire y las patadas de los mulos, para
armar en él un mullido nidal donde descansase el noble viejo. Fue
robando puñaditos de paja en este y el otro montón; apartó toda la
basura; hizo mudar de sitio a un gallo con varias gallinas, y la obra
quedó terminada pronto a satisfacción del que debía disfrutarla. Todas
las mantas que tenía las aplicó a la comodidad de Don Beltrán, unas
debajo, otras encima de su cuerpo. Mientras \emph{Mero} le quitaba las
botas, envolviéndole los pies en la manta de Tomé, Saloma le liaba a la
cabeza una ancho pañuelo de seda, despojándole antes de su levitón y
dejándole en mangas de camisa. Ofrecía el aristócrata una extraña
figura, de la que él mismo se reía, cuando se tendió de largo a largo
sobre la paja. Con refajos y ropa suya improvisó Saloma una almohada, y
no pareciéndole bastante, propuso que ella se acomodaría sentada junto a
la pared, formando como cabecera del improvisado lecho, y sobre sus
rodillas se apoyaría la almohada, sosteniéndola en alto de modo que no
se hundiese la cabeza de D. Beltrán. Para completar la obra, se convino
en que Galán pasaría la noche a los pies del señor, para contener el
frío por aquella parte, mientras por la otra sostenía el calor el gentil
cuerpo de Saloma. Hallábase Urdaneta algo acatarrado, y estornudaba
constantemente; mas no sintiendo otra molestia real que el frío,
procuraba agazaparse bien, y en medio de las mantas recobró su buen
temple y jovialidad, dando por excelente tal situación y creyéndola un
especialísimo favor de Dios en aquellos tristes días. «Paréceme, hijos
míos, que no debo quejarme---les dijo risueño,---¿pues qué más puedo
ambicionar que este tranquilo reposo, este abrigo que me habéis dado, y,
sobre todo, el calor de vuestra compañía cariñosa? Os veo como a dos
ángeles que Dios me envía para asistirme. Y es como si con vuestra
presencia me dijera: «Ya ves, Beltrán mío, que no te abandono.» En
verdad os aseguro, que no cambiaría este lecho por el del Papa o el
Emperador de Rusia. Aquí se está muy bien, con un guardián y calentador
por la cabeza y otro por los pies\ldots{} y esta sencillez, y esta
libertad\ldots{} Vamos, que estoy contentísimo, y ahora me permito
despreciar todos los cuartos de fonda, con sus camas frías y sucias, y
su soledad triste\ldots{} Bien, bien: \emph{Mero} y Saloma, mis buenos
amigos, sed caritativos hasta el fin; y pues el sueño se ha declarado mi
enemigo, contadme alguna cosita para engañar el tiempo.»

Reclinado a los pies del señor, Galán habló largamente de la campaña del
Centro, a la cual se daría gran impulso para exterminar de golpe a
\emph{los satélites del obscurantismo}. No lejos de ellos había otros
grupos; y a medida que avanzaba la noche, fueron entrando en la cuadra
más huéspedes, y se formaron entre paja y dornajos montones de humanidad
que producían extraños ruidos: aquí conversaciones y disputas
vehementes, allá un roncar estruendoso.

«\emph{Mero}, hijo mío---dijo al alférez D. Beltrán, de cuya persona no
asomaba entre las mantas más que la nariz,---por alguna palabra que
llega a mis oídos de lo que hablan esos tres hombres que están a tus
pies, entiendo que son de Rubielos. Acércate y pregúntales si conocen a
Juan Luco, rico propietario en término de Mora, alcalde que era de esta
villa hace dos años.» Poco después se aproximó un hombre, de estatura
más que alta gigantesca, vestido a estilo aragonés neto, con su
pañizuelo en la cabeza, faja morada y muy caída, mal envuelto en una
manta, como herido o enfermo, un brazo en cabestrillo, la faz atezada,
ruda, huraña. De su andar no debía decirse que era cojo, sino que
cojeaba, y uno de sus pies, envuelto en un lío de trapos, abultaba como
la pata de un elefante. Sus primeras palabras, al acercarse al grupo,
fueron torpes, balbucientes: «El señor alférez me manda\ldots{} que le
diga\ldots{} Gran señor, yo no veo dónde está su Ilustrísima, ni sé
quién dimonios es\ldots{} ¡Otra!\ldots{} Ya le veo como enterrao en el
panizo\ldots{}

---Siéntate\ldots{} tú eres de Teruel: no puedes negarlo---dijo D.
Beltrán sin moverse, no enseñando de su persona más que los ojos sin
vista y la nariz sin olfato.---Descansa, que, por las trazas, bien lo
necesitas.» Con lentitud y ayes de dolor fue doblando su corpachón el
aragonés hasta hundir la paja con sus asentaderas, no lejos del puesto
de Galán, y cuando halló postura cómoda, dijo que de Teruel mismamente
no era, sino de Cuatro Dineros, barrio de Montalbán, y que conocía todo
el país entre Ademuz y Puerto de Beceite como la palma de su mano.

«¡Ah---exclamó Saloma prontamente,---si ya te conocemos! Yo bien decía:
conozco a este bruto. Tú eres Joreas, el que hace dos años trajinaba con
mulas desde Vinaroz a Tudela\ldots{} Y después te fuiste a la facción, y
de la facción vienes ahora, puerco.

---Con perdón de la señá tinienta y de la compañía, digo que lo de
puerco no es razón, y sí lo es que me llamo Tanasio Joreas. Como hombre
honrado y cabal, no niego haber estuvido en la faición a las órdenes del
\emph{Serrador} primero, del \emph{Royo de Nogueruelas} dispués, porque
sentía de mi natural que debíamos ensalzar los divinos derechos del Rey
D. Carlos\ldots{} Pero aquí me tienen harto de desengaños, con más
balazos en mi cuerpo que pelos en la cabeza, muerto de hambre, con mi
casa y familia perdidas, porque una de mis masadas la arrasó el liberal,
otra el legítimo\ldots{} mis hijos muertos, todo hecho cenizas, y yo
poco menos que cadavérico. Lo que no me ha quitado el neto, me lo ha
quitado la usurpadora; y al fin, cansado de pelear, y de sufrir, y de
ver espantos, y de pisar tripas de cristianos, dije: «No más derechos
legítimos ni no legítimos, no más, no más,» y me escapé, y huyendo de la
tremolina vengo por trochas y atajos en busca de un terreno donde haiga
paz, donde los hombres sean cristianos, no carniceros\ldots{} Yo he sido
malo; yo he sido, como tantos, lo que dice la señora, faicioso y
peleador y verdugo de mi natural; pero ya le he tomado asco al matadero.
Me llamo \emph{Joreas el escarmentado}, y voy a Zaragoza en busca de un
pedazo de pan que yo pueda meter en la boca sin que, al mascarlo, me
parezca que lo han amasado con sangre.»

Callaban todos los oyentes, entristecidos por las lúgubres palabras del
escarmentado, y al fin rompió el silencio D. Beltrán, diciendo: «Pobre
Joreas, tu arrepentimiento es de celebrar, y ojalá se convencieran todos
como tú y siguieran tu camino\ldots{} Pero vamos a lo que me importa.
Conocerás a Juan Luco.

---De los mejores hombres de Aragón\ldots{} sí, señor\ldots{} gran
presona\ldots{} Y con muchas talegas. Suyas eran las dos masadas de
Rubielos, y en Mosqueruela y Forniche Bajo tenía más de mil
cabezas\ldots{} hombre cabal, buen amigo y padre del pobre\ldots{}

---Hablas como si Luco no existiera. Explícate: ¿ha muerto?

---Señor, no se enfade conmigo, que yo no he sido más que destrumento. A
la vuelta de Manzanera nos salió con catorce hombres armados de
escopetas\ldots{} Le cogió la partida de Peinado, donde yo iba, y no
tuvimos más remedio que afusilarle\ldots{} Señor, puede creérmelo: como
Dios es mi padre le digo que le digo la verdad\ldots{} Fue que cuando me
mandaron tirarle y le tiré, las lágrimas me corrían\ldots{} Yo decía
para mí: perdóneme, Don Juan, que no soy más que destrumento\ldots{}

\hypertarget{iii}{%
\chapter{III}\label{iii}}

---¡Qué horror!---exclamó D. Beltrán, haciendo sonar la paja con el
estremecimiento de todo su cuerpo.---Bandido, quítate de mi
presencia\ldots{} No, no te vayas: da más explicaciones\ldots{}

---Bandido no, señor\ldots{} Yo lloraba\ldots{} Es la guerra, señor, la
guerra. Aluego que le enterramos fuimos a quemarle la masada de Cabra de
Mora.

---¿Y la incendiasteis?

---No pudo ser, señor, porque\ldots{} la habían quemado ya los cristinos
el día antes, llevándose dos yeguas. Fue la columna del coronel Buil,
uno muy perro, que fusiló en Concud a mi hijo Agustín.

---Ojo por ojo y diente por diente. Los hijos de Luco vengarán a su
padre.

---No, señor. ¿Les conoce Vocencia?

---Sí, y sé que son valientes.

---Eran.

---¿También han muerto?

---No me eche a mí la culpa, sino al Nogueras, el más bruto que hay en
la Usurpación.

---¿Luego eran carlistas?

---Bruno sí, señor: desde el tiempo de Carnicer se alistó en las sacras
banderas. Luego andaba con el \emph{Fraile Esperanza} y con el
\emph{Organista de Teruel}. No tenía trato con su padre ni con su
hermano Cinto, el cual seguía la bandera puerca de Isabel\ldots{} Por
esto dicen que esta guerra se ha vuelto tan farisea o faricida.

---Fratricida, que quiere decir guerra entre hermanos.

---Y entre padres e hijos, y maridos y mujeres. Cinto Luco, casado en
Aliaga con la hija mayor de Crescencio Marlofa, salió con los urbanos de
la villa y un destacamento de tropa. D. Ramón, el propio D. Ramón, les
deshizo\ldots{} Escapó Cinto con su mujer y el chico menor de Marlofa, y
se escondieron los tres en una cueva de Peñarroya de los Pinares, donde,
descubiertos por el cura Lorente\ldots{}

---¿También fusilados? ¡Qué villanía!

---No, señor\ldots{} les pusieron en cueros, sin distinguir\ldots{}
vamos, que a la chica le quitaron hasta la camisa, y luego les
alancearon\ldots{}

---Cállate, por Dios\ldots{} Vete, vete a expiar tus delitos.

---Es la guerra, señor. Yo no tuve culpa, ni estuve en eso\ldots{} Me lo
contaron.»

Habíanse agregado otros dos al grupo, recostándose junto a Joreas. Por
las trazas eran sus compañeros, como él, escarmentados o arrepentidos.

«Yo le vi---dijo uno de ellos, joven y de palabra fácil y correcta,
revelando mejor educación y origen social que sus compañeros,---y desde
aquel día me escapé con otros seis de la partida de Lorente, y nos
agregamos a Forcadell. Nos teníamos por guerrilleros, no por bandidos.

---No sigáis---dijo D. Beltrán, que no sentía ya frío, sino un calor
sofocante, y sacó los brazos fuera de las mantas;---no sigáis, por Dios,
pues también vais a decirme que el hijo menor de mi queridísimo Juan
Luco, el pequeño, mi ahijado, Francisquín, ha perecido también en esa
guerra de cafres.

---Francisquín fue pasado por las armas en la acción de Liria---afirmó
Joreas.

---Tú no sabes de eso---dijo prontamente el segundo escarmentado.---Yo
estuve en Liria, y puedo contarlo.

---Mi parecer---dijo \emph{Mero},---es que todas esas historias
fratricidas deben quedarse para mañana.

---Lo mismo pienso---manifestó Saloma.---El señor necesita descanso, y
no se le han de contar tragedias, sino chascarrillos y donaires.

---Gracias, hijos míos; pero la ocasión es trágica: no podemos
sustraernos a estos horrores\ldots{} Que sigan: usted, joven, infórmeme
de lo de Liria y de la suerte de mi ahijado Francisco Luco. ¿Es usted de
este país?

---Eustaquio de la Pertusa, natural de Binéfar, en tierra baja de
Huesca, para servir a usted; estudiante de Teología y Cánones hasta
febrero del 35; después ayudante de Cabañero, alférez en la columna de
Pertegaz, y, al fin, escarmentado y desengañado. Pues el 29 de
marzo\ldots{} recuerdo bien la fecha, porque eran mis días: San
Eustaquio, Obispo\ldots{} sorprendimos la plaza de Liria. Don Ramón
recorría el llano de Valencia recogiendo mozos, dinero y caballos.
Pertegaz fue el encargado de la sorpresa. Antes de romper el día nos
llegamos callandito a las puertas de la ciudad, defendida por
nacionales. Abrieron ellos confiados, sin tener noticia de que estábamos
en acecho, y fácil nos fue entrar, despachando en la primera embestida
siete, después nueve, y cogiendo veintisiete prisioneros, con algunos
vecinos del pueblo. Saqueamos no más que dos horas; y al salir, D.
Ramón, que acampado estaba en Puebla de Balbona, nos mandó ir a Chiva
con los prisioneros.

---¿Y entre ellos estaba el pobre Francisquín?\ldots{} ¡ay!

---Sí señor. Yo le conocía del Seminario de Huesca, donde juntos
estudiábamos Teología, y por el camino de Chiva hablamos, y le dije que
tuviera paciencia, que de fusilarles, lo haríamos previa confesión,
según costumbre y ley de nuestro ejército, con lo que, si se perdía el
cuerpo, se ganaba el alma, que es lo principal.

---Grandísimo perro\ldots{} la hipocresía de tu ferocidad me causa
horror---exclamó Don Beltrán sin poder contenerse.---¡Pobre Francisquín!
Sigue, sigue.

---Pues en Chiva se mandó confesar a los prisioneros, que para estos
casos lleva cada partida, por pequeña que sea, su capellán\ldots{}
y\ldots{}

---Basta. ¿Tendrás valor para referir que hiciste fuego sobre tu pobre
amigo, tu compañero de estudios teológicos?\ldots{} ¡Bonita Teología
aprendiste, mal hombre, mal subdiácono, si lo eres, mal español!\ldots{}
Si vives tranquilo será porque no tienes conciencia, porque no sabes lo
que es Dios, aunque mil veces le hayas nombrado estudiando cosas que no
has entendido\ldots{} No me levanto---agregó el señor excitadísimo,
retirando su abrigo y removiéndose sobre la paja,---no me levanto y te
doy un par de pescozones, porque creería deshonradas mis manos de
caballero poniéndolas en la cara de un bandido.

---¡Eh! sepa el vejete---dijo el otro levantándose de un brinco,---que
mi cara no han de tocarla manos nobles y plebeyas. Y si es usted una
senectud y no puede hacer la prueba, destaque alguno de estos, y
salgamos afuera.

---El que sale afuera bailando, con una patada que voy yo a darte ahora
mismo, eres tú, so deslenguado---dijo con fosca serenidad Baldomero,
disponiéndose a ejecutar lo que decía, como la cosa más natural del
mundo.»

D. Eustaquio se engalló también; pero Joreas y el otro le contuvieron
diciéndole: «Guarda, hijo, que es tiniente.

---Y sepan---añadió Galán,---que si los señores escarmentados no guardan
el respeto debido a las personas, aquí no faltará quien les dé la última
mano del escarmiento.

---También aquí fusilamos---dijo Saloma iracunda.---¿Pues qué creen
estos? ¿Que somos de manteca?»

El tercero, que aún no había dicho nada, y era inclinado a la paz y
enemigo de pendencias en tal sitio, tiró del brazo del teólogo D.
Eustaquio para apartarle, ayudándole también Joreas, que venía de la
guerra con el cansancio y aborrecimiento de toda querella homicida.
Terminó el lance de buena manera; alejáronse los dos más levantiscos;
sólo quedó en el corrillo de D. Beltrán el tercero, que se declaró
escarmentado incondicionalmente, con propósito firme de no volver a las
andadas; y aproximándose, como deseoso de ganar confianza, hizo la
siguiente manifestación: «Yo soy de Ablitas, Sr.~D. Beltrán de Urdaneta,
y con nombrarle ya está dicho que le conocí desde que le vi meterse en
la paja. Conozco también a Saloma Ulibarri y a Baldomero Galán, y a
todos me recomiendo para que no me estimen en menos de lo que soy por
esta locura de haber ido a la facción.»

Maravilláronse todos de aquel encuentro, y el primero que rompió a
reconocerle fue Baldomero, que le dijo:

«¡Ajo! ¿no eres tú Vicente Sancho, hijo de José Sancho? Desde que te vi
me chocó el cariz tuyo, y dije: «Yo conozco a este pícaro.»

---El mismo soy. A todos les conocí; pero no quería dar la cara, por
vergüenza.

---¡Vaya con Sanchico!---dijo Urdaneta.---Hombre, me alegro de que seas
tú de allá\ldots{} Oye: ¿no era tu abuelo Bartolomé Sancho albéitar en
Monteagudo?

---Sí, señor\ldots{} Pues verán\ldots{} Son estos dos amigos el uno muy
bruto, y el otro, el \emph{Epístola}, que así le llamamos aunque no
tiene las órdenes, muy vivo de sangre\ldots{} No quisieron ofender al
Sr.~D. Beltrán; y como les pidió que refirieran, empezaron a contar,
poniendo las cosas como fueron, que harto malas son ellas, sin que tenga
la culpa el que cuenta con natural.

---Cierto: yo me acaloré---dijo el prócer.---Si a ellos se les ha pasado
el enfado, que vuelvan y acaben de contarme lo de Chiva.

---Yo le enteraré mejor que ellos---dijo Sanchico.---Yo estuve también
en Liria y Chiva; formé en el cuadro de los fusilamientos, y puedo
asegurar que no matamos a Francisquín. En el camino de Chiva se nos
perdió, bien porque lograra escapar, bien porque algún amigo le
amparase. Matamos a los prisioneros en el patio de un convento, después
de desnudarles. Luego, los que tenían gusto para estas cosas y mala
entraña, se entretenían en quemarles los bigotes cadavéricos y en
pegarles cuchilladas\ldots{}

---¡Qué espanto! ¡No puedo oír esto!---murmuró D. Beltrán\ldots---¿De
modo que el pobre Francisquín\ldots?

---Bien pudo ser que estuviera entre los que quedaron para otro día.
Nosotros seguimos con D. Ramón, que dio una batalla al general Palarea,
en la cual no salimos bien. Nos retiramos ordenadamente hacia Liria. Sé
que en Villar del Arzobispo fusilaron \emph{el sobrante} de Chiva, menos
unos cuantos que fueron llevados prisioneros a Beceite y de allí a
Cantavieja. Tengo por muy probable que entre esos esté Francisquín Luco.

---Dime, Sanchico---preguntó Baldomero.---¿Estuviste tú en lo de
Alcotas? Porque allí pasaron por las armas a un primo mío, cabo primero
en el regimiento de Ceuta.

---Aquel día estaba yo en Torrijas, a donde se nos mandó para pegar
fuego al pueblo, después de fusilar al alcalde porque no suministró las
raciones que se le pidieron. Al volver al Cuartel general supe lo de
Alcotas. Fue que a D. Ramón le llevaron el soplo de que estaban allí los
de Ceuta\ldots{} Corre allá: los de Ceuta habían salido del pueblo; les
sigue, les alcanza, les envuelve.

---Capitularon cuando se les concluyeron los cartuchos\ldots{} Así lo
oí\ldots{} Y el tigre les dio palabra de respetar las vidas.

---Pues el no cumplir fue porque el Padre Escorihuela llevó el cuento de
que los de Ceuta habían hecho el entierro de Cabrera, en chanza,
cantándole responsos por las calles de Alcotas, y que en la iglesia
hicieron burla de los santos. Como D. Ramón tenía el alma requemada por
lo de su madre, les mandó fusilar. Eran ciento cuarenta y cinco.

---Les confesarían antes---dijo Urdaneta, que había recobrado su actitud
de momia egipcia, y adormecía su pensamiento en una resignación
filosófica no exenta de humorismo.

---El mismo Padre Escorihuela que le contó al General las picardías de
los capitulados, se puso a confesarles de prisa y corriendo. Pero como
D. Ramón quería llegar de día a Manzanera y no sobraba el tiempo, no
confesaron más que los oficiales\ldots{} los soldados no.

---Dime tú, Sanchico---preguntó D. Beltrán inmóvil.---Cuando pasaban
esas cosas, ¿no caían del cielo rayos y centellas que hicieran polvo a
ese padre Estercolera, o como quiera que se llame?

---De eso de caer rayos nada sé: yo no estaba presente, señor. Mi
partida se incorporó a Quílez, que nos llevó a tierra de Monreal, cerca
de Daroca, donde derrotamos a los Voluntarios de Soria, mandados por
Valdés.

---¿Y a cuántos fusilasteis?

---Cayeron treinta y tres Oficiales y diez miñones.

---Bien, hijo, bien. ¿Y hay todavía humanidad, género humano quiero
decir, en esa condenada tierra?

---Fuera de los que combaten, señor, por ver quién reina, hombres,
ninguno hay; mujeres y caballerías, pocas.

---Ahora que hablamos de mujeres: mi amigo y protegido Juan Luco, además
de sus tres hijos varones, tenía una hija.

---Que es monja penitente; no sé\ldots{} De esto le noticiará Joreas,
que, como de Rubielos, conoce a toda la familia\ldots»

Diciendo esto, Sanchico miraba con recelo a un hombre que entró a dar
pienso a dos caballerías. A la mortecina luz del candilejo que alumbraba
la anchurosa cuadra de negro techo festoneado de telarañas, apenas se
distinguía el rostro del tal sujeto; pero el chico debía de conocerle y
temerle, porque al verle pasar cerca, en dirección de una de las
puertas, se tiró boca abajo sobre la paja, haciéndose el dormido. Pasado
el susto, el muchacho se incorporó diciendo: «Es mi padre, José Sancho,
que anda al servicio de un señor italiano, muy rico y principal. Llegó
esta mañana, y cuando le vi no supe dónde meterme, de la vergüenza que
me daba\ldots{} y del miedo, porque mi padre, al saber que yo me había
ido a la facción, dijo que si no me mataban en la guerra, me mataría él
cuando me encontrase, por haberle deshonrado\ldots{} que a deshonra le
sabe el ver a un hijo suyo debajo de la bandera de Carlos V.»

\hypertarget{iv}{%
\chapter{IV}\label{iv}}

Ya tenía D. Beltrán la palabra en la boca para pedir más referencias de
aquel señor extranjero, cuyo nombre y diplomático carácter no le eran
desconocidos, cuando se armó un gran tumulto al otro lado de la cuadra.
Empezaron peleándose dos, se enredaron luego cuatro, dándose morradas y
coces; la querella habría pasado quizás a mayores, si no intervinieran
Baldomero Galán y dos sargentos que a la sazón entraron, los cuales,
sacudiendo de plano, y deshaciendo a tirones el racimo que formaban los
contendientes, restablecieron el orden. A unos les hicieron salir, a
otros arrojáronles sobre la paja, y ya no se oyó más que el resoplido de
las cóleras sojuzgadas. «Es la de todos los días---dijo Baldomero
volviendo al lado de D. Beltrán,---la cuestión entre \emph{Cabreristas}
y \emph{Nogueristas}. Unos dicen y sostienen que la madre de Cabrera
estuvo bien fusilada, como castigo de ese tigre sanguinario, y otros que
no, que el haberla matado sin culpa de ella ha traído esta situación tan
\emph{fratricida}. Ya les hemos aplacado los humos; y como repitan, se
mandará dar un recorrido de palos, para que callen y nos dejen en paz.

---Y ahora, señor, que tenemos algún sosiego---dijo Saloma,---haga por
dormirse, que ya es tarde, y todos necesitamos cobrar fuerzas para el
ajetreo de mañana.

---Procuraré seguir tu sabio consejo---replicó el anciano, tomando
postura cómoda y cubriéndose bien de nariz para abajo.---Pero dudo que
pueda coger un buen sueño, pues ahora me doy a cavilar si ese señor
italiano será o no será quien yo me figuro: uno que de Madrid y Nápoles
fue comisionado al Cuartel de D. Carlos para tratar de un arreglo que
pusiese fin a estos horrores. No me acuerdo del apellido de ese sujeto,
pues ya no hay nombre que quiera guardarse en la jaula deshecha de mi
memoria; pero me da el corazón que es el mismo de quien tuve noticia por
cierto caballerito que conocí y traté caminando hacia Villarcayo. Lo
primero que has de hacer mañana es llegarte a Sancho y sonsacarle todo
lo que de su señor quiera decirte: te informas de si va para Zaragoza, o
para Levante, pues en este caso me convendría su amistad, que de seguro
irá el hombre bien pertrechado de pasaportes. Y no sería malo que tú,
tan despabilada y francota, te fueras a él, metiéndote en su cuarto, si
es que lo tiene, con el pretexto de saber cuándo se va para ocuparlo yo,
y una vez metida le dijeses quién soy, y como me veo en estas
estrechuras impropias de mi nobleza\ldots»

Prometiole la hermosa navarra conquistarle al italiano, y a toda la
Italia si fuese menester; y en aquel punto, Galán, que había salido a
recorrer los alojamientos de los soldados, volvió diciendo que corría
por el pueblo el notición de la muerte de Cabrera. Sobre esto hicieron
los tres comentarios prolijos, conviniendo en que si resultaba cierto,
sería gran merced de Dios, apiadado al fin de la pobre España. Y ya no
pensaron más que en dormir lo que pudiesen, cosa no fácil, por los
ruidos que a cada instante en el ancho local se levantaban, así de
inquietudes de animales como de personas, y por los feroces ronquidos de
algunos durmientes. Pudo vencer D. Beltrán la molestia que estos le
causaban; y cuando ya iba cogiendo el sueño, le despabilaron las voces
de un condenado hombre que, sentado en el suelo, en postura turquesca,
junto a la pared, solo, parecía rezar en alta voz con plañidera
monotonía desesperante.

«¿No podríamos conseguir---dijo D. Beltrán entre suspiros,---que ese
demonio de hombre se fuese a rezar a la calle? Si se va por una peseta,
dásela, Saloma.

---Es el pobre Muel---dijo condolido Galán,---que de ver morir a tres de
sus hijos, fusilados en Alventosa, se ha vuelto loco, y se pasa la vida
predicando por estos caminos en canto llano.

---Alventosa\ldots{} ya sé\ldots{} es en tierra de Rubielos. Alguna de
las propiedades que vendí a Luco allí están\ldots{} Creo que fue un
espanto la matanza que ordenó y ejecutó ese bribón del cura Lorente.

---Fusiló setenta y siete hombres y un niño de diez años, hijo de un
capitán. Eran del regimiento de Extremadura, donde yo he servido. Les
cogieron el Royo y Peinado en Arcos; les llevaban prisioneros, y el
capellán Lorente propuso fusilarlos. Los dos cabecillas no querían; el
clérigo, a fuerza de ruegos y amenazas, consiguió que mataran veintidós.
Al siguiente día, en ese pueblo de Alventosa, volvieron a cuestionar
sobre si mataban o no a los demás: Lorente, que sí; Peinado y Royo, que
no. En un descanso, el capellán mandó destapar un barrilito de
aguardiente que llevaba. Bebieron, y con la borrachera, el Royo se puso
de parte de Lorente. Salieron los vecinos del pueblo con su párroco a la
cabeza, y de rodillas imploraron la vida de los desgraciados
prisioneros. Lorente le dijo al párroco: «Confiéselos ahora mismo; y
para acabar más pronto, yo empiezo a confesar por una punta y usted por
otra.» Negose el cura de Alventosa, y se echó a llorar\ldots{} El
capitán pidió entonces a los cabecillas que no matasen al niño; pero
para más crueldad, fusilaron primero a la criatura, por que el padre lo
viese, y luego a este y a todos los demás después de desnudarlos\ldots{}
Al ponerse en marcha, Lorente dijo al cura de Alventosa que, so pena de
la vida, dejara los cuerpos insepultos para escarmiento de las tropas
cristinas que pasasen\ldots{}

---¿Y no ha habido un hombre honrado, valiente y justiciero---dijo D.
Beltrán, dando un salto en su lecho;---no ha habido un hombre, un
aragonés, que haya cogido a ese vil clérigo, a ese sacrílego, y le haya
colgado vivo, por las patas, de la más alta rama de un alcornoque, o del
campanario de una iglesia, para que se lo comieran los buitres?\ldots{}
Desconozco a mi raza\ldots{} esto no es Aragón. Si yo fuera mozo,
créanlo, iría a esa guerra, no para defender ambiciones y derechos de
reyes más o menos legítimos, sino para perseguir y castigar tan salvajes
crímenes, para vengar a Dios de los ultrajes que unos y otros le
infieren; sería implacable con los cobardes asesinos de uno y otro
bando, llamáranse Nogueras, llamáranse Cabrera, y vengaría a la madre de
este, y a la esposa de Fontiveros, y a todos esos infelices sacrificados
con barbarie tan horrenda y estúpida.

---Está muy bien, señor---le dijo Saloma, cogiéndole de los brazos para
hacerle acostar;---pero sosiéguese y no se desabrigue, que puede coger
una pulmonía.»

No había medio de aplacarle; de rodillas sobre la paja, apoyaba con
enérgico ademán su ardiente protesta: «No, no puedo sosegarme oyendo
estas cosas. Esto no es Aragón, esto no es mi raza, la raza justiciera
por excelencia, fuerte y benigna, guerrera y cristiana, iracunda y
generosa\ldots{} ¡Y ese pobre hombre es víctima de este furor de
matanzas! ¡Y ha perdido la razón viendo cómo los hombres se vuelven
maestros de las fieras en la crueldad!\ldots{} Ven acá tú, buen amigo, y
hallarás aquí un corazón aragonés compasivo, no más que compasivo, pues
que la vejez no permite otra cosa\ldots{} Ven acá, y nos consolaremos
todos los buenos, abominando de los que pisotean la justicia humana y
remitiéndolos a la divina.»

El otro infeliz, oyéndose llamado, acudió allá con paso lento. Era un
hombre de aventajada estatura, flaco, de tez tan morena, que a la escasa
luz de la cuadra parecía negra; el pañizuelo liado a la cabeza; el
cuerpo cubierto de un luengo camisón, sin faja; los pies desnudos,
negros también, como la cara, como las manos, semejantes a manojos de
sarmientos; todo él perfecto plagio de un santón árabe. Al aproximarse,
venía rezando en alta voz, y una vez junto al grupo soltó esta
terrorífica declamación con duro y ronco acento: «No te salvas, no te
escapas, malvado Lorente, aunque te escondas entre pajas, teniendo por
guardianes, por los pies a tu Rey y señor, y por la cabeza a la Reina de
tu Iglesia maldita\ldots{} No te escapas ya, clérigo de Satanás,
serpiente, que mis ejércitos rodean ya toda esta fortaleza, y no
hallarás puerta ni hendidura ni resquicio por donde puedas
escabullirte\ldots{} No morirás, no\ldots{} Con el zumo de unas hierbas
que hay en la torre de Pepo, nada más que allí, se te untará todo el
cuerpo, y vivirás mil años, ¡mil años! infame Lorente; y en todas las
partes de tu persona, pecho, espalda, muslos, barriga y lo demás, te
nacerán, por la virtud de aquella hierba, ojos, ¡ojos como los de la
cara! que vean, y delante de cada uno de estos ojos se te pondrá un
fusilado para que lo estés viendo día y noche\ldots{} Y horrorizado de
lo que ves con tantos ojos, querrás descansar y dormir; pero no podrás,
no podrás, porque esos ojos no duermen, ni pestañean, ni lloran, y los
tendrás siempre bien abiertos y despabilados, mirando con cada uno de
ellos a un fusilado por ti\ldots{} y así estarás mil años, trescientos
sesenta y cinco mil noches y días\ldots{} Luego se te dejará otros mil
años ciego y sordo, para que veas dentro de tu conciencia, y se te
quitará la razón para que no puedas arrepentirte ni confesarte\ldots{} y
se te pondrá una lengua venenosa para que blasfemes a todas horas, y se
te secará el agua de lágrimas para que no puedas llorar ni
afligirte\ldots{}

---Basta, basta ya---dijo D. Beltrán horrorizado\ldots---No tanto, pobre
Muel\ldots{} Es demasiado castigo, infinitamente mayor que la
culpa\ldots{} Perdóname ya.

---Todavía no, todavía no\ldots{} Otros mil años disparándote a cada
minuto por el oído izquierdo un tiro de fusil con bala, la cual, después
de retumbar dentro de tu calavera, saldrá por el oído derecho sin
matarte\ldots{}

---No más, no más, Muel\ldots{} Perdón, perdón.

---Otros mil años\ldots{}

---No, no\ldots{} Baldomero, quítame de aquí a ese hombre\ldots{} Por
Dios te lo pido.»

Suavemente le cogió de un brazo Galán y se lo llevó sin que hiciera
resistencia, pues su locura era pacífica; inocente en las acciones,
desbordada en las palabras. Día y noche se le oía la perorata cadenciosa
y lúgubre: arengaba a sus imaginarias tropas, vencía y aprisionaba a
Lorente; llevábale arrastrando por valles y montes hasta la torre de
Pepo; encerrado allí el vencido monstruo, le imponía los sutiles
castigos por series de mil años, hasta que, cansado de inventar
horrores, volvía a los de la realidad y a la tragedia de Alventosa.
Había sido maestro de escuela y diestro pendolista; no pedía limosna,
comía lo que le daban; dormía en despoblado, o bajo techo si se lo
permitían, y vagaba en un radio de cinco leguas alrededor de Quinto, su
patria. Echado al corral por Galán, volvió este al lado del señor, a
punto que Saloma, vencida del cansancio, cerraba los ojos y hacía
reverencias. Durmiose al fin, apoyada la cabeza en la pared, y el prócer
y Baldomero siguieron charlando en voz baja de cosas de guerra y
política hasta que oyeron el diligente estridor de la diana, que,
avisando a todos el fin del sueño, fue principio del de D. Beltrán, el
cual, por añeja costumbre, dormía las mañanas.

\hypertarget{v}{%
\chapter{V}\label{v}}

Cuando el pobre anciano despertó, después de dar a sus huesos algunas
horas de plácido reposo, contáronle sus amigos las novedades ocurridas
en el parador durante su sueño. Había conseguido Galán reconciliar a
Sanchico con su padre Sancho, no sin que este se mostrara largo rato
rebelde a las paces, haciéndose el inflexible con desmedida afectación,
hasta que, desahogando su severidad en una descarga de bofetadas, lloró
el chico, se aplacó el padre, y todo quedó perdonado, a condición de que
el joven partiese aquel mismo día para Ablitas y no volviese a separarse
de sus tíos. En la ruidosa querella de hijo y padre, salió a relucir que
Sanchico se había largado a la facción por contrariedades lastimosas de
amor. Entre tirarse al Ebro y hacerse faccioso para que una bala le
matase, prefirió esto último. El cuento fue que las balas no se metieron
con él, y que el trajín de la guerra le curó de la morriña que le
enfermaba el alma. Volvía, pues, mejor de lo que fue, saludable, fuerte,
aleccionado del mundo, y habiendo visto sucesos mil, lisonjeros o
desgraciados, que servían de grande enseñanza. Por lo demás, su afecto a
la causa de D. Carlos había sido puramente circunstancial, y lo mismo le
importaban a él los derechos del Rey legítimo que la carabina de
Ambrosio. Cuando Urdaneta supo que Sancho iba para la Ribera, ordenó que
se fuese con él uno de sus criados: se arreglaría sólo con Tomé; que los
tiempos eran apretados, y había que mirar por la economía.

Pero la gran novedad de aquella mañana fue que la gentil y desenvuelta
Saloma logró avistarse con el italiano, sorprendiéndole en su cuarto
cuando daba la última mano en su retoque personal. Desempeñado había con
extraordinaria agudeza el encargo que le confirió D. Beltrán, ganando,
si no la confianza, las atenciones de aquel señor. Por las referencias
de Saloma y el nombre del criado, se afirmó Urdaneta en que el tal no
era otro que el siciliano de que Fernando Calpena le habló,
intermediario clandestino entre las dos ramas borbónicas que se
disputaban el Trono. Toda la madrugada, hasta que se durmió, había
estado el prócer devanándose los sesos por recordar la gracia de aquel
sujeto. Su memoria era ya para los nombres un verdadero caos. Mas cuando
Saloma le contaba su entrevista, se le metió súbitamente en el cerebro a
D. Beltrán el perdido nombre, y gritó: «¡Rapella, Rapella! Ya me
acuerdo. En la punta de la lengua lo tenía.» Díjole por fin la navarra
que el señor extranjero se alegró mucho al saber que en el propio
parador se hallaba persona de tan alta alcurnia, a quien conocía de fama
por sus amigos de Madrid, y que deseando el honor de tratarle, le
invitaba a almorzar.

«¿Ves?---dijo Urdaneta con alborozo, dando pataditas en el portal para
entrar en calor.---Tú me has traído la suerte, pues yo venía con mala
pata, y desde que te encontré, todas las cartas me salen buenas.»

Antes de la hora del almuerzo juntáronse el viejo aristócrata y el
pintado diplomático en la calle, y cambiando mil finuras, hablaron
después cuanto les dio la gana, sin parar hasta que terminó el
comistraje. Hizo gala Rapella de su cortesanía, y derrochó sin tasa el
énfasis de su especial oratoria familiar. Aseguró a D. Beltrán que le
conocía por lo que de él le habían hablado sus grandes amigos Bernardino
Frías, Luis Córdova, Paco Malpica, Martínez de la Rosa, Quintana y
otros. Hablaron luego de Fernando Calpena, mostrándose Rapella muy
gozoso de saber que vivía, pues ya le consideraba muerto; y por fin se
eternizaron en el comentario de las cosas políticas y militares, la
revolución de la Granja, las nuevas Cortes, la situación política en
Madrid y en la corte carlista, las intrigas de una y otra parte,
Espartero, Cabrera, las expediciones de Gómez, D. Basilio y
Batanero\ldots{} el buen giro de la guerra en el Norte, el mal cariz de
la misma en el Maestrazgo.

Por más empeño que en ello puso, no pudo el viejo conseguir que Rapella
se clareara en lo de las misiones y recados que traía y llevaba de corte
en corte. Se escabullía gallardamente de todas las trampas que el otro
le armaba con capciosas preguntas. A veces la agudeza de D. Beltrán le
cogía en contradicción. Dijo primero que iba hacia Vinaroz, donde le
aguardaba un barco que debía llevarte a Nápoles; después indicó que el
objeto de su viaje en tal dirección era sólo avistarse con su íntimo
amigo Borso di Carminati, para darle un abrazo y pasar unos días con él.
Tenía en el ejército del Centro excelentes amigos, entre ellos su
paisano Cialdini, muchacho de gran porvenir, ayudante de Borso. Inútil
fue también el empeño que puso D. Beltrán en sonsacarle noticias y
cuentos de las interioridades del Cuartel de D. Carlos\ldots{} Nada: el
siciliano no daba lumbres. Y si no su locuacidad perdía un poco de su
finura cuando el otro quería llevarle a cierto terreno, apartándole de
los temas que él elegía, siempre vagos, de generalidades y lugares
comunes. Por fin llevó la conversación a la persona y hechos de Cabrera,
de quien se mostró admirador, sosteniendo que era ya vulgaridad insigne
tenerle por uno de tantos cabecillas, notable sólo por su inquietud y
ferocidad. Desde que apareció en la guerra, conmoviendo y abrasando el
país como fuego del cielo, mostrose gran caudillo, tan buen conocedor
del suelo como de los hombres, táctico y estratégico de primera, audaz,
incansable, heroico; y por entre estas cualidades apuntaba ya un gran
político.

«¡Oh, no tanto! ¿Ya quiere usted hacer de él un Napoleón?

---Un Napoleón de montaña, amigo mío.»

Respecto a las tan cacareadas crueldades del jefe carlista, dijo Rapella
que habían sido estrictamente de carácter disciplinario militar hasta
que los cristinos derramaron con bárbara torpeza la sangre de María
Griñó. El asesinato de una mujer, sin más delito que ser madre de
Cabrera, creó nueva ordenanza militar, dando una infernal lógica las
horrendas carnicerías consumadas por uno y otro ejército. Fuera de esto,
para abrirse camino el travieso bigardón de Tortosa, y pasar en breve
tiempo de seminarista pendenciero a caudillo y gobernador de hombres en
los campos de batalla, no podía menos de emplear, como resorte de
dominio, el terror, la fiereza y la brutalidad. No se había formado
dentro de un organismo, sino que tenía que sacar el organismo del caos
social, y esto no se hace sino desplegando desde los primeros momentos
un genio implacable, aterrador, extraordinaria viveza para aplicar
justicias rápidas, de moral severa y primitiva; haciendo sentir el peso
de su mano antes de que pudiera discutirse el derecho con que la
levantaba. En las guerras civiles, los hombres culminantes nacen así, o
no nacen nunca.

No le parecieron mal a Urdaneta estas razones, y como sacara a relucir
la especie, muy corriente en aquellos días, de la muerte del famoso
guerrero, negola el siciliano, sosteniendo que había, sí, corrido
grandísimo peligro en los últimos días de Diciembre; pero que estaba
vivo, aunque al parecer no muy sano. En Septiembre del año anterior
habíase unido Cabrera en Utiel a la expedición de Gómez. Juntos
recorrieron Cuenca, Albacete, la Mancha, Andalucía y Extremadura\ldots{}
Si las tropas cristinas que les perseguían no pudieron deshacerles,
tampoco ellos lograron su intento de sublevar las comarcas que invadían.
Un correr continuo; exacciones y rapiñas en ciudades y aldeas; aislados
lances de guerra sin plan ni concierto, gloriosos unos para los
liberales, como el de Villarrobledo, ventajosos otros para los
carlistas, pero sin que de ninguno resultara el aniquilamiento de la
expedición, ni tampoco su triunfo; tal fue la obra combinada de Cabrera
y Gómez, caracteres antitéticos, de cuya unión no podía resultar nada
eficaz. La falta de engranaje entre uno y otro temperamento militar fue
marcándose en desavenencias, luego en discordias, y los dos cabecillas,
que juntos no podían formar una cabeza, riñeron al fin, a la vuelta de
Cáceres, campando cada uno por sus respetos. Cabrera se escabulló fugaz
y resbaladizo por el caminito que creyó más seguro para volver a sus
riscos y barranqueras del Maestrazgo, donde en su ausencia las cosas de
la guerra no iban muy prósperas, y amenazaba desbaratarse lo que él con
paciencia, rigor y firme mano organizado había.

Lo primero que intentó al pisar su terreno fue pasar al Cuartel general
de D. Carlos en el Norte, para dar cuenta a este de la desavenencia con
Gómez y proponerle un nuevo plan de campaña en el Centro. Llegose al
Ebro, eligiendo el vado de Rincón de Soto como el único que en aquella
estación cruda era practicable; pero le salió mal la cuenta, porque fue
sorprendido por la columna de Iribarren, que le deshizo, matándole
muchos hombres y dispersándole los que quedaron con vida. La suya estuvo
en gran peligro. Acribillado de balazos, quedó al amparo de la
obscuridad junto a una pared, donde le recogió uno de los suyos, el
cabecilla que llamaban \emph{La Diosa}, y le llevó atravesado en una
caballería, como un saco, pues montar no podía. Perseguido por las
tropas de Iribarren, debió su salvación a un cura que le escondió en el
sótano de su casa; allí pasó largos días y noches entre la vida y la
muerte, hasta que, mejorado de sus heridas, le trasladaron a un abrupto
monte, espesura más propia de lobos que de seres humanos, donde
permaneció en escondite, recobrando poco a poco la sangre perdida, y con
ella el brío y la ferocidad. De este apartamiento provino la noticia de
su muerte, que corrió por toda España, descorazonando a los suyos, y
llenando de tristeza y confusión a todo el carlismo de aquende y allende
el Ebro; pero ya en los últimos de Enero (como unos quince antes de la
fecha en que esto se relata) se supo a ciencia cierta que vivía, y que
sin reponerse de sus heridas y enfermedades, preparaba nuevas correrías
por la Plana de Castellón y riberas del Turia: que en tal hombre la
ociosidad era imposible, mientras alguna vida le quedase. Cuando esto
narraba el señor Rapella, no podía decir fijamente dónde se hallaba el
famoso caudillo; presumía que, medio muerto o medio vivo, recogía sus
fuerzas, las reorganizaba, lanzándose al terreno que la Naturaleza
parecía haber amoldado a la hechura intelectual y física del que bien
podía llamarse, si no el león, el gato montés de la guerra.

«A fe mía---dijo D. Beltrán,---que está usted bien informado. Ya cuidará
de decir a su amigo Borso que se ande con tiento, pues este mozo no es
de los que fácilmente se dejan destruir y aniquilar.»

Por lo que a renglón seguido hablaron, comprendió el buen Urdaneta que
en los cálculos de su flamante amigo no entraba el llevarle en su
compañía, aunque en ello tuviera gusto, como se dejaba traslucir de lo
que manifestó con exquisita urbanidad y palabras equívocas. Delicado en
extremo, y muy ducho en artes mundanas, dio a entender D. Beltrán que
los fines de su viaje exigíanle también ir solo, sin más acompañamiento
que el de sus criados; manifestación que puso en gran cuidado al otro,
recelando que llevase también misión diplomática, quizás como apoderado
o mensajero del patriciado aragonés. Pero no atreviéndose a entrar en
explicaciones, cada cual, como de zorro a zorro, se encerró en su
discreción, preparándose para continuar su caminata. D. Beltrán partiría
con la columna que a la sazón estaba en Fuentes, y a que pertenecía
Baldomero; D. Aníbal aguardaba otra fuerza que llegaría por la tarde,
mandada por un coronel, íntimo amigo suyo.

Apercibiéndose para la partida, preguntó Galán a su antiguo señor que de
dónde había sacado el hermoso caballo que traía, el cual, mientras Tomé
lo limpiaba en el corral, era objeto de la admiración y curiosidad de
todos los allí presentes. Replicó Don Beltrán que había ganado aquella
joya en una donosa y feliz apuesta; sin dar pormenores del caso, mandó
venir a su presencia a los dos escarmentados Joreas y el
\emph{Epístola}, y en un poyo del portalón les interrogó acerca de los
hijos supervivientes del desgraciado Juan Luco. De Francisquín nada
sabían a ciencia cierta; de su hermana, monja profesa en el Monasterio
de Sigena, a cuatro leguas de Sariñena, dio el \emph{Epístola} informes
más concretos. Había despuntado Marcela, desde su entrada en religión,
por su ciencia grave y su lúcido ingenio; sabía latín, y dándose a la
lectura, lo mismo platicaba de teología que enjaretaba versos y prosas
en loor de los sagrados Misterios.

«Hace tiempo---dijo D. Beltrán,---que a mí llegó la fama, no sólo de su
santidad, sino de su vivo entendimiento.

\hypertarget{vi}{%
\chapter{VI}\label{vi}}

---Me contaron---añadió Joreas,---que otra más leída y escrebida no la
hubo nunca en aquel sacro monasterio, más antiguo que las Tablas de la
Ley, pues lo hicieron en cuantico que empezó la cristiandad, hace unas
docenas de miles de años. Oí que Sor Marcela pasmaba a todos con sus
latines hablados por gramática, y que a verla iban el arcipreste de
Mequinenza, el abad de Veruela y muchos calonges y prestes de Huesca,
Tarragona y hasta de Aviñón, que es la Roma de esta parte de Francia.

---Me consta---dijo el \emph{Epístola},---porque lo he visto y leído en
parte, que escribió un lindo poema sobre el milagro de los Corporales de
Daroca, y también conozco unas quintillas a la Transfiguración del
Señor. Sé que de diversas partes iban personas eruditas a consultar con
ella puntos graves de moral, de filosofía o de religión, y que el meollo
de sus sentencias era el asombro de cuantos la oían. En el monasterio,
con ser ella de las monjas más jóvenes, considerábanla como autoridad, y
como a vieja la respetaban. En los principios de la guerra, dicen que
llamó a D. Ramón para iniciarle a no emplear medios de crueldad, y lo
mismo hizo con Nogueras. El general Mina la visitó, y también fueron a
platicar con ella en el locutorio Masgoret y Tristany. Pero el año que
acaba de pasar, allá por Septiembre, si no recuerdo mal, cuando Maroto
vino a mandar en Cataluña, que más valía que no viniera, la partida de
Llarch de Copons y la de otro cabecilla que llaman \emph{Camas-Crúas},
bajaron huidas de la parte de Lérida, donde Gurrea les pegó de firme;
tomaron la vuelta de Benabarre y Albalate para pasar el Cinca, y con el
furor que traían cometieron mil desmanes, saqueando las aldeas y
arrasando cuanto encontraban. Incendiados por estos bárbaros el claustro
alto y aposentos capitulares de Sigena, salieron dispersas las señoras
monjas, como las abejas cuando les ahúman la colmena. Cada religiosa
tiró por su lado, buscando el amparo de otros conventos o de casas
honestas; y Sor Marcela, a quien se creyó muerta o extraviada, apareció
en una ermita solitaria de la Sierra de los Monegros, vestida con un
saco al modo de penitente, el cabello suelto, como pintan a la
Magdalena, sólo que más corto; los pies descalzos, una cuerda a la
cintura; y diz que iba predicando a los pastores y gente rústica para
que se apercibiesen a la guerra en nombre de Cristo, peleando contra los
dos ejércitos, cristino y carlino, según ella legiones de Satanás, que
quieren dominar la tierra y establecer el imperio de la injusticia.

---¡Vaya con la sabia!\ldots---dijo D. Beltrán.---Pues no me parece
descaminada su locura, o más bien, creo que debajo de ese desvarío se
esconde la misma discreción\ldots{} Y díganme ahora, señores
escarmentados: ¿qué tal cariz tiene la monjita? ¿Es su rostro de buen
ver? Su facha y apostura, ¿responden a la hermosa raza de los Lucos?

---Señor---dijo el \emph{Epístola} con extremos de admiración,---es
mujer de tanta gallardía y belleza, que aun con aquel desavío de
penitente, da quince y raya a las señoras más bien aderezadas. Y no diré
yo que el empaque de santidad a lo anacoreta, como figura de retablo, la
desfavorezca, que más bien me inclino a creer que su traje, al modo de
mujer de la Biblia, hace lucir más todo aquel contorno de cuerpo que no
tiene semejante, pues no ha visto usted escultura que pueda
comparársele.»

En esto se alejó el \emph{Epístola}, llamado por sus amigos, y Joreas
hubo de completar las informaciones con un dato, que apuntó en la forma
más descarnada y picante: «Este bribón de \emph{Epístola} se calla lo
mejor del cuento, señor, y es que, habiendo encontrado sola a la Marcela
en un camino junto al Pueyo, la requebró de amores, uniendo a las
palabras de solicitación las acciones atrevidas. Pero no contaba con el
geniecico de la que él llama estatua de bulto. Arreole Doña Marcela tan
fuerte bofetada, que le tiró al suelo, y cuando pataleaba para
levantarse, con un madero, que unos dicen era cruz y otros una tranca,
le dio tales golpes en la cabeza, que, si no acuden a la defensa del
chico los compañeros que por allí cerca andaban, la santa habría dado
cuenta del \emph{Epístola} y del mismo \emph{Evangelio}, si así se
llamara este pillo.

---¿Qué me cuentas? ¡Sobre la sabiduría, ese tesón, ese poder!\ldots{}
Vamos, que ya rabio por conocer a ese prodigio; y si no tuviera
precisión de verla para que me informe de ciertos asuntos de su padre
que me interesan como los míos, sólo por apreciar sus méritos, y
admirarlos en lo que mi corta vista me lo permita, iría en su busca.»

Lo último que dijeron Joreas y el \emph{Epístola}, al despedirse para
continuar hacia Zaragoza, fue que la Marcela penitente andaba por
aquellos meses en el Desierto de Calanda o en tierra de Alcañiz. Observó
Don Beltrán, al quedarse solo reflexionando en lo que veía y oía, que
desde que llegó a Fuentes de Ebro todo le anunciaba la entrada en el
reino de lo excepcional y maravilloso. Nada era ya común ni vulgar.
Personas y cosas traían la impresión de un mundo trágico, el cuño de una
poesía ruda y libre, emancipada de toda regla. No sentía más el buen
señor que ser tan viejo y andar tan mal de la vista: que si él tuviera
treinta años menos y sus ojos bien listos, había de serle muy grato el
ver y tocar de cerca un mundo que de modo tan peregrino quebrantaba las
rutinas sociales. También le contrariaba mucho su escasez de dineros;
mas como los fines de su viaje no eran otros que proveerse del precioso
metal, a quien amaba más que a las niñas de sus perdidos ojos, la
esperanza de alcanzarlo y poseerlo le alentaba.

Salió en su hermoso caballo, marchando a retaguardia de la columna, y
gran parte del camino fue al estribo, si así puede decirse, del carro en
que con una señora capitana y otras dos mujeres iba Salomé Ulibarri; y
por no desmentir su índole caballeresca y hábitos de sociedad, no cesó
de entretener a las cuatro hembras con frases galantes, de refinada
gracia sin faltar a la decencia, y a todas festejaba por igual
llamándolas hermosas, sin distinguir entre la belleza de la mujer de
\emph{Mero} y la fealdad repulsiva de la capitana, entre la desabrida
juventud de la tercera y la vejez de la cuarta. Pero como él no veía
bien, todas le parecían iguales, y por no haber allí género más noble y
elegante, tratábalas como a damas de alta educación. Por dicha, la
columna no encontró facciosos en el camino, y el viaje fue de los más
felices, fuera de las molestias, del hambre, polvo y frío, que alguna
tarde y mañana se dejó sentir, llegando el buen señor bastante molido a
la ciudad del \emph{Compromiso}, la noble Caspe.

Constante la fortuna en favorecer al caballero, encontró este en la
histórica ciudad a su antiguo amigo D. Blas de la Codoñera, que allí era
de los más pudientes, propietario de tierras y montes, padre de numerosa
familia. Llevole a su casa, y le aposentó como a tan insigne caballero
correspondía, tratándole a cuerpo de rey. Mucho agradecieron los
asendereados huesos del buen Urdaneta la blandura de aquella cama, tan
grande como la Colegiata, y las suculentas comidas y cenas con que le
regalaron. Aún estaba la familia de luto por la muerte del hijo mayor,
uno de los urbanos que fusiló Cabrera cuando entró a saco la ciudad en
mayo del 35. La señora y señoritas de Codoñera no se hallaban exentas de
la rudeza baturra: su habla carecía de finura; su educación, perfecta en
lo moral y religioso, era muy rudimentaria en lo social. Con todo, D.
Beltrán se hallaba en tal compañía muy a gusto, y se desvivía por
corresponder con su exquisita urbanidad a los obsequios de la hidalga
familia. Había sido el D. Blas constitucional templado hasta el día
funesto de la entrada de Cabrera; pero desde tal fecha se trocó en
furibundo \emph{patriota}, enemigo acérrimo del obscurantismo y de las
antiguallas que quería traernos D. Carlos. En la exacerbación de su
sentimiento liberal, que ya era insano, llegaba hasta la impiedad y el
volterianismo, abominando de la hipocresía, de la piedad extremada y
hasta de las prácticas religiosas, con excepción del culto de la Virgen
del Pilar. No pensaba abandonar a Caspe, pues ni él ni su familia tenían
miedo; y como volviera Cabrera con su patulea de ladrones y asesinos, D.
Blas se batiría en la muralla rodeado de sus hijos de ambos sexos: los
chicos bien armados de fusiles, las niñas y la señora bien preparadas
con piedras y ollas de agua hirviendo. Eran los hijos guapos, aunque
abrutados, y tan \emph{liberalicos} como su padre. A todos ellos pidió
D. Beltrán noticias de la monja de Sigena, y los muchachos, que la
habían visto y oído, se dividían en sus opiniones, pues mientras Rafael
sostenía que era una mujer estrafalaria y medio loca, que ocultaba con
las formas de penitencia sus ganas de corretear por el mundo, Pepe la
tenía por hembra superior y de pasmosa virtud, que la distinguía de
todas las gentes de nuestra edad, y a los mismos santos la equiparaba.
Como expresara Urdaneta el firme propósito de ir en su busca, hízole
presente Don Blas el gran peligro a que se exponía viajando por aquellas
tierras; expuso el otro lo inexcusable de su determinación, y,
hallándose en estas conferencias, trajo uno de los chicos la noticia de
que la monja Marcela se hallaba cerca de Alcañiz asistiendo a su hermano
Francisco en una grave enfermedad, con lo cual se le avivaron al anciano
las ganas de ir a donde su interés le llamaba. De nuevo le pintó el
Sr.~de la Codoñera lo arriesgado de tal expedición, maravillándose de
que D. Beltrán hallase gusto en el trato de una monja \emph{retrógrada}
y obscurantista.

«A mí no me hable usted de gente \emph{levítica}---dijo, recalcando esta
palabra, que recientemente había adquirido en la tertulia de la botica
de Cornejo.---Tengo declarada la guerra a esas ideas rancias, tan
contrarias al \emph{espíritu del siglo.»}

Tampoco le gustaba a D. Beltrán la gente levítica; pero sus necesidades
le obligaban a emprender aquel viaje, que felizmente no se alargaría más
allá de Alcañiz. Todo se presentaba favorable al ilustre aristócrata,
pues Borso di Carminati, desde Maella, ordenó que la columna recién
venida se incorporase a las fuerzas acantonadas en Alcañiz.
Disponiéndose Saloma para seguir a su esposo, se lamentaba de no poder
acompañarle en las operaciones, pues había orden deque la impedimenta
\emph{faldamentaria} no saliese de los puntos de guarnición. Despidiose
a la mañana siguiente D. Beltrán de su generoso amigo. Tanto este, como
su esposa, e hijos de ambos sexos, vieron salir con pena y lástima al
noble anciano; y sospechando que tales calaveradas revelaban falta de
seso y desvaríos de la senectud, presagiaban una desgracia. Las señoras
le encomendaron a Dios, y lo mismo hizo Don Blas, pues su aborrecimiento
de lo levítico no le quita el ser buen cristiano.

Muertas de miedo iban Saloma y las otras militaras, y a cada rato creían
oír tiros y ver un nublado de boinas aparecer por los cerros lejanos, lo
que no era absurdo, pues días antes había pasado por allí el Royo de
Nogueruela en dirección a Graus y Benabarre; tampoco andaban lejos
Cabañero, Tena y Maestre. Contrastando con las señoras, Don Beltrán era
todo intrepidez y desprecio del peligro; y en su imaginación de viejo,
reverdecida en la puerilidad, no veía más que bienandanzas. Habiéndole
manifestado Saloma la inquietud con que le veía entrar en el teatro de
tan bárbara guerra, le dijo: «Cuando lleguemos a la gran Alcañiz que,
entre paréntesis, es patria de mi abuelo materno, D. Diego de Paternoy,
almirante de Aragón, señor de las casas y encomiendas de Isún de Basa y
Usé, \emph{etcétera}\ldots{} te contaré por qué voy a donde voy, y por
qué busco a quien busco. Y si ahora supones en mi conducta un desarreglo
del sentido, verás luego en ella la misma cordura\ldots{} Es para mí
cuestión de vida o muerte, de dignidad o vilipendio\ldots{} No creo que
nos salgan partidas; y si salen, ya les sacudiremos. También te digo,
que si es Cabañero el que nos acomete, no temo nada. Le cuento entre mis
mejores amigos, y no había de consentir que me tocaran al pelo de la
ropa.»

A la caída de la tarde entraron en la noble Alcañiz, que desde Roma
viene fatigando a la Historia, ciudad vieja, como un libro de
antigüedades de Aragón y un muestrario de piedras elocuentes. A la luz
crepuscular, los esquinazos góticos y mudéjares parecían bastidores de
teatro, dispuestos ya, con las candilejas a media luz, para empezar el
drama. Resonaban las herraduras de los caballos en el pedernal de las
calles levantando chispas, y el ruido de tambores jugaba al escondite,
sonando aquí, apagándose allá, en los dobleces de la edificación, y
volviendo a retumbar a retaguardia de la tropa. Las plazuelas se unían
por pasadizos, y las calles se retorcían unas sobre otras, obscuras,
ondulantes. Soldados y algunos viejos se veían discurriendo por las
calles; mujeres en algunas puertas\ldots{} Triste y belicosa parecía la
ciudad, como un guerrero herido que se ve forzado a combatir con la mano
que le queda.

\hypertarget{vii}{%
\chapter{VII}\label{vii}}

Metieron a D. Beltrán en una casona llamada \emph{Corte} que hace
esquina con el Ayuntamiento, gótica, de ojivales porches al exterior,
interiormente muy capaz, con ventanas pequeñas, las puertas no muy
holgadas. Allí se alojaban oficiales de distintas graduaciones. Al pasar
por un gran aposento abovedado, donde había gran chimenea encendida con
troncos de encina, a cuyo calorcillo se arrimaban ateridos todos los que
entraban de la calle, vio D. Beltrán, agrupados en torno a una mesa, a
varios oficiales y urbanos de tropa que se engolfaban en el juego,
atentos con alma y vida a las manos del banquero y a las cartas que
lentamente pasaba. Fuéronsele a Urdaneta los ojos hacia la timba, y
subió con ánimo de volver luego, pues vio también que cubrían de
manteles las mesas, como si aquella pieza fuese comedor. El cuarto en
que le pusieron, juntamente con las militaras, no tenía camas; cada cual
se arreglaría con las mantas, alforjas o sacos que llevase. Seis
personas debían repartirse el suelo, que venía como a la medida, sin que
sobrase ni una cuarta.

El cenar fue más difícil operación; y si no se plantan Saloma y la
capitana en la cocina, no les tocara nada de las judías y gachas, que
era lo único que había, con pan moreno y algunas raciones de cecina.
Pero al fin aplacaron su hambre las afligidas damas; D. Beltrán, gozoso
y dicharachero, tratando de alegrarlas con sus galanterías y con
enfáticos elogios de las miserables viandas que comieron. Observó Saloma
que al viejo aristócrata se le iban los ojos a la mesa de los jugadores,
y como ya tomaba confianza con él, se permitió decirle: «Señor D.
Beltrán, noto que mira Vuecencia para el vicio, como si más en él que en
nosotros y nuestra conversación tuviera toda el alma. Pues yo le digo
que sería muy feo que con sus años y su respetabilidad diera el mal
ejemplo de ponerse a tallar o apuntar entre aquellos perdidos. Si así lo
hiciera y se dejara vencer de la tentación del juego, que ha sido la
causa de su ruina, sepa que me enfado, y no le quiero, ni le cuido, ni
le mimo, ni nada.»

Dicho esto a hurtadillas, sin que los demás se enterasen, contestó
Urdaneta en la misma forma, reconociendo el buen juicio que tal
advertencia revelaba, y ofreció no discrepar ni un punto de lo que su
decoro y años le imponían. Si miraba era por observar las caras y ver
quién perdía y ganaba. Antes de levantarse de las flacas mesas hizo
conocimiento, por mediación de Galán, con dos oficiales muy simpáticos,
uno de los cuales se había separado poco antes de la mesa de juego con
los bolsillos totalmente vacíos. Informados de que el señor deseaba ver
y tratar a la monja Marcela, brindáronse a llevarle hasta su presencia,
en el cerro de Santa Lucía, donde a la sazón moraba; ambos la conocían y
habían tenido más de una entrevista con tan extraña mujer, platicando de
cosas de guerra, filosofía y religión, permitiéndose bromear con ella y
echarle requiebros, que Marcela, en la multiplicidad pasmosa de su
disposición y en la riqueza de su entendimiento, para todo tenía una
palabra feliz y oportuna. No se le cocía el pan a Urdaneta hasta que no
llegase la hora de la mañana que los oficiales fijaron para la visita, y
pensando en ella se pasó la noche de claro en claro. Un poquito durmió
el viejo después de amanecer, levantándose con los huesos doloridos de
la dureza de aquellas mal cubiertas tablas. Saloma le preparó un
aceptable desayuno, con huevos y chorizo que afanó como pudo en la
cocina, y a las nueve ya estaba mi hombre junto a la chimenea esperando
a sus flamantes amigos. Sólo uno se presentó, por tener el otro servicio
extraordinario en el castillo, y sin más espera condujo al anciano hacia
la puerta de la ciudad que da al río Guadalope y al grandioso puente.
Fría estaba la mañana, los campos escarchados, el aire empañado por una
niebla que borraba toda visión a regular distancia. Iba D. Beltrán asido
al brazo de su criado, necesaria precaución por la cortedad de su vista,
que con la niebla era casi ceguera total. Pasado el puente, avanzaron
buen trecho por una alameda interminable; y como levantara la bruma, el
teniente hizo notar la gallardía de los desnudos álamos del paseo, y
mirando hacia atrás, la hermosa vista de la ciudad, coronada por el
castillo y ceñida por el Guadalope. Sin enterarse bien, manifestó D.
Beltrán su admiración, pues no gustaba de dar a entender que veía poco.

«¿Con que es usted aragonés?\ldots{} Repítame su apellido, pues ya no me
acuerdo.

---Estercuel.

---¡Hombre, Estercuel!\ldots{} ¿Es usted de Ayerbe?

---Sí señor. Mi padre, D. Celestino Estercuel, administraba los estados
de Ayerbe y de Boltaña; mi tío, D. Bernardino Estercuel, canónigo de
Jaca\ldots{}

---Ya, ya\ldots{} ¿Y usted por dónde me conoce a mí?

---No hay en todo Aragón persona más nombrada y famosa que D. Beltrán de
Urdaneta, a quien pobres y ricos señalan como el tipo de la grandeza, de
la caballerosidad\ldots{} Era yo muy niño y oía contar casos muy
singulares de esplendidez\ldots{}

---A ver, a ver\ldots{} ¿qué casos?---dijo D. Beltrán, risueño y
malicioso, deteniéndose.

---Pues que usted, poseedor de una riqueza incalculable, había mandado
traer de París seis perros de caza, los cuales vinieron cuidados y
asistidos por cuatro monteros y un mayordomo\ldots{} Y un día, siendo yo
muchacho, vi pasar unos trenes magníficos que iban para Canfranc. ¡Qué
sillas de postas, qué caballos, qué galera con provisiones de cama y
boca!\ldots{} Pues mi tío, que entonces era capellán en la casa de
Ayerbe, dijo: «Ahí va D. Beltrán el Grande con los Duques de tal y de
cual\ldots»

---¡Ay, hijo mío!---exclamó Urdaneta melancólico, acelerando el
paso.---Aquellos eran otros tiempos. ¡Lo que va de ayer a hoy!\ldots{}

---Y decía mi padre que sólo en Mora de Rubielos y en la Sierra de
Mosqueruela poseía usted más de diez mil cabezas.

---Sí, sí: muchas cabezas tenía entonces, y ahora creo que ninguna, ni
aun la mía propia. Pues en Mora de Rubielos me resta algo, y aun algos,
que intento recobrar\ldots{} Pero hablar de mí es mirar a lo pasado,
visión triste; alegremos nuestro espíritu hablando de lo presente, de la
juventud, de usted\ldots{} ¿Qué tal, vamos adelantado en la carrera
militar? ¿Siente usted ambición de gloria?\ldots{}

---No mucha, señor\ldots{} Un año llevo en esta vida, y le aseguro a
usted que deseo la paz, aunque me quede en el grado que tengo. Y esta
campaña del Centro no es para despertar verdaderas aficiones a la
milicia regular. Aquí todo es cuestión de picardía, astucia y agilidad;
todo cuestión de Geografía\ldots{} andada, ciencia de los pies. Además,
el carácter de cacería feroz que va tomando esta guerra, no es para mi
genio. He sido poco afortunado, pues desde que salí a campaña no he
visto más que horrores; y desgracias de nuestras armas. Para tener mala
pata en todo, me estrené con un acto militar que ha dejado en mi
espíritu una sombra lúgubre, algo como una mancha que no puedo borrar:
el fusilamiento de la madre de Cabrera.

---¡Qué dolor!\ldots{} ¡Barbarie inútil, impolítica!

---Empecé mi carrera destinado al regimiento de Bailén, 5º de Ligeros,
que daba guarnición en Tortosa, y mandé el piquete que dio muerte a la
infeliz mujer. Cuando al amanecer del 16 de Febrero del año pasado se
nos dijo que a las diez íbamos a fusilar a María Griñó, no lo creíamos.
Los Nacionales negábanse a cumplir la sentencia. Nosotros no podíamos
menos de obedecer; pero aún esperábamos que tal atrocidad se aplazara
indefinidamente, y aplazarla era como un indulto disimulado. Entre
nosotros se decía que el alcalde de Tortosa, Don Miguel de Córdova,
protestaba de tal iniquidad, y que quiso inducir al Gobernador, general
D. Gaspar Blanco, a no dar cumplimiento a la bárbara orden. Ello era
cosa Nogueras, que ofició al General Mina, y de los allegados de
este\ldots{} Reconocía el Gobernador que disponer tal muerte no era
propio de caballeros, y que si en algún caso procedía desobediencia,
había llegado la hora de poner en el oficio la fórmula: \emph{se acata,
pero no se cumple}. Mas el hombre no se atrevió, y su desmayada voluntad
y su corazón vacilante nos dieron aquel terrible ultraje de la justicia.
Dicen que al resistirse a los ruegos del alcalde y de otras personas
calificadas de la población, se echó a llorar\ldots{} Sus lágrimas
fueron de ésas que no producen ningún bien ni evitan los males\ldots{}
Ello es que metimos a Doña María en el calabozo, y la cargamos de
grillos, y le llevamos al cura D. José María Trench, hombre bueno y
compasivo, que también, llorando a moco y baba, fue a interceder con el
Gobernador, sin conseguir ablandarle. Confesada, mas no comulgada, pues
para esto no le dimos tiempo, la llevamos a la barbacana. Por el camino,
al paso de la pobre víctima, se agolpaba poca gente, pues la mayoría de
los vecinos no se había enterado todavía; de los que vio, se despedía
con palabras sencillas y cariñosas, como si para un viaje saliera. No
puedo olvidar su figura modesta ni su traje, el mismo que tenía en la
prisión: saya de cotolina azul, ya muy usada; jubón de pana verde.
Llevaba al cuello un pañuelo obscuro con fleco, y a la cabeza otro,
blanco, sin atar las puntas. Era delgada, de mediana estatura, rostro
moreno y curtido con arrugas en la frente, el mirar dulce, expresión
candorosa. En sus manos atadas llevaba una cruz. Su resignación, la paz
de su alma, su tranquilidad sin artificio, nos maravillaban; el no
pronunciar palabra ofensiva para nadie, nos colmaba de pena,
oprimiéndonos el corazón. La fortaleza con que afrontaba el suplicio
hacía más vergonzosa la innoble cobardía con que nosotros, con tanto
aparato de fuerza, destruíamos aquella vida que no había hecho daño a
nadie. «¿Qué resulta contra ella?» nos preguntábamos, o lo pensábamos,
por no atrevernos a decirlo. No resultaba más sino que había dado el ser
a Cabrera\ldots{} Llegados a la barbacana, la hicimos avanzar como a
veinte pasos del baluarte\ldots{} El cura que la asistía, D. Joaquín
Curto, no se separaba de su lado tan pronto como convenía. La mirada que
nos echó María Griñó al entrar en el cuadro no se me olvidará si mil
años vivo. ¿Fue de menosprecio, de compasión? De cólera no era, ni
tampoco suplicante\ldots{} no nos pedía que la perdonásemos. Tal vez
quiso decirnos que ansiaba terminar pronto, concordando en esto
fatalmente con las órdenes que habíamos recibido. Se le vendaron los
ojos. Fue preciso, para abreviar, tirarle suavemente del manteo al cura
para que se retirara. El pobre señor estaba turbadísimo: le dijo de
cerca que rezara el Credo, y luego en voz más alta, alejándose, le
anunció que iba a gozar de Dios\ldots{} Yo tenía que dar la orden de
fuego agitando un pañuelo. Me pasó por la mente la idea de no darla,
sublevándome en nombre de Cristo. Pero la fuerza de la disciplina, de
que no nos damos cuenta, se impuso. Ello es que sonaron los tiros, y
cayó la mujer al suelo, de golpe, sin ruido ni contorsiones, como un
vestido, como un colgajo de trapos que cae de una percha\ldots{}

---¡Horrible\ldots{} y estúpido!---exclamó Don Beltrán.---Si tiene usted
más hazañas de estas en su hoja de servicios, no me las cuente. Mi pobre
corazón viejo no resiste esas emociones ni aun contadas.

---Tres días estuve enfermo, sin poder apartar de mí la mirada de María
Griñó, ni aquel modo de caer al suelo, como un vestido que se desprende
de un clavo\ldots{} El vecindario de Tortosa quiso alborotarse, y
tuvimos que contenerle. Los Nacionales trinaban y creían que se habían
deshonrado por formar en el cuadro media compañía. Aseguraban que si se
les hubiera mandado formar el piquete de fuego, no habrían
obedecido\ldots{} Desde aquel día es para mí esta guerra una nube de
plomo posada sobre mis ojos, como un telón a medio echar. Ni sube, ni
baja\ldots{} ni veo bien la guerra, ni veo la paz\ldots{} No habrá ya
paz en la tierra de España. ¿Sabe usted lo que dijo Cabrera cuando supo
la muerte de su madre? Mirando a las cumbres que cercan a Valderrobles,
dijo que la sangre subiría hasta las cimas más altas. Y va subiendo, va
subiendo\ldots{} Para no cansar a usted, Sr.~Don Beltrán, le diré que
mis campañas desde entonces no han sido más que una cacería infatigable.
En multitud de encuentros me he visto, todos encarnizados: estuve en las
acciones de La Jana y de Toga, al mando de Buil; allí tuvimos la suerte
de derrotar al \emph{Serrador}. En Ulldecona, cuando Iriarte dio una
tremenda paliza al \emph{Organista} y a Llangostera, también tuve la
honra de encontrarme. Marchas penosas, hambres y trabajos mil he pasado;
peleando sin cesar, no veo que el aspecto de la guerra cambie. Siempre
es lo mismo: las ventajas de hoy son el descalabro de mañana. Si una
columna vence aquí, otra sucumbe dos leguas más allá. Se les echa de un
valle, y aparecen en otro. Creyérase que salen de debajo de las piedras,
y que la sangre de tantas víctimas, caliente y rabiosa, aun después de
derramada, engendra facciosos en los bosques, en los charcos de los
barrancos, en los escombros de las masadas destruidas. Esto no es
guerra, digo yo: es un duelo feroz, nunca suspendido. Nogueras conoce el
terreno, pero le falta cabeza. Borso tiene intención, pero no domina el
suelo. Sin darse de ello cuenta, conduce sus tropas por el camino más
largo. No encuentra nunca al cabecilla que busca, sino a otro que le
sale inesperadamente por retaguardia, cuando no le salen dos. Así no
acabamos nunca. Si no traen un ejército muy grande para ocupar todas las
posiciones y pueblos de importancia, a la defensiva, tapándoles los
boquetes y pasadizos para sus correrías, matándoles de hambre y
provocándoles a que se enzarcen unos con otros, tenemos guerra para un
siglo. Yo me doy a pensar en esto, y digo: «¿Por qué combatimos?»
Ahondando en el asunto, encuentro que no hay razón para esta carnicería.
¡La Libertad, la Religión!\ldots{} ¡Si de una y otra tenemos dosis
sobrada! ¿No le parece a usted?\ldots{} ¡Los derechos de la Reina, los
de D. Carlos! Cuando me pongo a desentrañar la filosofía de esta guerra,
no puedo menos de echarme a reír\ldots{} y riéndome y pensando, acabo
por convencerme de que todos estamos locos. ¿Cree usted que a Cabrera le
importan algo los derechos de Su Majestad varón? ¿Y a los de acá los
derechos de Su Majestad hembra?\ldots{} Creo que se lucha por la
dominación, y nada más, por el mando, por el mangoneo, por ver quién
reparte el pedazo de pan, el puñado de garbanzos y el medio vaso de vino
que corresponde a cada español\ldots{} ¿No opina usted lo mismo?

---Lo mismo, querido Estercuel, lo mismo. Es usted un sabio. ¡Tan joven,
y ya profundiza!

\hypertarget{viii}{%
\chapter{VIII}\label{viii}}

En esto llegaban al término de la extensísima olmeda, de donde a los
ojos se ofrecía un hermoso espectáculo: la cascada que forma el río Alto
al precipitarse en el Guadalope. Cerros enhiestos formaban el marco de
tan bello paisaje, que D. Beltrán pudo gozar, porque despejada la
niebla, daba el sol relieve y colorido a todos los objetos.

«Si es este el lugar que esa sierva de Dios ha elegido para sus
penitencias---dijo el anciano,---a fe mía que ha tenido buen gusto.

---En aquella casucha que ve usted junto a dos peñas muy grandes,
sombreada por una encina que parece partida por un rayo, moraba estos
días la que llamaré ermitaña trashumante.

Aunque no estaba seguro D. Beltrán de ver lo que su amigo le indicaba,
allá se encaminó a buen paso; y antes de llegar al sitio designado,
vieron que hacia ellos venían dos vejetes con trazas de pastores, por
sus vestiduras de pieles más parecidos a osos que a personas, uno de los
cuales, al llegar a donde pudo ser oído, les dijo: «Si van en busca de
la maestra, vuélvanse, que no la encontrarán.

---¿Pues dónde ha ido mi señora y capellana?---preguntole Estercuel,
sospechando que no le decía la verdad.

---¡Por vida de\ldots!---exclamó Urdaneta, golpeando airado el suelo con
su bastón.---No creí que la buena estrella que me guía en este viaje se
eclipsara tan pronto. ¿Sabéis, buenos amigos, si ha ido muy lejos?
Porque si supiera que no estaba distante, iría en su busca, que con mis
setenta y tantos años, no me arredran un par de leguas.

---Ayer de mañana---dijo el viejo,---fue a la Ginebrosa con mi sobrino,
y nos mandó que por hoy al mediodía la esperáramos en Castellseras, para
ir juntos a donde ella disponga.

---Entre paréntesis: ¿sabéis si vive y dónde está Francisquín Luco,
hermano de Marcela?

---Vive, gracias a Dios\ldots{} pero del paradero no le diré,
señor---replicó el anciano receloso, después de pensar lo que
decía.---No sé\ldots{}

---Sí sabes, tunante; pero no quieres decirlo. ¿No estaba gravemente
enfermo? ¿No le asistía su hermana?

---Así parece, señor\ldots{}

---Está bien\ldots{} Por ventura, ¿no tendríais en vuestra covacha algo
de comer? Porque con el fresco de la mañana y el paseo me siento un
tanto desfallecido.

---Cuando les vimos venir estábamos cortando el pan para hacer unas
pobres migas. Si los señores quieren participar de esta humildad, el
gusto será nuestro, y la penitencia de los señores.

---Discreto eres\ldots{} Ea, preparad esas migas con prontitud, y allá
va con vosotros mi criado para que nos avise cuándo podemos ir a matar
el hambre.»

Al quedarse solos D. Beltrán y Estercuel, sentaditos en una piedra, dijo
el militar al prócer: «Se me había olvidado informar a usted de lo que
en el país se cuenta de las idas y venidas de la monja suelta, y de la
prontitud, al modo teatral, con que aparece y se oculta, sin que nadie
pueda saber de dónde viene ni por dónde se escabulle. Es una conseja, y
a título de tal se lo cuento, advirtiéndole que esta guerra ha
resucitado en el país la Edad Media, tan bien acomodada a su naturaleza
bravía, a la rudeza de sus habitantes y a la muchedumbre de castillos,
monasterios y santuarios que por todas partes se ven.

---Ya había pensado yo eso de que por ensalmos nos encontramos en siglo
de feudalismo. Cuente, cuente pronto esa leyendita, que quizás no lo
sea.

---Pues se dice, y hay quien lo jura, que el padre de esta señora
ermitaña o peregrina era hombre muy rico.

---¿Y a eso llama usted conseja? Puedo dar fe de las propiedades que
poseía Juan Luco, las cuales fueron mías\ldots{}

---Y a más de la propiedad, dicen que poseía grandes cantidades de
dinero metálico\ldots{}

---Naturalmente: era hombre que apenas gastaba el tercio de sus
rentas\ldots{} ¿Y qué más?

---Que antes de lanzarse a pelear por Isabel, Juan Luco puso en un lugar
seguro una olla de onzas\ldots{}

---Precaución muy acertada\ldots{}

---Y en otro lugar seguro, a bastantes leguas del primer sitio, otra
olla de onzas.

---Tenía propiedades en Rubielos\ldots{}

---Y en Valderrobles, y en Calanda, y en Morella\ldots{} sus hijos
hicieron lo propio. El primogénito sepultaba ollas en este monte, y el
segundo en aquel barranco\ldots{} De modo, señor mío, que por todas
estas tierras y por parte de las del Maestrazgo, están esparcidas las
riquezas de Luco.

---Pues, amigo mío---dijo D. Beltrán grandemente excitado, levantándose
y haciendo rápidos molinetes con su bastón,---no veo la conseja\ldots{}
no veo más que un caso muy natural, la pura lógica, señor mío, el puro
sentido común.

---Ollas en los montes de Gúdar, ollas en el desfiladero de Vallivana,
ollas en Mosqueruela, ollas en Beceite, ollas en Calanda, en
Peñagolosa\ldots{} y quién sabe si aquí mismo, bajo nuestros pies, habrá
un puñadito de oro\ldots{}

---Hijo, podrán ser más, podrán ser menos---dijo D. Beltrán con grande
animación, iluminado el rostro, brillantes los ojos, revelando una
credulidad infantil.---El número de ollas no lo sé\ldots{} pero que las
hay\ldots{} ¡ah! lo creo y lo creo, como si las hubiera enterrado yo
mismo\ldots{} Y no me contradiga usted, porque cuando afirmo verdades
como esta, no es prudente contradecirme\ldots{}

---No, si no me parece absurdo\ldots{} Pero falta lo mejor de la
conseja. Dice el pueblo, y cuando el pueblo lo dice es porque lo cree
como el Evangelio, que esta señora monja ha tomado ese empaque
ermitañesco y peregrino para recorrer y vigilar los lugares donde yacen
escondidas las preciosas tinajas\ldots{} Sin duda conoce los sitios por
inspiración del cielo, o por topografías milagrosas que le ha comunicado
el Espíritu Santo\ldots{}

---No se burle usted, amigo mío, que estas cosas no son para tratadas
con genio maleante\ldots{} Y le advierto que me desagrada oír chanzas
aplicadas a cosas y objetos de la mayor seriedad.

---Serio, profundamente serio es cuanto digo, si aceptamos la ficción de
hallarnos en plena Edad Media. Prepárese usted, si persiste en penetrar
en el país, a ver milagros y hazañas, casos inauditos de santidad o
sortilegio, brujas, duendes, apariciones; subterráneos que empiezan en
un castillo y acaban en un monasterio a siete leguas de distancia; verá
usted hombres feroces, hombres heroicos, mujeres endemoniadas o
angelicadas; verá usted, en fin, a la hermosa y andante Marcela, con
aliento guerrero y olorcillo de santidad, corriendo por montes y
barrancos para tomar nota de las mil y quinientas ollas de Luco, y
trasladar a lugar seguro y profundísimo las que fueron escondidas a flor
de tierra en parajes muy transitados; prepárese usted a ver todo esto, y
si algo descubriese contante y sonante, avise, Sr.~D. Beltrán, que no ha
de faltarle un buen amigo que, armado de pala y azadón, le preste ayuda.

---¡Tunante!---dijo el anciano, que gozoso se lanzaba a la confianza
paternal,---si tuviera usted la suerte de encontrar uno de esos nidos,
ya sé que le faltaría tiempo para ponerlo a un maldito caballo, o a un
as indecente\ldots{} No quiero dejar pasar esta ocasión sin echarle un
réspice\ldots{} mi ancianidad me da derecho a ello\ldots{} Yo te vi a
usted anoche encenagado en el feo vicio. Paréceme que era usted el que
tallaba\ldots{}

---Sí, señor, por mi desgracia. No sé si advertiría usted que me
desplumaron.

---Tanto como eso no reparé\ldots{} Y ¿qué tal? ¿Eran atrevidos aquellos
puntos? ¿Se traían alguna martingala?\ldots{} Sea lo que quiera, un
joven de sus méritos no debe dejarse dominar por la pasión del
azar\ldots{} Todo el dinero que caiga en sus manos guárdelo usted, hijo,
guárdelo para sus necesidades de mañana. Piense en la vejez, que si en
todo caso es triste y desabrida, sin dinero es suplicio grande. Pero, si
no me engaño, oigo la voz de Tomé que nos llama, señal de que esas
benditas migas nos esperan.»

No tardaron en llegar a la choza; y tan grande apetito se le había
despertado al buen señor por causa de la frescura matinal, del paseíto,
o quizás por la risueña visión de las ollas auríferas, que empezó a
tragar migas, todavía calientes, a riesgo de abrasarse el gaznate; y
comiendo decía: «Pues de tal modo me interesa avistarme hoy mismo con la
venerable madre Marcela, para tratar con ella de un grave punto de
religión, que si estos señores van en su busca, les acompaño\ldots{} No,
no puedo detenerme\ldots{} No trate usted de disuadirme, amigo
Estercuel. Ni a mí ni a mi criado nos arredran ladrones ni carlistas. Si
usted los teme, vuélvase tranquilo a Alcañiz.

---No por miedo, Sr.~D. Beltrán, sino porque mis deberes militares al
pueblo me llaman, me veo precisado a dejarle partir solo.

---¡Ah! la obligación es antes que la devoción. El buen militar no se
pertenece\ldots{} Pues iré con Tomé y estos ancianitos. ¿Qué distancia
me ha dicho? ¿Legua y media? A pie mejor que a caballo. Me conviene un
poco de ejercicio\ldots{} sí\ldots{} Aún tengo bríos para andar largo
trecho. Si he de decir la verdad, me siento\ldots{} así como
rejuvenecido\ldots{} Sin duda es el aire de esta tierra, no sé qué gozo
del ánimo\ldots{} Hasta parece que veo mejor\ldots{} Sí, sí\ldots{}
distingo perfectamente las pieles de estos hombres, la sartén,
todo\ldots{} No hay duda, no hay duda: veo mejor, amigo
Estercuel\ldots{} Y apostaría que, después de un paseo de dos leguas, se
me aclarará la vista notablemente\ldots{} ¿Y qué tal?, ¿Se conserva bien
la hermana Marcela? No la he visto desde que era muy niña\ldots»

Atacado de una locuacidad que no podía contener, enjaretaba cláusulas
sin el debido enlace entre unas y otras. Como los ancianos no decían una
palabra ni comían, pidioles cuenta D. Beltrán así de su silencio como de
su falta de apetito, y el uno de ellos respondió que delante de tan gran
señor no era decente que ellos, infelices mendigos, hablasen ni
comiesen. Replicó a esto el afable aristócrata, que ante Dios, Padre
común del género humano, todos los hombres eran iguales, y que, pues
allí les reunía el acaso, no se acordasen de vanas categorías. Si ellos
eran pastores, ¿qué oficio y estado superaba en nobleza y antigüedad al
de conducir rebaños? Pastores fueron los patriarcas en aquel pueblo que
Dios llamó suyo; pastores fueron los primeros que adoraron y
reconocieron al Redentor del Mundo en Belén, y este había representado
su misión debajo del simbolismo de un pastor del gran rebaño de la
Humanidad. A esto replicaron los vejetes que no eran ellos pastores, y
que usaban aquellos pellejos, y los peales y zurrón por ser el traje más
adecuado a la frialdad del tiempo y a la fragosidad del país.

«¿Pues qué sois?---dijo el prócer, suspenso, preparándose a probar de un
queso que le ofrecían.

---Nuestro oficio es el de sepultureros; sólo que ya hemos dejado aquel
empleo tan humilde por acompañar y seguir a la divina Marcela.

---¡Hombre, hombre\ldots{} sepultureros, enterradores!---exclamó
Urdaneta con asombro.---Pues también es ocupación noble, antiquísima
como el mundo, pues desde que hubo vida, hubo muerte. Y oficio santo
además, que en él se cifra una de las obras de Misericordia. Muy bien,
muy bien, pobrecitos. Me agrada vuestra compañía. Enterrar los muertos
es noble misión. Dios manda que, después de recoger Él el alma, se dé a
la tierra lo que le pertenece. ¿Y quién sabe si revivirá algo de lo que
habéis soterrado? No todo lo que entra en la tumba es muerte. La fosa
recoge también la vida, para sustraerla a la codicia y al
latrocinio\ldots{} Y difuntos aparentes habréis sepultado, que volverán
a la vida y\ldots{} Pero de estas filosofías no entendéis
vosotros\ldots{} Y dime otra cosa: desde que os encontré, tú solo
hablas. ¿Por qué no hemos oído la palabra de tu compañero?

---Porque se le traba la lengua, y no quiere que le oigan\ldots{}

---Es tartamudo\ldots{} mudo quizás. Ya sabe Marcela lo que hace,
rodeándose de hombres callados, silenciosos, y cuando no, discretos como
tú\ldots{} Pero no perdamos más tiempo y pongámonos en camino.

Levantose ágil, sin esfuerzo, con sorpresa de todos, y emprendieron la
bajada al camino, al llegar a este se despidió del amable militar, que
deseándole un regreso pronto y feliz, le dijo: «Ya ve el Sr.~D. Beltrán
cómo va resultando lo que anuncié. Edad Media, pura Edad Media\ldots{}
Supongo que le veremos esta noche por Alcañiz, y ya nos contará, ya nos
contará\ldots{} Quiera Dios que no tenga un mal encuentro\ldots{} Es
posible que pueda ir y volver felizmente, porque no hay noticias de que
ahora anden por aquí partidas. Abur. A Sor Marcela le da usted
expresiones de mi parte, y que se deje ver\ldots{} De buena gana me
ajustaría yo en su cuadrilla de sepultureros, si supiera que tocaban a
desenterrar\ldots{} lo que usted sabe. Adiós.»

Internándose a buen paso en la olmeda que conduce a la ciudad, decía
para su sayo el bueno de Estercuel: «El pobre señor, reverdecido en la
niñez, está ya en su elemento: la conseja.»

\hypertarget{ix}{%
\chapter{IX}\label{ix}}

Anduvo larguísimo trecho D. Beltrán por la margen izquierda del
Guadalope, sin encontrar alma viviente, pues los caseríos estaban
desamparados, los ganados dispersos, hombres y animales del campo
huídos; y tan presuroso iba por el estímulo de su deseo, que al llegar a
las primeras casas de una aldea desierta, que debía de ser Castellseras,
faltáronle súbitamente al anciano los alientos, y dejándose caer en un
montón de tierra, cercano a un edificio en ruinas, dijo a sus
acompañantes: «Amigos míos, la costumbre de andar en coche y a caballo
ha quitado vigor a mis piernas para la marcha peonil. Vosotros andáis
sin fatigaros muchas leguas, yo no puedo. Me rindo, me entrego, y pues
ya no estamos lejos del punto en que os habíais citado con la maestra,
os ruego que os adelantéis y le digáis que la espero aquí. Recordad bien
mi nombre: D. Beltrán de Urdaneta\ldots{} el grande amigo y en otro
tiempo protector de su padre\ldots»

Obedecieron sin chistar los dos viejos, y D. Beltrán se quedó solo con
su criado Tomé, el cual no hacía más que mirar a los cerros cercanos,
pues en todos veía fusiles y boinas su medrosa fantasía. Por indicación
suya, se pusieron al abrigo y sombra de aquellas derrumbadas paredes, de
donde vigilarían quién viniera, y podrían esconderse si alguien se
acercaba con malas intenciones. Allí se aguantaron como unas dos horas,
y ya se impacientaba Urdaneta, cuando Tomé, encaramado en lo más alto,
avisó la presencia de cuatro personas por el camino que habían seguido
los viejos al partir.

«¿Ves a los enterradores?---preguntó Don Beltrán ansioso.---¿Viene con
ellos una señora vestida de monja o penitente?

---A los dos abuelicos les veo---dijo Tomé cuando las cuatro figuras se
aproximaron;---pero no viene ninguna monja, sino dos chicarrones, uno de
ellos con sotana.

---¿Estás bien seguro del sexo?

---¿Qué dice, señor? Si llama sexo a lo de distinguir de machos y
hembras, apuesto lo que quiera a que los cuatro son hombres naturales,
aunque al uno no le veo piernas por bajo, y por arriba le veo melenicas
como las de una imagen.

---¿Luego viene uno con faldas?

---Mas no son faldas ni andares de mujer, sino al modo de las túnicas de
los santos, que siempre usaban sayos o camisones.»

Y cuando ya cerca estaban, y amo y criado salían de las ruinas para
recibirles, gritaba Tomé: «Señor, señor, déjeme que me santigüe, pues
esto no es cosa buena. El de los pelos largos y caídos es un muchacho
amujerado, o mujer hombruna. No he visto otra\ldots{}

---Cállate, simple, y ponte a un lado, que ya veo los bultos, y me
adelanto a saludar a Marcela.»

Del grupo que venía, se adelantó una figura híbrida, tal y como Tomé la
había descrito, para mozuelo, de regular talla, para mujer, de elevada
estatura, con gallarda medida y proporción. Era el rostro moreno, tan
tostado del sol que semejaba al de una efigie secular, cuyo barniz el
tiempo ha obscurecido dándole una dulce pátina con vislumbre sienoso.
Los ojos grandes, negros y de profundo mirar, parecían de hombre; de la
nariz para abajo representaba cara fina y graciosa de hembra, con
hoyuelos en la barbilla, y un poco de vello sobre el labio superior. El
cabello caía en guedejas que parecían plumas de un gallo negro, y le
llegaba hasta mitad del pescuezo, no menos tostado que el rostro,
partiéndose en la frente en dos ramales espesos, ásperos, que a veces
nublaban los ojos. Era el cuerpo de rara perfección, más de hombre que
de mujer, pues no se le notaba elevación del seno, el cual era poco más
alto que el de un mocetón de anatomía lozana; bien sentidas la cintura y
cadera, sin ofrecer curvas muy acentuadas; el pie desnudo, de color de
antigua caoba, de mediano tamaño tirando a grande, y admirable forma. El
sayal que vestía, de parda estameña, remedaba un hábito franciscano de
varón; pero sin cuello ni capucha, sencillísimo en su traza y corte,
ceñido a la cintura por una cuerda. Llevaba el rosario en un bolsillo
interior del hábito, que se manifestaba en una abertura vertical al
costado derecho, por donde asomaba la cruz de bronce. Mayor bulto que el
de un rosario se veía por aquella parte; señal de que guardaba otros
objetos, pañuelos quizás, o sabe Dios qué. La voz, que hirió con sonoro
timbre los oídos de Don Beltrán en el primer saludo, era como de
muchachón tierno, engrosada por la constante vida al aire libre en país
tan frío.

«Aunque estos pobrecitos---dijo Marcela,---equivocaron el nombre\ldots{}
\emph{Don Jordán de la Beltraneta}, ya comprendía, señor, ya comprendí
que era usted\ldots{} el que me hacía el honor de venir en mi
busca\ldots{}

---El honor es mío---replicó D. Beltrán descubriéndose y besándole la
mano,---y me considero feliz de ver en opinión de santa a la que conocí
muy niña\ldots{} Ya, ya se anunciaba en ti la mujer superior,
extraordinaria, eminente\ldots{}

---Mi padre le apreciaba a usted de veras---dijo Marcela, cortando el
elogio.---Diez días antes de morir, estuvo a verme, y hablamos
largamente del Sr.~D. Beltrán\ldots{}

---Siempre tuve a Luco---afirmó el prócer, gozoso de lo que la ermitaña
relataba,---por uno de mis mejores amigos. De cuantas personas he
tratado en mi larga vida, Juan fue la única en quien vi siempre la flor
de la gratitud\ldots{} Sabrás que a mi protección decidida debía tu
padre los adelantos de su fortuna.

---Lo sé\ldots{} y a gala tenía el recordarlo\ldots{} A mis hermanos y a
mí, cuando éramos niños, nos enseñó a pronunciar con el mayor respeto el
nombre para él sagrado de Urdaneta\ldots{} Pero si el señor gusta de que
hablemos, no piense en volverse hoy a Alcañiz, y véngase conmigo
despacito hacia Calanda, que allí tengo un alojamiento regular, y podré
darle algo de comer, siempre dentro de la suma pobreza.»

Tan grata impresión habían hecho en el viejo las primeras palabras de la
santa mujer, que a todo se prestó gozoso, diciendo: «Vamos a donde tú
quieras, hija mía, y no creas que me asusta la pobreza, pues he llegado
a una situación en que mi gloria es confundirme con los humildes.

---Vivimos en el reino de la desventura---dijo la ermitaña con
austeridad.---El azote de Dios nos ha reducido a todos, ricos y pobres,
hombres y mujeres, a las extremidades de la miseria, y a no contemplar
más que espectáculos de tristeza y dolor. El Señor nos ha castigado, nos
somete a prueba durísima, desatando a la Muerte para que a ninguno
perdone. Convenzámonos de que sólo breves instantes nos faltan para
morir, que no hemos muerto ya por cansancio de la misma Muerte, la cual
apenas tiene aliento para cortar tantas vidas, y preparémonos\ldots{}

---¡Oh! sí, bien preparado estoy para cuando el Señor lo
disponga\ldots{}

---Y en tanto, fortifiquemos nuestras almas con la paciencia, con el
gusto de las adversidades, y celebremos las miserias y trabajos que Dios
nos envía.

---Sí, hija mía, sí\ldots{} celebrémoslo\ldots{} ya lo creo que debemos
celebrarlo\ldots{}

---«Que los trabajos bien recibidos y padecidos son, no sólo útiles y
provechosos, sino gustosos y sabrosos\ldots» Esto lo dijo Nicéforo,
famoso historiador de la Iglesia, y añade que «son las adversidades
satisfactorias por los pecados, y que los trabajos nos son útiles por la
fortaleza que con ellos se gana.» Tengamos fortaleza, Sr.~D. Beltrán,
esta soberana virtud con que se vencen y encadenan todos los males.

---Sí, hija mía, sí---murmuraba D. Beltrán:---seamos fuertes; yo busco
la fortaleza.

---Dice el bienaventurado San Juan Crisóstomo que «aunque los trabajos
no tuvieran otro bien sino el que el hombre recibe con su paz y quietud
cuando le faltan, fueran de muy grande codicia.»

---Paz y quietud anhelo yo, hija mía, y por Cristo, que a mis años,
después de tantas luchas y fatigas, bien merezco el reposo. Y bien
podría el Señor concedérmelo en premio de la valentía con que me lanzo
por estos caminos infestados de facciosos. Cierto que cuando Dios nos
manda trabajos y adversidades, ya se sabrá por qué lo hace; pero yo te
digo ahora, con perdón de San Nicéforo y San Crisóstomo, que maldita
gracia me hará que nos salga una partida carlista y nos deje en cueros,
o nos apalee o nos fusile\ldots{}

---El verdadero cristiano---dijo la beata peregrina con acento firme,
sin afectación,---no sólo no teme la muerte, sino que la desea. Cuenta
Eusebio en sus \emph{Anales} que, «hallándose los mártires presos, se
alegraban creyendo habían de ser los primeros que sacasen a martirizar,
y cuando no lo eran, quedaban desconsolados.»

---Pues perdóneme el señor Eusebio\ldots{}

---Y testifica San Jerónimo que el bienaventurado mártir San Ignacio
escribía a Siria desde Roma, poco antes de su martirio: «Plegue a Dios
dejarme gozar de las bestias que me esperan, las cuales ruego a Dios no
sean perezosas en acabarme\ldots» Donde dice bestias ponga usted
facciosos, y digamos: «Que vengan cuando quieran y nos despedacen.»

---Todo eso es muy bonito para dicho; pero como no soy santo, quiero
guardar de ésos los pocos días que me restan.»

Si en los comienzos del diálogo le encantaba a Urdaneta la firmeza de
convicciones de la peregrina y el severo estilo con que la manifestaba,
en cuanto empezó a largar citas se le hizo un poquito indigesta tanta
sabiduría. Preguntole que cómo podía repetir sin equivocarse tantos
textos de sagradas escrituras, y ella lo explicó por su prodigiosa
retentiva\ldots{} Lo que una vez leía, no se le olvidaba nunca, y su
mente era una copiosa biblioteca, que usaba sin compulsar libros. Por
todo el camino fue soltando citas de Santos Padres y de Aristóteles y
Cicerón; que también éranle familiares los filósofos profanos; y ya un
tanto mareado D. Beltrán con aquella erudición fastidiosa, diputó a
Marcela por un papagayo con más memoria que discernimiento. Aún era muy
pronto, dice el narrador, para formar juicio tan terminante.

Al caer de la tarde, llegaron a un barrio de Catanda, y metiéronse en
una casa mísera, donde había tres mujeres. Ningún hombre se veía en todo
el lugarejo ni en sus contornos. Impaciente por hablar largo y tendido
con la santa, hizo propósito D. Beltrán de plantear el magno asunto en
cuanto despacharan la frugal cena de alubias, habas secas, y algunos
huevos con que fue regalado el huésped. Como si le leyese en el rostro
los pensamientos, Marcela se apartó con él a un rincón de la estancia
donde comieron, que era un establo de cabras, sin cabras, y le dijo:

«Sr.~D. Beltrán, antes que empiece yo mis rezos y ejercicios de la
noche, y antes que usted se acueste\ldots{} que para su nobleza se
prepara en esta humildad un mediano lecho\ldots{} quiero que me diga la
razón de venir a buscarme.

---Precisamente, ya se me hacía tarde el hablarte de ello, hija mía.
Bien comprenderás que si a los riesgos de este viaje expongo mi
ancianidad, es porque me lo exige mi decoro, el honor de mi nombre.

---Fuertes razones habrá sin duda. Recordando lo que del Sr.~D. Beltrán
me dijo mi padre días antes de morir, lo que después oí a mis hermanos,
y agregando lo que yo con mi pobre entendimiento adivino, creo conocer
los motivos que acá le traen.

---Si lo has adivinado, me libras del enojo de decírtelo, que nunca es
grato en un hombre de mi condición declarar sus necesidades. Pero algo
debo referirte como antecedente necesario, y es el hecho de las
desavenencias graves con mi familia, y mi resolución de abandonar la
casa de Idiáquez para no volver más a ella.

---También sé algo de esto---indicó la monja con un dejo de
severidad,---y creo que no es toda la culpa de su familia, que buena
parte de esa culpa debe recaer sobre usted.

---Puede\ldots{} sí\ldots{} no digo que no\ldots---murmuró desconcertado
el aristócrata.

---Porque las opiniones están conformes en que ha sido usted un pródigo
incorregible\ldots{} Ha derramado su caudal, y ahora se encuentra escaso
y pobre. \emph{Effusus es sicut aqua; non cresces}. «Derramado has como
agua, y ahora no creces, no tienes,» como Jacob dijo a su hijo Rubén.

---Sí, es cierto\ldots{} sí\ldots{} Pero yo, por mi condición generosa y
mis hábitos de gran señor, desprecié siempre las cosas menudas,
pequeñas\ldots{}

---¡Ah! señor mío. El \emph{Eclesiástico} lo ha dicho: \emph{qui spernit
modica, paulatim decidet}. ¿Lo entiende usted?

---Hija mía, se me ha olvidado el poco latín que aprendí en mi niñez.
Háblame castellano. En castellano neto te digo yo que si es cierto que
con mi conducta he creado mis daños, ya no estoy en edad de corregirme.

---Bueno, señor. Pues mi padre\ldots{}

---Tu padre era, el primer año del siglo, un triste labrador que llevaba
en arrendamiento algunas de mis tierras de Rubielos. Gran trabajador,
gran economizador, el año 6 y 7 quiso comprarme las piezas de Alventosa
y el prado grande de Alcalá de la Selva. Aunque otros compradores me
ofrecían mayores ventajas, preferí a Luco, atento a su honradez y
puntualidad\ldots{} Además, siempre me ha gustado dar la mano al pobre.
Quedose tu padre con aquellas tierras, luego con otras, y me pagaba
cuando quería, a su comodidad y desahogo. ¿Es esto cierto?

---Usted lo ha dicho.

---Siempre se mostró tu padre agradecido, y andando los años recibí
pruebas de la estimación en que me tenía.

---Y jamás le apremió usted por los pagos; lo sé.

---Ni le cobré intereses por las demoras. Al fin, todo fue suyo; todo
no: quedábanme el monte de Mosqueruela y la encomienda de Forniche Bajo.
El 22, hallándose ya Luco en gran prosperidad, por las buenas cosechas y
el gran incremento que tomó el comercio de lanas, propúsele yo que me
comprase la Mosqueruela para que redondeara sus estados, y accedió a
ello, abonándome, desde aquella fecha hasta el 30, los plazos en que
estipulamos la venta. El año 33, hallándome yo algo escaso de fondos, y
necesitando reunir una cantidad para atenciones ineludibles, pedí a Luco
dos mil duros, que me mandó al instante. Le cedí las rentas de la
Encomienda por todo el tiempo que fuese preciso hasta la extinción de la
deuda, y al año siguiente le propuse que me comprase también esta finca
por la valoración que estimara justa. Todo se hizo conforme a la
voluntad de tu padre, pues ni yo regateaba con un hombre de tanta
rectitud y conciencia, ni me hallaba en aquellos días, por el
aturdimiento que me causaban mis afanes, en disposición de apreciar mil
duros más o menos en mis negocios. Siempre he sido lo mismo. Pasó
tiempo; y hace unos meses, hallándome yo en Villarcayo, recibo una carta
de tu padre en que me decía: «Sé, mi noble señor, que por ruindad de los
tiempos y caídas de grandezas humanas, se halla Vuecencia en escasez de
posibles. Si con el caudal no ha perdido la memoria, recuerde que está
en el mundo Juan Luco, y no olvide que Juan Luco no consentirá jamás que
padezca necesidades el primer caballero de Aragón.»

---Así es---dijo la venerable, afirmando además con una fuerte cabezada.

---Y hay más, hay más, mi bendita señora---dijo D. Beltrán, animándose
con el buen giro que, a su parecer, llevaba el asunto.---En la misma
carta decía: «Recuerde también el señor, y medite y repare que lo de la
Encomienda fue más ventajoso para un servidor que para usía; y pues Juan
Luco ha sido siempre hombre de conciencia, hoy, ante la verdad clara de
sus adelantos de fortuna, quiere serlo en mayor grado, y más que
condenarse por egoísta, le gustará salvarse por generoso. Dígame, pues,
el señor lo que necesita, y no será él tan presuroso en decírmelo como
yo en acudir a su alivio y remedio\ldots» Esto decía; y si lo dudas,
angélica mujer, aquí tengo la carta\ldots{}

---No, no ha de mostrármela, señor, pues lo que me dijo pocos días antes
de morir mi honrado padre es en todo conforme con el tenor de su carta.»

\hypertarget{x}{%
\chapter{X}\label{x}}

Echó D. Beltrán de su pecho, al oír tan consoladoras palabras, un
suspiro muy grande, con el cual pareció que se descargaba de la
pesadumbre de sus desdichas. Miró a la santa mujer, que al suelo
inclinaba sus ojos sin expresar nada inteligible en su rostro de imagen.
Pasado un ratito, la penitente miró al anciano, diciéndole: «Hora es ya
de que descanse, señor. Por lo que hemos hablado, bien se ve que sus
deseos son recoger ahora lo que le ofreció mi buen padre, cosa en verdad
fácil en mi voluntad, pero dificultosa en la de Dios, que es quien
dispone las cosas\ldots{} No puedo darle tan pronto respuesta
terminante, pues ello ha de ser muy pensado\ldots{} Recójase ya, duerma
tranquilo, y persuádase de que, puesto su negocio en mis manos, de la
hija de Juan Luco no ha de recibir usted ningún mal, sino todos los
bienes posibles\ldots»

Aunque estas vaguedades no satisfacían por entero las aspiraciones de
Urdaneta, que quería solución clara y pronta, fuese el hombre al
camastro esperanzado de lograr sus deseos, y confiando en la rectitud de
la piadosa mujer. Pasó la noche intranquilo, febril, y en los breves
ratos de sueño creíase transportado a subterráneos de castillos o
criptas de iglesias, donde entre tumbas aparecían ánforas llenas de
plata y oro. Despabilado desde el alba, llamó a su criado para que le
vistiera, y Tomé se apresuró a comunicarle lo que pensaba de la monja y
de su compañía. «Señor, debe de ser santa, porque la vi de rodillas más
de cuatro horas, y a ratos echábase de cara contra el suelo, y parecía
que lloraba con ansias y congojas\ldots{} Las otras dos mujeres también
rezaban, aunque con menos figuraciones; para mí son, como ella, monjas
desperdigadas y salidas\ldots{} Yo no pude dormir del frío que hacía en
aquella cuadra, y viendo tanto rezar, me puse a hacer lo mesmo\ldots{}
Los viejos y el muchacho, arrimaicos a la pared roncaban como
\emph{tocinos.» }

Algo más hablaron, comunicándose uno a otro sus impresiones. Sirvieron a
D. Beltrán las mujeres, muy de mañana unas sopas que le supieron a
gloria; y mientras las comía, díjole Marcela que habían de ponerse en
camino inmediatamente, tomando ella con los viejos la vuelta de Alcañiz,
por el vado de Torrevelilla, pues tenían que hacer en la Codoñera. Irían
juntos, y por el camino sabría D. Beltrán lo que ella durante la noche
había pensado del asunto que al señor tanto interesaba. Para resolverlo
del modo más equitativo había pedido luces a la Divina Ciencia,
recogiendo su espíritu en oración muy fervorosa, a fin de que Dios la
iluminase en el fallo que tenía que dar sobre cosas temporales. Ya
empezaba el caballero a inquietarse con estos requilorios, y se dispuso
a seguir a la santa, ansioso de escuchar pronto su resolución o
sentencia.

Salieron por un caminejo de herradura en busca del Guadalope, que por
aquella parte corre encajonado entre cerros de mediana elevación.
Marcela echó por delante a Tomé y a los dos viejos sepultureros, y
abordó con D. Beltrán el magno asunto: «Ante todo, hija mía---le
preguntó el prócer,---¿por qué tus viejos, a quienes no sé si llamas
discípulos o hermanos, llevan el uno una pala y el otro un azadón?

---Se han impuesto por penitencia dar sepultura a todos los muertos que
dejan tras de sí, en sus horribles batallas, liberales y absolutos. Por
mi cuenta han enterrado ya como tres centenares de cristianos
sacrificados a la ambición de los poderosos del mundo.

---Dios les reciba en su santo seno\ldots{} Pues satisfecha esta
curiosidad, dime ahora si debo esperar que des cumplimiento a la
voluntad de tu padre con respecto a mí; voluntad bien manifiesta\ldots»

Con el estilo severo y elegante, aunque algo duro, que en la lectura de
autores místicos se había asimilado, interpolando a cada instante citas
de Santos Padres, o de Aristóteles, Longinos, Teofrasto Paracelso y
otros sabios, como si con la erudición quisiera dilatar la sentencia,
Marcela manifestó a D. Beltrán que ella y su hermano Francisco ignoraban
dónde yacían soterrados los dineros que Juan Luco poseía en sus últimos
años, salvo una pequeña parte, cuyo paradero, por declaración de su
difunto hermano Cinto, conocían; que si lograban descubrirlo y
asegurarlo todo, cosa en extremo difícil en medio de guerra tan
desaforada, lo destinarían a una obra de gran piedad, como desagravio al
Señor por las iniquidades que las dos catervas de combatientes cometían.
Ambos hermanos estimaban, en su acendrada fe, que dar tal destino a las
riquezas de su buen padre sería muy grato al alma de este, ya se hallara
purgando sus pecados en el fuego del Purgatorio, ya estuviese gozando de
Dios, purificada y limpia por su martirio. Francisco Luco, el menor de
los tres hermanos varones, había hecho en Huesca sus estudios
eclesiásticos y disponíase a recibir las sagradas órdenes, cuando el
maldito clarín de guerra, hiriendo sus oídos y despertando en él ideas
de bandería política y militar soberbia, le indujo a tomar parte por
Isabel en la querella. Breves y no felices habían sido sus hazañas. En
Liria fue verdadero milagro que no le fusilaran. Dolorosos meses de
cautiverio pasó en Cantavieja. Libre al fin, al tomar la plaza el
General San Miguel, volvió a sus anhelos pacíficos y religiosos,
horrorizado de la guerra y de sus desmanes. Ante su hermana, y cuando
esta le asistía en la penosísima enfermedad contraída en el cautiverio,
hizo voto solemne de consagrar a Dios su vida, su alma y sus
pensamientos todos, sin esperar a ponerlo por obra más que el tiempo que
se tardase en preparar las cosas materiales para tal objeto\ldots{}

«Según eso---dijo D. Beltrán, a quien con tales santidades se le había
puesto un nudo en el tragadero, sin poder pasarlo para arriba ni para
abajo,---tu hermano entra en religión\ldots{} cantará misa, profesará en
alguna Orden. ¿Dónde está? Yo quiero verle.

---Espérese usted\ldots{} Francisco abrazará la vida religiosa; pero
antes de abandonar el siglo, tratará de descubrir y reconocer dónde se
hallan los bienes en especie que padre trató de sustraer a manos
rapaces. Y con decir yo esto, y usted con oírlo, queda manifestado, y
por usted comprendido, que hemos de destinar íntegro todo el caudal a
una fundación santa para religiosos de la Orden que abrace mi hermano, y
a restaurar mi glorioso convento de Sigena.

---Sí, hija mía, sí\ldots{} comprendido. Pero dime: tu hermano, ¿dónde
está?

---Hállase actualmente no muy lejos de nosotros, atento a lo que a él y
a mí tanto nos importa; mas para poder efectuar sus pesquisas en materia
tan delicada, ha sido menester que se agregase a una columna cristina,
so color de prestar en ella servicio hospitalario, que otro servicio más
guerrero no podría, por causa del grave detrimento de su
naturaleza\ldots{}

---No dudo---dijo D. Beltrán, cuya vista se nublaba, como si su pena
fuera una obscurísima visera que le caía sobre los ojos,---que si yo
hablara con Francisco Luco en tu presencia, ambos me darían prueba
inequívoca de su piedad y rectitud declarándome poseedor de aquello que
vuestro padre determinó que había de ser mío.

---Si he de hablar al Sr.~de Urdaneta con la plenitud de verdad que se
desborda de mi corazón---dijo la monja endulzando la voz,---le
manifestaré que me parece impropio de sus años ese insano apetito de las
riquezas. En la declinación de la vida, y cuando Dios ha decretado ya
para usted el acabamiento de todas las vanidades, ¿para qué quiere lo
que no puede disfrutar, ni tiempo tiene para ello?

---Hija mía, es que\ldots{}

---Padre y señor mío, la verdad sale de mis labios sin que mi respeto
pueda contenerla. Debiera usted despreciar las riquezas, y alegrarse de
haberlas perdido, renegar de que quieran dárselas\ldots{} y apartarlas
de sí como se aparta la podredumbre pestilente\ldots{} Sí, D. Beltrán.
Le recordaré, por si lo ha olvidado, lo que dijo San Pablo a los
hebreos: «Con alegría recibisteis el robo que os hicieron de vuestros
bienes.» Sí, sí, noble señor: alégrese de que le hayan despojado de sus
tesoros, y no ansíe volver a poseerlos\ldots{}

---Pero\ldots»

No siguió el desgraciado anciano por que tanto se le apretaba el nudo en
su gaznate, que no pudo articular palabra.

«Llénese, señor---continuó la santa con inspirado acento,---llénese de
aquella virtud de la paciencia, que todas las demás virtudes compendia y
resume; ame la pobreza, bendiga el no tener\ldots{}

---¡Pero\ldots{} hija mía\ldots---pudo decir al fin D. Beltrán,---si a
paciencia nadie me gana!\ldots{} Verás\ldots{} Yo\ldots{}

---Tertuliano dijo: «Donde Dios se halla, allí está con Él su amiga la
paciencia.»

---Estamos conformes\ldots{} Tertuliano y yo\ldots{}

---Y no olvide, Sr.~D. Beltrán, que la Divina Sabiduría dice en los
\emph{Proverbios: O viri, ad vos clamito, et vox mea ad filios
hominum.}.. \emph{Mecum sunt divitiae}\ldots{} fíjese D. Beltrán\ldots{}
\emph{mecum sunt divitiae, et gloria, opes superbae, et justitia}.

\emph{---¡Oh, la mère latiniste!\ldots{} Je n'aime pas les gens qu'a
tout propos crachent du grec et du latin. } ---Señor D. Beltrán, yo no
sé francés.

---Señora Doña Marcela, yo no sé latín. Hablemos en la lengua común.

---Pues en ella digo a usted que ya estamos en el Guadalope, y que
callemos ahora, pues juntamente con Tomé y con los ancianos que allí nos
esperan, emprenderemos el paso del río por aquel vado.»

Efectuado sin contratiempo alguno el tránsito de una orilla a otra,
siguió D. Beltrán por aquellos vericuetos, taciturno y suspirante; a su
lado iba la peregrina, rosario en mano, rezando al compás de la marcha
lenta y fatigosa, al través de montes solitarios, en un día destemplado
y brumoso. En las agrias pendientes solía D. Beltrán pedir descanso,
para dar paz a sus viejos pulmones; y en una de estas paradas, Sor
Marcela, terminando presurosa entre dientes una oración, dijo a su
aburrido acompañante: «No se aparta de mi pensamiento, noble señor mío,
su malestar, y me duele mucho la desazón que yo, sin quererlo, haya
podido causarle. Pensando vengo en ello todo el camino y pidiendo a Dios
que me ilumine con nuevas ideas. De Dios debe de venir, pues, esta que
ahora me asalta y que voy a manifestarle.

---Sí, sí, de Dios tiene que ser, si es idea benéfica y compasiva.
Dímela pronto.

---Pues he venido pensando por el camino que usted, en su vejez, triste
occidente de una vida de prodigalidad y disipación, habrá contraído
deudas, compromisos que afectan al honor y buena fama, y que desea, como
caballero cristiano, darles cumplimiento antes de morir.

---Hija de mi alma, hablas ahora como la misma sabiduría---dijo D.
Beltrán casi llorando, con ganas de arrodillarse y besarle la orla del
sayal.

---Bien, señor: se le dará lo que necesite para ese objeto, siempre que
adopte vida religiosa consagrando a la oración y penitencia el resto de
sus días. No tiene usted que inquietarse de cosa alguna, tocante a la
providencia de pagar sus deudas y demás negocios mundanos. Mi hermano, o
persona que él designe, se encargará de dejar bien puesto el nombre de
Urdaneta, pagando lo que usted debe a los hombres. Usted no vivirá ya
más que para pagar a Dios lo que a Dios debe\ldots{}

---Pero\ldots{} entendámonos\ldots{} La idea no es mala\ldots{}
Explícate mejor\ldots{} ¿Antes de que me arregléis mis asuntillos, tengo
yo que meterme fraile?\ldots{}

---Parece como que le espanta la idea.

---No, hija, no\ldots{} es que\ldots{} verás\ldots{}

---¿Se tiene acaso por persona más alta que el Emperador y Rey Carlos V?

---No, no\ldots{} ¡Si estamos conformes! Yo deseo el descanso, la
abdicación---dijo D. Beltrán, pensando que le sería forzoso dar su
asentimiento, a fin de obtener después, por concesiones graduales,
sentencia más conforme con sus deseos.---No tengo inconveniente\ldots{}
La idea es muy acertada\ldots{} Pero hazte cargo de la urgencia de mis
compromisos.

---Sobre toda urgencia está la de dar a las riquezas de Juan Luco la
aplicación santísima que hemos determinado.

---Aprobado, hija, aprobado\ldots{} La idea es grandiosa, ea\ldots{}

---En obsequio al amigo y protector de mi padre, hacemos una sola
excepción, consagrando parte de aquel caudal a poner en salvo la buena
fama de un noble caballero aragonés. Pero esto no ha de hacerse sino
consagrando usted previamente los días que le restan de vida a la
oración y a la austeridad. Hágase cuenta de que Dios le da el miserable
puñado de metal que necesita para cumplir con el mundo; pero no se lo da
por su linda cara, sino a cambio de su alma, en lo cual se ve patente la
bondad infinita.»

No pudo dar por de pronto el pobre viejo más respuesta que un suspiro
hondísimo, y afilando luego su entendimiento, trató de acomodarse al
deseo y planes de la monja con eufemismos delicados y vaguedades
ingeniosas. En esto se les pasó una parte del camino, y cuando ya
avistaban la villa que lleva el nombre de la Codoñera, situada en
escarpado y agreste sitio, vieron venir por el sendero abajo a Tomé
despavorido y dando voces. Detrás de él venían los ancianos con menos
veloz carrera. Diole a D. Beltrán un vuelco el corazón, viéndose cercano
a un gran peligro, y así era ciertamente, pues Tomé gritaba: «¡Los
facciosos, los facciosos!»

No pasaron dos minutos sin que se viera justificado el pánico del chico:
a la revuelta del sendero aparecieron seis hombres, luego más de veinte,
y por fin un tropel de ellos, que a D. Beltrán se le antojó un grande
ejército. Todos traían boina y fusil, vestidos con un abigarrado
desorden enteramente contrario a la uniformidad.

Suspenso y aterrado, Urdaneta apretó los dientes, mascullando palabras
airadas y blasfemantes; los ancianos temblaban; Marcela, impávida, se
plantó en medio del sendero, mirándoles con sosegado rostro, en que no
pudo advertirse la menor alteración.

\hypertarget{xi}{%
\chapter{XI}\label{xi}}

Llegó de los primeros al grupo de los peregrinos un mocetón con zamarra,
chaleco rojo y polaina de cazador, blandiendo una espada, único signo de
su jerarquía de oficial en aquella desalmada tropa, y encarándose con
Marcela, descompuesto y groserote, le gritó con acento valenciano: «En
Mas Nuevo, la semana pasada, te dije que si te volvía a encontrar, te
fusilaba. ¿A dónde vas ahora? ¿Quién es este vejete?

---Voy a donde quiero, y este señor es quien es.

---No eches roncas\ldots{} Mira, Marcela, que me tienes frita la sangre,
y si te desmandas, cumplo lo que te ofrecí.

---¡Salvaje, mátanos cuando quieras!---exclamó Marcela con tanto desdén
como energía, lanzando un rayo de sus ojos.---Delante de las bocas de
tus fusiles, yo y estos santos varones te decimos: ministro de Satanás,
toma nuestras vidas, que Dios recogerá nuestras almas. ¿Ves estos
hombres humildes; ves este anciano, de la primera nobleza de Aragón, que
abandona su casa y honores por amor a la penitencia y los trabajos? Pues
ni él, ni yo, ni los demás, tememos la muerte. Moriremos, ¿verdad D.
Beltrán? Moriremos alegres, pidiendo a Dios que perdone a nuestros
verdugos.»

Tales manifestaciones de santidad heroica, y la fosquedad siniestra que
vio impresa en el rostro del cabecilla, persuadieron a D. Beltrán de que
había llegado su última hora. Miró en derredor suyo buscando a Tomé.
Lamentaba que la vida juvenil de su escudero fuese también sacrificada
en aquel lance. Pero el despabilado chico, al dar aviso a su amo de la
presencia de los facciosos, y antes de que estos se acercaran,
desapareció como un ave en la cercana espesura.

El bárbaro capitán contestó a las provocaciones de Marcela con estas
palabras: «Pues te juro que habremos de daros gusto. Sólo que como no
tenemos aquí capellán que os confiese, forzoso será llevaros al pueblo.
Y puesto que todos sois santos, preparaos unos con otros\ldots{} Ea, en
marcha.

---¿A dónde nos lleváis?---preguntó D. Beltrán, que con el ejemplo de la
monja procuraba reforzarse de serenidad y entereza.

---A la Codoñera.

---¡Pues a la Codoñera! Y ojalá estuviéramos a cuatro pasos de ese
pueblo donde hay capellán, que ya nos pesa el cuerpo, ya nos pesa la
vida, como hay Dios.»

Era el capitán un hombracho hermoso, atezado de rostro, gallardo de
postura, vestido con cierta bárbara elegancia de buen ver. Mandó a los
prisioneros que se pusieran en camino, los dos enterradores delante,
entre la tropa de vanguardia; detrás, junto a él, Marcela y D. Beltrán.
No debía de ser hombre tan fiero como Urdaneta creyó en los primeros
momentos, porque aproximándose a la peregrina por la izquierda de esta,
le dijo: «Todo esto te pasa porque quieres. Sabes que te estimo.
Marcela, quédate conmigo, que más has de valer señora sentada que no
monja andariega.

---Monstruo, prefiero que me maten de un tiro a morirme de
asco\ldots---replicó la religiosa sin mirarle.---No te acerques a mí.

---Me da la gana de acercarme, y te digo que si haces lo que te propuse
la semana pasada en Mas Nuevo, serás feliz; vamos, que no te
arrepentirás de ser mía, y se te quitarán de la cabeza esas murrias del
misticismo.

---En verdad te digo, Nelet, que los escarabajos, salamanquesas y
cucarachas que adornan el trono de inmundicia de Satanás, son menos
asquerosos que tú.

---Y yo digo que los ángeles negros, que negros los hay también, son
menos bonitos y menos salados que tú\ldots{} ¿Qué ojos hay como los
tuyos? ¿Qué boca se iguala con ese panal de santa miel? ¿Pues y ese
cuerpo que debajo de las estameñas se cimbrea como un junco con carne?
Marcela, si has olvidado que eres mujer, yo haré que lo recuerdes y te
alegres de recordarlo.

---¡A matar pronto! Abra el dragón sus fauces, y tráguenos. Extienda el
buitre su garra, y destrócenos. Somos de Dios, y a él van nuestras
almas.

---Si no me das la satisfacción de tenerte a mi lado, me consolaré con
el goce de fusilarte\ldots{} Es un gusto, créelo, un gusto fusilar a
quien se ama; así sabe uno que no ha de ser para otro\ldots{} y ver ese
lindo cuerpo retorciéndose\ldots{} y luego cogerlo uno, y meterlo en el
hoyo y agazajarlo con tierra\ldots{}

---¡A matar, a matar pronto!---repitió Marcela, iluminado el rostro, la
boca seca.---Morir por Dios, morir en la pureza, viendo cómo el alma se
aparta de tanta inmundicia, es la mayor gloria\ldots{}

---Bueno es que el señor capitán entienda que no somos espías---dijo D.
Beltrán, que al ver los afectos amorosos del cabecilla empezó a cobrar
esperanzas.---Nosotros íbamos por aquí a nuestros asuntos de penitencia
y a practicar las obras de misericordia.

---¡Por Dios, no sea usted cobarde, D. Beltrán!---dijo Marcela viva y
colérica, dándole tan fuerte pellizco que le hizo ver las estrellas.

---¡Hija!\ldots{} ¡ay! Pues bien valiente que soy, ya lo ves\ldots{}
¡Ay! tus dedos son tenazas. Me has arrancado un pedazo de carne. ¡Si yo
también quiero que me fusilen!\ldots{} Pero, francamente, si hemos de
morir, suprimamos los pellizcos.»

Antes de llegar a la Codoñera, se les unió el grueso de la partida. El
jefe, que debía de ser de graduación equivalente a la de comandante o
más, examinó a los prisioneros.

---¿Es usted Peinado?---le preguntó D. Beltrán plegando los ojos y
aproximándose.---Si es Peinado, no extrañe que por causa de mi corta
vista no le reconozca. Fuimos amigos hace años.

---No señor, no soy Peinado: soy Tena---dijo el cabecilla, hombre
pequeño y vivo que no carecía de formas corteses.---Peinado está con el
General en jefe.»

Y volviéndose luego a Nelet, le dio órdenes: «Llévales contigo, pero no
entres en la Codoñera. Por el atajo te corres esta tarde hacia Belmonte,
y de allí, sin parar, a Valderrobles, donde me juntaré contigo mañana
temprano. Yo tomaré la vuelta de Torrecilla, a ver si cojo la columna
del Marqués del Palacio\ldots{} A estos cuatro simples no se les fusila.
Si ella no fuera hembra, y ellos unos vejestorios, les daríamos
cincuenta palos\ldots{} Eso les vale. Llévales a Valderrobles, y yo les
recogeré allí. Ya sabes que me dijo D. Ramón que si otra vez cogíamos a
esta saltamontes, se la lleváramos. Quiere conocerla.»

No se habló más, y en marcha todo el mundo. Muy entrada la noche,
llegaron los prisioneros de Nelet a Valderrobles; y aunque en el camino
se les había dado algún reparo de alimento, D. Beltrán no podía tenerse
de hambre y fatiga; los huesos le dolían como si se los hubieran
machacado; pero más que la molestia física le agobiaba la desesperación,
resultante del tristísimo fin de su loca aventura. «Tenía razón
Estercuel---decía desplomando su humanidad sobre el suelo de un establo
vacío en que les encerraron.---En plena Edad Media. ¡Maldita sea la tal
Edad Media y el perro que la inventó!\ldots{} Esto es horrible, Dios
mío\ldots{} A tal desastre me ha traído una vana ilusión, más propia de
niño inexperto que de anciano sesudo\ldots{} ¿Y cómo salgo yo ahora de
esta cisterna en que me ha precipitado mi desatino? ¿Cómo me libro de
estos cafres?\ldots{} ¡Para que me fíe otra vez de santos y de
peregrinas, de monjas milagreras que entierran ollas. Esta tarasca me ha
perdido, y ahora no será ella quien me salve\ldots{} Merezco, sí, lo que
me pasa, por codicioso, por crédulo, por niño\ldots{} chocho\ldots{} ¡Ay
de mí! ¡Vaya una vejez, vaya un fin de la existencia! ¡Yo que vine por
proveerme del pan de la vida, y ahora me veo prisionero, amenazado de
muerte, envilecido entre esta canalla, y teniendo que aguantar sus
groserías\ldots! Me pegaría si no estuviera en tal estado, que no caben
ya más dolores en mi cuerpo miserable\ldots»

Esto se dijo, procurando el descanso. Por suerte suya, aproximáronse a
él otros viejos, y acumulado el calor de todos, hallaron alguna defensa
contra el frío. Por el opuesto lado se arrimó después Marcela, que
sentadita rezaba en alta voz, hasta que D. Beltrán, incomodado del
sonsonete, le dijo: «Rece para sí, hermana, que los viejos necesitamos
dormir. Sabe Dios qué mañana nos espera.» Y a la madrugada, sintiendo el
prócer un gran peso que le oprimía, y comprendiendo que era el cuerpo de
la santa mujer, que en el abandono del sueño se caía de aquel lado, le
dijo: «Suspéndase un poco, hermana, que me agobia todo el lado izquierdo
y no puedo respirar.» Retirose la monja lamentándose de haberse dormido,
pues su ánimo era velar la noche entera, y se aprovechó del empujón de
D. Beltrán para despabilarse y continuar sus rezos.

Lleváronles al otro día muy de mañana por el desfiladero de Beceite a
salir a la Cenia, camino endemoniado, propio de cabras y guerrilleros.
Intenciones tuvo D. Beltrán de pedir que le arrojaran en el camino para
que se lo comieran los buitres, o que le fusilaran de una vez, pues así
se acababan sus martirios. Compadecido el capitán Nelet, que no era mal
hombre, aunque de genio harto arrebatado, le dio de comer, socorriole
además con un trago de aguardiente, y por fin, le atravesó sobre la
carga de una mula que llevaba sacos de forraje. Marcela continuaba a
pie, y a ratos se aproximaba al anciano para dirigirle estos o parecidos
consuelos: «En opinión del Beato Padre San Juan de la Cruz, tratándose
de trabajos, cuanto mayores y más graves son, tanto mejor es la suerte
del que los padece.

---Déjame a mí de Padres Beatos de la Cruz---le contestó Urdaneta,---que
la que tengo sobre mí pesa bastante\ldots{} ¿Cómo quieres que me alegre
de esta situación? Dime que me resigne y hablarás con seso\ldots»

Al cansancio y tristeza de tal viaje, uníase el temor de que la columna
carlista encontrase otra de la Reina, y rompieran el fuego cogiendo en
medio a los infelices que no habían hecho armas ni por Carlos ni por
Isabel. Dos días pasaron en esta ansiedad sin que nada de particular
ocurriese; y al ver que descendían con precaución por ásperas
pendientes, D. Beltrán, sin poder apreciar por sí mismo el territorio,
entendió que iban hacia la Plana. En una aldehuela poco distante de
Albocácer, se agregaron a una numerosa tropa carlista, que más que
columna era ya división, y allí tuvo el pobre anciano la suerte de
encontrar un alma compasiva, un capellán que sin conocerle, o más bien
reconociéndole por su traza y modo de hablar caballero de nacimiento, le
prodigó atenciones y cuidados, acomodándole al fin en un carro de
provisiones, donde el pobre señor se creía transportado del Infierno a
la Gloria. «Dios no desampara a los buenos---se dijo,---y yo soy bueno,
aunque otra cosa crea esa bigardona volandera, que ahora se empeña en
meterme monje del Císter. Yo no hice nunca mal a nadie, como no fuera a
mí mismo\ldots{} ¡Fraile yo! ¡Y me da a escoger entre la cogulla y una
miseria deshonrosa!\ldots{} ¡Vive Dios! que puesto en tan horrible
dilema, no sé a qué carta quedarme.

Completó el capellán sus atenciones con la constante compañía, de que
sobrevino una real amistad. Llamábase Mosén Putxet, y era tortosino, un
si es no es ilustrado y muy corriente en todo. Contole que en aquellos
días habían trabado pelea, en los campos de Torreblanca, Cabrera y
Borso, llevando este la mejor parte. No cerradas aún las graves heridas
que en Torre de Arévalo le pusieron a la muerte, Cabrera recibió un
balazo en el muslo. A su serenidad y arrojo debió la salvación. Retirada
su gente a Cuevas de Vinromá, el caudillo se ocultó en la casa del cura
de la Jana, donde permaneció algunos días en lastimoso estado, febril,
exangüe. La suerte suya y de sus tropas fue que Borso no supo
aprovecharse de la victoria, y con su inacción dio tiempo a que Cabrera
se curase, como él lo hacía siempre, de prisa y corriendo; a que en el
lecho dictara disposiciones para rehacer su ejército; a que este, con
ligereza inaudita, le secundase, marchando de nuevo en busca de nuevos
triunfos, con su General a la cabeza, llevado en parihuelas.

«Por lo que usted me dice, nos encontramos en el cráter de un
volcán---observó Urdaneta,---y estoy a punto de presenciar sangrientas
batallas. Sea lo que Dios quiera. Sin duda es su voluntad que acabe yo
mis días en medio de estos horrores.

---A todo se acostumbra uno---le dijo Putxet.---Mire: los primeros días
no podía yo habituarme a la guerra; pero ya me voy haciendo a tales
crueldades, y pienso que Dios las consiente para que venga pronto el
triunfo de su religión santísima.»

No se atrevió el ladino viejo a manifestar al capellán lo que sobre esto
pensaba, temeroso de perder su amistad, que en aquella triste aventura
érale tan provechosa. Pasaba D. Beltrán en vela las noches, y gran parte
del día durmiendo, sin que supiera por qué adquirió en la molicie del
carro esta costumbre. Una tarde, hallándose en letargo dulcísimo,
después de comido y bebido con cierto regalo, soñó con terremotos,
incendios, erupciones de volcanes. Despertando de súbito, hirió sus
oídos horrísono estruendo de tiros, y echando la cabeza fuera del toldo,
vio que en una loma cercana estaban batiéndose facciosos y cristinos.
Ellos debían de ser; mas no distinguía el pobre señor las boinas y
morriones. Vio, sí, los fogonazos, y oía el murmullo de la pelea, como
la reventazón contra peñascos de las olas embravecidas.

Parado su carro junto a otros, poca gente vio Urdaneta en su derredor.
Tras el pánico primero, una esperanza risueña animó el afligido corazón
del anciano. ¡Si resultaba que la división o columna cristina que allí
peleaba era la brigada de Borso, y obtenía la victoria; si resultaba que
en dicha columna venía \emph{Mero}, y \emph{Mero} quedaba ileso, qué
felicidad! Muy hermosas eran estas ideas para que la realidad las
confirmase. Al ponerse el carro en movimiento, creyó Urdaneta que los
carlistas se pronunciaban en retirada. Mas ¡ay! era lo contrario. Los
cristinos abandonaban sus posiciones, y los facciosos iban sobre ellos
ebrios de furor. Así lo comprendió por las voces que de lejos venían,
por la alegría que en derredor suyo estallaba\ldots{} El carro avanzó a
retaguardia de los batallones victoriosos, y a poco vino la noche.
Pararon. Urdaneta preguntó: «¿Dónde estamos? ¿Cómo se llama el lugar
donde se ha dado esta batalla?» Respondiéronle con una retahíla
valenciana, de la cual no sacó nada en limpio. A poco de la parada, y
cuando repartían el rancho, oyó entre ásperas voces del dialecto alguna
castellana que anunciaba el fusilamiento de los prisioneros del ejército
vencido. A D. Beltrán se le erizó el cabello, y recostándose sobre los
sacos, se hizo un ovillo\ldots{} «Que me fusilen también a mí, Señor, y
así acabarán mis sufrimientos.» Pasado un rato, un extraño ruido le hizo
abrir los ojos. Vio a lo lejos fulgor de antorchas; sonaron luego
disparos, una descarga\ldots{} después otra y otras, a las que siguió un
lúgubre silencio. «¡Pobre \emph{Mero}!---murmuró el anciano:---que Dios
te acoja en su santo seno.»

Al poco llegó Mosén Putxet, y subiendo al carro, donde tenía las
alforjas, dijo a su amigo: «Estoy rendido; no puedo más. Cada lance de
estos es para mí una enfermedad grave.» Y sacando de las alforjas el
buen repuesto que llevaba, invitó a su compañero a participar de la
cena. Excusose el prócer con su falta de apetito, y el otro,
declarándose también inapetente, afirmó que sin gana tenía que
alimentarse, so pena de hallarse al día siguiente desfallecido. «¡A diez
y seis he tenido que confesar!---dijo con dolorido acento.---Esto es muy
triste. Las leyes de la guerra, implacables, lo conceptúan
necesario\ldots{} para asegurar el triunfo del Trono legítimo y de la
Religión veneranda\ldots{} Hace usted mal, amigo mío, en no tomar algo.
No se puede abandonar la nutrición del cuerpo. Verdad que usted, si
ahora no tiene gana, a cualquier hora de la noche tomará lo que guste.
Yo, pasadas las doce, no puedo, porque tengo que decir misa. Mañana es
domingo: por eso se ha determinado que la \emph{aplicación de los
castigos} se efectuara esta noche\ldots{} ¡Cruento sacrificio! ¡Maldita
guerra! ¡Que sea necesaria la destrucción de vidas para llegar a la paz,
y la injusticia para llegar a la justicia, y la crueldad para llegar a
la benignidad!\ldots{} Como usted ve, tengo que alimentarme. Son muchas
horas desde las doce de la noche a las diez de la mañana, hora de la
misa de campaña\ldots»

Terminada la cena, no tardó el capellán en dormirse. D. Beltrán velaba;
veló hasta el amanecer, engolfado en severas reflexiones. En tan triste
noche, precursora de días más tristes, el viejo aristócrata,
despreciando a su amigo, se sintió religioso, profundamente religioso.

\hypertarget{xii}{%
\chapter{XII}\label{xii}}

Al amanecer, cuando empezaba a conciliar el sueño, le llamaron\ldots{}
Creyó al pronto que iban a fusilarle, y dijo: «Vamos, estoy pronto.
Acabemos de una vez.» No tardó en enterarse de que le mandaban a
desempeñar una obligación harto triste: enterrar los muertos de la
hecatombe de la noche anterior; y si el primer impulso de su orgullo y
dignidad fue rechazar colérico un cometido tan impropio de su categoría,
luego dejó lugar el orgullo a la conformidad cristiana que había criado
en su alma con las meditaciones del pasado insomnio, y dando un gran
suspiro, dijo: «Vamos a donde ustedes quieran. Merezco esto y mucho más:
seré sepulturero.» La idea de que entre los cadáveres podía encontrar el
de Baldomero Galán, hizo flaquear un momento su entereza; pero logró
rehacerse con estas consideraciones: «Si está, ¡qué puedo hacer más que
llorarle! Contra los hechos, dispuestos o consentidos por Dios, nada
podemos. Enterraré al pobrecito \emph{Mero}, alférez y mártir.»

Terrible duelo y consternación produjo a D. Beltrán la vista de los diez
y seis cadáveres ya desnudos, rígidos en sus violentas contorsiones; y
como no podía reconocer al que buscaba sino acercándose mucho, a todos
les fue observando uno por uno, y tocaba los fríos rostros. Eran
jóvenes, lozanas existencias destruidas bárbaramente en la plenitud del
vigor. «No está \emph{Mero}---pensó, sintiéndose aliviado de un gran
peso.---¡Pobres muchachos! ¿Por qué se les ha quitado la vida? España se
desangra, España se aniquila. Asisto al suicidio de una nación.
Sepultémosla en su propia tierra\ldots» Cogió el azadón que le
destinaron, y se puso a cavar tan tranquilo. La resignación, con este
humilde trabajo, fue ganando más y más espacio en su alma, y con ella la
certidumbre de que sus desdichas venían del Cielo y eran el contrapeso
lógico de una vida de disipación y goces. Junto a él vio que cavaban los
dos sepultureros de la compañía de Marcela; pero ni ellos le hablaron,
ni D. Beltrán les dijo una palabra. Abiertos tres hoyos de gran
capacidad, fueron cogiendo muertos y arrojándolos dentro, unos sobre
otros, y después rellenaron y apisonaron. «Yo creí---pensó Urdaneta
cuando concluían,---que esto me impresionaría horriblemente. Pero a todo
se acostumbra uno. Me desayunaría yo ahora mismo, si me dieran con
qué\ldots» A su carro se dirigía con esperanza de encontrar las alforjas
de Mosén Putxet, cuando le mandaron al campo donde se diría pronto la
misa. «Pues a misa,» murmuró, declarándose pasivo, dispuesto a cuanto le
mandasen.

En el campo habían puesto el altar, al pie de un soberbio algarrobo,
vestido de espléndido follaje. El sol relumbraba esparciendo claridad y
alegría, y picaba más de la cuenta. Del Mediterráneo, que no se veía,
pero se adivinaba, venía una brisa suave y consoladora. Las tropas
formaban para oír misa; los jefes, a caballo, ocuparon su puesto delante
de los Cuerpos. Allí vio D. Beltrán al cabecilla Forcadell, uno de los
lugartenientes de Cabrera, y general de una potente división. Su rostro
inflado, risueño y de hombría de bien, lo mismo podía ser de canónigo
que de mayoral de diligencias. Pesaba sobre un poderoso alazán; vestía
zamarra peluda, chaleco blanco abotonado hasta el cuello; la boina era
de gran vuelo, blanca con fleco dorado. De una tienda de campaña salió
Mosén Putxet revestido: eran acólitos dos granaderos del 1.º de Tortosa.
Oyó la misa D. Beltrán con recogimiento, en el sitio que le designaron,
y no lejos de sí, por delante, vio a Marcela de rodillas y a los dos
sepultureros. Cuando alzaban, entre el estrépito de cometas y tambores,
el recuerdo de los pobrecitos que había enterrado le distrajo un poco de
su devoción. Mas rehaciéndose, a punto que se santiguaba, decía para sus
adentros: «Diablos de la guerra, mucho tenéis que llorar por vuestros
crímenes para que \emph{Ese} os perdone.»

De vuelta a su carro, Urdaneta preguntaba: «¿Pero dónde estamos? ¿Es
esta la Plana de Castellón?» y sin enterarse de la respuesta, tendiose
de largo a largo, y comiendo de lo que le dio el buen Putxet, pronto se
quedó dormido. Por la noche, el zarandeo del carro era fuerte y molesto,
señal de que iba de prisa por ásperos declives\ldots{} A la madrugada
vadearon un río de escaso caudal. La división, dotada de extraordinaria
presteza de pies y pezuñas, avanzaba tragándose las leguas. Al día
siguiente, vio D. Beltrán un pueblo que llamaban Olla; después otro que
nombraban Chestalgar o cosa así. A la siguiente medianoche pararon.
Creyó notar Urdaneta que se juntaba más tropa; no cesaban de sonar
tambores y clarines. Salió a estirar las piernas y dar un paseo\ldots{}
Putxet, cuando se retiró a dormir, díjole que estaban cerca de Buñol o
cerca de Siete Aguas, no lo sabía con certeza, y que Don Ramón,
acompañado de Llangostera, había venido a conferenciar con Forcadell. Al
amanecer oyose tiroteo por la parte que el noble cautivo, mirando las
estrellas, estimó como Levante. Por allí avanzaba una división cristina.
Serían las ocho cuando, no D. Beltrán que apenas veía, sino Putxet y
otro clérigo que con él estaba, observaron que las tropas de la Reina,
vigorosamente atacadas en terreno desfavorable, se desbandaban; que
luego se rehacían reforzadas por su caballería\ldots{} El tiroteo se
generalizó, llegando a ser continuo por la zona del Sur\ldots{} Dos
batallones carlistas salieron corriendo como demonios a embestirles por
el flanco. «¿Qué pasa, qué pasa, amigo Putxet?---preguntaba Don Beltrán;
y el clérigo le dijo gozoso: «La caballería no les vale esta vez\ldots{}
Allá va como deshecha y desmoralizada\ldots{} ¡Qué victoria,
\emph{¡pacho!} qué victoria!\ldots» Sobrevino luego un incidente que
determinó cierta indecisión\ldots{} Fuerzas carlistas retrocedieron. Los
carros tuvieron que alejarse. De la otra parte empujaban con furia.

De pronto vieron venir un gran tumulto, muchos jinetes que no corrían,
volaban, los ágiles corceles saltando zanjas y cercas, desmandados,
locos. Frente a ellos, en un caballo blanco, venía un hombre vestido de
colorines. Al pasar el jinete junto a la impedimenta, vieron los que
allí estaban su rostro, harto parecido al de un gato, los ojos
flamígeros, la color verdosa, henchida la nariz, como si las ventanillas
de ella quisieran rasgarse para dar paso al aliento. Su capa blanca con
vueltas rojas sujeta al cuello, ondeaba como una bandera. En la mano
blandía la espada; se le oía claramente gritar: \emph{Per así, fills
meus\ldots{} Seguidme\ldots{} Els destrosarem\ldots{} ¡Viva Carlos V!
¡Mueran eixos pillos, cobards!\ldots{}}

La tromba, que tal parecía, de innúmeros caballos, seguidos de tropel de
infantería, describió un vasto círculo por la llanura. Cuando se alejó,
la nube de polvo que levantaba impidió ver dónde había caído. Oyose un
chasquido formidable, como un desgarrón de masas enormes\ldots{} Los
carros hubieron de alejarse más, metiéndose en una hondura desde donde
poco se distinguía. «Este D. Ramón es tremendo---gritaba Putxet alzando
los brazos al cielo, ebrio de gozo:---menuda paliza les está dando. No
quedará uno para contarlo. ¡Viva el Trono legítimo, señores! Esta
brillante jornada nos abrirá las puertas de la hermosa Valencia, de la
reina del Turia\ldots{} Vean\ldots{} ya se disipa el polvo: por allí van
desbandados los de Isabel\ldots{} distingo perfectamente los
morriones\ldots{} ¡Hola, hola! la caballería parece que quiere volver
grupas y hacemos cara\ldots{} Ya es tarde, \emph{¡pacho!\ldots{}} ya es
tarde. D. Ramón, que es el dios Marte en persona, les da una carga
horrorosa, y se deja caer sobre el propio Buñol\ldots{} Adelante,
valientes. ¡Viva la Virgen de los Desamparados, nuestra Madre!»

Con estas exclamaciones, de un entusiasmo pueril, iba señalando el
clérigo las peripecias del combate que desde allí podían apreciarse.
Hacia el mediodía todo el ejército carlista iba sobre Buñol,
persiguiendo a los liberales fugitivos. «Me estoy temiendo---dijo Putxet
a su amigo tomando un piscolabis en la carreta en marcha,---que
tendremos función esta tarde\ldots{} Sería yo muy dichoso si, variadas
las condiciones en que hoy se hace la guerra, diéramos cuartel. Es
ciertamente más humano perdonar al vencido, ¿verdad?» Llegados a la
Venta de Buñol, se procedió con método, parsimonia y naturalidad a
fusilar a veintisiete oficiales y sargentos. Afortunadamente para
Urdaneta, no le mandaron a enterrar. Oyó los tiros, vio llegar a su
amigo desconcertado y melancólico, y nada más.

Dos días después, sabedor Cabrera de que una columna cristina andaba por
Alcanar, mandó contra ella a Llangostera, que la deshizo y fusiló
victorioso todo lo que quiso. Enterado también el fiero caudillo de que
el Capitán General de Valencia había salido hacia Castellón con fuerzas
para relevar las guarniciones del Maestrazgo, mandó a la Plana al
\emph{Serrador}, y desplegando una actividad increíble, prodigiosa,
organizó al propio tiempo la expedición de Forcadell a Orihuela. No
satisfecho aún con la victoria de Buñol, y habiendo recogido armas y
caballos, amén del fruto de las depredaciones en país tan rico, se fue
hacia Requena, simulando un amago a esta ciudad; mas no se detuvo hasta
Utiel: establecido allí su Cuartel general, apresurose a fortificar la
posición.

Estaba de Dios que en aquella parte de su cautiverio se agravaran las
desdichas del noble D. Beltrán, obligándole Dios con esto a mayor acopio
de paciencia; su amigo, el buen Putxet, se separó de él antes de llegar
a Requena, agregado a la expedición que invadir debía la tierra de
Alicante, y ya no disfrutó el pobre viejo el beneficio del carro sino en
contadas ocasiones, viéndose obligado a llevar peonilmente la carga de
sus añosos huesos. Sacando fuerzas de flaqueza pudo llegar a Utiel, el
calzado roto, los pies llagados, molido y hambriento, harto de trabajos,
incomodidades y miserias. Pero le bastaba considerar que más había
padecido Cristo por nosotros, para sacar alientos de su propio desmayo y
prepararse a mayores infortunios.

Metiéronle en un zaguán húmedo, y de allí le pasaron a una bodega, con
salida a un jardinillo petiseco, cercado de tapias; le acompañaban los
dos enterradores. De Marcela, ni estos ni D. Beltrán sabían dónde había
ido a parar. En el piso alto de la misma casa se alojaba, con otros
oficiales, el capitán Nelet, que viendo desde el balcón a los viejos
sentados en el jardinillo tomando el sol, dijo a sus amigotes: «No sé
para qué nos traen acá tales estafermos. Son tres bocas y ningún hombre.
O fusilarles, en el caso de que se compruebe que son espías, o echarles
a un camino para que se mantengan de limosna.» Como D. Beltrán mirase
para arriba, y con lastimero acento dijese que lo mismo le daba a él la
muerte que la mendicidad, mandole Nelet que subiera; obediente el
anciano subió la escalera con paso lento, tomando resuello a cada cuatro
peldaños, pues no podía de otro modo, y fue recibido en una sala por el
dicho Nelet y otros dos tagarotes. Entró Urdaneta con digno continente,
descubriéndose, y permaneció en pie esperando las órdenes de aquellos
bárbaros. Nelet, apoltronado en un sillón, y rascándose las
pantorrillas, le dijo: «¿Es cierto que es usted de la aristocracia?

---Sí, señor: me honro de pertenecer a la primera nobleza de Aragón.

---¿Es usted Marqués?

---Mis títulos son los Señoríos de la Torre de Albalate, de Olid, con
Grandeza, de\ldots{}

---Acabe usted, hombre, con esa letanía\ldots{} Pues mire: de algún modo
ha de ganar el pan que le damos.»

Diciendo esto, se quitó las botas llenas de cuajarones de barro, y
alargándolas al prócer, le dijo: «En aquel cajón hallará usted cepillo y
betún. Me las pondrá como un espejo.»

Permaneció un instante D. Beltrán con su mano extendida hacia las botas,
inmóvil y rígido, empeñada su voluntad en terrible lucha entre dos
movimientos: o coger las botas y estamparlas en la cabeza del grosero y
estúpido capitán, o resignarse a tanta humillación y aceptarla por los
méritos de Jesucristo. Prevaleció este último intento, y recibió con
noble pausa las botas, recogiendo luego los adminículos de embetunar.

«¿Fuma usted?---le preguntó Nelet, haciéndole retroceder desde la
puerta.

---Sí señor.»

Le ofreció un cigarrillo, y pareciéndole poco, le dijo: «Tome usted más,
para sí y sus compañeros, que la vejez entretiene sus tristezas con el
tabaco.

---Gracias.»

Y bajó el anciano tan gravemente como había subido, escalón por escalón,
sin decir nada, casi sin pensar nada\ldots{}

\hypertarget{xiii}{%
\chapter{XIII}\label{xiii}}

Ya por despistar a los cristinos, ya por otras razones o ardides
estratégicos, determinó Cabrera fortificar a Utiel, y lo primero en que
puso mano fue el Convento o Colegio de Escolapios y la iglesia
parroquial, gótica, de buena y sólida fábrica. Para despejar las
inmediaciones del primero de aquellos edificios, mandó demoler varias
casas y cortar todos los árboles de una alameda que al camino salía.
Empleáronse en tales obras noche y día multitud de hombres, y no hay que
decir que el Señor de Albalate y los dos ancianos fueron aplicados a
este trabajo. Vierais allí al primer noble de Aragón descargando
hachazos en los añejos troncos. Por primera vez en su vida era leñador,
oficio que le pareció menos innoble que el de sepulturero y limpiabotas.
El sargento que les mandaba y dirigía era por demás insolente y grosero,
de estos que se envalentonan con los humildes. Grande era la resignación
de Urdaneta, que se había propuesto tomar por modelo al patriarca Job;
mas hubo ocasiones en que se vio a dos dedos de perder su pasiva
actitud, por la fuerza explosiva de la dignidad aristocrática, que
romper quería sus cadenas, atropellando paciencia, humildad y
cristianismo. Viendo que aquel bruto abofeteaba inhumanamente a dos
infelices que no habían entendido sus órdenes, o que por exceso de
fatiga se mostraban perezosos, sintió el prócer vibración en todo su
ser, efecto de la honda crisis o lucha de opuestos sentimientos, y se
dijo: «Haré un esfuerzo sobrehumano por contenerme si ese gandul pone
sus manos en mi cara; pero dudo que pueda conseguirlo, pues antes de que
el corazón se humille, el estallido de mi dignidad hará que le parta la
cabeza de un hachazo.»

Felizmente, con él no se desmandó el bárbaro sargento; no hacía más que
rezongar, dar voces y decir a los viejos: \emph{El que no traballa no
menja; que aquí no estem para mantindre vagos}. Terribles hambres
pasaban los tres al volver rendidos a la bodega y patinillo en que
tenían su alojamiento. Nadie se cuidaba de darles de comer. El
enterrador que hablaba, y que tenía por nombre Pedro Zaida, salía en
demanda de alimentos; no hiciera lo propio D. Beltrán, prefiriendo
perecer de necesidad a pedir su ración; el otro, nombrado Alfajar,
tampoco pedía, por carecer de palabra. Así pasaron algunos días,
manteniéndose de mendrugos de pan y de sobras de rancho, que Zaida
recogía en los vecinos alojamientos, hasta que Nelet y los oficiales del
piso alto se apiadaron de la miseria de los prisioneros, y les mandaban
los restos de su comida. En un caldero bajaban la bazofia; de ella
comían los infelices viejos, siendo tan atentos Zaida y Alfajar, que
escogían para el señor los huesos vestidos aún de hilachas de carne, los
trozos de comida menos deshechos, y las que podrían llamarse golosinas,
reservándose para sí lo peor. «Hasta en esta región de miseria
bochornosa se encuentran seres delicados, se encuentran
caballeros---decía para sí Urdaneta, renunciando a tales preferencias, e
imponiendo el reparto equitativo de piltrafas. A menudo, en esta u otras
escenas semejantes, rodaban lagrimones por su cara. Una tarde salieron
los oficiales al balcón para verles comer. A poco llegó el asistente con
un pedazo de pastel en un plato y resto de bizcocho borracho, y
entregándolo a los cautivos, díjoles que aquello mandaban para el señor
Marqués. Luego volvió el chico con tres puros y el braserillo para
encenderlos. Fumaron, y dieron las gracias a los señores, que riendo les
miraban. Uno de los de arriba decía: «Ese Marqués del Cuerno paréceme un
grandísimo pillastre\ldots»

Don Beltrán calló, no haciendo al deslenguado ni el honor de mirarle.
Luego, a una insinuación de Nelet, que parecía dicha en defensa del
anciano, se retiraron del balcón los oficiales. Volvieron los viejos al
trabajo, que aquel día consistió en arrastrar los troncos hacia las
entradas y puertas de la villa, para armar con ellos estacadas o
parapetos. Cuando Urdaneta llegaba por las noches a su alojamiento y se
tendía en el frío suelo junto a sus amigos, sin más abrigo que las
pellizas de estos; cuando, después de cenar lo que Zaida trajese o de
arriba les mandasen, procuraba embriagar con el sueño sus infortunios,
se le iba el pensamiento a la gran casa de Cintruénigo, la casa de
Idiáquez, y hacía revivir en su mente el edificio y las personas, la
vida toda de aquella señoril residencia. ¡Ay! lo que allá tuvo por
humillación, era ya como una broma inocente. Modificadas por las
enseñanzas de la realidad sus ideas y opiniones, lo que en Cintruénigo
conceptuaba contrario a su decoro, ¿qué era? Nada en comparación de la
presente ignominia y miseria. Las estrecheces que allá estimó
intolerables, eran abundancias y delicias en parangón de lo de Utiel.
Recordaba con desconsuelo el orden de aquella noble casa, donde todo
estaba a punto, donde nada faltaba para comodidad y regalo de sus
habitantes.

Y pensando en esto, se le representaba su nieto: le veía niño, tan
cariñoso, tan dulce, tan formalito, tan amante de su abuelo\ldots{} Era
su propia sangre, encarnación de su nombre y nobleza\ldots{} ¿Qué haría
Rodrigo si le viese en tan extrema desdicha? La misma \emph{Doña
Urraca}, si viese a su suegro, el noble Urdaneta, sufriendo tanta vileza
y oprobio, comiendo sobras y migajas de la mesa de los oficiales, ¿qué
pensaría?\ldots{} Frente a su conciencia, que severa se encaraba con él,
reconocía el grave error de no tolerar las asperezas o defectos de los
convivientes, para que estos toleraran los suyos. Bien claro veía que
todas sus querellas con la familia eran por motivos que ya se le hacían
vanos, pueriles. Veía también toda la fealdad de su soberbia, causante
principal del malhadado viaje a tierra de Teruel; veía su codicia, su
afán de atesorar dineros, que en su edad provecta casi no le eran
necesarios. Pero amaba el rumbo y quería ser siempre amo y señor,
dispensador de mercedes. ¡Bien le castigaba Dios, y cuán gallardamente
te aplicaba su justicia severa!\ldots{} Y mirándolo bien, no era
Rodriguito tan digno de menosprecio y rencor. Poseía todas las
cualidades que a su abuelo le faltaban. Actos de verdadera maldad, nadie
podía señalar en él. Y en cuanto a la impertinente, mandona y
atrabiliaria \emph{Doña Urraca}, sus defectos no eran motivo para
aborrecerla, Señor.

Estas reflexiones, en que se confundía la turbación de la conciencia con
la dulzura de las memorias de familia, le habrían llevado al sueño
reparador, si no lo estorbaran las picazones de su cuerpo, el sentirse
acribillado por atroces punzadas que parecían mordidas. Daba vueltas a
un lado y otro, y rascándose contra las durezas del suelo, volvían sus
reflexiones a distraerle del acerbo picor. «¡Vaya, que si Juana Teresa
conociera la cama en que duerme el padre de su difunto esposo, lloraría
de lástima; sí que lloraría!\ldots{} ¡Ella que cifra su orgullo en la
limpieza ideal de las camas, ella, en quien más que gusto es manía el
tenerlas pulcras, inmaculadas, como las vestiduras de los
ángeles!\ldots{} No hay en el mundo sábanas y almohadas como las de mi
casa de Cintruénigo: huelen a manzanas, a violetas, a algo más oloroso
que las flores, el aseo\ldots{} Si Juana Teresa y mi nieto me vieran en
esta inmundicia, llorarían\ldots{} ¡pobrecitos de mi alma!\ldots{} y no
sólo llorarían de compasión, sino de rabia por no poder remediarlo.»

Salía Cabrera con mil o dos mil hombres, los más de los días, como en
diversión militar, para hostilizar a Requena y figurar su propósito de
ponerle sitio. En una de estas excursiones, al regresar del campo
entrando por la puerta de Caudete, donde se trabajaba para hacerla
infranqueable, apeose del caballo y examinó las obras. Con seca frase
autoritaria hizo la crítica de lo que no le parecía bien; indicó los
defectos y el modo de subsanarlos con el menor trabajo posible. Viendo
avanzar a D. Beltrán, que a duras penas sustentaba una espuerta de
tierra, dio algunos pasos hacia él y le preguntó si era el caballero
Urdaneta.

«Para servir a usted, General---dijo el anciano, mirándole atento y sin
descargarse la espuerta.

---Lleva usted mucho peso\ldots{} \emph{Eh, tú, Lleuiset, no carregues
masa a eixe pobre home, qu' es un señor poch acostumat a traballs. Sous
molt brutos, y no teniu ni pizca de criteri ni talent, ¡caramba! Es
precis que sapian distinguir entre un home y un señor. A atres que son
burros de veritat, els trateu como si foren señorests, y no teniu
llástima d' este pobre vell, acostumat a anar sobre alfombres.»}

Comprendió el anciano que hablaba en su favor; y como al propio tiempo
le quitaran la pesada carga que llevaba, murmuró una frase de gratitud.
Cabrera no se hartaba de mirarle, fijándose últimamente en sus pies y en
las destrozadas botas. También D. Beltrán contempló a sus anchas al
afamado guerrillero, a quien vio por primera vez en el campo de Buñol,
pasando como un rayo al frente de infernal cabalgata. Reconoció en él la
cara de soberbio gato, que ya había visto, y quedó grabada en su
memoria: cara triangular, de pómulos salientes, ojos grandísimos y
negros con la ceja corrida, la nariz de mala forma con las ventanillas
siempre palpitantes. Vestía con elegancia y cierta presunción de
originalidad, no escaseando en su ropa los dorados y relumbrones; la
capa blanca con forro encarnado completaba su típica figura. Con militar
saludo se despidió para entrar en el pueblo. Por la noche, hallándose
los tres viejos en el patinillo, comiendo de las sobras enviadas por
Nelet, llegó un ordenanza que se puso a gritar en la puerta: «¿Quién es
aquí el Marqués?\ldots{} ¡Eh, Marqués!

---Yo soy, buen amigo---dijo Urdaneta, que respondía por aquel
título:---¿qué se ofrece?

---Pues aquí me manda el General con estas botas---dijo el chico
mostrando un par no muy nuevo, pero en buen estado.

---¡Ah\ldots{} ya!\ldots{} para que se las limpie\ldots{} Bien: déjalas
ahí.

---No es para que se las limpie, jinojo, sino para que se las
ponga\ldots{} Ya veo que le hacen falta. El General le manda estas, que
no se pone ya, y para usted están que ni pintadas; todavía en buen uso.
Ya le miró a usted la pata, y sabe que le vendrán bien.

---¡Oh!\ldots{} ¡Dios!---exclamó el aristócrata, decidiéndose a recoger
el regalo.---¿Y el General se acuerda de este infeliz?\ldots{} Dile que
estoy muy agradecido\ldots{} ¡Oh, botas de la paciencia, de la
humillación, venid a mis pies!»

Y cuatro días después, hallándose en Cheste, emprendida la marcha
sigilosa de todo el ejército hacia el llano de Valencia, fue sorprendido
D. Beltrán por un recado del General llamándole a su presencia en la
Casa Ayuntamiento, donde se alojaba. Allá se fue el noble viejo, y
encontró a D. Ramón en una estancia del piso bajo con trazas de escuela
pública, por los cartelones de letras gordas que colgaban de las
paredes. Estaba el caudillo de sobremesa con dos mujeres guapísimas, de
nacarada tez y ojos hechiceros, ataviadas a estilo popular. Los
\emph{caragols} sobre las sienes, cruzados por ganchos de oro; el moño
de trenzas, atravesado por las agujas, ofrecían el clásico modelo del
peinado valenciano. En sus orejas llevaban los arcaicos \emph{polques}
de oro con esmeraldas y perlas barrocas, joyas de apariencia bizantina,
y en el cuello hilos de aljófar. Toda la vestimenta, de tisú, era lujosa
y elegante dentro de la más escrupulosa propiedad. Sin verlas más que
como imágenes borrosas, o como bocetos de admirables pinturas, D.
Beltrán, olfateando belleza con su especial nariz de perito en mujeres,
las diputó por grandes señoras disfrazadas de campesinas ricas.
Sentábanse a izquierda y derecha del General, muy arrimaditas; luego
seguía un capellán, que parecía granadero, y al otro lado un cabecilla,
en quien, por la facha y rostro de clérigo afligido, creyó reconocer D.
Beltrán a Llangostera.

Sospechó el noble aragonés, no sin fundamento, que Cabrera le llamaba
para mostrarle a sus amigos como un objeto de curiosidad, como un ente
raro, consistente la rareza en el vivo contraste entre tanta nobleza y
miseria tanta. Mas no era este el único móvil del llamamiento: había
otro, que el General expresó después de contestar al cortés saludo del
caballero: «Pues le he mandado venir para advertirle que\ldots{} esté
preparado\ldots{}

---¿Preparado a qué, General?

---Haría usted mal en creer que le tenemos aquí por gusto de su
co\ldots{} mpañía---dijo Cabrera, que hablando familiarmente
tartamudeaba un poco: su lengua, disparándose en articulaciones
rapidísimas, tropezaba a cada instante.

---¿Para qué debo prepararme, General?

---El sistema de represalias, que, como usted sabe, es obra de esos
infames cristinos, me obliga a la crueldad con\ldots{} contra los
sentimientos de mi corazón.

---Ya entiendo. Es para fusilarme. Bien preparado estoy. Esta vida que
arrastro, señor, vale tan poco para mí, que el quitármela, más que de
cruel, le acreditará a usted de piadoso.

---Yo lo siento\ldots{} sabe Dios que lo siento. Co\ldots{} mpadezco a
los que me veo precisado a sacrificar\ldots{} Me duele, aunque mis
enemigos crean otra cosa y me llamen tigre\ldots{} Pero yo digo: todas
las inocencias del mundo juntas no valen la inocencia de mi madre.

---Aunque no temo la muerte, mi conciencia, mi respeto a la verdad, me
obligan a declarar que ni soy espía, ni he venido a esta tierra con
ningún fin político ni militar.

---Sé que no es usted espía. Me lo ha dicho la monja Marcela, que me
merece crédito\ldots{} Pero aquí cobramos vidas por vidas, y pagamos
muertes con muertes. ¿No se ha enterado usted de que la división de
Iriarte ha cogido prisionero al hermano del Conde de Catí, vocal del
Consejo de Su Majestad en este Reino?\ldots{} Pues en cuanto sepa yo que
le han fusilado, ya está usted de más en el mundo. ¿No le parece que
esto es natural, justo y equitativo? Noble por noble, caballero por
ca\ldots ballero.»

Mientras esto decía el implacable soldado, no se oyó una voz, ni un
murmullo, que indicaran protesta contra tanta barbarie, siquiera
compasión. O la costumbre de tales horrores embotaba en hombres y
mujeres todo sentimiento humanitario, o no se atrevían a manifestarlos.

«¿Puedo retirarme ya?---dijo el viejo sin hacer comentario a la terrible
conminación.

---Espere un poquito\ldots{} y sáquenos de una duda. ¿Es usted Marqués
de Sariñán?

---No señor: el Marqués de Sariñán es mi nieto, por enlace de mi hijo D.
Federico con una dama de la casa de Idiáquez.

---¿Ven como yo acertaba?---dijo una de las mujeres o damas disfrazadas,
por lo que comprendió Urdaneta que habían tenido discusión sobre su
personalidad.

---Y los títulos de usted ¿cuáles son?---preguntó el clérigo.

---Soy Señor de la Torre y Casa-Fuerte de Albalate, Señor de Rubielos,
Merino mayor de Monzón, poseedor de varios lugares, fortalezas, vasallos
y pechos en el antiguo reino de Sobrarbe; Señor también de la Puebla de
Olid con Grandeza de España, Caballero del hábito de Montesa, Maestrante
de Zaragoza\ldots{} y no sigo por no ser enfadoso a los que me
escuchan\ldots{}

---¿No es usted pariente de los Cárceres?---preguntó la otra hembra
bonita.

---Sí señora---replicó D. Beltrán, gozoso de oír la dulce voz, cuyo
timbre le sonó a nobleza y elegancia.---Ramón Cárcer, cuarto Marqués de
Castelbell, es mi sobrino, y primos de mi esposa son los Borrás y
Mezquita, así como Marianito Zagarriga, Marqués de Creixel.

---Otra cosa---dijo Cabrera, a quien ya parecía enojoso hablar tanto de
nobleza.---¿Qué tal le tratan a usted en mi Cuartel general? ¿Le dan
bien de comer?

---Señor, un ejército de campaña no puede cuidar del pobre cautivo
inútil, cuya vida no importa a nadie.

---Yo quiero que sea usted tratado con la co\ldots{} nsideración que
merece por su categoría\ldots{} Y si alguno le faltase al respeto, lo
que tarde yo en saberlo tardaré en ordenar que le den cincuenta palos.

---No vale hoy esta pobre vida que por ella se machaquen los huesos de
un cristiano.

\emph{---¡Pobre señor! Em dona molta llástima! ¡Y en quina dignitat
porta la seua miseria!»}

Algo pudo entender el prisionero de lo que la compasiva dama decía, y su
piedad le llegó al alma. En tanto Cabrera le ofreció un cigarro, que
rehusó, porque no solía fumar a tales horas\ldots{} Instó el General;
insistió la dama, que de manos de su amigo tomó el puro para alargárselo
a D. Beltrán. Cuando este salió del aposento, iba como fascinado por la
voz claramente oída y el rostro turbiamente visto de la beldad, y echaba
de menos sus verdes años para corresponder a la compasión de ella con un
amor grande, solitario y sin esperanza, como aquel inmenso infortunio de
su vejez.

\hypertarget{xiv}{%
\chapter{XIV}\label{xiv}}

Mejor tratado desde aquel día, el prisionero vio urbanidad y
benevolencia en algunos rostros; pero nada le maravilló como la radical
mudanza del capitán Santapau, a quien conocía por el familiar nombre de
Nelet. Empezando por mostrarse con él menos esquivo, se humanizó en un
día, en otro se trocaron sus asperezas en afabilidad cariñosa, y acabó
por declarar a D. Beltrán su sentimiento de haberle ofendido y su deseo
de trabar con él amistad. Aceptó gustoso este cambio de actitud el buen
viejo, y sospechando que alguna recóndita intención se traía su flamante
amigo, esperó a conocerle mejor para juzgarle. Respecto al paradero de
Marcela, a quien había perdido de vista desde antes de la acción de
Buñol, díjole Nelet que Cabrera la había mandado encerrar en un convento
de monjas, hasta que decidiera el Vicario General por D. Carlos, que
actualmente se hallaba en Navarra. A juicio de Cabrera, no era decoroso
ni ejemplar que una señora religiosa anduviese al zancajo por los
caminos, suelta de toda disciplina; pero Santapau no participaba de esta
opinión, pues las benedictinas de Sigena estaban exentas de clausura,
como había declarado nada menos que el Concilio de Trento. Conocedor del
monasterio y de su poética historia, el capitán había estudiado el
asunto, y podía demostrar a su jefe la razón y derecho con que
peregrinaba la santa señora y esposa de Cristo. Marcela Luco.

«Bien, hijo, bien---dijo D. Beltrán, barruntando a dónde iba a parar el
guapo Nelet.---También yo veo con simpatía la libertad monjil, y en este
caso la creo muy acepta a los ojos de Dios, pues, si no me engaño,
Marcela corretea en seguimiento de intereses que quiere aplicar a
grandiosas fundaciones pías, para mayor esplendor de la Fe y de la
Iglesia.»

Decían esto camino de Valencia, como a tres leguas de Chiva, donde
habían pernoctado. Las intenciones de Caín llevaba Cabrera en aquella
marcha, pues informado por sus espías de que los restos de la división
de Crehuet, derrotada tres días antes en Buñol, andaban por aquel
término, iba en su seguimiento, bien afiladas las uñas para
destrozarlos. ¡Espléndido país aquel, hermoso cielo, alegres campiñas,
que aun en invierno dan testimonio de su fecundidad! Aspiraba D. Beltrán
el templado aire, que por el aliento metía en los cuerpos la vida, la
esperanza, el contento del vivir; que duplicaba el vigor de los jóvenes,
y a los viejos les aliviaba el peso de los años. Pensaba que aun para
despedirse de la existencia es bueno un suelo feraz, un ambiente
templado, una tierra pródiga en flores y frutos.

Los mil doscientos cristinos de Infantería y el escuadrón de Lanceros,
que, con los milicianos de Valencia y Liria, habían recibido órdenes de
concentrarse en la capital, marchaban confiados, mal dirigidos,
desconociendo con angelical inocencia el país que pisaban y el enemigo
que tan cerca tenían. Como unos borregos de Dios se entregaron al
descanso en un pueblo llamado Pla del Pou\ldots{} Cuando más descuidados
estaban, vieron encima la caballería carlista. No les dio tiempo ni para
tomar posiciones, ni siquiera para escapar con algún orden. No fue
batalla, fue una carnicería sañuda: desordenada la caballería cristina,
se enredó en ella la infantería, como una deshecha madeja en las patas
de un animal que da vueltas sobre sí mismo. Los carlistas no combatían;
mataban a su gusto y satisfacción. Los liberales no eran soldados, sino
reses. Algunos de a caballo pudieron escapar; los pistolos que no
perecieron en la matanza, entregáronse a discreción, para que los
matarifes hicieran de ellos lo que quisiesen. Por de pronto, allá iban
todos, prisioneros y vencedores, hacia Valencia, y ya que para embestir
a esta grande y fuerte ciudad no tenía Cabrera poder bastante, se plantó
en Burjasot, lugar cercano, para verla al menos y que ella le viese.
Aunque de escaso relieve, la eminencia en que está fundado aquel pueblo
es como atalaya que domina la huerta feracísima, y a lo lejos el
apretado caserío de la ciudad, guarnecida del verdor perenne de los
naranjos, y destacando sus torres y chapiteles sobre una espléndida faja
de mar azul.

Tan contentos llegaron a Burjasot los soldados del absolutismo, que no
pensaron más que en celebrar su triunfo con la vena de abundancia que
aquella lozana tierra les ofrecía. Guerreros infatigables que devoraban
leguas y corrían de una comarca a otra con presteza gatuna, traían
hambre atrasada. El país donde comúnmente operaban, Maestrazgo, Desierto
de las Palmas, riberas del Palancia y Mijares, riberas del Guadalope y
Río Martín, puertos de Beceite y de Ademuz, estaban ya esquilmados.
Valencia era el oasis, la frescura, el descanso, la vida plácida con
regalos mil. No fue de iniciativa de Cabrera, como se ha creído, el
festín de Burjasot; fue idea de algunos jefes, y de la oficialidad y
subalternos, que ya anhelaban comer y beber sin tasa para reponer el
cuerpo de tantas fatigas. Bien se lo habían ganado: lo menos que podían
hacer era consagrar un día, unas horas a dar a sus cuerpos algún goce de
gula, pues todo no había de ser marchas, hambres y sofoquinas. Pedido
permiso al General, este lo dio de buena gana, porque si sabía utilizar
hasta la última tira de pellejo de sus soldados, también gustaba de que
se divirtiesen y solazaran cuando la ocasión lo permitía.

Parte del vecindario invadió el campamento, metiéndose entre la tropa.
Iban unos por afecto a la causa carlista; otros por curiosidad; muchos
por ofrecer y colocar hortalizas, carne, peces, patos, frutas y hasta
flores, que ya abundaban en aquel despuntar de la primavera. Habían
dispuesto celebrar la comilona en aquella parte culminante del pueblo,
formada de terreno calizo, bajo el cual se extienden los famosos silos o
graneros subterráneos para depósito de cosechas. La iglesia de San
Roque, objeto de gran devoción, situada también en la eminencia y no
lejos del pueblo, encara su frontis hacia Valencia y el mar, como
recreándose en tan bello panorama.

Pronto se vio la vasta planicie llena de cuanto Dios crió, viandas
regaladas, viandas adquiridas. Se nombró una comisión que cuidase de
allegar cucharas y tenedores, algo de mantelería y vasos para los jefes,
y el obsequioso vecindario facilitó al instante todo cuanto se deseaba.
Por aquí se encendían hogueras; por allá preparaban peroles y sartenes;
en un grupo de soldados desplumaban patos; en otro desollaban corderos.
Subían del pueblo en hombros zafras de aceite y pellejos de vino, cestos
de naranjas, rimeros de lechugas. Soldado había que en estos acarreos se
atracaba de forraje, como aperitivo. El vino empezó a correr desde el
primer momento, vaciando los pellejos en jarros, estos en los pocos
vasos que había para tantas bocas. Los carlistas más señalados en la
localidad por su fanatismo subieron sobre sus duros cráneos grandísimas
mesas, y montones de sillas, enganchadas traviesa con pata. Manteles
también vinieron, aunque no tantos como habrían sido menester. Toda
escasez se podía perdonar menos la del vino, que se remedió duplicando
la provisión de hinchadas corambres.

A las tres y media el aspecto de la bacanal era imponente: comían,
devoraban sin orden ni medida, la tropa en el suelo, diseminada en
grupos a los bordes de la meseta; los sargentos sentados también en
tierra, formando cuadros con relativa corrección; más allá oficiales,
unos de rodillas, otros \emph{ensillados}, algunos tendidos a la romana.
Frente a la ermita había mesas, donde se veía la figura clerical de
Llangostera y la cara de corcho de Tallada, en la cual se confundían la
picaresca malicia y la ferocidad. Otras personas calificadas se veían
por allí: el subdelegado castrense, del cual podían ser retrato los
odres de vino que acababan de traer; intendentes, cirujanos, mariscales
mayores. Los capellanes se señalaban por su ausencia, pues una grave
ocupación les retenía en el pueblo. Cabrera, mal humorado, sintiendo
algún recrudecimiento de sus achaques, y molestia en sus mal cerradas
heridas, se sentó un rato en la primera mesa; después iba de una parte a
otra, hablando con todos, recibiendo felicitaciones. Las miradas se le
iban hacia Valencia; apretaba las mandíbulas cuando sus íntimos le
decían: «D. Ramón, estamos a las puertas del cielo\ldots{} Haga una de
las suyas, y llévenos allá.»

En las clases inferiores reinaba una jovialidad frenética. Grupo hubo en
que empezaron por los postres, las dulces algarrobas; luego
descuartizaban un pato, tirando en cruz de las patas y alones. Aquí
comían las lechugas sin aliñar, en rama; allí naranjas a bocados
mordiendo la cáscara, y encima pescado frito, o a medio freír; vino sin
tasa; después bollos de aceite, y lonjas de tocino con azúcar. En las
mesas o tenderetes de preferencia hubo arroces quemados, arroces crudos,
anguilas, pajeles, pájaros y hasta morcillas; en otros comían el cordero
a medio asar, chorreando sangre, partiéndolo con las espadas, por no
abundar los cuchillos. El regimiento 1.º de Tortosa tenía una murga
militar de una docena de instrumentos, trombones abollados, bombo,
platillos y chinesco. Agregados a ella algunos músicos cogidos a las
tropas de la Reina, compusieron una mediana banda, la cual, desde los
comienzos del banquete, tocaba \emph{escogidos trágalas}, la jota y
otras piezas de baile. Su discorde ruido hacía juego con los manjares a
medio condimentar y con la desafinada alegría del festín. Aquí y allí
gritaban: «¡Que se callen esos perros!» y tenían razón, pues los de la
banda eran verdaderos sicarios del arte musical.

Casi a la fuerza fue llevado D. Beltrán por Nelet a uno de los grupos
que comían en el suelo; y apenas se había sentado, viendo que el capitán
se retiraba, le dijo: «¿Pero usted, Santapau, no come?» A lo que
contestó Nelet, condolido de sí mismo: «Ahora no puedo: tengo que
fusilar.

---¿Pero qué?\ldots{} ¡Ahora!\ldots---exclamó aterrado el viejo,
levantándose de un brinco, inverosímil para su edad.

---¿Pues qué creías tú, abuelo?---dijo un teniente, que desde el
principio de la comida estaba entre dos luces.---¿Creías que les íbamos
a perdonar\ldots{} y a convidarles encima?»

Antes de que pudiera contestarles, resonó el estruendo de una
descarga\ldots{} Corrió Don Beltrán hacia donde la humareda se veía, y
distinguiendo los desnudos bultos de cadáveres junto al tapial del
cementerio contiguo a la iglesia, lanzó una exclamación de horror y se
llevó las manos a la cara. Veinte infelices habían caído ya. A poco
trajeron otra cuerda: eran veinticinco, entre ellos los cadetes
valencianos que acababan de ingresar en el ejército, y se estrenaban en
aquella tragedia. Venían en cueros, resignados, los menos con pocos
ánimos, tropezando en el camino; los más altaneros, provocativos.
Algunos de ellos, alargando sus brazos hacia la embriagada turbamulta
del festín, gritaron frenéticos\ldots{} «¡Viva Isabel II!» La descarga
les cortó la palabra y el fervor de sus exclamaciones; luego, los tiros
sueltos para rematarles sonaban a cacería. Excitados con los vivas
insolentes de las víctimas, la soldadesca entregada a la gula prorrumpió
en gran vocerío aclamando a los suyos, escarneciendo a los vencidos, que
no tenían bastante con la muerte. Mientras traían otra cuerda del
cercano corral donde les desnudaban, en la explanada vaciaron más
pellejos. Los vacíos yacían en el suelo como cuerpos despanzurrados,
sanguinolentos. En algunos grupos, donde con la borrachera se había
perdido hasta el último destello de razón, gritaban: «Más, más.» ¿Qué
pedían? ¿Más bebida o más muertes? Las dos cosas: vino bautizado con
sangre.

Soldados del Serrador y de Tallada cogían entre dos los muertos, por
pies y cabeza, y los iban arrojando a un lado, formando montón. Las
gentes del pueblo, que al principio de la matanza se aproximaron con
instintiva curiosidad y querencia insana del terror, huían ya
despavoridas. La musiquilla seguía lanzando su chillar bufonesco en
medio de la melopea espantosa de tal tragedia, declamada por los fusiles
de una parte, de otra por los ayes lastimeros o los arrogantes
apóstrofes de las víctimas. Si pavoroso era el estruendo de las
descargas, no lo era menos el graznido lúgubre de la banda o murga y el
coro desenfrenado y soez de los que comían, bebían y pateaban sobre el
propio Calvario\ldots{} Movido de inmensa compasión, de un sentimiento
de protesta contra tanta barbarie, se fue D. Beltrán con paso torpe
hacia donde fusilaban\ldots{} Le entró el delirio de unir un grito suyo
al de los que gritando morían. No sabía por dónde andaba\ldots{} Una
mano vigorosa le apartó diciéndole: «¿A dónde va, buen hombre? Atrás, o
le coge una bala\ldots» Retirose, metiendo los pies en un charco de
sangre\ldots{} Vio los cuerpos desnudos retorciéndose en el suelo, y la
presteza con que los remataban, como quien extermina una plaga de
animales dañinos. Huyó el pobre señor horrorizado, sin saber a dónde iba
a parar; y más abatido por efecto del pavor que del cansancio, se dejó
caer en tierra. Una nueva descarga, alaridos, vivas y mueras, y el coro
de los bebedores, que ya era ronco, con voces arrastradas, grotescas,
llevaron al colmo su espanto. Se tapaba los oídos: sus miradas buscaban
en el movimiento de los grupos algo que indicase la terminación de la
matanza; pero nada veía. El humo cubría la hecatombe. Volviendo sus ojos
al cielo, ansiando ver algo que borrase de su espíritu la impresión de
tales horrores, contempló un instante la inmensidad azul. calmosa y
pura.

\hypertarget{xv}{%
\chapter{XV}\label{xv}}

No había concluido la función. Despachados los sargentos y oficiales,
empezaron a exterminar soldados. De arriba gritaban: «¡Más, más\ldots{}
todos!» Y los que se acercaron a Cabrera intentando convencerle de que
el escarmiento no debía pasar de allí, oyeron de él la fría respuesta:
«Hoy no les niego nada.» El General, molestado por horrible acedía, y
con su boca llena de un amargor insano, el rostro lívido, la mirada
menos brillante que de ordinario, no había tomado más que un poco de
vino con agua. Su inapetencia habría necesitado quizás, para remediarse,
espectáculos menos terribles; o era que ni aun con los triunfos
recientes se hallaba satisfecho, y su insaciable ambición pedía más al
adusto Genio que le protegía. En medio de las alegrías del festín y de
los horrores de la matazón, más que matanza, su espíritu se distraía de
la realidad presente, para volar hacia la ciudad cercana, bella y rica.
Los ojos se le iban hacia allá, como si contar quisiera las torres y
cimborrios de la que solemos llamar \emph{ciudad del Cid}. ¡Qué no daría
aquel nuevo dominador de pueblos por poderla llamar suya! Mirándola con
ojos de codicia más que de amor, parecía decirle: «Ya ves cómo trato a
mis enemigos. Permito a mis soldados que hagan esta pira de cadáveres,
para que en ella veas a Cabrera. Aquí estoy; mírame; quiero que tiembles
mirándome, quiero que toda España tiemble ante mí.»

Terminados los fusilamientos, un amigo de Nelet recogió a D. Beltrán,
atontado de la fuerza del susto, y le llevó a su alojamiento. A prima
noche, Nelet le hizo acostar, dándole vino caliente, y el pobre señor,
con los cuidados que su amigo, antes enemigo, le prodigaba, descansó del
molimento y de la pavorosa impresión, despertándose al toque de diana
con regular apetito y el espíritu fortificado de resignación, así
cristiana como filosófica. Vivía en los dominios del terror trágico y en
las fronteras de la muerte: cuando llegara para él la hora del martirio,
sabría, pues, afrontarlo con valor y dignidad.

Desayunándose con los restos del banquete, las tropas se pusieron en
marcha muy temprano, dejando intacta la pila de muertos para que los
enterraran los vecinos de Burjasot, si querían; algunos batallones se
aproximaron a Valencia simulando un ataque. El amago, sin más objeto que
amedrentar al vecindario, significaba un \emph{¡si voy\ldots!} Pero no
iba: para tal empresa no bastaban la audacia y la agilidad. Contentábase
Cabrera con aumentar su hueste, con organizarla y darle hábitos y
educación de ejército poderoso; sus crueldades no eran el nefando goce
del mal, como en el depravado cura Lorente: eran los resortes de una
infernal política, pues en su conocimiento del país y de los hombres, el
leopardo no veía más camino que la fascinación terrorífica para domar a
los pueblos. Destruyendo media España, aseguraba el imperio sobre la
otra media.

Hecha la demostración ante los muros de Valencia, emprendió Cabrera con
su ejército la marcha hacia la Plana de Castellón, sin decir a nadie a
dónde iba ni qué planes llevaba. Santapau, recién ascendido a
comandante, mandaba el 3.º de Tortosa, y en su estreno de plaza montada
brindó a D. Beltrán con la participación de su cabalgadura, llevándole a
la grupa en todo aquel caminar, que no fue de los más acelerados.
Dispuso el jefe una marcha por la margen derecha del Palancia, como si
quisiera embestir a Segorbe; descendió inopinadamente hasta Sot de
Ferrer; pasó el río, y a los dos días de lo de Burjasot, pernoctaba en
Alfandeguilla. Afirmose en tan larga correría la amistad entre D.
Beltrán y Nelet, ganando este con delicadas confianzas el corazón del
anciano. A poco de emprender la primer jornada, y observándole taciturno
y receloso, díjole que el General había manifestado, respecto a su noble
cautivo, sentimientos de benevolencia y estimación. La verdad de esto
demostráronla los hechos, pues en la parada que hicieron en Rafelbuñol,
presentándose la noche lluviosa y fría, Cabrera mandó a Don Beltrán un
capote suyo en buen uso para que se abrigase. Cuidaba en tanto Nelet de
apartar para él la mejor comida, y en los alojamientos le agenciaba toda
la comodidad posible. Tanta era en Urdaneta la gratitud como la
confusión, y llegó a sospechar que tales obsequios significaban un
refinamiento de crueldad, y que le regalaban como a los condenados a
muerte antes de quitarles la vida. Descansando y comiendo al pie de unos
robustos algarrobos, después de pasar el Palancia, Nelet intentó
quitarle de la cabeza los temores de fusilamiento, diciéndole que tal
vez Cabrera le retenía con fines muy distintos de los que supone la
prisión por rehenes. No comprendía el viejo qué fines podían ser
aquellos, dada su inutilidad, y ambos estimaron que el noble señor debía
esperar los acontecimientos, tomando lo que le dieran, comiendo de lo
mejor que hubiese, y abriendo su espíritu a la confianza.

«Dispuesto estoy---dijo Urdaneta,---a comer todo lo que me traigan, y a
ponerme la ropa del General, si continúa mandándome algunas piezas
útiles. Pero mi espíritu no puede estar sereno, pues no se aparta de mi
mente la matanza de Burjasot. Soy cristiano; protesto en silencio de
estos horrores, y pido a Dios que los castigue.

---Lo de Burjasot---replicó Nelet con fría naturalidad,---no es otra
cosa que una hilada más de la pirámide de justicia que juró construir D.
Ramón, hallándose en Valderrobles, en Febrero del año pasado. Esa
pirámide no es aún bastante alta para que pueda lucir en su cima la
imagen de aquella santa mujer, María Griñó\ldots{} Pero ya tocan marcha.
Andando, señor mío. Vamos a Nules, que es plaza nuestra. Yo le aseguro a
usted que allí tendremos ocasión\ldots{} y además motivos de hablar
largamente.»

A las diez de la mañana del siguiente día fue recibido Cabrera en Nules
con arcos de triunfo, cortinas, músicas y danzas populares. Salieron a
felicitarle y a ofrecerle ramitos de flores las chicas guapas del
pueblo; huelgas y merendonas tenían dispuestas los calificados, y por la
tarde corrida de toros en la plaza. En buena casa fue alojado Don
Beltrán, y tanto él como Santapau, tratados a cuerpo de rey. Salió el
comandante a obligaciones del servicio y a diligencias privadas, de que
su amigo no tuvo conocimiento hasta la tarde, en la ocasión y sitio que
pronto se sabrá. Comieron opíparamente, y cuando toda la oficialidad y
el Estado Mayor a la plaza se encaminaban para ver la función de toros,
Nelet propuso al anciano que, pues ellos no eran aficionados al barullo
y tenían algo que platicar, se fueran a dar un paseo por donde menos
ruido hubiese de festejo y de muchedumbre. Conforme en ello Urdaneta, se
metieron por calles y travesías buscando la soledad, que fácilmente
encontraron, por estar todo el golpe del vecindario en la corrida.

La villa, de construcción arábiga, blanca, de suelo plano y fácil, les
engañó con la tortuosa red de sus calles; y cuando creían haber andado
poco, halláronse lejos, en un arrabal separado del pueblo por anchas
acequias. Metiéndose por entre dos tapias, fueron a dar frente a una
iglesia de frontispicio blanqueado con excepción de la puerta de piedra,
barroca, de columnas salomónicas, de retorcidos follajes y garambainas.
«Como está usted cansado---dijo Nelet,---y esta iglesia nos brinda con
su soledad y silencio, tan a punto para el descanso como para la buena
conversación, entremos, señor D. Beltrán, y aquí hablaremos todo lo que
nos dé la gana.

---Dígame, compañero---indicó el viejo cuando Nelet, llevándole de la
mano, le metió en la iglesia y se sentaron los dos en un banco.---¿Es
que yo me he quedado completamente ciego, o que está esto más obscuro
que boca de lobo?

---No tema por su vista. Yo tampoco veo nada. Venimos deslumbrados de la
calle. Aquí nadie nos molesta ni nos oye. Voy a mi cuento, empezando por
decir a usted que el hombre más desgraciado del mundo, el más digno de
lástima, es el que con usted habla en este momento. Pensará usted quizás
que mis penas son obra de la imaginación, a lo que contesto que, aun
admitiendo esa idea, no dejan de ser efectivos, terribles, insoportables
los sufrimientos de su servidor. ¡Con decirle que en Burjasot, cuando
mandaba los fusilamientos, envidiaba a los pobres que allí matábamos
como moscas\ldots!

---Pasión de ánimo se llama esa enfermedad; y ella debe de ser motivada
por una mala impresión, por un vivo querer no satisfecho.

---Ya pone su dedo en mi llaga\ldots{} ¡y cómo me duele! No me equivoqué
al pensar que usted, hombre muy corrido, que ha vivido en esas
sociedades de tono, buen conocedor de hombres y mujeres y de todo el
tinglado social, es el único para confidente, quizás para médico de mis
males.

---¡Yo!\ldots{} Tate\ldots{} tate\ldots{} Amigo Nelet, o soy un niño
inocente, o es causa de sus desdichas ese trastorno del alma, a veces
del cuerpo, que llaman amor.

---Entre paréntesis\ldots{} Ya principio a distinguir los
altares\ldots{} ¿No hay allí dos viejas?

---No, señor: son dos sillas.

---Me da en la nariz, Nelet amigo, que esto es convento de monjas. He
sentido a mi espalda como un murmullo, como un roce de faldas\ldots{} y
un cierto olor de incienso de monja\ldots{} que es un olor eclesiástico
muy particular\ldots{} ¿Me equivoco?

---No señor.

---¿Está aquí detrás el coro?

---Y al través de la verja parece que veo un par de bultos
blancos\ldots{}

---Bueno, siga usted\ldots{} ¿Con que amor? Y admito, sí señor, que
pueda yo ser médico de tal achaque por mi consumada experiencia, por lo
que han visto estos ojos, por los innumerables afectos de diferentes
clases que han turbado este viejo corazón. Adelante, y abreviemos:
¿quién es ella?

---Antes de saber quién es ella, sabrá usted quién es él. Manuel
Santapau, nacido en un \emph{mas} próximo a Gandesa, de padres
labradores ricos, no debió a estos una crianza perfecta. Hijo único, sus
padres no supieron enderezarle desde niño por los buenos caminos, y en
vez de contener su natural voluntarioso, le dejaron tomar vuelo; sus
travesuras hacían gracia, y sus sinrazones eran alabadas antes que
reprendidas, resultando que cuando unas y otras, con la edad empezaron a
ser maliciosas, ya no había autoridad que las contuviera. En fin, señor:
yo, desde los diez y seis años, escandalicé la villa en que vivíamos,
que era entonces Gandesa, y más tarde hice campo de mis abominaciones a
Reus, a Vendrell y a Cambrils. Ausente de la casa de mi padre, salvo en
las pocas en que iba a reponer mi bolsa, me lanzaba yo con otros amigos
no menos inclinados a la vagancia, de pueblo en pueblo, cometiendo
tropelías sin fin. Mis estudios, que no pasaron de leer y escribir y
algo de cuentas, se completaron después en el libro del mundo, donde
aprendíamos toda la ciencia del mal. Era vasto nuestro terreno, y en él
ejercíamos diferentes artes malignas; pero la peor de estas, y en que yo
principalmente despuntaba, era la de seducir doncellas con mil engaños
para abandonarlas luego miserablemente. Si robábamos alguna vez en
ciudades o despoblados, era por modo de travesura; nuestro botín
consistía siempre en jamones y morcillas, aves y otros comestibles, y
jamás tomamos dinero de nadie. Esta es la verdad; y así como digo lo
malo, digo lo bueno o lo menos malo. Alguna muerte tuvimos sobre
nuestras conciencias, todas en riña, a veces por defendernos de padres
burlados, a veces por pendencias de ésas que, sin saber cómo, salen del
vino\ldots{} porque, eso sí, a borrachos y camorristas, nadie nos
ganaba. Aunque me esté mal el decirlo, mi buena figura era la mejor
ayuda de mi perversidad en la campaña de conquistar mujeres, embobarlas
y perderlas sin ninguna compasión. El demonio, que no Dios, me había
dado el rostro para enamorar y las palabras dulces y mentirosas; y con
tales medios, cada día era yo más terrible acosador del sexo femenino,
llegando a no respetar ya soltera ni casada, seduciendo también por
depravación a las que no eran bonitas, y a las religiosas, a las altas,
y a las bajas y a las medianas\ldots{}

---Perdone usted, Nelet---dijo D. Beltrán, que no podía contener las
ganas de interrumpirle.---El tipo de D. Juan, que existe desde el
principio del mundo y es de todas épocas, tiene en la nuestra, por lo
muy reglamentada que está la sociedad, poco terreno para sus audacias.
Se lo dice quien ha visto mucho mundo; quien, si se pusiera a contar
lances y aventuras donjuanescas, no acabaría en siete meses. Y yo
pregunto: ¿cómo pudo usted ejercer tan largo tiempo de caballero
seductor, sin tropezar con la justicia que le metiera en la cárcel, con
un padre que le descalabrara, o un marido que le partiera por la mitad?

---Lo encontré, sí señor: tuve mi castigo. Un marido, de Tortosa, me
cogió desprevenido una noche, y con una barra me abrió la cabeza.
Después agarrome por una pata y me tiró a una acequia, donde me habría
ahogado si esta llevara más de medio palmo de agua.

---Acabáramos\ldots{} Reconozca usted que ya era tiempo, querido
Santapau.

\hypertarget{xvi}{%
\chapter{XVI}\label{xvi}}

---Sí, era tiempo\ldots{} Yo me lo tenía muy bien merecido. Por poco no
lo cuento, señor D. Beltrán. Me recogió el santero de una ermita que hay
en Roquetas, y a su caridad y a la de su mujer debo la vida. No sé
cuántos días me tuvieron en aquella cueva, debajo de la iglesia, donde
había unos santos viejos tirados en el suelo, con las caras comidas de
polilla, y toda la pintura y la estofa de sus trajes descascaradas por
la humedad. Uno de ellos, que era por las trazas San Antonio de Padua,
pero sin niño, pues este y las manos se le habían quemado en un fuego de
los altares, se puso en pie una noche, y llegándose a mí, me
habló\ldots{}

---¡Nelet!\ldots{}

---Le veía y le oía, Sr.~D. Beltrán, como a usted le oigo y le veo.
Díjome que Dios estaba muy enojado conmigo por mis grandes pecados, y
que en castigo de haber yo perjudicado a tantas pobres mujeres
fingiéndoles cariño mentiroso, me pondría en el alma un amor
violentísimo y verdadero hacia persona que nunca me había de querer, y
con esta pasión no satisfecha, y con este fuego no apagado, padecería
todo lo que hice padecer a las mujeres que engañé.

---Soñó usted, en verdad, un ejemplo precioso de justicia y expiación.

---Verdadero o soñado, fue un aviso del cielo, según me dijo el fraile
mínimo con quien me confesé al siguiente día, porque yo estaba
arrepentido, sentía como un pestilente sabor de boca, la suciedad de mi
conciencia, y quería limpiarla. Meses después, el mismo fraile de
Roquetas (ya exclaustrado), que miraba por mi salvación espiritual y
corporal, me aconsejó que me alistase en la facción y peleara por los
derechos santísimos del Altar y del Trono. Así lo hice a fines del 35;
presenteme a Cabrera, que me recibió muy bien, y para que me fogueara me
mandó a la partida de Quílez, después a la de Tena. Gracias a mi arrojo
en los combates, a mi puntualidad en el servicio, adelanté bastante en
mi carrera. Era ya alférez y me hallaba en Valderrobles, en Febrero del
año pasado, cuando los monstruos liberales dieron muerte a la madre de
Cabrera; teníamos en rehenes en el dicho Valderrobles a cuatro señoras:
la esposa del coronel Fontiveros; Mariana Guardia, hermana de un urbano
de Beceite; Paca Urquiza y Cinta Foz, hermana y madre de otro urbano. D.
Ramón las trataba con mucho miramiento, convidándolas a su mesa algunos
días, y cortejaba a la Paquita: se corría la voz de que era su novio por
lo fino y que se casaría con ella. Pero cuando supo la muerte de María
Griñó, el furor de aquel hombre fue tal, que juró al cielo derramar
sangre inocente hasta anegar los valles y volver rojos los pequeños y
los grandes ríos. A mí me tocó el paso amargo de fusilar a las cuatro
mujeres. La Mariana Guardia me gustaba, y bromeando le había dicho yo
cuatro cuchufletas de tentación, picado de mi antiguo vicio\ldots{} Al
ponerlas de rodillas en el cuadro, después de confesadas por el Padre
Vallés, el mismo frailecico que a mí me auxilió en Roquetas, los pobres,
llorando como Magdalenas, me pidieron por Dios que no las matase. Pero
yo ¿qué había de hacer? La disciplina, que es más fuerte que la
conciencia, me hizo de hierro el corazón\ldots{} Murieron\ldots{} A
Mariana tuvimos que rematarla, porque con los tiros primeros no quería
morirse, y sus ojos se cuajaron, echándome una mirada que me
traspasó\ldots{} Ello fue que sentí luego un frío mortal, y al poco rato
caí con tremenda pataleta y convulsiones, blasfemando y clavándome las
uñas en el rostro\ldots{} Por la noche, hallándome en un catre, donde me
pusieron con los brazos atados para que no me golpeara, vino el demonio,
y cogiéndome por los cabellos me llevó a un alto monte que llaman
Cretas, y allí\ldots{}

---Alto, amigo---dijo D. Beltrán:---esa no cuela\ldots{}

---Por que no cree en ello. Pero yo sí, y sostengo todo lo que he dicho.
Tan cierto como que estamos aquí, lo es que me vi en el picacho de
Cretas, entre una caterva de demonios que allí estaban congregados; y
después de zarandearme jugando conmigo a la pelota, me mandaron que les
adorase, a lo que yo no accedí, y pusiéronme delante toda mi historia,
representada en las figuras de las mujeres que perdí y ultrajé, las
cuales iban pasando como las estampas de un libro\ldots{} Ni por esas me
conquistaron; y cuando el demonio mayor, o capitán de ellos, me volvió a
mi catre, arrojándome en él medio muerto, llamé al Padre Vallés, que me
consoló, haciéndome aprender de memoria oraciones que bien rezadas
ahuyentarían los espíritus malignos.

---¿Pero cree usted eso, pobre Nelet?

---¡Que si lo creo!---exclamó el guerrero con una convicción tan
profunda y tenaz, que D. Beltrán juzgó inútil emplear contra ella las
armas de la razón.---¡Pues si fuera tan cierto que he de salvarme!

---Siga, y lleguemos pronto al punto principal: ¿quién es ella?

---Ahora sale\ldots{} Restablecido de aquel mal demoníaco, de cuando en
cuando venía por mí el diablo que quería ser mi amigo, y me llevaba por
los aires, o al fondo de las cuevas que hay en la Portadilla, o a los
breñales espesos del río Nonaspe, en lugares adonde ni los búhos
penetran. Era el mes de Agosto, y me hallaba con el \emph{Fraile
Esperanza} en Calaceite, de vuelta del Mas del Hortal, donde nos
habíamos batido con Nogueras, cuando me encontré, sin saber cómo, frente
a una caverna, en noche cerrada\ldots{} y oí una música preciosísima,
que no puedo comparar a ninguna música de este mundo.

---Sobre todo a la de la banda de Tortosa.

---De la gruta salió una luz azul, muy suave\ldots{} y por fin, de en
medio de esta luz una mujer\ldots{} No puedo dar idea ni de la luz ni de
la hermosura de la señora, ni sé cuál de las dos cegaba y confundía más.

---Sería rubia\ldots{}

---No señor; morena, de ojos negros, el pelo suelto y corto, caído sobre
los hombros con infinita gracia, la mirada como de los santos en
oración, los pies desnudos, el cuerpo vestido de un sayal de
penitente\ldots{}

---Verde y con asa\ldots{} Marcela\ldots{} Ya me figuraba yo que en esto
habían de venir a parar todas esas jugarretas diabólicas\ldots{} Bueno,
¿y le dijo usted algo?\ldots{} ¿ella le habló?

---No señor\ldots{} palabras no hubo; nada más que el quedarme yo
estático, como sin sangre en las venas, la voluntad sobrecogida, y
sentir que toda la vida la tenía en el corazón, y que en él se me metió
un amor muy vivo y abrasador que de aquí no ha querido salir más.

---Pero se me ocurre una grave objeción. Fíjese usted en las fechas
antes de lanzarse a referir sus leyendas, Nelet. Ha dicho que en Agosto
fue la maravillosa visión. Pues en Agosto, según mi cuenta, Marcela no
había salido aún de Sigena, ni podía presentársele en esa traza de
penitente\ldots{}

---Pues ahí está lo maravilloso, lo sobrenatural, que confunde a los que
sólo creen y testifican las cosas ordenadas conforme al tiempo y a la
verdad que se toca. Yo vi a Marcela antes de que ella adoptase la vida y
hábitos de peregrina. Y en esta anticipación de las cosas advierto que
es ella la destinada por Dios a la obra del terrible castigo que quiere
imponerme, condenándome a una sed no saciada, y a un afecto no
correspondido.

---Bueno; concretemos. ¿Dónde vio usted a Marcela en realidad\ldots{} de
ella misma?

---En la Ginebrosa, y no me sorprendió el verla, pues ya la conocía por
su aparición, que he referido.

---¿Le habló usted?

---Le pedí amores, y me contestó muy esquiva, huyendo de mí. El segundo
encuentro fue en Nuestra Señora del Pueyo. Le hablé con galantería fina
y discreta que salía del corazón, y me dijo que no sentía por mí más que
asco y desprecio. Yo iba mandando una partida; en mi desesperación se me
ocurrió fusilarla, para matar con ella mi tormento\ldots{} Pero no me
atreví. Despidiéndola, le dije: «Vete, hechura de Lucifer, a donde yo no
te vea más, que si otra vez te cruzas en mi camino, te fusilo sin
compasión\ldots» Parecíame que sacrificándola me libraba de mi suplicio,
y que después podía seguir queriéndola hasta que me muriese o me
matasen\ldots{} Darle muerte no me parecía crueldad, sino una forma de
amar, a mi manera, estilo de gran pecador y visionario de cosas
grandes\ldots{}

---¿El tercer encuentro\ldots?

---De él fue usted testigo.

---¡Ah!\ldots{} En la maldita Codoñera. Tiemblo de recordarlo\ldots{} De
lo que sigue tengo noticia, y la última es que Cabrera la mandó a un
convento, porque no gusta de monjitas correntonas.

---Sí, señor\ldots{} y el convento donde está encerrada es este.

---¡Este! ¡Valiente pillo!---dijo D. Beltrán levantándose y dando
algunos pasos hacia el coro.

---Cuidado, señor\ldots{} que no nos conviene llamar la atención.

---Como si lo viera. Los tratos de usted con los demonios ya sé yo en
qué vendrán a parar, caballero Nelet---indicó el prócer, volviendo al
banco.---Estamos preparando una hazaña donjuanesca: violación de
clausura, rapto de virgen del Señor\ldots{} Pero entendámonos: ¿trata
usted de sacarla por su gusto, por el orgullo de robar una monja, o
porque ella le ha dicho: `Nelet, ¿cuándo tocan a robar?'?

---Ella no me ha dicho eso; pero constándome que le agrada la libertad,
hace días, por un propio muy listo que mandé a Nules, le propuse abrirle
las puertas de su encierro, y me contestó que en ello no había pecado,
sino observancia de las disposiciones del Concilio de Trento. El
capellán del 3.º, hombre muy leído, me ha prestado unos librotes en que
están la fundación e historia de Sigena, y con esa lectura mi conciencia
no se escandaliza del hecho de libertar a Marcela. Estoy tranquilo; he
tomado mis medidas\ldots{}

---Todo esto, mi querido Nelet---dijo Don Beltrán reverdecido,---es
hermoso, es poético y dramático. De la esencia de estas aventuras de
amor vive el alma\ldots{} Por tales emociones y otras semejantes, no es
el mundo un presidio. Dígame usted\ldots{}

---Ahora, mi ilustre amigo, no puedo decir más, porque tenemos que
separarnos. Es la hora precisa de ver a la demandadera, la cual ha de
darme de palabra o por escrito una razón, por donde sabré si la empresa
se acomete esta noche o se deja para la de mañana. Aguárdeme aquí, que
no estará solo más que el tiempo que yo tarde en esta diligencia.»

Mientras estuvo solo, Urdaneta se dio a reflexionar en el extraño caso,
que a su parecer justificaba el dicho del teniente Estercuel. La guerra,
el país, la raza, renovaban en todo los tiempos medievales. La vida
tomaba esplendores poéticos y risueñas tintas que se mezclaban con el
rojizo siniestro de la sangre, tan sin medida derramada. Exceso de vida
era quizás, plétora de sentimientos y pasiones. El fondo, por añadidura,
ofrecía característica decoración natural y el teatro más adecuado a tal
desbordamiento de vida. La mezquina civilización \emph{a la moderna} se
desvanecía, se borraba como un afeite mal aplicado, dejando sólo las
querellas feudales, el ardor místico, la superstición, las crueldades
horrendas y eminentes virtudes, el heroísmo, la poesía, la intervención
de ángeles y demonios, que andaban sueltos y desmandados por el mundo.

Volvió Nelet gozoso al cuarto de hora, y cogiendo del brazo a su amigo
le llevó fuera, a punto que un monaguillo a cerrar se disponía. «Y qué,
¿será esta noche?---preguntó el anciano, taconeando fuerte por el
puentecillo de la acequia.

---Aún no he leído su carta---replicó Nelet, que de la fuerza del
contento temblaba.

---¡Ha escrito!\ldots{}

---Y además me manda unos versos. Vámonos aprisa, que por el ruido que
se oye y la gente que se ve venir hacia estos barrios, paréceme que ha
terminado la corrida. Esta noche, después que yo lea la carta,
seguiremos hablando\ldots{} Aún me queda lo mejor. Porque yo no le he
contado a usted a humo de pajas mis desgracias y aspiraciones. Yo he
visto en el Sr.~D. Beltrán de Urdaneta, noble de antiguo cuño, caballero
muy corrido y de grandísima ciencia en cosas mujeriles, la única persona
del mundo que puede guiarme al fin que deseo tanto como mi salvación:
que Marcela me ame, que pueda yo, triunfando de su esquivez, dar al
traste con la leyenda de mi castigo, que me espetó San Antonio en la
ermita de Roquetas.

---Yo tendré un placer inmenso---dijo el aragonés, parándose para
hacerse oír mejor,---en ilustrar a usted con mis conocimientos en
materia tan grave. El corazón de la mujer no tiene secretos para mí:
ciencia dolorosa, amigo mío, porque los maestros no llegamos a este
doctorado sino a fuerza de amarguras y sufrimientos. En mí tendrá usted
un asesor desinteresado; pero deje aparte toda consulta referente a
espíritus más o menos diabólicos, pues yo no los he visto en mi vida, ni
sé nada de esos caballeros. Eliminadas las potencias infernales, yo le
aconsejaré lo más eficaz para conquistar el corazón y la voluntad de esa
doncella\ldots{} ¡Y que no es floja bestiecilla la que hay que
domar!\ldots{} Santa y arisca, filósofa y hombruna\ldots{} Pero ya
veremos, ya veremos\ldots»

Llegaban al centro de la calle Mayor, donde se aposentaban, y ya no
pudieron hablar más de su asunto, porque Nelet se vio rodeado de
compañeros y amigos. Todos ellos, y D. Beltrán no de los últimos,
pensaron en matar el hambre, lo que no era difícil entre un vecindario
casi totalmente afecto a la Causa, y que se desvivía por obsequiar a sus
defensores. En los bajos del Ayuntamiento, las estancias habían sido
convertidas en comedores, donde se agolpaban oficialidad, capellanes y
calificados vecinos del pueblo, mientras en el piso alto se hacían
regios honores al General y a su Estado Mayor. Los compañeros de Nelet
se acomodaron en una salita próxima a la escalera, donde se les dispuso
espléndido comistraje, con mariscos y pescado, arroz exquisito y otros
manjares de grande estimación. Con no poca estrechez se fueron
acomodando, no sin designar un puesto al noble cautivo. Mas no había
tomado la primera cucharada de sopa, cuando entró un ayudante del
General con este mensaje: «El señor de Urdaneta, que suba. El General le
convida a comer.

---¡A mí!\ldots{} ¿Está usted seguro de que\ldots?

---Vamos, dese prisa. Están aguardando por usted.»

\hypertarget{xvii}{%
\chapter{XVII}\label{xvii}}

La entrada de D. Beltrán en la sala del festín, donde ya ocupaban sus
asientos los comensales; el despejo y cortesía con que, adelantándose
hacia el General, compendió en una sola frase el saludo y las gracias
por el honor que se le dispensaba, cautivaron a todos los allí
presentes: bien se veía al aristócrata de raza, maestro en arte social.
Con raras excepciones, los jefes carlistas que se sentaban a la mesa del
General eran unos pobres gaznápiros, elevados por sus prendas militares
a posiciones de las cuales desdecía su educación. Tal coronel había sido
arriero, el otro pescador; sacristán, uno de los intendentes;
contrabandista de mar y bandido de tierra el jefe de la caballería, sin
que ninguno de ellos poseyese el genial instinto con que Cabrera supo
borrar de sus modales la humildad de su origen. Mal vestido y roto, D.
Beltrán descollaba entre aquella gente, que aun en el modo de mirarle
revelaba la conciencia de su inferioridad. Hubo uno, vecino de Nules,
que menos avisado que los demás, se permitió decir al prócer: «Vamos,
abuelo, que no estará usted poco \emph{inflao}. En toda su vida ha
tenido honor como este\ldots{} ¡comer con nuestro General ilustrísimo!

---Honor grande, que agradezco mucho---replicó D. Beltrán;---pero que no
es nuevo para mí. Yo he comido con Napoleón.»

Esto de comer con tan grande celebridad produjo estupor, que se fue
trocando en admiración. A lo largo de la mesa sonó un murmullo. Cabrera,
hombre muy desahogado en toda circunstancia, mandó a Cala y Valcárcel,
sentado a su izquierda, que desocupase el puesto, y haciendo una seña al
caballero aragonés, le dijo: «¡Con Napoleón!\ldots{} ¿Luego era usted su
amigo? Véngase a mi lado para que me cuente\ldots» Trocados los
asientos, ocupó Urdaneta la izquierda del General, y accediendo a sus
deseos, prosiguió así: «No debí decir Napoleón, sino Bonaparte, porque
ello fue antes de la primera campaña de Italia. Él tenía entonces menos
edad que tiene usted ahora; era delgado, melenudo\ldots{}

---Sí, sí---dijo Cabrera con admiración infantil,---poseo su retrato en
la \emph{Vida de Napoleón} con láminas, que he leído cien veces, pues no
ha existido hom\ldots{} bre en el mundo que yo admire más.»

Refirió D. Beltrán escenas y pasajes interesantísimos de 1795 y 96, años
IV y V de la República (¡ya había llovido!), por él presenciados, y
añadió anécdotas graciosas, más atractivas que la historia misma; y con
tal agrado Cabrera lo oía, que hasta se le olvidaba el comer por no
perder concepto ni palabra.

Y entre cuento y cuento, viéndose el aristócrata tan obsequiado, se
decía, comiendo tranquilamente: «Tanta finura me da muy mala espina,
pues de este tío no hay que esperar compasión: cuando se le hinchan las
narices, ni hay para él amigo, ni tienen valor alguno sus atenciones y
arrumacos. No puedo olvidar lo que me ha contado Nelet. A las cuatro
desdichadas mujeres que en rehenes tenía en Valderrobles, las sentaba a
su mesa, prodigándoles obsequios mil. A la de Fontiveros le permitía dar
paseítos en una jaca, que aparejaron para ella, y a la chica de Urquiza
le hacía el amor por lo fino con tanta insistencia, que hasta corrió la
voz de que se casaban. Todo lo cual ¡Dios mío! no impidió que las
mandara fusilar al saber la muerte de la Griñó. ¡Vaya un nene! Y no hay
que hablar de arrebato, pues Cabrera supo lo de su madre el 20, si no
estoy equivocado, y la matanza de las rehenes fue el 27. Sentenciadas
días antes, no las mandó ejecutar hasta que no supo que sus dos
hermanas, presas en Tortosa, se habían escapado\ldots{} No me fío,
leopardo, no me fío de tus halagos, y aunque me pases por el lomo la
pata blanda, con las uñas escondidas, sé que las tienes muy afiladas, y
que el mejor día, cuando más tranquilo esté el pobre Don Beltrán\ldots{}
¡pum! al otro mundo\ldots{}

---¿Por qué suspira usted?---le preguntó Cabrera.---¿Está descontento
del trato que le damos?

---¡Oh! no señor. Estoy muy satisfecho y muy agradecido. Encuentro
simpatías en su ejército, y en él he podido hacer algunas amistades
gratísimas. Pero bien sabe usted que la privación de la libertad
difícilmente halla consuelo.

---Es muy sensible---le dijo el leopardo hacia el fin de la comida o
cena,---que la ley de guerra, que no puedo eludir, no puedo\ldots{} me
obligue a tenerle a usted bien trinca\ldots{} ado en mi ejército, para
que su vida me garantice la de otro aristócrata que tiene en su poder
Iriarte\ldots{} Pero usted podría ahorrarme a mí el disgusto\ldots{} ya
me entiende, y al propio tiempo salir de esta situación molesta\ldots{}
sí señor, comprendo que es car\ldots{} gante eso de estar un hombre con
la idea de que le van a pegar cuatro tiros\ldots{} Sí señor, usted
podría\ldots{}

---Te veo venir---pensó el anciano, antes de preguntarle cuál era el
remedio de su angustiosa incertidumbre.

---¿Por qué el Sr.~D. Beltrán de Urdaneta, de la primera nobleza de
Aragón, no se presta a reconocer al único Rey legítimo de España? Para
S. M. sería muy grato; y a mi entender, si usted se decidiese, le
seguirían otros nobles de Aragón y de Castilla. Fírmeme usted una
declaración en el sentido que le propongo, y yo la co\ldots{} municaré
al instante a mi Rey\ldots{}

---Señor General---dijo el noble caballero después de toser y limpiar el
gaznate para expresarse con toda claridad,---estimo en lo que vale la
excelente intención con que usted me propone ese reconocimiento de los
derechos del Infante, y espero que usted estimará del mismo modo la
lealtad con que me veo precisado a evadir todo compromiso con la causa
carlista. En conciencia, y estudiado el asunto, creo que la sucesión a
la Corona pertenece a la hija de Fernando VII, y habiéndolo declarado
así solemnemente como prócer del reino, no es decoroso para mí deponer
ahora en favor del augusto Príncipe, a quien reverencio como a tío
carnal de nuestra Reina. Fácilmente comprenderá usted, ilustre soldado,
que en mi clase y en mi raza, la religión del honor y de la consecuencia
no nos obliga menos que la otra religión con sus dogmas santísimos. Ni
por cuantos bienes hay en el mundo, ni por la vida, que es el primero de
los bienes, mancillaría yo con una traición el nombre que llevo\ldots{}
Y dicho esto con toda la entereza de que soy capaz, y todo el respeto
que a usted debo, he de manifestarle también que aunque partidario de
Isabel, y convencido de la legalidad de sus derechos, no he tomado parte
a su favor en esta contienda ni con armas, ni con escritos, ni en
ninguna otra forma. Soy hombre de paz, y acato las leyes de la nación,
vengan como vinieren. Ni guerrero he sido nunca, ni tampoco político. La
pelea y la conspiración me son desconocidas. Soy un hombre honrado,
isabelino en la intención, neutral en la conducta. No desconozco la
convicción y lealtad con que tremola usted la bandera del Infante. Pero
yo no la seguiré nunca: ni puede usted catequizarme ofreciéndome la vida
mía, que hoy tiene en su mano. Y si en vez de tener usted en rehenes
este cabo de vida, ya caduca, triste y de ningún valor, tuviera usted
una vida robusta; si yo fuera joven y mirase ante mí un porvenir de
treinta, cuarenta o cincuenta años, lo mismo que ahora le digo, le
diría\ldots{} siempre con la consideración que debo a un hombre de su
valer y de su inteligencia.»

Oyó con atención y agrado el soldado del absolutismo esta declaración,
dicha con cierto énfasis oratorio, y estimó delicadas las razones del
caballero. «Basta, señor mío, y no hablemos más del asunto---le
dijo.---Yo lo siento por usted\ldots{} y también por la Causa, que,
digan lo que quieran, no se ve muy apoyada por la Grandeza de
sangre\ldots{} Pero ya vendrán, ya vendrán todos\ldots{} Sólo que
llegarán tarde, y les pondremos en última fila. Para entonces ya
habremos creado nosotros, digo, el Rey, una aristocracia nueva, sacada
de las filas de la lealtad\ldots{} ¿Qué hizo Napoleón cuando se vio sin
nobleza de abolengo? Pues fabricarla. De sus generales hizo duques y
príncipes, y hasta reyes\ldots{} Traemos entre manos la fundación de una
sociedad nueva, pueblo nuevo, ejército flamante, aristocracia acabadita
de salir\ldots{} Y ustedes, los de la Corte isabelina, se irán a cuidar
cabras, o a destripar terrones\ldots{} Sí señor, si yo lo dispusiera,
así sería. A todos los marqueses y archipámpanos que no han reconocido a
Carlos V, les pondría yo una azada en la mano, y\ldots{} ¡hala! a
labrarme las tierras del común\ldots»

Terminó la cuestión de un modo festivo, y con ella la comida. Retirose
D. Beltrán, expresando nuevamente al leopardo su estimación, \emph{quand
même}, y se fue a dar con Nelet, que ansioso le esperaba. En una sala
del mismo edificio, y en las propias mesas donde habían comido, los
oficiales jugaban al ajedrez o las damas. Cabrera, una vez alzados los
manteles, se puso a trabajar con dos secretarios, dictando oficios y
comunicaciones para el gobierno de lo que con visos de Estado tenía bajo
su mano. No sólo había creado un ejército, sino una administración
civil, tal como esta podría existir en aquella vida de constante
inquietud, de movilidad epiléptica.

Y en verdad que el Estado en esbozo y su terreno inseguro le venían
corto a D. Ramón Cabrera, hombre que por su inteligencia comprensiva, su
voluntad potente y sus dotes de organización, había nacido para más
altas empresas. Su inquietud continua, la palidez de su rostro, el
estado nervioso y febril en que de ordinario se encontraba, no eran más
que la impaciencia loca para llegar a donde quería ir, el sentimiento de
la desproporción entre sus facultades y la poca materia gobernable que
cogía entre las manos. Lo que había creado con esfuerzo monstruoso, con
los golpes fulminantes de su coraje guerrero, con su nativo conocimiento
de los hombres y del país era mezquino para quien se sentía capaz de
manejar un Imperio\ldots{} Algo de esto pensó D. Beltrán, recordando lo
que hablaron durante la comida, y el rostro siempre melancólico de
Cabrera: «Es un hombre que, con tener mucho entre las manos, aún tiene
más en la cabeza, y de este desequilibrio proviene su aspecto de gato
triste\ldots{} dormilón cuando sus ojos no despiden rayos. Su crueldad
es la irritación contra el género humano porque no se le somete de
golpe. Si este hombre triunfara y pudiera manifestar tranquilo y seguro
lo que lleva en su corazón y en su cabeza, sería un dictador severo y
paternal, rigorista y clemente, próvido para todo, y hasta liberal
dentro de su poder soberano indiscutible.»

No le dejó Nelet hablar de nada que no fuera su asunto, y en cuanto
tuvieron ocasión de arrinconarse, lejos del barullo de los jugadores,
reanudaron el sabroso tema. «No puede ser hasta mañana por la
noche---dijo el militar\ldots---Ahora sobreviene una dificultad que
trato de vencer esta noche misma. Dícese que el General vuelve mañana
hacia Liria: no se qué planes tiene. Llangostera se queda aquí para ir
sobre San Mateo, y después no sé a dónde. Yo he pedido que me destinen a
su división, pues deseo aproximarme a mi pueblo, donde necesito
proveerme de ropa y dar un vistazo a mis intereses.

---Y en tal caso, ¡ay de mí! habremos de separarnos\ldots{}

---Creo que no. He hablado a Llangostera, que es grande amigo mío y
paisano, y espero conseguir que se quede usted con nosotros. Daremos al
General esta noche una razón que no tiene réplica. A Llangostera
corresponde fusilarle a usted, en caso de que maten al hermano del Conde
de Catí, porque el tal estaba en su división cuando le cogieron. La cosa
es de clavo pasado\ldots{}

---¡Y tan pasado! No sabía que entre carlistas hubiera tales etiquetas.

---¡Anda\ldots{} y que se cumplen con todo rigor! En fin, que vendrá
usted con nosotros.

---Mucho me place; y en cuanto a mi fusilamiento, lo mismo me da que sea
Pedro que sea Juan el que me mande a mejor vida\ldots{} Me alegraré, sí,
de que sea usted el encargado de darme los tiros, pues no dudo que usted
mandará que me apunten bien al corazón, para que mi muerte sea
instantánea, y no esté yo pataleando como un buey a medio
degollar\ldots{} Con que vamos a nuestro negocio.

---Dígame sinceramente, echando mano de todo su saber mundano, si una
vez libre Marcela, debo ir tras ella y emprender su conquista por
asalto\ldots{} ¿Cree que es mejor poner paralelas?

---Hijo, sí: el asalto no es prudente hasta que la plaza no esté bien
castigada y con ganas de rendirse. No haga usted la tontería de
embestirla con violencia\ldots{} Al contrario, es muy hábil aparentar
desgana de entrar en el recinto, afectar que se desean los
procedimientos del asedio galante, colmarla de atenciones, sin mostrar
al principio una ansiedad viva de amor\ldots{} Mujer que vive en el
idealismo, fíjese usted bien, con el idealismo debe ser atacada.

---¡Oh qué talento tiene usted---dijo Nelet, abrazándole gozoso,---y
cómo conoce el corazón humano! Ha sido gran suerte para mí encontrar tal
amigo.

---Y para mí también. Entre paréntesis, si quieres que yo hable con el
desahogo que facilita la comunicación entre maestro y discípulo,
permíteme que te tutee\ldots{} Pues, sí: como ella tira a lo espiritual,
conviene que aprendas tú algo de fraseología mística y hojarasca de
librillos devotos. Nada de violencias. Paralelas, hijo, paralelas y
fuegos parabólicos\ldots{} por elevación. Según dices, no eres en este
caso un seductor vulgar; solicitas el alma, el amor\ldots{}

---El amor, sí, grande, abrasador como el mío---dijo Nelet con acento
teatral.»

Movido de compasión y de un paternal interés, quiso el buen Urdaneta que
sus consejos le llevaran por el camino menos aventurado y escabroso.
Díjole que de los infinitos casos y ejemplos que atesoraba el archivo de
su experiencia, escogía los de color más honrado y puro. Antes que
atacar a la hermosa Marcela con asechanzas o artificios de mala ley,
debía esperar a que ella se rindiera, poniendo en ejecución para esto
los ardides de un hombre lealmente enamorado. Bueno sería empezar con la
estratagema de los desdenes, la fingida frialdad o indiferencia, que en
multitud de casos subyugan más pronto que los extremos de cariño; bueno
sería también mostrarse rival de ella en lo de suspirar por este o el
otro santo, o por misterios de la religión; y si esto no resultaba
eficaz, se emplearía el galanteo fino y respetuoso, el anhelo de
sacrificarse por la persona amada, el propósito de emprender trabajos no
menos grandes que los de Hércules, para obtener por recompensa una
mirada dulce, una leve ternura, un favor sencillo. Tampoco vendría mal
manifestarse caballero amador sin esperanza, por el gusto y la
satisfacción espiritual, sin ningún melindre de los sentidos, haciendo
gala de constancia a prueba de desprecios, de una adoración pura, en que
el alma del galán fuera como esas substancias puestas al fuego que nunca
se derriten ni consumen. Alcanzado el primer éxito, se intentaría curar
a la beata mujer de su místico arrebato, sacándola de aquel soñar
continuo en una perfección imposible; y atraída al terreno de la vida
corriente, se le propondría el matrimonio cristiano, bendecido por Dios:
la unión honrada de dos almas y dos cuerpos por toda la vida. «Este y no
otro es el camino, querido Nelet---concluyó D. Beltrán con serena
entonación,---que puede aconsejarte un hombre cargado de años y de
experiencia. Creo que si vas resueltamente por él, Dios te ayudará,
indultándote del castigo que mereciste por tus pecados de libertinaje.
Sí, sí, hijo mío: pues amas a Marcela, hazla tuya honradamente, y
constituye con ella una familia, y ten hijos que criarás en la virtud y
en el santo temor de Dios.»

Tan grande entusiasmo despertó en el apasionado joven esta elocuente
exhortación de su amigo, que se le saltaron las lágrimas, y hubo de
dominar con vivo esfuerzo su emoción, para no manifestarla ruidosamente
ante la muchedumbre de jugadores que llenaba la sala.

\hypertarget{xviii}{%
\chapter{XVIII}\label{xviii}}

«¡Jesús, qué delicia!---exclamó Nelet, después de una corta
pausa.---¡Casarme con ella!\ldots{} ¡Marcela mi mujer! ¡Y retirarnos a
una vida pacífica, laboriosa y agradable!\ldots{} ¡Y tener hijos, muchos
hijos!\ldots{} Sepa usted, D. Beltrán, que hacienda no me falta.
Conservo parte de las heredades de la familia\ldots{} entre ellas un
\emph{mas} que es la gloria, cerca de Cambrils.

---Rico tú, más rica ella, el matrimonio se impone---dijo el anciano con
tal gravedad, que a Nelet pareciole que hablaba por su boca el Concilio
de Trento.---Has de saber que Juan Luco, padre de esa extraordinaria
hembra, poseía grandes caudales, que yacen sepultados bajo tierra en
diferentes puntos: me consta.

---Algo de esto oí; mas no le daba crédito.

---¡Si serás tú simple! Crees en los demonios, y pones en duda los
hechos más naturales y corrientes\ldots{} De acuerdo con su hermano
Francisco, que también ha dado en la flor de que le canonicen, Marcela
se propone consagrar todo ese metálico que hoy yace bajo tierra a una
grande obra de fundación religiosa\ldots{} figúrate qué desatino\ldots{}
¡Como si no tuviéramos en España bastantes conventos! ¡Y en qué ocasión
se le ocurre emplear dinero en albergues para frailes y monjas\ldots{}
cuando Mendizábal, de una plumada, ha echado por tierra las órdenes
monásticas\ldots! Pero poniéndonos en lo razonable, y a fin de no
contrariar abiertamente la voluntad de la monjita, la dejaremos que
consagre parte del tesoro a satisfacer aquel deseo santo, reservando un
buen pico para las obligaciones sacratísimas que dejó pendientes Juan
Luco. ¿No te parece?

---Si he de hablar claro, Sr.~D. Beltrán, amo a Marcela con amor del
alma y fuego de todo mi ser, sin que esta pasión sea turbada ni
envilecida por ninguna ambición tocante a intereses\ldots{} Por mi vida,
que más la quiero pobre; que a mis brazos venga sin otra propiedad que
la estameña que cubre la hermosura de su cuerpo; estameña que yo trocaré
gustoso por la sedas más ricas.

---Pero, hijo, lo que abunda no daña. Tú no tienes culpa de que la santa
sea una ricachona. La mejor demostración que puedes dar de tu delicadeza
es permitir que Marcela funde o restaure algún conventito no muy grande,
y que dedique luego una parte no floja de sus especies metálicas a dar
cumplimiento a la voluntad de su padre\ldots{} a restituciones que son
sagradas, hijo, sagradas\ldots{}

---Con todo estoy conforme, pues cuanto usted me dice, parece dictado de
la misma razón y del perfecto conocimiento de la vida humana. No ha sido
poca suerte para mí encontrar tal amigo y asesor.

---A buen árbol te has arrimado, hijo\ldots{} Lo que yo no hiciere en
este negocio, cuenta que nadie lo haría\ldots{} Y si te parece, yo iré a
recogerme, que me siento cansado y soñoliento\ldots{} Alguien habrá que
me diga dónde voy a tender esta noche mis pobres huesos.»

Llevole Nelet muy solícito a la cama que a él le habían destinado, y se
determinó, con insomnio y desasosiego de amante, a pasar toda la noche
en pie. Las solitarias calles de Nules le vieron rondar, al pálido
fulgor de las estrellas, y disparar suspiros contra los blancos muros de
las \emph{Mónicas}, santuario y prisión de la bella teóloga.

Habiendo partido Cabrera al día siguiente en dirección al Júcar, por la
noche se efectuó con facilidad y sin ningún tropiezo la evasión de
Marcela, facilidad en parte debida a las ingeniosas disposiciones de
Nelet, en parte a las ganas que tenían las señoras Mónicas de que la
prófuga de Sigena se fuera a otra parte con sus filosofías y sus
latines. Mucho sintió Urdaneta no haber sido testigo de un caso que
tenía por interesante y teatral. Contole el galán que Marcela había
salido con un empaque de penitente, tal como en libertad la habían
conocido, y que él, atento a seguir los sabios dictámenes de su amigo,
se había mostrado atentísimo y caballerescamente cortesano, pero con
cierta frialdad parecida al desdén, según el programa trazado. Habiendo
dicho a la monja que no le había movido a libertarla más que su amor a
la religión y su respeto a las decisiones del Concilio de Trento,
replicó ella que agradecía su libertad; mas para que el favor fuera
completo, había de buscar Nelet a los dos viejos enterradores que la
acompañaban comúnmente, y llevárseles para que la guiaran en su camino.
No le fue difícil al enamorado dar con Zaida y Alfajar, y aquel día muy
temprano la monja y sus servidores o discípulos habían partido juntos
hacia Villavieja. Tuvo buen cuidado Santapau de advertir a su ídolo que
no se alejase de la tropa que él mandaba, pues de otro modo podría topar
con quien de nuevo la cogiese y encerrara, obedeciendo las órdenes de
Cabrera.

«En todo, hijo mío querido---dijo D. Beltrán satisfecho,---has procedido
con tanto tacto como previsión. Atento, y al propio tiempo
desdeñoso\ldots{} solícito en buscar a los viejos, que sin peligro de su
virtud la acompañan\ldots{} y por último, precavido para tenerla siempre
a la mano y que no se nos escabulla.

---Trato de inspirarme en usted, que todo lo sabe, pues aunque yo, he
sido hombre muy corrido de mujeres, hacía mis conquistas al modo de
pueblo, y con la rudeza y malos modos de mi educación aldeana. ¿Cómo
dice usted que llaman a los que se dedican a engañar mujeres y hacen de
esto un oficio?

---Don-Juanes.

---Pues si yo he sido un Don Juanillo de pueblo bajo, sin finura, sin
retóricas, basto y llanote, usted ha sido un señor Don Juan cortesano.»

Echose a reír Urdaneta, y no tuvieron tiempo de más explicaciones,
porque tocaron marcha, y el regimiento de Nelet, componiendo con otros
dos una brigada, al mando de Pertegaz, fue al socorro del
\emph{Serrador}, que apretado se veía en el sitio de Burriana. Cuando
llegaron ya era tarde, porque el \emph{Serrador} venía en retirada por
causa de la gran resistencia que opusieron los valientes urbanos,
socorridos por una columna de Castellón. Pocos días después, los
urbanos, por orden de Borso, abandonaron la plaza, y entró en ella el
cabecilla faccioso con el sentimiento de no encontrar a ningún jefe de
la Milicia ni de tropa a quien fusilar. Pertegaz tomó la vuelta de
Cantavieja para unirse a Cabañero, y Nelet volvió a incorporarse a la
división de Llangostera, que marchó hacia Lucena y de aquí a Albocácer,
recogiendo cuanto encontraba, hombres y caballerías, víveres y forrajes,
animales y personas. En todas estas marchas y contramarchas D. Beltrán
se aburría de lo lindo, y Nelet no tuvo el gusto de encontrar a Marcela
más que dos veces: la una, en la rambla de la Viuda; la otra, en Nuestra
Señora de Hortiseda. Apenas pudo hablarle en el primer encuentro; pero
en el segundo sí platicaron, y por consejo de su noble maestro se lanzó
a demostraciones más expresivas, después de haber empleado los desdenes
sin resultado práctico. No debió de quedar satisfecho el comandante,
porque cuando partió con sus tropas en auxilio de Cabañero, que sitiaba
a Cantavieja, iba muy temeroso de que le cogieran por su cuenta los
demonios que atormentarle solían. Rendida Cantavieja por traición,
quedáronse las fuerzas de Nelet a mitad del camino, en Iglesuela del
Cid, donde recibieron orden de Cabrera para marchar a la Cenia, punto
fortificado por los carlistas a la subida de los puertos de Beceite.
Allí se enteraron de que Oraa era General en jefe del Ejército del
Centro, y que, decidido a dar impulso a las operaciones, había dividido
su hueste en tres cuerpos, que mandaban los brigadieres Nogueras, Corral
y Sequera; supieron asimismo que el infatigable y diabólico D. Ramón se
aprestaba a defenderse contra enemigo tan poderoso como \emph{el Lobo
Cano}, que así llamaban a D. Marcelino, y seguramente, si con él no
podía, había de \emph{marearle} con sus audaces movimientos y
prodigiosos brincos de un extremo al otro del país. Por de pronto,
apresuraba la expugnación de la histórica villa de San Mateo, para no
dar tiempo a que en su auxilio fuesen los de la Reina. Grandes
acontecimientos se preparaban: D. Beltrán, que era amigo de Oraa,
confiaba mucho en su pericia; mas conociendo ya el fragoso terreno de
aquella guerra, y la fiereza y dura condición de los que en él peleaban
por el absolutismo, no veía cerca ni lejos el menor vislumbre de paz. La
Naturaleza era allí tan guerrera como el hombre.

Estaba de Dios que antes de salir de la Cenia presenciaran Nelet y D.
Beltrán espectáculo tan lastimoso como el de Burjasot, pues, conducidos
allí los prisioneros de San Mateo (que se rindió como Cantavieja, por
flaqueza o deslealtad de algunos de sus defensores), se procedió con
toda tranquilidad a exterminarlos por un procedimiento fácil y barato.
Apenas llegaron, metiéronles en diferentes mazmorras; algunos fueron
recluidos dentro de un horno de pan. Y si por economía de víveres se les
mataba de hambre, por ahorrar cartuchos se determinó concluirles a
bayonetazos. Edificado el pueblo en eminencia rocosa, presenta por uno
de sus costados un tajo formidable, vertiginosa caída a la profundidad
aterradora de un barranco, donde brama un torrente entre peñas y
zarzales. Al borde de este precipicio fueron conducidos de dos en dos
los prisioneros, después de confesados por el padre Chambó, cura párroco
de la Cenia. Unos cuantos \emph{números} hacían de matachines; otros
tantos arrojaban los cuerpos a la hondura tenebrosa y fría. Treinta y
ocho oficiales y sargentos perecieron de este modo, sin contar un cadete
de doce años, que fue al matadero emparejado con su padre, comandante
del fuerte rendido de San Mateo. La última res sacrificada fue una
cantinera portuguesa.

No tuvo papel Santapau en esta tragedia, pues habiéndose trocado, por la
virtud de su amorosa llama, de feroz en benigno y humanitario, siempre
que le daba en la nariz olor de degollina, se ponía malo; y realmente lo
estuvo de la cabeza y del corazón. Sin quejarse tanto como su amigo, D.
Beltrán no gozaba de buena salud. Ambos se alegraron cuando se dio la
orden de que Nelet marchase con la mitad de su regimiento a relevar la
guarnición de Benifazá, lugar que también tenían toscamente fortificado
en el centro de aquel núcleo de montes elevadísimos que llaman la
Tinenza. Por los desfiladeros del río de la Cenia, faldeando la Peña del
Águila, pasaron de la zona de Rosell a Benifazá, y a la célebre abadía
cisterciense fundada por D. Jaime, edificio devastado sucesivamente por
tres guerras, la de las Germanías, la de Sucesión y la que ahora se
relata. Daba pena ver su noble arquitectura mutilada por bárbaras manos:
aquí señales de incendios, allá desplomados muros, la iglesia con medio
techo de menos, la torre melancólica y sin campanas, con sus espadañas
ciegas y mudas, las junturas pobladas de jaramagos y ortigas, y el
claustro, en fin, con sólo tres costados, más triste que todo lo demás,
y más poético y ensoñador. Aposentaron a D. Beltrán en un pasadizo entre
el claustro y la iglesia, donde gozaba de la hermosa vista del
despedazado monumento, que apreciar podía en su esbeltez de conjunto, no
en sus riquísimos detalles. No era lego en arqueología el buen aragonés,
y sentía verdadera pasión por el estilo llamado románico y su elegante
austeridad: en tiempos más felices había visitado con entusiasmo de
artista los monasterios de Veruela y San Juan de Peña; conocía el de
Rueda como su propia casa, y todo lo románico y gótico del siglo XIII
que encierran las ilustres villas y ciudades de Aragón. Se extasiaba
recorriendo los venerables restos de la construcción medieval, los tres
ábsides semi-circulares, el claustro, la sala del Capítulo, el palacio
abacial; y tan dulce encanto encontró en aquella paz y en el poético
lenguaje de las nobles y tristes piedras, que habría deseado permanecer
allí todo el tiempo que su prisión durase.

También Nelet se sentía muy a gusto en el monasterio, que perfectamente
cuadraba a su espíritu en aquella ocasión, como estuche ajustado a la
joya que guarda. La dolencia que trajo de la Cenia se le calmó el primer
día; mas repuntó al segundo con sus murrias negras y sus vibraciones
nerviosas, anunciándole la visita de los entes infernales que con él se
divertían. Los ratos libres de servicio pasábalos con D. Beltrán,
sentaditos en un rincón del claustro, hablando cada cual de sus
tristezas. Como el présbita que se hace leer un libro de letra menuda,
Urdaneta rogaba a su amigo que le leyese el claustro, esto es, que
examinara uno por uno los capiteles y el simbolismo que representaban,
para poder él juzgar de obra tan bella, como si con sus propios ojos la
deletreara. Después de describir varias esculturas en que no halló
ningún interés, dijo Nelet con estupor:

«¡Ay, aquí veo mi propia historia!\ldots{} No, no se ría: es mi
historia, que aquí representaron aquellos artífices algunos siglos antes
de que yo viniera al mundo.

---¿Qué ves, hijo?

---En este capitel del ángulo, por la parte de dentro, veo un guerrero
que adora a una penitente. Él está de rodillas; ella, en la tosquedad de
estos relieves, ofrece gran semejanza con Marcela, los pies desnudos,
suelto el cabello\ldots{} En el capitel de fuera se ve la misma
peregrina, con una cruz\ldots{} Yo no estoy aquí\ldots{} parece como si
me hubiera ido\ldots{} Debo de estar más allá\ldots{} Déjeme ver\ldots{}
Aquí no estoy; forman el adorno unos como perritos o leoncitos, y luego
sigue otro con cabezuelas de ángeles, entre las púas retorcidas de
cardos borriqueros\ldots{} ¡Ah! ya parecí\ldots{} aquí estoy, en este
otro capitel, y me tiene cogido por el pescuezo el demonio que se
permite conmigo sus bromas cargantes\ldots{} Sigue otro en que hay
muchas mujeres chiquitas, desnudas, entre llamas, que son las hembras
que deshonré y perdí, y por mi culpa están en el Purgatorio o en el
Infierno\ldots{}

---Hombre, no saques las cosas de quicio. Será otra leyenda que nada
tiene que ver contigo\ldots{} ¿Qué hay más allá?

---Pues un caballero con cruz en el pecho, como de Templario, con un
cuerno de caza en el cinto, en la una mano una pica y en la otra un
halcón.

---Caballero noble\ldots{} Ese soy yo\ldots{} No me niegues que puedo
ser yo.

---¿Cómo he de negarlo, si hasta se le parece en lo airoso de la
figura?\ldots{} pues en el rostro tiene un cierto aire\ldots{}

---Dime otra cosa\ldots{} fíjate bien. ¿No estoy hablando con alguna
dama de alta alcurnia, reina o princesa?

---No señor\ldots{} Está usted solo.

---No puede ser. Puede que el tiempo haya desgastado la otra figura.
Dama ilustre debe de haber, que me acompaña en el noble ejercicio de la
caza; y si no es así, no soy yo el que miras, Nelet.

---Créalo usted o no lo crea, yo sostengo, amigo mío, que vivimos en
estos pedruscos. Esto que aquí nos rodea no es cosa muerta; esto tiene
alma, como la tienen los montes, el viento, las cavernas y los torrentes
que cantan y rezan en las profundidades\ldots{}

\hypertarget{xix}{%
\chapter{XIX}\label{xix}}

---Más poeta eres de lo que yo creía---dijo D. Beltrán, cogiéndole del
brazo para pasear por el claustro.---Por cierto que una queja tengo de
ti, y es que, habiéndote escrito Marcela, según me has dicho, más de una
carta acompañada de versos, aún no me los has enseñado.

---No sólo he de mostrárselos, sino que quiero que ponga su mano de
maestro en los que yo, en respuesta de los suyos, estoy
inventando\ldots{}

Rompió D. Beltrán en una risa placentera; mas no pudieron seguir
ocupándose en aquel ameno asunto, porque se acercó el ayudante del
batallón, llamando a Santapau para urgentes resoluciones del servicio.
Toda la tarde y parte de la noche estuvo atareadísimo, dando
cumplimiento a órdenes de Cabrera para proveer a una corta de árboles,
con objeto de proteger el camino cubierto entre la casa del abad y un
\emph{mas} situado a tiro de fusil, dominando el río y el sendero. Al
día siguiente contó el comandante a su maestro que, no pudiendo dormir
después de su trabajo, había visto a Marcela, o más bien una parte no
más de su persona\ldots{} «pero tan claramente, amigo Urdaneta, como le
estoy viendo a usted ahora.

---¡Demonio! ¿Y qué parte de su persona veías? ¿Se puede saber?

---Los dientes\ldots{} Mire usted que es raro. No hay, créalo, en todo
el mundo dientes como los suyos, blancos como la leche, y tan iguales y
bonitos que se emboba uno mirándolos\ldots{} Por arriba y por abajo de
las dos hiladas veía yo un poco de los labios\ldots{} y nada más.

---Eso es que sonreía. Buena señal. ¿Y una sola vez lo viste?

---Más de veinte, y hoy también como unas ocho veces.

---Aunque aquí estamos muy bien, es lástima que las obligaciones
militares nos separen de la divina Marcela.»

Díjole Nelet que, desde antes de ir a la Cenia, era tal su anhelo de
verla y hablarle, que había discurrido establecer comunicación con ella.
Tanto tiempo ausente del ser que adoraba, era peor desgracia que la
muerte. Habiendo tenido la suerte de encontrar a un pastor viejo, muy
conocedor de aquellos montes y cañadas, devoto de Marcela, a quien como
santa miraba, le dio el encargo de rastrearla y descubrir sus guaridas.
Al segundo día de estar en Benifazá, le había traído el pastor razones
satisfactorias. Marcela habitaba con preferencia en las alturas, como
las águilas, y en los santuarios de más devoción del país. Había morado
algún tiempo en la Muela de Ares, después pasó a la cueva de la Balma,
de allí a la Virgen de los Ángeles, cerca de San Mateo, y en aquellos
días se hallaba en el santuario de la Traiguera, entre Chert y Vinaroz.
Como preguntara D. Beltrán si había recibido carta en prosa o verso,
replicó que las razones de que había sido mensajero el pastor eran
verbales. Marcela enviaba un cordial saludo a sus dos amigos, asegurando
que en todas sus oraciones pedía al Señor y a la Virgen que les diera
salud y buenas ideas. A Nelet, particularmente, le enviaba nuevas
expresiones de su agradecimiento, y la promesa de acudir al punto que él
designara, si algo tenía que decirle.

«¡Oh! Magnífico\ldots{} No repugna acudir a la cita. Vamos bien, querido
Nelet, pero muy bien\ldots{} ¡Ay! es triste cosa que ni yo por
prisionero, ni tú por militar, esclavo de la ordenanza, podamos
trasladarnos a donde nos llama nuestro deseo.

---En eso pienso, señor mío---dijo Santapau caviloso.---Y harto ya de la
esclavitud del servicio, estoy decidido a pedir mi licencia por enfermo,
instalándome donde ella esté, aunque para esto tenga que hacer vida de
penitente.»

Aseguró Urdaneta, suspirando y casi lloroso, que él haría lo mismo si
pudiese, agregándose a los enterradores que escoltaban a la divina
mujer, y dedicándose con ellos al manejo de la pala y azadón donde fuese
menester remover la tierra. Añadió Nelet que para la comunicación con la
monja había encontrado mensajero más rápido que el pastor, y era una
mujercita del barranco de Vallivana, a quien llamaban \emph{Malaena},
también con cariz de penitente o mendiga errante, envejecida por los
trabajos, la miseria y los sufrimientos. Madre fue de dos hijos que
andaban en la partida de Pertegaz, y cogido por Nogueras uno de ellos
con un parte del \emph{Serrador}, le fusilaron; al otro le aplicó Boil
la misma pena en Concud, cerca de Teruel. Sin parientes ni habientes,
viviendo de arrancar leña y vender teas, era \emph{Malaena} un puro
espíritu, pues entre sus huesos y su piel no encontrara el escalpelo más
diligente una hebra de carne. Frecuentaba los bosques; sabía escoger
hierbas oficinales; comía raíces y mendrugos de pan, reblandecidos en el
agua. En ligereza para pasar de un valle a otro, salvando las más altas
muelas, y los puertos pedregosos, no la igualaban más que los pájaros.
Aunque en algunos caseríos de Salvasoria la tenían por bruja y la
recibían a pedradas, era una pobre y santa mujer, sencilla, inocente y
fiel. Al escogerla Santapau para embajadora, vio en ella un ave discreta
y solícita; y para tenerla en su gracia, empezó por regalarle una saya
nueva, pañuelos, y todas las alpargatas que para sus montaraces
correrías necesitase. Estas fueron inauguradas por un mensaje amoroso,
en que puso Nelet sus cinco sentidos, consultándolo con D. Beltrán, el
cual hizo varias enmiendas, más para templar que para encender el ardor
pasional del desgraciado joven.

Si la guerra vino de improviso a perturbar estos planes, tan distintos
de la contienda entre Isabel y Carlos, luego los favoreció, como se verá
más adelante. El 4 de Mayo avanzó el General Oraa desde Vinaroz contra
Cabrera y el \emph{Serrador}, que ocupaban la Cenia y Rosell. En una y
otra parte les atacó con brío, desalojándoles después de reñido combate.
La fuerza de Benifazá acudió en apoyo del \emph{Serrador}, y tanto este
como Cabrera hubieron de buscar refugio en la sierra de Bel. Dos días
estuvieron tiroteándose en aquellas alturas las guerrillas de uno y otro
bando, hasta que Oraa, falto de provisiones, hubo de retirarse a
Vinaroz, y Cabrera y el \emph{Serrador} volvieron a ocupar la Cenia y
Rosell. Tal era la guerra del Maestrazgo, un tomar y dejar posiciones y
un perseguirse y sorprenderse, sin ventaja de los liberales, que no
podían abandonar largo tiempo su base de operaciones; el juego sólo
aprovechaba a los carlistas, que estaban en su casa, y, desalojados de
la sala, se metían en la cocina; perseguidos en esta, se escabullían por
el cañón de la chimenea, y desde el tejado seguían combatiendo.

El ejército cristino, como se ha dicho, tuvo que bajar a Vinaroz: comió
y volvió a subir, custodiando un convoy de víveres para socorrer a
Morella, algo apurada de bucólica en aquellos días. Queriendo cortarle
el paso, apostó Cabrera su gente en Chert; pero el \emph{lobo cano}
anduvo más listo; conocida la jugada, dispuso sus tropas con arte y
burló la astucia del leopardo. Trabose batalla, en que el \emph{lobo}
llevó la mejor parte, ganando sin dificultad el paso de Vallivana y
entrando en Morella sin grave tropiezo. Repitiose a la vuelta la jugada,
con mayor gasto de cartuchos y algunas bajas; pero el \emph{lobo} pasó,
rodeando las alturas de Catí, mientras su rival, desconcertado por este
hábil movimiento, bajó a esperarle en el valle de San Mateo, donde la
caballería cristina le hizo frente, obligándole a volverse a las
alturas. Poco afortunado Cabrera en aquellos lances, dividió de nuevo su
ejército, y dejando a Llangostera en el Maestrazgo, se corrió con el
\emph{Serrador} y el \emph{Fraile Esperanza} hacia Murviedro, donde
esperó inútilmente sorprender a Nogueras, y de allí le veremos volver
pronto hacia el Norte con la celeridad del rayo, para sitiar a Gandesa.

En el tiempo invertido en estas operaciones, que sólo por el cansancio
que producían al enemigo eran al carlismo provechosas, pasó el buen
Urdaneta días de ansiedad amarguísima, confinado primero en Chert, luego
en la Jana, deplorando la ausencia de su amigo, de quien nada sabía;
oyendo sin cesar el vivo tiroteo que por esta y la otra encañada, de
este y el otro monte venía; ignorante de quién perdiera o ganara en
aquellos combates, a su parecer fantásticos y aéreos, sostenidos en las
alturas o en los desfiladeros por bandadas de aves, más que por hombres.
Eran las guerras de fábula, entre animales de pluma o pelo, veloces, y
que prontamente corrían de un punto a otro, sin dejar rastro. Recluido
en la impedimenta de Llangostena, que escoltaban pocos infantes y
caballos, sufrió el hombre tristezas, hambres y tratos groseros, hasta
que puestas en marcha las acémilas, cuando ya toda la rambla de Cervera
y pasos de Vallivana estaban libres de cristinos, tuvo la satisfacción
de ver a Nelet, que al frente de un corto destacamento de soldados venía
de San Mateo, y lo primero que hizo el joven guerrero fue correr a
abrazarle cariñoso. Poco le faltó a D. Beltrán para echarse a llorar del
gusto que aquel encuentro le daba, y antes que pudieran comunicarse sus
afectos, hubo de notar Urdaneta que el rostro de su amigo, demacrado y
macilento, revelaba enfermedad honda o turbaciones del ánimo. No quiso
el comandante entretenerse en explicaciones, dejándolas para cuando
llegasen al poblacho donde habían de dormir. Sólo dijo que sintiéndose
mal de salud, había pedido permiso a Cabrera para reponerse con algunos
días de descanso, y para cumplir un voto a la Santísima Virgen de
Vallivana. Como se condoliera el maestro de no poder acompañarle ni en
el descanso ni en la piadosa peregrinación, díjole Nelet que pues en
Catí encontraría a Cabañero, bien se podía esperar que el bravo
aragonés, deudor de D. Beltrán por beneficios recibidos, le mostraría su
gratitud en aquella ocasión sin faltar a sus deberes militares.
Consolado con esta idea, recobró el noble señor su tranquilidad, ya que
no su alegría, y charlando de los sucesos recientes, se encaminaron uno
y otro a Catí, venciendo trabajosamente la subida asperísima que a tan
enriscada posición conduce.

Lo primero de que se ocupó en el pueblo Santapau fue de ver a Cabañero,
que con su legión zaragozana y oscense allí estaba desde el día
anterior, y hablarle del desgraciado prócer a cuya generosidad debía el
jefe aragonés los primeros zapatos que se puso en su vida. Así lo
reconoció el tal, manifestándose muy sorprendido de que en pasos tan
desdichados se viera el noble señor de Albalate y Olid. Corrió a verle,
y, besándole afectuoso las manos, oyó de D. Beltrán las explicaciones
que este quiso darle de los motivos por que había venido a ser cautivo
de Cabrera, y de hallarse en rehenes, la más aflictiva situación de un
hombre ¡ay! en tiempos tan calamitosos. Compadecido Cabañero, y
expresando su voluntad sincera de influir con el jefe para libertarle,
le convidó a su mesa, harto pobre en verdad; pero aceptable en tales
circunstancias. Tocado por Nelet el punto delicado de la escapadita a
Vallivana para cumplir el voto que \emph{los dos habían hecho} de
visitar a la Santísima Virgen, accedió Cabañero a que el prisionero se
ausentase del ejército por dos días no más, dejándole una garantía más
valiosa que todos los rehenes o \emph{prendas} vivas. «Mi palabra de
honor, ¿no es eso?---dijo D. Beltrán alargando su mano flaca.---Pues la
tienes.» Respondió el aragonés con gallarda confianza que la palabra de
tan insigne caballero le bastaba para tener bien cubierta su
responsabilidad, y no se habló más del asunto.

Vierais, pues, a la mañanita siguiente a Manuel Santapau y al Sr.~de
Urdaneta salir de Catí, solos, a pie, cada cual amparado de un nudoso
garrote: el uno inerme, el otro armado de pistolas y un cuchillo de
monte. Llevaba de añadidura Nelet provisión de pan y otras cosillas de
sustancia liadas en un pañuelo. En el descenso de la montaña, por
senderos de ovejas que sorteaban la pendiente con ángulos y curvas
dilatadas, pudiendo apreciar el grandioso panorama que a su vista se
ofrecía; belleza incomparable de que también gozó D. Beltrán, pues si no
apreciaba las menudencias y tonos medios del paisaje, percibía
claramente las grandes masas rocosas, que por su coronamiento romo y
achatado, en aquella formación geológica, son llamadas \emph{muelas}.
Las vertientes cubiertas de verde espesura son en algunos puntos suaves;
en otros caen rápidamente, querenciosas de la vertical: todas de
imponente majestad y hermosura. En una de las revueltas vieron el alto
de la Virgen de la Salud, cerca de San Mateo, coronado por el santuario
eminente; en otra revuelta, hacia el Oeste, la Muela de Ares, cima chata
en la sierra de la Higuera. Hacia el Norte distinguían el obscuro monte
de Vallivana cubierto de verdor, y más allá asomaban el Castell de
Cabres, la Moleta del Cid y los montes de la Cenia. Ningún ser humano
encontraron en el camino. Llegado que hubieron a un ameno grupo de
alisos entre peñas, se sentaron a descansar y a reponerse con un frugal
almuerzo, y tumbados allí, en medio de la paz y quietud más deliciosas,
Nelet empezó a desembuchar las noticias y peregrinos hechos que ansiaba
someter al consejo de su amigo.

\hypertarget{xx}{%
\chapter{XX}\label{xx}}

Sin ociosos preámbulos refirió que había pasado noches horribles de
insomnio y terror, pues al llegar a Calig, después de haberse batido en
guerrillas un día entero con las guerrillas de Oraa, le cogieron por su
cuenta como media docena de espíritus, a quienes primero tuvo por
ángeles, y luego hubo de reconocerlos por demonios efectivos, de la
familia o casta de bellacos y maleantes, pues se le presentaron en un
puesto de cantina, y convidándole a beber copas, invitáronle a dar un
paseo. Vestían de paño colorado, como oficiales de un ejército
extranjero; y cuando ya se hallaron solos con él en lugar apartado,
trocáronse por ensalmo en clérigos, y le dijeron que le casarían al
instante con la hermosa Marcela. Quiso huir Nelet; mas le cogieron, y de
un vuelo rapidísimo fue llevado al castillo de San Mateo, entrando por
la plataforma de la torre más alta.

«Nelet, si es sueño---dijo D. Beltrán bondadoso,---cuéntamelo como
sueño, y con la importancia que a tales figuraciones de nuestro cerebro
debemos dar.

---Lo cuento como me pasó y como lo sentí. Preste usted atención, y verá
si es sueño o qué es. Pues, señor\ldots{} El que parecía jefe de la
infernal comparsa me cogió por el brazo y me dio un rápido paseo por el
interior del castillo, arrastrados él y yo de un furioso ventarrón que
por todos los huecos entraba y salía, llevando consigo alimañas mil
volanderas y un polvo que cegaba. Y con las propias voces del aire y los
chillidos de las alimañas, mi demonio me hablaba. De todo lo que me
dijo, sólo saqué en limpio que el amor que Marcela tenía a las cosas
divinas se le había trocado, por arte maléfica, en afición a hombre, a
mí, en una palabra; que en aquel momento hallábase en el santuario de
Traiguera engañando a la Virgen para que la relevara de la obligación de
sus votos. Debí de manifestar al maldito diablo mi afán de trasladarme a
Traiguera\ldots{} no estoy seguro de ello\ldots{} sólo sé que llevándome
a un gran sótano que hay bajo la sala de armas del castillo, me mostró
un agujero al modo de escotillón, de donde arrancan escalones hacia lo
profundo\ldots{} Como polvo, como humo se desvaneció mi acompañante,
dejando tras sí un olor muy malo, y yo, precipitándome por aquella
abertura, me vi dentro de un angosto callejón labrado en la roca, y por
él me lancé, en la seguridad de salir a Traiguera. Una luz tristísima,
que yo no sabía de dónde demonios podía venir, me alumbraba en tan feo
camino. Seguí, seguí toda la noche andando; toda la noche, señor, y al
ser de día, o cuando a mí me parecía que alumbraba el sol en la región
externa de la tierra, oí ruido de aguas que manaban de aquellas peñas y
corrían por grietas y sumideros, haciendo unas como gárgaras muy
imponentes\ldots{} Halléme por fin en una caverna, cuyo techo parecía la
bóveda de una catedral; en el fondo de ella varios hombres cavaban la
tierra\ldots{} Acerqueme, y les vi sacar del suelo un objeto largo y
pesado de color de tierra. `¿Es eso una momia, amigos?', les pregunté. Y
ellos respondieron: `Mojama es de un muerto de metales, que agora
sacamos y resucitamos por orden de la sacra señora, para mayor grandeza
de Dios e de su religión'. Sin parar mientes en lo que hacían, les
pregunté por dónde saldría más pronto a Traiguera, y su respuesta fue
señalarme uno de los conductos que desde allí partían, abiertos en la
roca. Por él me metí, y a las seis horas de camino, por mi cuenta, salí
a la luz, y me encontré, no en Traiguera, sino en el castillo de Cervera
del Maestre.

---Para, querido Nelet, para---le dijo Don Beltrán,---y reconoce que
todo eso es un desatinado sueño.

---Lo reconoceré si usted se empeña en ello. Pero hay algo aquí que no
comprenderé si usted con su universal conocimiento de las cosas no me lo
explica, y es que al salir a Cervera del Maestre, encontréme tan molido
como si me hubieran dado carreras de baqueta; mis pies sangraban; en mi
cuerpo no cabían ya más cardenales\ldots{} Y otra duda: si ello fue
sueño y me dormí en Calig, ¿cómo desperté en Cervera?

---¿Estás bien seguro de no haber ido a Cervera\ldots{} por tu pie?

---Segurísimo. ¿Y cómo, sin creer en los poderes ocultos, se explica que
al bajar yo del castillo al pueblo de Cervera me encontré a
\emph{Malaena}, que muy sentadita en una piedra me esperaba? ¿Cómo sabía
ella que allí estaba yo, habiéndole advertido que fuera a buscarme a
Calig?

---Pues será bruja, como dicen\ldots{} Y en suma, ¿qué recado te traía
la mensajera?

---Que había visto a Marcela en el castillo de San Jorge, más abajo de
Traiguera, ocupada con dos viejos en apisonar la tierra de una sepultura
recién abierta y cerrada. Apisonaban dando pataditas encima los tres,
marcando el compás, como de baile, con una oración entre rezada y
cantada. Luego que acabaron, Marcela dijo a mi embajadora que si yo
quería verla pasase el jueves (por hoy) a Vallivana.

---Y por eso estamos aquí, y por eso vamos allá. Muy bien.

---La despaché en seguida con nuevo mensaje escrito, y hoy ha de traerme
la contestación. Me espera en Salvasoria, que es aquella aldeíta que
blanquea allá lejos, en el fondo de este valle, y que desde aquí parece
un hato de ovejas sesteando entre los matorros verdes.»

Siguieron; y como D. Beltrán intentara quitarle de la cabeza la pueril
creencia de los caminos subterráneos, obra de la Edad feudal, dijo Nelet
que a la tradición debía tal creencia y otras análogas, como la parte
fundamental que toman en nuestra vida las potencias invisibles, ora sean
ángeles, ora demonios. Replicó el anciano que la tradición era una vieja
loca, que había sido poetisa; pero que ya con la edad chocheaba; y
Santapau contó que su madre, natural de Ares del Maestre, el riñón del
Maestrazgo, hablaba de las galerías secretas entre los castillos de la
Orden de Montesa y los monasterios de frailes y monjas, como si las
hubiera visto y reconocido de punta a punta. Tomó la palabra Urdaneta
para denegar tales absurdos, asegurando que si había pasadizos bajo
tierra, eran cortos, y sólo servían para unir los castillos con algún
reducto cercano, caminos naturales del arte antiguo de la fortificación.
Respecto a la Orden de Montesa, de quien fue propiedad aquel territorio
que veían, y otros mayores en grandísima extensión por todo el reino
alto de Valencia, dijo que él era caballero de dicho Hábito; pero que ya
tales caballerías eran una ficción de vanidad, porque todo lo
substancial de ellas se lo había tragado el tiempo insaciable, que va
devorando, devorando, y no siempre crea cosas nuevas con que sustituir a
las pasadas. En la antigua ciudad de Olite, patria de su madre, y en la
casa solar de Urdaneta, en las Cinco Villas, subsistían no pocos
retratos de esclarecidos caballeros de San Jorge de Alfama, Orden que se
refundió en la de Montesa. Esta trocó su cruz negra flordelisada por la
roja y sencilla de San Jorge, que es la que aún dura. Uno de sus remotos
abuelos, según constaba en pergaminos de la casa, D. Gilaberto de
Monsoria, fue Gran Maestre de Montesa, y con esta dignidad murió en la
villa de San Mateo, donde seguramente se conservaría su sepulcro. «Otro
ascendiente mío por la línea materna, frey D. Pedro Luis de Garcerán de
Borja, fue Comendador mayor, y poseía por tal dignidad las villas y
pueblos de Cuevas de Vinromá, Albocácer, Tirig, Torre den Dumenje, y
otras más que no recuerdo ahora. Clavero fue el hermano del fundador de
mi Señorío de Albalate; frey D. Guillén de Corbera, almirante\ldots{}
Pues si las mudanzas de los tiempos y las revoluciones no hubieran hecho
escombros de todo aquel orden social, tu amigo D. Beltrán de Urdaneta
sería hoy quizás Gran Maestre, y dueño, por tanto, de las villas y
lugares de San Mateo, Traiguera, Chert, La Jana y algunos más.
Figúrate\ldots{} Nadie nos tosía en estos valles y montes; con mi gente
armada y esta red de castillos y fortalezas, haríamos aquí lo que nos
diera la gana: a ti te nombraría bailío para que me gobernaras todo mi
territorio; elegiríamos prior a un clérigo sumiso que a nuestro gusto
nos gobernara todo lo espiritual; a las monjas de nuestra jurisdicción
las obligaríamos a proporcionarnos todos los milagros que fueran
menester; haríamos excavaciones para sacar tesoros escondidos y\ldots{}
Pero despertemos a la realidad, y caigamos innoblemente en este lodazal
de miseria, de esclavitud y vulgaridad. Veamos nuestros castillos en
ruinas, poblados de lagartos y murciélagos; nuestro poder desvanecido
como el humo; veámonos tan impotentes que sobre nadie tenemos autoridad,
y a nosotros nos mandan cuatro canallas groseros y estúpidos. ¿Qué
somos? Unos pobres peregrinos que van tras de una monja suelta, de quien
esperamos, tú una limosna de amor, yo una limosna de pan\ldots{} Ya
ves\ldots{} ¡qué triste despertar!\ldots{} ¡Oh tiempos, oh fin de
fines!\ldots»

Callaron largo trecho: antes de llegar a Salvasoria, se les apareció
\emph{Malaena} saliendo de un matojo, y Nelet se detuvo un instante con
ella para recibir razones de su embajada. D. Beltrán distinguía de la
mensajera una figurilla delgada y ágil, brazos y manos ennegrecidos, con
rostro muy semejante en color y arrugas a una pasa, con ojos ratoniles.
No hablaba más que valenciano, dulce y lacónico, apoyando con sus flacas
manos los dichos, cual si quisiera estamparlos en el aire. \emph{Pos
hara}---le dijo Nelet,\emph{---adelantat y espéranos en la font, al peu
del mont. Allí pasarem la nit. Arreplega lleña y fes una bona fogata.
Pren estas provisions, y si pots conseguir unes criailles, fetnos un bon
guisado.}

En breve desapareció delante de los peregrinos la diligente pájara, y
ellos siguieron taciturnos: Nelet mirando al suelo, recitando entre
dientes algo que no se sabía si era oración o algún conjuro contra
diablos entrometidos y enredadores; D. Beltrán mirando al monte,
recreándose en aquella plácida soledad de sagrado bosque propicio a los
misterios. Sentíase el noble viejo a mil leguas de la sociedad y de sus
afanes; diríase que ni la guerra, ni la política, ni ninguna lucha de
humanos, habían de extender hasta allí su tumulto y vocerío. Por no ver
seres vivos, ni aun cabras veían. Era la soledad de los lugares no
estrenados aún por la historia y la leyenda\ldots{} La imaginación del
primer habitante los poblaba de seres invisibles, escondidos en el
silencio.

Oyendo suspirar a Nelet, su maestro le dijo: «Muy caviloso te veo. ¿Eso
que entre dientes hablas, es rezo o un ensayo de lo que quieres decir a
tu amada en la entrevista de esta tarde?

---No la veré esta tarde, sino mañana al amanecer, que así acaba de
anunciármelo \emph{Malaena}; y en cuanto a lo que mascullo, sepa usted
que es la contestación que debo dar a unos versos que hace días me envió
Marcela\ldots{} Mi plan es glosarlos estrofa por estrofa, devolviéndole
el discurso y dándole un giro peregrino, que al propio tiempo que
exprese mis afectos, sea muestra gallarda de un buen razonar\ldots{}
Compongo de memoria algunas de mis estrofas para que usted me las
corrija, y en eso vengo trabajando con los sesos bien afinados y
calientes.

---Ante todo, léeme o recita los versos de esa prodigiosa mujer, pues
sin conocer la proposición poética, mal podré yo juzgar si en la
conclusión rivaliza tu ingenio con el suyo.

\hypertarget{xxi}{%
\chapter{XXI}\label{xxi}}

---Recitaré a usted las primeras estrofas de ellas, que estampadas con
letras de fuego, como todas las demás, llevo en mi memoria. Dicen así:

\small
\newlength\mlena
\settowidth\mlena{\quad ¡Oh, Marcela! Si es Dios circunferencia }
\begin{center}
\parbox{\mlena}{\quad \textit{Es Dios la original circunferencia          \\
                De todas las esféricas figuras,                           \\
                Pues cercos, orbes, círculos y alturas                    \\
                En el centro se incluyen de su esencia.                   \\
                \null \quad De este infinito centro de la ciencia         \\
                Salen inmensas líneas de criaturas,                       \\
                Centellas vivas de las luces puras                        \\
                De aquella inaccesible omnipotencia.}}                    \\
\end{center}
\normalsize

---Enrevesadillo es\ldots{} pero no está mal. Yo que tú, me limitaría a
contestarle en prosa llana que la quieres, que ahorque el sayo de
peregrina, y se deje de ensueños y se case contigo, para que deis a Dios
y a la sociedad, ella robusta, tú también, una \emph{inmensa línea de
criaturas}\ldots{} Pero sin perjuicio de este consejo, veamos cómo se
compone tu cacumen para devolver esas estrofas.

---Pues verá usted\ldots{} yo le digo:

\small
\newlength\mlenb
\settowidth\mlenb{\quad ¡Oh, Marcela! Si es Dios circunferencia }
\begin{center}
\parbox{\mlenb}{\quad \textit{¡Oh, Marcela! Si es Dios circunferencia     \\
                              De la divina esencia,                       \\
                              Explana de los orbes el abismo              \\
                              En líneas, cercos, círculos y...}}          \\
\end{center}
\normalsize

Al llegar aquí, la ley del maldito consonante me obliga a buscar el modo
de meter la palabra \emph{profundas}, para poder rematar con el
concepto:

\small
\newlength\mlenc
\settowidth\mlenc{\quad ¡Oh, Marcela! Si es Dios circunferencia }
\begin{center}
\parbox{\mlenc}{\quad \textit{Tú que de amor y gloria te circundas,       \\
                              Eres del centro de Dios mismo.» }}          \\
\end{center}
\normalsize

Apretándose los ijares, rompió D. Beltrán en una tan fuerte risa, que el
bueno de Nelet, desconcertado, cortó la vena poética. «¿Qué, señor?---le
dijo:---¿es que no están bien hilvanados, o que no hay bastante sutileza
y delgadez de razonamiento?

---Por San Jorge de Alfama y por el nombre que llevo---replicó D.
Beltrán llorando de risa,---te juro que desde que hay poesía no se han
compuesto versos peores\ldots{} Hijo mío, vuelve en ti; acógete a la
opinión leal y a la experiencia del viejo Urdaneta, y abandona un camino
por donde vas, no a la conquista, sino a la total perdición de la plaza
que quieres sitiar. Ven acá, y en un abrazo de amigo te comunicaré las
ideas que deben curarte de esa enfermedad que padeces. Los demonios y
los versitos son dos síntomas de un mismo mal: el mal de tontería,
Nelet\ldots{}

---Por Dios, que voy creyendo que tiene razón---dijo el discípulo
dejándose abrazar.

---¡Que si tengo razón!\ldots{} Como que a no cambiar de sistema,
Marcela se reirá de ti y acabarás por volverte loco. De un mal semejante
al tuyo padece ella, y no has de curárselo sino con la aplicación de la
medicina que produzca humor contrario a esas simplezas. Vuelve en ti;
levántate de ese terreno, verdadero corral de pavos, en que te has
caído. Ten presente que Marcela no ha de quererte por pavo, sino por
hombre. No seas con ella poeta huero, sé gallardo, fuerte, enamorado,
siempre varonil; antes que ñoño y quejumbroso, sé atrevido y jovial. No
hagas caso de duendes, que son muy mala compañía, ni te calientes los
cascos componiendo endechas, que, aun siendo superiores, no agradarían a
tu señora tanto como un buen poema de amor, sentido y expresado en los
hechos, no en las palabras.

---¡Es verdad, sí, sí! ¡Viva D. Beltrán!---exclamó Nelet entusiasmado,
abrazándole más fuerte.---Lo veo claro\ldots{} Hay que ser hombre,
galán, fuerte, apasionado, dispuesto para todo\ldots{}

---Sí: que vea y entienda la grandeza y el ardor de tu pasión; que en ti
admire el tipo del caballero amante, de corazón fogoso y voluntad firme;
que te tema un poco, pues es bueno una chispita de miedo para encender
amor; vea también que a todos infundes respeto; que eres bravo,
verdadero gallo en guerras y amores. Esta es mi opinión. Si no haces
esto, no cuentes conmigo\ldots{} Que te aconsejen los demonios y te
amparen los versitos.

---No; no hay consejero como usted, ni quien sepa más de cosas de mundo
y mujeres. A mi D. Beltrán me atengo\ldots{} Fuera demonios, fuera
ensueños, fuera poesía, que no es tal poesía, sino lo que usted
dice\ldots{} cosa de pavos\ldots{} Fuera los quejiditos y el no comer, y
el miedo ridículo\ldots{} El cuento es que cuando yo enamoraba a tantas
sin quererlas, sabía cumplir de palabra y obra; y a lo bruto\ldots{}
porque yo era un bruto\ldots{} me desenvolvía muy bien\ldots{} Pero con
esta no soy lo que fuí, ni acierto a enamorarla\ldots{} Y es que me
tiene prendada toda el alma, y el seso completamente sorbido\ldots{} y
todo mi ser como derretido en ella y transformado\ldots{}

---Acógete a mi doctrina, hijo, y adelante. Ganarás, ganaremos la
partida, porque algo me ha de tocar a mí como maestro: la satisfacción
de ver coronados tus deseos, de verle feliz, contento, padre de
familia\ldots{} ¡Y que no se alegrará poco este viejo de ver en ti y en
Marcela florecer nueva rama de la honradísima familia de Luco! Así se
redondeará todo, y evitaremos que el caudal de mi amigo vaya a parar a
manos muertas\ldots{} Con él constituiremos una gran familia tronco de
numerosa prole; y en esa familia prosperará la agricultura, la
industria, y resplandecerá la moral, la\ldots{} Ya ves, ya ves cómo
discurro y voy atando cabos. Hay que estar en todo, hijo mío.

---Venga otro abrazo---dijo Nelet con efusión, sintiendo que al mágico
influjo de aquella palabra persuasiva, el alma se le vigorizaba, y se le
inundaba el entendimiento de vivísima luz;---ya lo veo, ya lo veo. ¡Vaya
un talento macho!\ldots{} Adelante: soy hombre; no creo en duendes;
quédense los versitos para barberos y estudiantes\ldots{} Apresurémonos
ya, que aún estamos distantes del sitio en que hemos de pasar la noche.»

Grandemente excitado, D. Beltrán fue charlando todo el camino, y el otro
escuchaba gozoso las explanaciones que hizo de su pensamiento, y los
ejemplos admirables que refirió en corroboración de sus teorías. Con
esto se les pasó la tarde, y ya anochecía cuando llegaron al borde de la
barranquera que les separaba del monte de Vallivana. Para dar descanso
al viejo pararon allí, recreándose los dos en el paisaje que a sus ojos
se ofrecía: soledad en lo hondo, quietud en las alturas, la majestad de
la Naturaleza campando en su silencio augusto. Con precaución
descendieron hacia el río profundo, que fácilmente se vadeaba, y paso a
paso emprendieron la subida de la vertiente opuesta, guiados por
\emph{Malaena}; que sin este auxilio no habrían podido encontrar el
escalonado sendero entre la peña cubierta de vegetación. Llegaron por
fin a la meseta, donde había una fuente de agua cristalina dentro de un
nicho de variadas florecillas. En una gruta cercana descansaron. La
noche se les pasó en coloquios muy entretenidos y en ratos de tranquilo
sueño, después de una cena frugal. Al amanecer, previo lavatorio de cara
y manos en la fuente, emprendieron la marcha hacia el santuario. Según
los informes de la vieja, allí encontrarían a Marcela, que había llegado
la noche anterior traspasando la sierra de Bel.

En efecto, serían las siete cuando, vencida ya gran parte del fragoso
camino, vieron descender por entre matojos la figura mística de la monja
Luco, seguida de los viejos. Estos se quedaron atrás, y avanzó sola
entre el verdor de los jarales con lento paso de procesión: traía en la
mano una rama de espino florecido. Cuando estuvo casi al habla saludó a
sus amigos con grave sonrisa y un movimiento de la mano en que tenía el
ramo, y se sentó en una peña. No lejos de ella, otra peña baja y extensa
parecía puesta allí para que se sentaran los caballeros. Esmerádose
había la Naturaleza en la hechura de aquel estrado, para pláticas de
novios o para honestas reuniones. Se miraron los tres un instante.
Rompió el silencio Marcela con palabras de relleno: «¿Verdad, Sr. D.
Beltrán, que es agria la subidita? Siéntese aquí, a este lado mío. Tú,
Nelet, enfrente.

---La más penosa cuesta---dijo el anciano con refinada galantería,---se
vuelve ligera y fácil cuando al término de ella estás tú.

---Es lisonja, Señor\ldots{} No le quiero tan lisonjero.

---Es la verdad---afirmó Nelet, que ya se enojaba de permanecer
mudo.---Por ti, Marcela, subo yo a este monte y a otros más altos; y
cuanto más te subas tú, más gozo yo elevándome hasta donde estés: que es
obligación de lo humano remontarse a lo divino.

---¡Jesús mío!---exclamó la monja risueña, santiguándose.---¡Cuán
desatinados vienen hoy los dos!

---Alto ahí---dijo D. Beltrán, tomando pie de las últimas palabras de
Nelet:---si divina es Marcela, y como a tal la adoramos, no ocultemos
que ahora la quisiéramos humana, sin menoscabo de su divinidad, pues a
mi entender, lo divino y lo humano deben compenetrarse, constituyendo el
mejor estado dentro de la Naturaleza\ldots{}

---Alto ahí, digo yo ahora, y a fe de Marcela sostengo que no soy
divina, aunque a la divinidad aspira mi pobre humanidad baja, y la
compenetración de lo humano y lo divino ha de ser por el modo que la
propia divinidad señala cuando quiere hacer suyo lo humano.»

Si Marcela gozaba en este torneo conceptuoso, Nelet sufría de verse en
tales laberintos, donde se perdía su intellectus. Así, con gallardo
arranque llevó la cuestión al terreno de la sinceridad y llaneza: «No sé
si es humano o es divino el sentimiento que aquí me trae, Marcela,
sentimiento por el cual iría yo tras de ti hasta el fin del mundo. Lo
que te he dicho en mis cartas, ahora lo repito con el apoyo de mi buen
amigo: y es que te quiero. Dios encendió en mí una llama que me devora y
consume. Si me niegas el amor que te pido, creeré que este fuego es un
pedazo del infierno metido en mí.

---¡Oh! eso no---dijo Marcela prontamente,---que el amor viene siempre
de Dios. Fuego del Cielo es lo que te quema el alma, Nelet; mas no has
de pretender que yo rompa mis votos para darte la tranquilidad. El amor,
nacido en el alma, puede en ella tener su remedio, pues como divino, con
divinos medios se modera y aplaca.

---Eso no---dijo el anciano:---con perdón de la ciencia, el amor como
sentimiento de pura humanidad, sólo en la esfera humana encuentra su
remedio.

---Perdóneme el Sr.~D. Beltrán; déjeme concluir. Ha dicho Séneca que el
afecto de amor no se rige por la razón. Es sabido que el demasiado amor
es muy peligroso y acarrea desastres y muertes. Y así, yo repito ahora
el dicho de Chilon Lacedemonio: «No amarás ni desearás nada
demasiadamente.» Y de que el amor no se rige por la razón, tenemos en la
antigüedad ejemplos mil. Pigmalión y Alcidas Rodio amaron estatuas;
Pasifae Reina amó a un toro; Semíramis a un caballo; Jerjes Rey a un
árbol plátano; Hortensio Orador amó a una murena pescado; Cipariso a una
cierva, y muerta la cierva, murió él también de pesar\ldots{}

---Pero yo no amo a una estatua, ni a un pez, ni a un árbol---dijo Nelet
con viveza,---sino a una mujer, a un ser vivo y hermoso, en quien Dios
puso todas las perfecciones\ldots{}

---Déjame acabar mi argumento.

---Dejarla\ldots{} sí, dejarla---indicó D. Beltrán, que notaba en
Marcela un gran gusto de hablar de amor, y el empeño de disimularlo con
frialdades eruditas.

---Hemos sentado que el amor no se rige por la razón---prosiguió la
santa.---Y ahora, tratando de penetrar en la esencia de ese sentimiento,
digo que lo que mueve el amor del hombre es toda perfección de
Naturaleza\ldots{}

---Muy bien.

---Admirable.

---No lo digo yo: lo dice Aristóteles. Las cosas que incitan y mueven el
amor en el hombre son: sapiencia, hermosura, eutrapelia, que es como
decir buena conversación\ldots{} Pues apartando el alma de estas
perfecciones de Naturaleza, a que llamo perfecciones imperfectas, y
embebiéndola en la única perfección perfecta, que es Dios, el amor
humano se extingue, y el alma se ve purificada, gozosa y satisfecha en
el verdadero amor.

---Todo eso es muy sabio---dijo Nelet en pie, impaciente, decidido a
llevar las cosas por lo humano, pues tanta divinidad y sutileza de
palabra le enfadaban;---pero a mí no me traigas ese cuento de que el
amor de Dios quita el amor de mujer\ldots{} No: a Dios se le quiere como
Dios y a la mujer como mujer. Hombre soy, mujer tú. ¿Por qué no hemos de
amarnos y ser felices? ¿Para qué nos ha criado Dios? ¿Para que nos
aborrezcamos uno a otro y le queramos a Él? No, Marcela\ldots{} Eso es
un disparate, aunque lo digan Séneca, Aristóteles o San Simplicio. En
cuestión de amor sé yo tanto como esos y más, más\ldots{} Si quieres
darme una razón para no amarme, deja a Dios y a los santos en el Cielo,
y háblame como se habla entre criatura y criatura. Dime que no te
agrado, que no soy de tu gusto, y ante este argumento, que no es sabio
ni está en latín, no tendré más remedio que callarme y devorar mi
amargura y morirme de pena. Sí, Marcela, porque tu desprecio es mi
sentencia de muerte\ldots{}

---Bien, muy bien, Nelet---gritó D. Beltrán radiante de
satisfacción.---Así habla un hombre, y así te quiero, hijo mío.

---Hemos venido a pedirte una contestación a lo que de palabra y por
escrito te he dicho. Yo estoy loco por ti. Desde antes de conocerle te
amaba, y antes de verte te veía, y tan llena de ti tengo mi alma, que no
hay en ella intención ni pensamiento que no sean tuyos\ldots{} de lo que
se sigue que has de escoger entre quererme y que yo acabe mi vida. Esto
es quererte a ti y querer también a Dios. Pero no me pidas, ¡ay! que
quiera a Dios sólo sin dejar nada para lo humano, porque eso es
imposible.»

Marcela mordía un palito de la rama del espino, sin fijar los ojos en
ninguno de los caballeros, perdida su mirada en vagos espacios. D.
Beltrán se aproximó a ella para observar su rostro, en el cual creía
notar cierta turbación o pugna de sentimientos, y aprovechando estado
tan ventajoso, hizo seña a Nelet de que callase, dejándola un rato en
aquel solemne careo consigo misma.

\hypertarget{xxii}{%
\chapter{XXII}\label{xxii}}

«No me negarás---dijo D. Beltrán, poniendo suavemente su mano en la
rodilla de la santa,---que el hombre en cuyo corazón has encendido fuego
de amor tan grande, es merecedor de tu cariño. Caballero leal en todas
sus acciones, será para ti el mejor compañero que Dios podría depararte.
¿Lo niegas?\ldots{}

---No señor---replicó Marcela mirando al suelo;---no puedo negar lo que
es verdad: reconozco sus buenas partes, y por su rendimiento y
constancia me veo precisada a tenerle estimación; la estimación que
permiten mis estrechos votos\ldots{}

---Por algo se empieza, hija mía. Y ahora te digo que a Dios no podría
ofenderle que trocaras la vida religiosa por la que llamamos mundana.
Dios hizo el mundo, hizo la humanidad para que en él viviese y de él
gozara, y creó el amor para que la humanidad se prolongase hasta lo
infinito, de padres a hijos\ldots{}

---Y no sé yo---dijo Nelet con bárbara lógica,---que hiciera Dios
conventos, ni mandase a hombres y mujeres que se apartaran de la
existencia material\ldots{} porque la existencia material es el
fundamento de toda vida y hasta del amor de Dios; porque para amar a
Dios tenemos que vivir, y para vivir tenemos que nacer, y para
nacer\ldots{}

---Aunque me ven ustedes silenciosa---indicó la penitente dando un
suspiro,---no crean que me faltan razones para contestar a lo que uno y
otro me dicen.

---¡Oh! Ya sabemos que silogismos y citas sagradas y profanas, no han de
faltarte\ldots{} Pero ahora nos harás el favor de guardar a todos los
sabios en el archivo de tu memoria, y no consultar más texto que el de
tu corazón. ¿Qué te dice este? ¿Que desprecies a Nelet?

---No me dice que le desprecie---replicó la monja sin mirar al
interesado;---pero me persuade a no cambiar la vida de penitencia por
otra vida.

---Pues yo he leído en no sé qué autor---dijo Nelet altanero,---que la
primera penitencia es el matrimonio, y la mayor gloria humana criar una
familia. Y si te decides a permanecer en el siglo, donde me encontrarás
amante, esclavo fiel, no te pesará, Marcela, y verás cómo Dios te quiere
más y te bendice\ldots{} pues la vida que llevas no es vida de persona
racional, ni Dios nuestro Criador puede querer eso.

---No creáis---repitió Marcela, inquieta y como azorada, sin mirarles,
mascando el palito,---que porque callo me faltan razones\ldots{} Mas no
quisiera que las razones que se me ocurren las tomara Nelet a
desprecio\ldots{} No, no: desprecio no es\ldots{} Y\ldots{} no sé cómo
decirlo\ldots{} Es que aunque yo me propusiera arrancar de mí el amor de
la vida religiosa y el gusto grandísimo de cumplir mis votos, no podría,
no podría\ldots{} Es más fuerte que yo mi devoción\ldots{} Pero el
afianzarme en ella no significa desprecio\ldots{} no\ldots{} Considero
lo que Nelet merece\ldots{} y yo pediría al Señor que le concediese, en
criatura mejor que yo, la satisfacción de su fina voluntad\ldots{} Que
las hay mejores, sí, mejores que yo, de superior mérito físico y moral,
así por la presencia como por las virtudes\ldots{}

---No, no hay quien te supere---exclamó Nelet levantándose con furor de
abrazarla,---ni siquiera quien te iguale. Marcela, en dos letras
pronunciadas por tu boca está la ventura y la salvación de un hombre.
Pronúncialas. Fácil, como el respirar, es decir \emph{sí}\ldots{} El
\emph{no} es sentencia de muerte, y tus labios divinos no me
condenarán.»

Levantose Marcela, y poniendo en su rostro y en su acento una severidad
que el menos lince habría tenido por afectada, dijo a los caballeros:
«Con su venia subiremos a la iglesia, que yo tengo que rezar, y ustedes
también, pues han venido a cumplir una promesa.»

Sin esperar respuesta, echó a andar hacia arriba con grave paso,
echándose al hombro la rama de espino que decoraba graciosamente su
gallardo busto. Quiso Nelet avanzar tras ella para proseguir el coloquio
interrumpido; pero D. Beltrán le detuvo vigorosamente por un brazo, y
aguardando a que la santa se alejara, le dijo: «Tonto, ¿no has
comprendido? Es nuestra, es tuya.

---Me ha parecido que su espíritu no es insensible al amor de hombre.

---Calla, hijo\ldots{} Desde que comenzó a soltar filosofías y citas de
autores, observé que viene transformada. ¿Qué eran aquellas sutilezas
más que un coqueteo de arte mayor? Es mujer, es mujer; hemos triunfado.

---¡Mujer!---repitió Nelet como en éxtasis.

---Pero ¿no ves esos andares?\ldots{} ¿No ves cómo se recoge la saya
para andar cuesta arriba? ¿Y esa manera de llevar la rama
florecida?\ldots{} No es mala sofoquina la que le hemos dado con nuestro
razonar irrebatible. Mírala, hombre, y dime si eso no es una mujer
disfrazada de santa\ldots{} El cuento es que está guapa de veras\ldots{}
La he visto muy de cerca; me he fijado bien. Los dientes son ideales; no
extraño que hayas soñado con ellos. ¡Y qué perfil el de su cara! ¿Pues y
los ojos?\ldots{} Nelet, dame un abrazo\ldots{} Estás de
enhorabuena\ldots{} Yo no la distingo ya más que como un bulto. ¿Va muy
lejos? ¿No mira para atrás?

---Todavía no ha mirado.

---Ya, ya la veo. Allá va. Pues bien, Nelet, yo te apuesto lo que
quieras a que antes de llegar a aquel peñasco negro\ldots{} ¿No hay allí
un peñasco?

---Es una encina.

---Pues te apuesto a que antes de llegar a la encina, se para y nos
mira\ldots{} a ver si la seguimos. No, no te muevas.»

Resultó, en efecto, lo que el ladino viejo decía. Parose la penitente, y
agitó la rama como diciendo con ella: «¿Pero qué hacen que no suben?»

Como el tardo paso de D. Beltrán no permitía la ascensión rápida,
Marcela se adelantó largo trecho. De rato en rato miraba, y Nelet le
hacía señas de que se detuviese; mas no hacía caso, y cuando los
caballeros llegaron al santuario, ya la monja y sus viejos rezaban ante
el altar con gran recogimiento. Arrodilláronse no lejos de la puerta, a
distancia de Marcela, para poder hablar a su gusto. «Trastornadita y
blanda la tienes ya---decía Urdaneta.---Y no debes atribuir esta mudanza
a la constancia de tus manifestaciones amorosas. Obra es del contacto
continuo con la Naturaleza, de la vida al aire libre, de la libertad, el
campo, las montañas, los bosques sombríos y las fuentes cristalinas. Ya
conocían el paño los que establecieron para penitencia de hombres y
mujeres los recintos cerrados. La sociedad es gran conductora de amor;
lo es también la Naturaleza\ldots{} Por más que aún se defiende con sus
sabidurías acartonadas, se ve que está vencida, tocada del mal de amor.
En los andares lo conozco, en el metal de voz. A mí no me engaña
queriendo hacer papeles de teóloga. Para rendir por completo su
voluntad, y que nos largue un sí tan grande como esta iglesia, hemos de
proceder con tino. Mucho cuidado, Nelet, con lo que ahora le
digas\ldots»

Nelet rezaba; el prócer hizo lo mismo, pidiendo a la Virgen que le
mejorara la vista y que le sacara del cautiverio que tan injustamente
sufría. Examinaron luego la iglesia, conducidos por la santera, pues
allí no había sacristán ni hombre alguno; vieron también el camarín y la
imagen, y se salieron al atrio a pasearse y fumar un cigarrillo\ldots{}
Marcela, terminados los rezos, apareció al fin, tras larga espera, y
tomando de la mano a D. Beltrán, guió a los dos caballeros a un lugar
abrigado junto a la hospedería, al pie de copudos robles. Sentados los
tres sobre la hierba, continuaron su coloquio, siendo ella la que rompió
con estas palabras: «He pedido a Dios y a la Virgen con todo fervor que
me iluminen. No siento aún desgana de mis votos benditos, ni sombra de
afición a otra vida. También he pedido al Señor que derrame alguna
frialdad sobre ese fogoso afecto de Nelet, y espero que\ldots{}

---Esto no lo enfría Dios---dijo el enamorado.---Lo que hace es avivar
la lumbre, y cuanto más te miro, más me enciendo, Marcela. Yo he pedido
a Dios que de este fuego que a mí me sobra te dé a ti algunas ascuas,
infundiéndote el gusto de familia, de vida doméstica\ldots{}

---Sí, hija mía: si te incitara Nelet a cosas impuras y pecaminosas, tus
escrúpulos serían muy justificados; pero te propone, y yo con él, la
unión bendita y santa ante el altar. ¿Qué sacas de esta vida errante? ¿A
quién haces feliz con tus penitencias? ¿No es más cristiano y caritativo
que libres de la muerte a un hombre honrado, y trueques sus martirios en
dulzura, su infierno en cielo?

---¡Vive Dios---exclamó Nelet con insana vehemencia,---que lo ha
expresado D. Beltrán como el mismo Evangelio! Quisiera yo ver a Dios,
como os estoy viendo a vosotros, para preguntarle delante de ti: «Dios,
¿no es verdad que tengo razón y ella no la tiene?»

---Cálmate, Manuel---dijo D. Beltrán, alarmado de tanto ardor.---Yo veo
en el mirar dulce de este ángel, que nuestras razones han ganado su
entendimiento, que Dios pone el dedo en su voluntad y le dice: «Hija
bendita, levántate y sigue a tu esposo.»

Pausa. Nelet, pálido como un difunto, miraba al suelo, y con su
temblorosa mano se agarraba los mechones menos cortos de su cabello.
Marcela tenía el rostro encendido, la respiración anhelante. Dejando
caer a un lado su cabeza en actitud de Dolorosa, arqueando las cejas y
bajando los párpados, pronunció estas palabras, sin autorizarlas con
sentencias de santos ni de filósofos: «Uno y otro, despiadados, me ponen
en grande suplicio. Yo quiero ver a mi lado el bien y veo el mal; por
causa mía inocente, enferma Nelet de la peor dolencia, de aquella para
que no hay consuelo ni medicina, como no sea ella misma y las punzadas
de su propio dolor; esto veo y no puedo remediarlo, que si en mi mano
estuviera, pronto lo haría. Así, les ruego que no me atormenten más y me
dejen partir.

---¡Partir!---exclamó Nelet suspenso, echando de sus ojos un siniestro
rayo.---¡Partir y dejarme en esta ansiedad! ¿Partir tú y no conmigo? ¿Es
que no quieres verme más? Marcela, por Dios, no me lo digas; no quieras
verme trocado de hombre en fiera\ldots{} no ofendas a Dios convirtiendo
en monstruo a una de sus criaturas\ldots{} Si por otra causa o razón no
te decides a quererme, hazlo por la santa obra de salvar un alma\ldots{}
¿No te convenzo al fin?

---Si con que yo te vea y te hable, tu alma se sostiene en Dios---dijo
la santa, bondadosa,---te veré siempre que gustes y haya buena ocasión
de ello. Al decir que me dejarais partir, no quería, no, alejarme de ti
para siempre\ldots{} decía que es hora de que por hoy nos separemos. Y
en esta ausencia, ofrezco yo a Nelet con toda lealtad que seguiré
pensando en el grave caso, y pidiendo a Dios fervorosamente que me
ilumine para resolverlo.

---Yo te aseguro---declaró Santapau con acento en que se revelaba el
propósito de una resuelta acción,---que si al decir que partías lo
hubieras hecho en son de despedida para siempre, antes de que te fueras
me habrías visto arrojarme por aquel despeñadero que da al barranco de
Vallivana.

---Hijo mío, Marcela te promete volver, y volverá---indicó Urdaneta
conciliando voluntades con frase cariñosa.---Yo quedo de fiador.
Tendremos otra entrevista dentro de pocos días, en el sitio que
designaremos\ldots{}

---Y no sólo he de consultar con Dios---agregó la beata,---sino con mi
hermano Francisco; que es bien le dé cuenta de esta terrible
novedad\ldots{} De aquí me iré en busca de un confesor, a quien
manifestaré las turbaciones hondísimas que han levantado en mí las
palabras tentadoras de uno y otro; luego iré en busca de mi hermano, y
hecho todo esto, les avisaré por \emph{Malaena} para que nos reunamos.

---Y me des respuesta de vida o muerte---dijo el galán.---Está bien. Si
me matas, mátame de un solo golpe. Si he de vivir, sépalo también
pronto, para no vivir muriendo\ldots»

Levantose Marcela, diciendo con gracia mujeril, que D. Beltrán apreció
como síntoma felicísimo: «Me dan permiso para retirarme?

---¿Tan pronto?---murmuró Nelet.

---Me equivoqué, señores míos---añadió ella con nueva emisión de gracia,
acompañada de sonrisa un tanto picaresca.---No debí pedirles permiso
para retirarme, sino para suplicarles que se retiren\ldots{} Perdónenme.
Y para que nadie se ofenda, ustedes y yo nos retiraremos al mismo
tiempo, por distintos lados\ldots{} Yo me voy monte arriba, a salir a
Bel.

---Y nosotros barranco abajo a salir a donde Dios quiera---replicó D.
Beltrán.---¿Ves?\ldots{} Nelet no se conforma con que nos prives tan
pronto de tu divina presencia\ldots{} Pero yo le persuadiré a la
resignación; descuida. Tiene en mí un aliviador de sus males de ánimo, y
un atemperante de sus nervios.

---Me conformo, sí---dijo Nelet con noble ademán.---Propuesta por ti la
separación con ese modo gracioso y\ldots{} de mujer, la acepto\ldots{}
Más te quiero mujer que santa, y entre santa de todos y mujer mía,
prefiero esto\ldots{} porque la santidad no llega tan adentro del alma
como el querer entre criaturas\ldots{}

---Yo celebro verte en esa conformidad---afirmó ella, dando los primeros
pasos hacia el sendero que había de seguir.---De las diferencias entre
santicio y mujericio, mucho podría decirte; mas ahora no puede ser.

---¿Tardarás mucho en decírmelas?

---Dios es quien ha de fijar el cuándo. Él solo es el marcador de las
ocasiones.

---Bueno: también me conformo. Esta mansedumbre que en mí ves no tiene
otra causa que el haberte visto benigna\ldots{} Has sonreído, Marcela, y
sólo con eso me desconozco, me siento mejor de lo que fuí.

---Ahora\ldots{} como si lo viera\ldots---dijo la penitente, sonriendo
con más gracia y viveza que antes,---irán ustedes caminando despacito, y
parándose a cada instante para mirar hacia atrás.

---¿Y tú no harás lo mismo?---observó Nelet más vivo que la pólvora.

---Si alguna vez vuelvo la cara---replicó ella conteniendo la
risa,---será por observar la tontería de los hombres, y porque no crean
que es desprecio el no mirar alguna vez\ldots{} Vaya, en marcha. Nelet,
D. Beltrán, el Señor les acompañe.»

Se separaron lentamente, y como a diez pasos gritó D. Beltrán: «Conste
que no soy yo el que mira, sino este truhán, vicioso del mirar.

---Adiós,» repitió la divina mujer.

A bastante distancia, hablaban así los dos caballeros: «¿Qué?\ldots{}
¿Se detiene a mirarnos?

---Ahora\ldots{} ¡Y que no haya tenido yo valor para darle un abrazo!

---Calma, hijo. Tiempo tienes. Y ahora, ¿vuelve la cara?

---Va despacito\ldots{} alza los ojos al cielo. Ya no la veo. Pasa
detrás de un grupo de árboles\ldots{} ¡Qué figura, qué aparición
celestial!\ldots{} Yo estoy loco.

---Calma\ldots{} Repito que tiempo tienes. A punto de completa madurez
la verás pronto.

---Ahora reaparece otra vez.

---¿Y mira?

---Sí señor\ldots{} Se ha puesto en la boca una ramita de hinojo. ¡Ay,
qué delicia de hinojo!\ldots{}

---Tiempo tienes\ldots{} Anda, anda\ldots{}

---No, no es de este mundo esa mujer.

---De este mundo o del otro\ldots{} tuya es.»

\hypertarget{xxiii}{%
\chapter{XXIII}\label{xxiii}}

Muy consolado el uno en sus fatigas amorosas, satisfecho el otro del
buen giro que a su parecer tomaba el asunto en que como consejero
intervenía, llegaron los dos caballeros a Catí. De lo que hablaron por
el camino no se hace mención. Baste decir que a los recelos que
manifestaba Nelet, como amante que con menos que la definitiva victoria
no se satisface, oponía Urdaneta las seguridades optimistas, fundado en
su conocimiento y larga práctica de negocios mujeriles. Para el anciano
prócer era como tenerlo en la mano. De allí a las bendiciones
matrimoniales poco trecho había que recorrer.

Hallaron en Catí la novedad de que Cabañero había salido con dos
batallones, por orden del General, y en su lugar quedaba Llangostera,
pronto también a partir con fuerza considerable hacia la frontera de
Cataluña. A la mañana del siguiente día, pasó por allí Cabrera con su
ejército en veloz marcha. Venía de cerca de Murviedro, donde se había
batido con las tropas de Oraa, y a Gandesa se dirigía llevando algunos
cañoneros para poner formal sitio a esta plaza. Grande fue la desazón
del pobre Urdaneta cuando le despertó Santapau para decirle: «Mi querido
viejo, la fatalidad, y en su nombre D. Ramón Cabrera, ha decretado que
nos separemos. Desde Salvasoria mandó aviso de que se incorporen sin
dilación a su ejército el 3.º de Tortosa y tres compañías del 1.º de
Valencia. Parece que vamos a sitiar a mi pueblo\ldots{} No puedo, ni con
pretexto de enfermedad ni con otra artimaña, librarme de la maldita
obediencia al superior\ldots{} Pero ya me canso, ya me canso de la
esclavitud, y a la primera oportunidad pediré la absoluta. Imposible
repicar y andar en la procesión que usted sabe. Amor y ordenanza no
casan bien\ldots{} Y no más, amigo mío. Le dejo bien recomendado a
Llangostera, que se ha de situar en Rossell, para cortar el paso del Pla
a las tropas que vayan en auxilio de Gandesa\ldots{} Con que
adiós\ldots{} No siento más sino que venga \emph{Malaena} y no me
encuentre. Pero ya le advertí que en este caso se vea con usted\ldots{}
Con decirle dónde estoy, basta. Es buen sabueso; dará conmigo\ldots{} No
puedo detenerme ni un segundo más. Adiós.»

Muy triste se quedó el pobre caballero, señor de tantas torres; y su
único consuelo fue que a poco de despedirse de Santapau le deparó Dios
una antigua amistad, el capellán mosén Putxet, que dos días antes había
llegado con destino al 1.º de Tortosa, de la división de Llangostera.
Aunque no podía sustituir el clérigo la franca y ya entrañable amistad
de Nelet, al menos le entretenía con su charla, y le prodigó no pocas
atenciones, entre ellas el agenciarle una buena mula para el paso desde
Catí a Rossell, que Llangostera, con seis batallones, efectuó en la
noche del 15 de Mayo y parte de la mañana del 16. Llegó D. Beltrán
molido y displicente por el duro trotar de la condenada bestia, y lo
primero que solicitó de la bondad de su amigo fue que le metieran en
cualquier mechinal, para poder estirar su esqueleto y darse algún
descanso. En un aposento de la sacristía de la iglesia mayor le colocó
Putxet, con gran satisfacción del noble, que no esperaba tan buen
hospedaje.

Lo que deseaba era que le dejasen allí, previo juramento solemne de no
quebrantar su esclavitud y estar siempre a disposición de la autoridad
carlista que le reclamase. Pero ¡ay! que si el cielo le concedió la
quietud material que por el momento deseaba, no fue benigno con él en
aquellos tristes días. El 18 muy temprano, cuando las claridades del
alba despuntaban por Oriente, despertó el caballero con sobresalto, sin
que nadie le llamase, por efecto de un súbdito golpetazo de su corazón.

«¿Quién está ahí?---dijo sin moverse, viendo avanzar hacia su lecho un
bulto negro.

---Soy yo, querido D. Beltrán---respondió al poco rato Putxet, pues no
era otro acercándose más.---No venía a despertarle, sino a ver si
dormía\ldots{} Pero es temprano\ldots{} Duerma una hora más\ldots{}
aunque sean dos horas\ldots{} todo lo que quiera.

---¿Qué sucede? ¿Tenemos que partir?

---No, no\ldots{} Por ahora no\ldots{} Es que\ldots{} Sentiría mucho que
usted se alterase\ldots{} Calma, ilustre señor. Me voy, para que duerma
otro poquito.

---Ya no podré dormir, caramba, pues esta entrada de usted a hora tan
intempestiva, la turbación que noto en su acento, son para despabilar al
sueño mismo. Me dice el corazón que tiene usted algo que\ldots{}
comunicarme.

---No es tiempo aún\ldots{} ¿Quiere usted que se le haga café?\ldots{}

---¡Demonio! Tan pronto me dice que duerma como me ofrece café. Ea,
Sr.~Putxet, ¿qué le trae acá? No valen melindres conmigo.

---Pues sí---dijo el capellán, que en su tristeza y azoramiento, cuanto
más hábil quería ser, más torpemente procedía.---Mejor será que se
despabile y se levante\ldots{} No se altere, señor, no pierda su aplomo
y serenidad\ldots{} A un hombre como usted, tan entero y\ldots{} y que
se hace cargo de las cosas\ldots{} se le puede decir\ldots{} Nada, no es
nada, señor; es que\ldots{} ha ocurrido una gran desgracia.

---Acabe usted, acabe, hombre pusilánime, hombre enclenque, hombre
femenino\ldots{}

---Pues sepa el hombre fuerte, sepa el hombre valeroso y grande, que
ayer, en un pueblecito llamado Belén, más allá de Tortosa, los infames
cristinos fusilaron a D. Alonso de Almela, hermano del Conde de Catí.

---Y, en represalias de esta barbarie, los infames carlistas harán lo
mismo con el noble D. Beltrán de Urdaneta---gritó el anciano, poniéndose
en pie, medio desnudo, sobre el camastro.---Bien, bien: aquí me tenéis,
asesinos; aquí estoy dispuesto a morir. Noble por noble, como me dijo en
Cheste el jefe de los matachines, Ramón Cabrera\ldots{} ¡Y para
anunciarme esto, Sr.~Putxet, ha estado ahí tartamudeando y poco menos
que haciendo pucheros!\ldots{} Aguarde a que me vista; dispense que
tarde en ello algún tiempo, pues acostumbrado a valerme de ayuda de
cámara, soy algo torpe en estas operaciones matutinas\ldots{} Pero si
tienen mucha prisa por despacharme, ¡demonio! llévenme a medio vestir,
que la muerte no ha de poner reparo. Por falta de ropa, ni he de ser
menos animoso, ni vosotros menos viles y cobardes.

---Si no hay prisa, señor---dijo el capellán abrazándole.---De aquí a
las nueve nos sobra tiempo\ldots{} Y pues tiene la costumbre del ayuda
de cámara, yo soy bastante humilde para prestarle ese servicio.

---Gracias, no pretendía yo tanto---replicó D. Beltrán sentándose en el
lecho, mientras el otro le traía las botas, el pantalón, disponiéndose a
vestirle.---Y pues con tanta generosidad mis verdugos me conceden estas
horas, sepa que no renuncio al café que me ha ofrecido\ldots{}

---Al momento mandaré que se lo preparen. ¡Pues no faltaba más! Sería
una desconsideración imperdonable privarle de alimento.

---Bien, hijo, bien: se agradece\ldots{} ¡Con qué destreza me ayuda a
vestirme! Parece que en toda su vida no ha hecho usted otra cosa.

---Fuí paje del ilustrísimo señor D. Víctor Sáez, Obispo de Tortosa.

---¡Sáez, el Ministro del absolutismo! ¡El que ayudó a Fernando VII en
su tarea de ahorcar a medio mundo! Bien, hombre, bien. Pues ya que usted
tiene la bondad de ser por un instante mi criado, no vacilará, si es tan
humilde, en prestarme todos los servicios que necesita un hombre como
yo\ldots{} Adelante\ldots{} Tenga usted cuidado con esta pierna. Trátela
con miramiento, que está reumática\ldots{} Ahora el chaleco\ldots{} Este
me lo dio D. Ramón, y me ha hecho un gran servicio. Bueno, bueno\ldots{}
No corra usted tanto\ldots{} Le recordaré el dicho de nuestro gran
tirano, el ahorcador de gentes, Fernando el Deseado\ldots{} contra una
esquina\ldots{} Ya sabe usted que él fue quien dijo: `Vísteme despacio,
que estoy de prisa'. Ahora, hágame el favor de pedir el café\ldots{}

---Lo tendrá usted a punto. Sabe Dios cuánta pena me causa tener que
notificarle\ldots{} Anoche me llamó Llangostera, que entre paréntesis,
está muy afligido por verse en el duro trance de\ldots{}

---¡Pobrecito! Si está tan afligido, le compadezco\ldots{}

---Pero el deber\ldots{}

---Claro, el deber\ldots{} En estas guerras salvajes, trastornadas las
conciencias, aplicáis a los crímenes palabras santas que se inventaron
para expresar la virtud, y asesináis en nombre de la justicia, que es
como poner al diablo en los altares\ldots{} Bien\ldots{} que sea pronto.

---Suplicome el Sr.~Llangostera que me encargase\ldots{} y con gran
sentimiento acepté comisión tan triste\ldots{} Era yo el más significado
para este paso, por la amistad\ldots{} de que me honro.

---La honra es mía. No sea usted tan modesto\ldots{}

---Y encargome al propio tiempo que le preparase\ldots{} si usted se
dignaba elegirme entre los cuatro señores capellanes que estamos hoy en
Rossell.

---Hijo, sí, por elegido\ldots{} Lo mismo me da.

---Mi amistad atribulada---dijo el capellán buscando una bonita
expresión retórica,---se consuela con esta preferencia que el noble
caballero se digna concederme.

---Mi confesión no será larga---indicó Don Beltrán paseándose por la
habitación,---y si usted quiere, ahora mismo\ldots{}

---Antes haré que se le sirva el café\ldots{} No hay en ello
inconveniente, pues no tendremos comunión\ldots{} y no por culpa mía. El
párroco del pueblo nos ha hecho la jugada de abandonar su iglesia para
unirse a la partida del \emph{Organista}. Está el hombre furioso desde
que los liberales le mataron al sobrino\ldots»

No necesitó el capellán separarse de su amigo para la diligencia del
café, pues el oficial de guardia en la estancia próxima, interesado
también por D. Beltrán, y de su desgracia compadecido, había dado las
órdenes para que se le llevase pronto aquella tónica bebida. Dio el
anciano las gracias a los que se la sirvieron, mostrándose con todos muy
afable. Tomado el café, que por singular merced no estaba mal hecho,
volvió al capítulo de su confesión, diciendo con animado lenguaje:

---Pues sí: mi conciencia ve su luz y su sombra perfectamente
deslindadas, y no vacila al señalarlas\ldots{} No hay en mí casos
dudosos, enigmáticos, obscuros. Soy claro y bien definido\ldots{} En
esta crítica hora, mi memoria se aviva, y no habrá nada que se me quede
en el tintero, llamando tintero al antro del olvido. Lo que Dios sabe,
yo lo digo sin rebozo y con facilidad al sacerdote que me auxilia, a
cuantos quieran oírlo, pues la vida de Beltrán de Urdaneta es pública,
su carácter, bien diáfano, y sería en mí ridículo melindre el hacer un
misterio de lo que sabe todo el mundo, todo Aragón\ldots{} Soy público
en Aragón; soy popular, mejor dicho\ldots{}

Y observando que oficiales y soldados, de guardia en la estancia
próxima, se asomaban a la puerta movidos de curiosidad, les dijo:
«Entren si gustan, y oigan; que los pecados que declara mi boca no son
tales que produzcan espanto, y refiriendo mis maldades, puedo decir que
el que se encuentre limpio de ellas, tire la primera piedra. No es que
yo deje de creerlas vituperables; al contrario, en esta hora clara de la
conciencia, veo y reconozco cuánto he ofendido al Señor, y qué mal uso
hice de las cualidades que se dignó poner en mi alma. Siempre fuí
religioso, creyente ciego de cuanto su Iglesia nos enseña, aunque muy
perezoso y descuidado en cumplir los preceptos que se nos dieron para
conservar y enaltecer el nombre de cristianos. He faltado en esto
gravemente, más que por desamor de Dios, por la continua distracción en
que me tenía el bullicio vano del mundo, y las frivolidades con que la
sociedad noble embelesa nuestros sentidos. Siempre fuí más devoto de los
placeres que de las abstinencias, y más gustoso de la buena vida que de
las mortificaciones, sin llegar nunca a la embriaguez ni a la
glotonería, y no porque ambos excesos son pecados, sino porque siempre
les creí de mal gusto\ldots{} He sido vanidoso, amante de la ostentación
y de la lisonja, mirando siempre a que lo mío fuese superior a lo ajeno,
a que ninguno me igualara en grandeza y lujo; y cuando veía por alguna
parte algo que me obscureciese, sufría mal de tristeza, y me lo curaba
con nuevos esfuerzos para extremar la presunción y humillar a los
demás\ldots{} Pero también digo que jamás cometí vileza contra nadie, y
que conservé la dignidad que mi raza y mi nombre me imponían,
mostrándome siempre caballero noble, con los iguales cortés, afable y
cariñoso con los inferiores\ldots{} Mi pecado mayor, manantial
inagotable, en vida tan larga, de innumerables errores, ha sido mi
locura, que así la llamo, de galantear y ser grato al bello sexo. Mi
goce más vivo fue en todo tiempo el trato de damas altas, bajas o
medianas, y llamo damas a cuanto se comprende dentro de la muchedumbre
femenina. Mi desatino ha sido tal, que todo lo he pospuesto a la
satisfacción de mis gustos. Verdad que dentro del fuero del amor no he
cometido vilezas; pero sepan que ese fuero es puro artificio inventado
para nuestro uso por los galanteadores, y que no vale ante la ordenanza
del Decálogo. Yo, pues, he pecado gravísimamente, y al declararlo,
reconozco sin atenuaciones ni disculpas todo el mal que hice, añadiendo
que mis infamias no tu vieron término por severidad de mi conciencia,
sino porque el desmayo de la naturaleza les puso freno, contraviniendo
mi liviandad y hábitos viciosos. De esto me acuso, y reconociendo mi
error, me encomiendo a la Misericordia divina.

»También es pecado grave el poco o ningún cuidado que puse en el manejo
de mi hacienda; que la riqueza, Dios nos la da para que la usemos con
templanza y la transmitamos a nuestros hijos. Yo he sido una mano
verdaderamente horadada. Ciertamente que algo atenúa este pecado mi
generosidad sin límites, pues todo se ha de decir: yo hacía partícipes
de mi bien a cuantos me rodeaban o se me acercaban en demanda de
auxilio. Yo he remediado muchas miserias, enjugado no pocas lágrimas.
Ningún colono ni sirviente mío puede decir que le oprimí; y si esto se
lleva como litigio a tribunal divino para fallar sobre mi alma, tengo
por cierto que innumerables seres depondrán en favor mío. Váyase lo uno
por lo otro, que si largamente derroché, con no menor largueza di mi
mano a los miserables para que se agarraran\ldots{} Defecto capital mío
ha sido el amor a ese resorte de vida material que llamamos dinero,
despreciado por los filósofos, vilipendiado por la religión, pero del
cual no podemos prescindir dentro de la sociedad a que pertenecemos,
porque su empleo y distribución se ha hecho ley que a todos nos sujeta,
so pena de volvemos salvajes o ermitaños, lo que no digo que sea peor ni
mejor que el estado social. Sólo afirmo que mis apetitos, mi presunción,
me han espoleado siempre para proveerme de ese metal, que no llamaré
precioso ni vil, dejándole en esta ocasión sin ningún título ni apodo.
Pero bien sabe Dios que en las situaciones aflictivas a que me condujo
el afán de prolongar mis goces y conservar mi fama de rumboso y señoril,
jamás tomé nada que no viniese a mí por caminos legítimos, aunque
ruinosos. Sobre mi conciencia pesan muchos pecados, muchos; pero no pesa
ni un solo maravedí que pueda llamarse ajeno. Si alguna vez me rebajé al
empleo de resortes que humillaban un tanto mi dignidad, nunca me movió
el intento de traer a mí lo perteneciente a otro\ldots{} eso nunca.
Limpio estoy de esa clase de manchas\ldots{} No puedo decir, ¡ay de mí!
que de todas esté limpio, pues pecador fuí, por pecador me tengo, y como
pecador empedernido me confieso en la hora de mi muerte. Ya lo habéis
oído; ya veis, señores, la conciencia de D. Beltrán de Urdaneta, a quien
todo Aragón llamó en otro tiempo \emph{D. Beltrán el Grande}. Ni cosa
mala he callado, ni cosa buena hay fuera de lo manifiesto. Si algo se me
olvida, quiera Dios ordenar mi memoria de modo que los olvidos sean de
cosas y hechos favorables, y que nada de lo malo se me quede escondido
en la mente. Creo que no\ldots{} Tal como fuí y como soy, a vosotros, a
mi confesor y amigo me presento; y sumiso, pesaroso de haber
menospreciado la divina ley, entrego mi alma a Dios, infinitamente
Justiciero, infinitamente Misericordioso.»

\hypertarget{xxiv}{%
\chapter{XXIV}\label{xxiv}}

Cuantos vieron y oyeron al infortunado caballero aragonés, quedaron
maravillados de su sinceridad y presencia de ánimo. Del grupo de
oficiales y soldados que en la puerta se arremolinaban, se destacó uno,
al parecer teniente, que adelantándose hacia el prócer y besándole la
mano, le dijo: «Señor, cuando esté usted en el Cielo, acuérdese de un
servidor, Nicasio Pulpis, que tiene sobre su conciencia los mismos
pecados de usted y no sus virtudes.

---Bien, hijo---replicó D. Beltrán abrazándole.---Que mis desgracias y
fin desastroso te sirvan de espejo para que en él te mires y procures
enmendarte.»

Putxet, en tanto, inconsolable, expresaba su consternación en estos y
parecidos términos: «Una y otra vez he dicho al señor Llangostera que
hoy no es día hábil para ejecuciones. Figúrese usted: domingo, y por
añadidura Pascua de Pentecostés\ldots{} ¡Cuando la Iglesia conmemora
nada menos que el grandiosísimo misterio de la venida del Espíritu Santo
en forma de lenguas de fuego sobre las cabezas de los Apóstoles, para
infundirles la divina ciencia!\ldots{} ¡cuando tal festividad augusta y
solemne celebramos, tener que consumar un cruento sacrificio, por más
que las leyes de guerra, ¡malditas leyes! lo autoricen y
sancionen\ldots! No, no puede ser: protesto\ldots{} y he de insistir,
pidiendo que se deje para mañana. Me parece que corriendo a mi cargo la
dirección espiritual del regimiento, tengo derecho a que se me
oiga\ldots{} No estamos aquí los capellanes sólo para confesar de prisa
y corriendo\ldots{} Vea usted, por no hacerme caso, hoy no puedo
celebrar: no tenemos formas\ldots{} Es inconcebible este
descuido\ldots{} ¡Pues cartuchos no faltarán! Todo lo de guerra está
corriente, eso sí\ldots{} y lo espiritual, nada\ldots{} Así anda ello.

---No se sulfure, amigo Putxet---le dijo D. Beltrán, que se había
sentado y quería meditar.---Y no se apure por el aplazamiento de
mi\ldots{} sacrificio. ¿Qué más da un día que otro? Si el día es
solemne, no importa. Bien sabe Dios que andan ustedes algo atropellados,
y no pueden acomodar sus acciones al almanaque. En la guerra, ya se
sabe, todo es permitido. Como si se presentara hoy buena coyuntura para
una batalla\ldots{} ¿iban ustedes a dejar de aprovecharla por ser
Pentecostés? No; y en Pentecostés matarían unos y otros gran número de
cristianos. Si admitimos como lógico y razonable el dar a nuestro Padre
Celestial el nombre de \emph{Dios de las Batallas}, que usan los
capellanes en sus sermones y los generales en sus proclamas a la tropa;
si Dios es, como dicen ustedes, capitán general o generalísimo, ya
pueden contar con su indulgencia por aplicar leyes de guerra en días de
solemnidad litúrgica\ldots{} Por mí, no deseo el aplazamiento, pues
aunque me encuentro tranquilo y resignado, no respondo de que en esas
veinticuatro horas se me conserve la resignación y tranquilidad. Somos
hombres, y el morir violentamente, en acto preparado y ceremonioso,
agobia\ldots{} sí señor\ldots{} Mátenme de una vez, y no pongan a prueba
mi fortaleza.»

No se dio por convencido el terco capellán, y perseverando en su idea,
dijo al infeliz prócer: «Quiero dar un nuevo ataque al jefe. En seguida
vuelvo; de paso mandaré que le sirvan a usted un par de huevos
fritos\ldots{} He visto que hay tomate\ldots{} y si usted quiere\ldots{}

---Bien, hijo, bien; lo mismo da\ldots{} Gracias por todo\ldots{} Haga
usted lo que quiera. Yo no tengo voluntad\ldots{} Quiero convencerme de
que ya no vivo.»

En el rato que estuvo solo, el pobre condenado cayó en reflexiones
tristísimas, buscando el por qué de su tragedia; que en tales trances y
en otros menos lastimosos propendemos a escudriñar los orígenes o el
móvil inicial de todo suceso que nos afecta. «Ello es de toda
evidencia---pensaba,---que Dios me envía mi muerte en forma tan terrible
para castigarme de mi enormísimo pecado de estos días. He prestado a
Nelet ayuda insidiosa para la seducción de la monja Marcela; y aunque
desde el primer momento le señalé forma y fines de matrimonio, cosa es
muy grave, y si se quiere sacrílega, el inducir a una esposa de Cristo
al rompimiento de sus votos. Y lo peor es que con malicia instruí al
enamorado y le aconsejé, dándole por norma las inicuas reglas que yo he
ido sacando de la experiencia de mi vida libertina\ldots{} ¡Ah, bien
merecido me está lo que ahora me pasa! ¡En ello veo tu mano, Dios de
justicia!\ldots{} Hice muy mal en tomar a mi cuidado las desazones del
pobre Nelet. ¿Quién me mete a mí a zurcidor de voluntades guerrilleras y
monjiles? ¿Qué voy yo ganando con que una tarasca y un endemoniado se
casen o dejen de casarse? ¡Ah, en el fondo obscuro de mis intenciones
veo la maldita codicia y el afán de allegar recursos! No fue otra la
causa de mi metimiento en tan feo negocio. Y que la monja andariega, por
las reglas infames que di a Nelet, se ha trastornado y siente el veneno
de amor en su sangre, no puede ponerse en duda. Por culpa mía y de mi
sabiduría pérfida, romperá sus votos y ofenderá a Dios\ldots{} Me ha
movido el villano interés, la idea de que, casándose, me habían de
entregar lo que para mí designó Juan Luco\ldots{} Mal pensé, mal hice, y
Dios, en pago de mi perversidad, permite que estos bribones me den
cuatro tiros\ldots{} ¡Ay de mí!»

Interrumpiole Putxet con la noticia de que, oídas las razones canónicas
expuestas por el capellán, que amenazó con poner el caso en conocimiento
del Vicario General, había decretado Llangostera aplazar el acto hasta
el día próximo de madrugada. No supo Urdaneta si la resolución del jefe
le causaba tristeza o alegría. Si fue esto último, era una alegría
triste. Almorzó con mediano apetito, departiendo con el capellán y el
teniente Pulpis, que le custodiaba en la capilla. Por la tarde, su
tristeza se exacerbó en grado sumo, y la compañía de aquellos señores le
causaba enojos. Y pues no le dejaban solo, echose en un camastro como
intentando dormir; mas lo que hacía era sumergirse en la contemplación
de lo pasado, y en traer al pensamiento su familia, su casa de
Cintruénigo\ldots{} «¡Ah! si Rodrigo y Juana Teresa me vieran en esta
horrenda situación, qué amargo llanto derramarían\ldots{} Sí, sí: porque
me quieren, aunque riñamos y nos enemistemos por tonterías que, vistas
desde aquí, son de una insignificancia que mueve a risa y desprecio.
¡Dios mío, qué lección me das al fin de mi vida! Paréceme que estoy ya
en la eternidad, donde presumo que hemos de ver todas las cosas del
mundo en su natural pequeñez. Me quieren, sí, me quieren, y yo también
quiero a mi nieto y a la madre de mi nieto, que es la esposa de mi
hijo\ldots{} Las contrariedades, que en mi necedad estimé graves
ofensas, ahora las perdono de todo corazón. Y cuando ellos sepan ¡ay de
mí! cómo ha concluido D. Beltrán \emph{el Grande}, también me perdonarán
los agravios que les hice, mis malas palabras, mis actos rencorosos.
¡Pues poco que se condolerán de mi suerte! Rezarán por mí, pedirán a
Dios que me acoja en su seno, y harán sufragios por mi alma. Ya estoy
viendo a todo el clero de Cintruénigo atareado por largo espacio de días
en misas, funerales y responsos\ldots{} Confío sobre todo en la eficacia
de mi arrepentimiento. Pésame, Señor, de todo corazón el haberte
ultrajado sistemáticamente, empleando tan mal la vida larguísima que me
has dado. Pésame también el rencor que sentí hacia los míos, y el
regocijo que tuve al ver descompuesta la proyectada boda de mi nieto con
la mayorazga de Castro-Amézaga. Pésanme mis bravatas, mi orgullo, mi
disipación, mi ansia de coger dinero para presumir y disimular mi
ruina\ldots{} Pésame todo el daño que hice, y esta última travesura de
querer arrancar a Marcela de la vida religiosa para satisfacer el
liviano amor de Nelet\ldots» Consagró también tristes pensamientos a su
hija y yerno de Villarcayo, perdonándoles sus últimos desaires; besó
mentalmente a sus nietos, y de todos se despidió con efusión de lágrimas
y suspiros. Sus amigos fueron pasando después por su mente, uno tras
otro, en melancólica y pausada procesión, siendo de los últimos Fernando
Calpena, por quien sentía paternal cariño. Condolíase de que en Bilbao
le hubieran birlado la novia. Si pudiera en aquel instante, ya no se
atrevería, no, a inducirle a solicitar bodas con Demetria\ldots{} No,
no: guarda, Pablo. Demetria debería ser para el Marqués de Sariñán. Que
Doña María Tirgo y Juana Teresa rehicieran los descompuestos planes.
Buscara Calpena otra mayorazga, que buenos partidos no habían de
faltarle\ldots{} Hasta del pobre \emph{Mero} se acordó y de Saloma,
deseándoles vida, salud, felicidades y rápidos ascensos\ldots{} ¿Y qué
sería de Tomé?\ldots{} ¿Y del caballo ganado a Calpena, qué se habría
hecho? En Alcañiz habían quedado también su breve equipaje y el reloj,
magnífica repetición que no llevó consigo al salir en busca de Marcela,
porque roto el espiral a poco de partir de Cintruénigo, para nada le
servía. Guardado con unos pocos duros y pesetas quedó en una bolsa de
vejiga que antes usara para el tabaco\ldots{}

La primera parte de la noche la pasó inquietísimo, hablando sin
fatigarse horas enteras, y ya refería sucesos de su vida, ya dictaba
disposiciones para que Putxet recogiera en Alcañiz su equipaje y
caballo, remitiéndolo todo, con la noticia y relato de su muerte, a la
villa de Cintruénigo. Hizo intención de escribir a su nieto y a su hija;
mas sintiendo muy desvanecida la cabeza y el pulso tembloroso, no trazó
más que unas seis líneas con la declaración de su inocencia y de su
trágico fin. Moría como caballero cristiano, dolorido del mal que había
hecho, y a todos perdonaba, sin excluir a los que inicuamente le
quitaban la vida. Esmerose en la firma, trazándola con todo el vigor y
claridad que le fue posible. Después dijo: «Quisiera que ahora mismo
acabáramos. Las horas que faltan pesan sobre mí como siglos futuros que
se convirtieran en presentes.» Repetida y ampliada la confesión con
piadoso recogimiento, incitole Putxet a dormir. Negose a ello D.
Beltrán, y estuvieron departiendo hasta la madrugada. Viendo cercana la
hora, llamó el reo a los oficiales del piquete para despedirse de ellos.
Formando rueda en torno a la mesa, oyeron esta manifestación tan
sencilla como substanciosa:

«Amigos, les agradezco la simpatía y delicadeza que en esta ocasión me
han manifestado. Son ustedes caballeros; yo también lo soy. Como tal
quiero morir; como tales se conducirán ustedes en el trance final,
acabando mi vida con rapidez y sin martirizarme inútilmente. Yo les
perdono de todo corazón. Y si me es permitido, por el fuero de
ancianidad, dirigirles algunos consejos, allá voy; y esto que ahora les
diga, sea para ustedes de autoridad, como expresión postrera del
pensamiento de un moribundo. Condenado sin culpa, no diré palabra
injuriosa ni vengativa contra el bando político que me arranca la vida,
ni contra vuestro ejército\ldots{} Todas estas cosas quedan para mí en
un término lejano. Sin vituperar esta causa ni la otra, sin enaltecer a
ninguna de las dos, os digo que no derraméis más sangre de españoles.
Guardad esta sangre para mejores y más altas empresas. No defendáis con
tesón tan extraordinario derechos de príncipes o princesas, pues voy
entendiendo yo que tanto valen unos como otros, y que cuando la cuestión
se dilucide y haya un vencedor definitivo, habréis desgarrado a vuestra
patria, que es la legítima poseedora de todos los derechos. Mientras
ponéis en claro, a tiros, cuál es el verídico dueño de la corona, negáis
a la nación su derecho a la vida, porque le estáis matando todos sus
hijos, y le destruís sus ciudades y le arrasáis sus campos. Será muy
triste que cuando de vuestras querellas salgan triunfantes un trono y un
altar, no tengáis suelo firme en que ponerlos. ¿Para qué queréis altar y
trono, si luego han de cojear como esos muebles a que falta una pata?
Allanad y afirmad el suelo ante todo, y esto lo haréis con las artes de
la paz, no con guerras y trapisondas. Haced un país donde haya todo lo
contrario de lo que unos y otros, a quienes no sé si llamar guerreros o
bandidos, representáis; haced un país donde sea verdad la justicia,
donde sea efectiva la propiedad, eficaz el mérito, fecundo el trabajo, y
dejaos de quitar y poner tronos\ldots{} Lo que va a resultar es que,
cualquiera que sea el resultado, estáis fabricando una nación de
bandolerismo, que en mucho tiempo, gane quien ganare, ha de seguir
siendo bandolera, es decir, que tendrá por leyes la violencia, la
injusticia, el favor, la holgazanería, el pillaje y la desvergüenza. En
un pueblo a que dais tal educación, cualquier trono que pongáis será un
trono figurado, de cuatro tablas frágiles y cuatro mal pintados lienzos.

»Quizás vosotros, llenos de vida y de ilusiones, no veáis esto como lo
veo yo, viejo y moribundo. Creéis que toda la vida vais a estar
guerreando, con miras de gloria y ascensos; creéis que España ha de ser
patrimonio y casa de guerreros, los cuales en la paz tendrían que ser
empleados. ¿Empleados de qué? ¿Guerreros para qué? Sois muchos a comer
rancho; sois muchos a vivir de distinciones, de cintajos y signos
categóricos. Y yo os pregunto: ¿quién trabaja? ¿De dónde sale el rancho,
el sueldo, la ropita con galones? Esto es absurdo: estáis matando el
país y haciendo de él un magnífico cementerio poblado por maniquís, que
ostentarán su presunción paseándose entre las sepulturas\ldots{} Y
ahora, puesto que me oís con tanta atención, me permitiré daros consejos
de otro orden. No es tan gran autoridad el virtuoso que nunca ha pecado
como el pecador que reconoce, aunque tarde, sus yerros. Y puesto que
conocéis mi vida, os incito a no imitarme en la parte corrompida de
ella. No seáis pródigos; adoptad con discreta medida las prácticas de
los miserables, llevando cuenta y razón de lo que tenéis y consumís,
para que nunca os salga la necesidad más larga que su remedio, ni la
sábana más corta que la pierna. Entre la sordidez y la excesiva
largueza, preferid lo primero, que os hará antipáticos, pero no
infelices. La generosidad practicada sin medida puede ser viciosa,
porque muchas veces la dicta la presunción antes que el verdadero
espíritu de caridad\ldots{} Y tocando, por fin, el punto más sensible,
no me atrevo a deciros que no seáis enamorados, porque esto sería
contravenir una gran ley de Naturaleza; pero sí os recomiendo que lo
seáis sin apartaros de las leyes eternas, y que evitéis toda empresa de
amor en que veáis probable daño de tercero. Esto es muy malo, hijos
míos, y os lo asegura quien, por seguir la regla contraria, ha tocado en
la experiencia sus perniciosos efectos. En todo caso, sed respetuosos y
veraces con las mujeres. Es más conforme a Naturaleza dejarles a ellas
el uso del engaño, arma con que compensan su debilidad, y tomar el
hombre para sí el uso continuo de la lealtad, que es la fuerza; y los
riesgos que de esto se ocasionen, cada cual los sortee como pueda,
buscando siempre el bien. Que las alabéis y las obsequiéis con flores
del ingenio, no es cosa mala, pues muchas con esto sólo quedan
satisfechas, y vosotros nada perdéis en ello. Los que sean casados,
harán bien en guardar la fidelidad matrimonial, aunque les haya tocado
un culebrón\ldots{} Por eso, conviene mirarlo despacio, y enterarse
antes de contraer esos vínculos que duran toda la vida. Sostened siempre
la paz dentro de la familia que os resulte del nacimiento y de las
uniones, y si hay en ella caracteres ásperos, procurad haceros a sus
asperezas para que los demás contemporicen con las vuestras, que de
seguro las tendréis. Espinas sufrimos, espinas tenemos, y el que crea
que no las tiene y se duela de que le pinchen, es tonto de remate. Y ya
no me queda que deciros sino que seáis trabajadores, que os procuréis un
modo de vivir independiente del Estado, ya en la labranza de tanta
tierra inculta, ya en cualquiera ocupación de artes liberales, oficios o
comercio, pues si así no lo hacéis y os dedicáis todos a \emph{figurar},
no formaréis una nación, sino una plaga, y acabaréis por tener que
devoraros los unos a los otros en guerras y revoluciones sin fin\ldots{}
Sed cultos, bien educados, y emplead las buenas formas así en el
lenguaje como en las acciones, que la grosería es causante de terribles
males privados y públicos. La rudeza y los procederes ordinarios han
sido aquí, bien lo veis, semilla de discordias entre los pueblos, y por
esa falta de formas se hacen interminables las guerras, pues la grosería
engendra el odio, y el odio nos lleva al salvajismo y a la
barbarie\ldots{} Y basta ya: no lloréis por mí, ni tengáis demasiada
lástima de mi muerte, pues soy muy viejo y no sirvo ya para nada. A
nadie soy útil, a nadie hago falta; mis días son de absoluta
esterilidad; ya he vivido bastante, y al quitarme de en medio, casi casi
no cometéis crueldad, pues no hacéis más que arrancar un tronco añoso y
seco, que estorba el nacimiento de nuevos árboles\ldots{} A todos ruego
que me perdonen, y yo en los presentes perdono a cuantas personas de
este y el otro bando hayan podido causarme algún agravio\ldots{}
Entereza no me falta, ya lo veis: confío en la Misericordia divina, a
quien entrego mi alma, abominando de mis culpas sin pedir un galardón
que no merezco, y deseando sólo la indulgencia que Dios no niega al
último pecador. Les ruego, además, que entierren mi cuerpo en lugar
decoroso, designando mi sepultura con una cruz y alguna inscripción,
pues mi familia pretenderá seguramente transportar estos tristes
despojos al panteón de Cintruénigo\ldots{} Por mí, los dejaría en
cualquier parte; pero los Idiáquez no lo consentirán\ldots{} Ea: ya he
concluido, y perdonen que haya sido hablador prolijo en este trance.
Acabemos pronto, y cumplan ustedes su deber, que es matarme, como yo
cumplo el mío muriendo en paz con Dios y con los hombres.»

\hypertarget{xxv}{%
\chapter{XXV}\label{xxv}}

Uno tras otro le fueron abrazando, admirados no sólo de su entereza,
sino de su talento y gracia. Algunos minutos habían pasado ya de la hora
designada para el suplicio, y D. Beltrán, impaciente, dijo con buena
sombra: «¿Pero qué hacemos, señores? Estamos perdiendo un tiempo
precioso\ldots»

El sol entraba por la ventana anunciando un esplendente día primaveral.
Suspiró Urdaneta próximo a la ventana, y dirigiendo miradas de tristeza
hacia el campo verde y risueño, vio en primer término unas cabras; junto
a ellas, un burro viejo, amarrado por las patas. «¡Pobre animal!\ldots{}
le harían ustedes un gran favor sacrificándole conmigo\ldots{} Pero él
no querrá, naturalmente. Aunque viejo y con los dientes gastados, aún le
gusta la hierba\ldots{} ¡glorioso!\ldots{} ¿Con que vamos\ldots{} o
qué?»

Entró Pulpis a decir que el jefe había mandado un recado urgente\ldots{}
¡Que aguardaran\ldots! Sin duda querría despedirse del señor D.
Beltrán\ldots{}

«Pues, hombre---dijo este, suspenso y ansioso,---que venga de una
vez\ldots{} ¿Viene ya?»

Dos minutos de cruel expectación transcurrieron hasta la entrada de
Llangostera en la estancia. Su rostro de clérigo afligido, si algo
expresaba, era la premura y el diligente afán del puntual servicio.
«Siéntese, señor---dijo al reo, sin más saludo.---No tenemos prisa. ¿Qué
tal le han dado de comer?

---¡Comer yo! ¿Para qué?\ldots{} No como nunca tan temprano.

---Que le traigan algo\ldots{} Hay cordero asado, que quedó de anoche.

---Gracias; no tomo nada entre horas.

---Pues ocurre\ldots{} Nada, que tenemos otro aplazamiento. Perdone
usted: bien sé que es molestísimo\ldots{}

---Sí, señor: eso digo\ldots{} De modo que\ldots{} un día más---murmuró
D. Beltrán mirando al campo y al sol.

---¿Un día?\ldots{} ¡Qué sé yo cuántos días serán!\ldots{} Este Ramón ni
descansa, ni deja descansar a nadie. Hace una hora que ha llegado de
Gandesa la partida del \emph{Arcipreste}. Recibo por ella este parte
\emph{(mostrándolo)} en que se me dice, entre varias cosas que no son
del caso, que\ldots{}

---Que me atormenten un poco más.

---No, señor: que antes de fusilarle\ldots{} naturalmente\ldots{} Vamos,
que no le fusilemos, y que hoy mismo se te mande a Gandesa. Quiere
interrogarle sobre cosas que sólo usted puede saber.

---¡Yo!\ldots{} ¡Cosas!\ldots{} ¿Estoy soñando?

---Presumo lo que será\ldots{} No es que él me lo haya dicho. Pero el
que más y el que menos, todos aquí sabemos por dónde va el agua\ldots{}
No se devane el caletre. A Gandesa hoy mismo, dentro de dos horas, con
dos compañías del 3.º y los pocos caballos que aquí tengo. Lo que Ramón
le preguntará es cosa de política\ldots{} de lo que pasa por
allá\ldots{}

---¿En la corte celestial?

---O en otras de más abajo\ldots{} En fin, allá ustedes.

---Pues, señor---dijo D. Beltrán levantándose como un niño entumecido
que quiere correr,---vamos a Gandesa, y hablemos de cortes y cortijos o
de lo que quiera D. Ramón. Yo no sé una palabra\ldots{} o tal vez lo
sepa sin saberlo, sin enterarme de que lo sé\ldots{} Sí, sí\ldots{} algo
podré decirle de grandísimo interés\ldots{} Sr.~de Llangostera, si esto
es una forma de indulto, Dios se lo pague, que alguna parte habrá usted
tenido en ello.

---Yo no; si no viene esta orden, ya estaría usted gozando de Dios. Con
que\ldots{} sea enhorabuena.

---Gracias\ldots{} Viva usted mil años, Sr.~Casa de Val, \emph{alias}
Llangostera\ldots{} Y acordándome ahora de su gallardo ofrecimiento, que
me traigan el cordero asado. Se me despierta un apetito horroroso.

---Pues que aproveche\ldots{} No descuidarse: a las ocho, en marcha.»

Apenas traspasó la puerta el cabecilla, arrancose Putxet a dar a su
amigo un abrazo tan fuerte, que a poco más le ahoga. «A mí, a mí me debe
usted su salvación, nobilísimo señor, pues sin la tremenda batalla que
ayer di, por ser Pentecostés, la orden de Don Ramón le habría alcanzado
a usted en la sepultura\ldots{} Y lo hice, puede creérmelo, más que por
ser Pentecostés, ¡pacho!, porque me dio la corazonada de que ganando un
día, salvábamos al hombre. Acerté\ldots{} Ya sabía yo que anda Cabrera
muy caviloso estos días con chismes que le han traído del Cuartel
Real\ldots{}

---¡Pero si yo estoy tan enterado de las cosas del Cuartel Real como de
lo que pasa en la luna!

---Quia\ldots{} eso no puede ser\ldots{} Por algo se fija D. Ramón en
usted, y espera que le aclare lo que ignora\ldots{}

---Juro que\ldots{}

---Y en todo caso, si usted no lo sabe, invéntelo,
\emph{¡pacho!\ldots{}} Para mí, ya está usted indultado, y puede que muy
pronto libre\ldots{}

---Sea lo que Dios quiera, amigo Putxet. He visto la muerte tan de
cerca, que no podré desechar la idea de que vivo de milagro. Cúmplase la
voluntad de Dios. Pronto estoy a todo, a vivir y a morir.»

A la hora designada salió de Rossell el gran aristócrata con las tropas
que marchaban a Gandesa, y todo le fue lisonjero aquel día: se le
facilitó un buen caballo, y para colmo de felicidad iban con él Putxet,
capellán del 3.º, y el teniente Pulpis, que en el corto tiempo de
conocimiento mostraba hacia el aragonés gran simpatía y cordialidad. Por
montes y laderas departían los tres de diversas cosas humanas y divinas,
hallándose D. Beltrán tan inspirado aquel día y con su inteligencia tan
despierta, que los otros no se hartaban de oírle. Refirió sucesos
interesantísimos de su vida y de la vida general, o sea Historia, con
sin igual donaire y expresión justa, ingeniosa, contestando sin fatiga a
cuanto le preguntaban. Y entre párrafo y párrafo introducía, a guisa de
estribillo, ponderaciones de los espectáculos de la Naturaleza que
contemplaba. Todo le parecía bello, aun lo que no lo era. «¿Y no saben
ustedes una cosa, amigos míos? Pues estoy asombrado de ver\ldots{} que
veo mejor que antes\ldots{} No sé a qué atribuirlo. Pero no hay duda: se
me aclara considerable mente la vista. No sé si será porque\ldots{}
\emph{¡pacho!} como estuve casi dentro del reino de la muerte, mis ojos
se preparaban para ver lo que aquí tenemos por invisible, y se
afinaron\ldots{} aprendieron algo nuevo en el arte de la visión\ldots{}
no sé\ldots»

Todo el día y parte de la noche emplearon en el paso de los puertos de
Beceite, pernoctando en la bajada de Monte Caro. Al amanecer se les
agregaron varias partidas, y avanzando cautelosos con buenos guías y
precavidos de espionaje, evitaron el encuentro con las fuerzas cristinas
que operaban en aquella zona. Al caer de la tarde supieron que D. Ramón,
atacado por Nogueras ante los muros de Gandesa, había tenido que
levantar el sitio de esta plaza retirándose a Bot. A este punto se
dirigieron a marchas forzadas, y a media noche encontraron a sus
compañeros, acampados al raso, en árida y polvorosa colina junto al río
Seco. La temperatura era ardiente; la tierra, caldeada por el sol,
apenas se refrescaba en la segunda mitad de la noche. Escaseaba el agua,
y los soldados abrían pozos buscando con qué aplacar su sed. En una mala
tienda hallábase Cabrera, desvelado, inquieto, en un grado de biliosa
displicencia que hacía temblar a cuantos para asuntos del servicio se le
acercaban. No bien se enteró de que habían llegado las fuerzas pedidas a
Rossell, mandó llamar al viejo Urdaneta, sin darle punto de reposo: tal
era su avidez de interrogarle. Muerto de cansancio y de sueño, llegó a
la tienda el buen aragonés, y con el saludo pidió al \emph{leopardo} que
le permitiese echarse en el suelo, pues ya no podía tenerse en pie:
antes de obtener la venia, se desplomó. Dos sillas de tijera había en la
tienda: en una se sentaba el General, envuelto en su capa blanca, pues
tenía frío a pesar del tiempo bochornoso; en la otra, convertida en
mesa, había papeles, un tintero de cuerno y un farol. El secretario se
sentaba en el suelo en postura turquesca.

«Póngase usted a su comodidad---dijo Cabrera al prócer.---Aquí no
guardamos etiquetas\ldots{} Yo voy a hacer lo mismo, pues el dolor de
riñones no me deja estar sentado.» Hizo una seña al secretario para que
se largara, y se tendió frente a D. Beltrán, apoyando la cabeza en un
rollo de mantas. No era hombre que se resignaba a perder el tiempo: los
minutos eran para él preciosos, y aborrecía las vanas palabras. Sin
preguntar al prisionero cosa alguna referente a su viaje ni a su
interrumpido suplicio en Rossell, abordó el asunto, que sin duda le
inquietaba hondamente.

«Con que\ldots{} va usted a responderme con claridad, con precisión, y
sobre todo con verdad, a lo que le pregunte, Sr.~de Urdaneta. No piense
usted en engañarme, porque a Ramón Ca\ldots brera nadie le ha engañado
todavía, ni guarde reserva sobre punto alguno de mi
interrogación\ldots{} porque se arrepentirá de ello. Lo que me oculte,
yo he de saberlo después\ldots{} y le pediré cuenta de su silencio; lo
que me diga con falsedad, lo descubriré al oírlo, porque Dios me ha dado
el don de distinguir lo falso de lo verdadero en lo que me dicen\ldots{}
Y si algo de lo que me manifieste es de carácter delicado, quedará entre
los dos; yo sé callar como nadie\ldots{} pero como nadie sé oír y
aprender.

---Sepa yo pronto de qué se trata, General---replicó D. Beltrán,---que,
por Dios, ni aun sospecho cuál puede ser el asunto de mi conocimiento
que a usted interese.

---Ahora lo veremos. Prepárese a responder con cla\ldots{} ridad, y
sobre todo con exactitud. En Febrero de este año pasó usted por Fuentes
de Ebro, camino hacia Caspe y Alcañiz. En el parador de Viscarrués comió
usted y habló largamente con un sujeto italiano, su amigo, llamado
Rapella, que iba en seguimiento de Borso, y venía del Cuartel Real del
Norte.»

Después de asentir con la cabeza a los primeros conceptos del
\emph{leopardo}, manifestole D. Beltrán con acento sincero que, en
efecto, había hablado con Rapella; pero que no era amigo suyo, y en
Fuentes de Ebro le vio y trató por primera vez. Por cierto que, movido
de la curiosidad y sin ningún interés positivo en ello, había intentado
tirarle de la lengua, para sorprender la clave de sus continuas viajatas
diplomáticas entre Cortes borbónicas; mas nada pudo obtener, como no
fuera la certidumbre de la cerrada discreción del siciliano.

Mostrose Cabrera incrédulo de esta declaración, y en tono agrio le dijo:
«Veo que es usted de la misma escuela. No me sirven los diplomáticos, y
usted tampoco quiere servirme\ldots{}

---He dicho a usted, mi General, que ni una palabra pude sacarle\ldots{}
Pero no he dicho que ignore los líos que se trae ese señor\ldots{}

---Pues si lo sabe\ldots{}

---Es que usted, General, debió empezar por decirme: `Urdaneta, ¿qué
sabe usted de esto?', y no interrogarme al modo capcioso, como se hace
con los espías enemigos.

---Tiene usted razón---dijo Cabrera, rindiéndose a la noble actitud del
aragonés.---Perdóneme; no supe distinguir. ¡La costumbre de tratar con
canallas\ldots! Es usted un caballero, y lo que sepa acerca de este
asunto, me lo dirá\ldots{} como de amigo a amigo.

---A ello voy. No sirvo a ninguna causa; no vendo ningún secreto;
referiré lo que sepa, para mí falto de interés, para usted quizás
no\ldots»

Minucioso y elegante narrador, maestro en el arte de dar interés al
relato más sencillo, D. Beltrán expuso gallardamente lo que sabía y
opinaba; que no todo fue relación de hechos, pues hubo también un
disertar gracioso sobre cosas políticas hondas, de las que rara vez
salen a la superficie. Habiendo trabado amistad, en su viaje desde La
Guardia a Villarcayo, con un joven madrileño muy simpático que el verano
anterior había visitado la Corte de Oñate en compañía de Rapella, pudo
conocer el carácter de este, sin más datos que las referencias de aquel
joven. Era el siciliano muy astuto, corrido en intrigas de mujeres y en
diplomacia menuda de gabinetes secretos, de combinaciones políticas a
hurtadillas de los ministros o cancilleres. Pintó Urdaneta la Corte de
D. Carlos, repitiendo lo que le había contado su amigo, y por cierto que
no escatimó las tintas burlescas en la pintura, sin que por ello se
escandalizara el que le oía. Diole noticias de la amistad del siciliano
con el Infante D. Sebastián, con quien al parecer no se había entendido
en las negociaciones o enredos que llevaba. Lo que resultó de las
conferencias del tal embajador con D. Carlos en Durango, su amigo no lo
sabía, pues un accidente inesperado le separó de él el día mismo de la
evacuación de Oñate a consecuencia de la toma de Arlabán. Presumía que
la base del proyectado convenio para poner fin a la guerra era la
reconciliación de las dos ramas borbónicas por medio de un casamiento;
mas como este no había de efectuarse hasta que la Reina Isabel y el hijo
de D. Carlos llegasen a edad de matrimonio, tal proyecto era un sueño; y
para celebrar la paz y que se abrazaran los dos ejércitos, se buscaban
otras fórmulas de transacción y avenencia.

Levantose Cabrera de un salto, nervioso y colérico, exclamando: «Yo no
me abrazo con nadie\ldots{} ¡Abrazos a mí!\ldots{} ¡Transacción!\ldots{}
Juro que no\ldots{} No saben quién es Cabrera\ldots{} Ni por un puñado
de oro, ni por grados y ventajas en la carrera, me cubro yo de
vilipendio entregándome a los cristinos. Si Don Carlos cede, allá se las
haya\ldots{} Él en su casa y yo en la mía\ldots{} ¡No quiero, no
quiero!\ldots{} ¡Matrimonios de príncipes!\ldots{} ¿Se casa la luz con
las tinieblas?\ldots{} ¿Se casa la justicia con la injusticia, la razón
con la sinrazón? Pues si se casan, con su pan se lo coman. Yo no me caso
con nadie. Ramón Cabrera no se casa.»

\hypertarget{xxvi}{%
\chapter{XXVI}\label{xxvi}}

Volviendo a ocupar su silla, acarició con movimiento maquinal los
papeles que en la otra tenía, alumbrados por el mustio farol. D.
Beltrán, sin cambiar de postura, flemático y perezoso, siguió
manifestando al caudillo apreciaciones que creía interesantes. Por lo
que había oído en Medina y Villarcayo, por algo que pudo descubrir
conversando con su grande amigo D. Baldomero Espartero, los tratos para
buscar fórmula de paz no habían cesado desde el principio de la guerra.
Proposiciones se hicieron a Zumalacárregui, proposiciones a Maroto, y el
mismo Cabrera no habría estado libre de que en su oído se murmuraran
palabras tentadoras\ldots{}

«A mí no, a mí no---dijo prontamente el \emph{leopardo}.---Ya saben que
mandaría fusilar al que me trajera recaditos de Doña Cristina o del Rey
napolitano.

---Del Rey de Nápoles, a quien entiendo yo que no debemos la invención
de la pólvora, es agente oficioso el tal Rapella. Anda también en estos
tratos y trotes un legitimista francés, Marqués de no sé cuántos.

---No es Marqués, sino Barón\ldots{} y ha entrado en España con el
supuesto apellido de Neuillet. Me da en la nariz que el nombre de
Rapella es también falso, y que bajo él se esconde un correveidile de
Cristina, maestro en intrigas, que en Madrid era conocido por Marqués de
Lagrua.»

Insistió Urdaneta en que no podía dar ninguna luz sobre esto, pues no se
había echado a la cara al tal D. Aníbal hasta su paso por Fuentes de
Ebro, y de él no tenía más noticias que las anteriormente comunicadas.
Asimismo ignoraba si el siciliano se había visto con Borso; pero Cabrera
te sacó de dudas, afirmando que tres días permaneció aquel en Castellón
en compañía del General y de un italiano llamado Cialdini, embarcándose
después para Marsella.

«Le tengo a usted por un caballero---añadió D. Ramón con cierta
solemnidad, después de larga meditación,---y estoy con\ldots{} vencido
de que me ha dicho todo lo que sabe. Sus opiniones parécenme muy bien
fundadas.»

Algo más dijo el leopardo; pero D. Beltrán, que ya venía dando fuertes
cabezadas, hundió al fin la barba en el pecho, y cogió un sueño
profundo, que por causa de la mala postura había de ser breve.

«Sí, sí: duérmase usted, amigo mío---murmuró el General con
lástima,---que bien necesitado está de descanso. Le envidio su facilidad
para el sueño.»

Y cogió de la mesa-silla, ávido de nueva lectura, la carta que desde
Sangüesa le había escrito Arias Teijeiro. De aquellas apretadas líneas
de menuda letra española, provenían sus inquietudes y desvelos.
Informábale con prolijas referencias su amigo, principal figura en la
camarilla del Pretendiente, de que la magna expedición al mando del
propio Rey había partido de Navarra el 17 con diez y seis batallones,
nueve escuadrones, el estandarte de la Generalísima y su lucida escolta,
y un inmenso bagaje, como correspondía al sinnúmero de funcionarios de
Corte y Administración que acompañar debían a la Real persona.

«Le tengo por un gran farsante---dijo Don Beltrán despertando
súbitamente.---¡Ah\ldots{} mi General! ¿No me pregunta usted su opinión
sobre ese Rapella? Opino que el ir a Marsella es para ganar más
fácilmente la frontera de Navarra y agregarse al llamado Cuartel Real.

---Así es, en efecto. Viene en la expedición magna.

---Pero ¿qué es eso? ¿Se lanza D. Carlos a una correría como las de
Gómez, Batanero y D. Basilio?

---No sé\ldots{} Eso se dice\ldots{} Allá veremos.»

Siguió pensando el \emph{leopardo} en lo que la carta decía y comentando
con interno juicio las noticias de ella. Traducida con la posible
fidelidad de expresión muda de su pensamiento en valenciano, resulta
\emph{mutatis mutandis}: \emph{«¡Pacho,} con la impedimenta que nos
traen! La caterva de empleaduchos, la taifa de gente allegadiza que
quiere comer a costa nuestra! ¡Vaya una plaga, \emph{pacho!} Aquí nos
vemos y nos deseamos para poder vivir\ldots{} El país esquilmado\ldots{}
apenas hay raciones para mal comer\ldots{} y ahora nos viene encima esa
nube. Tenemos un Rey que sabe tanto de guerra como yo de afeitar ranas.
¿Por qué no se estará quietecito en su Corte esperando a que le hagamos
Rey de todas las Españas?\ldots{} ¡Y que se trae unos consejeros y unos
ministros que no tienen precio para ayudar a misa, para pegar botones o
cepillar la ropa! Vendrán de generales el tontaina de Don Sebastián, el
buey cansino de González Moreno y el bribón de Gómez, a quien yo pondría
de capataz de un presidio, que es lo único para que sirve\ldots{}
Duérmase de una vez, D. Beltrán, que aquí no gastamos etiquetas. Me da
pena verle luchar con el sueño.

---Es que\ldots{} verá usted\ldots{} decía yo que indudablemente hay
tratos y contubernios entre \emph{Palacio} y ese\ldots{} ¿cómo le
llaman? Ya no me acuerdo\ldots{} El Rey, hombre\ldots{} Felipe V\ldots{}
digo, Carlos\ldots{} La Reina, que no perdona lo de la Granja, parece
que no quiere nada con liberales\ldots{} Luis Felipe desea que se acabe
la guerra de cualquier modo, por creerla un peligro\ldots{} y la
cuádruple alianza\ldots{} sí señor, la cuádruple\ldots{}

---A dormir\ldots{} Tenga usted este lío de mantas para que descanse la
cabeza.

---Muchas gracias, querido Nelet\ldots{} digo, no, Sr.~D. Ramón
V\ldots{} El sueño me rinde, me trastorna\ldots{} Gracias.»

Sin poder apartar de su mente las ideas que le atormentaban, Cabrera se
paseó en el estrecho espacio de la tienda, embozado en su capa blanca.
No se conformaba con que el Ejército Real, mal organizado y pésimamente
dirigido, viniese a compartir con él el dominio en la región valenciana.
Recordaba sus desavenencias con Gómez, por cuál mandaba más. Cierto que
al Rey no podía disputársele la supremacía. Aunque incapaz para la
guerra y para el Gobierno, era el Rey, por divino mandato, la sacra
bandera, el símbolo de la Causa; y de la regia persona, absolutamente
inepta para todo, provenía la fuerza moral de las cohortes del
absolutismo. No había, pues, más remedio que cargar con el ídolo, aunque
este fuera una de las obras más burdas del fetichismo dominante. ¡Y por
semejante figurón, hecho al modo de las imágenes vestidas, que por
dentro no son más que una armazón de madera tosca, se peleaban tantos
hombres valientes, y se vertían ríos de noble sangre!\ldots{} Claro que
todo se hacía por \emph{la idea}. El grosero ídolo era una idea. Por
ella combatían fieramente los de acá, mientras los defensores de la idea
contraria cifraban su valor en la adoración de una linda muñeca\ldots{}
En suma: lo que ponía en grande irritación al caudillo del Maestrazgo
era que se había de convertir en auxiliar y mequetrefe del Ejército Real
en cuanto este pasase el Ebro. Las operaciones ya no serían suyas:
tendría que subordinarlas a lo que dispusiese cualquiera de los
reverendos sacristanes que venían agregados al santón del
absolutismo\ldots{}

Verdad que la carta de Arias Teijeiro no escatimaba las lisonjas al
héroe del Maestrazgo. En el Cuartel Real se le tenía por un estratégico
de primer orden, firme columna de la Causa, y el Soberano deseaba
ocasión de mostrarle personalmente su Real aprecio. Pero tras estos
inciensos venían anuncios de resoluciones que desagradaban al
\emph{leopardo}. La expedición Real, a la que se uniría Cabrera para
engrosarla y fortalecerla, llegaría con la ayuda de Dios hasta el propio
Madrid, y entraría en la capital de la Monarquía \emph{sin disparar un
tiro.}

Esto de rematar la campaña \emph{sin combatir} sacaba de quicio al
ardiente Cabrera. Todo lo que no fuese ganar a sangre y fuego el triunfo
de la Causa, pugnaba con su temperamento batallador, con su corazón
fiero y ¿por qué no decirlo? noble. Los arreglos por concesiones
recíprocas de mercedes, o por casorios y pactos de familia, le olían a
podredumbre. Tan viles eran los unos como los otros si a ello se
prestaban. Uno de los dos rivales debía perecer: eso de que vivieran y
triunfaran los dos, partiéndose la torta disputada, no se acomodaba a su
lógica ruda, ni a su primitivo y elemental criterio de cosas políticas.
¡Entrar en Madrid unos y otros con \emph{sus manos lavadas!} ¡Ah,
\emph{pacho}, y reconocer a Doña Cristina y a D. Carlos como \emph{Reyes
Padres}, los dos en igual categoría dinástica\ldots{} y ver a los nenes
asistidos de un consejo mixto, y apoyados por un ejército mixto o
mestizo!\ldots{} Y en tanto, ¿que se haría de las ideas? Pues juntarlas
todas en una redoma para sacar otra mezcla indecente, que no serviría
para nada. ¡Libertad y absolutismo desleídos en agua, \emph{según arte!}
¡Rey y pueblo abrazaditos\ldots! ¡Religión y ateísmo en una pieza,
\emph{pacho!}

Colmaba la indignación del General esta frasecilla de la carta: «Aún no
puedo ser muy explícito, mi querido D. Ramón. Sólo me permitiré
anticiparle que las bases de un arreglo decoroso están sentadas por
manos muy peritas, y que no veo lejano el día glorioso en que podamos
descansar de nuestra ruda campaña, viendo triunfante lo más esencial de
nuestra doctrina.» ¡Descansar! ¡Si él no quería más descanso que
reventar combatiendo!\ldots{} La gentuza civil, la patulea de holgazanes
y vividores que acudían a la Causa como las moscas al panal, era la que
anhelaba el descanso de la paz, para chupar a sus anchas, repartiéndose
el momio de los destinos. Ese descanso de lo civil era el militar
vilipendio, y él no\ldots{} él no quería descanso sin honra, sino honra
con cansancio.

A esto llegaba, cuando despertó el noble caballero sobresaltado, con
ahogos de pesadilla. Soñó que le sacaban al cuadro para fusilarle, que
le ponían de rodillas y le vendaban los ojos\ldots{} «¡Al corazón, hijos
míos, al corazón! No me hagáis padecer,» murmuraba sin abrir los ojos; y
cuando los abrió, reconociéndose despierto, pidió perdón al General: «No
me haga usted caso. Estoy fatigadísimo, y si aquí molesto, me saldré a
dormir en campo raso.

---No, no; quédese aquí. Le diré, para su tranquilidad, que ya está
libre de la sentencia de rehenes. Aunque allá fusilen media
aristocracia, la vida de usted en mi poder no corre peligro. Rehenes por
gente civil, no me convienen.

---No sé con qué palabras expresar a usted mi agradecimiento por su
magnanimidad---dijo Urdaneta conmovido.---¿De modo que estoy
libre\ldots?

---Libre no. Aún será usted mi prisionero por una temporada. Puede que
te necesite, por su gran conocimiento de cortesanías y politiquerías de
Madrid\ldots{} y de toda la morralla civil. Tenga un poco de paciencia,
y por de pronto duerma en mi tienda todo lo que el cuerpo le pida.

---Me pide mucho, General\ldots{} Traigo un atraso horroroso en el
dormir. Lo menos me debe a mí el sueño cuatro noches. Figúrese\ldots{} a
mi edad.»

Ayudado de aquel sosiego que las últimas palabras de Cabrera dieron a su
espíritu, cogió D. Beltrán el sueño, quedándose en él con profunda
quietud hasta muy avanzado el día; pero cuando ya su cuerpo hubo
recibido la reparación de que estaba tan necesitado, el cerebro se
soliviantó, dándose a los sueños extravagantes. Después de mil visiones
vagas, indefinibles, viose atormentado por seres malignos y traviesos
que le traían y llevaban sin ningún respeto a su nobleza y ancianidad.
Eran, sin duda, los familiares demonios de Nelet, que por contagio de la
amistad, pasado se habían del joven al viejo, del creyente al incrédulo.
En medio de la turbación del soñar, su razón siempre vigilante le decía:
«De esto tiene la culpa Santapau, por contarte sus diabólicas aventuras
con tantos pelos y señales.» Ello es que la infernal cuadrilla cogió por
su cuenta al señor de Albalate, y de un vuelo me le transportó a
Cintruénigo, donde vio a Doña Juana Teresa echando trigo, y a Rodriguito
con la pluma tras la oreja, contando los garbanzos que se habían de
echar al puchero. Visto esto, volvieron los diablillos a cogerle por los
sobacos o por el cogote (no estaba bien seguro), y le llevaron a la cima
del Moncayo; de allí a Veruela, y metiéndole por un subterráneo, le
arrastraron hasta salir al castillo de Loarre en tierra de Huesca.
Entretuviéronse en jugar con él a la pelota, lanzándole de un torreón a
otro, y después te llevaron, cogido por las orejas, a la sierra de
Guara, desde cuyas cumbres le mostraron todo el territorio del antiguo
reino de Sobrarbe, diciéndole\ldots{} Pero de lo que decían no pudo
enterarse bien. Despertó con el cuello dolorido, y, viendo la necedad de
su ilusión, requirió nuevamente el sueño, tomando mejor postura.

No debía despertar el noble señor sin que su turbado cerebro se lanzara
a mayores travesuras, sucediendo a las imágenes de un orden bufonesco
otras de carácter lúgubre y penoso. Tan claramente como se ven cosas y
personas en la realidad, vio a Nelet, que, asistido de unos cuantos
facciosos con rabo (por donde se colegía su calidad demoníaca),
crucificaba a un hombre, clavándole en un largo madero. El hombre, que
debía de ser un bendito, se dejaba crucificar risueño, diciendo a su
verdugo: \emph{«¡Pacho!,} no sabes lo que haces.» Largo tiempo, si es
que la lentitud o rapidez de este son apreciables en una pesadilla,
atormentó al soñador la visión espantosa, que terminaba y se reproducía
como el ensayo de una escena teatral. El propio D. Beltrán, angustiado,
quiso más de una vez gritar a su discípulo: \emph{«¡Pacho!}, no sabes lo
que haces.» Pero no podía\ldots{} ¡Vive Dios, que no podía!\ldots{} Las
palabras se le pegaban al cielo de la boca cual si fueran obleas.

\hypertarget{xxvii}{%
\chapter{XXVII}\label{xxvii}}

Horrorizado y tembloroso despertó el anciano, y lo primero que vio fue a
Cabrera durmiendo, tendido en el suelo boca arriba sobre una manta,
envuelto en su capa blanca y roja, la boina sobre los ojos para
resguardarlos de la luz. El secretario, con violenta postura, escribía
en la silla de tijera, y un ayudante que hacía cigarrillos sentado en la
tierra, indicó a D. Beltrán con un signo que evitase el ruido para no
turbar el descanso del General, que se había dormido después de salir el
sol. A poco entró un ordenanza, y en voz muy baja dijo al prócer que
fuera le esperaba desde el amanecer un señor comandante amigo suyo.
Echose de la tienda D. Beltrán, andando poco menos que a gatas por la
gran debilidad que sentía, y encontrose a Nelet sentadito en una piedra,
la cabeza entre las manos, el espinazo en violenta curva, imagen de la
melancolía negra o de la desesperación. Después de tocarle en el hombro,
el desmayado viejo encaminose a una cercana tienda, de donde un
penetrante olor de fritangas le llamaba con reclamo irresistible. Tuvo
la suerte de tropezarse allí con el teniente Pulpis, que inspeccionaba
las sartenes; pidió que le dieran de comer, aunque sólo fuera pan y
cebolla, y obtenido algo más confortativo y suculento, se puso a
devorarlo mientras hablaba con Santapau, que se le arrimó al instante
con apetito de conversación.

«Hijo mío, te encuentro muy desmedrado. ¿Estás herido? ¿Has perdido tu
preciosa sangre en las acciones de estos días frente a los muros de
Gandesa?\ldots{} ¿O es que te sobrevino algún disgusto, quizás otra
jarana con los \emph{chicos de Lucifer?}

---No\ldots{} a esos no les temo ya. Curado estoy del mal de
demonios---replicó Nelet suspirando, agobiado de tristeza.---Un
saludador de mi pueblo me ha dejado las cámaras interiores bien limpias
de esas alimañas, con un bebedizo que, por lo amargo, debe de estar
hecho con la hiel de Judas. Al decir de ese médico, los diablos huyen
ahora de mí y se albergan en los cuerpos de mis amigos.

---Cierto debe de ser eso---dijo Urdaneta haciendo por la vida con ansia
fisiológica,---porque anoche se han dignado visitarme esos mequetrefes,
y en ellos reconocí a los que contigo se divertían. Pues que ya
desalojaron tu interior, haz que abandonen también el de tu maestro, que
no gusto de tales inquilinos\ldots{} Entiendo, por la murria que noto en
ti, que el desahucio no ha sido completo, y que algún intruso se quedó
trasconejado dentro de tu pobre humanidad.

---No es murria de diablura la que tengo, sino de conciencia, y tan
grave y honda, que anoche faltó poco para que pusiera fin a mi vida.
Suspendí el dispararme por esperar a consulta con usted acerca del caso
que me anonada, caso tremendo de los que no tienen solución.

---¿Qué sabes tú si yo la encontraré? Déjame que coma un poco más de
este guisado de cabra que me da la vida, y me fortalece el magín para
evacuar consultas\ldots{} Come algo, hijo, que del alimento corpóreo se
nutre también y conforta lo más espiritual de nuestro ser: la
conciencia.

---Las hambres de la conciencia no se aplacan sino echándole la propia
carne para que se la coma\ldots{}

---Cuéntame, cuéntame pronto, y veré la causa de tu aflicción.

---Acabe usted y salgamos de aquí. Vámonos a donde no haya personas que
vean y oigan. El oído y el ver humanos me dan tanto enojo, que a todo el
mundo dejaría ciego y mudo. Sólo Dios debe ver, y sólo deben sonar las
tempestades, que son su voz.

---Hijo, poético estás y lúgubremente metafórico\ldots{} sólo que tus
imágenes son de un cuño que está ya mandado recoger por anticuado y
candoroso. Ea, terminé mi almuerzo, que por el hambre que tenía me ha
resultado opíparo. Vamos a donde quieras.

Llevole Nelet a un ejido donde estaban herrando caballos, y allí, entre
relinchos, aún mejor sonantes que las palabrotas de mariscales y
soldados, refirió el caso que tan hondamente le perturbaba. «La
malhadada acción de Gandesa---dijo,---la perdimos porque, en lo mejor
del combate, muchos de nuestros hombres fueron atacados repentinamente
de un mal de estómago, por haber bebido en charcos corruptos, y con
fieros retortijones caían muertos. Mi regimiento fue de los que más
sufrieron de este maleficio. Creían mis soldados que el enemigo había
envenenado las aguas\ldots{} les entró el pánico\ldots{} entre el físico
y yo quisimos convencerles de que la ponzoña era natural en aquellas
estancadas lagunas\ldots{} Para abreviar: enfermos y desalentados nos
batimos en guerrillas en todo el flanco derecho. Nogueras embistió el
centro. Vi que flaqueaban; apretamos más y más, perdiendo gente y
ganando terreno; hice lo que pude, más de lo que podíamos y debíamos,
hasta que Cabrera nos mandó retirar. Hícelo yo con un orden perfecto,
pues conozco como los dedos de mis manos todos los caminos, atajos y
veredas que rodean al pueblo donde nací. Ninguna fuerza cristina me
atacó en mi retirada, que hice vadeando el río y tomando la vuelta de
Algás. No habíamos andado legua y media, cuando sorprendimos y copamos
unos veinte hombres cristinos que al parecer habían salido de
descubierta. Tan torpes andaban y tan ignorantes del terreno, que se nos
vinieron a la mano en sitio donde no podían escapar. Algunos, arrojando
las armas, emprendieron la fuga con pies ligeros; pero mis tiradores no
tardaron en cazarles: sólo dos piezas perdimos. Los otros se nos
entregaron como borregos atontados, pidiéndonos misericordia. «¿Qué
hacemos, mi comandante? ¿Les fusilamos, o qué? Nos da el corazón que
estos andaban por aquí envenenando todo el río\ldots» Respondí que
bueno\ldots{} Yo me sentía un poco emponzoñado\ldots{} estaba
furioso\ldots{} echaba fuego de todo mi cuerpo\ldots{} Por ahorrar
cartuchos, mi gente les iba despachando a bayonetazos\ldots{} Yo no sé,
amigo D. Beltrán, por qué me entró aquel día tal furor de matanza.
Demonios no llevaba dentro de mí; pero sí un amargor que me irritaba,
que me volvía feroz. Por la mañana había tomado el brebaje de que antes
hablé\ldots{} me escocía horriblemente el cuerpo. Las moscas que se
cebaban en mi pobre caballo, me tenían loco con sus furiosas picaduras.
Y además, yo sudaba\ldots{} ¿cómo diré? a mares, un sudor amargo y
venenoso, según creo, y mosca que me picaba, moría. Mas eran tantas, que
hube de apearme por huir de ellas\ldots{} Mientras mis soldados
exterminaban hombres, yo daba vueltas a pie por entre vivos, muertos y a
medio morir; y en esto vi a un cristino tumbado contra un árbol, herido
ya\ldots{} No sé por qué me dio el arrechucho de atravesarle con mi
espada\ldots{} le tomé por una mosca, o por el padre de todas las
moscas\ldots{} Apenas retiraba de su costado izquierdo mi espada, me
asaltó una idea\ldots{} sí, era una idea. ¿Qué vi yo en la cara y en los
ojos de aquel hombre? ¿Qué vi para lanzar un alarido, pues alarido de
rabia y dolor fue la pregunta que le hice? «¿Eres tú Francisco Luco?» Lo
pregunté dos veces, y él respondió que sí con la cabeza, moviéndola de
golpe\ldots{} así\ldots{} Con la cabeza dijo que sí, y también con los
ojos al mirarme; mas con la boca no dijo nada, porque entre el intento y
la palabra se metió la muerte.

---¡Dios nos tenga de su mano!---exclamó Urdaneta, desahogando su pena
con un gran suspiro.

---Dígame usted ahora si habiendo dado muerte con tan estúpida crueldad
al hermano de la que adoro, puede haber consuelo para mí. ¿No debo
desear que se abra la tierra y me trague? ¿Para qué está ya Manuel
Santapau en el mundo?

---Poco a poco\ldots{} no hay que perder la serenidad. Primero, pudo
haber error. Al dar el hombre esa fuerte cabezada, como dices, quizás no
fue su ánimo responder a tu pregunta\ldots{} Aquel movimiento debió de
ser la tensión de músculos propia del morir\ldots{}

---¿Y la semejanza con su hermana? ¡Si era su propio rostro! Los ojos,
en la mirada que me echó, pareciéronme los ojos de Marcela.

---Tampoco eso prueba nada. O pudo ser un parecido casual, o no había
tal semejanza más que en tu imaginación excitada por el combate, por las
preocupaciones, por el brebaje, y\ldots{} por las moscas. ¡Y quién sabe,
quién sabe, querido Nelet, si en esa tragedia habrán tenido alguna parte
\emph{los chicos de Luzbel}, valiéndose de un cubileteo, de una
simulación de rostros para trastornarte! Aquí donde me ves, influido sin
duda por el ambiente que respiro, por el aspecto romántico del país, voy
creyendo en la realidad de las travesuras diabólicas, de que antes me
reía\ldots{} Y ¡qué diantre! atenúa mucho tu responsabilidad el haber
sido cosa repentina, imprevista, como accidente de una batalla\ldots{}
La ocasión, la ley de represalias, que no puedes eludir como subordinado
de Cabrera, te disculpa en cierto modo\ldots{}

---No, no: mi conciencia no lo cree así\ldots{} Mi conciencia se ha
vuelto muy rígida, muy exigente y escrupulosa\ldots{} Natural es que el
amigo y maestro quiera consolarme\ldots{} Pero no hay consuelo para mí.
He cometido un verdadero parricidio. El querer matarme ahora, ¿qué es,
señor mío, más que el afán de huir de mí, por el horror que me causo?

---Calma, juicio, reflexión\ldots---dijo el maestro desalentado, mas
queriendo disimular su pesadumbre.---Repentino y fulminante parece tu
mal de conciencia; pero no faltará remedio para él: yo te lo fío, yo te
lo aseguro\ldots{} Has de prometerme no tomar ninguna resolución airada,
y oírme y consultarme en todo, que si experto soy en amores, no me
faltan luces ni conocimientos para los casos más graves de conciencia
turbada. Déjalo a mi cargo. Descansa en mi autoridad, triste ciencia de
los años\ldots»

Como a continuación expresara el ladino viejo la idea de que bien podía
Marcela ignorar siempre quién había sido el matador de su hermano, se
remontó Nelet de la tristeza lúgubre a la ira, diciendo: «¿Cree usted
que con esta cara puedo yo presentarme a ella y guardar el secreto de mi
crimen? En el estado de mi conciencia, es imposible el disimulo, porque
mi cara, mis ojos llevan retratado el crimen que cometí. En mis pupilas
verá Marcela la imagen de su hermano moribundo, respondiéndome sí con la
cabeza. Si usted me aconseja que le oculte la verdad, no es usted tan
completo caballero como creí: no, no lo es.

---Te perdono tus dudas acerca de mi caballerosidad. Tú no estás bueno,
querido Nelet\ldots{} En cuanto a que declares, a que confieses tu
crimen, admito y apruebo que lo hagas; pero sólo en el tribunal de la
penitencia. No veo por qué motivo ha de ser Marcela tu confesor\ldots{}

---Sí lo es\ldots{} debe serlo, y yo quiero que lo sea---gritó Nelet.

---No grites, por Dios\ldots{}

---O me mato para callar, o vivo para confesarme con ella.

---Pues colocada la cuestión entre los términos de ese terrible dilema,
decido, ea, que vivas y confieses.

---¡A ella! Este fuego que ahora prende en mi conciencia y que me está
quemando cuerpo y alma, no se aplaca más que con la verdad\ldots{}
Luego, que sea de mí lo que Dios quiera.»

Con la idea de calmarle, fingió D. Beltrán asentir a lo que Santapau
decía: confiaba que el descanso, el sueño, las obligaciones militares,
el roce con sus compañeros, le traerían pronto a la vida normal y al
equilibrio de su mente. Procuró distraerle, hablándole de diversos
asuntos, y después de contarle con pintoresco estilo, no exento de
gracejo, la escena de su interrumpido suplicio en Rossell, le notificó
que Cabrera, con benignidad increíble, le había levantado la sentencia
de rehenes, y que confiaba obtener pronto su libertad.

Tuvo esta palabra la virtud de animar un poco al atribulado Nelet.
«¡Libertad! ---exclamó.---Yo también quiero ser libre\ldots{} ¡Muerte y
libertad! ¿No es cierto que la conciencia oprime? Pues hay que matar al
déspota, como dicen los patriotas y jacobinos\ldots{} matar al tirano
para ser libre. Por eso digo yo: «Muramos, libertémonos.»

\hypertarget{xxviii}{%
\chapter{XXVIII}\label{xxviii}}

Con sutil ingenio trató de hacerle ver D. Beltrán lo disparatado de
aquel conceptismo, dando su verdadero valor a las ideas de libertad y
muerte, harto graves ambas para ser tratadas en estilo de madrigal, y en
estas y otras charlas llegó la hora de partida, dispuesta repentinamente
por Cabrera cuando con más descuido saboreaban todos el descanso después
de tantas fatigas. ¡En marcha! ¡A correr, a combatir! ¿A dónde iban?
Cabrera no acostumbraba decirlo, y marchando al frente de sus tropas les
señalaba el camino. Agregose D. Beltrán en un caballejo que le
proporcionó su amigo Putxet, y entre este, que hablaba por los codos, y
Santapau, que parecía privado del don de la palabra, emprendió la
caminata por un sendero ingrato y polvoroso. Y por Dios, que ya se
cansaba el buen señor de tanto ajetreo; sus huesos le pedían descanso;
quizás en el nuevo estilo de Nelet, le decían: «Libertad, muerte.»
Gracias a su vigorosa fibra, a su carácter jovial y un tanto aventurero,
podía resistir los molimientos y privaciones inherentes a la vida
militar; y cuando el cansancio físico parecía irresistible, su
imaginación, reverdecida en lo juvenil, le deparaba algún nuevo estímulo
para proseguir en la carrera. Por dicha suya, o por desgracia, que esto
es dudoso, ante su vejez declinante no se cerraban nunca los horizontes.

Grande fue el disgusto del prócer en aquel camino, viendo que Nelet, sin
mejorar de su desazón espiritual, decaía visiblemente, como atacado de
un mal físico grave. A media tarde observó su amigo en él fiebre
intensísima; al anochecer, entrando en Arenys de Lledó, cayose el
comandante del caballo. Recogiéronle como cuerpo muerto y le arrimaron a
una pared, en tanto que Urdaneta, consternado de ver a su discípulo en
tan mala disposición, se determinó a manifestar al General la
imposibilidad en que aquel se hallaba de continuar su marcha. En la casa
del cura, donde tenía su alojamiento, recibiole Cabrera malhumorado,
revelando en su ceñudo rostro que no se había podido escoger peor
ocasión para pedirle favores. Mas el intrépido aragonés, a quien no
acobardaban entrecejos, no sólo pidió que Santapau fuera dado de baja
por enfermo grave, y quedase hasta su restablecimiento en aquel pueblo,
donde tenía familia, sino que se arrancó a solicitar que a él se le
permitiese también permanecer allí para asistirle. Observando en Cabrera
el centelleo de los ojos, el bilioso color tirando a verde, y la
inquietud \emph{leopardina} con que se paseaba de un ángulo a otro de la
jaula, creyó que a cajas destempladas le despediría, sin acceder a sus
peticiones. Mas no fue así: como un hombre afanado que aparta su
atención de las cosas menudas para aplicarla por entero a las grandes,
Cabrera le manifestó que tanto él (D. Beltrán) como Santapau se
fueran\ldots{} a cualquier parte, o \emph{mucho con Dios}, pues ninguno
de los dos le hacía falta para nada. «Usted, Sr.~de Urdaneta---le dijo,
plantándose ante él,---está libre, y puede volverse a sus estados de
Aragón. Para rehenes no me dan juego los aristócratas, y para
prisioneros me convienen los que trabajan y toman las armas. No es
desprecio, señor\ldots{} En cuanto a Santapau, que se me presente así
que esté curado, y si no cura y se muere, Dios le perdone\ldots{} Puede
usted retirarse. Quizás no nos veamos más, porque usted es muy viejo, y
yo, aunque joven, moriré pronto\ldots{} de un berrinche\ldots{} Adiós.»

Retirose agradecido el señor de Albalate, y Cabrera celebró Consejo,
para someter a la deliberación de unos cuantos individuos, clérigos la
mayor parte, el asunto que revestir quería de autoridad consultiva,
conforme a las fórmulas de gobierno impuestas por D. Carlos. No
estorbaba tal trámite al caudillo del Maestrazgo, que sabía cubrir el
expediente de \emph{oír} a los señores, y afectando respeto a sus
dictámenes, hacía después lo que le daba la gana. Los consejeros
quedaban muy satisfechos, creyéndose ruedas indispensables de la máquina
administrativa, y si algunos pudieron entrever que en el gobierno de
aquella región no eran más que figuras de adorno, churrigueresco por
añadidura, se consolaban con la risueña esperanza de obtener plaza en la
\emph{audiencia de ministros} de Valencia, o en el Consejo y Cámara de
Castilla, el día del triunfo. Al salir de la visita al General, se cruzó
D. Beltrán con los consejeros que entraban, y, sin dársele un ardite de
aquella farsa, no pensó más que en la obligación de alojar a su amigo
enfermo, para lo cual lo primero que hizo fue buscar a los parientes que
tenía Nelet en Lledó; pero como estos no parecían ni nadie daba razón de
dónde habían ido a parar, no hubo más remedio que acomodarse en alguna
de las casas donde, mediante pago, se les brindaba regular albergue.
Eligió D. Beltrán, por despejado y saludable, un \emph{mas} a la entrada
del pueblo, con casa vieja y grandona entre arboledas. El
\emph{masovero} era un viejo catalán, asistido de dos nietas guapas, la
una más que la otra, y ambas obsequiosas, atentas, un poquito redichas y
algo coquetas, razón por la cual la tal familia se le entró a D. Beltrán
por el ojo derecho. Dieron al enfermo un cuarto alto de la casa, con
mediano lecho, y al caballero anciano otro contiguo, donde había
simientes y colgaderos de hierbas en manojos puestas a secar. No le
pareció mal su residencia, a pesar de la dureza de la cama, que a las
piedras igualaba, y habría vivido allí muy gozoso, si el mal cariz de la
dolencia de su amigo no le tuviera en tan grande sobresalto.

Pasó Nelet la primera noche en un estado que a su maestro le pareció
gravísimo, con fiebre muy alta, delirio y agotamiento de fuerzas. Al día
siguiente amaneció con una fuerte erupción en toda la cara y parte del
cuerpo, como si le hubieran picado abejas. D. Beltrán no se apartaba de
su lecho ni de día ni de noche, atento a cuidarle con ayuda del
\emph{masovero}, hombre tan bondadoso como amañado, y de sus nietas, más
amañadas aún para todo lo doméstico. Como en el pueblo no había médico,
ni siquiera albéitar, entre D. Beltrán y \emph{Chimeta} (que así se
llamaba la mayor de las muchachas, y al propio tiempo la más bonita y
dispuesta), celebrando frecuentes consultas, diagnosticaron y
prescribieron lo que les dio la gana, determinándose por el sistema
expectante, el más fácil y barato, y tal vez el más científico. Quietud,
limpieza y frecuentes tomas de agua bien endulzada, fueron la única
terapéutica en los ocho días que duró la gravedad de Nelet, y en que los
brotes de la cara tomaron un aspecto por demás alarmante. Según el
\emph{masovero}, no era caso de viruelas, que él conocía muy bien por
haberlas visto más de una vez en su familia; era tan sólo un hervor de
sangre motivado de \emph{berrinche suspenso}, es decir, de una sofoquina
que por prudencia no había salido del cuerpo. Decía que no hay cosa más
mala que enfadarse en día de calor y no desfogar la rabia con palos o
bofetones. El que tal hace, lo paga con la salud y a veces con la vida.
Sucedieron a los ochos días de gravedad otros ocho en que cedió la
erupción, resolviéndose en muda de la epidermis; desapareció la fiebre,
y el enfermo pudo tomar alimento, aunque siempre con repugnancia. Su
inteligencia, completamente obscurecida en aquel período, revelaba una
honda crisis: su palabra era torpe, cansada, regañona. Tanto D. Beltrán
como \emph{Chimeta}, persistiendo en la puntual asistencia, se
confirmaron en la superioridad incontestable del tratamiento acuático,
sin mezcla de ninguna droga, y proclamáronse curanderos de primer orden,
capaces de ejercer el arte con no poca fama y provecho. Era
\emph{Chimeta} muy graciosa, y a D. Beltrán se le caía la baba oyéndola
bromear y reír por cualquier fútil motivo. En su aturdimiento senil,
olvidado ya del trance terrible de Rossell y de los actos de
arrepentimiento con que allí limpió su conciencia, se le reverdecieron
las aficiones de toda la vida, y su habitual culto del bello sexo
encontraba ante aquella sencilla y tosca ninfa ocasiones de gran
lucimiento. Para ella era un deleite novísimo oír los galanteos
refinados, y hasta cierto punto paternales, del Sr.~de Urdaneta, y a él
se le refrescaba el alma, se le avispaba el entendimiento, se le
aliviaba el peso de los años. Todo era inocente, madrigalesco, puro
juego de frases agudas untaditas de miel: sobresalían en él las buenas
maneras y el propósito, casi siempre logrado, de no caer en lo ridículo;
en ella se veía la mujercita exuberante de vida que quiere adquirir
soltura en la esgrima y en el lenguaje de la lucha pasional.

Mas ¡ay! cuando \emph{Chimeta}, llamada de sus obligaciones, dejaba de
acudir al enfermo, y con este se encontraba sólo D. Beltrán, ya no podía
el hombre librarse de la tristeza. Cierto que había recobrado la
libertad, inapreciable don; pero el asunto que le trajo a tierra de
Teruel continuaba sin resolver. No creía ofender a Dios deseando que
viniera a sus manos lo que estimaba de su legítima pertenencia; y sin
apartarse del orden de sentimientos que el angustioso paso de Rossell
despertara en su alma, se condolía de tener que volver a Cintruénigo en
situación desairada y con las manos vacías. Las esperanzas de remedio
que había concebido se disipaban ya, pues Nelet tenía trazas de quedarse
idiota: no razonaba; sus conceptos eran incoherentes o de una
simplicidad rayana en la estupidez. Para mayor desdicha, nada se sabía
de la monja vagabunda y enterradora de caudales. No aportaba por allí
\emph{Malaena} ni para traer ni para llevar sus velocísimas embajadas,
sin que esta ausencia pudiera achacarse a ignorancia del lugar donde los
caballeros residían, pues por los oficiales del 3.º de Tortosa, a
quienes se dejaron instrucciones muy precisas, debía tener conocimiento
de la enfermedad de Nelet y de su forzosa estancia en Lledó. «Aunque no
sea más que para decirnos que nada sabe de la hija de Luco---pensaba Don
Beltrán en sus soledades tristes,---la mensajera tiene que venir.» Y
tanto deseó a la mujercilla ratonil, y con tanta fuerza la reclamaba su
voluntad, repitiendo el \emph{vendrá}, \emph{tiene que venir}, que una
mañana, como por virtud de conjuro, apareció la vieja. \emph{¡Hosannah!}
Veinte días llevaba ya de enfermedad el pobre Santapau, y su
entendimiento despertaba perezoso, tratando de cobrar con lenta cacería
las ideas dispersas, fugitivas, descarriadas.

En la huerta del \emph{mas} recibió D. Beltrán a la embajadora loco de
contento, y este subió de punto al saber que Marcela no andaba lejos de
allí, pues sabedora de la muerte de su hermano, se encaminaba con los
viejos a Gandesa por el Monte Caro, con el fin de recoger el cadáver y
darle sepultura. No quiso el buen caballero que \emph{Malaena} se
presentase a Nelet, pues aún no estaba este en disposición de recibir
emociones vivas, que podrían retrasarle en su penosa convalecencia; y
dando de comer a la mensajera, y aposentándola en la cuadra con
comodidades para ella desconocidas, la interrogó prolijamente, tratando
de indagar, no sólo los propósitos, sino el estado de ánimo de la santa
mujer. Poco pudo informarle \emph{Malaena} de estos particulares. La
última vez que vio a Marcela fue cerca de un castillo que hay a la
bajada de Monte Caro para ir hacia Pauls. Iban ella y los viejos cuesta
arriba, llevando una olla muy pesada, tan pesada, que se relevaban para
cargarla.

«¿Les viste saliendo del castillo o entrando en él?---preguntó D.
Beltrán con afectada indiferencia.

---Hacia él iban, señor---replicó la vieja en valenciano, que el
caballero tradujo fácilmente;---mas no sé si llegaron o siguieron de
largo, pues la sacra señora, dándome pan y queso, me mandó que me
retirara, y yo me retiré comiendo, sin mirar para atrás.»

Eran estas referencias como una mano blanda y tentadora que en el alma
del noble anciano revolvía, y con sus halagos despertaba la codicia,
sierpe aletargada desde las efusiones cristianas del terrible día de
Pentecostés. Se argumentaba para calmar su conciencia, diciéndose que
desear lo suyo y perseguirlo no era desatino grave, sino intención
equitativa; pero entre el desear y el temer, ello es que perdía el
sueño, y su espíritu se distrajo de las alegrías que el trato de
\emph{Chimeta} le daba, alegrías tras de las cuales se ocultaba con
senil rubor una honesta adoración, un sentimiento que casi no era más
que estético goce.

\hypertarget{xxix}{%
\chapter{XXIX}\label{xxix}}

Viendo muy mejorado a Nelet, diole cuenta de la reaparición de
\emph{Malaena} y de lo que habían hablado; excitose el enfermo,
recobrando de golpe su locuacidad, y a las primeras palabras hubo de
comprender D. Beltrán que se renovaba en toda su intensidad el enfadoso
mal de conciencia; no vaciló el maestro en atacarlo con brío diciendo:
«Entrégate a mí, pues no estás en disposición de resolver por ti mismo
cosa tan grave. Yo lo arreglaré con tan buena maña como pura honradez.
Tus escrúpulos se disiparán, y Marcela será tu esposa. Tu delicadeza es
ya locura. Conviene que moderemos hasta nuestras virtudes\ldots{} Y si
te encuentras en disposición de caminar, no será malo que salgamos en
seguimiento de la divina mujer.» Accedió Santapau, y se convino en
esperar dos días para mayor acopio de fuerzas, pues no teniendo caballos
ni posibilidades de adquirirlos, era forzoso emprender a pie la dura
caminata.

Llegado el día de la marcha, salieron, y fue un paso triste para D.
Beltrán el separarse de la linda \emph{Chimeta}, que con sus donaires y
risotadas se le había metido en el hueco preferente del viejo corazón.
No digamos que le turbaban pretensiones absurdas respecto a la muchacha:
no era sino que le dolía separarse de ella, como duele el arrancarnos
cualquiera raicilla que penetra en el alma, y la de D. Beltrán tenía un
terruño muy propicio al arraigo de toda hierba. ¡Nunca más ¡ay! volvería
a ver a la ninfa tosca de Lledó! Era un adiós en la puerta de la
eternidad, adiós dado al bello sexo, a la humana belleza, a las únicas
flores que alegran este valle de lágrimas. Casi con ellas en los ojos,
realmente conmovido, se despidió el señor de tantas Torres, besando la
mano áspera y gordezuela de \emph{Chimeta}. Le deseó un buen novio para
hacer de él un buen marido, y le recordó los consejos que le había dado
para dominar a los hombres y hacerse querer locamente de ellos.
Agradecida la ninfa, así como su hermana y abuelo, a las bondades de los
dos señores, les vieron partir con pena, pidiendo a Dios para ellos
salud y prosperidades.

Acompañados de \emph{Malaena} se metieron por los atajos y recodos que
conducen a Horta, donde pensaban terminar su primera jornada. Parecían
dos pobres titiriteros, seguidos de un perrillo con faldas, o mejor, de
un cuadrumano con cuyas monadas y brincos pedirían limosna de pueblo en
pueblo. Iba Nelet vestido como en la facción, sin insignias, armado de
cuchillo y pistolas; mas en la traza total de la cuadrilla, las armas, a
primera vista, parecían trebejos para el arte de volatines o
prestidigitación. Muy mal de ropa estaba el primer noble aragonés; pero
aun así no se despintaba su empaque de persona principal. Andaban
despacio, guardando silencio en largos trayectos, charlando a veces con
lánguida conversación. Temerosos del encuentro con alguna columna
cristina, mandaban a \emph{Malaena} por delante, a la descubierta, para
que ojeara toda emergencia de gente sospechosa en aquellos horizontes.
En el descanso de Horta, albergados en una paridera a la entrada del
pueblo, explayose Nelet a contar a su maestro \emph{las cosas que le
andaban por dentro del espíritu}, en verdad muy extrañas, y las visiones
que desde los comienzos de su enfermedad le acosaban, alguna de las
cuales tuvo poder bastante para obscurecer a las demás y resplandecer
sola y continua en el campo luminoso de la óptica interna.

«Esto que voy a contarle---dijo Santapau, recostándose en el suelo junto
a su amigo después de mal cenados,---lo vi muy claro la primera noche de
mi enfermedad en Lledó; después se me fue apagando\ldots{} lo veía
turbio, desvanecido, mezclado con otras imágenes; pero al entrar en
convalecencia, volví a verlo claro, cada noche más, y más\ldots{}
llegando a tanto su claridad, que ya lo veo también de día y con los
ojos abiertos.

---Cuéntamelo pronto, que ya estoy ardiendo en curiosidad. No dudo que
ello tendrá relación con el fin y empresa que mueven tu vida, y que la
imagen de Marcela será centro de todas esas esferas y círculos de tu
soñar loco\ldots{}

---Pues oiga usted. Desde que \emph{me entra}, ya me tiene usted
corriendo a caballo tras de la monja de Sigena.

---¡Y ella\ldots{} a patita! Poca ventaja te llevará.

---No puedo decir cómo va, pues no la ven mis ojos\ldots{} Sé que va
delante, la siento, la olfateo. Yo grito; ella no me oye.

---Y sigues, sigues\ldots{} arrimando espuela.

---No espoleo porque voy desnudo de arreos, de ropa y hasta de carne.
Soy un esqueleto. Mi caballo es también esqueleto\ldots{} de caballo, se
entiende\ldots{} y ni yo tengo más espuela que el hueso del carcañal, ni
él tiene barriga en que yo pueda espolearlo\ldots{} Mas no es preciso,
porque corre, corre sin que yo le diga nada, haciendo con sus cuatro
cascos un compás de música que no se aparta ya de mi oído.
\emph{Pataplás, parrataplás}\ldots{} siempre así.

---Sufrirás mucho corriendo tras un fantasma sin alcanzarlo nunca.

---Más que la persecución del fantasma, me hace padecer el
\emph{pataplás} de mi cabalgadura y los estragos que causa al sentar
alternadamente los cuatro cascos como mazas de hierro\ldots{} ¿Por dónde
voy en esta carrera? Por un campo que parece árido y no lo es. Lo
parece, porque en él no nace ningún árbol, ni mata, ni hierba; no lo es,
porque está todo lleno de seres vivos, chiquitos, que nacen en él y por
entero lo cubren\ldots{} No se ve el suelo: no hay dónde poner una pieza
de dos cuartos. ¿Qué son? dirá usted, ¿qué vidas son aquellas? Pues son
niños, señor D. Beltrán; no ángeles, que alas no tienen, sino criaturas
como las de acá, como las del mundo, como nosotros cuando teníamos un
año, dos años\ldots{}

---Hombre, sí que es rara, estupenda visión\ldots{} ¿Pero esos
niños\ldots?

---Nada, señor, niños. ¿No sabe usted lo que son niños, criaturas, o
como dicen los gitanos, \emph{churumbeles}? El campo absolutamente lleno
de ellos. ¡Y qué lindos, qué graciosos! Gorjean, ríen con esa carcajada
del chiquillo que se embelesa mirando una luz. ¿De dónde salen? De la
tierra, pienso yo, apretados unos contra otros, como los tallos de la
hierba\ldots{} desnuditos, rollizos, ligeros\ldots{} Bueno: pues por
este campo de niños paso yo a la carrera. Mi caballo les va destruyendo
con sus patadas, y ellos vuelven a salir, vuelven a nacer, y a gorjear y
a reír\ldots{} siempre chiquitos y monos; ya digo, de año y medio o dos
años, y en número incalculable. En todo lo que alcanza mi vista, no se
ve más que el campo lleno de nenes. Se agita el sin fin de cabecitas
haciendo ondas, como un campo de trigo, y las ondas traen y llevan el
gorjeo. Mi caballo recorre como el viento leguas y leguas, y siempre lo
mismo, machacando criaturas, que vuelven a salir vivitas,
alegres\ldots{} Si le digo a usted que son cuatro mil cuatrillones, no
digo nada, pues son más, más\ldots{}

---¿Y su única voz es el gorjeo? ¿No has reparado sí dicen \emph{papá} y
\emph{mamá}?

---No lo dicen; pero es como si quisieran decirlo.

---Está bien. ¿Y qué hace mi señora beata en el campo de niños?

---No sé\ldots{} allá lejos va\ldots{} yo no la veo. Se me antoja que al
golpe de sus pisadas brotan las criaturas.

---Hijo, visión más peregrina no atormentó jamás a ningún cristiano. Lo
que no alcanzo es qué relación pueda tener ese campo infantil con tus
cuitas, Nelet.

---Yo tampoco lo alcanzo\ldots{} Pero ello es que la visión no me deja.
Hasta de día y muy despierto la tengo ya. Los gorjeos también se agarran
a mi oído. Y no miento si le digo a usted que a toda esa inmensa
chiquillería la quiero ya\ldots{} ni más ni menos que si fueran mis
hijos\ldots{} ¿Lo serán? pienso yo. ¿Serán los que tuve o debí tener en
cuatro mil cuatrillones de siglos que viví antes de esta vida?

---¡Demonio, echa siglos y generaciones!\ldots{} ¿Sabes que tu
fantástico sueño es para marear y confundir la cabeza más firme?

---La mía no puede ya con más confusión.

---Y eso es contagioso\ldots{} Temo que me pegues tu mal. Cállate ya,
por Dios, que yo voy a soñar también lo mismo\ldots{} pisoteando
nenes\ldots{} quita allá\ldots{} ¡qué atrocidad!\ldots{} Cállate, que no
quiero yo soñar eso, no quiero.»

Guardaron silencio, y a poco dormían ambos; mas se ignora lo que
soñaron, y si fue un hecho el contagio que D. Beltrán temía. A la mañana
siguiente, que se presentó lluviosa, continuaron andando con no poca
molestia, amparándose bajo los árboles cuando el llover arreciaba. El
suelo arcilloso, lleno de charcos, les causaba grande enojo, y tan
pronto se detenían ateridos al abrigo de un paredón, como aceleraban su
andadura, afanosos de llegar pronto a poblado. Renegando de tales
contratiempos y de las perversas condiciones en que viajaban, dijo
Santapau a su amigo, guarecidos en una aldea mísera: «Ni usted ni yo nos
resignamos a andar de camino como unos miserables titiriteros,
careciendo de todo, mal vestidos, perdiendo la paciencia, el tiempo y la
salud. Necesitamos caballos, vestidos, dinero. Puesto que estamos tan
cerca de Cherta, donde tengo familia, amigos y un \emph{mas}, cuya renta
de doscientos ducados no he cobrado este año, nos llegaremos allá, o me
llegaré yo solo, si usted no se halla muy dispuesto. Sólo estaré el
tiempo preciso para recoger todo el dinero que pueda y proporcionarme un
par de caballos o mulas, o aunque sean borricos\ldots» Pareciole de
perlas a Don Beltrán este propósito; mas se declaró perezoso de
acompañarle, pues se hallaba rendido, aspeado, lleno el cuerpo de
dolores y con ganas de guardar sus huesos en abrigo media semana para
repararlos de los efectos del último remojo. Convinieron en que iría
solo Santapau al romper el día: conocía perfectamente todos los senderos
y atajos, y no contaba emplear, andando sin sofocarse, arriba de tres
horas. D. Beltrán se quedaría en la aldea, que era el barrio más lejano
de Prat de Compte, al cuidado de \emph{Malaena}, reponiéndose del
quebranto producido por la caminata y la mojadura. Partió Nelet
tempranito, agregado a una cuadrilla de mujeres que iban a Cherta con
haces de leña, y el ilustre señor se quedó en un blando lecho de paja,
arreglado por la que había venido a ser su camarera. En la memoria del
buen viejo se reprodujo la noche pasada en Fuentes de Ebro, bien
apañadito en montones de paja. ¡Pero qué diferencia entre la bella
Saloma, tan graciosa y diligente, y aquella desmañada viejecilla de
Vallivana, que no servía más que para correr de monte en monte! La
compañía de la navarra, su excelente disposición y cháchara festiva,
trocaban en palacios las cuadras de los mesones, mientras que
\emph{Malaena} todo lo afeaba y envilecía. Encargole D. Beltrán unas
sopas de ajo, y tan mal las hizo, que sólo a fuerza de hambre pudo
pasarlas el pobrecito viejo. Por su ineptitud para todo lo doméstico,
por su salvajismo y suciedad, se le había hecho antipática, y le azoraba
con su prurito de confianza y de palique cuando más deseaba él estar
solo, callado y libre; el brillo y la continua vigilancia de sus
ratoniles ojos le ponía nervioso; sus familiaridades llegaron a ser de
una pesadez impertinente, como si desconociera el respeto que a tan alta
persona debía guardarse. Creyérase que le tomaba por titiritero
arruinado en el oficio. Sentadita frente a él sobre la paja, le dijo en
dulce valenciano, que es forzoso traducir: «¡Qué hace ahí tan metido en
su magín, cavilando maldades! \emph{Vosté} no está ya más que para
ponerse en paz con Dios.

---Pienso lo que me da la gana---replicó D. Beltrán, esquivando la
mirada de las cuentas de azabache que \emph{Malaena} tenía por
ojos.---¿Quién te manda a ti meterte\ldots? ¡vaya!

---Me meto por llamarle a Dios, que ya es tiempo. Más vejestorio es
vosté que yo. Me da lástima de que la muerte le coja descuidado.

---¡La muerte! ¿Acaso estoy yo para morir?

---Yo no sé leer escrituras, pero leo la muerte en la cara de la
persona.

---Vete al demonio\ldots{} Te encargó Nelet que me acompañaras, no que
me faltaras al respeto.

---No falto al respeto diciéndole a \emph{vosté} que se muere. No me
equivoco.

---¡Embustera, quítate de ahí! Aunque algo cansadito, me siento fuerte,
y paréceme que aún tengo años por delante.

---Días tiene, y los dedos de una mano le sobran para contarlos.

---¡Lárgate pronto, condenada!» gritó Don Beltrán estirando
violentamente una pierna contra la paja.

La vieja se fue. Y en su imperfecta vista creyó el pobre caballero que
desaparecía como un ratón por entre los informes y obscuros objetos que
llenaban la cuadra, revestidos de telarañas y polvo\ldots{} Solo ya,
meditaba. ¡Si tendría razón la maldita vieja! No, no: él no hacía caso.
¿Qué podría saber de vidas y muertes una pobre rústica, salvaje, casi
idiota? ¡Vaya que estaba divertido! ¡Después de una mala noche, soñando
con el campo de niños y oyendo sus gorjeos, un día de prisión junto a
semejante sabandija, que no era, no, que no podía ser cosa buena\ldots!
Sintió un ruidillo de dientes sobre cosa dura, y a poco se le apareció
\emph{Malaena} royendo algo que llevaba de la mano a la boca con
movimiento jimioso. Acercose a él y le observó, aproximando su rostro de
pasa. Al verse mirado por los ojos ratoniles, D. Beltrán sintió frío,
miedo. «Vete---le dijo.---Me molestas.» Y ella: «Ya me voy. ¿Quiere
estar solito para calentarse los cascos con sus malas ideas?\ldots{}
Diviértese \emph{vosté} jugando con el pecado de la codicia, y piensa
que le van a dar ollas de dinero\ldots{}

---¡Calla, vete pronto!» gritó Urdaneta ronco, fuera de sí.

Y tan sobresaltado quedó el hombre para todo el día, que cuando
\emph{Malaena} se acercaba al lecho de paja, sentía el hombre verdadero
pánico. Tomó el partido de cerrar los ojos y rodearse la cabeza con los
brazos como para llamar el sueño; pero este no le favoreció, ni tampoco
Nelet, regresando aquella tarde como había prometido. ¡Qué soledad, qué
triste abandono! Pasó la noche agitadísimo, sintiendo que \emph{Malaena}
le tiraba de los pies para llevárselo\ldots{} ¿Era bruja, era un diablo
humanizado en la forma más odiosa? No hacía el pobre más que dar golpes
en la paja, al modo de coces, murmurando: «Vete, demonio, vete; déjame.»

Pero ¡ay! mientras Santapau no volviese, ¿qué remedio tenía más que
vivir resignado bajo el poder de la infernal bestiezuela de Vallivana?
Dejábase cuidar de ella, y probaba con repugnancia los bodrios que le
servía\ldots{} Pasó todo el día entregado a las absurdas creencias. Él,
que nunca fue supersticioso, ya creía en demonios aviesos, en asquerosas
brujas y en trasgos maleantes. Y como a la segunda noche tampoco
pareciese el bueno de Nelet, viose el señor de Albalate tan desamparado,
que hubo de volver los ojos a Dios. Sólo con esto se le fue del alma la
superstición, y abominando de tales torpezas, se sintió profundamente
religioso, como lo había sido en algunas ocasiones aflictivas de su
cautiverio, y singularmente en el tremendo paso del día de la
Pentecostés. Sobrevino, pues el estado de arrepentimiento y contrición,
dolor de haber ofendido a Dios con una vida de libertinaje; sobrevino el
desprecio de las riquezas, el espanto de las malas acciones, así pasadas
como presentes. Al amanecer del tercer día llamó a su ratonil guardiana,
y con buen modo le dijo que hablase a los dueños de la casa antes que
salieran al campo, concertando con ellos que le llevaran un sacerdote,
pues sentía vivísimo anhelo de confesarse. Cumplió la vieja el encargo
con toda diligencia; mas como no había en el lugar ni en sus contornos
clérigo alguno, hubo de quedarse el noble señor sin el consuelo y
descanso que deseaba.

Enojosas fueron para él las horas de aquel día, pues sin que se calmara
el infantil terror que la seca viejecita le inspiraba, le atormentó el
tumulto de su alborotada conciencia. Veía muy clara su abominación, pues
cuando Dios le conservó la vida en Rossell, en vez de mostrar gratitud
conservando su alma en la pureza y descargo de su arrepentimiento, lo
que hizo fue reincidir en sus antiguos vicios. No fue cosa grave el
encandilarse un poquito con la gentil \emph{Chimeta}; pero sí lo era el
incurrir de nuevo en la fea codicia, afanándose por el legado de Juan
Luco, y más aún la persistencia en agenciar con móvil egoísta el casorio
de Nelet y Marcela. La situación moral había empeorado, pues al pecado
antiguo de querer secularizar a una esposa de Cristo, se unía el
propósito de engañarla, ocultándole que su galán o pretendiente era el
matador de Francisco Luco. ¡Oh qué grande malicia, Señor! ¡Y de este
modo y con intenciones tan protervas, pagaba la inmensa benignidad de
Dios, que le había concedido la vida cuando ya casi apuntaban a su pecho
los fusiles facciosos!

Encendida su alma en fuego de contrición, gritó llamando a su guardiana.
«\emph{Malaena}, ven. Ya no me inspiras miedo. ¿Verdad que no eres
demonio ni bruja? Yo veía en ti el daño y corrupción que en mí propio
llevaba. Perdóname. Eras para mí lo que para los niños el coco. Pero
¡ay! ya he visto que el coco dentro de mí lo tenía yo: era mi
conciencia\ldots{} Pues te digo que Dios me ha iluminado, y vuelvo al
bien y a la virtud. Si me muero, que me muera. No más, no más pecar, no
más pensamientos infames. Corra quien quiera tras un puñado de oro; yo
no. No más supercherías con Marcela\ldots{} Gobierne la santísima verdad
los días que me restan, pocos o muchos. Quiero salvar mi alma. Mi alma
merece salvarse\ldots»

En esto sintieron ruido de gente y caballerías. Era Nelet que llegaba de
Cherta.

\hypertarget{xxx}{%
\chapter{XXX}\label{xxx}}

No fue el gozo de D. Beltrán, al abrazar a su amigo, proporcionado a una
ausencia de tres días: fue como por ausencia de tres años, y de la
fuerza del contento se le trastornó el sentido, viendo a Nelet más
fuerte, más gallardo, restablecido de su reciente mal, la cara limpia
del rojizo color de quemadura. Era ilusión del pobre viejo que veía lo
que deseaba. Por su parte, Santapau encontró a su maestro más caduco,
encorvado, jadeante, algo ido del cerebro, progreso de senectud excesivo
para tres días. Mostrole muy satisfecho lo que traía: dos soberbios
burros, pues caballos no los encontrara ni a peso de oro. Eran
excelentes piezas, de cómoda andadura, y muy bien enjaezados. Traía
también ropa para los dos, y un repuesto copioso de vituallas en una
cesta barriguda. Despedido el criado del \emph{masovero} que había
venido en el segundo pollino con la cesta y equipaje, los dos caballeros
pusiéronse a cenar. Tiempo hacía que Urdaneta no probaba cosas tan
ricas: butifarras, diversas clases de suculentos embutidos, pollos
asados, frutas escarchadas, chocolate, bollos de sartén\ldots{} Más que
en saborear aquellas viandas, gozaba D. Beltrán viendo a \emph{Malaena}
devorar, con atrasadas hambres, comidas tan finas en increíbles dosis.

«Pues he tardado tres días---dijo Nelet,---porque las grandes novedades
que encontré en Cherta, y el barullo de gente y amigos, me
imposibilitaron el despacho de mis diligencias en el tiempo que yo creí.

---Oí que estos días ha pasado hacia allá mucha tropa de uno y otro
ejército. ¿Qué ocurre?

---Sí\ldots{} tropas de Isabel, tropas de D. Carlos. Se ha batido bien
el cobre\ldots{} Vámonos pronto de aquí, antes que nos coja el paso de
los míos, que ahora son en número mayor que antes. En una palabra: ya
tenemos la expedición Real del lado acá del Ebro, en Cherta, gracias al
talento militar de Ramón Cabrera y a su arrojo y prontitud. Está el
hombre que no cabe en su pellejo de puro orgulloso\ldots{} Sí, sí: no se
asombre usted. Don Carlos ha pasado el Ebro. Yo le he visto\ldots{} he
visto al Rey, a nuestro ídolo, y le aseguro que me quedé como si hubiese
visto a cualquiera que no fuese ídolo de nadie. Yo me figuraba otra
cosa, otro empaque, otra representación de gran Monarca, hijo de reyes y
ungido de Dios. De esta hecha, nuestro \emph{leopardo}, como usted dice,
ha puesto una pica en Flandes, porque gracias a su buen tino para
ordenar las cosas, ha podido Don Carlos librarse de Borso y Nogueras,
que le perseguían. Junto a Cherta dio Cabrera una batalla al Sr.~de
Borso, obligándole a retirarse. Nogueras cometió la mayor pifia que se
puede cometer en la guerra, que es no llegar a tiempo. La guerra no es
más que el arte de la oportunidad, y este lo posee D. Ramón como nadie,
y lo completa con su diligencia y conocimiento del terreno. Pasó Carlos
V tranquilamente el gran río de España, en lanchas al caso preparadas, y
los gritos de entusiasmo de las tropas competían en estruendo con el
instrumental de las músicas. Enronquecieron gargantas y trombones. Ayer,
la Sacra Majestad y todo su séquito mataron el hambre en Cherta, que la
traían atrasadilla, porque la batalla que ganaron en Huesca les dio más
prisioneros que bucólica. El comistraje que se les preparó era de lo más
opíparo, y para que hubiera de todo, hasta helados hubo\ldots{} Cabrera,
el día del paso, que fue anteayer, estaba como loco, demacrado, los ojos
del tamaño de toda la cara, echando rayos y centellas. Daba sus
disposiciones ronco de tanto gritar, vestido con su peor ropa, pues ni
para engalanarse como acostumbraba tuvo tiempo. Cuando se presentó al
Rey en la orilla izquierda para pasarle acá, no le conocían, y los
cortesanos se preguntaban asombrados: «¿Pero ese es Cabrera?\ldots{}
¿ese?» El Soberano le manifestó su Real agrado. No serán flojas envidias
las que van a salir ahora, pues corre la voz de que S. M. quiere
nombrarle Generalísimo, y poner bajo su mando todos los Reales
ejércitos.

---Así tiene que ser, pues según tengo entendido, de los figurones que
rodean al Infante, poco debe esperar este\ldots{} Y dime otra cosa:
¿oíste o viste si con el Rey viene un italiano llamado Rapella, que es
el correveidile entre cortes verdaderas y falsas para tratar de un
arreglo por bodorrio?

---Creo haber oído algo de un italiano de campanillas, y de otros
extranjeros que en la comitiva del Rey vienen, entre la turbamulta de
empleados y gentileshombres. Pero como yo, por el estado de mi espíritu,
no podía prestar a lo que allí veía una gran atención, no puedo asegurar
nada de italianos ni correveidiles.

---¿Y qué se dice? ¿La expedición, con su Rey a cuestas, dirígese a
Castilla o a Valencia? Puede que reforzada con Cabrera, y quizás mandada
por este, no se detenga hasta Madrid. ¿Oíste algo?

---Oí, oí\ldots{} no sé lo que oí---dijo Nelet aturdido.---¿A usted le
interesa saberlo?

---Absolutamente nada.

---Lo mismo que a mí. Que vayan, que vengan, que suban, que bajen. No me
interesa ya más que un reino, el mío. Cada cual se arregle en su reino
como pueda.

---Muy bien dicho. Peleáis por poner en el Trono a un buen hombre, cuya
incapacidad es bien manifiesta. Si tus amigos triunfan, estableceréis un
imperio caedizo, pues en los tronos disputados, el vencedor no lo será
definitivamente si no posee estas cualidades: bravura, don de mando,
ciencia militar. Gane quien gane en este pleito, querido Nelet, la
Monarquía carecerá de fuerza y vivirá con vilipendio, entregada a las
facciones. Ten presente que no se hace nada de provecho sin fuerza,
entendiendo por esto, no el poder de las armas, sino una virtud eficaz y
activa, que a veces reside en una persona, a veces en las leyes. Ni las
leyes tienen aquí fuerza, o llámese energía gobernante, ni hay Rey o
Príncipe que tal posea. Puede que nazca algún día; mas yo te aseguro que
a la fecha no ha nacido. De modo que paz, lo que se llama paz, no la
veréis en mucho tiempo los que sois jóvenes, ni quizás lo vean vuestros
hijos y nietos\ldots{} Con que lo que tú dices: cada cual a su
reino\ldots{} y en el reino chico de cada uno, que no falte una
ventanita para ver pasar la Historia.»

No prestaba Nelet a estos profundos juicios la debida atención, ni se
extendió tampoco en pormenores de lo que presenciara en Cherta, porque
sus impresiones eran confusas, como de quien ve muchas y abigarradas
cosas en corto tiempo, sin interés ni recreo alguno de su ánimo. Mirando
no más que a su reino, propuso que partieran a la mañana siguiente en
busca de Marcela, pues por fidedignos informes que en Cherta adquirió,
venía ya de vuelta de Gandesa, después de recoger el cuerpo de su
desgraciado hermano y darle sepultura. Al capellán del 1.º de Tortosa,
que la encontró una mañana junto al río Seco, dijo la monja que pensaba
detenerse un día en el Santuario de San Salvador y encaminarse luego a
Arenys de Lledó a visitar a un enfermo. Iba, pues, en busca de sus
amigos, los cuales se apresurarían a salirle al encuentro. Para mayor
seguridad, dispuso Nelet que partiera la embajadora aquella misma noche,
con instrucción precisa de las etapas que los caballeros seguirían y
puntos de descanso, y la consigna de que esperasen en Lledó los que
primero llegaran.

Conforme con tan acertado plan, y admirando el tino con que Nelet lo
concertaba, creyó D. Beltrán llegada la oportunidad de manifestar al
discípulo el estado de su ánimo, y sin más exordio le dijo: «Durante tu
ausencia, hijo mío, no he cesado de reflexionar en el caso de Marcela,
complicado ahora con la desastrosa muerte del pobre Francisco, y,
discutiendo la solución que debemos darle, me sentí acometido del mal
tuyo reciente, el mal de conciencia. Dios ha entrado en mí. Como avisos
o presagios de la naturaleza flaca, procedieron a mi mal miedos
supersticiosos, la idea de una muerte próxima. Era Dios que llamaba a la
puerta de mi alma: no entendía yo su llamamiento, hasta que te vi entrar
y me iluminó con su divina gracia. ¡Ay! querido Nelet, no quiero en mis
postrimerías comprometer mi alma. ¿Amo a Dios, le temo? Amor y temor por
igual me consuelan y sobrecogen; amor y temor me infunden el anhelo de
ser bueno en lo que resta de vida, de sostener con una conducta ejemplar
la paz, mejor será decir la salud de mi conciencia\ldots{} Reniego ya de
aquel propósito y consejo mío de ocultar a Marcela la verdad de tu
culpa, pues si con ese artificio ganaríamos bienes terrenales,
perderíamos seguramente los eternos. No, no, Nelet: tú estabas en lo
cierto y yo en lo errado; tú en la verdad, yo en la mentira; tú
procedías como cristiano caballero, yo como un hombre vil\ldots{} Pero
ya no\ldots{} Ahora te digo que la ocultación, o siquiera disfraz de la
verdad, es gran pecado; me paso a tu partido, y en él te fortalezco.

---Pienso, amigo mío---dijo Nelet con gravedad,---que esta concordia de
la voluntad de usted con la mía es cosa muy feliz. No hay duda: Dios o
los ángeles han andado en ello. Hoy como ayer considero felonía el negar
a Marcela mi culpa; mas, no teniendo yo valor ni cara para confesarla
ante ella, convendría que usted le hablase antes, y de la sentencia que
se sirva dar depende mi destino.

---Muy juicioso me parece lo que has discurrido. Yo le hablaré antes, yo
le diré\ldots{} ¡Oh, si te perdonara reconociendo que fuiste víctima de
un arrebato!\ldots{} ¡qué triunfo, hijo! Me da el corazón que así será,
pues los caminos de la verdad siempre llevan al bien.

---¡Perdonarme!---exclamó Nelet clavando sus miradas en el
suelo.---¡Pues si así fuera\ldots! Pero lo dudo\ldots{} Ya no veo la
cabeza de Francisco Luco diciendo que sí con aquel movimiento
fuertísimo\ldots{} la veo diciendo que no\ldots{} así\ldots{}
así\ldots{} que es decirme: no hay perdón.

---Basta ya de visiones, hijo. Tus desvaríos me contagian, y estas
noches he soñado que también yo cabalgaba por el campo de niños\ldots{}
sólo que mis nenes, los nenes que yo destruía, no volvían a nacer.

---Pues los míos\ldots{} en las noches últimas\ldots{} ya no reían, sino
lloraban\ldots{} Vi a Marcela cogiéndoles a puñados y metiéndoseles en
el seno\ldots{} Pero, lo que usted dice: basta ya\ldots{} No duermo, no
quiero dormir: pasaré la noche pensando en que ella viene en nuestra
busca y en que le salimos al encuentro. Descanse usted, y yo le llamaré
cuando sea hora de partir. Voy a despachar a \emph{Malaena} y a dar un
pienso a nuestros burros.»

Cumpliose con toda puntualidad lo que Santapau disponía, y antes del
alba salieron ambos caballeros oprimiendo los lomos asnales, D. Beltrán
algo remediado de ropa, Nelet bien provisto de armas, pues ignoraban qué
clase de gente encontrarían. Anduvieron toda la mañana sin ver alma
viviente, entreteniendo las lentas horas con el inagotable y pavoroso
tema: «¿Me perdonará?» Llegados al caer de la tarde a una ermita en la
derecha margen del río Seco, que era el punto de cita con la embajadora,
recibieron de boca de esta las deseadas noticias. Había dejado a Marcela
con sus viejos en el convento abandonado de San Salvador, y allí pasaría
la noche en rezos y meditaciones; al amanecer recalaría en el castillo
de Horta, donde los señores podían reunirse con ella para seguir juntos
a Lledó, o al punto que de acuerdo fijaran. Aumentose hasta lo increíble
la ansiedad de Nelet. ¡Ya estaba cerca! Sólo una noche y un breve
espacio de terreno le separaban de la solución del temido enigma: «¿Me
perdonará?» Incapaz de todo sosiego, acordó seguir hasta Horta, y en
ello emplearon las primeras horas de la noche. Con no poco trabajo
pudieron hallar albergue y pienso para los burros y \emph{Malaena} en
una reducida cuadra; descansó en el mismo recinto D. Beltrán algunas
horas, mientras Nelet se paseaba suspirando, a la luz de la luna, en un
próximo corral, como caballero que vela sus armas; y antes que fuera de
día salieron los dos a pie hacia el castillo, distante sólo del pueblo
veinte minutos de marcha cómoda, y situado en un mogote de mediana
elevación entre el río y el camino de Bot. Esqueleto de muros
despedazados, recompuestos y vueltos a despedazar por sucesivas guerras,
era el tal castillo, festoneado de hiedras y jaramagos, y conservando en
algunas de sus gastadas piedras cruces y escudos de San Jorge de Alfama.
Ponían el pie los dos caballeros en el primer cerco de ruinas, cuando
Nelet, asaltado de súbito terror, se paró y dijo a su amigo: «Pienso,
señor D. Beltrán, que Marcela se nos habrá anticipado, llegando aquí por
alguna galería subterránea que comunica esta fortaleza con el monasterio
de San Salvador. La encontraremos más adentro, y es tal mi miedo de
verla, o de que ella vea mi cara y mis ojos, que me clavo en tierra sin
poder dar un paso hacia adelante.»

Trató Urdaneta de devolverle la tranquilidad, negando que hubiese tales
conductos por donde pudiera la monja presentarse, al modo teatral y
fantástico, y le indujo a no ser temeroso y afrontar con varonil aplomo
la entrevista. Mas advirtiendo en él señales de mayor pánico, antes que
de entereza, le dijo: «Y en suma, si no puedes vencer tu aprensión, y
persistes en que le hable yo primero y explore su ánimo, manifestándole
la verdad que tanto temes, retírate a Horta y déjame aquí en espera de
la señora penitente y de los viejos Zaida y Alfajar. Pero ten la bondad
de conducirme a un sitio donde pueda yo sentarme, que apenas veo, y no
acierto a llevar mis pobres huesos por entre tanto pedrusco.» Condújole
Nelet, evitando tropezones, a un lugar despejado con buen asiento, y más
medroso cuanto más avanzaba, le faltó tiempo para escabullirse, diciendo
a su amigo: «Parece que la siento ya\ldots{} como si subiera por un
pozo\ldots{} Me voy al pueblo. Allí espero mi sentencia\ldots{}
Adiós\ldots»

\hypertarget{xxxi}{%
\chapter{XXXI}\label{xxxi}}

Quedose D. Beltrán solito en las ruinas, lo que no era muy divertido
para el pobre señor, pues el frío de la mañana le obligaba a requerir su
abrigo, envolviéndose bien en el capote que le había traído Nelet. No
menos de una hora estuvo rezando, atacado también de vagos temores,
semejantes a los de Santapau, y a cada instante creía sentir blandos
ruidos que le parecían el roce del sayal de la monja contra las piedras.
«No---se decía,---no sentiré roce de vestidos ni de pisadas. Veré
aparecer primero la cabeza, después los hombros, y sin hacer ruido
alguno se me pondrá delante\ldots» No veía nada el buen caballero; pero
vio amanecer, y distinguió los telones de piedra desgarrados, en el
centro de los cuales se encontraba; y cuando reconocía con su menguada
vista la decoración, oyó voces efectivas, sílabas vibrantes de mujer y
catarrosas de hombres, y\ldots{} era ella, sí, Marcela, seguida de los
enterradores, que aparecían por un hueco de los muros\ldots{} «Aquí
estoy, hija mía,» gritó el anciano gozoso, sin miedo ya. Ligera como una
corza saltó la beata por entre las piedras, y fue a besarle la mano al
prócer, que besó también la de ella. Entre beso y beso dijo la
penitente: «¿Y Nelet?

---Hija mía, no te asustes---replicó D. Beltrán, pesaroso de la mentira
venial a que le obligaban las circunstancias.---Está bueno; pero tan
delicado en su convalecencia, que no le he permitido abandonar el lecho
antes del día. En Horta le dejé, y allá nos vamos en cuanto tú y yo
descansemos. Como se fijó este lugar para nuestro encuentro, he venido
yo solito para que no creyeras que faltábamos a la cita.

---Pudo usted mandar a \emph{Malaena} y evitarse este madrugón, que no
le sentará bien. A su edad, señor mío, no hay que jugar con la salud.

---Verdad, sí\ldots{} pero\ldots{} no mandamos a la vieja\ldots{}
porque\ldots{} verás---dijo Urdaneta tartamudeando, pues se le
atragantaba la nueva mentira venial que le exigía la
situación\ldots---\emph{Malaena} se nos puso anoche mala de un
cólico\ldots{} de tanta butifarra como comió, la pobre\ldots{} Pues
descansemos y hablemos un poquito antes de bajar al pueblo\ldots{}
Siéntate a mi lado\ldots{} Más cerca\ldots{} así. Desde Vallivana no te
hemos visto. Hora es ya de que resuelvas. El pobre Nelet espera tu
determinación. ¿Has pesado bien el pro y el contra?

---Se asombrará usted---dijo Marcela, vacilando en las primeras
declaraciones,---y quizás me tache de ligera\ldots{} pero no es
ligereza, no señor\ldots{} cuando me oiga\ldots{} no sé cómo
expresarlo\ldots{} Pues bien: sabrá que han ahondado en mi ánimo las
razones de mis dos amigos y el rendimiento y constancia del pobre Nelet.

---¡Ah, qué felicidad!\ldots{} Yo esperaba\ldots{} en efecto\ldots{}

---Y a esta mudanza de mi voluntad, creo firmemente que no es extraña la
voluntad de Dios\ldots{} Divina es a mi parecer la voz que me incita a
querer a Nelet, y a cambiar de vida y vocación\ldots{} Por santo tengo
el matrimonio\ldots{} sus votos severos y sus obligaciones nos llevan a
una vida eficaz\ldots{}

---Es cierto. ¿Y has consultado el caso con tu confesor?

---Sí señor, y me ha dicho que dedicando a una fundación religiosa parte
del caudal de mi padre, si sentía honrada inclinación a la vida secular,
la adoptase, previas las dispensas de Roma, teniendo en cuenta el
trastorno que nos traen estas guerras y revoluciones.

---¿Y consultaste con tu hermano Francisco? Ante todo, sabrás que la
noticia de su muerte me ha llegado al alma. Eres ya la única
descendiente de Juan Luco, y este hecho debe pesar en tus
resoluciones\ldots{} ¿Tuviste tiempo de consultar con tu hermano el caso
extrañísimo de tu cambio de vida?

---¡Ay! sí señor\ldots{} y mi pobre hermano, que sabía desentrañar lo
presente y lo futuro, me aconsejó que abrazase el nuevo estado, pues si
grave es el quebrantamiento del voto, debíamos mirar también a la
conservación de los bienes de nuestro padre, así raíces como en especie,
recogiendo los que aún están esparcidos, y librándolos de la perdición.
Díjome que él en las propias circunstancias que yo se encontraba, pues
habiendo topado en término de Falset con una honesta, discretísima y
bella joven, nacida de noble familia, y prendándose de ella, creía que
este suceso era como aviso de Dios, con que le mandaba trocar una
vocación por otra; y así, era su propósito no pensar más en vida de
claustro, y adoptar las penitencias y dura regla de matrimonio con
aquella bendita niña de Falset. Largamente hablamos de nuestro negocio,
y él expuso ideas tan juiciosas, que parecen dictadas de la misma
sabiduría. Pensaba que debíamos apartar un tercio del caudal específico
de nuestro querido padre para consagrarlo a una fundación pía, y que con
los otros dos tercios y los bienes raíces, equitativamente partidos,
podríamos constituir dos familias cristianas, dedicadas a servir a Dios
y a perpetuar el nombre y patrimonio de Luco. Declaró también que de
estos dos tercios metálicos debíamos, en conciencia, retirar una suma
para dar cumplimiento a la moral obligación contraída por mi padre con
su grande amigo y protector D. Beltrán de Urdaneta, fijando, de acuerdo
con este, la cifra prudencial para tan sagrado objeto\ldots{}

---¿Eso dijo?\ldots{} ¡Oh Providencia, oh divina equidad!---exclamó el
viejo, sintiendo que un rayo penetraba en su alma,
trastornándola.---Bien, hija, bien\ldots{} Pero dime otra cosa: ¿tenía
Francisco conocimiento de la pasión que has inspirado a Nelet?

---Ya lo sabía, pues en los comienzos de nuestra conversación se lo
dije. Su parecer fue que, si yo gustaba de Nelet, le aceptase, pues
tiene fama de valiente y leal, aunque algo arrebatado, y posee bastante
hacienda en Cherta y Cambrils.

---¡Eso te dijo!\ldots{} ¿Estás segura de que tal era su pensamiento?»

A las manifestaciones afirmativas de la monja, contestó el anciano con
nuevas alabanzas del poder de Dios. El pobre señor veía más claro;
recobraba la vista, y en su turbación no sabía por qué caminos llevar la
interesante conferencia. Por fin, salió del paso con esta pregunta:
«¿Cuántos días antes de morir te dijo tu hermano lo que acabas de
manifestarme?

---Dos días. Después, el pobrecito siguió a su ejército, y la tarde
misma de la batalla de Gandesa, volviendo con otros veinte de cumplir
una orden del General, fue sorprendido por una partida de facciosos en
retirada, y le asesinaron con saña, vileza y cobardía.

---¡Oh, qué desgracia!\ldots{} Y sabiendo su triste fin, sin duda por
los compañeros suyos que lograron escapar, ¿cómo no supiste quién
dispuso y consumó hazaña tan inicua?

---Dijéronme que un capitán o no sé qué, cabeza de aquellos sayones,
traspasó a mi hermano con su espada.

---¿De modo que no sabes\ldots?

---No, señor: no lo sé.

---Y si conocieras al matador, ¿le perdonarías?

---¡Oh! como cristiana tendría que perdonarle; como cristiana,
señor\ldots{} ¿Acaso lo que yo ignoro lo sabe mi D. Beltrán\ldots?»

Como anillo al dedo venía en aquel punto de la entrevista la temida,
pavorosa revelación; mas el noble caballero, Señor de tantas torres, no
se atrevió a sacarla del pensamiento a los labios. Era hombre: careció
del valor necesario para un acto que requería verdadera santidad.
Habíase propuesto ser bueno, purificar sus últimos días con virtuosas
acciones; mas no era santo, no: no lo era.

«¿Lo sabe usted?» repitió Marcela espantada de su silencio.

Y D. Beltrán, sintiéndose a cien mil leguas de la cristiana perfección,
dijo en un grave suspiro: «Hija mía, no sé nada.»

Apareció en aquel instante Santapau por entre el hueco de unas altas
piedras, y bajando de un brinco, como sillar desplomado con estruendo,
gritó: «Sí lo sabe; mas no tiene valor para decirlo.» Marcela se levantó
bruscamente como un ave que quiere emprender el vuelo, y saltando sobre
piedras, se alejó despavorida.

«Ven, Marcela, ven\ldots{} no huyas---dijo Nelet.

---¿Cómo vienes aquí?\ldots---balbució la penitente con sílabas
entrecortadas.---¿Por qué vienes así, en esa forma, que más que de
hombre es de demonio?

---Porque lo soy. Demonio del Infierno es quien dio villana muerte a
Francisco Luco. Nuestro amigo no tiene valor para decirlo: lo tengo yo.»

Horrorizada, Marcela se llevó las manos a las sienes, volviendo la
cabeza. Luego cayó de rodillas.

«Levántate---dijo Nelet acudiendo a ella.---Yo soy el que debe
humillarse. Humillado te diré que, aunque no merezco tu perdón, lo
solicito, lo quiero\ldots{} Fue una ceguera, embriaguez de
sangre\ldots{} el maldito hábito de esta guerra, el matar por
matar\ldots{} por destruir vidas contrarias\ldots{}

---¡Perdón, perdón!---exclamó D. Beltrán, también de rodillas, llorando
como un niño.

---¡Monstruo---dijo Marcela encorvada, las manos en la cabeza, mirando
de soslayo torvamente al infortunado guerrero,---monstruo de
maldad!\ldots{} Como cristiana, te perdono. Pero huye, vete al fin de la
tierra, o donde yo no te vea más\ldots{} Condenado, no quiero condenarme
contigo\ldots{} tus miradas corrompen\ldots{} Yo no quiero verte ni
respirar el aire que respiras.

---¡Paz, paz!\ldots---repetía el buen Urdaneta alargando sus flacos
brazos.---Hijos míos\ldots{} sed cristianos\ldots{} No habléis de
condenaros. Salvaos, salvémonos todos.»

Huyó Marcela, y tras ella, saltando de piedra en piedra, corrió Nelet,
como anhelante cazador.

«No te acerques a mí---gritaba la monja.---Condénate tú solo; yo no.»

Poseído de insano furor, Nelet dijo: «Solo no. No más soledad. Tú
conmigo\ldots» Y viendo a la desdichada mujer buscar refugio tras unas
altas piedras, como res acosada que se esconde, allí la persiguió, y
allí, antes que los atontados viejos pudieran acudir en defensa de su
maestra y señora, le dio bárbara y pronta muerte. Retumbó el pistoletazo
en la tristísima cavidad del castillo como si todas sus piedras de golpe
se derrumbaran. Sobrecogido, exánime, el rostro contra el suelo, D.
Beltrán dijo: «Nelet, ¿qué haces?\ldots» Pasados algunos segundos de
pavoroso silencio, oyó el anciano la respuesta, que fue otro tiro no
menos estruendoso y lúgubre que el primero.

Los pobres sepultureros, a quienes el estupor y su propia debilidad
senil paralizaron en la fugaz duración de la tragedia, no supieron ni
aun requerir sus azadones para impedirla. Al primer tiro, cayó Alfajar
de espaldas con temblor epiléptico. Zaida, más animoso, blandió su
herramienta de sepultar, abalanzándose hacia Nelet con móvil de venganza
o justicia; mas no pudo anticiparse al criminal, que la hizo rápida y
eficaz con su propia mano.

Transcurrido un lapso de tiempo, que ninguno de los tres ancianos
apreciar podía, Zaida se llegó a D. Beltrán, y tocándole en el hombro,
con angustiada voz le dijo: «Señor, señor, ¿vivimos o morimos?

---No sé, amigo---replicó el caballero, despegando del suelo su
rostro.---¿Vives tú? ¿Qué es esto?\ldots{} Dame la mano: probaré a
levantarme\ldots{} ¡Ay! la juventud perece\ldots{} a sí misma se
destruye. Nosotros, tristes despojos de la vida, aún respiramos\ldots{}
¿Y para qué? El siglo no quiere soltamos, ¡ay de mí!

---Señor, nuestro deber ahora no es otro que abrir dos hermosas
sepulturas\ldots{}

---Amigo, no: abramos una sola, hermosísima, y enterrémosles juntos.»

Todo el día permanecieron los tres ancianos en el lugar de la tragedia;
y cuando se retiraban, al caer de la tarde, consternados y llorosos,
oyeron lejano bullicio de clarines y tambores. A medida que iban
venciendo con lento andar el camino de Lledó, arreciaba el marcial
rumor. A la puesta del sol, Zaida, que era de los tres el que gozaba de
mejor vista, distinguió por Oriente, en las áridas colinas de la margen
del río Seco, líneas de gente armada, las cuales avanzaban ondulando
como serpientes en las curvas del terreno, mitad en sombra, mitad en
luz. Eran las mesnadas de vanguardia de la expedición Real que marchaban
hacia la frontera de Aragón.

\flushright{Santander (San Quintín), Abril-Mayo de 1889.}

~

\bigskip
\bigskip
\begin{center}
\textsc{fin de la campaña del maestrazgo}
\end{center}

\end{document}
