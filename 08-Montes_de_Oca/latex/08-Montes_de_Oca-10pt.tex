\PassOptionsToPackage{unicode=true}{hyperref} % options for packages loaded elsewhere
\PassOptionsToPackage{hyphens}{url}
%
\documentclass[oneside,10pt,spanish,]{extbook} % cjns1989 - 27112019 - added the oneside option: so that the text jumps left & right when reading on a tablet/ereader
\usepackage{lmodern}
\usepackage{amssymb,amsmath}
\usepackage{ifxetex,ifluatex}
\usepackage{fixltx2e} % provides \textsubscript
\ifnum 0\ifxetex 1\fi\ifluatex 1\fi=0 % if pdftex
  \usepackage[T1]{fontenc}
  \usepackage[utf8]{inputenc}
  \usepackage{textcomp} % provides euro and other symbols
\else % if luatex or xelatex
  \usepackage{unicode-math}
  \defaultfontfeatures{Ligatures=TeX,Scale=MatchLowercase}
%   \setmainfont[]{EBGaramond-Regular}
    \setmainfont[Numbers={OldStyle,Proportional}]{EBGaramond-Regular}      % cjns1989 - 20191129 - old style numbers 
\fi
% use upquote if available, for straight quotes in verbatim environments
\IfFileExists{upquote.sty}{\usepackage{upquote}}{}
% use microtype if available
\IfFileExists{microtype.sty}{%
\usepackage[]{microtype}
\UseMicrotypeSet[protrusion]{basicmath} % disable protrusion for tt fonts
}{}
\usepackage{hyperref}
\hypersetup{
            pdftitle={MONTES DE OCA},
            pdfauthor={Benito Pérez Galdós},
            pdfborder={0 0 0},
            breaklinks=true}
\urlstyle{same}  % don't use monospace font for urls
\usepackage[papersize={4.80 in, 6.40  in},left=.5 in,right=.5 in]{geometry}
\setlength{\emergencystretch}{3em}  % prevent overfull lines
\providecommand{\tightlist}{%
  \setlength{\itemsep}{0pt}\setlength{\parskip}{0pt}}
\setcounter{secnumdepth}{0}

% set default figure placement to htbp
\makeatletter
\def\fps@figure{htbp}
\makeatother

\usepackage{ragged2e}
\usepackage{epigraph}
\renewcommand{\textflush}{flushepinormal}

\usepackage{indentfirst}

\usepackage{fancyhdr}
\pagestyle{fancy}
\fancyhf{}
\fancyhead[R]{\thepage}
\renewcommand{\headrulewidth}{0pt}
\usepackage{quoting}
\usepackage{ragged2e}

\newlength\mylen
\settowidth\mylen{...................}

\usepackage{stackengine}
\usepackage{graphicx}
\def\asterism{\par\vspace{1em}{\centering\scalebox{.9}{%
  \stackon[-0.6pt]{\bfseries*~*}{\bfseries*}}\par}\vspace{.8em}\par}

 \usepackage{titlesec}
 \titleformat{\chapter}[display]
  {\normalfont\bfseries\filcenter}{}{0pt}{\Large}
 \titleformat{\section}[display]
  {\normalfont\bfseries\filcenter}{}{0pt}{\Large}
 \titleformat{\subsection}[display]
  {\normalfont\bfseries\filcenter}{}{0pt}{\Large}

\setcounter{secnumdepth}{1}
\ifnum 0\ifxetex 1\fi\ifluatex 1\fi=0 % if pdftex
  \usepackage[shorthands=off,main=spanish]{babel}
\else
  % load polyglossia as late as possible as it *could* call bidi if RTL lang (e.g. Hebrew or Arabic)
%   \usepackage{polyglossia}
%   \setmainlanguage[]{spanish}
%   \usepackage[french]{babel} % cjns1989 - 1.43 version of polyglossia on this system does not allow disabling the autospacing feature
\fi

\title{MONTES DE OCA}
\author{Benito Pérez Galdós}
\date{}

\begin{document}
\maketitle

\hypertarget{i}{%
\chapter{I}\label{i}}

En los cuarenta andaba el siglo cuando se inauguró (calle de la Abada,
número \emph{tantos}) el comedor o comedero público de Perote y
Lopresti, con el rótulo de \emph{Fonda Española}. No digamos, extremando
el elogio, que fue el primer establecimiento \emph{montado} en Madrid
según el moderno estilo francés; mas no le disputemos la gloria de haber
intentado antes que ningún otro realizar lo de utile dulci, anunciándose
con el programa de la bondad unida a la baratura, y cumpliendo
puntualmente, mientras pudo, su compromiso. La exótica palabra
restaurant no era todavía vocablo corriente en bocas españolas: se
decía\emph{fonda} y \emph{comer de fonda}, y \emph{fondas} eran los
alojamientos con manutención y asistencia, así como los refectorios sin
pupilaje. Es forzoso reconocer que si nuestros antiguos bodegones y
hosterías conservaban la tradición del comer castizo, bien sazonado y
substancioso, los italianos, maestros en esta como en otras artes,
introdujeron las buenas formas de servicio y un poco de aseo, o sus
apariencias hipócritas, que hasta cierto punto suplen el aseo mismo. No
fue tampoco reforma baladí el sustituir la lista verbal, recitada por el
mozo, con la lista escrita, que encabezaban los \emph{ordubres},
estrambótica versión del término \emph{hors d'oeuvre}. Lo que
principalmente constituye el mérito de los italianos es la introducción
del precio fijo, la regla económica de servir buen número de platos por
el módico estipendio de doce reales, pues con tal sistema adaptaban su
industria a la pobreza nacional, y establecían relaciones seguras con un
público casi totalmente compuesto de empleados y militares de mezquino
sueldo, de calaveras sin peculio, o de familias que empezaban a gustar
la vanidad de comer fuera de casa en días señalados o conmemorativos.

Para dar a cada uno lo que le corresponde con imparcial criterio
histórico, conviene indicar que no fueron Perote y Lopresti verdaderos
innovadores en materia y formas de comer, sino más bien los que
divulgaron aquel arte precioso en la vida de los pueblos. Ya Genieys
había dado a conocer las \emph{croquetas}, los asados un poquito crudos,
las chuletas a la \emph{papillote} y otras cosillas; pero Lopresti
popularizó estos manjares poniéndolos al alcance de los bolsillos
flacos, acreditando su saber, así como la equidad paternal de sus
precios. Al propio tiempo superaba a Genieys en los arroces a la
valenciana y milanesa, así como en el bacalao en salsa roja; era maestro
en el cordero con guisantes, en el besugo a la madrileña, en la
pepitoria, en los macarrones a la italiana, y principalmente en los
guisotes de pescado y mariscos a estilo provenzal o genovés. En el
renglón de vinos, el poco pelo de la clientela limitaba el consumo a los
tintos de Arganda o Valdepeñas para pasto, y un Jerez familiar y
baratito para los libertinos domingueros, y para los que iban de
jolgorio, con mujerío o sin él, a horas avanzadas de la noche. En estas
francachelas de un carácter confianzudo y pobretón, no se conocía el
\emph{champagne}. El agua, de que algunos parroquianos hacían
considerable gasto, se anunciaba como de la Fuente del Berro; mas era de
la Academia o de la Escalinata. En el servicio de vinajeras introdujeron
los italianos cristalería fina en armaduras elegantes, y presentaban los
mondadientes en gallitos y monigotes de porcelana. Inferior era el lujo
en la mantelería y lienzos de mesa, de dudosa blancura los más días del
año.

Por todo ello tuvo la \emph{Fonda Española} un éxito tan rápido como
lisonjero, y el público invadió desde los primeros días el modesto y
lóbrego local de la calle de la Abada, recinto que aún conservaba olor y
trazas de logia masónica, piso bajo con dos rejas a la calle y entrada
por el portal. Era éste ancho, con zócalo de azulejos negros y blancos
como tablero de ajedrez, bien alumbrado a prima noche por un farolón de
dos mecheros, obscuro a última hora y expuesto a tropezones, que a veces
eran graves, sin contar el desagradable quién vive de las humedades
mingitorias. Adoptaron los dueños, porque no podía ser de otro modo si
habían de tonificar el establecimiento, el horario francés, dando la
comida fuerte por la noche, con supresión de cocido. Al mediodía,
servían almuerzos de seis y ocho reales, con huevos fritos y uno o dos
platos, y el invariable postre de pasas y almendras con añadidura de un
bollito de tahona, régimen que las casas huéspedes han perpetuado como
una institución hasta nuestros días, y será preciso un golpe de
revolución para destruirlo.

Fue uno de los primeros fundadores de la clientela el benemérito D. José
del Milagro, que, aunque cesante en todo el tiempo que vivieron los dos
Gabinetes moderados presididos por D. Evaristo Pérez de Castro, habíase
agenciado algunos modos de vivir, honradísimos, y podía permitirse
almuerzos de seis reales, y comiditas de ocho. Como tributo a una firme
amistad antigua, los italianos le concedían rebajas discretas y abríanle
créditos de una y de dos semanas, confiando en que el agraciado
guardaría reserva sobre este privilegio para no desmoralizar a la
parroquia. Debe advertirse aquí, para evitar juicios temerarios acerca
de aquel digno sujeto, que estaba viudo desde el 38; que una de sus
hijas, notable arpista, se había casado con un bajo italiano de la
compañía de la Cruz, la otra con un subteniente de la Guardia Real, y
que los chicos menores vivían en Illescas con su tía Doña Tránsito.
Campaba, pues, el buen hombre por sus respetos, y ganándose la pitanza
con traducciones de leyendas históricas o de historias poéticas, y con
tareas de contabilidad, vivía suelto, libre, en solitaria y a veces
triste independencia, viendo venir las cartas políticas, esperando la
ruina del llamado \emph{Moderantismo} y el triunfo del \emph{Progreso},
que debía llevarle a la holgura y descanso de la Administración. En
cuantito llegara el \emph{Progreso}, y agarraran la sartén sus ilustres
prohombres, nadie podía disputarle a Milagro su placita de diez y ocho
mil, digno premio del fervor consecuente, acendrado, incorruptible con
que había defendido siempre las libertades públicas.

Correspondía Milagro a la generosidad de los italianos corriendo la voz
de la excelencia y baratura del establecimiento, y a los pocos días ya
eran feligreses D. Víctor Ibraim, castrense del 2.º de la Guardia, y uno
de los hermanos Fonsagrada, teniente del 4.º, con otros individuos de
que se dará conocimiento. El más calificado entre estos era un D. Bruno
Carrasco y Armas, manchego de buena sombra, de insaciable apetito y de
mucha correa en el discurso, que llevaba cuatro años en Madrid
gestionando la resolución de un embrolladísimo expediente de Pósitos;
hombre que pasaba por rico y que lo acreditaba convidando
espléndidamente a los amigos cuando las esperanzas del pronto arreglo de
su negocio le ponían de buen temple. Siempre que almorzaban juntos
Milagro, Ibraim y Carrasco, se establecía entre los tres una feliz
comunidad de criterio para juzgar las cosas públicas. Unánimes convenían
en el aborrecimiento del régimen imperante, persuadidos de que la viuda
de Fernando VII era la mayor calamidad arrojada por Dios sobre las
pobres Españas.

A todos excedía Milagro en la firmeza de su convicción y en el ardor con
que últimamente la manifestaba. Aquel hombre sin ventura, a quien
hicieron escéptico las turbaciones políticas; aquella víctima, aquel
mártir que había sufrido con admirable resignación los desastres que al
individuo y a la familia ocasiona todo cambio de gobierno, llegó a
comprender que la neutralidad y la falta de convicciones son la mayor de
las desventajas en el orden social, y que por tal camino, por lo mismo
que es el más derecho, no se va a ninguna parte. Sus dolorosas
cesantías, sus hambres y escaseces demostráronle la necesidad de poseer
un temperamento vivo, ya sea real, ya figurado, para no quedarse a la
cola en el movimiento general. El manso, el prudente, el descreído que
se planta y espera, es arrollado por la multitud que avanza ciega y
ardorosa. Sentó plaza, pues, el buen Milagro, curado al fin de su insana
neutralidad, en las falanges del \emph{Progreso}, y se puso en las filas
de vanguardia, enarbolando, si no la bandera, el primer trapo de
colorines que encontró a mano.

Una noche de Julio convidó el manchego sin tasa, agregando Jerez y
licores, no ciertamente porque tuviera buenas noticias de su asunto,
sino porque las tenía detestables, y la desesperación le indujo a echar
la casa por la ventana, difiriendo sus esperanzas y colocándolas en el
día no lejano del triunfo de los \emph{libres}. En la boca y en el
corazón de los amigos reverdecieron las tales esperanzas con el contento
que dan el buen comer y un beber abundante a costa de generoso
anfitrión. Al segundo plato el gozo era inefable, a los postres
vocinglero. Los roncos acentos de Ibraim y su ceceo bárbaro llenaban la
sala expresando las ideas más audaces, con escándalo de algunas orejas
timoratas. De pronto se levantó un vejete que con tres individuos comía
en una mesa lejana, y llegándose a la del manchego, insinuó una protesta
en tono humorístico un tanto destemplado. Véase la muestra: «Oí patadas
y dije: `caballería tenemos'. Señores, se les saluda. ¿Qué hablan
ustedes ahí de Reinas y Ministerios, ni qué entienden de esto los
caballeros del margen?\ldots{} Y usted, señor de Milagro, no se agazape
ni vuelva la cabeza, que ya le he conocido, y sus facciones, aunque hace
un siglo que no nos vemos, no se me despintan. No vale, no, hablar mal
de los moderados, después de haber comido con ellos a mandíbula
batiente. ¿Pues qué quería usted, alma de Dios? ¿Que le tuvieran
colocado toda la vida, y encima\ldots{} le nombraran canónigo? ¿No han
de comer los demás? ¡A fe que hay pocos padres de familia entre los
moderados, con seis, siete y hasta doce criaturas!\ldots{} Hoy les toca
el pesebre a los morenos, mañana a los blancos\ldots{} Si usted quería
pan perpetuo, ¿por qué no aprendió un oficio, como lo aprendí yo, que a
los catorce años ya me ganaba un cocido trabajando en la orfebrería con
mi amigo Leandro Moratín? ¡Ja, ja, pues no me sale usted ahora con pocos
humos!\ldots{} ¿Qué espera mi hombre del \emph{Progreso}? Tonto, más que
tonto: pida limosna antes que limpiarle las botas a Linaje, y no se fíe
de Espartero, que repartirá todos los piensos, digamos destinos, entre
los animales manchegos, o sea los vecinos de Granátula. Esto lo veo
yo\ldots{} ¡ja, ja\ldots{} y el que no lo vea es porque tiene ojos en la
cara, no en el entendimiento\ldots{} ja, ja!»

---No le había conocido, Sr.~D. Carlos Maturana---dijo Milagro adoptando
el tono zumbón, después de pintar en su rostro, en sucesivas
expresiones, la sorpresa, el enojo y la hilaridad.---Con esas barbas que
se ha dejado, da usted el pego a sus buenos amigos.

---No me disfrazo para conspirar, como usted, ni uso bigote de moco para
adular al Duque.

---No adulo\ldots{} los pelos de mi cara siempre significaron
\emph{libertad}.

---Antes iba usted afeitado.

---Ya no, para no parecerme a los curas.

---Cuéntele eso a su compañero, el castrense que me oye.

---Este no es obscurantista.

---Ya; es retinto.

D. Víctor Ibraim echó mano a una botella. Acudió D. Bruno a contener la
ira del Capellán, y apaciguándole con un gesto y cuatro voces de lo más
crudo, volviose risueño hacia el diamantista y le ofreció una copa de
Jerez, acompañada la oferta de estas campechanas expresiones:

«Si me ha llamado usted animal, y recojo la alusión como hijo de
Granátula, aunque no pariente de D. Baldomero, yo le llamo a usted
zopenco, y con estos insultos terribles no hacemos más que pasar el
rato\ldots{} porque aquí venimos a pasar el rato, no a pelearnos por una
Reina ni por un General. Beba usted, y luego nos diremos cuatro
cuchufletas, si tiene humor de jarana. Estos amigos son
pacíficos\ldots{} Yo no he venido a Madrid a pedir un puesto en el
pesebre, sino a que me hagan justicia.»

---¡Justicia!---repitió Maturana empinando.---A eso vienen todos, y
luego\ldots{} En fin, señores, perdonen mi desenfado. Hablaba como
hablamos hoy todos los españoles, como un loco. No hagan caso: sin
quererlo, dice uno mil desatinos. ¡Feliz España si fuera la tierra de
los mudos! Sr.~Ibraim, si le llamé a usted retinto fue por pasar el
rato. Seamos amigos.

\emph{---Siéntese el buen Aguilera.}

---¿Qué hay de noticias?

---Nunca sé nada que sea de oposición\ldots{} Sólo sé que nuestra
excelsa Reina sigue su viaje triunfal por Cataluña, y que no faltará
quien le acuse las cuarenta al caballero de Granátula.

\hypertarget{ii}{%
\chapter{II}\label{ii}}

Entablaron luego coloquio amistoso: si la acción del Jerez lo encendía
más de la cuenta, no tardaba en enfriarlo D. Bruno arrojando en las
ascuas su buen sentido, su pasta conciliadora y un lenguaje hábil para
contentar a todos. Según Maturana, por el comunicado de Mas de las
Matas, que más bien era manifiesto, Espartero merecía la destitución, y
Linaje cuatro tiros. Cierto que no había un Gobierno bastante fuerte
para ponerle el cascabel al gato\ldots{} Un hombre existía con hígados
bastantes para arrancar el bastón de manos del Duque; un hombre, sí, de
grande ánimo y convicciones profundas: D. Manuel Montes de Oca; ¿pero
qué podía un solo individuo, por animoso que fuera, entre tantos que
creían resolver las cuestiones con discursos, con arreglitos y dimes y
diretes? ¡La conciliación! ¡Buena conciliación nos diera Dios! La
soberbia de Espartero no cabía dentro de las leyes, y era forzoso
resquebrajarlas para hacerle hueco.

Con no poca dificultad, tartamudeando y corrigiéndose a cada instante,
expresó el castrense andaluz opiniones enteramente contrarias a las del
diamantista. D. Manuel Montes de Oca no era más que un barbilindo que no
servía para nada. Sus habilidades consistían en componer versitos
clásicos de la escuela del Sr. Reinoso, y pronunciar discursos
acaramelados imitando a Martínez de la Rosa. Todos sus actos como
político y como escritor eran los de un Quijote chico que había tomado a
María Cristina por Dulcinea, y al moderantismo por ley de la andante
caballería. Esto lo dijo Ibraim con formas premiosas y groseras, que
traducimos al lenguaje usual para no afear con ellas estas páginas.

Con palabra más fácil, aunque algo entorpecida por el Jerez, hizo
Milagro el panegírico de Espartero llamándole libertador, pacificador y
\emph{apóstol de todos los adelantos}. ¿No había concluido la guerra, o
estaba a punto de concluirla? ¿No le debía España el completo exterminio
de las \emph{hordas de la reacción}? Pues suyo era el país, suyas las
leyes, suya la autoridad y todo aquello que llamamos \emph{cosa
pública}. Desde que el mundo es mundo, desde Moisés a Bruto, desde
Guillermo Tell a Cromwell, y desde Bonaparte a Espartero, el que ha
tenido la fuerza y la razón ha tenido \emph{la cosa pública} en el
bolsillo. ¿Para qué nos servía esa Reina, viuda de Fernando VII, casada
hogaño con un Muñoz, dama graciosa y bonita, cuya linda mano movía el
\emph{timón de la nave} como si este fuera el abanico? ¡Cuánto mejor
gobernaría Espartero, hombre de buen puño! El trono de Isabel necesitaba
un protector macho, y España un Regente bien bragado y de muchísimos
riñones. Que viniera pronto y colocara en sus puestos a los
\emph{funcionarios probos}, destituidos por la infame \emph{moderación}.
Viniera, sí, antes hoy que mañana, a traernos la justicia, eliminando de
las oficinas a los pancistas, intrigantes y gorrones, y dando la
merecida redención a los pobres mártires de la política.

Acogía Maturana con cascada risilla senil las manifestaciones egoístas
de su amigo, y el buen manchego, tomando muy en serio su papel
conciliador, discurría una componenda que sería felicísima si fuese
práctica. ¡Lástima grande que Doña Cristina hubiera incurrido en la
flaqueza de emparentar secretamente con Muñoz; lástima grande también
que Espartero se hubiera precipitado a desposarse con Doña Jacinta
Sicilia! Si uno y otro estuvieran solteros en aquel crítico momento de
la historia patria, con una simple boda se realizaría la felicidad de la
nación, afirmando la paz para siempre y repartiendo entre las dos
familias o bandos los puestos administrativos. Casado el \emph{Progreso}
con la Corona, se casaban y refundían todos los derechos, y comían todas
las bocas y se acababan todas las hambres; el contento general traería
la general justicia, y la hartura sería el fundamento de la felicidad;
no habría ya pronunciamientos, ni logias ni cadalsos, y daría gusto ver
cómo marchaban fácilmente los asuntos, cómo prosperaba el trabajo, cómo
hallaban su acomodo los pobres, y los acomodados la riqueza, y los ricos
la opulencia; daría gusto ver despachados en un periquete los
expedientes de arbitrios, los expedientes de Pósitos, los Pósitos,
¡Señor!, que eran la tela de araña en que se enredaban y perecían, como
pobres moscas, los hombres más honrados de la nación.

Soltó la risa con mayor estrépito D. Carlos Maturana, y levantándose se
volvió a la mesa de donde había venido. Su reír picante, recorriendo la
sala, era como si al andar se soltaran rodando por el suelo las cuentas
de un rosario. Un tanto corrido, dirigía Milagro hacia la distante mesa
los cristales de sus gafas; mas como era tan cegato, ni aun con los
vidrios podía distinguir a los dos comensales del diamantista, a quienes
este comunicaba su risa burlona.

«Dígame, Ibraim---preguntó al capellán:---¿conoce usted a esos tipos que
comen o han comido con D. Carlos?»

---El que ahora se burla de nosotros---replicó D. Víctor,---no es para
mí cara desconocida. Le he visto mil veces; me han dicho su nombre; pero
en este momento no puedo traerlo a mi memoria. El muy sinvergonzonazo se
ríe en nuestras barbas mirándonos con un ojo solo, porque es tuerto.

---Ya, ya le conozco---dijo el manchego:---es ese poeta\ldots{}
demonches\ldots{} autor de una comedia que la llaman \emph{Moríos y
vereislo}.

---¿Poeta, tuerto\ldots{} \emph{Muérete y verás}?---exclamó el buen
Milagro dando un palmetazo en la mesa.---Bretón de los Herreros.

Presuroso y también tocado de risa, corrió a la mesa del rincón más
distante, y acogido por el poeta con un apretón de manos, oyó estas
palabras de cordial benevolencia:

«También aquí disputamos, también nuestra mesa es un campillo de
Agramante, o Cortes en miniatura, con izquierda y derecha, oposición y
mayoría. Maturana y yo somos el orden establecido, \emph{vulgo}
Ministerio, y este señor\ldots»

En un paréntesis hizo el poeta la presentación de su amigo, un joven
alto, moreno, de rostro varonil y hermoso, que denunciaba la profesión
de las armas, disimulada por el traje civil: «Mi amigo, casi paisano y
casi pariente, D. Santiago Ibero, teniente coronel de los Ejércitos
Nacionales, propuesto ya para coronel\ldots{} Fabulosa carrera, pero
bien ganada; que éste no es de los de farsa.»

---¡Vivan los héroes---vociferó Milagro,---que nos han librado al fin de
esa plaga indecente de la facción! ¡Ibero!\ldots{} un nombre que no
falla. Llamándose así no hay más remedio, señor mío, que ser español
valiente y liberal.

---Lo que decía---continuó Bretón.---D. Carlos y yo somos en esta mesa
el pobre Gobierno, y Santiago \emph{los señores de enfrente}. Figúrese
usted si estará forrado de liberalismo el niño este, que ha sido y es el
brazo derecho de Zurbano. Un cuerpo cubierto de heridas y una cabeza de
viento. Ya me lo dirá, ya me lo dirá cuando los años le amansen el
genio, y cuando vea\ldots{} porque todo es cuestión de ver pasar cosas y
personas, reinados y gobiernos, tiranías y revoluciones. ¿Qué edad
tienes, Santiago? ¿Treinta y dos? Ya me contarás tu \emph{progresismo}
cuando rebases de los cuarenta, si es que yo puedo alcanzar el tiempo de
tus desengaños, pues la vida que llevamos los españoles no es para
llegar a viejos. Sólo los que se pasan el día y la noche politiqueando,
como este Milagro, realizan el de vivir mucho, porque con todos comen, y
en todas las salsas mojan su mendruguito.

---Pido la palabra. El pelo que ha echado un servidor de
ustedes---replicó el aludido,---bien a la vista está, y los frutos de
mis intrigas pueden calcularse por la opulencia en que vivo\ldots{}
Bromas a un lado, el Sr.~de Ibero nos dirá si podemos dar la guerra por
concluida, o si aún nos queda en Cataluña y Aragón algún rabo faccioso
que desollar.

---Atrasado está de noticias el amigo Milagro---dijo Maturana, echándole
familiarmente mano al cuello.---¿No sabe la noticia de esta tarde, la
retirada de Cabrera después de la paliza que le ha dado León en Berga?
Ya no hay guerra, señores; ya no hay más que política, lo que a mí no me
parece un grave mal, pues España es un enfermo que no puede vivir sino a
fuerza de sangrías\ldots{} No reírse. La política sola paréceme más
mortífera que la política con guerra. La una corrompe, la otra
purga\ldots{} En fin, los que vivan lo verán.

---Se acabó la facción\ldots{} ¡Viva Espartero!

---No cantemos victoria tan pronto---indicó Bretón guiñando el ojo con
malicia,---que en este bendito suelo, el último tiro de una guerra civil
es el primero de otra. Ya nos estamos preparando para un
pronunciamiento; que nuevas tropas, ¡vive Dios!, no es bien que estén
ociosas. ¿Verdad, Santiago, que os pronunciaréis?

Contestó Ibero gravemente que en el ejército del Norte y del Centro
nadie pensaba en insurrecciones, a menos que la libertad peligrara. «Ya
pareció aquello---manifestó el autor de \emph{Marcela}, acompañando su
dicho con toquecillos de tambor sobre la mesa.---Siempre que queréis
sublevaros nos habláis de los peligros que corre la señora Libertad, a
la cual yo comparo con la monja pudibunda que preguntó cuándo tocaban a
violar. Eso decís ahora vosotros, pillos, demagogos, jacobinos; eso
decís: `¡Que violan!'. Y os equivocáis, porque nadie ha pensado ni
piensa en atropellar la virtud de vuestra diosa. Aquí no viola nadie más
que vosotros, los liberales, que cada día os fumáis una ley más o menos
virgen.»

---D. Manuel---dijo Milagro, vivamente interesado en la cosa
pública,---déjese de bromitas y vamos al grano. Sr.~de Ibero, si no hace
mucho que ha venido usted del Maestrazgo, sabrá qué opiniones privan en
el Ejército, si seguiremos con la regencia una o la estableceremos
trina\ldots{}

---Yo no sé nada de eso---replicó el militar.---Allá no pensamos más que
en perseguir al enemigo.

---Que nos cuente sus hazañas---propuso el diamantista,---pues más debe
interesarnos un poquito de historia, por breve que sea, que todos los
chismes masónicos.

---No tengo hazañas que contar---afirmó Ibero, sacando la petaca y
ofreciendo puros, que todos aceptaron, menos Bretón.---Mis proezas no
han sido más que el cumplimiento de un deber sagrado, sin ninguna
función heroica ni cosa que lo valga. Estuve en las acciones de Segura y
Castellote, ambas muy reñidas. Me encontré en el sitio de Morella y en
los combates que hubimos de dar para posesionarnos de parte del país
circundante; pero no presencié la rendición de la plaza ni la fuga de
los carlistas, porque tuve que venir a Madrid con una comisión del
servicio\ldots{}

---Comisión de que no nos dirá una palabra, ni nosotros hemos de
fastidiarle con preguntas---apuntó Maturana.---Ya sabremos del
pronunciamiento cuando oigamos el primer berrido.

En este punto de la conversación, y mientras Ibero denegaba
festivamente, riendo y gesticulando, llegó el mozo con la botella de
Jerez, brindándoles \emph{de parte de los señores de la otra mesa}. Un
gesto campechano del diamantista y un llamamiento jovial de Milagro
produjeron la reunión de dos grupos; mas no cabiendo en una mesa, parte
de Milagro y la totalidad del corpacho de Ibraim, ocupaban la inmediata.
Una rápida presentación hecha por D. José, cantando los nombres, unió a
los seis individuos en accidental intimidad. El rumboso D. Bruno, que ni
a tiros quería soltar el lucido papel de anfitrión, mandó traer vino y
puros de a dos reales; rechazó Bretón el exceso de bebida, protestando
de su templanza, ya que hacerlo no podía de la de los demás; festejaron
Maturana y Milagro la esplendidez del conterráneo de D. Quijote,
abalanzó su ávida manaza Ibraim hacia los puros, y todos parecían
dispuestos a prolongar la placentera reunión hasta hora muy avanzada. Y
cuando por la retirada lenta de los parroquianos íbanse quedando solos
los seis puntos de la improvisada tertulia, gozosos de poder alborotar
un poquito si el cuerpo y los espíritus así lo pedían, dejábase ver
Lopresti con mandil y gorro blanco, saludando risueño a los señores con
su atiplada mujeril voz. Era en él costumbre salir, terminado el
trabajo, a recrearse oyendo las observaciones que sus feligreses le
hicieran sobre los platos del día, o las alabanzas de su maestría
culinaria. Acercose tímidamente dando las buenas noches, y Milagro, con
el sombrero echado atrás, la mirada fulgurante y el labio trémulo,
llegose a él y le ofreció una copa, diciéndole: «Ínclito Cayetano,
brinda por la libertad, por la regencia \emph{trinitaria}, por el Duque
nuestro padre, que a todos nos sacará del
Purgatorio\ldots{}\emph{Amén.»}

En tanto, interrogado por Carrasco, amplió Maturana las noticias
recientes: la Reina, después de ser recibida en Lérida por Espartero con
todos los honores de rúbrica, continuaba su viaje a Barcelona. Trabose
en seguida acalorada discusión de principios, llevando la voz D.
Santiago Ibero y D. Carlos Maturana por las ideas liberal y moderada
respectivamente. «Yo no entiendo de política---dijo el militar con
sinceridad y convicción;---no sé lo que son partidos, ni para qué
existen las logias; pero declaro que creo en la libertad y la tengo por
cosa excelente. Antes de haber leído lo mucho y bueno que sobre la
libertad han escrito hombres muy sabios, sentía yo en mi alma la fe de
esta idea, y con entusiasmo la adoraba. Antes que en mi entendimiento,
estuvo en mi corazón el deseo de que los pueblos fuesen libres. Amo a mi
Patria tanto como a mi familia y a mí mismo: quiero para ellos los
bienes del progreso. Alguno me hablará de los males que ocasiona: yo los
reconozco; pero los males son chicos y pasan, los bienes son grandes y
quedan. Creo que con libertad, igual para todos, tendremos ilustración,
dignidad, riqueza; sin libertad caeremos en la ignorancia, en la pobreza
y en la ignominia. Si esto es un disparate, no pierdan el tiempo en
demostrármelo, pues no hay razones que destruyan mi idea. Más que
convicción clara es esto fe ciega. Yo no discurro: creo. Yo siento; no
razono. Así soy, y así pido a Dios que me conserve.»

Murmullo de entusiasmo, en el cual el vocerrón de Ibraim y la voz
femenina de Lopresti formaban las notas extremas, acogió las palabras
del militar, que a fuer de sencillas y leales casi eran elocuentes.
Bretón se levantó, y abrazando a su amigo le dijo: «Te admiro,
Santiago\ldots{} y te compadezco. Adiós, hijo mío. Señores, divertirse.
Mi mujer me riñe si entro tarde.»

Maturana reservaba en lo profundo del pensamiento sus opiniones: antes
del discursillo de Ibero había reclinado su cabeza, haciendo almohada
con los brazos en el respaldo de la silla, y se quedó dormidito, como
una criatura a quien padres viciosos obligan a trasnochar.

\hypertarget{iii}{%
\chapter{III}\label{iii}}

Pasaron días. De nuevo aparecen en la \emph{Española} comiendo juntos
Carrasco y Milagro, y en una mesa próxima Ibero con un señor
desconocido. Una y otra vez los parroquianos fundadores se aproximaron
con llaneza cordial al caballero alavés, movidos de una simpatía
misteriosa. Dígase, para encontrar la explicación de tal sentimiento,
que movía sus corazones la confianza en las ideas que Ibero expresaba.
El fatigado pretendiente y el viejo cesante buscaban los rayos de un sol
que desde el momento de la aurora, y aun antes de ella, ya calentaba un
poquito. Maturana fue alguna vez con su sobrino, y gracias a este supo
mantenerse en una templanza que le quitaba todo su mérito de personaje
cómico \emph{per accidens}. Era, en el estado ordinario, un señor
apreciabilísimo, de una sensatez ejemplar y desabrida. A Bretón no se le
vio más por allí. Ibraim fue una noche con Fonsagrada, al cual se
juntaron luego dos sujetos de los llamados del bronce, acompañados de
una bulliciosa trinca de mozas alegres\ldots{} Corrieron más días. El
calor arreciaba; Madrid era un páramo ardiente sin agua, sin alegría,
sin placeres, ambiente apropiado a la desesperación y a la locura; el
Ministerio Pérez de Castro había sufrido nueva metamorfosis, echándose
por tercera vez tapas y medias suelas; en el quita y pon de Ministros,
sólo permanecía inmutable D. Lorenzo Arrazola, el conciliador
sempiterno; tenebrosa confusión reinaba en la cosa pública, y todo
anunciaba sucesos inauditos.

Una noche de aquel Agosto triste de Madrid, de aquel bochornoso mes casi
siempre precursor de tempestades en nuestro calendario histórico, comió
Ibero en la \emph{Española} con un capitán de la Guardia, y hallábanse
ya rematando el postre de pasas y almendras, cuando se presentaron
Milagro y el manchego, ya bien comidos al parecer, pues el uno traía
puro en la boca y el otro palillo, y llegándose a la mesa con aire
misterioso, dieron a entender a medias palabras que tenían que tratar
con el alavés de un asunto grave y delicadísimo. Para dar mayor
solemnidad a su mensaje, Carrasco propuso a Ibero que se dejase llevar
al rincón opuesto de la sala, vacío de gente, donde podrían secretear a
su gusto. No creyendo bastante reservado aquel sitio, hubiérale llevado
Milagro a la cocina, o a lugares más recónditos. Impaciente Ibero, y
tomando a broma los aspavientos de sus amigos, que parecían padrinos de
duelo o conspiradores de profesión, les incitó a explicarse pronto y con
menos arrumacos.

«Calma, señor mío, que ya le enteraremos con todo el sigilo que el caso
requiere.»

---En este ángulo, hablando bajito y con disimulo, como si tratáramos,
verbigracia, de una cuestión \emph{faldamentaria}, estaremos bien
seguros. El hecho es que\ldots{}

---Yo, yo\ldots---dijo D. Bruno reclamando la primacía.

---El caso es que\ldots{}

---Déjeme a mí, querido Milagro. Vamos por partes. No nos hagamos un
lío. Recordará el señor de Ibero que hace días le hablé de mi sobrino,
Modesto Gallo\ldots{}

---Sí, sí; grande amigo mío.

---Como usted, teniente coronel del Ejército.

---Subtenientes nos conocimos, y desde capitanes hemos peleado juntos,
haciendo vida común, compartiendo las penas y alegrías de la guerra, los
peligros de siempre, las glorias de algún día.

---Pues bien---dijo Carrasco con una solemnidad que casi era
terrorífica:---Modesto ha llegado.

---¿Aquí? ¿Dónde está? Quiero verle---exclamó Santiago con no menos
sorpresa que alegría.

---Poco a poco---indicó Milagro, queriendo llevar el espinoso asunto con
la pausa que su extrema gravedad requería.---Ha llegado ayer. Le hemos
visto esta noche.

---Y al tiempo de saludarnos nos ha preguntado si sabíamos de usted, y
dónde podríamos encontrarle.

---Yo también quiero verle, ¡caramba! ¿Dónde está?

---Calma; no perdamos la serenidad.

---Necesita hablar con usted esta noche\ldots{} ¡ojo!\ldots{} esta
noche.

---Vamos allá.

---Quieto\ldots{} No se entere la gente. ¿Ve usted? Ya nos miran.

---Esto es ridículo. Parecemos conspiradores.

---Chitón\ldots{}

---Prudencia, amigo mío.

---En fin, ¿dónde veré a Modesto? ¿Para en casa de usted o en alguna
posada?

---No sabemos dónde mora. Vino a mi casa con la urgencia de que le
buscáramos a usted esta noche\ldots{} sin falta.

---Tiene usted que verse con él inmediatamente---susurró Milagro en voz
tan baja que apenas se le oía.

---¿Pero si no sabemos dónde está, ¡Cristo!, cómo he de verle?

---Silencio. Usted le encontrará sólo con dirigirse sigilosamente y sin
comunicarse con nadie a donde yo le indique.

---¡Pues acabe usted de explicarme, ajo!\ldots{}

Adoptando las formas de disimulo más exquisitas, el manchego sacó de su
bolsillo un papel, un cartón, la mitad de una tarjeta, y presentándola a
su amigo con delicadas afectaciones de naturalidad, le dijo: «Con esto
se encaminará el Sr.~D. Santiago al sitio propuesto por mi sobrino. Yo
le diré la calle, número de la casa y piso\ldots{} y no hay que perder
tiempo, señor mío; despáchese usted\ldots{} pague la comida, despídase
de su amigo, sin darle a conocer este negocio, y vámonos a la calle.
Aquí no me atrevo a decirle lo que aún ignora.»

Obedeció Ibero, y una vez los tres en la calle obscura, desembuchó
Carrasco lo que del misterioso mensaje aún quedaba en su cuerpo. La casa
en que el teniente coronel Gallo citaba a su amigo era un piso segundo
en el número 13 del Postigo de San Martín.

«¿Y qué tengo yo que hacer allí?---dijo Santiago perplejo y de mal
talante.---Esto me huele a tapujo masónico\ldots{} Yo no soy masón, para
que ustedes lo sepan.»

---Ni yo---iba a decir D. Bruno; pero Milagro se apresuró a cortarle la
palabra con manifestaciones que, si no revelaban escuetamente las
fórmulas rituales del masonismo, eran de la casta más próxima.

---Tampoco nosotros; pero \emph{blasonamos} de liberales, queremos la
felicidad de la patria, y contribuimos en nuestra esfera humilde al
triunfo de los buenos.

---Nada, Sr.~de Ibero---declaró con austeridad Carrasco:---cuando mi
sobrino le llama a usted a ese punto, es porque se le necesita, es
porque\ldots{} se le estima útil, indispensable como quien dice. No se
trata de un cualquiera: se trata de un bizarro jefe, cuya autoridad
puede ser de gran peso\ldots{} en fin, yo no sé\ldots{} hablo por
corazonadas, pues Modesto nada me ha dicho\ldots{}

---Como si lo dijera---añadió el cesante.---Podrá ser que si usted no
acude a la cita, los acontecimientos sigan un curso\ldots{} es un
suponer\ldots{} un curso torcido, distinto del que anhelamos todos. Poco
tiene que andar el Sr.~Ibero, pues el Postigo de San Martín lo tenemos
aquí propiamente\ldots{} a dos pasos\ldots{} Le llevaremos hasta el
mismo portal del 13. Conozco la casa, que es la más antigua de la calle,
y en ella estuvo la panadería de los frailes allá en los tiempos
ominosos.

Ibero se dejó llevar. Si el cariño de su compañero le avivaba el paso,
se lo contenía el temor de lo desconocido y la sospecha de que le
llevaban a una encerrona para envolverle en alguna maraña política.
Recordaba el carácter de Gallo, un chico excelente, intrépido militar,
amigo intachable; pero de cascos muy ligeros, así en cuestiones de
mujerío como en las que atañen a la vida pública. De una
impresionabilidad excesiva, se remontaba fácilmente de los afectos a las
pasiones; su fácil palabra y su asimilación más fácil todavía fomentaban
en él los entusiasmos bruscos, el ardor sectario; no sabía querer sin
violencia, ni profesar opiniones sin llevarlas hasta el delirio. Tal era
Modesto Gallo, a quien Ibero reconocía en aquella forma novelesca de
darle cita con media tarjeta, con el sigilo teatral del mensaje confiado
a dos amigos. Por último, a los temores del alavés se sobrepuso la
curiosidad, y cuando se aproximaba con sus conductores al Postigo de San
Martín, ya se le hacían largos los minutos para llegar a la solución del
enigma.

«¿Será prudente que nos veamos a la salida?» preguntó el honrado
manchego, vacilando entre el miedo y la curiosidad.

---¡Sabe Dios---indicó Milagro alardeando de discreción,---si el Sr.~de
Ibero podrá contarnos\ldots! No, no: resoluciones tan graves no se
comunican ni al cuello de la camisa. Vámonos; no está bien que rondemos
la calle.

---¡Oh!, no\ldots{} ¡Podrían creer que nosotros\ldots! Vámonos de
aquí---dijo Carrasco sintiendo frío.

Era el temor de la persecución policiaca, que por primera vez en su vida
contristaba su ánimo; y no era sólo temor, sino repugnancia y algo que
ofendía su dignidad. ¡Verse él, español pacífico y acomodado, padre de
familia, señor de ganados y tierras; verse, pensaba, en trotes de
persecución, traído y llevado por guindillas inmundos! No: el papel de
víctima política, fuera por esta o la otra causa, no cuadraba, no, a su
hidalga condición. Amaba la libertad, mas que por propio conocimiento,
por lo que de tal señora había oído decir y contar; pero no sentía ganas
de martirio por ninguna religión política; sólo un gran ideal le movía:
el satisfactorio despacho de un triste expediente de Pósitos.

Al despedir al militar, quiso Milagro arrancarle la promesa de que
acudiría luego al café del Siglo; mas no consiguió sino una vaga
respuesta condicional sobre este punto. Retiráronse viéndole perderse en
el lóbrego zaguán, y avanzando silenciosos hacia la plazuela de las
Descalzas, Milagro imaginaba trastornos inminentes, a la medida de su
loco deseo, y Carrasco sentía, en medio del sofocante calor estacional,
ráfagas de frío, el sobresalto de la conciencia, que le afeaba sus
concomitancias con gente intrigante y revoltosa. Las sombras de la noche
aumentaban su recelo, y agarrándose al brazo de su amigo, aceleró el
paso para llegar pronto a la Puerta del Sol, que ya en aquellos tiempos
era lo menos obscuro y solitario del viejo Madrid. Para mayor
intranquilidad del buen compatriota de Sancho Panza, creyó ver desusado
movimiento en la Puerta del Sol, y grupos más compactos que de ordinario
frente al Principal. Quiso D. José aproximarse y meter sus narices en el
gentío; pero el manchego le llevó a empujones diciéndole: «Amigo mío, no
olvidemos que somos ciudadanos pacíficos, honrados\ldots{} quiero decir
que no nos metemos en quitar y poner ministros. Allá se las
hayan\ldots{} Adelante: allí, junto al reverbero, parece que leen
\emph{La España} o \emph{El Guirigay} maldito. ¿Habrá noticias? Ya nos
lo dirán en el café. ¿Tendremos libertad? Que la traigan con mil pares;
pero nosotros no nos metamos, D. José, no nos metamos\ldots{} Somos
gente de orden.»

Subió el buen Santiago al segundo piso de la misteriosa casa; llamó
tirando de un sucio cordón; le atisbaron por un ventanillo reforzado con
cruz de hierro, y franqueada al fin la puerta, viose ante un hombre
escueto que lo mismo podía parecer torero de invierno que sacristán de
las cuatro estaciones, el cual, previo examen de la media tarjeta, le
introdujo y encerró en una estancia con las paredes cubiertas de
imágenes, estampas piadosas y objetos de devoción. Oyendo el intenso
murmullo de pláticas muy vivas en próximos aposentos, entretuvo el corto
plantón contemplando a la luz de un quinqué pestífero las estampas
milagrosas y un retrato de Gregorio XVI con el ropaje bordado en
mostacilla, y de pronto se vio sorprendido por su amigote Gallo, que le
abrazó con toda la rudeza cariñosa que gastar solía.

«\emph{Chiquío---}dijo Ibero,---explícame pronto esto. ¿Qué me quieres?»

---Ya te lo diré\ldots{} aguarda un poco---replicó Gallo, sintiéndose
cohibido, premioso en su sinceridad.

---Siento voces. ¿Qué gente es esa? ¿En dónde estoy?

---Entre amigos\ldots{} Llamado por mí, no debes temer cosa mala. Esta
noche conocerás a un hombre\ldots{} ¡qué hombre, Santiago! Es el más
noble, el más digno, el más caballero\ldots{} ¡con un talento, chico,
con una penetración y conocimiento de las personas, de las cosas\ldots!

---¿Quién es? ¿No puedo saber su nombre antes de verle?

---Ya lo sabrás, sí\ldots{} ahora lo sabrás---repuso el otro algo
turbado.---Pero empecemos por decirnos tú y yo algunas palabras,
pocas\ldots{} tirémonos, si es necesario, cuatro mordiscos\ldots{} Somos
amigos; debemos confiarnos el uno al otro el estado de nuestros
corazones en punto a\ldots{} Más claro, confesémonos tú y yo lo que
pensamos de esta quisicosa que llaman \emph{la situación}.

---Lo que yo pienso ya lo sabes---afirmó Ibero con severidad.

---No se piensa lo mismo en campaña que en Madrid. Allá pensamos en
batirnos, en defender la vida; aquí en buscar la verdad, en defender los
principios\ldots{}

---¡Metafísico estás!\ldots{} Menos retórica y más franqueza, Modesto.
¿Somos o no somos amigos?

---Amigos hasta morir. Siéntate un momento. Hablaremos como hermanos.

Arrogante y garboso era el tal, y muy bien le caía el nombre de Gallo.
Menos que mediana era su estatura; pero no había otro mejor plantado ni
que mejor retratara en su continente la bravura indómita y en ciertas
ocasiones provocativa. Su encrespado cabello parecía una cresta; sus
ojos despedían el fuego de Marte; usaba bigote espeso, largo y caído,
imitando el personal estilo de su adorado jefe, Don Diego León. Su voz
vibraba en las disputas como clarín guerrero, y acompañábala con
gesticulaciones de una energía despótica. Argumentaba cerrado el puño y
solidificándolo como una maza de hierro. Empuñaba el argumento como una
lanza.

\hypertarget{iv}{%
\chapter{IV}\label{iv}}

«Tú has venido aquí---dijo Gallo,---como uno de los hombres de confianza
del General en Jefe, para preparar\ldots»

---No hemos preparado nada; no hacemos más que sostener\ldots{} La
misión del ejército es apoyar a la opinión pública y oponerse a los que
quieren ir contra ella.

---¡Inocente! Hablas como El Correo Nacional.

---Hablo como hombre de verdad y como soldado de honor.

---No lo dudo, \emph{chiquío}. Pero niego que seas tú el único que
interprete lo que llamamos la opinión. No constando en ninguna parte de
una manera clara lo que la Nación siente y desea, todos usamos el
derecho de ser sus intérpretes; y yo, que también tengo mi criterio y
aunque bruto, ¡ajo!, sé formar juicio de las cosas, sostengo que el país
no quiere la preponderancia de D. Baldomero, ni ve con buenos ojos que
se pretenda rebajar la dignidad de doña María Cristina, tratándola como
a una mala patrona.

---Veo, querido Gallo---dijo Ibero levantándose,---que no estamos ya
juntos frente a un enemigo común: estamos el uno frente al otro, cada
cual en su terreno. Somos dos amigos\ldots{} enemigos. Apenas sofocada
una guerra civil, inventamos otra para nuestro uso particular.

---De los capitanes afortunados nacen los grandes ambiciosos, digo yo.
Ahí tienes la peor calamidad de las guerras, que nunca son tan malas y
desastrosas como cuando concluyen.

---En suma, querido amigo, creo que ya hemos hablado bastante. Confieso
mi error. Yo creí que todos los Generales formados a la sombra de
Espartero apoyaban la Causa liberal en contra de la camarilla moderada.

---Algunos hay para quienes no existe más causa que la de la Reina, cuyo
nombre está escrito en nuestras banderas y en nuestros corazones\ldots{}
corazones muy brutos, ¡ajo!, pero muy leales.

---Háblame con franqueza. ¿Es León el único que resueltamente está
contra Espartero?

---No: hay más. Pasa revista en tu memoria a la plana mayor de nuestro
ejército. Fíjate en lo más brillante, en lo más ilustre, en lo más
inteligente. Entre esos, elige lo mejor de lo mejor. Resultará que todo
lo bueno está contra el ídolo.

---¡Ay, amigo mío!---dijo Ibero con profunda tristeza.---Convéncete de
que has dado un golpe en vago llamándome, y déjame salir de aquí. Estoy
violento, y de seguro te estorbo.

Hizo ademán de retirarse, y el otro le retuvo estrechándole
afectuosamente las manos. En el mismo instante oyéronse voces y ruido de
pasos. Alguien entraba o salía.

«Aguárdate---indicó Gallo, bajando la voz,---deja que salgan esos. Aún
tenemos algo que hablar\ldots»

---La reunión concluye---dijo Ibero, poniendo atención a los ruidos de
voces y pasos que indicaban la salida cautelosa de un número de personas
difícil de apreciar por el oído.---Tienes razón: algo falta que decir.
Lo que ha pasado esta noche entre nosotros, y lo que no ha pasado, todo,
todo, quedará en el mayor secreto.

---Naturalmente. He contado con Santiago Ibero ¡re-Dios!, que es contar
con la decencia misma, con la caballerosidad.

---¿Puedo ya tomar la puerta, \emph{chiquío}?

---No.~Abuso de tu amistad reteniéndote un poquito más. Has formado mala
idea de mí, y quiero rehabilitarme en tu concepto. No quiero que quedes
bajo la mala impresión de la tosquedad con que yo expreso mis ideas:
quiero que estas lleguen a ti por boca mejor que la mía, por la persona
de que antes te hablé\ldots{} No: no te dejo ir sin que le veas. Dame
ese gusto, hombre\ldots{} no te hagas el interesante. Ya no hay en esto
compromiso alguno, ni aquí se trata de conspiración, ¡ajo!, ni de
narices. Somos dos amigos que oyen la palabra hermosa de un tercer
amigo, y así le llamo porque sé que te cautivará. No tengas ningún
recelo, ¡contro! Repito que no hay en ello compromiso\ldots{}

---Por agradable que sea hablar con hombres tan eminentes, yo creo que
debo retirarme.

---Aguarda un momento. Pues nada tenemos que hacer aquí, ni se ha de
resolver cosa alguna, quizás salgamos juntos los tres. Aguarda te digo,
no más que dos minutos.

Sin dar tiempo a que Ibero hiciese nuevas observaciones, salió Gallo, y
a los pocos instantes volvió acompañado de un sujeto, a quien presentó
en forma solemne: «D. Manuel Montes de Oca, ex-Ministro de la Corona.»

La primera impresión de Ibero fue de disgusto, como de quien se ve
objeto de una emboscada. Permanecieron los tres un instante mudos,
esperando cada cual que uno de los otros dos dijese la primera palabra.
En este breve lapso de tiempo, el enojo de Ibero se dulcificó ante la
fisonomía grave, dulce y melancólica del joven gaditano, a quien conocía
por su nombre, un poco altisonante, y por su fama de caballerosidad.
Habíasele imaginado viejo, adusto y con cara de pocos amigos; y viendo
su juventud, su hermosura, su expresión soñadora y romántica, sus azules
ojos, que antes revelaban las tristezas del poeta que las energías del
sectario, reconoció que nuestra existencia no es más que un tejido de
errores, y que gran parte del tiempo que vivimos lo empleamos en la
necesaria rectificación de juicios y creencias.

«No sé cómo pedir a usted perdón---dijo Montes de Oca a punto que los
tres se sentaban,---por haberle traído a una reunión, cuyo objeto,
momentos antes de llegar usted, dábamos ya por fracasado. Ha sido usted
muy oportuno en llegar tarde, y así no hay para nadie ni sombra de
compromiso. Con media palabra me ha dicho Gallo que no habríamos podido
contar con usted. Más vale así, ya que nada hemos hecho ni podremos
hacer por ahora. Ello es muy triste; pero de una realidad que a todos se
impone.»

---Llegando tarde es menos violenta para mí la negativa que yo habría
dado a los que, por lo visto, se reunían para defender una causa
perdida.

---Por el momento quizás---dijo Gallo.---Luego veremos.

---En política---afirmó Montes de Oca acentuando su expresión de
tristeza,---el momento presente es lo que más importa. Al intentar dar
una batalla nos hemos encontrado sin fuerzas y, lo que es peor, sin
terreno. Usted, Sr.~de Ibero, piensa que somos locos, y en ello tiene
mucha razón\ldots{} Pero no: el único loco soy yo, y las personas a
quienes he querido hacer partícipes de mi delirio han tenido el buen
acuerdo de dejarme solo. Respetando las ideas de usted, y en la
esperanza de que usted, como hombre leal, respetará las mías, yo me
permito emplazarle para dentro de un año, de dos\ldots{} Entonces
veremos dónde está la sinrazón y dónde la cordura.

---Las convicciones arraigadas, señor mío, aunque sean erróneas, merecen
siempre respeto. Reconociendo que el proceder de usted en este asunto es
obra de una alucinación, celebro infinito que mis compañeros no hayan
querido o no se hayan atrevido a secundarle.

---El tiempo hará lo que yo no he podido hacer. Quizás es conveniente
que el mal madure y crezca, para destruirlo más pronto y desarraigarlo.
En los momentos críticos de la vida de los pueblos, no es fácil saber
dónde está la alucinación y dónde la claridad del juicio. Alucinan los
triunfos repentinos, no la desgracia; la usurpación puede ser un
delirio; el derecho no lo es. Y en cuanto a la nobleza de los móviles,
yo le invito a usted a que haga un paralelo, una comparación entre los
que defienden la fuerza material y los que patrocinamos la espiritual.
Dígame usted qué cree más digno y noble: si alentar el poder ciego de
las armas, o apoyar la ley representada en lo más augusto, que es la
Monarquía; en lo más hermoso, que es la mujer; en lo más sagrado, que es
la infancia.

---Sr.~de Montes de Oca, usted es elocuente; yo, pobre soldado, no sé
más que sentir. Siento las ideas\ldots{} no sé si digo un disparate. En
mi corazón, en mi cabeza dura, las junto con el honor, con el deber
militar, con la idolatría de mis jefes, bajo cuyas órdenes he derramado
mi sangre; las junto también con el amor de mi querida Patria, de la
libertad, a quien adoro sin saber por qué\ldots{} y con todas estas
cosas hago un solo sentimiento, que es mi vida. Así soy, y así me
encontrará usted siempre. Conmigo no podrá usted ganar batallas, y yo
haré cuanto pueda para que las pierda.

---Dejemos para lo futuro las lecciones que podamos recibir el uno del
otro---dijo Montes de Oca.---Por hoy, ya que entre los dos no resulte
amistad, separémonos como caballeros que somos. Me reconozco vencido
antes de combatir. No abusen ustedes de su poder antes de ser
vencedores. Declaro que si yo tuviera fuerza material, impediría la
usurpación que se prepara. Entre los defensores de ella hay muchos que
la creen odiosa, brutal; pero no se atreven a combatirla. Yo me
atreveré, por poco que me secunden, y espero que mi ejemplo traerá
prosélitos a esta santa causa. Prepárese usted, y los que como usted
piensan, a las audacias de un enemigo terrible: ese soy yo, se lo
advierto desde ahora para que sean implacables conmigo, como yo lo seré
con ustedes. De seguro verán en mí una actitud quijotesca, una pasión
que por querer remontarse a lo heroico, resulta ridícula. No me importa:
está en mi naturaleza el acometer las empresas grandes que casi parecen
imposibles, y no porque lo sean me acobardan a mí\ldots{} En la
expresión de su cara, oyéndome, veo que mis arrogancias no le asustan ni
le enfadan.

---En efecto---replicó Ibero:---me agrada su tesón y lo admiro.

---No, no puede considerarse perdida---afirmó Gallo con cierta
brutalidad de gesto y de palabra,---una causa que tales leones cría.

Levantose Montes de Oca, y después de dar algunos pasos por la estancia
detúvose ante los militares y les dijo: «Lo que ahora tememos algunos,
lo que ustedes preparan, lo que unos amigos de Espartero niegan con
hipocresía y otros anuncian con insolencia, será un hecho dentro de
veinte, treinta o más días\ldots{} qué sé yo cuándo. ¿Y este atentado se
consumará sin que en el ejército español, donde hay tantos hombres de
honor, se desenvaine una sola espada para impedirlo\ldots? Usted lo cree
así\ldots{} yo no, yo no puedo creerlo\ldots{} Si lo creyera, maldeciría
a mi Patria\ldots»

Honda impresión hicieron en el alavés estas palabras, a las que no pudo
contestar sino con otras torpes y balbucientes. Era un rudo soldado
incapaz de filosofar sobre cosas públicas, un monomaniaco del
patriotismo, que no entendía bien las razones contrarias a la breve
fórmula de su demencia. Ello no impidió que sintiera misteriosa simpatía
por el gaditano, viendo en él un desdichado caballero que se prendaba de
los imposibles y a pelear se disponía, solo y triste, por una idea
rancia y sin lucimiento\ldots{} ideas de capa y espada, cosas de la
\emph{edad media}, o de cualquiera edad donde no había progreso.

La despedida fue breve. Ibero le estrechó la mano, sintiendo, y así lo
dijo, no ser su amigo. El otro se fue delante, dejando tras sí un
suspiro, y hasta que no le sintieron en el tramo más bajo de la
escalera, no se determinaron los militares a salir, para dejar entre
ellos y el paisano un largo espacio de calle. Descendieron silenciosos,
lentamente, y en la calle no vieron más que media docena de vecinos que
huyendo del calor de las habitaciones, hacían su tertulia en las aceras,
mientras los chicos jugaban en el arroyo. De los portales y cuartos
bajos salía un olor de humanidad comprimida que destapa sus madrigueras
para no ahogarse.

«Estáis locos» fue lo único que Ibero dijo a su amigo, aproximándose a
las Descalzas.

---Es verdad---replicó el otro, sacando con dificultad las palabras del
cuerpo.---Locura es pelear uno contra veinte. Que triunfen, y sólo con
el hecho de triunfar, nos ponen en la proporción de veinte contra
uno\ldots{} ¿Qué es lo que ahora pasa? Que no hay oposición. Pero en
España la oposición se forma en cuatro días después del éxito. Nace como
la mala hierba, y crece como la espuma. Verás, verás\ldots{} Yo lo he
dicho: para poder apedrear bien a un ídolo hay que ponerlo
arriba\ldots{} Arriba, y bien alto, para que no se pierda ni una china,
¡ajo! Di que estamos locos. Los locos son ellos, y tú, Santiago,
tú\ldots{}

\hypertarget{v}{%
\chapter{V}\label{v}}

Al día siguiente de este suceso recibió Ibero orden de partir para
Valencia, conduciendo cuatro compañías de \emph{Borbón} y dos de
\emph{San Fernando}. Las únicas personas que de política le hablaron el
día de su partida fueron los inseparables Milagro y D. Bruno, sin que
ninguno de los dos obtuviera de él ninguna referencia de la misteriosa
reunión del Postigo de San Martín. Medroso, turbado por la visión
continua de graves disturbios, el manchego se mostraba pesimista, con
más ganas de volver al terruño que de continuar en Madrid su inútil
via-crucis de oficina en oficina. En cambio, D. José soñaba despierto
con una revolución pacífica y absolutamente limpia de sangre, que nos
trajera la justicia y el reinado de la honradez; jarana filosófica, ante
la cual habrían de prosternarse todos, reconociéndola buena, eficaz y
definitiva, como principio de una era de perdurable ventura. «Este país
se gobierna con una hebra de seda, señores---decía con tenaz
convencimiento, que parecía fe religiosa.---Y lo que es una revolución
pacífica, que resuelva de una vez todas las cuestiones, no ha de
faltarnos. Yo, yo me comprometo a ello sólo con que me dejen tres días
de \emph{Gaceta}. Nada, nada: es cosa sencillísima\ldots{} Tres días de
\emph{Gaceta} me bastan, y si me apuran, dos\ldots{} Soy sastre viejo,
conozco el paño. Pero, Señor, ¿no es principio de los principios la
voluntad nacional? Pues teniendo esta bien manifiesta, basta con un
\emph{cúmplase}. Que se cumpla, y todo el mundo boca abajo. Y no me
salgan los moderados con la tecla de que la santísima voluntad de los
españoles no es clara como el agua. España clama libertad con justicia,
y honradez en todas las esferas, y al pedirlo señala con el dedo bien
tieso quién puede darnos el bien que no gozamos. ¿Lo quieren más claro?
Pues para claridades, ahí tienen lo que ocurrió al paso de la Reina por
Zaragoza. El pueblo aclamó con mayor estruendo a la señora Duquesa de la
Victoria que a la propia doña María Cristina. ¿Y eso? Pues a cada
instante vemos demostraciones no menos elocuentes de la voluntad de la
Nación\ldots{} Yo gobernante, ustedes gobernantes, ¿qué haríamos? Decir
\emph{cúmplase}\ldots{} y cumplir\ldots{} Ya ven a qué sencilla fórmula
se reduce todo mi sistema. ¡Cumplir, cumplimiento! Y no más trapisondas,
no más discusiones, no más derramamiento de sangre\ldots{} Declaro que
no soy partidario de la violencia, ni de los tumultos, ni de que se haga
uso de las armas\ldots{} No se ofenda usted, querido Ibero, si le digo
que a todos los militares, en tiempo de paz, les mandaría yo a sus
casas, quedándose sólo una corta fuerza para contener a los
malhechores\ldots{} ¿Para qué necesitamos tanta tropa una vez que todo
quede establecido en regla? Para nada. Más bien servirán ustedes de
estorbo que de ayuda\ldots{} Y luego, un gasto fabuloso, inútil, mi
querido D. Santiago. Yo emplearía las tres cuartas partes del
presupuesto de guerra en fomentar la riqueza pública, y por cada fusil
que suprimiera plantaría un árbol, y en vez de regimientos, pondría
Sociedades de Amigos del País, y los cuarteles se convertirían en
Universidades, y las banderas servirían para adornar las imágenes en
nuestros templos\ldots{} en fin, poca fuerza y mucha ilustración. Que me
dejen la \emph{Gaceta}, y verán qué pronto\ldots»

Hubiera seguido desarrollando con fácil vena sus proyectos, producto
inagotable de su reciente desvarío político, si el buen Ibero, que
comúnmente se interesaba poco en la aplicación de los principios, por
serle más grata la contemplación mental de los mismos en abstracto, no
acelerase la despedida. Deseáronle los dos amigos, y otros que a la
sazón llegaron, un viaje feliz, y partió a la cabeza de las seis
compañías. Era un anochecer caluroso. Para no fatigar inútilmente a sus
soldados, Ibero dispuso aumentar las jornadas nocturnas, abreviando las
caminatas durante el día. No podría imaginarse peor tiempo de viaje,
siquiera este fuese de tropa, que en toda ocasión debe y sabe ir a donde
la llevan. Los caminos eran polvo, el aire fuego; del sol diríase que
arrojaba la luz a torrentes y con ella el polvo, y del suelo, que
ensuciaba y resplandecía. Y al través de aquel territorio arábigo, seco
y ardiente, que media entre las puertas de Madrid y las riberas del
Jarama, los soldados iban locos de alegría; el calor y la sequedad eran
su elemento; ni el peligro ni el temor de guerra podían inquietarles; no
aguardaban ni perseguían a un enemigo fiero; no les faltaban alimentos,
ni agua, ni obsequios de vino; era su viaje un paseo triunfal por los
pueblos feos tras de los cuales vendrían pueblos bonitos, y en todos
ellos encontraban muchachas de distintos pelajes a quienes embromar. El
contento de la tropa, soltando chispas a lo largo del árido camino, iba
prendiendo fuego y levantando llamas de alegría: para los pueblos era
una dicha el paso de la tropa, y esta no deseaba sino que España fuese
del tamaño de todo el mundo para que la marcha no tuviese fin. ¿Qué más
podía desear el soldado sino que le pasearan por el mapa, viviendo y
gozando sin funciones de guerra? Vida más deliciosa no podrían soñar los
pobres hijos del terruño español, destinados poco antes a matarse
despiadadamente. Hermosa era la paz, y grande entre los más grandes el
que la había traído\ldots{}

En medio de la infantil alegría de su tropa, Ibero iba triste, agobiado
por el calor. Recorría largas distancias sin hablar con los compañeros
que le rodeaban más que lo necesario para los actos del servicio. Como
D. Quijote en sus horas de melancolía soñolienta, dejaba tomar al
caballo el paso que quisiese, y contemplaba las vagas líneas del
horizonte, o las nubes, si por acaso las había en el cielo, o las ondas
de polvo que el viento llevaba consigo, arreándolas como a una recua de
fantasmas. No se crea que el militar adormecía su entendimiento en un
éxtasis de cosas políticas, discurriendo si tendríamos mayor o menor
grado de libertad. Esto le interesaba, le había interesado en los
tiempos de la campaña activa; mas desde los meses que precedieron al
abrazo de Vergara, Ibero había sufrido la brusca invasión de una
enfermedad del espíritu muy propia de sus años viriles, la cual por
venir algo tardía entró con más fuerza, cogiéndole de un extremo a otro
todo el campo de la naturaleza física y moral, sin que quedase parte
alguna que no estuviese afectada por tan grave dolencia. Que esta era el
amor, fácilmente se comprende; un amor como los que se estilaban en
aquella época: abrasador, exclusivo, con tendencias lloronas y
funerarias, sabores de amargura y relámpagos de lirismo.

La historia era de las más comunes. Apenas conocía Santiago el amor más
que por inclinaciones o caprichos insubstanciales, cuando se prendó de
una señorita de La Guardia, a quien había conocido en la niñez y en la
juventud florida de ella, sin que jamás se le ocurriera que viniese a
ser la dama de sus pensamientos. Ello fue repentino, obra de un par de
tardes apacibles; se inició en una fiesta popular; siguió
desarrollándose en un paseo junto a la iglesia, después en un refresco
que dio el cura párroco al señorío principal de la villa; y para
determinar el incendio de la grande alma de Ibero, no hubo más
combustible que unas palabritas de simpatía disimulada con donosas
burlas; después otras de él que debieron de ser de lo más atrevido
dentro del comedimiento social, y luego\ldots{} un par de cartas muy
respetuosas con las indispensables fórmulas de rendimiento y ternura.
Obtuvieron estas una cortés acogida, que ya significaba mucho en la
condición de la niña, y a tal demostración siguieron bromas delicadas
que encerraban veras muy dulces; en sucesivas entrevistas se marcó el
gusto que recibía la señorita de verse amada por un joven tan gallardo
como Ibero y de tan honrosos adelantos en la carrera militar; mas no
queriendo entregar su alma sin la preparación y trámites que pide la
decencia, echó por delante risueñas esperanzas, con las cuales el hombre
se tuvo por amante dichoso. Pero ¡ay!, en cuanto le alejó de La Guardia
la dura obligación militar, ya no fue vida su vida, sino un martirio
continuado, pues lo mismo le atormentaban sus alegrías delirantes que
sus lúgubres tristezas. Un correo amoroso, enviado y recibido de tarde
en tarde, sostenía su pasión en el punto de mayor ardimiento. Cartas
recibió en Miranda, en Morella, en Madrid, y cartas expidió desde
aquellos y otros puntos. No queriendo dudar, dudaba; la niña, fuese por
estudio, fuese porque así lo dictaba la realidad, a lo mejor salía
proponiendo ruptura. ¿En qué se fundaba? En razones de familia muy
atendibles que no podían exponerse por cartas; en repentinas veleidades
de vocación religiosa, que despertaban en Ibero furiosos celos de
Jesucristo\ldots{} Ello es que el hombre no vivía, y sus inquietudes
subían de punto con la idea mortificante de no ser grato a la familia,
que si le apreciaba como a un joven de mérito, de honrada progenie y
buen acomodo, quizás no le creía digno de poseer un bien tan grande como
la niña de Castro Amézaga, noble por los cuatro costados y poseedora de
un rico patrimonio. En el aburrimiento y soledad de aquel viaje a
Valencia, sus temores y tristezas se resumían en el propósito de dirigir
una expresiva carta al Sr.~de Navarridas no bien llegara al término de
su caminata. Urgía despejar la terrible incógnita. Pensando en ello,
ocupada la mente noche y día por la linda imagen de su dama, iba el
hombre tragando leguas, bebiendo polvo, espaciando la vista por las
llanuras abrasadoras o distrayéndola en los cerros piníferos; cumpliendo
como una máquina sus deberes militares, sin más gusto que el de tolerar
a los soldados todos los esparcimientos que no fueran escandalosa
violación de la disciplina.

Nada ocurría en el cansado viaje que alterara la desazón tediosa del
alma de Ibero. ¡Si al menos hubiera guerra, enemigos que combatir,
ocasiones de exponer la vida y de ganar nuevos laureles! ¡Acabarse la
guerra cuando él se hallaba casi a las puertas del generalato! La faja
hubiera sido un título ante el cual los Navarridas no podrían mostrarse
inflexibles\ldots{} Pero ya no había que pensar en nuevas campañas, pues
Espartero había asegurado la paz por mucho tiempo. ¡Qué cosas trae la
vida, Señor! ¡Él, Santiago Ibero, que había peleado sin ambición, movido
tan sólo del ardiente amor de la libertad, y del gusto de afianzarla con
las armas, apenas terminada la lucha sentía en su alma el gusanillo, la
avidez de más altos títulos y empleos para deslumbrar con ellos a una
noble familia! Y no era él hombre para despreciar la paz, ni haría cosa
alguna que contribuyese a renovar los pasados horrores. Su conciencia
antes que todo. Si no le daban la niña de Castro, no podría vivir. La
muerte sería la solución, un morir no menos glorioso que el de los
campos de batalla, pues lo mismo daba caer a los pies de Cupido que a
los pies de Marte, que tan dios era Juan como Pedro.

Por efecto del calor y del cansancio que le quitaban el apetito, al
pasar las Cabrillas iba el hombre tan espiritado, que el caballo, si en
ello pensara, habría podido darse cuenta de una notable disminución en
el peso de su jinete y señor. Ya en el llano de Valencia, donde los
soldados se entregaban a locas alegrías, convidados de la dulzura del
clima y de las abundancias de aquella tierra, Ibero se sintió invadido
por tristezas más crueles, que se le agarraron al hígado, al corazón, y
luego despedían negros vapores hacia la cabeza. Marchando en las serenas
noches, se complacía en ver espectros, que surgían a uno y otro lado del
camino, y pausadamente se alejaban ante el regimiento, mirando hacia
atrás con fúnebres ojos. No eran, no, las nubes de polvo que levantaba
el viento, eran ilusorios o verdaderos fantasmas, seres de otro mundo,
que venían a penar en este y en el propio lugar donde fueron despojados
de su carnal vestidura; eran las sombras de los infelices españoles
brutalmente fusilados en los mataderos de Utiel, Chiva y Burjasot. De un
suelo harto de sangre se evaporaban los trágicos horrores de la guerra
para turbar los días serenos y las noches plácidas de la paz. Fuese
porque estas imaginaciones le trastornaran, fuese porque al pasar
bruscamente del achicharradero de la meseta central a las humedades
ribereñas contrajera algo de paludismo, ello es que entró en Valencia
con escalofríos y sed insana. El físico le recomendó descanso y pócimas;
mas no hizo caso, atendiendo sólo, antes que a su salud, a buscar en el
correo, o en la Capitanía General, las cartitas de La Guardia. Contaba
con ellas como con la salida y puesta del sol a la hora marcada por los
almanaques. Pero los divinos papeles ¡ay!, o no habían llegado, o
andaban perdidos en los laberintos de la ciudad, quizás en manos de
personas extrañas que los profanaban leyéndolos. ¡Qué abominación!
Horrenda catástrofe era que se perdiesen, y crimen nefando que los
violara la curiosidad. Lo primero merecía una revolución; lo segundo, un
cruel castigo\ldots{} fusilar sin piedad a todo español que desflorase
una carta.

\hypertarget{vi}{%
\chapter{VI}\label{vi}}

La enfermedad de Ibero no fue grave ni larga, y aún habría durado menos
si llegaran las deseadas epístolas. En cambio de esta soledad del
corazón, veíase mentalmente asaltado de continuas impresiones, pues los
amigos le llevaban todo el fárrago de noticias que diariamente llegaban
de Barcelona y de Madrid. El capitán D. Jacinto Araoz, que amaba a su
superior como a un hermano, le ponía en autos de las graves ocurrencias,
refiriéndolas con el calor que todo español pone en las cosas del
procomún, principalmente cuando no le afectan ni mucho ni poco. En
Barcelona, \emph{archivo de la cortesía} según Cervantes, arca del
liberalismo según los modernos, había estallado un motín. Decían los
enemigos de Espartero que la trifulca era obra de Linaje. ¿Qué querían
los revoltosos? Pedían, a juzgar por sus gritos, cosas muy buenas. ¡El
Duque, la Constitución, nuevo Gobierno! La Reina y el General no se
habían entendido en la formación del Ministerio ni en el programa de
este, pues de un lado tiraban a que la nueva ley de Ayuntamientos,
violación de un principio constitucional, fuese sancionada, y de otro a
que no lo fuera. D. Baldomero se atufaba y anunciaba la dimisión de
todos sus cargos; la Reina no sabía de qué lado volverse, pues los
hombres civiles de valía no eran la fruta más abundante en el país.
Todos resultaban enanos, medrosos, obedientes a la espada o bastón de
quien había sabido mantener el uso exclusivo de estos emblemas de
autoridad\ldots{} El motín fue escandaloso, repugnante: ni los
amotinados sabían hacer revoluciones, ni las autoridades el arte y modo
de contenerlas. Ocurrieron desmanes vergonzosos, actos de estúpida
crueldad; moderados y liberales se injuriaban o se agredían en medio de
las calles. Ante los balcones de la residencia del Duque vociferaban los
unos, y ante el carruaje de la Reina los otros hacían demostraciones
ridículas. Hubo no pocas víctimas, algunas gloriosas; rasgos personales
de caballeresca audacia, que contrastaban con el salvajismo de la soez
multitud. Terminó al fin la jarana con prisiones y bandos, y el
indispensable cambio de personas en los primeros cargos militar y civil.

Por variar, el mismo 18 de Julio estallaba en Madrid otro motín, y los
pobres ministros no sabían a qué santo encomendarse. Todo lo arreglaban
dimitiendo; con delicadezas y remilgos querían gobernar un país revuelto
y desquiciado. Felizmente, la Milicia de Madrid supo cumplir, y todo se
redujo a los himnos y vociferaciones de costumbre en calles y plazuelas,
a los atropellos de gente pacífica por gente desalmada. Libertad pedían
los revoltosos, y en nombre de este ideal acometían a las mujeres que
llevaban galgas, o a los hombres que por su traza elegante ¡oh
contradicción!, parecían enemigos del \emph{progreso}. La tropa
permaneció fiel a la disciplina; los ministros, pasado el peligro,
acordaron que se cantara un solemne \emph{Te Deum} para celebrar la paz.
¡Bonita paz nos daba Dios!\ldots{} Lo más grave de todo, según el bueno
de Araoz, era que Inglaterra y Francia, las dos potencias más poderosas
y camorristas del mundo, tomaban partido en nuestras discordias,
declarándose los ingleses por la libertad y Luis Felipe por la
\emph{moderación}. «Era lo que nos faltaba---decía el ingenioso
capitán:---que las naciones extranjeras vinieran a enzarzarnos más de lo
que estamos. ¡Vaya una paz que hemos traído, chico! Ya voy viendo que la
mejor de las paces es la guerra, y que nunca están los españoles tan
sosegados y contentos como cuando les encharcamos con sangre el suelo
que pisan. Preparémonos para otra campaña, querido Santiago, la cual no
veo clara todavía, pues no sé quiénes serán ellos ni quiénes seremos
\emph{nosotros}; pero entre media España y la otra media andará el
juego. A prepararse digo, que aquí la paz es imposible, y si me apuran,
desastrosa, porque el español ha nacido eminentemente peleón, y cuando
no sale guerra natural, la inventa, digo que se distrae y \emph{da gusto
al dedo} con las guerras artificiales.»

Poco interés ponía Ibero en estas cosas, pues para él guerra y paz,
progreso y obscurantismo, se borraban en su mente ante el inmenso
problema de que llegara o no la deseada carta. Corrieron días, y al
anuncio de que la Reina saldría de Barcelona para Valencia, comenzaron
atropelladamente los preparativos para la recepción. Llegó Su Majestad
por mar, en un vapor mercante, y desde que fue avistado por el vigía,
acudieron las tropas a formar en el Grao. Agregado a la sazón Ibero al
Estado Mayor, debía escoltar a la Reina hasta su alojamiento, que era el
suntuoso palacio de Cervellón. Desde muy temprano se agolpaba la
multitud en el puerto. Desembarcó la Gobernadora, y las primeras
aclamaciones con que fue recibida al poner el pie en tierra no revelaron
un delirante entusiasmo popular. Ibero la vio en el momento en que al
coche subía, oído el breve saludo de las autoridades, y quedó encantado
de la gentil presencia de Cristina y de la incomparable gracia de su
rostro. El mirar dulce, las lindas facciones, los hoyuelos que al
sonreír se le hacían a uno y otro lado de la boca, le fascinaron. No
había visto jamás mujer tan bonita, con excepción de una, de una sola,
que por soberanía de amor no podía tener semejante. Y lo más extraño fue
que entre aquella, la suya, y María Cristina encontraba misterioso
parecido. No eran iguales el color del cabello ni el corte de la frente;
pero la boca y singularmente los hoyuelos decían: «aquí estamos todos.»
Con tal semejanza y la impresión que hizo en él la Reina, cuya imagen
llevó estampada en la mente mientras duró el trayecto del Grao al centro
de la ciudad, tuvieron gran alivio las melancolías del buen alavés; casi
estaba contento; veía rosados y luminosos los horizontes de la vida, que
horas antes se le presentaban negros, y se sentía menos desconfiado y
pesimista.

En el tránsito de las Personas Reales, las manifestaciones del pueblo
resonaron débiles y frías. Habría querido Ibero más calor, más
entusiasmo, que bien lo merecían los peregrinos hoyuelos y la seductora
expresión de aquella sonrisa. ¿Qué importaba la insana preferencia de la
Gobernadora por los moderados, si encantaba al mundo con su gracia
hechicera\ldots? Nuevamente vio el alavés a Su Majestad al parar el
coche para recibir a las muchachas que le ofrecieron ramos, y mayor fue
entonces su admiración de tanta belleza, y más vivo el sentimiento
plácido que invadía su alma, algo como confianza en lo futuro y retoños
de esperanza. Un cuarto hora después de la entrada de la Reina en
Palacio, y hallándose Santiago en el cuerpo de guardia, se le acercó
presuroso su asistente, y con voces de alegría confianzuda le dijo: «Mi
Teniente coronel, ¡dos cartas, dos! Ahora mismo llegaron por el
correo\ldots{} Ahí las tiene. Y que no abultan poco.»

Cogió las cartas Santiago, quedándose un rato como si no viviera en este
mundo, y las guardó en el pecho para leerlas en la primera ocasión. Como
una tarabilla continuó charlando con sus compañeros, a quienes no pudo
ocultar la alegría que inundaba su alma. Todas las cosas tomaron risueño
color a sus ojos: la oficialidad era más diligente en el servicio; los
soldados ganaban en marcialidad y compostura; los generales estaban ya
de acuerdo para dar patriótica solución a las graves cuestiones; los
políticos de clase civil deponían su ambición y sacrificaban al interés
público todo interés personal. ¡Y la Reina!\ldots{} ¡oh!, ¡la
Reina!\ldots{} Al retirarse a su alojamiento, metiéndose la mano en el
pecho para acariciar lo que pronto había de leer, se decía: «Esa mujer
divina es quien os ha traído, adoradas cartas\ldots{} Me parece poco
llamarla Reina: es un ángel, una diosa\ldots»

Las cartas no decían nada y lo decían todo. Traían las mismas dulzuras
de otras, y las propias esperanzas. No fijaban el porvenir de un modo
concreto, y esquivaban la cuestión capital; referían con gracia
encantadora mil cosas de familia, y en medio de estas intimidades
dejaban entrever el obstáculo que era la mayor tristeza del valiente
militar. Luego expresaban el gozo de que hubiese terminado la guerra, y
entonaban un himno a la paz. De la paz resultaría mucho bien, así a los
grandes como a los pequeños. No bien las hubo leído Santiago, le asaltó
el formidable tumulto de ideas para la respuesta; había tanto que decir,
que difícilmente podría decirlo todo.

Días después, habiendo tomado el mando de \emph{Borbón (17 de línea)},
entró de guardia, y Su Majestad le convidó a comer. En su vida se había
visto en trances de tanta etiqueta. El honor de la invitación le
vanagloriaba, y el miedo de hacer un papel desairado le afligía; mas se
tranquilizó pensando que para salir del paso bastábale su buena
educación castiza, sus hábitos de caballero y militar. No necesitaba,
pues, experiencias cortesanas, pues al soldado de temple no se la había
de exigir un conocimiento prolijo de la vida social. Durante la comida,
y en la breve recepción que la siguió, Su Majestad estuvo con todos
amabilísima, y a cada cual supo decir un concepto grato. Distinguió a
Ibero, consagrándole algunas palabritas más de lo que acostumbraba, y
ellas fueron tales que el agraciado no pudo olvidarlas en mucho tiempo.

«Sr.~de Ibero---se dignó decir la Reina,---se cuenta por ahí que anda
usted terriblemente enamorado. Aunque me consta lo que usted vale, temo
que una pasión tan fuerte le distraiga del servicio\ldots»

---¡Señora!\ldots---murmuró Ibero en el colmo de la turbación, trémulo
como un niño, viendo de cerca los lindísimos hoyuelos que daban infinita
gracia a la boca de Su Majestad.---Señora\ldots{} yo\ldots{} digo que el
servicio de mi patria y de mi Reina es antes que todo.

---Si lo sé\ldots{} Mientras más enamorados, más caballeros y mejores
servidores de estas pobrecitas Reinas. Y qué, ¿no se piensa ya en
casorio? No descuidarse, Sr.~de Ibero. Ya, ya sé\ldots{} me lo ha dicho
Jacinta\ldots{} Una huérfana, mayorazga. Son dos hermanas.

---Las dos de muchísimo mérito.

---Todo se arreglará. Cree Jacinta que todo irá por buen camino\ldots{}

---La señora Duquesa de la Victoria---dijo Ibero, arrancándose con una
audacia de pretendiente,---podría interesar en mi favor a la familia
de\ldots{} a los señores de\ldots{}

Bruscamente cambió de asunto Su Majestad, como herida de un recuerdo
vago que anhelaba precisar.

«Me parece---dijo;---no estoy bien segura\ldots{} paréceme que en la
lista de ascensos a coronel que me han presentado ayer está el nombre de
Ibero.»

---Bien puede ser, señora---replicó el militar:---sé que el señor Duque
me había propuesto.

---¡Coronel!\ldots{} Lo habrá usted ganado bien.

---Deberé mi ascenso, más que a méritos míos, a la munificencia de
Vuestra Majestad.

---Subir un escalón más en la milicia es cosa muy buena para los
enamorados que desean casarse, pues cuanto más suben, más fácilmente ven
a sus novias.

Oyó esto Ibero como un rumor lejano, pues atraían y fijaban toda su
atención los hoyuelos jugando en derredor de la regia boca. Distrájose
Cristina recibiendo el saludo del General D. Leopoldo O'Donnell y del
regidor D. José Félix Monge, que entraron en aquel momento; cambió con
ellos algunas palabras; volvió luego junto a Ibero, o más bien frente a
él, y por despedida le dijo: «Señor teniente coronel, ¿a quién quiere
usted que hable: al Ministro de la Guerra o a Jacinta?»

---A los dos, señora---replicó Ibero con una espontaneidad que al poco
rato turbó gravemente su conciencia.

Al retirarse no tenía consuelo, y furioso consigo mismo se echaba en
cara la grosería de aquella respuesta. «¡Qué gaznápiro me hizo
Dios!---se decía.---Debí contestar de una manera fina, con gracia y
modestia, no a lo bruto\ldots{} ¡En qué estaba yo pensando! Di una
respuesta egoísta, ambiciosa\ldots{} una respuesta \emph{moderada.»}

\hypertarget{vii}{%
\chapter{VII}\label{vii}}

Viviendo en sus soledades, sin dejar de atender con mecánica regularidad
a su militar obligación, nada le importaban a Ibero los acontecimientos
políticos, y las noticias del motín de 1.º de Septiembre en Madrid le
afectaron muy poco. El movimiento no fue iniciado por la plebe ni por
los militares. El Ayuntamiento rompió plaza, declarando su propósito de
no cumplir la Ley Municipal y poniéndose en frente del Estado. Era una
nueva forma de revolución, a lo pacífico, como la preconizaba el buen
Milagro, y ello debía de estar bien guisado, porque la Milicia se apiñó
resueltamente al lado de los ediles, y el ejército fraternizó con el
pueblo. Con este modo de señalar, claro es que no había de correr
sangre, ni había para qué.

Llegaban a Valencia las noticias abultadas y con cierto cariz poético.
¡Qué orden tan admirable! Verdaderamente no había pueblo más digno de la
libertad que el español. Así se engrandecían las naciones. Los
extranjeros se admiraban de nuestra cordura, de nuestra cívica
virilidad. Se repetían las frases ardientes de González Bravo,
pronunciadas en el Ayuntamiento, las proclamas de San Miguel a la
Milicia, y los dichos catonianos de este y el otro individuo, que
entonces empezaban a figurar en la historia. Naturalmente, se formó una
Junta, que asumió todos los poderes, y su primer cuidado fue dirigir una
respetuosa exposición a la Reina. Todo se hacía con respeto: con respeto
se convirtió un Municipio en Estado, y la fuerza pública se ponía a las
órdenes de un Alcalde, con muchísimo respeto. Oyendo contar a su amigo
Araoz estas novedades, Ibero lo encontraba todo muy natural; pero no
pudo menos de reír al enterarse de que en la flamante lista de
secretarios de la Junta de Madrid figuraba el claro nombre de José del
Milagro.

Dígase entre paréntesis que la ley de Ayuntamientos, causa de toda la
trapisonda, no era más que una triquiñuela legal de los moderados para
reducir a su mínima expresión la fuerza popular en los comicios, y matar
de raíz las aspiraciones progresistas. Revelaron en ello, si no la
suprema inteligencia de que blasonaban, una trastienda frailuna de que
sus contrarios carecían. Los caballeros del Progreso, aferrados a la
política sentimental, todo lo resolvían con himnos, abrazos y
banderolas; los otros iban un poco más al bulto.

Cundió por toda España el ejemplo de Madrid, y el pronunciamiento no
tardó en ser nacional. Vencida por un superior juego, la Reina no tenía
ya más que una carta, y la jugó sin vacilar: Espartero fue Presidente
del Consejo de Ministros\ldots{} Vio en ello Ibero la solución más
natural y conveniente, pues el Duque y la Reina, las dos personas más
altas de la Nación, encontrarían la forma y manera de hacer felices a
los españoles, dándoles leyes justas y gobernando con prudencia y
eficacia. Siempre había sido Ibero un gran inocente, y bajo la
influencia soñadora y narcotizante de su refinado amor, lo era mucho
más. Pensaba como un niño, y en la paz los tonos rudos de su fiereza
militar se avenían singularmente con el carácter incoloro y anodino de
sus ideas. Por aquellos días recibió su nombramiento de coronel, y fue a
dar las gracias a la Reina, que le recibió muy afable, sin repetir las
delicadas bromas acerca del noviazgo. Sin duda la señora no se acordaba
ya de tal cosa: su semblante revelaba insomnios y tristeza. La gravedad
de la situación política la reconoció Ibero claramente en los hoyuelos,
que aparecían algo desvanecidos y con pocas ganas de broma. Salió de la
regia estancia compadeciendo a Su Majestad, y deseoso de que el
Pronunciamiento le trajese días gloriosos, cosa en verdad menos fácil de
lo que parecía.

Recibió el coronel con su honroso grado el mando del \emph{Príncipe}, y
en la toma de posesión y en los trabajos de revista de material,
documentación, caja y demás, se le pasaron algunos días. Consagrose
después a un rudo trabajo epistolar, mandando para La Guardia en
plieguecillos de papel toda su alma y tiernísimos memoriales, y mientras
escribía con destreza febril, apenas se enteró de que el recibimiento
hecho en Madrid al Duque fue un delirio, de que la Junta revolucionaria,
como quien no dice nada, se permitía pedir a la Reina que diese un
manifiesto \emph{reprobando los consejos de los traidores que la
rodeaban; que separase de su lado a todos los funcionarios palatinos y
personas notables que habían concurrido a engañarla}, etc. Poco después,
no fueron tan sentimentales los acuerdos de la Junta, pues se arrancó a
proponer al Duque la reforma de la Regencia, con arreglo a los buenos
principios. La Reina era excelente persona, según la Junta, y
\emph{estaba animada de las mejores intenciones; pero en su
inexperiencia encontraban un campo fácil de explotar los que aspiran a
perdernos}. Para no cansar (el documento es largo y mal escrito),
querían los junteros asociar a la augusta persona otras que participaran
con ella de carga tan pesada\ldots{} y merecieran la estimación y
confianza nacional.

En esto, formaba el Duque su Ministerio, lo que no le fue difícil, dueño
como era de la fuerza y de la opinión, y con sus ministros en el
bolsillo, tomó el camino de Valencia, a donde llegó el 8 de Octubre,
harto de ovaciones, siendo la más solemne y estrepitosa la que en la
ciudad del Turia dispuso y efectuó la gran mayoría del vecindario, el
Ejército y Milicia. A la Cruz Cubierta salieron a esperarle generales y
jefes, el Ayuntamiento, y gentío inmenso de todas las clases sociales.
Locos de entusiasmo, los chicos de Milicia y pueblo desengancharon los
caballos de la carretela y tiraron de ella tan guapamente hasta el
interior de la ciudad, en medio del estruendo de las aclamaciones
patrióticas, que semejaba a los fragores de la Naturaleza. Comparsas y
músicas unían su clamor a la delirante voz del \emph{Progreso}. De
balcones, ventanas y azoteas llovían flores, coronas, dulces, confites,
versos del inspirado Arolas. Al llegar el pacificador a su alojamiento
en casa del Marqués de Mascarell, cantaron un himno los coristas del
teatro, digno remate de función tan lucida y grandiosa\ldots{} No ha
existido en España popularidad semejante, tanto más hermosa cuanto eran
más efectivos los méritos que la justificaban. ¡Qué caminito para fundar
algo grande y duradero! Ya se irá viendo, a medida que vaya clareándose
el balance histórico, lo que España debió a Espartero, y lo que
Espartero quedó a deber a España. Esta pobre vieja siempre sale
perdiendo en todas las cuentas.

«Eso de que la Regencia sea doble---dijo Ibero en aquellos días,
imponiendo su opinión a la oficialidad, mientras tomaban café en el
cuarto de banderas,---me parece una inspiración del cielo. Los dos
partidos, las dos ideas se juntan y gobiernan y transigen como un
matrimonio, que no se puede disolver. Si esto no cuaja, señores, será
porque aquí ya no hay patriotismo.»

Opinaron todos como él, y pusieron en el cuerno de la luna lo que
llamaban la co-Regencia, invención de la Junta municipal y constituyente
de Madrid. Mientras de esto platicaban los militares, haciendo de paso
sátiras muy acerbas de los personajes moderados que componían la
camarilla de la Reina, esta escuchaba en su cámara la lectura del
programa ministerial, en el cual, entre vanas retóricas, se soltaba esta
idea: \emph{Pero lo que más generalmente se desea es que Vuestra
Majestad se acompañe de hombres prácticos en la ciencia de
gobierno}\ldots{} Luego remachaban con este otro parrafito: \emph{Es
opinión tan generalizada, que hasta en los pueblos más pequeños y que
menos parece se ocupan de las cosas públicas, existe; y es tal la
exigencia respecto a este punto, que la creemos irresistible, y un
escollo contra el cual se estrellaría cualquier Gobierno que intentase
contrarrestarla}. Oyó la Reina, y no dijo si le parecía bien o mal el
documento, discreción en verdad muy extraña, pues para saber lo que
opinaba del programa se lo habían leído. Como para quitar a los
consejeros el mal efecto que había hecho su mutismo, requirió Cristina
el Crucifijo y Evangelios para que los tales juraran, y con esto y el
acto solemne de tomarles la prenda de sus conciencias, les tranquilizó,
y ellos se tuvieron ya por ministros efectivos. Salieron de Palacio, y
pasó un lapso de tiempo que por su importancia en aquella comedia hubo
de merecer diversos cálculos acerca de su duración. Fue lo que podría
llamarse un rato \emph{histórico,} y su longitud la apreciaron unos en
más, otros en menos. D. Joaquín María Ferrer lo fijaba en veinte
minutos, D. Manuel Cortina en quince, y Gómez Becerra en media hora.
Ello es que no había transcurrido después de la jura una larga
existencia ministerial, cuando Espartero, que aún no había salido del
palacio de Cervellón, fue llamado precipitadamente. Su instinto le
anunció algo grave, y no se equivocaba el señor Duque, hombre de olfato
seguro, pues al entrar en la regia estancia, la Gobernadora, nerviosa y
demudada, retorciendo en sus lindas manos el pañuelo, le dijo sólo tres
palabritas: «Espartero, yo abdico.»

¿Qué hablaron en el resto de la conferencia, que duró más de una hora?
Claro es que Espartero empleó aquel tiempo en disuadir a Su Majestad de
la resolución expresada. Debió de argumentar como ministro, como general
y como caballero, y las varias razones salidas de sus labios no debieron
de tener otro fin que la demostración del daño grande que al país
ocasionaría la renuncia. En ningún archivo histórico consta ni puede
constar aquel diálogo; pero la verosimilitud y el arte hipotético pueden
reconstruirlo. Lo verdaderamente indescifrable es el pensamiento de uno
y otro mientras hablaban; lo que dijeron no ofrece dificultad grande al
historiador. Claro como el agua se ve que el Duque agotó todo su caudal
lógico para quitarle de la cabeza a la bella Cristina la ventolera de
abandonar su cargo, y que la Reina se obstinó en la renuncia, como quien
ha tomado un acuerdo irrevocable, con su cuenta y razón. O anhelaba
descanso, vida doméstica, goces más tranquilos que los del poder,
despojado ya de todo encanto para ella, o vislumbrando un porvenir de
dificultades insuperables, hacía la jugada de endosar al vecino su parte
de responsabilidad. Cualesquiera que fuesen los móviles, estrategia o
fatiga, ello es que la Soberana y el Soldado se separaron cada cual con
su tema. No hubo acuerdo más que en la conveniencia de que sólo el
Gobierno supiese la grave resolución, y de que al día siguiente se
celebrara Consejo para discutirla.

Pero ¡ay!, el Gobierno no fue más afortunado que su Presidente: los
pobres ministros, que se creían en situación muy desairada ante una
Reina que, mientras tomaba juramento, tenía guardado el escrito de su
renuncia en la gaveta de la mesa donde estaban el Crucifijo y los
Evangelios, hablaron sin tasa para disuadirla. Todo inútil. «Yo me voy,
yo me voy, y yo no puedo más.» Con esta misma tenacidad categórica
rechazaba María Cristina todos los extremos del programa ministerial,
negándose a suspender la ley de Ayuntamientos y a reconocer la legalidad
de las Juntas, y abominando de la co-Regencia.

«¿Por qué Vuestra Majestad no nos dijo todo eso antes de hacernos
jurar?»

---Porque no podíamos prescindir del juramento, señores míos; porque era
forzoso que hubiese un Ministerio en quien resignar el poder, para que
la Nación no quedase sin gobierno.

Ni con estas razones ni con otras que expuso la dama se dieron por
convencidos, y acordaron dejar en suspenso la discusión, celebrando
nuevo Consejo al siguiente día. En el intermedio preparose Cristina de
nuevas armas dialécticas, que fácilmente encontraba en el arsenal de su
camarilla, y el Ministerio, tras una fatigosa disputa en que la fuerza
lógica de seis hombres de autoridad se estrellaba en la tenaz porfía de
un ser débil (hecho en verdad muy humano, que ocurre constantemente en
el orden privado), se declaró vencido\ldots{} Espartero y los suyos
hubieron de aceptar la situación creada por la renuncia; mas no se puede
determinar, a estas distancias cronológicas, si al acto de aceptar el
hecho acompañó tristeza o alegría de los corazones. La actitud de
Cristina tomaba toda su fuerza de la propia debilidad mujeril y del
respeto y exquisitas consideraciones con que era forzoso tratarla. Había
pronunciado con toda la majestad del mundo un \emph{ahí queda} eso, y ya
podían venir a predicarle abogados, generales y hacendistas. Si estos
querían hacer un poco de historia con elementos más o menos políticos y
literarios, ella sabía componerla con un mohín tan enérgico como
gracioso, con un rasgueo de abanico y un estira y afloja de los
expresivos hoyuelos.

No tardó en hacerse público el estupendo caso, y cada cual lo comentó
como quiso, prevaleciendo el criterio de que Doña María Cristina daba
muestras de gran patriotismo, quitándose de en medio para que viniesen
otros \emph{a labrar} la felicidad de la patria. Entre tantas opiniones,
el historiador debe preferir las que rompían los vulgares moldes del
juicio de los más, revelando en su propia extravagancia un cierto poder
de adivinación. A la tertulia del cuerpo de guardia de Palacio asistía
diariamente un señor de edad madura, a quien llamaban \emph{Don
Nicolás}, no se sabe por qué, pues no era este su verdadero nombre.
Gustaba de andar entre militares, sabía revolver la historia de su época
y apuntar sobre cosas y personas juicios muy donosos. Valenciano neto,
poseía la perspicacia levantina, el decir sentencioso, y un sentido de
la realidad que los ribereños del Mediterráneo deben a la frecuencia con
que les visita el espíritu de Maquiavelo.

\emph{D. Nicolás} expresó una opinión que fue motivo de risa y chacota
entre los circunstantes, bebedores de café y copas, fumadores de
tagarninas. «Pues la razón de todo esto---dijo,---es el odio que la
señora ha tomado a Espartero. Le aborrece; no puede matarle con su
autoridad, y le mata con su dimisión. La cosa es bien clara. ¿Cuál es
para Cristina la mejor manera de hundir al Duque y de inutilizarle para
siempre? Un hombre, un Rey, le arrancaría de las manos el bastón de
generalísimo. Una mujer posee otros medios de venganza y castigo más
eficaces\ldots{} ¿Qué es ello? Pues ponerle en la situación de que la
patriotería le haga Regente. Cátate Regente por virtud y gracia de los
patriotas, cátate perdido. Esto es juego muy fino, señores, la quinta
esencia del saber político y humano. Para poseer esta ciencia sutil hay
que ser de la otra banda, haber nacido al pie del Vesubio o del Etna.
Acá somos más llanotes y atacamos al enemigo por lo derecho\ldots{}
¿Qué, se ríen? Le da la Regencia; él la toma; y ella, sentadita a la
otra orilla, le ve patalear y hundirse\ldots»

Rieron, porque si el juicio era tan disparatado que \emph{no merecía los
honores de la refutación}, en él resplandecían la originalidad y el
ingenio. Por toda Valencia cundía, entre carcajadas, con el estribillo
de \emph{Cosas de D. Nicolás}.

\hypertarget{viii}{%
\chapter{VIII}\label{viii}}

Ya en el trance de dar forma legal a la renuncia, el Gobierno se aplicó
a endilgar del mejor modo posible la página histórica, para que los
venideros tiempos no tuvieran nada que decir en punto a formalidades, y
allí hubo de lucir todo su talento el que luego adquirió fama
imperecedera, D. Manuel Cortina, hombre muy fuerte en jurisprudencias y
en el conocimiento de la humanidad. Resultaba dificilísimo fundamentar
la renuncia de la Gobernadora, que en 16 de Septiembre había dicho en un
decreto famoso que \emph{satisfaría las necesidades de los pueblos}.
¿Con qué razones se justificaba la ligereza de negar en Octubre lo que
un mes antes había ofrecido conceder? Aquí del ingenio político, aquí de
las elasticidades del pensamiento y de la palabra, para concertar un sí
con un no y fundar encima el catafalco de la renuncia. Si por su
entendimiento descollaba Cortina, no valía menos por la rectitud de su
conciencia; y no hallando razones públicas con que motivar ante la
posteridad el paso de la Reina, creyó que debía buscarlas en el orden
privado. Demostró en ello más inclinación a resolver todo conflicto con
resortes humanos que con artificios forenses, y rebosando de sinceridad
y buena fe, propuso a la Reina que por cimiento de la dimisión se
pusiera el hecho firme, bajo el punto de vista legal, de su casamiento
morganático.

Debe decirse que si lo del casamiento no era más que un rumor, la
naturaleza maligna del caso le daba tanto crédito, que ya en 1840
poquísimas personas lo negaban. Últimamente, la desavenencia ruidosa
entre Cristina y su hermana contribuyó a difundir el secreto, pues Doña
Carlota, refugiada en París, no halló mejor modo de distraer los ocios
de su proscripción que refiriendo con pormenores de verdad todo el
idilio palatino y morganático. Se cuenta que Su Alteza patrocinó un
libelo que sobre la regia historia escribieron plumas venales en la
capital de Francia, el cual no pudo ver la luz pública porque nuestro
Embajador, Marqués de Miraflores, se cuidó de recoger toda la edición y
destruirla, no sin que se escaparan algunos, muy poquitos, ejemplares.

Bueno, Señor. El sabio, el íntegro Cortina, que creía verdad lo del
casamiento, y sin duda no lo tenía por delito, sí por impedimento para
ejercer la Regencia, se atrevió a ser sincero con Su Majestad. Mas la
viuda de Fernando VII no juzgó que había llegado aún la oportunidad de
hacer público aquel suceso, o entendía que su figura histórica se
achicaba enormemente si aparecía prefiriendo la actitud amorosa a la
política, y sin mostrar sorpresa ni indignación denegó el caso. Ya no
tuvo más remedio D. Manuel que devanarse los sesos para construir el
castillete retórico que debía ser una página más de esa historia
falsificada que elaboran diariamente los gobiernos con ideas muertas y
palabrería de mazacote, historia indigesta, destinada al olvido. Otra
cosa será cuando no haya tanta distancia entre la psicología de Reyes o
gobernantes y los moldes de la \emph{Gaceta}; entonces tendremos la real
historia escrita al día. Pero es muy dudoso que este tiempo llegue;
resignémonos a una vida de ficciones, y a recoger los granitos de verdad
que a duras penas extrae la observación del fárrago indigerible de la
literatura oficial.

Aplicáronse los señores ministros a resolver diversos problemas
secundarios, nacidos de la renuncia, tales como la cuestión de tutela,
la disolución de Cortes, etc., y no se cayó el firmamento, ni subió el
vino, ni vieron los españoles la menor alteración en su vida bonachona.
Comía el que tenía qué, y todos hablaban cuanto querían de lo humano y
lo divino, derrochando su aptitud crítica, que era y sigue siendo la
virtud o el vicio del siglo.

Santiago Ibero, cuyas tristezas se exacerbaron cruelmente en los días de
la renuncia, por los motivos que él mismo dirá, se fue una mañana, la
del 10 según los informes más autorizados, a la residencia del Duque su
ilustre jefe, y solicitó audiencia de la señora Duquesa, que aquel día
no prestaba servicio en Palacio al lado de la Reina. Tras corta antesala
se dignó la señora recibirle, y no manifestó en aquella ocasión crítica
toda la afabilidad que en su bello rostro hallaban comúnmente los que
tenían la dicha de tratarla: sin duda la inquietaba la próxima partida
de la Reina, y anticipándose mentalmente a volver aquella hoja
histórica, veía quizás obscuras y garrapateadas las páginas siguientes.

«¿Qué traes por aquí, Santiago?»

Se sentó indolente, señalándole el asiento próximo. Como Ibero, indeciso
y turbado, permaneciese en triste mutismo, continuó la dama: «¿Qué te
parece de esta renuncia? ¿Has visto cosa más inesperada y sin
fundamento? ¿Qué opinas tú?»

---¿Yo, señora? Nada, absolutamente nada---replicó el coronel con toda
su alma.---No he tenido tiempo de pensar en ello, abrumado por\ldots{}
En fin, no quiero aburrir a usted con mis lamentaciones.

---Sí, hijo; no hagas el Jeremías, que no estamos para llorar. ¿Qué te
pasa?, dímelo de una vez.

---Vengo a suplicar a usted que interceda con el señor Duque para que me
mande a Vitoria. Me ha dicho el ayudante del Sr.~Linaje que el mismo día
de la partida de la Reina saldrá \emph{El Príncipe} para Madrid. Yo, que
en tiempo de guerra jamás solicité cambio de destino, en tiempo de paz,
y viendo una absoluta incompatibilidad entre mis intereses particulares
y el real servicio, estoy decidido a pedir la absoluta si no se me manda
al Norte.

---¿Y por qué esa prisa de ir a Vitoria? ¿Qué se te ha perdido allí?

---Se me ha perdido, o se me quiere perder, lo que para mí vale más que
cuanto existe en el mundo. Perdone usted: debí empezar por ponerla en
antecedentes, para que se haga cargo de las causas de mi desesperación.
En la carta que recibí momentos antes de saber la renuncia de la
Reina\ldots{} parece que el demonio lo hace, señora: mis alegrías y mis
penas coinciden con los sucesos políticos más graves\ldots{} pues
momentos antes recibí una carta\ldots{} ya me esperaba yo este jicarazo,
que se me había anunciado en cartas anteriores\ldots{} Total, que la
familia quiere que rompa a todo trance, porque se ha determinado que
Gracia dé su mano al Marqués de Sariñán, a fin de unir las casas de
Idiáquez y Castro-Amézaga.

---¡Dios nos asista!\ldots{} ¿Pero es ella quien te lo propone?

---Ella, movida, según dice, de la obediencia, del respeto a los
superiores\ldots{} Bien quisiera protestar de tal tiranía; pero se halla
sin fuerzas para la rebelión: su voluntad, no muy fuerte, se halla
cohibida por la de su hermana, que es, como usted sabe, la que piensa y
obra por las dos. A usted sorprenderá, como me ha sorprendido a mí, que
Demetria, la gran Demetria, sacrifique la felicidad de su querida
hermana por el marquesado de Sariñán.

---Santiago Ibero, tú no estás en tus cabales, y la pequeñuela de Castro
juega con tu corazón, sin duda para ponerlo a prueba. Eres un niño; el
amor te tiene tan ciego, que no ves toda la picardía de ese angelito
juguetón de quien te has enamorado.

---Quizás habría pensado como usted si con la carta de Gracia no hubiera
recibido otra de Navarridas en que me canta la misma tonadilla\ldots{}
que renuncie, que no insista; que la familia determina otra cosa por
razones muy respetables\ldots{} y todo ello en un tono seco y
autoritario que me ha puesto, como usted ve, fuera de quicio, y con
ganas de adoptar los medios revolucionarios. No me resigno, señora; no
me estimo en tan poco como Navarridas quiere tasarme. Quiero que el
señor párroco de La Guardia me diga esas cosas en mi cara; que Demetria
también me las diga\ldots{} que no me lo cuenten por cartas\ldots{} que
me suelten el tiro a boca de jarro si se atreven a ello\ldots{} Decidido
estoy a todo: si el jefe no accede a lo que le pido, me iré de paisano.
¿Qué vale ya mi carrera militar, ni para qué la quiero?

---Pero, tonto, si pides la absoluta, bien podría ser que te hicieran
menos caso. Pongamos que convenzo a Baldomero y te da el mando de un
regimiento de los que están en el Norte: \emph{Farnesio, Cuenca}, no
sé\ldots{} Vas, llegas\ldots{}

---Y me persono en La Guardia, y pido explicaciones, y propongo a Gracia
la rebeldía, la evasión, la fuga\ldots{} Cerco la casa, la incendio;
arrebato a Gracia, la robo, hago el trovador: no me arredran los lances
de comedia\ldots{} Y si no pudiera conseguir lo que intento, porque la
familia, el enemigo, se me anticipara con precauciones y defensas, el
volcán de mi alma reventaría por el cráter de la venganza\ldots{} Ya lo
ve usted: sin quererlo me vuelvo poeta\ldots{} y hago versos\ldots{} en
prosa\ldots{} sin que ello me resulte ridículo\ldots{} Pues sí:
¡venganza, justicia!\ldots{} Cintruénigo me la pagará\ldots{} Pegaré
fuego al palacio de Idiáquez, arrasaré la villa, no dejaré piedra sobre
piedra\ldots{} ¿Para qué estamos los militares más que para castigar la
maldad, para meter a todo el mundo en cintura?

Rompió en franca risa la señora Duquesa, y le dijo: «Pues, hijo,
medrados estamos con tus ideas\ldots{} No se os han dado las armas, no,
para que con ellas atropelléis a la gente pacífica, ni para esas
venganzas de teatro. ¡Pues estaría bueno!\ldots{} Santiago, si sigues
diciendo esos disparates, creeré que eres capaz de hacerlos; y Baldomero
que se interesa por ti más que tú mismo, te mandará a un castillo hasta
que te pase la calentura. Ten formalidad, y yo te prometo interceder
para que te dejen ir a ver a la niña, y puedas echar un párrafo con
María Tirgo\ldots{} Vamos, hombre, que no serán las cosas tan negras
como tú las pintas\ldots{} Es que con la paz, los valientes os volvéis
otros, digo yo, y todo el furor de guerra que teníais en el cuerpo os
sale en forma de tonterías, y os ponéis babosos, y qué sé yo\ldots»

---¡La guerra!---exclamó Ibero dando un gran suspiro.---Los días más
penosos de la campaña, aquellos en que me vi en mayores peligros, en que
sufrí más hambres, fueron, ¡ay! los más felices de mi vida\ldots{} Ya no
volverán.

---Ni falta que nos hace. ¿Pues qué, siempre hemos de estar peleando
para dar gusto a estos señoritos alocados?

---No digo que siempre estemos en guerra\ldots{} digo que aquello para
mí era mejor, que me gustaba más.

---Buen provecho te haga. No, no: España quiere ahora paz, y una paz
larguísima, para que prospere todo, hijo, y seamos un pueblo ilustrado y
rico.

---Así lo he pensado yo; pero no me sale la cuenta, señora.

Algo más quería decir; pero le interrumpió la entrada de Espartero.
Levantose Santiago con marcial presteza al sentir el ruido de la
mampara, y dando media vuelta se encontró ante la cara cetrina del
pacificador, que aquel día no revelaba un temple muy favorable a las
conversaciones ociosas.

«¿Qué quiere Santiago?» preguntó casi sin mirarle.

---Quiere que le mandes a Vitoria---dijo la Duquesa entre seria y
festiva, poniendo toda su bondad generosa al servicio de una causa de
amor harto simpática.---Y realmente tiene que hacer allí. Es una
iniquidad que le quiten su novia y la casen por fuerza con otro, a
estilo de comedión pasado de moda. Los Navarridas dan un bofetón al
Ejército español, y esto no debe consentirse.

---¿A Vitoria\ldots?---repitió Espartero, que engolfado en otros asuntos
y pensamientos no se hizo cargo de lo que oía.---¡Válgame Dios, qué
jaqueca nos está dando esa buena señora! Hoy hemos salido locos\ldots{}
¿Pero no comemos, Jacinta? No es que yo tenga ganas; pero hay que comer,
no sólo para vivir, sino para salir pronto de esa obligación de la
comida, y ocuparse uno en lo que ha de hacer por las tardes\ldots{}
Ahora me acuerdo: tenemos que esperar a Cortina, a quien he
convidado\ldots{} Me parece que ya está ahí: ese es de los puntuales.
Santiago, te quedarás a comer con nosotros\ldots{} No hay excusa: yo lo
mando. ¿Con que a Vitoria? Por ahora no puede ser. Ibero irá siempre a
donde yo le necesite, y yo le necesito a mi lado\ldots{} en Madrid.

\hypertarget{ix}{%
\chapter{IX}\label{ix}}

La repetición de este concepto, al siguiente día, quitó a Ibero toda
esperanza de que el General accediese por el momento a trasladarle al
Norte; y para colmo de desdicha, siempre que de esto se le hablaba,
respondía Espartero con mayor severidad y firmeza, tomando a broma lo de
la licencia absoluta, que calificó de chiquillada indigna de un hombre
serio. No tuvo al fin Santiago más remedio que resignarse, ayudándole en
su conformidad la bonísima doña Jacinta, que le prometió escribir a La
Guardia para informarse de la intriga o cábala matrimonial que hacía de
un bravo coronel de ejército un desairado personaje de comedia
sentimental. En los días que precedieron a la partida de la Reina, se
distrajo con las precauciones que hubieron de ser tomadas para impedir
que se turbara el orden, pues corrían voces de que la caterva
reaccionaria produciría un motín en el momento de salir Su Majestad de
la misa en la Virgen de los Desamparados para dirigirse al muelle. El
plan era precipitarse al coche, cortar los tirantes, y haciendo de
borriquitos los señores y pueblo, llevarse a la Real persona con rápida
tracción a Palacio. Así desbarataban caballerescamente todo el plan de
embarque, dando por nulo y sin ningún valor el acto de la \emph{llamada}
renuncia.

Bueno será indicar el extrañísimo estado psicológico de Ibero con
respecto a la Reina, para que a nadie sorprenda que se alegraba de verla
partir, aun conservando hacia ella una simpatía dulce, y compadeciéndola
por la pena de separarse de sus hijas. El amor, que desatado con
violencia desequilibra las facultades y centuplica la sensibilidad y la
fantasía a expensas de la razón, probó de un modo excepcional todo su
poder en el valiente Ibero, llevándole al delirio, y haciéndole ver en
la Naturaleza y en la Sociedad fenómenos y relaciones propias de la edad
primitiva. No llegó ciertamente a un estado de locura como el de
Cardenio; pero sí a creer y sentir como hijo de las selvas, de las
espeluncas o de cualquier otro sitio donde no había civilización, ni
ciencia, ni pacto social, sino rebaños de hombres soñadores y pacíficos
ante los sublimes espectáculos del cielo y de la tierra. El bravo
coronel veía signos de celestial escritura en las dispersas estrellas y
constelaciones, o figuras humanas que recorrían con pausada solemnidad
la inmensa bóveda; animaba con su naturalismo creador los objetos
terrestres, atribuyendo a los árboles, a las peñas, a las sombras de los
edificios, y aun a las cosas más innobles, figura, existencia y
personalidad, y separando todas estas visiones en las dos categorías de
benéficas y maléficas. Un poste, a veces, le miraba con saña; un
ventanucho le sonreía; una caja de cigarros le decía: «cuidado Ibero»
con fraternal interés; una banderola ondeando al viento le gritaba:
«tonto, ¿por qué vives?» Aprendió mil supersticiones sin que nadie se
las enseñara, y mil formas de \emph{jettatura}. Reconociendo él mismo la
ridiculez de aquel trastorno, actuaba sobre sí con la voluntad, y
trataba de quitarse tales tonterías de la cabeza, diciéndose al fin:
«¡Qué bien te vendría, Santiago, que estallase otra guerra, para que los
cuidados y peligros te limpiaran el entendimiento de esta mugre!»

Cuando llegó Doña María Cristina, la peregrina casualidad de que en un
mismo día y hora apareciesen la Reina y la carta que le hizo tan feliz,
fue parte a que Santiago creyera su destino amoroso asociado a la
persona de la Soberana. Los hoyuelos del divino rostro de Cristina eran
la cifra o representación de la divinidad misteriosa que preside al
amor, y ellos le infundían esperanza, le señalaban un camino, le
recomendaban la perseverancia y la fidelidad, anunciándole nuevas dichas
cada vez que en público se mostraban. Pero de pronto el influjo benéfico
de la regia persona trocose en maléfico influjo. Con la renuncia de la
Regente en el mundo político, grande y ruidoso trastorno, coincidieron
en el individual mundo del enamorado, las tristísimas nuevas venidas de
La Guardia en las cartas de Gracia y Navarridas. No fue preciso más para
que la Reina se trocase de ángel en demonio, entendiendo por esto un ser
muy bello, pero de muy malas intenciones, que provoca desastres y ruinas
sólo con una mirada. Otras mil ocasiones de probar la sombra maligna de
la viuda de Fernando VII se le presentaron, pues observó más de una vez
que siempre que la veía, le pasaba algo desagradable. En fin: convertida
la Reina en el genio adverso del buen militar, en la fase negra de su
destino, ¿qué había de desear sino que se marchara? Y para desbaratar su
poder maléfico, convenía que saliese por donde había venido: por la
inmensidad del mar. Mares y cielos traen y llevan las fuerzas invisibles
del mal y del bien.

Conjurado por el Gobierno y las autoridades el peligro de aquel tremendo
complot para no dejar salir a la Reina, se preparó todo para la mañana
del 17 de Octubre. Ibero y otros jefes recorrían desde el amanecer la
carrera, disponiendo la distribución de fuerzas del Ejército y la
Milicia, desde la Puerta del Mar hasta el Grao, y reforzando los puntos
débiles como si se tratara de tomar posiciones para una batalla. Cuando
se aproximaba la hora, vio pasar, camino del puerto, a todos los
personajes que eran figuras de primero y segundo orden en el mundo
oficial, luciendo sus uniformes con bandas y cruces. A las seis de la
mañana, ya el corto muelle del Grao se hallaba tan obstruido por la
muchedumbre de funcionarios, como en tiempos modernos por las cajas de
naranjas en días de embarque. Militares había en gran número, y
magistrados y clérigos, y a unos y otros hubo de señalarles Ibero los
puestos convenidos para que pudiesen ver a Su Majestad y saludarla sin
confusión. Allí estaban, representando al Ejército y la Marina, el
Mariscal de campo Borso di Carminati, el Subinspector de Artillería D.
Casimiro Valdés, el Comandante de Ingenieros D. Juan de Quiroga, el
Comandante del Tercio Naval D. José de Julián. Por el clero, iban el
Chantre de la Catedral Don Miguel Soler, el Magistral D. Vicente Llopis,
el Penitenciario D. Juan Broto y multitud de curas párrocos, señalados
algunos por concomitancias \emph{cabreristas}. De la justicia eran
dignos representantes D. Vicente Fuster, Regente de la Audiencia, y el
Fiscal D. Andrés Ruiz Morquecho, amigo y conmilitón de Ibero. Empleados
de categoría formaban la masa obscura, sin casacas ni relumbrones,
figurando entre ellos el Administrador de Loterías, el de Aduanas, el
comisionado de Amortización, y tras estos los síndicos del Ayuntamiento
y el \emph{Administrador interino de ramos decimales}, que no era otro
que aquel D. Nicolás, filósofo de la historia y profesor de
maquiavelismo.

Antes de las seis llegó la Reina en coche de cuatro caballos; había
recorrido el trayecto desde la casa de Cervellón sin que saliesen las
\emph{turbas moderadas} a desenganchar para ponerse a tirar del coche.
Todo resultaba fácil y corriente en la realidad, y la Gobernadora
dimisionaria salía del Reino sin producir más entorpecimiento que una
partida de naranjas. Tras ella, en otros coches, llegaron los individuos
que la opinión señalaba como figuras culminantes de la camarilla: el
Duque de Alagón, Capitán de Guardias de la Real persona; el Conde de
Santa Coloma, Mayordomo mayor; el Marqués de Malpica, Caballerizo, y
algunos otros, cuya celebridad iguala a su insignificancia. Y seguían
Doña Jacinta y otras damas de la Reina llorando: algunas partían con Su
Majestad, otras se separarían de ella para siempre por disposición de
los hados políticos. Los Generales Seoane y Espartero formaron a un lado
y otro de Su Majestad para conducirla a la falúa. La despedida fue
tiernísima. María Cristina tan pronto se llevaba el pañuelo a los ojos,
como saludaba a la multitud agitándolo, sin poder decir más palabra que
\emph{adiós}, \emph{adiós}\ldots{} Viola Ibero embarcarse y partir sin
apartar los ojos de tierra y del gentío que vitoreaba. Los hoyuelos, si
para todo el mundo eran la afabilidad y el cariño, para él fueron la
expresión de una ironía diabólica.

Íbase al fin bendita de Dios; su ausencia daba al enamorado militar
esperanzas de un cambio feliz de su sino. Era el ocaso de una
constelación adversa, que no volvería, no, a traspasar la línea del
horizonte. Vio Ibero a la Reina subir la escalera del barco y agitar en
lo más alto de ella su pañuelo mirando a tierra. El vapor, que humeaba
ya, presuroso de salir, levó anclas y empezó a dar paletadas, no
tardando en tomar carrera fuera del puerto y en emprender su marcha
ceñido a la costa. El Mediterráneo, tranquilo aquel día, se puso de azul
intenso para recibir y transportar a la \emph{ninfa de Parténope}. Debió
de recapitular la Reina en su mente, mirando las costas españolas de que
se alejaba, los diez años de su vida en nuestra tierra. ¡Qué cosas
pensaría, qué cosas debió de decirse!\ldots{} Recordaría también su
salida de Nápoles en 1829, cuando vino a casarse con el Rey odioso y
feo, y cotejando aquella salida con la de Valencia, diez años después,
quizás pensó que su vida transcurría entre volcanes: allá el Vesubio,
aquí la Guerra Civil, y tras esta la inmensa pira del \emph{Progreso},
que no esperaba más que una mecha encendida para arder por los cuatro
costados\ldots{} A tiempo se iba, después de haber desempeñado un
glorioso papel político. Si enemigos crió, de amigos y sectarios
entusiastas dejaba también buena empolladura. Hijos no le faltaban; que
la Naturaleza habíala hecho bien prolífica, y si dos tiernas criaturas
quedaban aquí, otras hallaría en Francia, sin contar lo que viniera
después. Más satisfecha como mujer que como Reina, se consolaba de sus
desgracias políticas considerando la dificultad del cargo. Pero, en
conjunto, no le había sido adversa la fortuna, y recapitulando al son de
las paletadas del vapor, le salía más crecida la cuenta de los bienes
que la de los males.

No tardó en perderse el vapor \emph{Mercurio} mar afuera, y a las diez
de la mañana, los \emph{moderados} doloridos que desde el Miquelete o en
altos miradores seguían con catalejos el curso de la nave por la azul
inmensidad, no descubrían ya más que un tiznón sobre el horizonte. Por
allí iba\ldots{} ¡Qué dolor! ¿Volvería?\ldots{}

\hypertarget{x}{%
\chapter{X}\label{x}}

Antes de terminar Octubre, ya estaba Ibero de nuevo en Madrid, hastiado
del viaje de regreso, igual al de ida en aburrimiento y monotonía, sin
más diferencia que la producida por el estado atmosférico, pues si le
achicharraron en verano los calores, en otoño las pertinaces lluvias le
mojaron y refrescaron más de lo que quisiera. Fue casi todo el camino en
la custodia y acompañamiento del General Espartero, viéndose obligado a
presenciar unas cuarenta ovaciones en pocos días. Habríale gustado dar
convoy a la Reina y a su hermanita; pero casi todo el camino fueron una
o dos jornadas por delante, con su lucido acompañamiento de damas,
caballerizos, escolta y numerosísima servidumbre. Sólo en la subida de
las Cabrillas las vio y fue junto al regio coche un buen trecho. Por
cierto que iban las dos niñas muy monas, picoteando con las damas que
ocupaban la delantera, y dirigiendo a cada instante su voluble atención
con juguetona risa hacia toda novedad de cosas o personas que hallaban
en el camino. Por cierto que se fijaron en el Coronel, y aun le hicieron
un poquito de burla, porque habiéndose interpuesto unos gitanos que
bajaban el puerto con media docena de jumentos, se desorganizó un tanto
la marcha de coches y jinetes. Ibero trató de restablecer el orden,
arreó latigazos a los borricos, y la gitanería defendió su derecho al
camino con graciosos denuestos. Tal incidente fue muy del agrado de las
niñas, y la incomodidad de Ibero, no proporcionada quizás al motivo del
lance, les hizo mucha gracia.

Desde aquel día las niñas se adelantaron y no las vio más. Iba el
Coronel en compañía de personas fastidiosas, de funcionarios sin ninguna
amenidad, que no hablaban más que de política, como si nada existiese en
la Naturaleza digno de atención. El 28 llegaron a Madrid, siendo
recibido el Gobierno Provisional o Ministerio-Regencia, que de ambos
modos se le llamaba, con todas las músicas disponibles y con las
aclamaciones de ritual. El mismo día de llegada se confirió a Ibero el
mando de \emph{Saboya} (6.º de línea), acuartelado en el Pósito, y la
primera ocupación del Coronel fue arreglar su instalación personal no
lejos del cuartel y de la Inspección de Milicias, donde fue a morar el
Duque con su familia. El 30 tenía ya su acomodo en una casa de la calle
del Turco, agregándose a una familia riojana en calidad de huésped, pues
firme en su tenaz idea de marchar al Norte a la primera ocasión que se
presentase, no quiso poner casa ni embarazar su libertad. El continuo
trajín militar, la dignidad de su mando, y más que nada las noticias
consoladoras que recibió de La Guardia, aplazando el conflicto y
reverdeciendo esperanzas, le aliviaron grandemente de su mal, y su mente
se despejó de aquellos delirios supersticiosos que le habían atormentado
en Valencia. Alguna vez le sobrecogían temores hondos, sin otro motivo
que presenciar la caída de una cafetera, o escuchar la desafinada voz de
un ciego que pregonaba el \emph{Huracán}. Pero se dominaba, consiguiendo
llevar a sus nervios la disciplina, a su razón la luminosa fuerza.

Reanudando sus amistades de otros días, encontró a Bretón amable y
gracioso, a pesar de las tristezas de su cesantía. Por ley que parecía
obra de la Naturaleza, tal era su regularidad, el nuevo régimen le había
separado del comedero de la Biblioteca, para poner en él a persona más
conforme con las ideas dominantes; frecuentaba Ibero su trato y el de su
familia, gozoso de la paz de aquella casa, donde moraban la honradez, la
modestia y todas las gracias castizas en verso y prosa. De muy distinto
género era la amistad de González Bravo, el periodista impetuoso del
\emph{Guirigay}, el que se puso a la vanguardia del motín de Septiembre,
penetrando a la cabeza de los primeros grupos en el Ayuntamiento. El
triunfo del pueblo había hecho de Luis González un energúmeno: en vez de
aplacarse con el acabamiento de la tiranía moderada, se inflamaba más en
ardor patriotero y en ansias de libertad. Se decía que, contrariado
porque no le habían metido en la Junta, quería llevar las cosas a los
extremos de la licencia y la anarquía, ayudado de su amigo Nocedal, tipo
del perfecto miliciano, el primerito en el servicio como en las
asonadas. Más desinteresado que estos, movido de su loco idealismo, como
poeta, y de un sentimiento popular sano y hermoso, Espronceda escribía
con un rayo, pidiendo, no ya la libertad, sino la República. No se
paraba en barras; no sabía contenerse retorciéndose y achicándose dentro
de los moldes circunstanciales, ni quería mantenerse en el terreno común
a moderados y progresistas. Su ardiente imaginación, su temple audaz,
familiarizado con el libre vuelo del pensamiento, le lanzaba a las
grandes empresas, y las acometía presagiando la inutilidad de sus
esfuerzos. Pero ¿qué le importaba si satisfacía su ideal y se recreaba
con los fantasmas creados por sí mismo? No tardó Santiago en afirmar la
amistad que en el verano había contraído con Espronceda, afinándola y
robusteciéndola con recíprocas confidencias. Bien conocía el alavés que
las ideas de su amigo eran irrealizables, ideas poéticas y de otro
mundo, ¡pero qué hermosas! Arrancaban del pasado y nos conducían a un
porvenir risueño; se fundaban en lo más hermoso de nuestra alma, y
pertenecían al propio tiempo al ensueño y a la razón. Contradiciéndole,
movido de los respetos inherentes a su posición militar, el Coronel
gustaba de oírle, y le incitaba a desbocarse por los espacios donde
jamás penetró el pensamiento de los hombres comunes. Era Espronceda el
vate político, y bajo su influjo la religión liberal de Ibero se iba
convirtiendo en un culto secreto de dioses lejanos.

Muy distinta era la amistad que reanudó con el buen Milagro, pues en
este no veía más que un pobre maniaco inofensivo, de estos que lo
sacrifican todo al ansia de vivir y a las complejas necesidades de que
se ven cercados. Infatigable en su propósito de no quedarse atrás en la
procesión social, D. José había logrado meterse en la Secretaría
provisional de la Junta, y tales servicios prestó allí desplegando todo
su saber burocrático, que a la llegada del Ministerio-Regencia halló
fácil medio de colarse en Gobernación con \emph{veinte mil}, y a los
pocos días se le indicaba para un puesto de jefe político en provincia
de tercer orden. Transformado de ropa y cara le encontró Ibero, pues se
había rapado completamente al uso antiguo, quitándose el bigote de moco
que le sirviera de emblema revolucionario, y se había provisto de la
ropa indispensable para un funcionario de su fuste, que pronto tomaría
el mando de una provincia. Sorprendió a Ibero el aire de dignidad y
mesura que en sus ademanes ponía el buen Milagro, y las ideas sensatas
que derramaba de su boca.

«Vea usted, Sr.~D. Santiago---le dijo la segunda vez que fue a visitarle
en su oficina,---cómo se ha realizado lo que yo presagié con buen ojo.
Ya tenemos el gobierno del pueblo por el pueblo; ya no hay tiranías
palaciegas ni camarillas indecentes; ya no hay más que legalidad,
justicia, y libertad perfectamente hermanada con el orden. Ahora
procuramos que el gobierno de la Nación entre en su cauce natural,
cesando en sus funciones la Junta de Madrid, después de cumplida su
misión salvadora. Cierto que algunas juntas de provincias no quieren
disolverse; pero la razón a todas se impondrá\ldots{} ¡Qué cordura la de
nuestro pueblo! ¡Qué energía en la acción, qué prudencia en el triunfo!
Aquí vienen todos los días en la prensa de Inglaterra y Francia
demostraciones de lo que nos admiran los extranjeros. Pasar del
despotismo a la libertad sin derramamiento de sangre es gran cosa,
¿verdad? Ahora se verá lo que es España y qué reformas, qué progreso,
qué adelantos\ldots{} No dirá usted que se duermen los ministros, pues
cada día larga la Gaceta un decreto reformista que da gusto\ldots{} Así,
así se gobierna. Luego tendremos el gran problema de la Regencia, que
resolverán las Cortes legalmente elegidas. ¡Y qué Cortes!\ldots{} las
más liberales, puede usted decirlo, que se han visto desde que tenemos
régimen. ¿Será la Regencia una o trina? Eso lo dirán los doctores
políticos. Luego, con un \emph{cúmplase}, queda todo concluido\ldots{}
Aquí me tiene usted sacrificándome por la patria, pues el Ministro me
retiene hasta media noche\ldots{} Como hay tanta gente nueva, no saben
por dónde andan. Luego la taifa moderada dejó esto en el mayor desorden:
no venían aquí más que a fumar cigarrillos y a hablar mal de
Espartero\ldots{} Y a propósito: ¿qué tal está el Duque? Dijeron ayer
que se había metido en cama molestado por un ligero ataque de retención
de orina. Yo sé lo que es eso, y empleo la zaragatona, uso interno y
externo. Recomiéndeselo usted, que le ve todos los días\ldots{} Que se
cuide, que se cuide, pues él es la columna en que descansa todo este
gran edificio de la libertad\ldots»

Hablaron luego del amigo D. Bruno Carrasco, con quien conservaba Milagro
relaciones muy cariñosas. La circunstancia de tener amistad antigua y
aun algo de parentesco por afinidad con el Sr.~Gamboa, Ministro de
Hacienda, fue para D. Bruno como la venida del Espíritu Santo, pues a
más de prometerle resolver a su gusto el expediente de Pósitos, el
gobierno deseaba utilizar sus servicios en la administración,
nombrándole para una plaza de consejero de Hacienda o cosa tal. Había
llegado el reinado de los buenos, el predominio absoluto de la honradez.
A la sazón estaba Carrasco en su pueblo, ocupado en la faena de levantar
la casa para venirse a vivir a Madrid con toda la familia, apretándole a
ello el vivo afán de aplicar su inteligencia y su respetabilidad a la
cosa pública. «Eso, eso es lo que nos hace falta, señor mío---decía
Milagro con enfática suficiencia,---y eso es lo que yo sin cesar
predico. Que vengan a Madrid los hombres pudientes a dar tono a la
política, para que esta no sea patrimonio de cuatro danzantes\ldots{} Si
me promete usted la reserva, amigo Ibero, le confiaré un proyecto que
\emph{acariciamos} Carrasco y un servidor de usted. Aún no sé qué ínsula
me darán; pero se trabaja para que esta sea Ciudad Real. Si allá me
mandan, tenga usted por cierto que Bruno sale diputado\ldots{} vaya si
sale. Allí tiene arraigo, bastante propiedad, numerosos amigos y
deudos\ldots{} y a mí, \emph{a mí que las vendo}, ja, ja\ldots{} No le
digo a usted más.»

Anunciole también D. José que viéndose mejorado notablemente de recursos
pecuniarios y de posición social, había traído a su lado a los niños
pequeños y a las hijas mayores, la esposa del bajo y la del subteniente
Piquero, separada de su marido por la mala vida que este le daba.
Contento de la restauración de su hogar, había tomado un pisito en la
calle de las Infantas, modesto asilo que se permitía ofrecer al Sr.~de
Ibero, por si gustaba de honrar a la familia con su presencia, en los
ratos de ocio. Allí no encontraría lujo ni etiquetas; pero sí
cordialidad, franqueza y alegría. Las muchachas eran muy instruiditas
para lo que aquí se acostumbra; aficionadas a la música y a la
literatura, y en el poco tiempo que llevaban de su nueva instalación en
Madrid, comenzaban a formarse un núcleo de excelentes relaciones.

Agradeciendo mucho la oferta de casa, prometió el Coronel no privarse de
la honra y agrado de tal sociedad. Milagro, al despedirle, se condolió
de que no fuese la dicha completa en su familia, pues si su hija mayor y
los chiquillos no le daban ningún motivo de tristeza, érale muy penosa
la situación de su hija menor, esposa de un perdido y separada de él: ni
casada, ni soltera, ni viuda\ldots{} ¡Qué dolor! Gracias que la niña era
un ángel, la misma virtud. D. José padecía lo indecible viéndola en
aquel divorcio de hecho, criatura perdida para los grandes fines
sociales, destinada a vivir como una monja en el hogar paterno. Véanse
aquí las consecuencias de un mal matrimonio, contraído precipitadamente,
por ventolera irresistible de la muchacha y audacias del mozalbete. En
fin, ya no tenía remedio. «Mis hijas---agregó D. José,---son dos obras
maestras, aunque me esté mal el decirlo, notables por su talento y por
todo lo tocante a la exterioridad, belleza, donaire, \emph{etcétera}.
María Luisa, que era una notabilidad en el arpa, ha descuidado el
instrumento desde que se casó; pero la obligaremos a estudiar un poco
para que usted la oiga. Su esposo, Romano Cavallieri, es uno de los
primeros bajos del mundo, y en Madrid no hay otro que ponga más alta la
buena escuela italiana, así en ópera como en funerales. Mi hija segunda,
Rafaela, fue siempre tan suave por la figura, los modales, las
aficiones, y al propio tiempo tan melosa y atractiva en su manera de
hablar, que unas vecinas nuestras dieron en llamarla \emph{Perita en
dulce}, y en casa casi siempre le dábamos ese nombre. Después de su
infeliz matrimonio nada ha perdido de su dulzura y delicadeza: al
contrario, parece que la conformidad con su desgracia la hace más tierna
y cariñosa\ldots{} En fin, amigo, no se ría usted de mis debilidades
paternas\ldots{} Nada quiero decirle a usted de los chicos pequeños, que
han hecho en Illescas unos exámenes brillantísimos. ¡Si viera usted las
planas que me ha traído el mayorcito! Creo que serán hombres de
provecho, buenos ciudadanos, buenos \emph{progresistas}\ldots{} Quedamos
en que usted nos honrará\ldots»

Con una afirmación cordial se despidió el Coronel, que desde el
Ministerio se fue a casa de Espronceda, y después a casa de Olózaga, y
de allí a ver a Pacheco, a quien había conocido en Valencia, y luego al
café, donde encontró a varios militares amigos. Así mataba el tedio con
sucesivas y amenas visitas, y si no lo mataba lo hería gravemente.

\hypertarget{xi}{%
\chapter{XI}\label{xi}}

¿Se disolvían las juntas? ¿Sería disuelto el Senado? ¿Era cierto que el
Infante D. Francisco salía con la gaita de reclamar la Regencia? ¿Qué
tal el Manifiesto de la Reina Cristina, tronando contra la situación que
había creado con su renuncia? Vamos, que el Ministerio-Regencia no se
mordió la lengua en la refutación de aquel documento. ¿Qué había del
conflicto eclesiástico? ¿Nos quedábamos sin Nuncio, absolutamente
incomunicados con la Corte y curia romanas? ¿Qué se decía de casamiento
de príncipes y princesas brasileñas con infantes e infantas españolas? Y
a la Reinita, ¿con quién la casaban?\ldots{} De todas estas cosas y de
otras menudencias políticas y sociales que en aquellos días (ya entrado
Noviembre) fatigaban la opinión, habló y oyó hablar Ibero en sus
primeras visitas a la modesta casa de Milagro. Fue allí por añadir un
recurso más a los que empleaba para combatir su aburrimiento, y en
verdad que no le pesó, pues la familia era muy agradable, las niñas muy
despiertas, el bajo muy complaciente, y en la tertulia nocturna,
alrededor de la camilla, no faltaban señoras dicharacheras ni aun
\emph{hombres políticos} que decían cosas muy atinadas sobre los
problemas del día. Los chiquillos pequeños eran el único desconcierto de
la grata armonía doméstica, porque no brillaban por su buena educación,
ni sabían hacerse agradables en la edad que precede al último estirón de
la infancia. Eran pegajosos, entrometidos, preguntones y cargantísimos.
Pero, en fin, a esta pejiguera servía de compensación la discreta
amabilidad, la risueña juventud de María Luisa y Rafaela.

Ya llevaba Ibero algunos días de conocimiento y no podía conseguir que
María Luisa tocase el arpa. Se excusaba pretextando rigidez de dedos por
el abandono de la ejecución en largos meses; y en tanto, como el
instrumento padecía también de la dilatada inacción, el bajo, que era
hombre para todo en cosas musicales, se pasaba las horas componiéndolo y
echándole nuevas cuerdas. María Luisa no había dado aún nietos al buen
Milagro; a los siete meses de casada, un mal parto malogró las
esperanzas paternales, que de nuevo reverdecían en el invierno de 1840;
y como se hallaba ya de cinco meses, a D. José no le gustaba que se la
instara demasiado a lucir sus habilidades de arpista, no fuera que con
el ajetreo de pies y manos y con las sofoquinas que suele producir la
inspiración, cuando es de ley, se malograse el fruto. El pobre
Cavallieri era un hombre excelente, conocedor de sus deberes como
presunto padre de familia. A pesar de hallarse sin contrata, pues por no
lanzarse a viajes costosos había rechazado las que le propusieron de los
regios teatros de San Petersburgo, Londres y Nápoles, sabía traer dinero
a casa, sacando un jornal de todas las solemnidades religiosas y de los
funerales de primera. Además daba lecciones de canto, y también componía
su poco de música, ora un invitatorio, motete o tanda de villancicos,
ora alguna canción con letra de Espronceda para acompañamiento de
guitarra. En casa era de una seráfica mansedumbre: respetuosísimo con su
suegro, obediente a su mujer, sin exigencias en las comidas, dispuesto a
todo, aun a cosas tan contrarias al arte de Rosini como el planchar
vuelillos y el peinar a las señoras.

La casa era modestísima, los muebles viejos y descabalados, simbólica
expresión de la vida procelosa de Milagro y de las cesantías, traslados
a provincias y demás accidentes de la vida del funcionario público en
esta desordenada tierra. Notó Ibero los apuros que había cuando los
visitantes vespertinos o nocturnos excedían del número de sillas,
contando para los \emph{grandes llenos} con las de la cocina. Mas no por
estas escaseces de mobiliario, ni por otras faltas que a cada paso se
ponían de manifiesto, perdían aquellos benditos el gusto de la vida
social, y cada vez querían atraer y recibir a más personas, sin reparar
que fueran de mejor pelo y de clase superior a la suya. No les era
difícil sostener la casa con el sueldo de Don José y las ganancias no
deslucidas de Cavallieri: la experiencia de Milagro y sus dotes de
gobierno impusieron desde el primer día el sistema salvador de gastar
menos de lo que ingresara, y por nada del mundo se alteraba este método,
al que debían la tranquilidad, un comer apropiado a las necesidades, y
una vida, en fin, decorosa, aunque humilde. Los chicos no iban rotos a
la escuela, ni D. José a la oficina con facha indigna de su posición;
para todo había, y aun se juntaba duro a duro el presupuesto de
sastrería que había de dotar a Milagro de todas las prendas
indispensables a un jefe político.

La menor de las hermanas, la que, según el dicho de su padre, no era
viuda, ni casada, ni soltera, Rafaela, en fin, por mote familiar vigente
aún, la \emph{Perita en dulce}, daba quince y raya a todos en lo
hacendosa y hormiga para su atavío particular. Otra que mejor y con más
gusto se arreglara los cuatro pingos que poseía, y los lazos, cintas y
moños, no ha existido en Madrid. ¿Qué arte secreto era el suyo para
vestirse y emperifollarse, y qué hacía para parecer tan bien con su
trajecillo pobre y con cualquier trapo de bien combinados colorines que
se pusiera? Verdad que ayudaban al mágico efecto su rostro bonito y la
perfecta conformación de su talle; pero algo más había, y era un
instinto, una adivinación, y el conocimiento genial de todas las modas y
sus cambios, sin sobar figurines ni andar entre modistas. Descollaba
Rafaela entre sus iguales como la rosa entre mastranzos; su superioridad
consistía quizás en que nunca delató con la afectación su prurito de
elegancia, en que su sencillo atavío no revelaba el estudio previo y
paciente para obtener tan feliz resultado. Era su rostro finísimo y algo
picaresco, de un estilo (si estilo hay en las formaciones de la
Naturaleza) que bien podría llamarse Pompadour, pintiparado para el
traje de pastoras de abanico, con empolvado pelo, corpiño estrecho y
espléndidas faldas recogidas. El pelo era rubio, la tez de una blancura
porcelanesca, los ojos obscuros, reveladores de amor, de ensueño, a
veces de inteligente malicia.

No se creía D. Santiago infiel a su compromiso de amor porque la
\emph{Perita en dulce} le gustase. Le gustaba, sí; pasaba ratos muy
entretenidos a su lado; pero todo el goce que recibía en ello era
superficial y no le llegaba al corazón. Le divertían los conceptos
extravagantes que expresaba Rafaela sobre cualquier tema de los usuales
de la vida, y reconocía en ella una inteligencia no común. «¿No les
parece a ustedes---dijo la \emph{Perita} una noche, no hallándose
presente su padre,---que esto de la libertad es una paparrucha? La
libertad, como el retroceso, ¿qué son sino los motes o letreros que se
ponen estos o los otros señores para mangonear? ¡Ay, Virgen del Rosario,
que no me oiga mi padre!\ldots{} Me mataría.»

Oído aquel disparate gracioso, le soltó Ibero un discursillo
enalteciendo las ventajas que obtienen los pueblos del régimen que
felizmente disfrutábamos, y no fueron risas y chacota las de ella para
zaherir tan manoseadas retóricas. «¿Con que libertad? ¿Y para qué sirve
esa libertad? Para escribir en los papeles mil disparates, para insultar
a los ministros y no dejarles gobernar; libertad para los que alborotan,
y entre tanto el pobre, pobre se queda, y los ricos se hacen más ricos,
y nosotras las mujeres seguimos esclavas. Dígame usted qué libertad es
esta que a mí me tiene prisionera de una equivocación. Mi marido es un
mal hombre, y no soy yo quien lo dice: es el juez, es su propia familia,
es todo el mundo. ¿Pues por qué no había yo de poder descasarme y volver
a la soltería?»

---Por mí---dijo Ibero,---que vuelva usted. No me opongo.

---Poco sacamos de que usted no se oponga.

---Las Cortes, el Rey, el Papa, el Concilio de Trento, tienen que poner
mano en eso para reformarlo.

---Pues ya verá usted cómo no lo reforman\ldots{} Tanto hablar de
libertad, y no nos traen el divorcio. Que mi padre no me oiga decir la
herejía de que no tendremos una buena Constitución hasta que no traigan
las reglas de descasar\ldots{} y otras cosas, Señor, otras cosas que por
ahora me callo, para que usted, Sr.~de Ibero, que es tan remilgado y
para poco, no se nos escandalice. En fin, vengan libertad y pobreza, que
me parece a mí que andan unidas\ldots{} Yo, si ustedes no se asustan y
me prometen no contárselo a papá, diré que, a mi modo de ver, en tiempo
del moderantismo y de la camarilla había más dinero.

---¡Qué cosas tienes, mujer!---dijo María Luisa, que no por contradecir
a su hermana dejaba de gozar oyéndola.---¿Más dinero entonces? ¿Dónde?

---En casa no; a eso no me refiero. En Madrid, quiero decir.

---No va descaminada Rafaelita---indicó una señora mayor, esposa de un
compañero de oficina de Milagro, muy rolliza de carnes, y de ideas harto
enjuta, pues no hablaba más que de novenas y modas, o del eterno sisar
de las criadas.---Ello es que en todos los comercios se quejan. Es lo
que dice Gerardo: que aquí los ricos no tienen patriotismo.

---Lo que yo sé---declaró el músico Cavallieri,---es que sólo en los
tiempos moderados se ha sostenido aquí una buena compañía de ópera.
Cuando yo salí en la \emph{Serva Padrona}, ¿se acuerdan?, y cuando hice
el \emph{Don Magnífico} de la \emph{Cenerentola}, era en lo más crudo de
los tiempos ominosos, mandando el Sr.~Cea Bermúdez.

---Hoy me ha dicho Doña Rosaura, la de la tienda de encajes---afirmó
Rafaela,---que desde que han venido los \emph{libres} no venden ni la
mitad. Y en casa de Bárcena, hay días en que no entran en el cajón
arriba de catorce reales. ¡Ya ven\ldots{} una casa como aquélla\ldots!

---El dinero---observó María Luisa,---como dice papá, no se pierde: lo
que hace es ocultarse.

---Pero ya le haremos salir, ¿verdad, Sr.~Don Gerardo?---dijo Ibero
dirigiéndose a un sujeto acartonado, esposo de la no ha mucho citada
señora rolliza.---En Gobernación, según oí, preparan sin fin de decretos
para desarrollar la riqueza pública.

---Así es---replicó D. Gerardo, hombre comedido, discreto, que se oía
cuando hablaba, y no hablaba más que lo preciso; funcionario excelente,
de procedencia masónica de los \emph{Tres años}, que no había llorado
largas cesantías, y usaba en invierno y verano levita muy larga y
sombrero de copa de desmedida elevación.---Ya ve el país que el Sr.~de
Cortina no se duerme. Hombres como D. Manuel son los que han de
regenerarnos. Prepara reformas en todos los ramos, en minas, en policía,
en caminos vecinales, y sobre todo, en Instrucción pública, que es el
\emph{barómetro}, ya lo saben ustedes, \emph{el barómetro de la
civilización de los pueblos}. Con esto, y el buen gobierno de la
Hacienda y las economías, la riqueza pública y privada tomará gran
desarrollo. Un buen Gobierno trae la confianza, y la confianza trae la
riqueza, el curso de los capitales, la circulación del numerario\ldots{}

---Esas son tonadillas, D. Gerardo---dijo Rafaelita, burlándose con
gracia del rígido funcionario,---tonadillas que nos cantan todos
mientras tienen la sartén por el mango. Pero como al fin resulta que lo
que es buen Gobierno aquí no lo hay nunca, tampoco tenemos confianza, y
todo se queda en música: música de himnos, música de discursos, y en
tanto el dinero no parece\ldots{} que es a lo que vamos.

---No hagan caso de mi hermana---dijo María Luisa,---que ha tomado ahora
este tema del dinero por pasar el rato y matar el fastidio. Rafaela
varía de gustos a cada triquitraque; no es como yo, que siempre soy la
misma.

---Cuando eran ustedes solteras---observó la señora rolliza,---pasábamos
ratos muy divertidos oyéndolas recitar versos.

---Sí\ldots{} y los aprendíamos de memoria, y parecíamos cómicas; en
casa no nos podían sufrir. Naturalmente, con la edad cambian los gustos.
El tiempo pasa, y una se va formalizando. Vienen las necesidades, y ante
la cara dura de las necesidades ya no está una de humor de poesías. Pero
a mí me gustan siempre.

---A mí no---dijo Rafaela.---Hace algún tiempo les he tomado una tirria
tremenda. Ellas tienen la culpa de muchas desgracias. Los poetas son los
que traen los malos casamientos, la falta de verdadera libertad y la
pobreza\ldots{} y lo digo\ldots{} y lo pruebo.

Celebraron todos con risas estos donaires de la \emph{Perita en dulce},
y el Sr.~D. Gerardo se permitió defender la poesía, que adoraba por
haberla cultivado sin fruto en su mocedad.

«Hace cuatro noches---dijo,---tuvo María Luisa la bondad de recitar unos
versos divinos\ldots{} Yo me entusiasmé de tal modo, que se me saltaron
las lágrimas. El Sr.~de Ibero no estaba presente aquella noche, y como
yo deseo que oiga lo que oímos, y que goce lo que gozamos, solicito de
la simpática señora de Cavallieri que nos repita la función.»

¡Ay, Dios mío! ¡Qué melindres los de la señora de Cavallieri! No se
acordaba\ldots{} Tenía un poquito de ronquera\ldots{} ¡Qué compromiso!
No sabía recitar sino en familia, o entre amigos muy íntimos\ldots{} Le
daba vergüenza.

Las súplicas de Ibero, a las que unió tímidamente su autoridad
Cavallieri, no vencieron la modestia de la dama. Intervino Rafaela
diciendo: «Lo que recitaste fue \emph{La gloria y el orgullo} de Pepe
Zorrilla. Yo lo sabía también; pero se me ha olvidado. Si lo recordara,
verías qué pronto despachaba yo. No me acuerdo más que de aquel pasaje:

\small
\newlength\mlena
\settowidth\mlena{\quad De un dios hechura, como Dios concibo;}
\begin{center}
\parbox{\mlena}{\textit{\quad De un dios hechura, como Dios concibo;      \\
                        Tengo aliento de estirpe soberana...»}}          \\
\end{center}
\normalsize

Con tal estímulo se arrancó por fin María Luisa, y recitó la composición
entera con tono y énfasis de teatro, exagerando un tanto la expresión
del rostro para comunicar vida y color a cada concepto y a cada palabra.
Oyéronla con religioso pasmo los presentes, fijos en el rostro bonito de
la declamadora, y a medida que avanzaba, graduando la sensibilidad y el
entusiasmo, se iban quedando sin aliento. Cuando llegó al pasaje
culminante en que dice el poeta:

\small
\newlength\mlenb
\settowidth\mlenb{\quad Gloria, madre feliz de la esperanza,}
\begin{center}
\parbox{\mlenb}{\textit{\quad Gloria, madre feliz de la esperanza,        \\
                        Mágico alcázar de dorados sueños,                 \\
                        Lago que ondula en eternal bonanza                \\
                        Cercado de paisajes halagüeños,}}                 \\
\end{center}
\normalsize

D. Gerardo tuvo que llevarse el pañuelo a los ojos, y el buen Ibero,
fácil a la emoción, hacía visajes, pestañeaba, componía su rostro para
que no vieran llorar a un soldado rudo, ni flaquear una entereza forjada
en las batallas.

\hypertarget{xii}{%
\chapter{XII}\label{xii}}

«No es propio de una dama---dijo Ibero a Rafaelita, otra noche, en grupo
apartado de la piña de tertuliantes,---mostrarse tan
\emph{materialista}, tan aficionada al dinero, que, según todos los
filósofos, es cosa despreciable.»

---Como no he leído a ningún filósofo, no sé lo que dicen, D. Santiago,
ni creo que me haga falta saberlo. Todo eso de los filósofos estará
escrito en latín. Cualquier día lo leo yo\ldots{} En cuanto al dinero,
si es cosa tan mala, yo tiraré a la calle el poquito que tengo, si los
demás hacen lo mismo\ldots{} Pues vería usted lo que pasaba. El dinero
que los ricos tirasen lo cogerían los pobres, y volveríamos a estar lo
mismo: unos con mucho, otros con nada.

---Pero usted\ldots{} vamos a ver, ¿por qué se nos ha hecho tan
ambiciosa? La ambición es pecado de hombres, como la modestia es la
virtud de las mujeres.

---¿Que por qué soy ambiciosa? Pues porque no soy tonta ni ciega. ¡Ay!,
¿no lo entiende? ¡Qué torpe se hace usted cuando le conviene!

---Tiene usted talento.

---Puede. Si Dios me lo ha dado, ¿qué quieren que haga con él?

---Naturalmente, emplearlo.

---Es usted hermosa.

---No digo que no.

---Y no se resigna a una vida obscura.

---Dígame usted: ¿para qué nos ha dado Dios la vida?

---Para amarle y servirle, según el Catecismo.

---Con perdón del Catecismo, Dios nos ha dado la vida para que
vivamos\ldots{} No nos la ha dado para que nos muramos.

---Y naturalmente, usted no quiere morirse\ldots{}

---Si Dios me manda una enfermedad y la muerte con ella, ¿qué remedio
tengo más que conformarme?\ldots{} pero lo que es de fastidio no quiero
morirme.

---Ni yo tampoco: por eso vengo aquí, y viéndola a usted y gozando de su
conversación, y admirando sus gracias, no sé lo que es hastío.

---Vamos, que viene usted a pasar un rato en mi compañía. Yo se lo
agradezco. Algo es algo. No soy tan desgraciada como parece.
Pero\ldots{} verá usted lo que va a pasar. El mejor día se cansa usted,
o encuentra mejor entretenimiento en otra parte, con personas de más
viso, de la clase que a usted le corresponde, y adiós D. Santiago. El
pájaro voló de esta casa, huyendo de la pobreza.

---Se equivoca usted, amiga mía. Soy muy constante, y si no tuviera esa
virtud, los méritos de usted me la darían.

---Fíjese usted en lo que dice.

---Sé lo que digo. No sabe usted lo que vale, Rafaela, ni la atracción
que ejerce.

---Mire, D. Santiago, que eso que me dice es muy grave, y que podría yo
tomarlo por declaración\ldots{} volcánica.

---¿Y qué?

---Que si usted se empeñara en declararse, yo tendría que decirle que
soy casada.

---¿Y qué más?

---¿Le parece poco? También podría darme la ventolera de
admitirle\ldots{} con la condición de ser ¡ay!, sumamente platónico.

---A eso iba, a que seamos\ldots{} ¡terriblemente platónicos!

---Hable usted bajito. Mire que estamos llamando la atención. Mi hermana
me mira.

En esto entró D. José con la cara muy larga, afectando seriedad y
mordiéndose los labios para contener la risa. Era la cara de las buenas
noticias, que sus hijas conocían muy bien, y en cuanto le vieron se
lanzaron hacia él, cogiéndole cada una por un brazo. «¿Qué hay, papá?
¿Eres ya jefe político?»

---¿Jefe político? ¡Qué cosas tenéis! ¿De dónde habéis sacado el
disparate de que vuestro pobre padre fuese mandarín de una provincia?

Con estas denegaciones festivas preparaba siempre el buen hombre sus
anuncios de felicidades. D. Gerardo se abalanzó a estrecharle la mano,
diciéndole: «No martirices a tus hijas, Pepe, y dales pronto el
alegrón\ldots{} Ya lo supe esta mañana; pero no he querido decir nada
por no quitarte el gusto de las albricias.»

---Mil enhorabuenas a usted, mi queridísimo D. José---le dijo Ibero
abrazándole con efusión,---y otras tantas a D. Manuel Cortina por un
nombramiento que le honra. ¡Viva el Gobierno! Así se regeneran las
naciones, así, llevando la probidad y la inteligencia a los puestos de
peligro\ldots{}

---Y recompensando a los buenos liberales.

---A los probados, a los consecuentes.

---¿Y qué provincia al fin?

---La que yo quería. ¡Pues hemos bregado poco por ella, en gracia de
Dios!---dijo D. José paseándose por la salita, como si padeciese un
delirio de actividad.---Querían mandarme a Lérida. Se han convencido de
que para mejor servicio de la situación y de la libertad debo ir a la
Mancha. Ciudad Real es mi ínsula, y los compatriotas de Sancho mis
súbditos. Buena gente, según me han dicho; país sano; excelente carne de
cabra, y a veces de carnero\ldots{} Su capital goza fama de sucia y
villanesca; pero la mejoraremos, introduciendo los adelantos. Los
moderados han tenido al país aquel en un abandono lamentable\ldots{} ¡Ya
se ve!\ldots{} son gente que no gobierna, que no instruye a los pueblos,
que no les inculca la civilización\ldots{}

---No te olvides, Pepe---díjole D. Gerardo con una gravedad
administrativa que fue la admiración de todos,---de llevar allá la
Memoria y estudios de los Pozos artesianos\ldots{}

---¡Vaya si los llevaré!\ldots{} con planos y presupuesto.

---Como allí entren por ese adelanto, a la vuelta de un par de lustros
todo el páramo manchego será un vergel magnífico.

---Y la cosa es sencillísima\ldots{} Figurémonos un tubo\ldots{} varios
tubos que se van hincando en la tierra\ldots{} hasta llegar a la capa
húmeda\ldots{} y una vez en ella, frrrrr\ldots{} sale el agua con un
chorro que da gusto. Me ocuparé de eso, procuraré hacer un ensayo a poco
de mi llegada, para lo cual me llevaré tubos en cantidad
suficiente\ldots{} Allí tenemos un ingeniero muy listo, que ha estado en
Francia\ldots{} Tengo tiempo de prepararme aquí para la implantación del
\emph{artesianismo}, porque no puedo irme antes de diez o doce días.
Sobre que no me ha concluido el sastre los trapitos de gobernar,
Carrasco quiere que le espere aquí, para celebrar con él y con el
Ministro, quizás con Espartero, un par de conferencias. ¿Conviene o no
conviene que al Parlamento, que ha de elegir la Regencia, vengan hombres
de probado amor al \emph{progresismo}, hombres de arraigo, hombres de
circunstancias\ldots? Pues no tengo más que decir. Bruno estará en
Madrid dentro de pocos días, pues en su última carta me dice que
activaba la venta de sus cosechas de vino y pan, que ya tenía ultimados
los arrendamientos de sus propiedades, y se ocupaba en el levantamiento
de casa y transporte de toda la familia.

Tratose luego de si D. José llevaría consigo a los hijos menores, o a su
hija Rafaela, para que en las soledades de la ínsula le cuidase; pero
esta idea fue pronto desechada por la resistencia ingeniosa que la
\emph{Perita en dulce} opuso al proyecto paternal. Érale forzoso
permanecer en Madrid, a la mira de los incidentes del pleito que había
de entablar reclamando alimentos. No se reiría de ella, no, el bribón de
su marido. En cuanto a los muchachos, mejor seguirían sus estudios en
Madrid que en la Mancha, y el papá, sin la incumbencia de cuidarles y
vigilar su educación, podría dedicarse en cuerpo y alma al gobierno
político y a la grande innovación de los pozos. Apoyó María Luisa el
sesudo dictamen de su hermana, sosteniendo que el gasto sería menor
permaneciendo en Madrid toda la familia y D. José solito en su
Barataria, donde viviría como un príncipe, casi de balde, pues había que
contar con regalos de comestibles y con el servicio de ordenanzas. Del
mismo parecer fue D. Gerardo, que por triste experiencia conocía los
dispendios y molestias de cargar con familia cuando se iba destinado a
provincias, y en apoyo de su aserto expresó la contingencia de que,
efectuadas las elecciones, fuese trasladado D. José a un \emph{mando} de
primera clase. Por gusto de hacer coro, Ibero sostuvo la misma opinión.

Toda la noche, hasta la avanzada hora en que terminó la tertulia, estuvo
el buen Milagro dándose un tono fenomenal, ora llevando a gloriosa
regeneración los graves asuntos nacionales, ora los manchegos. Las
esperanzas optimistas, los risueños programas, afluían de su boca como
un fresco manantial inagotable que fecundando toda la tierra, la poblaba
de venturas. Las extensas plantaciones de arbolado darían a la Mancha
frescura y sombra, y la desecación de las lagunas de Ruidera aumentaría
en muchos miles de fanegas los terrenos laborables. Con una
administración proba y activa y unos cuantos \emph{toques de Gaceta}, el
país de D. Quijote sería un edén, y vendrían en tropel a establecerse en
él los extranjeros, cargados de capitales; y el día en que Inglaterra y
Francia probaran el valdepeñas, adiós Burdeos y toda la porquería de
vinos de la Gironda. Retiradas las visitas, reiterando los plácemes,
entregose la familia al descanso, y se adormecieron grandes y chicos en
rosados ensueños de gloria. No podía María Luisa apartar de su mente los
versos

\small
\newlength\mlenc
\settowidth\mlenc{\quad No baste a mi placer la inmensa copa}
\begin{center}
\parbox{\mlenc}{\textit{\small \quad No baste a mi placer la inmensa copa   \\
                                Del báquico festín, libre y sonoro,         \\
                                De esclavos viles la menguada tropa         \\
                                Sin las llaves de espléndido tesoro,}}      \\ 
\end{center}
\normalsize

\justifying{\noindent y dormida los recitaba con la misma entonación de teatro y el propio juego de
ojos y boca. Rafaela no concilió fácilmente el sueño, rehaciendo en su mente
los últimos coloquios con Santiago, el cual le agradaba en extremo por su
condición blanda, dentro de la superficial fiereza militar; por su corazón sano
y potente, sin picardía; por su poco mundo y el candor honrado con que juzgaba
de cosas y personas.}

En la tarde que siguió a los sucesos referidos, Rafaela cogió por su
cuenta al Coronel, y sin cuidarse de la presencia de su hermana que
cosía la ropa de los niños, le trasteó con gran maestría.

«¡Qué lástima, amigo Ibero, que se haya concluido la guerra!»

---¿Y qué razón hay, señora \emph{Perita}, para que usted no se crea
dichosa en la paz que disfrutamos?

---Porque si tuviéramos guerra todavía, usted, tan valiente y
pundonoroso, sería muy pronto general.

---Más quiero la paz que cien fajas; puede usted creérmelo.

---Pues yo no pienso lo mismo. Porque usted se pusiera dos entorchados,
por lo menos, vería yo con gusto una tremolina muy gorda.

---Agradezco el buen deseo por lo que a mí se refiere; pero tengo que
decir que es usted muy inhumana.

---Diga usted lo que quiera; pero yo pienso que con las guerras, aunque
sean civiles, las naciones crían callo y se hacen más fuertes\ldots{} Y
qué sé yo\ldots{} me parece a mí que las peleas encarnizadas ilustran,
quiero decir que despabilan a la gente. En fin, si es disparate que lo
sea. Lo que usted no me negará es que con las guerras se aumenta el
dinero.

---¡Anda, morena! Si la guerra, señora mía, es la paralización, la ruina
del comercio, de la industria\ldots{}

---Ya pareció el estribillo\ldots{} A mí no me venga usted con
estribillos, D. Santiago, si no quiere que le tenga por tonto.
¡Paralización! ¡Vaya una música! Bien a la vista está que concluida la
guerra salen por ahí hombres riquísimos que antes eran pobres. ¿Usted no
ha oído hablar de uno que hace años, no sé cuántos años, iba vendiendo
paja con una reata de tres mulas? Pues ahí le tiene usted hecho un
caballero millonario, que de algo le ha valido el suministrar a los
ejércitos tanta paja y cebada. ¿Y qué me dice de los maragatos que antes
venían aquí con sus cargas de trigo de Castilla, y después, llevando
víveres al ejército, o haciendo que los llevaban, se han forrado de
dinero? Mi padre conoció a uno que vendía por las calles piezas de
lienzo, y ahora revuelve con pala los montones de onzas. En pocos años
de guerra ha salido de pobre. Pues eso quiere decir que con la guerra
hay más hombres ricos que antes, y que estos, si mucho tienen, mucho han
de gastar\ldots{} o lo gastarán sus hijos, sus mujeres\ldots{} sabe Dios
quién se encargará de dar aire al dinero.

---En todo eso que se cuenta, crea usted que hay mucho de leyenda o
fábula.

---Pues mi padre, hombre que lo entiende, nos decía: «hay más de cuatro
que desean la continuación de la campaña, porque con ella se están
cubriendo el riñón.»

---Esos pesimismos de D. José eran el desahogo natural de las tristezas
de la cesantía. Vea usted cómo ahora no lo dice.

---Bueno: ya veremos quién tiene razón, si usted, que es un ángel, o yo,
que, aunque me esté mal el decirlo, soy más lista que usted, y no se
ofenda. No ha de pasar mucho tiempo sin que vea usted construir en
Madrid casas magníficas\ldots{} me lo ha dicho quien lo sabe\ldots{}
casas como no se han conocido aquí nunca, con portales al modo de
palacio, y comodidades por dentro y decorado muy bonito. ¿Y usted no
sabe que a esta fecha están llegando de París todos los días modistas
que traen la última novedad, y además una caterva de perfumistas,
camiseros, estufistas? ¿Pues esos a qué vienen sino al olor del dinero
que ahora saldrá? ¿Cree usted que vienen por la libertad? ¡Ay qué
simple!

---Es el resultado de los adelantos, Rafaela.

---¿Y usted no ha oído decir que van a poner en Madrid una cosa que se
llama el gas, para alumbrar toditas las calles?

---Sí, sí; y también se habla de caminos de hierro para ir de aquí a
Aranjuez en dos o tres horas. Pero eso no es porque hayamos tenido
guerra civil.

---Es porque ahora hay ricos y antes no los había---prosiguió la
\emph{Perita} con gracia,---es porque nos hemos despabilado con la
sacudida de las guerras. Pues otra: ¿y qué me cuenta de los ricos nuevos
que van a salir, de todos esos que están comprando por un pedazo de pan
las tierras y casas que fueron de frailes? ¿Y los que afanaron, como
dice papá, el papel de Deuda que tenían las monjas? Vamos, que habrá
cada millonario que meta miedo, y eso, eso es lo que conviene. La
grandeza tiene cada día menos dinero, así lo cuenta D. Gerardo, que
entiende de estas cosas. Pero ya le oyó usted anoche: ahora va a salir
otra grandeza nueva, la de los que vendieron paja y después compraron
dehesas de frailes; la de los que daban de comer a las tropas, y luego
establecerán \emph{los adelantos}, haciendo caminos nuevos y poniendo
máquinas para todo\ldots{} qué sé yo, cosas muy buenas. El cuento es que
haya dinero y que corra.

---Vamos, que es usted una \emph{materialista} tremenda---dijo Ibero,
que a medida que la \emph{Perita} se metalizaba la veía más graciosa y
dulce. Sus atractivos no despertaban en él un afecto puro, sino más bien
curiosidad ardiente, como un deseo de conocer a fondo aquel carácter
extraño, y de ver hasta dónde llegaba el vuelo de sus ideas atrevidas.

---Me han hecho materialista mis desgracias---replicó Rafaela mirando un
trazado ideal que con el dedo hacía en el tapete de la mesa,---y la
necesidad en que me veo de abrirme sola los caminos de la vida\ldots{}
También me hace materialista el que no me siento yo\ldots{} ¿cómo
decirlo?\ldots{} de manera de pobre\ldots{} cosa rara, ¿verdad?, pues en
la pobreza nos hemos criado. La pobreza es cosa muy mala, y hay que huir
de ella sin faltar a la decencia.

---Ahora comprendo por qué le son simpáticas las guerras y desea que se
repitan.

---¿Por qué?

---Porque en una nueva guerra podría perecer Piquero, víctima de su
arrojo. Usted se quedaría libre y en disposición de arreglarse mejor con
otro.

---¿Con otro marido? Falta que lo encontrara como yo me lo merezco.

---Yo sé de algunos que se determinarían\ldots{} corriendo el riesgo de
que usted les volviera locos.

---Veo que me está tomando miedo por esto del materialismo. Yo lo
conozco en que ya no me hace declaraciones.

---¿Es que quiere que las repita? Ya me he cansado de hacerlas
inútilmente.

---Porque usted, por otro lado, es también de un materialismo que da
miedo. No es fácil que nos entendamos.

---Porque usted me pide como medida previa que la divorcie, y yo lo haré
con mucho gusto el día en que me nombren Papa.

---Lo que hay es que usted quiere que \emph{toquen a divorciar}, como
mandaría tocar fajina.

---Diga usted de una vez que no soy su salvador, su libertador, y así
habremos acabado.

---No digo eso, y bien podría decir todo lo contrario\ldots{} ¿Ve
usted?, ya está lleno de fatuidad, porque esto que he dicho, casi sin
pensarlo, lo toma el Sr. D. Santiago a declaración.

---Claro; como que lo es.

---Silencio: viene mi hermana.

---Y me temo que venga también Cavallieri a cantarnos el aria que acaba
de componer.

---Para que se convenza usted de que aquí no podemos hablar.

---Imposible que hablemos aquí con libertad; ya lo he dicho.

---Yo he sido quien primero lo dijo.

---Y yo quien propuse que buscáramos otro sitio donde\ldots{}

---Si no fuera usted tan pillo, desde luego.

---Basta de melindres. ¿Mañana\ldots?

---¿Dónde?

---Un paseíto y nada más.

---¿A qué hora?\ldots{} ¡Chitón!\ldots{} Luego veremos.

\hypertarget{xiii}{%
\chapter{XIII}\label{xiii}}

Cómo pasó Ibero por suave pendiente desde las alturas del amoroso ideal
caballeresco a una liviandad caprichosa y pasajera, lo comprenderá quien
considere su soledad triste, su juventud misma vigorosa y la fuerza de
los hábitos militares en tiempo de paz, y a veces de guerra. Emprendió,
pues, la fácil aventura, manteniendo en su espíritu con secreto culto la
fe del amor verdadero, sin que le costase muy grande esfuerzo establecer
la distinción, el deslinde de campos, conforme a las ideas vigentes en
nuestra edad y a la imperfecta educación moral y religiosa del hombre
del siglo. Trazada la raya entre lo accidental y lo permanente, entre la
superfluidad de unos días y el deber de siempre, se divirtió el hombre
todo lo que pudo, con no poca ventaja de su espíritu y de sus nervios,
porque en verdad se hallaba necesitado de esparcimiento y también de
variedad en su monótona existencia de caballero soñador. No se tome por
giro retórico esto de la fidelidad que a su ideal señora conservaba, y
adviertan los que le critiquen que se pasaba la vida sin verla más que
en figuración de la mente. Cualquiera sale indemne de semejante prueba.

Lo más gracioso del caso fue que con los deslices del señor Coronel
coincidieron las buenas noticias de La Guardia, y ello hubo de
producirle alguna inquietud de conciencia, no mucha, y bastante
confusión en los pensamientos, porque era en verdad cosa muy peregrina
que el destino le recompensara sus traicioncillas con esperanzas en lo
que más amaba. No por este contrasentido se despertaron sus hábitos
mentales de superstición, ni aquella manía de ver en todos los objetos
signos de felicidad o ventura. Por el contrario, la distracción, el
contento que recibía de aquella forma de vida, siquiera no fuese un
contento integral, pleno y comprensivo de todo el ser, le aliviaron de
sus murrias, haciéndole olvidar las aberraciones que sufrió en Valencia,
donde a punto estuvo de practicar la quiromancia y otras artes
diabólicas. \emph{Similia similibus}: un diablo bonito le había sacado
del cuerpo los feos diablos; al propio tiempo se divertía, se recreaba,
como quien espacia su ánimo admirando las hermosuras de la Naturaleza, y
además aprendía, pues seguramente aquellos fugaces amores eran muy
instructivos, ¿quién podía dudarlo? El carácter de Rafaela, que iba
observando día por día, viéndolo manifestarse en mil accidentes y
ocasiones, le producía la satisfacción del que adquiere conocimientos,
del que descubre mundos, aunque sean áridos; del que viaja y ve
panoramas bellos, lugares donde no ha de vivir, pero que contempla y
examina para poder describirlos.

¡Y que no tenía poco que estudiar la dichosa \emph{Perita en dulce!} Si
al descubrirla en la casa paterna la tuvo el militar por una pizpireta
de mucho cuidado, luego, en el trato íntimo, pensó que se había quedado
corto en la opinión que formara de sus hechiceras malicias. Si al
principio se dejó coger en el sentimentalismo que con supremo arte
tendía Rafaela como una suave y fina red para cazar a los tiernos de
corazón, pronto supo escabullirse rompiendo las mallas. Cualidades
extraordinarias desplegaba la hija de Milagro en la seducción; era en
ella un don nativo, y así como conocía, sin que nadie se las enseñara,
las artes del adorno y de la elegancia, sabía emplear mil sutilezas para
establecer su dominio. La dulzura, los alardes de puntillosa estimación
de sí misma, el llanto, la risa, la seriedad o el abandono, los
admirables métodos de disimulo que empleaba para revestir de decencia su
liviandad, y evitar el escándalo, todo era de una admirable
falsificación psicológica, imitando sabiamente la verdad. Pero en nada
se revelaba su inspirado histrionismo como en los superiores artificios
para inspirar lástima, haciendo una pintura muy patética de su situación
social, ni casada ni viuda, queriendo ser buena y no pudiendo
conseguirlo, incapacitada por ley de su naturaleza para ser vulgar.
Habíala hecho Dios para un fin, y si a él no se dirigía, era porque el
mismo Dios le cortaba los caminos, como arrepentido de su obra.

Con todas estas artimañas estuvo a dos dedos del peligro el valiente
Ibero, y por espacio de una semana se vio el hombre aturdido, sintiendo
que algo profundo, negro y aterrador, como una sima sin fondo, ante sus
ojos se abría. Tuvo la suerte o la entereza de contar los pasos que le
faltaban para llegar al borde, y se propuso atajar vigorosamente su
carrera, valiéndole de mucho para conseguirlo la serenidad con que fijó
el pensamiento en la ideal señora de La Guardia, pidiéndole con mental
invocación que en aquel trance le socorriese.

Dígase ahora, para componer el buen orden de los sucesos, que Milagro no
había podido detener su viaje a la ínsula manchega tanto como quería:
apremiado por el señor Ministro para ponerse en camino, partió antes de
que D. Bruno Carrasco abandonase el terruño. Allá conferenciaron días y
noches cuanto les dio la gana y exigía el grave negocio de la elección;
y dejando el manchego bien preparados los trastos caciquiles, y
arreglado lo tocante a sus haciendas en los pueblos de Peralvillo y
Torralba de Calatrava, cargó con toda la familia y se vino a Madrid,
pensando en la falta que haría en la Corte su presencia para deshacer
tantos agravios entre pueblo y Monarquía, y resolver tanto litigio
hispánico, ultramarino y europeo.

Cuando el manchego y su gente llegaron a Madrid, medio derrengados todos
del traqueteo de la infame galera, ya habían pasado muchos días, lo
menos veinte, del enredillo de Ibero con la hija del jefe político; mas
tan sutil era el arte de Rafaela para rendir el debido homenaje al
formalismo de una sociedad dominada por la etiqueta religiosa y moral,
que los allegados a la familia no tenían de aquel lío conocimiento.
Siempre encontraban a la \emph{Perita} en casa, por las noches, con
rarísimas excepciones, tan simpática, graciosa y elegantita con cuatro
pingos muy bien puestos, haciendo la víctima interesante y encantando a
todos con su sencillez y modestia. Los amigos de Ibero sí que lo sabían;
pero estaban en esfera tan distante de la casa y relaciones de Milagro,
que la opinión respecto a Rafaela no había podido variar todavía. En el
ámbito de Madrid, que es lugar grande, pero lugar al fin, corrían ya
malignas especies; mas la murmuración andaba todavía muy desorientada, y
como toda ruindad de pensamiento tiende aquí a envilecerse más y más
revistiéndose de ruindad política, los comentaristas, que veían a
Rafaela vestida de seda, dijeron: «¡Cómo le luce la jefatura política a
ese buey cansino de Milagro!\ldots{} ¡Y decían que era ciego! Pues si
llega a ver el hombre, ¡pobre Mancha! Se trae para su casa hasta la
langosta.» Quedaba el recurso de menospreciar estas malicias con la
muletilla: «Cosas de los moderados.»

Una vez lanzado a la irregularidad, fácilmente recayó Ibero en otros
vicios muy propios de la vida militar, y de los ocios de la guarnición
en tiempo de paz. Por distraerse dejábase llevar de la corriente
licenciosa de sus compañeros y amigos. En la calle de la Aduana tenían
una timba, exclusivamente para militares, algo como casino o
\emph{cuartón}, que había sido logia en tiempos no lejanos, y en el
callejón de Sevilla había otro \emph{asilo} de esta clase para pasar las
noches, no menos corrupto, pero más divertido: el local era más bonito y
casi lujoso, y en él no reinaba sólo el naipe, sino la galantería, si
este nombre puede darse al trato de mozas guapas: no puede negarse que
la disipación era allí más amena. A uno y otro sitio concurría Santiago,
y anegaba en el azar su hastío, con tan mala sombra que al poco tiempo
tuvo necesidad de pedir dinero a su familia para salir de compromisos.
Por dicha suya, era su carácter de los que, poseyendo lo que hoy
llamaríamos \emph{freno automático}, saben contenerse en el filo de la
perdición, y esta entereza, que le había salvado en el caso de
Rafaelita, le salvó asimismo en los desórdenes del juego.

Ya que se ha nombrado a \emph{la Milagro} (así solían nombrarla ya),
sépase que Ibero no se habría desprendido tan pronto de sus redes si a
ello no le ayudara un amigo, llamado Manuel Catalá, comandante de
Caballería con grado de teniente coronel, valenciano, de buena
presencia, muy corrido en lances amorosos. En aquella ocasión las cosas
vinieron rodadas del modo más feliz para D. Santiago, pues Catalá quiso
jugar una mala partida a su compañero, quitándole su hembra; hizo a ésta
la corte ganoso de alcanzar una victoria; aprovechó el otro la ocasión
con seguro instinto, haciendo una retirada hábil, y Catalá se encontró
dueño del campo creyendo deber tan fácil triunfo a sus propios méritos.
Gozoso de su liberación, hizo Ibero el papel de sentirse herido en su
amor propio; mas este fingimiento no fue de larga dura, pues no tardó el
valenciano en comprender que había sido estratégica la sustitución. Por
su desgracia fue cogido muy estrechamente en la red y ya no tenía
escape: el arte sentimental de Rafaelita hizo su efecto, y el comandante
se prendó de ella con pasión tan viva y ardiente, que allí fenecieron
sus vanaglorias de conquistador y empezaron sus martirios de
conquistado.

La nube de rivalidad entre Ibero y Catalá se disipó bien pronto, pues el
uno supo sostener su papel con dignidad y el otro no hizo alarde de
vencedor; volvieron a ser amigos; no dejó de serlo Santiago de
Rafaelita, y siempre que la veía en su casa o en la calle le hablaba en
tonos de protección fraternal, recomendándole aplicara enérgicos
emolientes a la llaga lastimosa de su materialismo. Entre el defecto
capital de Rafaela y la pasión cada día más loca de Catalá, hubo de
entablarse colosal lucha, y esta trajo conflictos graves, estallidos de
ira, dolor intenso, riñas y reconciliaciones en que uno y otro ponían el
fuego de sus almas. A tal extremo llegó la desesperación de Catalá
algunos días, que hubo de recurrir a Ibero solicitando su amistosa
mediación: «Chico---le dijo, echando lumbre por los ojos, balbuciente y
trémulo,---soy hombre perdido: ni puedo consentirle sus infamias, ni
puedo dejar de quererla. ¡Ya ves, yo, tan corrido, tan dueño de mí en
otros lances de mujeres! Pues aquí me tienes loco, niño, imbécil; no sé
qué soy. Creo lo que nunca hubiera creído, que se dan y se toman filtros
o venenos para enloquecer. Yo no me conozco. Antes de dos días haré lo
único que cabe para poner fin a esta situación: la mato y me mato. He
comprado dos pistolas muy seguras y las tengo bien cargadas\ldots{}
Porque no me quiere la mato a ella; porque la adoro me mato yo.»

Esto dijo en la calle con frase entrecortada, sin añadir explicaciones
que permitieran a Santiago formar juicio exacto de los motivos de la
inminente tragedia; pero luego, solos en el cuarto de banderas del
cuartel del Conde Duque, dio suelta el lastimado amante a sus agravios,
refiriendo al Coronel cosas que le afligieron y abrumaron en extremo,
pues si no amaba a Rafaela, no gustaba de verla tan despeñada por la
pendiente del mal.

\hypertarget{xiv}{%
\chapter{XIV}\label{xiv}}

Sin pérdida de tiempo trató Ibero de ver a Rafaela en su casa, decidido
a hablarle severamente; pero encontrose con un obstáculo formidable,
porque, habiendo llegado aquel día D. Bruno con todo su rebaño, las
hijas de Milagro se consagraban con alma y vida a la instalación de la
familia manchega. Se les había tomado el principal de la misma casa; mas
como no estaba aún pertrechado de camas, se les daba vivienda
provisional y comida en la casa de Milagro, para lo cual no hubo más
remedio que poner colchones en el suelo y arreglarse todos como Dios
quisiera. La casa era una Babel, y los chicos manchegos y matritenses,
enredando juntos, producían un estruendo insoportable. Atendían Rafaela
y María Luisa, multiplicándose, al menester de preparar comistraje para
tantas bocas, y las viajeras, hijas y señora de Carrasco, descoyuntadas
y muertas de fatiga, dormitaban en sofás y sillones, mientras Don Bruno
y Cavallieri se ocupaban en clavar escarpias en las paredes del nuevo
domicilio, y en abrir baúles y colgar perchas. Vio Santiago que no era
ocasión para lo que se proponía, y se fue, no sin anunciar a Rafaela que
se preparase para una buena reprimenda.

A primera hora de la noche se fue Ibero a pasar un rato en casa de D.
Antonio González, con quien había contraído amistad recientemente, por
Seoane. ¡Cuánto mejor aquella sociedad que los garitos en que se había
dejado su dinero y su decoro! Diríase que en las moradas de cierto tono
a que por entonces concurría, restauraba su personalidad, medio deshecha
en la borrascosa vida del vicio. El único inconveniente de los salones
era que en ellos se hablaba demasiado de política, hasta el punto de
producir mareo y confusión en los que como él tenían ideas fijas, que
apenas admitían controversia. Pero esta dificultad se obviaba dejándose
llevar de la corriente general, y no haciendo gala de un radicalismo
chocante en las opiniones. En casa de González jugaba sus tresillos con
Sartorius y con la señora de Seoane, o con Beltrán de Lis y el brigadier
Latre; de allí solía irse al café Nuevo, donde encontraba a Espronceda,
a veces a González Bravo y a los Escosuras. De la primera tertulia
sacaba la impresión de que todo iba como una seda: vendrían unas Cortes
elegidas con libertad, representación genuina del \emph{Progreso}, que
era la voluntad del país; se elegiría la Regencia, una o trina, y
entraríamos en un periodo de bienandanzas y prosperidad. De la segunda
reunión, ahumada por los cigarros, sacaba impresiones contrarias: íbamos
a un cataclismo si no venía pronto el gobierno del pueblo por el pueblo,
la verdadera igualdad, la supresión de monigotes y de ficciones
ridículas. ¿Qué saldría del cataclismo? Pues la regeneración grande y
sólida, un Estado potente, costumbres europeas y una civilización de
nueva planta. Retirábase Ibero a dormir, procurando conciliar en su
mente unas opiniones con otras, estas y aquellas esperanzas, y en su
tarea de imposible conciliación, dando vueltas al endiablado problema,
concluía por anegar sus ideas en el sueño.

Volvió a casa de Milagro a la hora del siguiente día que le pareció más
oportuna; pero Rafaela estaba ausente, pues había tenido que ir de
compras con la señora de Carrasco para proveer a lo más apremiante en
cosas de vestimenta. María Luisa también revoloteaba por tiendas de
telas y comestibles. Ya se iba el hombre, huyendo de las arias
mortíferas de Cavallieri, cuando le cogió por su cuenta el Sr.~de
Carrasco, que no quería soltarle a dos tirones, y le invitó a comer,
para que probara los chorizos, hechos en casa, que había traído de su
pueblo, cosa excelente sobre toda ponderación, y las perdices
escabechadas y el mostillo.

¿Qué debía de hacer Ibero más que quedarse, cediendo a los agasajos y
carantoñas del buen Carrasco? Su aquiescencia le deparó el gusto de
conocer a la noble familia, transportada como una tribu desde las
soledades manchegas al bullicio de la Corte. Doña Leandra Quijada,
esposa de D. Bruno, era una señora flaca, más que vieja envejecida, muy
descuidada de su persona, llena de arrugas la faz, los ojos lacrimosos,
áspero el cabello entrecano y partido en \emph{bandós} aplastados sobre
la frente y sienes. Estaba la pobre mujer atontada, en una estupefacción
triste, como quien no se da cuenta de lo que pasa ni entiende lo que
oye. El ruido, \emph{la mucha gente que iba por las calles}, el paso
continuo de coches, la altura de las casas, los gritos de los
vendedores, todo cuanto veía y escuchaba, le había infundido más terror
que asombro. Su anhelo era huir de este barullo, volviéndose al sosiego
de donde había venido; pero la timidez no le permitía manifestar su
tristeza y miedo más que con suspiros. Su vestido, totalmente negro, de
lana, y el pañuelo del mismo color anudado bajo la barba, dábanle
aspecto lúgubre. Hablaba poco, respondía con urbanidad concisa a cuanto
Ibero le preguntaba del viaje y de sus primeras impresiones en Madrid, y
cuando nada le decían tomaba una actitud meditabunda, cogiéndose la
barba y fijando los ojos en el suelo.

Nacida en Peralvillo, casada con D. Bruno en Torralba de Calatrava, de
donde no había salido más que una vez para visitar a sus primas en la
ciudad de Almagro; hecha desde muy niña a la vida de propietaria rica, a
los espectáculos de la Naturaleza y a las faenas de la labranza; formado
su carácter en una sociedad de cariz feudal, en la cual se pasaban los
años viendo pocas y siempre las mismas caras; acostumbrados sus ojos a
la horizontalidad expansiva de su tierra, su oído al silencio campestre,
su vida a las casonas grandísimas, no podía menos de sentir, traspasados
ya los cincuenta, el brusco salto de aquel medio a otro tan distinto. La
casa en que había venido a parar le pareció un gallinero, un palomar,
algo peor y más estrecho aún; las personas que aquí veía le hicieron
efecto de estar locas o borrachas. Hablaban para ella tan aprisa, que
comúnmente no entendía palotada. Ni era el lenguaje de Madrid como el de
allá. En su tierra se hablaba más fuerte y con tono más reposado, y las
palabras sonaban con más pompa. Las primeras comidas que probó le
supieron a broza desabrida, insustancial. ¡Qué chocolate! ¡Y el caldo
qué insípido! El pan no alimentaba ni tenía gusto. Se aterró cuando le
dijeron lo que en Madrid costaban dos palominos, un cabrito o una docena
de huevos. Sin duda en Madrid no vivían más que ricachones. Y toda
aquella gente que veía por las calles, ¿qué gente era, en qué se
ocupaba, a dónde iba?

Compadecido Ibero de la buena señora, y deplorando lo violento del
trasplante, procuró consolarla con la esperanza de un próximo cambio de
hábitos y gustos. «Verá usted---le dijo,---qué pronto se hace a esta
vida, y cómo acaba por encontrarla mejor, más cómoda y placentera que la
de Torralba de Calatrava. Madrid es un pueblo en el cual se aclimatan
fácilmente los españoles de todas castas y terruños. Comprendo que le
costaría un gran esfuerzo arrancarse de su concha\ldots{} La cosa es
dura, lo veo; sé lo que es una casa donde han vivido tres o cuatro
generaciones de nuestra sangre, una cocina que huele a las carnes
ahumadas de un siglo, de dos\ldots{} sé lo que es una tierra propia, un
árbol que ya era grande cuando nacimos, un burro que nos mira diciendo:
«yo también soy de la familia\ldots»

Doña Leandra echó media docena de suspiros, y sin abandonar su actitud
de melancólica resignación, dijo: «Sí que me dolió el arrancarme, señor;
pero Bruno lo quiso, y yo\ldots{} La verdad, al principio no me entraba
en el pensamiento la idea de venir. Yo quería meterla, y ella\ldots{} no
entraba. Pero Bruno decía que nos desterráramos, porque así nos
convenía, y por dar carrera a los hijos, y yo\ldots{} todo lo que Bruno
quiera se hace, cueste lo que cueste\ldots»

Calló, y sus ojos húmedos volvieron a mirar al suelo.

Componíase la familia de Carrasco de los mismos elementos que la de D.
José: dos hijas mayores y dos chicos pequeños, entre los ocho y los doce
años. Solteras eran las muchachas, de la misma edad, próximamente, que
María Luisa y Rafaela, pero de tipo, casta y educación muy diferentes.
Ambas eran negruchas, desgarbadas, desapacibles. A la primera ojeada que
Ibero echó sobre ellas las diputó por feas; observándolas mejor y
aseándolas mentalmente; suponiéndolas despojadas de los horrorosos
vestidos de pueblo y trajeadas a estilo de Madrid, vio que eran
susceptibles de una mejora radical en su cariz y facha. \emph{En
principio}, no pertenecían al odioso reino de la fealdad; pero mucho
había que desbrozar en ellas para obtener dos mujeres bonitas.

A la mayor, bautizada Leandra, por su madre, la llamaba su padre
\emph{Lea}, para evitar el inconveniente de la igualdad de nombre en dos
personas de la familia. Eufrasia era la segunda, y los chicos Bruno y
Mateo. No fue tan penoso como el de la madre el trasplante de las dos
señoritas, por razón de la edad, por la ilusión de ver Madrid y de
afinarse y embellecerse. Con todo, a su llegada no podían vencer el
azoramiento y confusión, que era la conciencia de su inferioridad.
Hablaban muy poco, temerosas de decir algún disparate, o de pronunciar
algún término que pareciese ridículo a la gente de Madrid. Apenas
echaron la primera ojeada por las calles, comprendieron que venían
hechas unos adefesios, y que ningún pingo de los que habían traído de su
lugar les servía para lucirse en la coronada villa. Miraban a María
Luisa y a Rafaela con arrobamiento, asombradas del lindo talle de la
segunda, del aire garboso de la primera, a pesar de su embarazo de cinco
meses; admiraban su ropa, su aire de soltura y elegancia, los andares,
el habla fácil y descarada con airosas cadencias, la gracia del reír, y
la movilidad de expresión en sus bellas facciones. Las pobrecitas
Eufrasia y Lea habían recibido la mejor educación posible en las
soledades manchegas. Un preceptor muy hábil les había enseñado a
escribir con letra española de casta de archivo, redonda, y ponían una
carta con bastante primor. Sus lecturas habían sido escasas; sus
labores, la costura casera y puntilla de Almagro. De conocimientos
generales andaban medianas, porque el preceptor no daba de sí más que la
aritmética elemental, una geografía y una gramática primitivas.
Avergonzadas reconocían las dos muchachas su rusticidad, al llegar a
Madrid, comparándose con María Luisa y Rafaela, que, por lo que hablaban
y las cosas lindísimas que decían en su conversación, debían de ser unas
sabias de tomo y lomo.

Traían a Madrid las hijas de Carrasco las virtudes castizas en grado
eminente: la fe religiosa, el sentimiento del honor y la dignidad, el
culto de la opinión y el respetuoso amor a los padres, a quienes daban
el tratamiento de \emph{su merced}, conforme a la tradicional costumbre
manchega. En los pequeñuelos, la adaptación fue repentina, pues apenas
se juntaron con los chicos de Milagro, hiciéronse todos unos; se
asimilaban cuanto en sus amiguitos hallaron de novedad en habla y modos,
y no querían más que estar siempre en la calle viendo cosas, y saltando
y brincando con libertad y alegría.

Cuando Rafaela y María Luisa se encontraban solas, hacían apreciaciones
reservadas de la familia Carrasco, que conviene consignar. «Es buena
gente ---decía la \emph{Perita en dulce};---corazones muy sanos, con
toda la honradez que da la vida de pueblo; pero trabajo les ha de costar
desasnarse. La pobre Doña Leandra me parece que ha venido tarde para
rasparse la corteza. De Lea y Eufrasia no digo lo mismo, y como son
mozas, aprenderán pronto la civilización. ¡Mira que vienen salvajes las
pobres! ¡Qué cuerpos, qué talles y qué manera de vestirse! Si bien se
las mira, mal formadas no son; pero con aquellos justillos y aquellas
faldas son verdaderos espantajos. También te digo que no tienen un pelo
de tontas: anoche hablé largo rato con Eufrasia, y si vieras cómo se
suelta\ldots{} Estas paletas lo que tienen es mucha hipocresía.»

---Ya verás cómo se transforman en poco tiempo---dijo María Luisa.---Son
mujeres, y eso basta. El problema es que aprendan a lavarse, que no hay
costumbre más difícil de quitar que la del desaseo. Luego vendrá el
vestirse bien. Lea no ha cesado de hacerme preguntas: quién nos hace los
vestidos; lo que cuesta una buena modista; cómo se estilan ahora los
cuerpos. Yo, que no me paro en barras y me intereso por ellas,
¡pobrecillas!, le dije: «Mira, Lea: lo primero es que tires a la basura
todos los pingos del pueblo, los cuales dan el quién vive con el olor
ovejuno.» ¿No has reparado que traen también pegado a la ropa un tufo de
cominos, de anís o no sé qué?\ldots{} En fin, dinero no les falta. Doña
Leandra no se desprende de un pellejo, a modo de vejiga, que parece
lleno de onzas. Querrán vestirse, y hemos de procurar presentarlas como
personas ricas de provincias, que vienen a Madrid a ocupar una posición,
y quizás a figurar más de lo que ahora parece.

---Ha dicho D. Gerardo que D. Bruno es de madera de ministros\ldots{}
¡Mira que si nos le hicieran ministro!\ldots{}

---Eso me parece mucho. Pero de que viene diputado no tengas duda, que
allí está papá, lanza en ristre, para sacarle por encima de todo. Y una
vez diputado, sabe Dios lo que le harán.

---Eufrasia y Lea tienen de su padre una idea que ya ya\ldots{}
Creen\ldots{} así me lo ha dicho Lea, que Espartero y D. Bruno se pasean
del brazo, y que Cortina le consulta todo lo que hace. Así se contaba en
Torralba de Calatrava y en Peralvillo.

---No me parece disparatado que a D. Bruno le den la poltrona---dijo
María Luisa con segura dialéctica.---Mira lo que son otros, de dónde han
salido, y compara. Cierto que no sabe lo que papá. Papá sí que es de
madera de ministros. Yo siempre lo he dicho\ldots{} Pero su cortedad de
genio le pierde, y a nosotras más, y siempre estaremos lo mismo, pobres,
olvidadas, viendo

\small
\newlength\mlend
\settowidth\mlend{los turbios días y las altas horas.}
\begin{center}
\parbox{\mlend}{\textit{\small \qquad \qquad \qquad caminar lentos        \\
                               los turbios días y las altas horas.}}      \\
\end{center}
\normalsize

\hypertarget{xv}{%
\chapter{XV}\label{xv}}

Ven a mis manos, ven, arpa sonora. Baja a mi mente, inspiración
cristiana, y enciende en mí la llama creadora que del aliento del querub
emana.

\small
\newlength\mlene
\settowidth\mlene{Baja a mi mente, inspiración cristiana,}
\begin{center}
\parbox{\mlene}{\textit{\small \quad Ven a mis manos, ven, arpa sonora.   \\
                                Baja a mi mente, inspiración cristiana,   \\
                                y enciende en mí la llama creadora        \\
                                que del aliento del querub emana.}}       \\
\end{center}
\normalsize

Esto recitaba María Luisa una tarde, atizando el fogón para poner a
calentar unas planchas, cuando sintió entrar a Ibero en el comedor,
donde estaba Cavallieri copiando música. Presurosa salió a recibir al
Coronel, que en aquella casa merecía de continuo extremadas
consideraciones, y con oficiosa y dulce voz, antes que la del bajo
acabase de saludar al visitante, le dijo: «Santiago, por Dios, aguárdela
usted, que no puede tardar: ha salido con Doña Leandra a comprar loza.»

Con pretexto de trasladar a sitio más decoroso la visita, fuese con
Ibero a la sala, donde acabó los conceptos que expresar no quería
delante de Cavallieri. «No pase lo de ayer y anteayer, ¡por
Dios!\ldots{} Usted no tuvo paciencia para esperarla, y así se nos va el
tiempo, y se escapan los días sin que Rafaela oiga las verdades que
usted tiene que decirle. Crea usted que está muy echada a perder. Si
usted no la sujeta, no sé, no sé, amigo Ibero, a dónde va a parar mi
hermana. Anoche también entró en casa a las doce dadas\ldots{} Ya no sé
qué decir a los amigos, ni cómo explicar estas ausencias\ldots{} Luego
no pasa día sin que lleguen aquí unos recados estrambóticos, traídos por
mujeres de mala traza\ldots{} ¡Ay, Santiago, estoy afligidísima!\ldots{}
¡Pues si llegara a mi padre y viera estas cosas! Usted, usted es quien
puede traerla a la razón, y ya que no a la virtud, a la decencia, Señor,
al buen parecer, al recato\ldots{} Yo le digo: `Mujer, ten cuidado,
piensa en tu familia, piensa en el nombre sin tacha de nuestro padre,
que ahora, por hallarse en alta posición, es \emph{el foco} de las
miradas de sus amigos y enemigos'. Responde que sí, que tendrá cuidado,
y ya ve usted el cuidado que tiene. Yo, que la conozco, estaba contenta
cuando vi que se entendía con usted, guardando las debidas reservas.
`Del mal el menos', dije. Cuando se da con personas nobles y decentes,
queda el consuelo de que no habrá escándalos\ldots{} Pero viene el
rompimiento, que sentí, me lo puede creer, como si se nos cayera la casa
encima, y mi hermana se disloca, y una tarde nos arma ese bruto de
Catalá una gritería en el portal, y una mañana se planta en casa el
otro, el \emph{Don Frenético}, que así le llamo yo, y con pretexto de
encargar música de bajo, le cuenta a mi marido mil historias que parten
el corazón\ldots{} Nada, nada, sea usted cariñoso y al mismo tiempo
terrible: que ella vea su amistad, y que coja miedo, mucho miedo. Yo sé
que a usted le respeta más que a nadie, Santiago; que le estima\ldots{}
y es natural que así sea. Duro en ella; pegue usted fuerte\ldots»

---A duro no me gana nadie, amiga mía; yo pegaré\ldots{} Tengo una mano
como la maza de Fraga\ldots{}

---Chitón, que ahí está\ldots{} Es ella la que entra. Yo me escabullo
por la alcoba\ldots{}

Dos minutos después, Ibero y Rafaela, solos en la sala, producían una
escena que, sin ser histórica, merece ser puntualmente relatada. ¿Y por
qué no había de ser histórica, siendo verdad? No hay acontecimiento
privado en el cual no encontremos, buscándolo bien, una fibra, un cabo
que tenga enlace más o menos remoto con las cosas que llamamos públicas.
No hay suceso histórico que interese profundamente si no aparece en él
un hilo que vaya a parar a la vida afectiva.

«Al fin---dijo Ibero,---te cojo a tiro, y ahora no te me escapas. Buena
la has hecho, y contento tienes al pobre Catalá. No creí nunca que tu
ambición te enloqueciera hasta ese punto\ldots{} Ya sé lo que vas a
decirme: que yo, por haber contribuido a corromperte, no tengo derecho a
predicarte ahora la moral. Pero no tienes razón, Rafaela: yo te cogí
dañada y bien dañada, y traté de que anduvieras todo lo derecha que
podías con el daño que tienes. No habrás olvidado cuánto bregué contigo.
El día de nuestra separación te dije que\ldots{} ¿no lo recuerdas?»

---Que te dabas de baja como amante, y de alta como inspector
mío\ldots{} así dijiste\ldots{} pues pensabas vigilarme, no permitir que
yo descarrilara\ldots{}

---Así me lo propuse, pensando en el pobre D. José. Si yo fuera un
egoísta, habría dado media vuelta, diciendo como aquel Rey: «Después de
mí el diluvio.» Pero no puedo hacer esto; no soy tan malo; y aunque
rabies, me constituyo en tu fiscal, en tu juez, y si es menester, en tu
verdugo, por mucho que me duela. Con que tú verás, Rafaela. Ya me
conoces: soy un pelma terrible.

---Pega todo lo que quieras. He venido al mundo para víctima, y víctima
seré siempre, hoy de un marido villano, mañana de otros que no lo son y
quieren gobernarme como si lo fueran.

---No debías tener queja de Catalá, Rafaela. Arréglate pacíficamente con
él, porque es un hombre de corazón muy bueno. Sabiendo manejarle, harías
de él lo que quisieras: Como todos los vehementes, en el fondo es un
niño; como todos los que gritan mucho, en el fondo es la misma
docilidad. Pero le has irritado, has cogido una tea encendida, y con
ella le has chamuscado el corazón. ¡Los celos!, ¡qué cosa tan mala! El
que debía ser cordero se te hace tigre.

---Estoy divertida, como hay Dios---dijo Rafaela, sacudiéndose con
gracia los golpes que recibía,---con estos protectores que me salen
ahora. Yo les pregunto qué es lo que me dan, sepámoslo, a cambio de esta
esclavitud en que quieren tenerme. ¿Me han descasado, para que yo pueda
volver a casarme y tener una posición decente? ¿Me han hecho más persona
de lo que yo era? ¿Qué pretenden, que yo les guarde fidelidad y me
sacrifique por ellos, sin que de ellos reciba nada de lo que me falta:
dignidad, nombre, posición?

---Nosotros no podíamos descasarte. ¿Somos por ventura el Papa? En eso
de las posiciones, tú no has pensado bien lo que dices, porque\ldots{}
posición totalmente honrada no puedes tenerla sino resignándote a estar
metida entre cuatro paredes haciendo la viuda inconsolable. Al
declararte independiente, podías aspirar a lo mejor dentro de las
posiciones falsas, a un bien relativo, a una moral de circunstancias.
Pues todo eso lo habrías tenido con Catalá, que se ha enamorado de ti
como un trovador\ldots{} Por lo que me ha dicho el pobre, casi llorando,
habría llegado hasta la bondad inaudita de casarse contigo, en caso de
que enviudaras\ldots{} Ya ves si esto es bondad, si esto es amor, y amor
de los que gastan la venda más espesa.

---¡Casarse conmigo! Si tan largo me lo fías\ldots{} Mi marido goza de
buena salud, según me cuentan; es de familia de vividores, pues su
abuelo tiene ochenta y seis años y lee sin gafas, y da paseos de dos
leguas; familia de Matusalenes\ldots{} ¡Vaya un consuelo!

---Confiésame con sinceridad---dijo Ibero un tanto confuso, sin saber en
qué terreno ponerse,---que ni a mí ni a Catalá nos has querido con
verdadero amor. Confiésamelo; ten franqueza y alma grande para declarar
que fue mentira todo lo que a mí y a Manuel nos dijiste\ldots{}

---Si te empeñas en ello---replicó la \emph{Perita en dulce}, gustosa de
mostrar la grandeza de alma que su amigo le recomendaba,---te daré una
prueba de rectitud declarando que ni tú ni Manuel habéis sabido
interesar mi corazón. ¿Quieres más franqueza? Pues por mí no queda,
Santiago. Sabrás que a uno y otro no los he mirado más que como
escalones\ldots{}

---¡Como escalones\ldots!---repitió Ibero aturdido del golpe, pues la
arrogancia calmosa y un tanto cínica de Rafaelita le desconcertó.---No
te servíamos más que de peldaños para subir hasta D. Federico Nieto, a
quien tu hermana llama \emph{Don Frenético}. Bien. Vale más que te
expliques con claridad para saber qué clase de armas debo emplear
contigo.

---Y es ridículo, Santiago---prosiguió, más altanera y fría
Rafaela,---que tú me pidas amor, cuando no me tomabas más que por
pasatiempo: me alquilabas, Santiago, no me hacías tuya. ¿Me explico
bien? No podía ser de otro modo, porque el amor verdadero se lo guardas
a la señorita de La Guardia con quien estás en relaciones honradas, y
con quien quieres casarte\ldots{} Hace poco lo he sabido, como sé
también que están verdes. Nada me dijiste de estos amores tuyos, tan
finos, ¡ay! Y tomándome por \emph{mujer-simón} para una carrera, o unas
horas, pretendías que yo te amase, que me pusiera flaca y ojerosa y
lánguida por ti. ¡Pero qué tonto eres, qué cosas tiene mi maestro!

---Si no recuerdo mal---dijo Ibero, más desconcertado por la certeza
lógica de la que fue su amante,---te manifesté que tenía un compromiso
antiguo, serio\ldots{} Pero Catalá no se encontraba en ese caso: Catalá
no estaba ni está ligado a otra mujer por una cadena espiritual, y
tenía, por tanto, derecho a tu amor.

---El amor no es cosa que se reclama por derecho. Se inspira sabiéndolo
inspirar, se siente cuando se siente; pero no pueden venir alcaldes y
alguaciles a decirle a una: «pague usted el amor que debe.» Manuel
Catalá será todo lo bueno que tú quieras; pero su carácter violento y
sus celos furibundos no son para enamorar a nadie\ldots{} Luego, hijo
mío, si quieres que te lo diga todo, yo\ldots{} vamos, soy algo
ambiciosa\ldots{}

---El \emph{materialismo} es tu locura y será tu perdición. ¿Qué
entiendes por bienes de la vida? ¿Das este nombre a lo que puede
adquirirse con dinero?

---Dime una cosa, Santiago: ¿por qué te has batido tú, por qué has
pasado tantas fatigas y trabajos en la guerra? ¿Lo has hecho por
quedarte siempre de soldado raso? ¿No soñabas tú con ascensos, con ser
lo que eres, más aún, brigadier, general? Claro; ahora que has ascendido
dirás que no, que lo hacías todo por la gloria, ¡angelito!

---¿Y qué tiene que ver la carrera militar con esa carrera tuya,
despeñadero del vicio? ¿Adónde vas tú? ¿Qué quieres? ¿Riquezas,
posición? Aquí no hay eso para las mujeres que se salen del camino
derecho. Somos, gracias a Dios, un pueblo muy morigerado, un pueblo
virtuoso\ldots{}

---No era mala virtud la que me predicabas tú cuando\ldots{}

---No te burles\ldots---gritó Ibero, que enrojecía del calor de la
discusión.---Lo que yo afirmo, y no puedes desmentirme, es que aquí no
hay posiciones ni riquezas para las mujeres que descarrilan. En Francia
sí lo hay; pero esa es una moda que no ha de venir.

---Yo no traigo modas, Santiago, las traéis vosotros, los que hacéis las
guerras, los que hacéis las revoluciones, los que perseguidos emigráis y
luego venís diciéndonos que aquí somos salvajes, que no hacemos más que
rezar, y que España está infestada de clérigos; tú lo has dicho,
tú\ldots{} y que las mujeres apenas sabemos leer y escribir, y no
tenemos el aquel de las extranjeras, ni la coquetería extranjera, ni la
finura extranjera\ldots{} Con que yo no traigo modas, ¿sabes?

---Ni yo. Lo que haré contigo---dijo Ibero, sospechando que Rafaela
manifestaba tan sólo la parte menos interesante de su ser, que en su
alma había un doble fondo, en el cual no era fácil penetrar,---lo que yo
haré contigo es cortarte los vuelos, no dejarte correr con la velocidad
que quieres tomar.

---¿Y qué harás para cortarme los vuelos?---dijo Rafaela con altanería
desdeñosa,---¿amarrarme a Catalá?

---Amarrarte, no: convencerte de que debes ser benigna para él, de que
debes limitarte a su amistad, sin buscar otras.

---¿Y si no me dejo convencer?

---En ese caso, emplearé otros medios, pues por el estado en que se
encuentra el pobre Manuel preveo una tragedia, y no quiero tragedias en
ti ni en tu casa. No lo hago sólo por ti, lo hago principalmente por tu
padre.

Y encrespándose y tomando bríos, como quien siente muy sólido el terreno
que pisa, se levantó, y con arrogante ademán continuó el vapuleo: «Que
no te escapas, Rafaela, que no tienes salida. Tú a que has de ser mala,
y yo a que no. Tú a caminar torcida, y yo a cogerte y a llevarte
derechita. ¿No quieres de grado? Pues a la fuerza. Soy muy bruto: tú lo
has dicho, y ahora vas a verlo\ldots»

---Veamos, pues---dijo la infortunada fingiéndose asustadica.---Lo
primero que me manda mi sátrapa es que haga buenas migas con Manuel.

---Que le guardes fidelidad, que seas suya y sólo suya\ldots{}
Después\ldots{} no, no, antes o al mismo tiempo, que despidas,
quitándole toda esperanza, a ese D. Federico Nieto\ldots{} Eso has de
hacerlo prontito, Rafaela, porque si no, yo, yo me encargo de romperle
el espinazo al \emph{Don Frenético}, para que no te trastorne más. Si
hay \emph{materialismo} de por medio, y lo habrá, porque ese caballero
es rico, no me importa. Él y su dinero van rodando\ldots{} Créelo como
te lo digo\ldots{} Con que ya ves cómo las gasto. Me he propuesto que
seas buena, y lo serás, vaya si lo serás. Y para que te convenzas de la
energía, de la honradez de mi resolución, te diré que me constituyo en
tu hermano. Con el esposo perdido, el padre ausente, ¿qué sería de la
pobrecita Rafaela si ahora no tuviese el amparo de un hermanote muy
bruto, muy leal, muy honrado?\ldots{} ¡Ay!, honrado no fui, ahora lo
soy, y derecha has de andar, mal que te pese, porque yo, con la voz de
tu padre y la mía juntas, te digo: «¡Rafaela, cuidado; Rafaela, que soy
tu hermano, y como tal te dirijo, te castigo, y si es preciso\ldots{} te
mato!»

---¡Matarme!---exclamó la \emph{Perita en dulce} abstrayéndose,
balanceando su pensamiento en vaguedades recónditas, lejanas.---Puede
que esa fuera le mejor corrección.

---Lo dicho\ldots{} Ya me conoces. No gasto palabras ociosas.

Desde hoy, ten en cuenta que te vigilo, que no darás un paso sin que yo
lo sepa\ldots{} Por mucho que te recates, por grande que sea tu
habilidad para escabullirte, no te librarás de mi vigilancia\ldots{}
Mucho ojo, señora Doña Rafaela del Milagro.

---¡Vaya por Dios!\ldots{} ¡Qué hermanito tan fiero! ¿Y me libraré de la
tragedia queriendo a Catalá?

---Queriendo a Catalá, que bien lo merece el pobre; a él solo,
solo\ldots{} Adiós\ldots{} Ya es hora de comer. Hasta mañana.

Salió dejándola más meditabunda que asustada, y en el pasillo se
encontró a María Luisa, que había oído lo más substancial de la
conferencia, agazapadita tras la vidriera de la alcoba, y no quiso
dejarle partir sin expresarle su entusiasmo y gratitud por la buena
obra. No estimando discreto el hablar del caso donde Rafaela pudiese
oírla, se contentó con besar las manos del valiente y generoso amigo de
la casa.

\hypertarget{xvi}{%
\chapter{XVI}\label{xvi}}

Obediente quizás a estímulos de su conciencia, o a otros móviles que por
el momento nadie conocía, volvió Rafaela a la vida regular, entendiendo
por esta el no excederse demasiado en los desatinos, no dar motivo a los
desplantes furiosos de Catalá y suspender las salidas nocturnas.

No pudo gozar todo lo que quisiera el buen Catalá de la dichosa enmienda
de su ídolo, porque a consecuencia de los pasados berrinches cayó
gravemente enfermo de un ataque a la cabeza, y por poco toma el portante
para el otro mundo. Con algo de espontaneidad por su parte, y con no
poca docilidad a los mandatos de Ibero, Rafaelita se portó muy bien en
aquella ocasión, visitando diariamente a su amigo enfermo, asistiéndole
con exquisitos cuidados y consolándole con su presencia. En cuanto al
\emph{Don Frenético}, no fue posible espantarle tan pronto como se
quisiera. El enamorado petimetre limitábase a obsequiar a su ídolo, no
ya con ramos de flores, que no eran admitidos, sino con novelas,
mostrando una preferencia de buen gusto por las pocas de Balzac que en
aquellos tiempos se habían traducido al castellano. Rafaela no sabía
francés; pero \emph{Don Frenético}, galómano furibundo, como recriado en
París, había querido iniciar a su amada en el conocimiento y en la
admiración del gran pintor de las pasiones, miserias y vanidades
humanas. Un día y otro dejó en la casa \emph{Úrsula Mirouet},
\emph{Honorina}, \emph{El lirio en el valle}, \emph{La piel de zapa}.
Leía María Luisa, tardando algún tiempo en tornar gusto a una literatura
en todo diferente de la poesía caballeresca de acá; y después tocaba el
turno a Rafaela, que comprendía y apreciaba los profundos análisis de
aquel soberano ingenio mejor que su hermana. «Esto es muy
filosófico---decía María Luisa,---y no va con nosotras\ldots»

A los entretenimientos que retenían en el hogar a las dos hermanas, se
unió bien pronto la faena de ayudar a las de Carrasco en la magna obra
de vestirse a la moderna para presentarse en público como les
correspondía. Largos días y semanas largas se emplearon en esto, primero
con la elección de modelos y de telas, después con las tareas prolijas
del corte y costura. La primera lección que dieron las de Milagro a sus
amigas fue la de prescindir de modistas, trayéndose a casa buenas
costureras que bajo su dirección trabajasen. María Luisa era maestra en
el corte, y Rafaela no tenía rival para el ajuste, combinación de
colores, conforme al modelo vigente de la elegancia, ni para la
adaptación de cada forma al tipo, talle, estatura y corte de cara de la
persona que había que vestir. Poseía el don especialísimo de ver el
efecto, y en todo lo que trazaba ponía un sello personal de gracia y
tono. Instalado el taller en la casa de Carrasco, allá se pasaban todo
el día cortando y cosiendo, con ayuda de buenas oficialas, y no duró
menos de un mes la campaña. En las probaturas que se hicieron para cada
pieza, resultaban las chicas manchegas completamente transformadas; eran
otras, y Doña Leandra creía soñar viendo a sus niñas tan elegantes. Ante
el espejo, Eufrasia y Lea reventaban de satisfacción observando que las
caras se les ponían más bonitas sin necesidad de afeites, y los cuerpos
más esbeltos y airosos por la virtud de aquellos corsés, que parecían
obra de magia.

A cada una de las señoritas de Carrasco se le hicieron dos vestidos de
calle, y uno para teatro y sociedad. Para los primeros eligió Rafaela
las telas llamadas \emph{bareges} y \emph{popelines}, entonces muy en
boga, y resultaron lindísimos, claro el uno, obscurito el otro. En los
faralaes dispuso la directora una gran sobriedad; hubo fuerte discusión
entre ella y su hermana, y al fin, en la primera prueba, todas le dieron
la razón, rindiéndose a su maestría. Los cuerpos o jubones con el cuello
alto, ostentando una imitación de camisa con chorreras, fueron el éxito
más brillante de las Milagros. No se verían en Madrid cuerpos tan
bonitos. Pero en lo que extremaron su ciencia fue en los vestidos de
sociedad, verdaderas obras de arte por la interpretación fiel de la
moda, dejando algo a la invención y \emph{fantasía} personal. Eran de lo
que llamaban \emph{Pekín glacé}, con rayas arrasadas de colores pálidos
y guarnecidos de encajes, canesús de batista bordada con hilo de
Escocia, y cuellito fruncido \emph{a la Lucrecia}. ¡Vamos, que el día
que los estrenaran darían golpe!

Para doña Leandra se confeccionaron dos vestimentas, una de calle y otra
para teatro, entrambas muy apropiadas a la seriedad y modestia de señora
tan respetable. Echaron en el primero no pocas varas de muselina de la
India, de color llamado \emph{de escarabajo}, y en el segundo tafetán
negro de Italia, que adornaron con plegado de cintas \emph{à la
vieille}, todo muy rico, muy bien compuesto, sin extremar el adorno,
porque así lo recomendaba de continuo Doña Leandra, que no quería
desmentir su nativa sencillez, y hacía un verdadero sacrificio en
ponerse aquellos ringorrangos. En las pruebas no disimulaba su mal
humor, repitiendo que tales magnificencias no eran para ella; que no se
acostumbraría jamás a ir por la calle vestida de señorona, y que ya se
sofocaba pensando que la gente se mofaría de su facha. ¡Qué dolor, qué
Madrid este! En los trapos que ella había de lucir, violenta, forzada,
vistiéndose de máscara por dar gusto a la familia, se había empleado el
valor de seis cochinos, y todo el trapío y galas de las hijas suponían
una piara entera, ¡Señor!, la más lucida de Torralba de Calatrava.

Rematado hasta en sus últimos perfiles el grandioso aparato de los
trapitos, lanzáronse todas a la calle, rivalizando en elegancia, pues
las Milagros no querían dar su brazo a torcer, y endilgaron sus más
lindos trajes y perifollos. Hubo días espléndidos de sol en aquel
invierno, lo que a todas vino muy bien para lucirse: iban al Prado y al
Retiro, sin descuidar las visitas de presentación, y al propio tiempo
las madrileñas mostraban a las novatas todas las curiosidades de Madrid,
no olvidando llevarlas, como había recomendado expresamente desde Ciudad
Real el buen D. José, a ver la Historia Natural y Caballerizas. No sólo
se iban soltando con este ajetreo social Lea y Eufrasia, adquiriendo
modales y la desenvoltura madrileña, sino que en sus cuerpos y rostros
se determinó radical mudanza; el encogimiento desapareció al primer
revuelo, y nadie diría que habían venido de la dehesa, cogidas con lazo.
Desprendiéronse pronto del pelo, por virtud del poder asimilativo de la
mujer y de las lecciones vivas que continuamente recibían de las chicas
de Milagro. El éxito coronó la aplicación de las discípulas, así como la
dirección de las maestras, pues a las pocas tardes de andar por el Prado
y Retiro, ya llevaban tras sí las manchegas una reata de novios,
señoritos elegantes que las miraban y las seguían haciendo mil
cucamonas.

Doña Leandra, pasados los primeros días, se resistió a los largos
paseos, no sólo por cansancio, sino porque la mareaba el gentío, y
aumentaban su murria el barullo y regocijo de las tardes de Madrid.
Prefería quedarse en casa, adormecida en triste éxtasis, indelebles
memorias del abandonado terruño, o bien rezando rosarios y pidiendo a
Dios que se realizaran las esperanzas que trajo a Madrid toda la
familia, pastoreada por Bruno. Ya le daba en la nariz a la buena señora
olor de reveses, porque habiendo salido del Ministerio de Hacienda el
señor de Gamboa se rompían los asideros de Carrasco en aquella casa; el
expediente de Pósitos no acababa de resolverse, y lo de la Diputación no
se veía claro, a pesar de los lisonjeros vaticinios que mandaba en todas
sus cartas el seráfico D. José.

Siempre que el servicio se lo permitía, acompañaba Ibero a las señoras y
señoritas en su paseo, pues con Bruno no había que contar: se pasaba la
vida en los ministerios y en tertulias políticas de café y redacciones.
Algunos amigos de Santiago, paisanos y militares, se agregaban a la
feliz cuadrilla, y la charla sabrosa y galante no tenía término. Entre
ellos se señaló un teniente coronel, que hacía continuo derroche de
finezas sin decidirse por las solteras ni por la casadita, como si fuera
su plan tocar todas las teclas a ver cuál le sonaba mejor. Era de cuerpo
pequeño, de carácter francote y comunicativo, cetrino de color, escaso
de bigote y barba, el habla durísima, gorda, catalana. Una tarde que
iban las manchegas y sus amigas con Ibero por la calle de Alcalá, le
encontraron en la \emph{esquina de la calle del Turco}; parose Santiago
al reconocerle, se abrazaron, y al instante hizo la presentación: «Mi
amigo muy querido Juan Prim.»

Siguieron todos hacia el Retiro. Prim, que vestía de paisano, contó a
Ibero rápidamente sus tribulaciones militares y políticas, y luego pegó
la hebra con las damas, que le oían con singular agrado, maravilladas de
su simpática franqueza, de sus atrevimientos gallardos, que se
acomodaban, como al vaso el líquido, a la ruda lengua catalana.
Hallándose María Luisa un poco pesada, próxima ya a meses mayores, solía
ir a retaguardia con Ibero y D. Gervasio. En una de estas, interrogado
el Coronel por su amiga, refirió que el tal Prim era un bravo militar
que había empezado su carrera de \emph{pesetero} en la guerra de
Cataluña, adelantando rápidamente por su valor sereno y su militar
instinto en la dirección de tropas. Chico despejadísimo, llegaría a
donde llegan pocos; y si por entonces parecía fuera de juego y no tenía
mando, no era por falta de méritos, sino por significarse en política
más de lo prudente, con ideas harto exaltadas.

«Pues abran ustedes mucho ojo para vigilar a este pájaro---dijo D.
Gervasio parándose para acentuar mejor el tono profético.---Yo podría
sostener que las ideas del teniente coronel Prim más que exaltadas, son
jacobinas: me consta que no hace muchas noches pronunció en casa de
Pacheco palabras que le valdrían una temporada de castillo si el Duque
las supiera. Hay en este mozo algo que contradice las costumbres que
observamos diariamente en todo joven que politiquea. Fijémonos bien en
esta circunstancia: su amigo de usted profesa ideas que casi, casi tocan
en el republicanismo, y no obstante, se junta con retrógrados, y sus
principales amigotes son lo más granado de la \emph{moderación}. Le verá
usted siempre con Carriquiri, con Salamanca, con Sartorius, y creo que
con Fernandito, el hermano del General Córdoba. ¿No le sorprende a usted
esta contradicción entre las ideas políticas y los gustos sociales?»

---Le diré a usted, amigo D. Gervasio---replicó Ibero:---antes que ese
contraste, veo yo otro más fundamental en ese bravo chico, y es que,
siendo de origen muy humilde, no le gusta tratarse más que con
aristócratas. Ya ve usted qué bien viste: no hay otro que lleve mejor la
ropa, ni quien le iguale en el refinamiento de los gustos; su rumbo, su
esplendidez nos harían creer que es noble de nacimiento; sus ideas dicen
que es hijo de la plebe. Yo le quiero y le admiro.

---Pues a mí me da mala espina\ldots{} Mi opinión es que se vigile a
estos plebeyos que andan demasiado elegantes, y a estos \emph{peseteros}
que adquieren costumbres de próceres.

---La contradicción yo no la temo, y hasta le creo natural, D. Gervasio.
Todo hombre es una carrera, una vida que viene de un punto y a otro se
dirige\ldots{} Si el hombre no se aleja del punto de partida, ¿en dónde
está el progreso, nuestro \emph{Progreso}, que tanto amamos y por el
cual hemos dado terribles batallas? En Prim ve usted las ideas avanzadas
de origen plebeyo y las aficiones aristocráticas: las primeras son los
principios, las segundas son los fines.

Creyera o no D. Gervasio paradójica y vana la explicación de Ibero, ello
es que no añadió más que lo siguiente: «Estamos perdidos si no se vigila
a los exaltados que andan entre obscurantistas. Lo dice un hombre de
larga y dolorosa experiencia de las cosas públicas. Si yo tuviera, como
usted, mi querido amigo, acceso diario en la casa del señor Duque, le
saludaría siempre con estas palabras sibilíticas:\emph{Palo al
jacobinismo, palo al retroceso.»}

Procuró Ibero quitar importancia a estos vaticinios del funcionario que
se pasa la vida temblando por su nómina, y siguieron. A la semana
siguiente, agregado también Prim al convoy, halló ocasión de quedarse
atrás con su amigo, y le dijo:

«Sé que vas a la parte en los favores de la viudita, y\ldots»

---¿Qué viudita? ¿Rafaela?\ldots{} es casada.

---¡Ah!, sí\ldots{} la casada solitaria, de quien me han contado\ldots{}
¿Qué? ¿Seré indiscreto?

---Sigue hombre, sigue.

---Es monísima, y sabe como ninguna hacerse la candorosa. Diríamos que
no rompe un plato. ¿Pero es verdad que tú\ldots?

---Sí, hombre, sí. Sigue, ¡ajo!

---Pues me alegro de tu franqueza, porque así puede la mía serte de
algún provecho. Al amigo la verdad\ldots{} Esa\ldots{} te engaña.

---Sí, hombre, sí. Acaba pronto. ¿Quién\ldots?

---Vas a saberlo. Ayer salíamos de almorzar en casa de Carriquiri,
Narciso Ametller, Luis Sartorius y yo\ldots{} Al volver la esquina de la
calle de las Huertas, vimos a tu amiga salir de un coche con Federiquito
Nieto, y entrar\ldots{} ¿sabes ya dónde?

---Basta; no sigas: esta noche la mato.

---Hombre, no es para tanto.

---¿Qué sabes tú?

---Siento\ldots{}

---No sientas nada\ldots{} te digo que la mato\ldots{} Y a ese \emph{Don
Frenético} le pisotearé en medio de la calle, en cuanto le encuentre.
Ella me había prometido\ldots{} No, no fue a mí\ldots{} no soy yo.
Cállate, déjame. Yo sé lo que tengo que hacer.

---Pues Ametller me contó algo más\ldots{}

---No sigas: estamos llamando la atención. Ya ves: se paran todos
esperándonos.

---Creerán que conspiramos. Y si quieres, por mí no ha de quedar.
Conspiremos, Ibero.

---¿Ves? Se ríen de nosotros.

---Se reirán de ti\ldots{}

---Cállate ya\ldots{} ¿En dónde nos veremos mañana para poder hablar?

---En ninguna parte, porque yo me voy a Tarragona, donde espero salir
diputado.

---Bien, hombre, bien\ldots{} Para ti es el mundo. ¿Y votarás la
Regencia una o trina?

---Creo que con un solo Regente basta y sobra. De lo malo, poco.

Uniéronse al grupo, y el paseo tuvo su desarrollo natural sin incidente
alguno. En torno de las damas revolotearon los pretendientes,
derrochando su gárrula estolidez amorosa. Ibero, metido en sí, no cesaba
de pensar: «¡Pobre Catalá! Bien le decía yo a María Luisa que estas
saliditas de mañana no tenían explicación, y ella me porfiaba que
sí\ldots{} que iba a la cordonería, al tinte\ldots{} Enredos\ldots{}
María Luisa tapa. Pues aquí estoy yo para destapar a la tapada y a la
tapadera.»

\hypertarget{xvii}{%
\chapter{XVII}\label{xvii}}

No tuvo Ibero reposo hasta que vio llegar la mejor coyuntura para
interrogar a Rafaela. La increpó con severidad, afeándole su hipocresía
y falta de juicio, y ella, negando al principio, balbuciendo luego una
tímida confesión, sin descubrir el doble fondo, echó por fin un raudal
de lágrimas sobre la disputa. El rígido censor, apiadado, no quiso
añadir un martirio más a los que a la pecadora infligía su conciencia, y
calló, mandándole que se sosegara. Aquella misma tarde habló a solas con
María Luisa, de cuya boca oyó conceptos que cayeron como lluvia glacial
sobre su corazón. No esperaba, ciertamente, aquella filosofía de comodín
que era al propio tiempo censura y tolerancia de los deslices de
Rafaela, ni el desdén con que apreciaba la intervención caballeresca de
él en asunto tan grave como el honor de la familia.

«No podemos hacer carrera de ella---decía María Luisa.---Y lo que
siento, amigo Ibero, es que usted se dé tan malos ratos para no
conseguir nada\ldots{} Hablando con franqueza, yo no creo que Rafaela
sea un monstruo, ni mucho menos\ldots{} Los actos de las mujeres no
deben juzgarse sin mirar un poco a las circunstancias, y las de mi
hermana ya sabe usted cuáles son. Hay que verlo todo, amigo mío, y no
ser demasiado severo. Francamente, yo me pongo en el caso de
Rafaela\ldots{} El tal Catalá no es hombre de tantísimo mérito que
merezca sacrificios extremados. Si se tratara de usted, ya sería otra
cosa\ldots»

Aterrado más que sorprendido, Ibero no supo qué contestar.

«Yo comprendo---prosiguió María Luisa,---que si usted no hubiera rifado
con ella, haría muy bien en ponerle el grillete\ldots{} Tal como están
las cosas, no podrá usted enderezar a mi hermana todo lo que deseamos, y
de veras lo siento yo; no podrá enderezarla, digo, porque usted la
enseñó a torcerse\ldots{} No es esto censura, líbreme Dios\ldots{} ojalá
durara\ldots{} es decirle a usted que no se aflija porque sus sermones
sean de tan poco efecto\ldots»

---Tiene usted razón, María Luisa---dijo Ibero, cayendo de un nido, de
las nubes, de más alto aún:---soy un necio, el mayor mentecato de la
orden de diablos predicadores. Usted me abre los ojos\ldots{} No es sólo
Rafaela la que está dañada en esta casa.

Las señales del grave daño estaban a la vista, pues rodeaban a María
Luisa muestrarios de telas, piezas riquísimas de \emph{barege eoliana},
de muselina de la India, de tafetán de Italia, y cachemiras, crespones y
\emph{popelines} de dobles reflejos. Tantos y tan lucidos trapos se
veían allí, que el gabinete parecía un taller de modas de los más
elegantes. Ya había notado Ibero que la transformación indumentaria de
las manchegas fue para las Milagros como súbito envenenamiento: la
elegancia de sus amigas les inoculó el virus del lujo, y este prendió al
instante con aterradora intensidad. La primera envenenada fue Rafaela,
que no tardó en comunicar a su hermana el pegajoso mal. Bien pronto
invadieron la casa figurines y piezas de tela, mil arrumacos elegantes
de seda y encaje, modelos de los abrigos llamados \emph{twines} y
\emph{kasadawekas}, que se adornaban con pieles riquísimas, y Rafaela
frecuentaba la famosa casa de Madame Petibon, depósito de todas las
monerías parisienses de última novedad.

«Aunque tarde---dijo Ibero melancólico, tirando a la indulgencia que un
hombre debe a la flaqueza mujeril,---caigo en la realidad, y veo la
ridiculez de mis pretensiones puritanas. ¿Me permite usted, María Luisa,
que le hable con la libertad a que tiene derecho un amigo que se
despide? Pues si usted no se me enfada, le diré que el dinero enviado
por Don José para gastos de ropa (y conozco la cantidad porque ha pasado
por mi mano), no basta ni con mucho para ese aluvión de trapos\ldots»

---No hay que asustarse, amigo Ibero\ldots{} Mucho de esto se devuelve;
lo hemos traído sólo para verlo\ldots{}

---Déjeme seguir. Si ustedes pensaban que debían estirar los pies a
mayor largo que el de las sábanas, ¿por qué no me pidieron a mí el
dinero necesario, como mil veces le he dicho a Rafaela?\ldots{} No se
enfadará usted tampoco si, como leal amigo de D. José, le digo que es un
grandísimo peligro esa ostentación\ldots{} vamos, ese insulto a la
medianía de un jefe político que blasona de honrado, y que lo es\ldots{}
lo es.

---Papá nos autoriza para vestirnos decentemente, contando con lo que
nos mandará luego. No quiere que hagamos mal papel al lado de las
manchegas. Además, diré a usted que a Cavallieri han venido a buscarle
para que cante los meses que quedan de temporada en la Cruz; un contrato
ventajosísimo, amigo Don Santiago. El público no está contento de
Reguer, y Becerra se ha puesto ronco. Tendrá usted a mi marido de primer
bajo, con obligación de cantar \emph{Chiara di Rosemberg}, \emph{Marino
Faliero}, \emph{Il Conte Ory}, del gran Rossini, y la ópera que ha
escrito nuestro celebrado Saldoni, \emph{Cleonice, Regina di Siria}.

---Lo celebro infinito. Iré a dar mi aplauso al amigo Cavallieri, y a
admirarlas a ustedes en su palco de la Cruz. No se ofenda por lo que he
dicho, ni aquí hay nada que censurar, como no sea mi conducta: me daría
de bofetadas\ldots{} tal rabia me tengo, puede usted creerlo\ldots{} por
meterme yo en donde no me llaman. Todo lo que dije de querer ser su
hermano, y de guiarlas y protegerlas, como tal, contra los infinitos
riesgos de este Madrid diabólico, no es más que un quijotismo que, ya lo
ve usted, viene a parar en lo que para siempre el meterse a pelear con
aspas de molino. Aquí me tiene usted caído y con los huesos
quebrantados; pero aprovecho la lección, vaya si la aprovecho,
¡canastos! No volveré, no, a romper lanzas por el honor de nadie, ni a
enderezar mujeres que quieren torcerse. Hermoso me parecía lo de ser
hermano de estas pobrecitas, y ello me servía como de un buen descargo
de mi conciencia; pero ya veo que el oficio de hermano postizo tiene sus
quiebras, y\ldots{} dimito el cargo.

---Siempre será usted un buen amigo nuestro, por más que no
quiera---dijo María Luisa, un poco asustada de verle con tal impresión
de tristeza y desaliento.---Diríjanos y aconséjenos todo lo que guste,
que bien sabe Dios cuánto hemos de agradecérselo. Lo único que le pido
es que no sea demasiado regañón con nosotras, vamos, que no nos grite ni
ponga los ojos fieros, porque me asusto\ldots{} crea que me
asusto\ldots{} y como entro ya en meses mayores, cualquier sobresalto
repentino podría\ldots{} ya sabe\ldots{}

---Esté usted tranquila, que por culpa mía no ha de fracasar la
criatura. Le deseo un felicísimo alumbramiento, y a Cavallieri ovaciones
sin fin. Con que\ldots{} a ver si acaban ustedes todo el traperío, para
que se pongan bien guapas y tiemble Madrid.

---¡Burlón, mala persona!

---Adiós, amiga mía. Adiós.

Se fue, no ya triste, sino consternado, pues era hombre a quien
afectaban hondamente las rupturas o interrupciones de amistad, de
cualquier orden que fuesen. Aquel mismo día visitó al pobre Catalá, y le
halló tan tranquilo, tan confiado, que habría sido no sólo impertinente,
sino criminal, turbar su almo reposo. Por todo ello, se confirmaba en su
propósito de abandonar definitivamente la redención de pecadores, obra
que a Dios pertenecía, no a los hombres, y menos aún a los que se hallan
distantes de la perfección. «Hagamos todo el bien que podamos---se
decía;---pero dejando siempre a un lado los trastos de redimir.»

En los siguientes días, atraída su alma solitaria con nueva fuerza desde
La Guardia, fue a ver a Doña Jacinta y después al Duque, con la
pretensión de que, si no le trasladaban al Norte, como era su deseo, se
le diera al menos una licencia de un mes, de dos semanas. Don Baldomero,
meditabundo, mas como nunca benévolo, le dijo: «Ten paciencia, Santiago.
Ahora no puede ser. En cuanto se reúnan las Cortes y estas elijan la
Regencia, podrás ir a donde quieras.»

Por algo que dejó escapar la suma discreción de Espartero, por lo que
poco antes le había dicho la Duquesa, y por lo que oyó después en la
Secretaría, entendió Ibero que el Gobierno olfateaba conspiraciones.
Síntomas de displicencia apuntaban en ciertos círculos, resto nefando de
las antiguas logias; cuchicheos misteriosos sonaban en los cuarteles. El
\emph{retroceso}, abrazando con sentimental quijotismo la causa de
Cristina, y declarándola víctima inocente de una intriga brutal, se
apiñaba para adquirir una fuerza de que carecía. Los moderados elegantes
y ricachones usaban del resorte social de las suntuosas comidas para
producir la agrupación lenta de adeptos, para definir y caldear las
ideas que, por el pronto, sólo se expresaban en forma de chistes y
agudezas contra el Duque, su familia y adláteres. Figuras importantes
del Ejército iban marcando su actitud \emph{paladinesca} en favor de la
ilustre proscripta, que recibía corte de descontentos en su residencia
de la Malmaison, comprada a los herederos de Josefina. No era sólo
Belascoain el que cerdeaba. Manuel de la Concha tenía muy arrugado el
entrecejo, y su hermano Pepe, amigo de Espartero y a punto de emparentar
con él, no podía vencer la sugestiva atracción de su hermano; de Juanito
Pezuela nada podía asegurarse; O'Donnell era declarado cristino; mas su
fría cara irlandesa no revelaba sus intenciones. Seguros eran Seoane,
que mandaba en Valencia; Van-Halen, en Cataluña; Ribero, en Navarra. En
cuanto a la Milicia Nacional, se creía en su fidelidad como en Dios,
viéndola cada día más firme en su liberalismo chillón, ardoroso,
pintoresco.

Dos días después de la visita a Espartero hizo otra a Linaje, que le
retuvo más de una hora, encareciéndole la necesidad de vigilar con cien
ojos y de aplicar el oído a las conversaciones de la oficialidad,
siquiera fuesen de las más íntimas. Se habían emprendido trabajos en
algunos cuerpos por el sistema llamado del \emph{triángulo}, y no eran
pocos los jefes y oficiales que andaban en estos enredos. Urgía
conocerles y desenmascararles antes que las cosas fuesen a mayores. Por
lo demás, no se temía nada serio, y la popularidad y buen crédito del
Duque garantizaban una paz durable\ldots{} Con todo se mostró conforme
Ibero, y prometiendo ser un Argos de buen oído, y no perdonar medio
alguno, por duro que fuese, para imponer castigo a los que se salieran
de la estricta disciplina, se despidió del famoso secretario del Duque,
creyéndole atormentado por pesadillas horrendas, a no ser que inventara
las conspiraciones para dar a sus servicios un valor que fuera del
terreno policiaco no podían tener.

Recibió en aquellos días Ibero una carta de Navarridas muy grata y
consoladora. ¡Cuánto habría dado el hombre por poder llegarse allá y
recrear sus ojos en la contemplación del dulce objeto de su amor fino, y
hablar con Gracia, con la sin par Demetria y con Navarridas de proyectos
felices cuya realización no debía de estar lejana! Pero ¡ay!, vana
ilusión, sueño de esclavo era pensar en esto. Viéndose tan sin libertad
privada por servir a la pública, fue acometido de un tedio sombrío, con
desvío de la sociedad y repugnancia del trato de gentes; se pasaba en su
casa largas horas leyendo novelas, sin distinguir de géneros y estilos,
devorándolas todas con igual atención; y en medio de aquel fárrago
pasaron también las de Balzac que semanas antes le había dado María
Luisa, y procedían de la mano dadivosa de \emph{Don Frenético}.
Volvieron a despuntar en su mente los delirios supersticiosos que le
habían trastornado en Valencia, y por las noches cualquier sombrajo en
la habitación obscura o en la calle tomaba forma de animado ser para
significarle sucesos terroríficos. Una mañana fue a coger su bastón del
sitio donde comúnmente lo ponía, y el bastón cayó al suelo, y al bajarse
para recogerlo movió con el hombro un colgadero portátil de ropa, que
vino a desplomarse sobre la mesa. En esta había un plato (del servicio
de chocolate), que al golpe se rompió por la mitad, mostrando en uno de
los pedazos rotos el perfil perfectísimo de una cara burlona, la cual
cobró vida y voz en el instante de la rotura, y así le dijo: «Teme a los
traidores.»

\hypertarget{xviii}{%
\chapter{XVIII}\label{xviii}}

A los traidores ya les temía y execraba, sin necesidad de que el maligno
ente se lo advirtiera. Lo que hacía falta era descubrirles y saber por
dónde andaban, para meterles mano y hacer en ellos un cruel escarmiento.
Coincidieron estas travesuras de la imaginación con un soplo que en
aquellos días le dio el Mayor del segundo batallón de su regimiento, D.
Gabriel O'Daly. Mandaba la primera compañía del mismo un capitán llamado
Vallabriga, tildado de inquieto y sospechoso. Según O'Daly, hombre de
carácter muy serio y de bien probada veracidad, Vallabriga andaba en
malos pasos y en peores trotes. No era difícil comprobar que había leído
proclamas clandestinas a varios sargentos de su compañía; se supo que
frecuentaba una reunión nocturna de \emph{jovellanistas} en una de las
calles jorobadas y tortuosas que caen detrás de Buenavista, no lejos de
las Salesas, conciliábulo a que concurrían otros militares de distintos
cuerpos. Con estos nada tenía que ver D. Santiago; pero como descubriera
y evidenciara al traidor de su regimiento, sorprendiéndole \emph{con el
puñal levantado sobre el corazón} de la patria, no se contentaría con
menos que con atravesarle de una estocada sin más dimes ni diretes, ni
sumaria ni consejo de guerra. Nunca le había gustado el tal Vallabriga,
que componía versos de moros y cristianos, blasonaba de ideas
estrambóticas, y solía concurrir a las tertulias de café peor reputadas.
Hizo propósito de seguirle la pista y de echarle la zarpa, sin dar
cuenta a nadie de su cacería, ni valerse de persona alguna militar ni
civil.

Pero estaba de Dios que en aquellos días su alterada mente no tuviera
reposo, porque tras una impresión desagradable venía otra de un orden
distinto, y el hombre no ganaba para disgustos. Hallábase una tarde en
el Cuarto de banderas, durante el acto de pasar lista, tocando la música
en el patio, cuando entró Catalá demudado y trémulo, y con balbuciente
voz le dijo: «La mato, Santiago, la mato, la degüello\ldots{} Ahora no
la salva ni el \emph{Sursum corda.»}

A las preguntas de Ibero no respondía sino con expresiones
desconcertadas y delirantes, acariciando una pistola que llevaba en el
bolsillo interior de la levita. «¿Sabes tú dónde podré
encontrarla?\ldots{} Porque en su casa no está\ldots{} ¡Cuatro noches
pasadas fuera! Es un demonio, es la mentira, la traición. De hoy no pasa
que le meta una bala en el cráneo\ldots{} No me mato yo\ldots{} yo
no\ldots»

Y diciéndolo salió disparado sin oír las exhortaciones de su amigo, que
a la moderación le incitaba. No se sentía Ibero con ganas de tomar en la
cuita del comandante un papel activo: bastaba con tenerle lástima y con
desear que las cosas se arreglaran por las buenas, sin catástrofe. Desde
que renunció al desairado papel de paladín de la honra Milagrera, sus
comunicaciones con las graciosas hermanas eran casi nulas. Supo que
María Luisa había dado a luz con toda facilidad un niño que se parecía
mucho a Cavallieri, y se enteró de que a este le habían dado una grita
fenomenal en la Cruz, cantando \emph{Le Prigioni d'Edimburgo}, de Ricci,
con la Mazarelli, la Lombía y Ojeda, y que a consecuencia de este
desastre enmudeció en los teatros la espléndida voz de bajo para tronar
de nuevo en los responsos y funerales. De Rafaela no supo más sino que
la habían visto sola, por la calle de Alcalá abajo, luciendo un twine de
todo lujo, guarnecido de pieles, y que en el teatro del Circo había
llamado la atención en un palco, con elegantísimo vestido, en compañía
de las manchegas. Las relaciones de Ibero con Catalá no eran ya muy
íntimas. Como el pobre Comandante no acababa de restablecerse del mal de
su desconcertada cabeza, Santiago influyó para que se le retirase del
servicio activo, y a sus instancias le colocó Linaje en la Secretaría
del Montepío Militar.

La tarde en que se presentó Catalá en el cuarto de banderas de
\emph{Saboya} con aquel rapto de ira, no pudo Santiago ir en su
seguimiento para impedir una barbarie, porque había recibido invitación
para comer con los señores Duques, y el meterse a componedor habría
comprometido su puntualidad. Por la noche, en el café de Pombo, supo que
no había ocurrido tragedia clásica ni romántica, porque los compañeros
de oficina de Catalá habían recogido a este, llevándosele a su casa y
quitándole las pistolas y todo instrumento que pudiera ocasionar muerte.
Mas no pudiendo permanecer de guardia indefinidamente en su alcoba,
temían la repetición del acceso de furia, el cual no era un fenómeno
morboso, sino arrechucho normal producido por discordias terribles con
su amada infiel.

A los tres días de esto, el 19 de Marzo, se abrieron las Cortes, y ya no
se hablaba en Madrid más que de la elección de Regencia, y de si esta
sería una, trina o cuaternaria. Muchos amigos tenía Ibero en el
Parlamento que había de resolver cuestión tan peliaguda. Triunfaron Prim
y Olózaga; elegidos fueron también González Bravo, Ametller y Posada
Herrera. En cambio, el pobrecito D. Bruno Carrasco había sufrido una
derrota ignominiosa, a pesar de \emph{tener el padre alcalde}; y el
bonísimo D. José del Milagro, a quien el fracaso produjo terribles
amarguras, fue acusado por los amigos de no entender la mecánica
electoral, de haber conducido a las urnas el rebaño votante con el modo
y pasos de la más candorosa legalidad y de una corrección infantil. Por
no parecerse a los moderados, había dejado indefensa la candidatura del
amigo, y él quedaba como un modelo de la probidad más imbécil. Tal era
el criterio de la llamada razón política, enteramente reñido \emph{et
nunc et semper} con toda idea moral.

Ya se aproximaba la elección de Regente, cuando Ibero, libre de todo
compromiso social y militar, escogió una destemplada noche de Marzo para
lanzarse al ojeo de aquel indigno Vallabriga, que era el oprobio de la
brillante oficialidad de \emph{Saboya}. Un dato de la policía,
transmitido por O'Daly, le dio a conocer que la junta secreta de
jacobinos y moderados (¡nefando amasijo!), a que concurría el pérfido
capitán, se había trasladado a una de las calles próximas a la plazuela
de Afligidos, entre el cuartel de Guardias y la Cara de Dios. Allá se
fue el hombre, en traje de paisano y trazas de cesante, bien embozado en
su pañosa, y con un sombrero del año 23 que completaba el disfraz de un
modo perfecto. Calles arriba, calles abajo, midió todo el barrio durante
dos lentas horas, sin descubrir rastro ni sombra de lo que perseguía; y
cansado ya de su inútil acecho, se retiraba por la calle del Limón,
cuando vio salir de un portal tenebroso a una mujer, cuyos andares y
figura le revelaron persona conocida, sin poder discernir quién era,
pues iba bien entapujada con manto negro y cuidadosa de no dejarse ver
la cara. El corazón, más que los ojos, fue quien le dijo a Ibero: «O yo
veo visiones, o esta es Rafaela.» La siguió a distancia. Avivaba ella el
pasito como si hubiera notado la persecución; al llegar a lo alto de la
calle torció a la izquierda por un solar vacío, y tomó la calle de
Amaniel; acortó Ibero la distancia, y observando mejor a la luz de los
reverberos, se confirmó más en su sospecha. Entró luego la tapada en la
calle de San Hermenegildo, lóbrega, solitaria, de aspecto mísero, y el
galán tras ella. La macilenta luz de los escasos faroles apenas permitió
al ojeador distinguir el bulto, que no ya de prisa, sino a la carrera,
por la calle avanzaba. De pronto se filtró en un portal. Reconoció
Santiago la casa donde había desaparecido la mujer, y observó que no era
de mal aspecto; la mejor de la calle sin duda. Una luz pitañosa,
semejante a la mirada de un ojo enfermo, brillaba en lo más hondo del
portal larguísimo y angosto.

Hasta aquí la aventura era por demás insípida, pues aun suponiendo que
la hembra escurridiza fuese Rafaela, ¿qué interés podían tener ya para
Ibero los pasos rectos o torcidos de la que fuese su amante? Pensó
retirarse, y una fuerza íntima, nacida de su suspicacia y de su
curiosidad juntamente, le retuvo. «Me da el corazón---se dijo,---que aún
he visto poco, y que debo quedarme aquí para ver más.»

Aunque comúnmente no era hombre para largos plantones, determinó hacer
aquella noche pruebas de paciencia, y buscando el sitio más adecuado
para garita, dio con un cerrado portal, que parecía un nicho, en uno de
los trozos más obscuros de la calle, en la acera opuesta a la de la casa
misteriosa, y a una distancia tal de esta, que no era difícil observar
quién entraba y salía. Porque en la tal casa había de ocurrir algo
extraordinario; a Ibero se lo dijo la singular fisonomía que resultaba
de la disposición de sus huecos; se lo dijo la ordenada fila de las tres
repisas de balcones, la combinación de pintura roja imitando ladrillo, y
de pintura blanca imitando piedra; díjoselo también una ventana
figurada, y, por último, se lo confirmó un letrero pendiente entre las
dos rejas del piso bajo. Pudo leer el primer renglón, \emph{Imprenta}, y
el de que había más abajo; pero el nombre expresado en la tercera línea
no era legible, ni hacía falta por el momento.

No habían pasado quince minutos de plantón, cuando Ibero vio salir a dos
hombres, embozados en luengas capas. Tiraron hacia la calle de San
Bernardo. Parecían señores. Diez minutos después salió uno solo,
enfundado en un gabán con alzacuello altísimo. Aquel sí era señor
efectivo. Le vio Ibero pasar cerca, porque tiró hacia la calle de
Amaniel. No pudo ver su cara; no le conocía por el cuerpo y andadura. De
pronto, el tal sujeto retrocedió como azorado, vaciló un instante, y al
fin salió por pies hacia la calle Ancha con no poca prisa. Antes de
perderle de vista, vio salir a otro, y luego a dos\ldots{} «¿Pero qué
jubileo es este? Aquí hay una guarida de conspiradores---pensó, dejando
caer el embozo.---Vamos, no aguanto más. Me pondré en la misma puerta, y
si sale \emph{mi traidor}, el \emph{Judas de Saboya}, no le dejaré hueso
sano\ldots» Con paso resuelto avanzó hacia la casa, y al aproximarse al
portal, casi estuvo a punto de chocar con dos bultos que salían\ldots{}
un hombre y una mujer. Esta era Rafaela: la vio cara a cara; no podía
dudar de lo que veía. Y como en aquel súbito encuentro, obra de un
instante, aplicara toda su atención a la hembra, no pudo distinguir bien
la persona del hombre, que al verse sorprendido se embozó hasta la
nariz. No obstante, en rápida visión, que Ibero pudo comparar a la fugaz
claridad del relámpago, se le manifestó un semblante hermoso, un bigote
rubio\ldots{} nada más. Quedó en su retina la vaga impresión de un
rostro conocido; mas ni en aquel instante ni en los que sucedieron al
encuentro, pudo discernir quién era.

Avanzó la pareja por la calle adelante, hacia la de San Bernardo, y a
distancia les siguió Ibero. Iban hombre y mujer muy pegaditos, hablando
en intimidad confianzuda. Al pie de la mole churrigueresca de Montserrat
se pararon un rato; el desconocido parecía reñir amorosamente a Rafaela.
Siguieron, y en otra parada comprendió Santiago lo que podría llamarse
el sentido escénico de aquel coloquio. Sin oír nada, pues la distancia
no lo permitía, pudo, con la sola observación de la pantomima de ambos,
comprender que el galán la incitaba a que se separaran. No convenía, por
estas o las otras razones, que fuesen juntos. Ella se obstinaba en
acompañarle; él en que no. Hubo sin duda transacción entre las opuestas
voluntades, porque siguieron hasta el Noviciado. En una nueva paradita,
reparó Ibero que la Milagro lloraba, llevándose el pañuelo a los ojos, y
que el caballero le apretaba las manos. Pareció indicarle que se
retirara por la calle de los Reyes al punto que debía de ser su
residencia eventual. Ella se resistía; cedió al fin ante exhortaciones o
mandatos impuestos con voluntad firme\ldots{} La despedida fue tierna,
penosa, lenta: se apartaban y volvían a reunirse, siendo ella la que
tras él corría, como desconsolada de verle partir\ldots{} Esto fue obra
de un minuto, quizás de dos, y por fin el hombre arrancó presuroso calle
abajo, y la sombra de ella se desvaneció en la travesía más próxima.

Dudó un instante Ibero\ldots{} ¿A cuál de los dos seguiría? El primer
impulso fue dar caza a Rafaela; pero de pronto una sospecha vivísima le
indujo a la determinación contraria: seguir al hombre. Creyó haber
encontrado en sus recuerdos la clave del enigma de aquel rostro, visto
en un relámpago, y quería comprobarlo con nueva observación. El hombre
iba de prisa por la acera del Noviciado, Ibero por la opuesta, avivando
el paso con intento de tomarle la vuelta y mirarle de frente. Pero
cuando ya el desconocido iba cerca del Rosario, vio pasar un simón: lo
tomó precipitadamente y metiose en él, dando al cochero la orden desde
dentro. Santiago, que se aproximó cuando el caballero cerraba con
violencia la portezuela, no pudo ver lo que deseaba. Fue luego en
seguimiento de Rafaela; mas ya era tarde. Ni aun pudo determinar la casa
de que la vio salir, en la mísera y tenebrosa calle del Limón.

\hypertarget{xix}{%
\chapter{XIX}\label{xix}}

Al día siguiente visitaron los corchetes la casa de la calle de San
Hermenegildo en cuyo piso bajo estaba la imprenta de Minutria; mas no se
encontró nada que transcendiese a conspiración. En el principal había un
colegio de niñas, y los vecinos del sotabanco eran vendedores
ambulantes, un cochero y dos limpiabotas. En la imprenta se había tirado
\emph{El Eco del Comercio}, después \emph{El Huracán}, y a la sazón se
imprimían dos papeles, cuyo ministerialismo no podía ponerse en duda; el
dueño de ella era miliciano nacional, considerado en el cuerpo como de
intachable adhesión al Duque.

Pensó Ibero, como síntesis de sus cavilaciones de aquella noche y del
siguiente día, que no cuadraban al decoro de su posición militar las
correrías y acechos de polizonte, desfigurando su persona; y creyendo
haber descubierto un rastro de criminales \emph{liberticidas}, se
propuso seguirlo, mas no con tapujo, sino a cara descubierta, de
uniforme y a plena luz. Comenzó por la tarde sus indagaciones en la
calle que fue principio de su aventura, y tan propicia le fue la suerte,
que a primera hora de la noche ya conocía el escondrijo de Rafaela, el
cual resultó ser la vivienda de una planchadora llamada Encarnación,
nodriza que fue del chiquillo mayor de Milagro. Comió el Coronel a la
francesa, con unos amigos, en el próximo cuartel de Guardias, y a punto
de las ocho se personó en la casa, presumiendo, como en efecto sucedió,
que al preguntar por la extraviada la negarían. «¿Cómo se entiende? Sé
que vive aquí; sé también que está en casa---dijo en tono que no admitía
réplica,---y si se obstinan en negarla, ya veré yo la manera de
despabilar a los que ocultan la verdad.» Diciéndolo, empujaba suavemente
a la mujer que abrió la puerta, y sin reparo alguno se colaba por un
pasillo, a cuyo extremo compareció un hombre corpulento, en mangas de
camisa, al modo de tapón para cerrar el paso. Antes que el tal formulase
una protesta, le echó mano al cuello D. Santiago, diciéndole: «Que salga
pronto Rafaela, ¡ajo!; y renuncien a ocultarla si no quieren ir a la
cárcel todos los inquilinos, empezando por usted y concluyendo por el
gato.»

El gato apareció detrás del dueño, mirando receloso al intruso; dos
chicos tiznados salieron detrás del gato, haciendo pucheros; se
persignaba la mujer, rezongaba el hombre, escupiendo palabras
descorteses; y en esto se abrió una puerta vidriera al opuesto extremo
del largo pasillo, y la turbada voz de Rafaela dijo claramente: «Sí, sí,
Santiago, aquí estoy. Puedes pasar.»

---¡A mí con estas bromas de negarte! Ya comprenderás que vengo como
amigo, y que no te causaré ningún daño\ldots{}

Entrando en la sala con esta breve insinuación, y posesionándose de la
primera silla que se le vino a mano, invitó a la Milagro a sentarse.
Alumbraba la estancia un quinqué bastante avaro de claridad, con
pantalla de cartón, puesto sobre una cómoda, y en todos los muebles se
veían prendas de vestir, esparcidas con desorden, ropa blanca recién
planchada, zapatos y ligas. Rafaela, envuelta en un mantón, despeinada,
los pies metidos en pantuflas turquescas de tafilete amarillo bordado de
plata, se acomodó en un sillón frente a Ibero, mediando entre los dos un
brasero sin lumbre. Parecía enferma o profundamente atribulada, y en su
bello rostro, que nunca fue romántico, se advertían las transparencias
opalinas y el nácar violáceo de las penas hondas y del llorar frecuente.

«¿Qué te pasa, mujer?---dijo Ibero compadecido de veras.---¿Se te ha
muerto alguna persona querida? Es la primera vez que veo en ti un dolor
vivo, y esto, dejando a un lado nuestra discordia, no puede serme
indiferente. Acaba de suspirar y cuéntame\ldots»

---Soy muy desgraciada---fue lo único que respondió.---Si con esto no te
basta, peor para ti, pues poco más podré decirte.

---No creas que voy a mortificarte con interrogaciones, aunque el caso
de anoche las justificaría---dijo Ibero.---Pero algo tendrás que
decirme\ldots{} No; no te asustes antes de tiempo.

En aquel punto, juzgó Santiago que sería muy estratégico no atacar de
frente la cuestión que bien podría llamarse política. Para obtener claro
informe acerca de los visitantes de la casa misteriosa, convenía figurar
que esto no interesaba, desviando las indagaciones hacia otro objeto, y
suponiendo en este objeto convencional un interés que no existía.
Embistió, pues, por el lado de las liviandades y de los desvaríos
amorosos, hablando de Catalá, de su estado de furor, y de los accidentes
graves que podrían sobrevenir si Rafaela no ponía fin a sus locuras.

«¿Pero no habíamos quedado---dijo ella,---en que ya no éramos hermanos,
y en que no te importaba lo que yo hiciese o dejara de hacer? Son cosas
mías, Santiago, cosas malas si quieres, pero mías, y lo que es mío no es
de los demás.»

---Perfectamente; pero las cosas tuyas afectan a otras personas, a
muchas personas, Rafaela\ldots{} ¡Quién sabe si también a mí!

---¿A ti?

---Tus \emph{cosas}, como dices, van tomando tal carácter de gravedad,
que será difícil ya que tu padre deje de tener conocimiento de ellas. Tu
hermana misma, a quien yo vi tan dañada como lo estás tú, y que ha
contribuido a lanzarte por el mal camino, ya se asusta de su
complicidad\ldots{} Hasta los pequeños, Rafaela, hasta tus hermanitos
sienten o adivinan que hay en ti algo que no es honroso para la familia,
y van aprendiendo a pronunciar tu nombre con miedo, con vergüenza. ¿Esto
no te dice nada?

---Eso me dice algo, me dice mucho, Santiago---contestó Rafaela, la voz
cortada por la emoción;---y si te aseguro que ahora me encuentro
verdaderamente arrepentida, oirás una verdad como un templo\ldots{} y no
la creerás. Pues tienes que creerla, tienes que creerla.

---¡Si supieras, amiga mía---dijo Santiago dando un gran
suspiro,---cuánto me gusta creer en el arrepentimiento de las personas
que han hecho algún mal! Pero en este caso, para que yo vea clara tu
enmienda, es preciso que conozca el estado de tu ánimo, tus pensamientos
todos y los motivos de tu pena. Que la cosa es grave lo veo en el
desorden de tu vida, en tu cara demudada, en tu llanto, en este
encierro, en ese acento tuyo tan distinto de lo común en ti, que parece
otra la que habla. Para que tu carácter se me presente cambiado, lo
ocurrido en ti debe de haber sido, más que un suceso, una revolución.
Cuéntame esa revolución, y sólo el contármela te aliviará de tu pena.

Le oía Rafaela sin mirarle, inclinado el rostro sobre el pecho.

«Apostaría yo---dijo Ibero después de una pausa en que se cansó de
esperar respuesta,---a que el origen de tus desgracias no es otro que el
infame \emph{materialismo}. ¿Dices que no? ¿Por qué niegas con la
cabeza? ¿Te has quedado muda?»

---No es la ambición, Santiago; te digo que no es la ambición.

---En los días en que yo pude enterarme de lo que hacías, te vi
menospreciando la medianía decorosa por buscar la amistad de personas
que no tenían otros atractivos que su dinero.

---Una vez en el mal camino---dijo Rafaela con una sequedad que
contrastaba con su pena,---me parecía una simpleza perderme sin
gracia\ldots{} Para pobreza ya tenía la de la honradez\ldots{}
¡Perdición pobre\ldots!, es como ahogarse en un mar hediondo.

---La pobreza y las privaciones son cosa mala, es cierto\ldots{} Pero
podías haberte limitado a una situación media\ldots{}

---Me entró la locura de las cosas grandes, y no podía contenerme.
Quizás no comprendan esto los hombres que pueden satisfacer sus
vanidades de mil modos, con los títulos, con los galones, con la gloria,
qué sé yo. Nosotras no tenemos más que un medio de satisfacer el
orgullo\ldots{} Por eso yo decía: «Ya que no tengo nada de lo bueno,
Señor, tenga de lo malo lo más bonito.»

---Y tu hermana, que al principio te contuvo, viéndote después por el
camino de las conquistas lucrativas, te jaleaba para que siguieras.

---María Luisa es más ambiciosa que yo, y también más cobarde. Tiene,
para mantenerse honrada, motivos que yo no tengo: es casada de hecho y
madre de un niño. Para ella ha sido muy cómodo que yo peque. De este
modo resalta más su virtud, y como nos queremos y siempre hemos partido
lo que teníamos, no sale mal librada\ldots{} sin pecar, por supuesto.

El terrible juicio que en pocas y secas palabras hizo la dolorida de las
relaciones morales entre las dos hermanas, causó al Coronel verdadero
terror. Quedose un largo rato absorto, contemplando aquel cuadro
siniestro en que la virtud y la maldad comían en el mismo plato.

«De todo lo que me cuentas---dijo saliendo al fin de su
meditación,---resulta que tu hermana y tú no tenéis idea ninguna de la
verdadera decencia; resulta también que tú, Rafaela, eres una preciosa
muñeca que habla y ríe por mecanismos naturales; pero que no piensa ni
siente; no conoces la fe, ni el amor, ni ningún sentimiento grande. Si
algún mérito hay en ti es la sinceridad; pero esta virtud no compensa,
no, la falta de tantas otras. El día que nos separamos me dijiste que
eras incapaz de amar, que\ldots»

---Que no comprendía el amor, ya me acuerdo. ¿Y a eso llamas sinceridad?
Nunca he dicho una mentira tan atroz. Como nos despedíamos, y no te
había engañado nunca, quise echar sobre ti de una vez el engaño mayor
del mundo, para que te fueras con todos los sacramentos, bien
engañadito, sin entenderme ni tanto así, sin conocer a la mujer que
habías tenido, sin poseer de ella lo que más vale, que es el
corazón\ldots{} En esto que te digo ahora sí hay sinceridad, y lo
aseguro, aunque te duela.

---Si crees que esa franqueza tuya, tardía, me mortifica, estás muy
equivocada, Rafaela---dijo Santiago haciéndose el valiente.---Más te
quiero sincera y leal que engañosa\ldots{} por más que me lastime un
poco el no haberte conocido cuando debí conocerte, y el descubrirte
ahora, cuando la verdad de tu mentira, como dijo el otro, no debiera
importarme\ldots{}

Siguió a esto un largo silencio. Las aficiones policiacas contraídas en
una noche habían despertado en Ibero curiosidades impertinentes; no pudo
contener su avidez de examinar los objetos que le rodeaban, y
dirigiéndose a la cómoda miró dos o tres libros, un retrato de señora
colgado de la pared y un papel a medio escribir, que resultó apunte de
ropa, hecho por mano para él desconocida. Ni esto, ni los grabados de
periódico, adheridos con obleas a la pared blanca, eran materia
sospechosa en la que pudiera encontrarse relación con los preparativos
de un pronunciamiento militar.

«Lo que me has contado me hace el efecto de ver entrar la luz en un
cuarto obscuro. El cuarto obscuro eres tú, y el que a ciegas estuvo en
él, dándose trompicones contra las paredes, era yo\ldots{} Enciendes tú
la luz, y ahora te veo. Me alegro de conocerte. Resulta que la que yo
tuve por muñeca, linda figura rellena de serrín, es mujer, con todo su
relleno interior de sentimientos elevados\ldots{} Tienes corazón\ldots{}
¡vaya!, me alegro\ldots{} Sabes lo que es amor, eres capaz de
amar\ldots{} Digo que me alegro y te felicito. Ya veo claro que tu
desgracia viene de\ldots{} de eso, del amor. Y ahora, completa tu
confesión declarándome\ldots{} porque aquí encaja la cuestión magna,
Rafaela:\emph{¿Quién es él?\ldots{}} ¿Te cuesta confesarlo? Pues yo te
ayudaré, yo digo: «la que para mí y para todo el mundo ha sido muñeca,
mujer ha sido para uno solo, y este uno es el caballero de anoche.»

\hypertarget{xx}{%
\chapter{XX}\label{xx}}

«Para el caballero de anoche\ldots{} ¿Acierto? Respóndeme.»

---Es verdad lo que dices. No puedo negártelo.

---Pues ahora\ldots{} has de decirme quién es.

---Te cuento el milagro, el santo no.

---¿No me darás siquiera alguna explicación para que pueda yo formar
juicio\ldots? ¿Es pasión antigua?

---Sí.

---¿Anterior a tu casamiento?

---No: después\ldots{}

---¿Y es el único amor de tu vida?\ldots{} el único verdadero y
desinteresado, quiero decir.

---El único\ldots{}

---¿Por qué lo tenías tan oculto? ¿Cómo llegó a tanto tu disimulo de esa
pasión, que te formaste un carácter artificial para desorientar a
cuantos te conocíamos?

---Lo guardaba porque era verdadero, porque era lo único bueno que yo
conocía\ldots{} Lo tenía bien guardadito en mi sagrario, sin que nadie
lo viera, y a solas lo adoraba.

Daba Rafaela estas respuestas sin mirar a su confesor, inclinada hacia
adelante, con las manos ante la boca, soltando las palabras por entre
los dedos, como si estos fuesen la reja del confesionario.

«Muy bien---dijo Ibero, abrasado de curiosidad.---No me conformo, amiga
querida, con que cuentes el milagro sin nombrar el santo. Necesito
conocer a este; dime pronto su nombre.»

---Eso sí que no puede ser.

---No hay excusa. Si no me dices el nombre, la confesión no vale.

---La confesión vale sin el nombre. Ningún confesor pregunta nombres,
Santiago.

---Pues yo los pregunto, porque no soy un confesor como otro cualquiera;
soy un amigo.

---Los buenos amigos deben ser discretos.

---Dímelo, por Dios\ldots{} te lo suplico.

---Imposible\ldots{} no insistas.

---Pues necesito saberlo---dijo Ibero alzando la voz.---Es conveniente
que lo sepa. Rafaela, no me obligues a tratarte con dureza.

---Con amenazas conseguirás lo mismo que sin ellas, pues aunque yo viera
la muerte sobre mí, y aunque de contestar yo a tu pregunta dependiera mi
vida, respondería lo que has oído ya. No puedo decirte más.

Tenacidad tan formidable reveló la pobre mujer en esta declaración, que
Ibero retrocedió dolorido y algo colérico. No esperaba tal entereza; y
como a terco no le ganaba nadie, hizo mental juramento de no salir de
allí sin domar la fierecilla. No habiéndole resultado eficaz la
investigación directa, acordó emplear la parabólica, con rodeos y
hábiles artificios de palabra. «Ya que no me digas el nombre, dame al
menos alguna referencia de tus relaciones con ese sujeto, para que yo
conozca la extensión de tu desgracia y pueda aconsejarte los mejores
remedios. Quedamos en que le conociste después de casada\ldots{} ¿Fue
antes de separarte de tu marido?»

---Antes.

---Corriente\ldots{} Le conociste y te agradó\ldots{} Sin duda es
persona de superiores atractivos\ldots{} aunque también se dan casos de
que las mujeres se vuelvan locas por hombres vulgares y sin ninguna
gracia\ldots{} Bueno: quedamos en que le quisiste ciegamente. ¿Tuvo tu
hermana noticia de esta pasión?

---Sospechas, indicios\ldots{} siempre sin saber quién era la persona.

---Es, sin duda, persona de posición más alta que la tuya. Esto se ve
claramente y no puedes negarlo.

---No lo niego\ldots{} Es mucho más alta.

---Bien. En tus amoríos, de fijo hubo interrupciones, ausencias\ldots{}
A pesar de esto ¿era tu pasión durable, continua?

---Para mí como eterna, como lo que no puede tener cambio ni fin\ldots{}
Para él\ldots{} Pero muchas cosas quieres saber.

---«Para él no» ibas a decir\ldots{} Vamos, que le veías un día, otro
día\ldots{} pasaban semanas, meses quizá sin verle\ldots{} ¿Puedes
decirme si esto era antes o después del primer Ministerio Pérez de
Castro?

---No me hables a mí de ministerios. ¿Qué entiendo yo de política?

---Es para precisar fechas\ldots{} Otra cosa: ¿ese hombre tan amado por
ti te daba esperanzas de que tú llegarás junto a él a una posición más
regular, a una posición en que no tuvieras que avergonzarte de
quererle?\ldots{}

---Nunca me dio esas esperanzas.

---Luego eras para él un pasatiempo, un juguete para días, para horas
quizás, menos, mucho menos de lo que has sido para nosotros\ldots{}

---No sé\ldots---murmuró Rafaela, los ojos húmedos, mirando al techo.

---Ahora, compagíname esa pasión que has pintado como sublime, con la
otra pasión tuya de lujo. Yo, la verdad, no acierto a juntar en un solo
corazón, en un solo carácter, dos querencias tan distintas, la una tan
ideal y por lo fino, la otra tan baja.

---¿No entiendes eso?---dijo Rafaela mirándole como compadecida de su
ignorancia en punto a pasiones.---Pues yo gustaba del lujo, y me lo
procuré por todos los medios que se me venían a la mano. No pudiendo
subir a las alturas por la escalera natural, dejaba que los diablillos
me subieran volando. Yo quería subir\ldots{} Más fácil me era
verle\ldots{} a él\ldots{} arriba que abajo, y arriba podría de algún
modo atraerle, abajo no.

---Otra pregunta se me ocurre, y es delicada. Vas a darme la mejor
prueba de amistad, contestándola lealmente. Dime: en el tiempo mío, en
mi corto reinado, ¿veías y tratabas a ese hombre?

---No: te juro que no. No estaba en Madrid.

---¿En dónde estaba?

---Eso no te importa. Si hubiera estado aquí, ya ves si soy leal, te
habría\ldots{}

---Me habrías engañado\ldots{} dilo claro.

---Quizás no. Habría tenido el valor de decirte: «Santiago, no te
quiero, no puedo ser tuya.»

---Bien. En tiempo de Catalá y de \emph{Don Frenético} has tenido
frecuentes entrevistas con tu ídolo. Eso no lo negarás\ldots{} Bueno.
Lleguemos a lo que podríamos llamar historia contemporánea,
calentita\ldots{} En estos días, deseando retenerle, te determinaste a
salir de tu casa para gozar de alguna libertad. ¿Puedes decirme si le
veías siempre en la calle de San Hermenegildo?

---Allí nunca\ldots{} Fue una casualidad que nos vieras salir de aquella
casa.

---Ya, ya comprendo. Vuestro nido era este, este el asilo de amor. Pero
anoche supiste, no sé cómo\ldots{} eso ya me lo dirás algún día\ldots{}
supiste que los que se reunían en aquella casa corrían peligro de ser
descubiertos, y te faltó tiempo para llevar a tu amante el aviso de que
se pusiera en salvo.

---No\ldots{} no\ldots{} eso no es cierto---replicó Rafaela
desconcertada:---fue porque tenía que hablarle\ldots{}

---¿A qué iba tu hombre a esa casa?

---Yo no lo sé\ldots{} ni me importaba\ldots{} Nos veíamos allí\ldots{}

---Has dicho antes que allí no eran las citas de amor. Te contradices.
Si en todo lo anterior has dicho la verdad, ahora no la dices: te lo
conozco en la cara. Fuiste a dar el aviso, la voz de alarma\ldots{} Por
eso, a poco de entrar tú, salieron los mochuelos, uno a uno, o en
parejas\ldots{} ¡y que no llevaban el paso poco vivo! ¿Ves cómo sé la
verdad, aunque tú quieres ocultarla?

---No sabes la verdad: la supones, la inventas para desorientarme. Ya no
contesto a más preguntas. He confesado lo que debía confesar: lo demás
no te importa.

---Ya verás si me importa---dijo Ibero lanzándose al método capcioso
para buscar la luz.---Tampoco querrás revelarme los nombres de los que
estaban reunidos con tu ídolo. Yo los sé\ldots{} Aquel alto, que salió
con otro de regular estatura, era D. Leopoldo O'Donnell\ldots{}

---¡Pero si O'Donnell está de cuartel en Pamplona!

---¿Y tú cómo sabes eso?

---Lo sé\ldots{} no sé cómo.

---¿Niegas que uno de los que salieron era O'Donnell?

---Yo no niego ni afirmo; no sé.

---¿Niegas que el que salió solo, después de la pareja, era D. Manuel de
la Concha?

---¡Yo qué sé de Conchas ni conchos! Déjame en paz\ldots{}

---¿Y me negarás también que entre los conjurados estaba un capitán
indigno, llamado Vallabriga, pequeño, lívido, bilioso?\ldots{} Claro,
todo lo niegas\ldots{} no has visto nada\ldots{} Esta niña inocente pasa
junto a los volcanes sin enterarse\ldots{}

---Estás loco\ldots{} yo no entiendo una palabra de eso---dijo Rafaela
temblando de frío.---Me harás un gran favor dando por concluida mi
confesión. No puedo más.

---Mucho siento mortificarte, Rafaela; pero la confesión no está
concluida. Vuelvo a mi tema. Fáltame la clave de todo\ldots{} el nombre.

---He dicho que no.

---¡El nombre!\ldots{} Es necesario que yo lo sepa---dijo Ibero
golpeando el suelo con el pie.

---Si hasta el día del Juicio final estás preguntándomelo, por los
siglos de los siglos te responderé yo que no lo sé, que no me da la gana
de decírtelo.

La obstinación de Rafaela, absolutamente inexpugnable al parecer,
produjo en Santiago un arrebato de ira. Nunca la creyó capaz de guardar
un secreto, imitando a los héroes, defensores de plaza sitiada. Nuevas
intimaciones del Coronel dieron el mismo resultado. Ni había podido
escalar por sorpresa los muros, ni abrir brecha en ellos con furiosa
embestida.

«¡Mira que conmigo no se juega; mira que estoy decidido a no salir de
aquí sin tu respuesta!»

---Estate todo lo que gustes.

---Pues aquí me planto---dijo sentándose.---No lo tomes a broma. Primero
te cansarás tú que yo.

---Cansada estoy de oírte, puedes creerlo; pero no por eso me rendirás.
El callar es fácil\ldots{} Yo callo y tú alborotas.

---Te digo que conmigo no juegas---gritó Ibero poniéndose en pie con
súbito movimiento, y conminándola con reiteradas expresiones de amenaza,
airado, descompuesto, brutal.

---No te vale tu fiereza---dijo la Milagro con dignidad flemática,
envolviéndose en su manto, como un romano en su toga.---¿Qué es lo peor
que podrías hacerme? ¿Matarme? Pues a ello, Santiago. Aquí me tienes. No
chistaré, ¿Crees que muerta he de decirte lo que viva me callo? ¿O
piensas que amenazándome con puñal o pistola has de hacerme hablar no
queriendo yo? Pruébalo. ¿Traes pistolas?

---No juegues, te digo.

---Espero el tiro en completa tranquilidad. Apúntame a la sien\ldots{}
aquí. Ya ves. No me muevo\ldots{} ¿O es que no traes arma de fuego? Pues
ahí tienes la espada. ¿De qué te sirve ese chisme, si con él no me
atraviesas el corazón, en castigo de que no quiero responderte? Haz la
prueba, hombre\ldots{} Ya ves\ldots{} soy más valiente que tú.

La actitud de la Milagro, que sentía o afectaba una rigidez de voluntad
y un estoicismo a toda prueba, desconcertó a Ibero, sin aplacar su ira,
antes bien, encendiéndola más. En un tris estuvo que las amenazas
verbales se trocaran en bárbaras obras; pero el hombre supo echarse todo
el freno, que tal era su principal virtud, y espaciando su cólera con
pasos de tigre por la estancia, vinieron a resolverse sus furores en una
brutalidad pueril. Cogió una silla, y de un solo golpe contra el suelo
la hizo pedazos. Las astillas saltaron. El trozo de respaldo que le
quedó en la mano voló a estrellarse contra la pared.

«¡Qué culpa tendría la pobre silla!» exclamó Rafaela.

---Alguna tuvo\ldots{} En ella se sentaría ese hombre---dijo Ibero casi
sin aliento, poniendo en su voz un matiz de humorismo lúgubre.

Siguió una pausa larguísima: en el espacio de ella sonó un reloj en la
vecindad; después otro más lejano. ¿Qué hora era? Ninguno de los dos lo
sabía; ninguno se cuidaba de apreciar la marcha del tiempo. Pero debía
de ser muy tarde, porque el velón parecía próximo al total consumo de su
aceite. Ibero se sentó al fin, diciendo, ya con voz más reposada:
«Quedamos en que de aquí no me muevo hasta que hables. Mestizo de las
razas de Aragón y Álava, soy más terco que la terquedad.»

Se acomodó en un sillón, poniendo delante una silla para estirar las
piernas. Y ella, entapujándose más y cerrando los ojos: «Eres dueño de
estarte aquí todo el tiempo que quieras: así se verá quién es más terco.
Nos dormiremos, tú con tu curiosidad, yo con mi angustia.»

Transcurrió otro largo espacio de tiempo, en cuya longitud bostezante
sonaron los relojes, dando un número de campanadas que ninguno de los
dos se cuidó de contar. De improviso, y como si continuara una tranquila
conversación suspendida por la pereza, Ibero preguntó a su amiga:

«¿Y ese hombre es casado? ¿Tampoco esto querrás decírmelo?»

---Es soltero; pero como tú, vive prendado de una señora ideal, de una
Dulcinea; a esa mujer, más que verdadera para él, soñada, consagra su
alma toda\ldots{} Le pasa lo que a ti, que la dama está muy alta, y no
podrá, no podrá llegar a ella\ldots{}

---La altura de la mía no es tanta\ldots{} ¿Por qué no he de llegar?

---Pues él no llegará, no llegará.

---¡Enamorado de otra!---dijo Santiago compasivo y triste.---Y a ti que
tanto le quieres, que sólo por él tienes alma y corazón; a ti, Rafaela,
que para él vives, te trata como a una mujer a quien se encuentra en la
calle, y en la calle se deja\ldots{} ¿No es esto?

---Eso, o poco menos es.

---¡Dime su nombre, y te juro\ldots! Vamos\ldots{} no me conoces, no
sabes de lo que es capaz Santiago Ibero\ldots{} te juro que le persigo,
le cazo, y te le traigo amarradito de pies y manos. Voy viendo que es un
miserable ese hombre\ldots{} Merece una lección dura.

---Pero no podrás tú dársela ni hay para qué. Mi destino es el
sufrimiento, la muerte, y nadie me salva\ldots{} Todo por querer a un
hombre\ldots{} Naturalmente, ha visto en mí una mujer extraviada\ldots{}
¿Y cómo podría yo convencerle de que tal vez no lo sería si él me
quisiera?

---Esas cosas no caben dentro del convencimiento. Lo que tú dices, es el
sino\ldots{} Tu desgracia no tiene remedio. Pídele a Dios que te dé el
olvido.

---Lo pido; pero ya verás cómo no me lo da. Le querré siempre, y ahora
más, ahora más.

---Explícame una cosa. ¿Anoche, disputabais sobre si os separaríais o
no?

---Cierto: él decía que no era conveniente que nos viéramos más; yo que
no puedo vivir sin verle. Las razones que él daba no puedo decírtelas.
Fue una lucha tremenda\ldots{} y en medio de la calle\ldots{} Él no
quería más que alejarse\ldots{} alejarse\ldots{} y yo correr tras él,
trincarle fuerte y no soltarle más. Por fin, hizo lo que quería. Yo me
vine aquí desolada, el corazón partido en no sé cuántos pedazos. Pasé
una noche horrible, y esta mañana ¡ay de mí!, recibí una carta
suya\ldots{}

---En que te daba la despedida\ldots{}

---¡Para la eternidad!\ldots---dijo Rafaela, rompiendo en un llanto
desgarrador.---Se despedía\ldots{} ya no nos veremos más\ldots{} Su
esfera y mi esfera son tan distintas, que no caben más
aproximaciones\ldots{} Así lo escribía\ldots{} Me recomendaba la calma,
la formalidad, y buscar en otro amor más ajustado a mi esfera\ldots{}
la\ldots{} no sé qué\ldots{} Me mató con esta carta\ldots{} ¡y adiós
para siempre! ¡Qué ingrato!

---¡Qué infame, dirás, qué monstruo de egoísmo!\ldots{} Rafaela, dame la
carta.

---La he roto---respondió la infeliz, anegada en llanto.

---Se podrá leer recogiendo los pedazos.

---Los he quemado.

---¿Las cenizas\ldots?

\hypertarget{xxi}{%
\chapter{XXI}\label{xxi}}

El amarguísimo llorar de Rafaela, inútilmente combatido por las palabras
consoladoras del buen Ibero, vino a parar en una congoja o espasmo con
imponente anarquía de nervios, gritos de dolor, convulsiones, tentativas
de arrancarse mechones de su espléndida cabellera. El Coronel y los
dueños de la casa se confundieron en el auxilio de la dolorida,
prodigándole cuantos cuidados eran del caso; pero entre tantos médicos
que aplicaban, ya remedios comunes, ya las exhortaciones cariñosas, sólo
el tiempo obtuvo resultado feliz. Al rayar el día, desgastada la energía
nerviosa de la pobre mujer, acostáronla, y pena y trastorno entraron en
la natural sedación. «Ahora te duermes---le dijo Ibero al
despedirse,---y mañana, más tranquilos tú y yo, te diré lo que pienso.
No te asustes: ya no te haré preguntas. Nada quiero saber; me doy por
vencido, y levanto resueltamente el sitio que te puse\ldots{} Veremos si
aplaco a Manuel, y una vez reducido a la conformidad, lo que no creo
difícil si tú me ayudas un poquito, te llevaré a tu casa\ldots{} Adiós,
hija, que duermas.»

Se fue el hombre, rendido del largo asedio, no satisfecho de sí mismo,
pues habría sido más caballeroso que desde las primeras declaraciones de
ella respetase su silencio. Aún no sabía si Rafaela, después de la
incompleta confesión, se había empequeñecido o agrandado a sus ojos.
Sintió un estupor extraño ante la imagen de ella, que apartar no podía
de su mente: era Rafaela otra persona; la \emph{Perita en dulce}
perdíase en las brumas del pasado, como el recuerdo de personas muertas;
en su lugar otra mujer aparecía.

Los quehaceres de aquella semana no distraían a Ibero de su cavilación
tenaz. Rafaela era otra; Rafaela no era tal y como él la había visto,
víctima de una equivocación, de un error de los sentidos y del
entendimiento. ¿Valía más o valía menos después de manifiesta en su
genuino ser? A la resolución de este acertijo consagraba el Coronel sus
horas, y si fatigado del mental devaneo lo arrojaba de su mente, pronto
se le introducía en la bóveda cerebral con sutileza de ladrones,
agregándose a otras ideas de muy distinta calidad, a ideas políticas, a
ideas del servicio militar. Véase por qué no puso sus cinco sentidos,
como parecía natural, en la elección de Regente, suceso memorable que
debía despertar en él entusiasmo vivísimo por haber prevalecido la
Regencia única, recayendo el voto parlamentario en el salvador y
pacificador de las Españas, D. Baldomero Espartero. El día en que acudió
a felicitarle con la oficialidad de su regimiento, encontró Santiago al
señor Duque menos satisfecho de lo que creía, sin duda porque abrumaba
su conciencia el peso de la responsabilidad que la Nación había echado
sobre sus hombros. No encontró tampoco en Doña Jacinta ninguna señal de
vanagloria, y oyó de sus labios esta frase donosa: «¡Ay, Santiago, más
quiero reinar en la Fombera, en medio de un pueblo de patos y gallinas,
que \emph{regentar} en España! Este corral no es para nosotros.»

Cuando esto pasaba, ya el Coronel había dado nuevo testimonio de su
inaudita bondad a las desdichadas hijas de Milagro, pues no sólo
consiguió arrancar de la mente de Catalá, con un trasteo ingenioso, las
ideas trágicas que hacían temer mayores escándalos, sino que condujo a
Rafaela a su casa y la devolvió al cariño de sus hermanas y hermanitos,
inventando todas las historias necesarias para cohonestar la ausencia.
Fuera que su sino adverso no se hartaba de perseguirla, fuera que el
Señor quisiera imponerle el castigo que merecían sus culpas, ello es que
la pobre mujer no pudo gozar de la tranquilidad que su casa, tras las
pasadas tormentas, le ofrecía, porque a los pocos días de entrar en
ella, cayó con una insidiosa enfermedad que hubo de agravarse
inesperadamente, degenerando en tabardillo. Altísima fiebre, delirio,
pérdida de toda energía fueron los síntomas predominantes, y pasaban
días y semanas con alternativas de mejoría y retroceso, sin que a la
postre pudieran la familia y el médico esperar una solución que no fuese
la irremediable. Ibero no dejaba pasar día sin ir a informarse, y en los
de peligro acudía dos y hasta tres veces, traspasado de compasión cuando
las noticias eran tristes, y alegrándose si observaba en las caras de
Cavallieri o de María Luisa señales de esperanza. En ningún caso
pretendía verla, temeroso de que su presencia despertara en la paciente
recuerdos desagradables. María Luisa le contaba todo: el número de
cucharaditas de medicina que había tomado, las tazas de caldo, las
personas por quienes preguntaba.

Se administraron a Rafaela los Santos Sacramentos en primeros de Mayo,
siendo la confesión larga y compungida, y en el acto del Viático edificó
a todos por su piedad. A la semana siguiente sobrevino un estado que
calificaron de mejoría por la desaparición de la fiebre; pero su
debilidad era tan extremada, que se le trastornó el sentido. «Anoche y
esta mañana---dijo María Luisa al Coronel, que ni un solo día desmintió
su puntualidad,---nos ha dado una sesión de política. ¡Cómo tiene la
cabeza la pobre! Dice que vamos a tener otra revolución; que se
sublevarán las tropas para quitar a Espartero la Regencia que ha robado,
y dársela otra vez a \emph{la patrona de los moderados}, \emph{Doña
María Muñoz}\ldots{} ¡Qué risa! Lo cuenta todo como si lo viera. Dice
que O'Donnell y otros militares que andan por París lo están fraguando,
y que muchos que aquí pasan por fieles están metidos en el ajo.»

Ninguna observación hizo Ibero sobre estos delirios, y a los pocos días,
cuando se decidió a penetrar en la alcoba, apenas cambió con Rafaela las
expresiones comunes en visitas a enfermo. Grande era la demacración de
la joven, tristísima la máscara que el estrago morboso había puesto en
su dulce fisonomía. Los ojos se comían toda la cara, según la expresión
de María Luisa. Nunca había visto Ibero retrato más vivo de la
Magdalena, por su expresión de espiritualidad y de sentimiento
intensísimo. El cabello espléndido aumentaba la semejanza; sólo faltaban
una calavera y una cruz tosca, para que fuese perfecta. Después de
recomendarle que se alimentara poquito a poco, para recobrar la salud y
el vigor, salió el hombre de la visita más triste que había entrado, y
la imagen de la convaleciente ejerció una tenaz persecución sobre su
espíritu. Llegó en aquellos días a sentirse contagiado del
enflaquecimiento de su amiga: también él se contaba los huesos; también
era mucho espíritu y escasa materia, y tomaba el cariz de un escuálido
penitente del yermo; también perdía el apetito, y sentía un ardentísimo
amor de la meditación y la soledad.

Entre las innumerables cosas raras que le pasaron al Coronel Ibero en la
primavera y verano del 41, se mencionarán algunas que no parecen
indignas de la historia. Apenas se dio cuenta de que había disminuido el
interés ansioso que comúnmente le inspiraban las noticias y
correspondencia de La Guardia. Fue, en verdad, estupendo que faltasen un
mes las cartas sin que él lo advirtiese ni de ello se inquietara; y
cuando el correo las trajo, no se alegró todo lo que debía leyéndolas,
ni saboreó con la fruición de otras veces su lisonjero contenido. Otro
singular caso de los que le turbaron entonces fue que una tarde, casi
una noche, pues anochecía, vio en la Puerta del Sol a un caballero que
le recordó al misterioso acompañante de Rafaela en la calle de San
Hermenegildo. ¿Era o no era? Su primera impresión fue la de una perfecta
conformidad del rostro de aquel sujeto con la imagen que en rápido
instante vio la noche de marras. Después dudó\ldots{} Salía del
Principal el señor aquel con tres personas, una de las cuales era
conocida como de las más afectas a la situación; los otros dos pasaban
por moderados rabiosos. Violes Ibero perderse entre el gentío, y se
retiró tratando de cotejar en su mente facciones con facciones: las del
sujeto que acababa de ver y las de otro personaje de historia que
conoció en cierta casa donde por sorpresa le metieron una noche amigos
oficiosos. No le fue difícil establecer la concordancia entre el sujeto
que a la sazón veía y el del Postigo de San Martín; mas la de éste con
\emph{el caballero de Rafaela}, no le salía. Tan pronto le parecía el
uno más alto, tan pronto más bajo el otro, y el aire, color y andares no
eran los mismos. No hubieran quizás embargado su ánimo estos cotejos en
otras circunstancias; pero en aquellas no podía librarse, por extraña
rutina de su mente, de consagrarles más atención de la que sin duda
merecían.

La persona con quien más intimó en aquel tiempo fue su paisano Bretón,
que desde el tropiezo de \emph{La Ponchada}, pieza infeliz en que
ridiculizó a la Milicia, venía corriendo un temporal duro, agravado por
la cesantía. El infeliz poeta, desamparado de la administración, no tuvo
más remedio que tirar de pluma, y su fecundidad fue en aquel año
progresista más prodigiosa que nunca. En el Liceo y en el Príncipe
estrenó varias obras con vario éxito, ocultando a veces su nombre,
temeroso de los milicianos, que, por atreverse con todo, también
faltaban al respeto a las musas. Ibero tomaba la causa de Bretón como
propia, llevando al teatro en las noches de estreno una imponente
\emph{alabarda}, compuesta de los amigos de \emph{Saboya} y de toda la
gente decidida que podía reunir. Pero así como el más tranquilo refugio
de Bretón fue por entonces el Liceo, mansión neutral, apacible templo de
la poesía y las artes, también Ibero buscó en aquel oasis la frescura y
descanso que su alma necesitaba. Allí se encontraba una noche, oyendo
recitación de versos, cantorrio de cavatinas y salmodia de discursos,
cuando fue a buscarle con prisa el ayudante de su regimiento de parte
del Brigadier Linaje. «Creo, mi Coronel---le dijo al salir,---que nos
mandan a Vitoria.»

No tardó en confirmar el secretario del Regente la disposición que a
satisfacer venía, después de tanto tiempo, los deseos del caballero
alavés. Al fin los hados benignos, y en su nombre el señor Duque de la
Victoria, levantaban el destierro de un corazón amante para que corriese
al lado de su ídolo, poniendo fin a una separación que era la mayor
crueldad de los tiempos pretéritos y futuros. Pero como en aquel periodo
de la vida de Ibero venían a producirse todos los sucesos con un
contrasentido que era burla de la lógica y juguete de la razón, resultó
que la noticia de su traslado, en vez de inundar de resplandores el alma
del Coronel, condensó sobre ella nubes, dudas, tristezas\ldots{} No
sabía lo que era aquello, porque si su voluntad, por el movimiento
adquirido, persistía en querer lanzarse al Norte, al propio tiempo el
alma toda se le inmovilizaba en una inercia estúpida, contra la cual
poco podían voluntad, antiguos deseos y compromisos.

Como era forzosa la obediencia, no había que pensar ni en discutir la
orden. De labios de Linaje oyó confirmada la disposición; mas no era
Vitoria el punto de su destino, sino Pamplona, a las órdenes del Capitán
General Ribero: probablemente se le daría un mando en la brigada de su
antiguo jefe y maestro, Zurbano. Esto le desconcertó, pues había
presumido que al frente de \emph{Saboya} iría. No, no: \emph{Saboya}
quedaba en Madrid, y él iba sin mando, con otros dos jefes, Seisdedos y
Clavería, criados también a los pechos de D. Martín. «Se conoce---pensó
Ibero,---que del Norte vienen soplos de frío, y hay que templar aquel
ejército con soldados de los que echan lumbre. Pues, Señor, al Norte, a
la guerra. Olor de sangre me da en la nariz. Venga bendita de Dios, si
ha de ser para bien mío y de la libertad.»

Habiéndole marcado un plazo improrrogable para la salida, no se despidió
más que de los que habían sido sus subalternos, de los amigos Bretón y
Espronceda, y de la familia de Milagro, a la cual consagró todo el
tiempo de que en su último día de Madrid pudo disponer. María Luisa
lloraba su partida como la mayor desgracia que sobre la casa podía
recaer, y Cavallieri no hacía más que suspirar con grave diapasón de
bajo profundo. Rafaela, ya levantada, mas sin poder moverse de un sillón
y recobrándose muy lentamente de su debilidad, sostuvo con él un corto
diálogo en que le aconsejó precaverse contra las balas. Cuando al Norte
se le mandaba con tanta prisa, era que teníamos guerra en puerta. Que
esta sería implacable, cruelísima, sanguinaria, ella lo sabía por
seguros indicios, por sueños terroríficos y pesadillas espantosas que
aquellas noches la atormentaban. Llevado de una secreta inclinación a
pensar y sentir como Rafaela, apoyó Ibero los vaticinios lúgubres de su
amiga. También él había tenido sueños horribles, y veía los ríos del
Norte enrojecidos por española sangre. Si Dios así lo permitía y quizás
lo ordenaba, ¿qué remedio había más que cumplir la soberana voluntad?
Diéronse las manos, apretándoselas fuertemente durante un tiempo, que no
se sabe cuánto tiempo sería, y Rafaela le miró de un modo singular, con
piedad y dulzura inefables, o al menos, él así lo veía. Tal impresión le
hizo aquel mirar, que creyó llevarse los ojos de ella dentro de los
suyos\ldots{} Y pronto hubo de ver que ni cuando él dormía, dormían los
ojos intrusos\ldots{} siempre alerta, siempre cebándose en un mirar
continuo, eterno.

\hypertarget{xxii}{%
\chapter{XXII}\label{xxii}}

Como se le señaló la ruta de Soria y Alfaro, no había que contar por el
momento con una escapadita a La Guardia. Divertido habría sido para
Ibero el viaje si el hombre se encontrara en mejor disposición de ánimo,
porque sus compañeros Clavería y Seisdedos eran los caracteres más
abonados para la vida de bromas y regocijo: de un viaje molesto y con
mil peripecias fastidiosas hacían un divertido Carnaval. La caminata por
pueblos alcarreños y sorianos fue una continuada serie de escenas
cómicas, incluyendo en este género, no sólo los encuentros felices, los
galanteos y comilonas, sino también los peligros, retrasos, vuelcos y
fatigas. Pasaron el Ebro por Alfaro, y ansiosos del descanso que sus
molidos cuerpos necesitaban, siguieron hasta la nobilísima ciudad de
Olite, asentada en un llano fértil. En la corta guarnición encontraron
no pocos amigos, entre ellos Baldomero Galán, Gobernador militar de la
plaza, que se tuvo por dichoso de obsequiar al Coronel Ibero
aposentándole en su casa, donde tanto él como su digna esposa, Doña
Salomé \emph{de} Ulibarri, se desvivieron por hacerle grata la
existencia en el tiempo que durara su descanso. Regalada era allí la
vida por la abundancia y variedad de comestibles de la tierra, y por el
esmero y arte que en el condimento de almuerzos, comidas y cenas ponía
la señora de Galán, la cual, entre paréntesis, era una mujer guapísima,
de ojos negros, deslumbrantes, de aire desenvuelto y franco, el habla
graciosa con golpes de baturrismo.

Muy bien lo pasó D. Santiago en tal compañía. Llevole Galán a visitar
las hermosas ruinas del castillo, predilecta morada de los Reyes de
Navarra en siglos remotos, y estando el Coronel con su patrón y amigo
embebecido en la admiración de los soberbios baluartes corroídos por el
tiempo, de los gallardos torreones festoneados por lozanas hierbas que
en las grietas crecían, vio salir por entre los muros del despedazado
monumento a un hombre, cuyo rostro le causó singularísima impresión de
estupor y miedo. Era rostro conocido; no se le despintaba. Mirándole
atentamente, advirtió su condición de caballero, que el traje no
desmentía; tras él, saltando también por entre los sillares arrumbados,
iba otro sujeto, que saludó cortésmente a Galán al paso. Apenas les vio
desaparecer entre las ruinas, pidió Ibero informes acerca de ellos a su
comilitón, y éste le satisfizo con lo que sabía: «El que me ha saludado
es D. Francisco Tiemblo, vecino de Olite, persona, según dicen, de mucha
sabiduría en achaque de historias, el que aquí más entiende de lo que
fueron y significan estas piedras antiguas. De memoria le refiere a
usted todos los letreros, y le explica las figuras que vemos en las
iglesias de San Pedro y Santa María y en el convento de Claustrales,
madriguera frailuna que fue. El señor que va con él es de Madrid, según
creo, y aficionado también a estas quisicosas de la caballería andante.
Por lo que le oí no hace muchas tardes, paseándonos aquí con varias
personas del pueblo, pienso que es poeta, de estos que lo tienen por
oficio, pues a cada triquitraque soltaba un verso y se pasmaba delante
de las ruinas, que visita de día y de noche. No le conozco, ni D.
Francisco me ha dicho su nombre, o porque no lo sabe o porque no quiere
decirlo.»

---Y ese señor arqueólogo ¿qué opiniones tiene? ¿Es liberal?

---¡Anda, anda\ldots{} si es más moderado que Judas!

No hablaron más del asunto, y se fueron a comer. Ibero pasó una tarde
tristísima, negándose a toda distracción y a los pasatiempos de billar,
damas o ajedrez que Galán le propuso. Creyendo la patrona que le había
hecho daño la copiosa comida, preparábale infusiones de diferentes
hierbas, que el Coronel no quiso tomar. Pasó la noche luchando con el
insomnio, perseguido por la imagen de Rafaela, que no le dejaba vivir,
le mordía el pensamiento (así como suena), y armaba un terrible
desconcierto revolucionario en su desquiciada voluntad. Procuraba
desecharla; pero ella, la pertinaz imagen de Magdalena penitente y
cabelluda, hacía que se iba, y tornaba más luminosa, más bella. Lo peor
del caso era que a ratos gozaba tanto en contemplarla, que no cambiara
aquel goce por ningún otro del mundo. Hacía propósito de imponerse el
correctivo de huir de la Milagro para siempre, de no volver a estar
donde ella viviese; pero dudaba que pudiera cumplirlo. Grande era la
atracción del abismo: ¿se arrojaría en él, o emplearía el tiempo
mirándolo desde la boca, para que con la continuada vista de la hondura
se le pasaran las ganas de arrojarse?

Dispuso la partida hacia Pamplona para la mañanita del tercer día, y la
noche precedente, después de cenar, le dijo Galán con misterio: «Ha
venido a verme el amigo Tiemblo con la incumbencia de que el señor aquel
de las ruinas, el poeta, desea hablar un ratito con usted, mi Coronel.
Me pregunta si debe venir aquí, o si prefiere usted ir a su casa, que es
la posada de Fadrique, la mejor del pueblo. Me temo que éste trae la
mala idea de leerle a usted versos, pues yo sé que a otros los ha leído,
y en verdad que no han quedado satisfechos, por ser muy melancólico lo
que el tal discurre. Para mí que está algo tocado. ¿Qué contesto?»

---Que iré a su posada mañana temprano---replicó Ibero con propósito de
hacer lo que decía; mas al amanecer, después de otra noche de cruelísimo
desvelo, sintió desgana horrible de acudir a la cita. El personaje
incógnito, fuera o no quien sospechaba, le infundía miedo, un terror
instintivo, primario, y no porque de él temiese ningún daño material:
temía que su presencia, su habla, despertasen un tumulto de pasiones,
quizás sentimientos contrarios, el odio, la admiración, el cariño, la
envidia\ldots{}

No, no, no. Mejor era que partiese sin verle. ¿A santo de qué venía tal
entrevista? No mil veces: él se largaba por su camino, y quisiera Dios
que en ningún punto se encontrara con el caballero hermoso y triste.
Firme en su propósito, partió con sus compañeros, y el otro, que desde
muy temprano le esperaba, saliendo al balcón de su aposento siempre que
sentía pasos en la angosta calle, se descorazonó cuando ya muy avanzada
la mañana le dijeron que el señor Coronel Ibero, con los comandantes
Seisdedos y Clavería, llevaba una hora de camino en dirección de
Tafalla.

«Nos ha dado un buen quiebro---dijo al melancólico, sin ánimo de
consolarle por aquel contratiempo, un individuo que desde la tarde
anterior le acompañaba y al nombre de Gallo respondía.---No lo esperaba
de un chico tan atento\ldots{} Por supuesto, como nada habíamos de
sacar, más que nosotros pierde él, perdiendo su opinión de persona fina.
Y pues este pueblo ruin ha dado ya de sí todo lo que podía dar, vámonos
con viento fresco a Estella, de donde bajaremos a Viana, para seguir
luego a La Guardia\ldots{}

---Vámonos---repitió el otro suspirando, sin poder desechar el enojo del
desaire sufrido,---y en otra parte seremos más afortunados, aunque voy
viendo que no se encuentran caballeros a la vuelta de cada esquina. El
siglo los va descastando, y llegará día en que no se halle uno para un
remedio. Vámonos.

En un cochecillo derrengado partieron antes de mediodía hacia Tafalla, y
sin entrar en esta ciudad siguieron a Estella por Larraga y Oteiza, con
calor sofocante, respirando un aire seco y polvoroso. A media tarde
comenzó a cubrirse el cielo de nubes pardas, que avanzaban del Oeste, y
con ellas de la misma parte venía un mugido sordo, intercadente, como si
por minutos se desgajaran los montes lejanos y rodando cayeran sobre la
llanura. No era floja tempestad la que se echaba encima. Para zafarse de
ella, apalearon los viajeros al infeliz caballejo que tiraba del coche;
mas no obtuvieron la velocidad que deseaban. Descargó la primera nube
antes que llegasen a Oteiza. El iracundo viento quería revolver los
cielos con la tierra, y durante un rato el polvo y la lluvia se
enzarzaron en terrible combate, como furiosos perros que ruedan
mordiéndose. Los giros del polvo querían enganchar la nube, y esta
flagelaba el suelo con un azote de agua en toda la extensión que
abrazaba la vista. El polvo sucumbía hecho fango, y retemblaba el suelo
al golpe del inmensísimo caer de gotas primero, de granizo después. Los
campos trocáronse un instante en lagunas; retemblaba el caserío de las
aldeas como si quisiera deshacerse, y los relámpagos envolvían
instantáneamente en lívida claridad la catarata gigantesca. Grandiosa
música de esta batalla era el continuo retumbar de los truenos, que
clamaban repitiendo por todo el cielo sus propias voces o conminaciones
terroríficas, y cada palabra que soltaban era llevada por los vientos
del llano al monte y del monte al llano. Como al propio tiempo caía el
sol en el horizonte, y la luz se recogía tras él temerosa, iban quedando
obscuros cielo y tierra, y la tempestad se volvía negra, más imponente,
más espantable. En la confusión de ella se perdieron, como la hoja seca
en medio del torbellino, los cuitados viajeros que a media mañana habían
salido de Olite en un mezquino carricoche. Se les vio luchar contra los
elementos desencadenados, avanzar por en medio de la espesa lluvia y del
desatado viento, queriendo achicarse y escabullirse; pero tal navegación
era imposible, y en la revuelta inmensidad desaparecieron bien pronto el
carro y caballo y caballeros.

Para encontrar nuevamente a los que aquel día desafiaron a la irritada
Naturaleza, hay que dejar pasar días, meses, y no habrá que rebuscar
media España para dar con ellos, pues reaparecen a cara descubierta y a
plena luz en la por tantos títulos ilustre ciudad de Vitoria, cabeza del
territorio alavés. Álava, con Navarra, Guipúzcoa y Vizcaya, es la tierra
que podríamos llamar del martirio español, el fúnebre anfiteatro de sus
luchas de fieras, y el redondel en que se han despedazado los
gladiadores, por el gusto de las peleas y la embriaguez de la sangre.
Allí las cañas han sido siempre espadas, los corazones hornos de coraje,
la fraternidad emulación, y las vidas muertes. Allí las generaciones han
jugado a la guerra civil, movidas de ideales vanos, y se han desgarrado
las carnes y se han partido los huesos, no menos ilusos que los niños
jugando a la tropa con gorros de papel y bayonetas de junco. Pues allí,
en una de las cabeceras del territorio éuskaro, que los liberales no
entregaron jamás a la facción, aparece el melancólico galán de la causa
de María Cristina, levantando bandera negra contra el Regente, a quien
declara usurpador, y haciendo tabla rasa de toda ley y estado
posteriores a la renuncia de la Gobernadora en Octubre del año anterior.
Ya tenemos en campaña otra guerra fratricida, en nombre de principios
más o menos claros, invocando el sagrado lema de la defensa de la débil
mujer contra el varón fuerte, de los derechos de la sangre contra los
artificios de la soberanía nacional.

D. Manuel Montes de Oca, el más ardiente paladín de la Regencia de
Cristina, el que la proclamó condensando en una idea política el
sentimiento poético y la caballeresca devoción de su alma soñadora,
noble en su delirio, grande en su loco intento, al propio tiempo
insensato y sublime, gigantesco y pueril, aparece en Vitoria al frente
de un artificio de Gobierno, con poderes reales o figurados del soberano
ausente. Sin pararse en barras, contando con la insurrección de
generales en Zaragoza y Pamplona, sostenido en Vitoria por la guarnición
que se subleva al mando del Comandante General Piquero, entra en
funciones como Presidente de la Junta Suprema de Gobierno,
\emph{mientras llegaba Doña María Cristina}, con una resonante proclama
en que dice:

\emph{La Nación no reconoce, vosotros} (a los nobles vascongados y
navarros se dirigía)\emph{no podéis reconocer como válida y legítima la
renuncia del Gobierno de la Monarquía hecha por Su Majestad en Valencia,
porque fue, y así lo ha declarado Su Majestad, un acto insolente de
fuerza\ldots{}}

\emph{Doña María Cristina es la única Regente y Gobernadora del Reino;
la única tutora de las ilustres huérfanas llamadas a regir los destinos
de esta Nación tan rica de gloria como escasa de ventura. Esta es la
bandera de los leales; esa bandera se levanta hoy en todos los ámbitos
de la Monarquía española\ldots{} Los generales más ilustres, los
militares valientes, los que ganaron en campos de batalla honrosas
cicatrices, los que nunca faltaron a la fidelidad ni nunca cometieron el
crimen de perjurio, siguen esa bandera magnífica y radiante que conduce
a la victoria. Ella es el símbolo de nuestra santa Religión y de nuestra
católica Monarquía\ldots{} Con ella triunfaremos nosotros como
triunfaron nuestros padres}.

\hypertarget{xxiii}{%
\chapter{XXIII}\label{xxiii}}

Armado de nuevo el sangriento juego nacional, los desgarrados pendones,
un tanto sucios ya del largo uso sin la renovación conveniente, se
vieron otra vez en alto añadiendo a sus lemas el de la sacratísima
Religión. Para mayor gloria de esta, se levantaban en armas cuatro
caballeros, hijos de la política los unos, del ejército los otros, y por
dar mayor fuerza a su audaz aventura, agregaban a su bandera el
programita de restablecimiento de fueros, cebo magnífico para llevarse
consigo a toda la población éuskara, pisoteando el Convenio de Vergara.
Bien, bien. ¡Qué delicioso país, y qué historia tan divertida la que
aquella edad a las plumas de las venideras ofrecía! Toda ella podría
escribirse con el mismo cuajarón de sangre por tinta, y con la misma
astilla de las rotas lanzas. El drama comenzaba a perder su interés, por
la repetición de los mismos lances y escenas. Las tiradas de prosa
poética, y el amaneramiento trágico ya no hacía temblar a nadie; el
abuso de las aventuras heroicas llevaba rápidamente al país a una
degeneración epiléptica, y lo que antes creíamos sacrificio por los
ideales, no era más que instinto de suicidio y monomanía de la muerte.

Los primeros días del alzamiento fueron risueños, días de esperanzas y
de ciego optimismo. Vista la insurrección desde Vitoria, que parecía ser
su centro y atalaya, la idea sediciosa prendía en todo el territorio
vasconavarro como el incendio en la seca mies. A la voz de Montes de
Oca, que lanzaba a los pueblos endechas rimbombantes, responde Bilbao,
sublevándose también con su Diputación al frente, y parte de la Milicia
Nacional. Montes de Oca tira de pluma y devuelve a la invicta villa en
un decreto el derecho de Bandera y otros privilegios abolidos; en
Miranda toma partido por Cristina el Provincial de Burgos, que a Vitoria
se dirige para dar su apoyo al movimiento; Portugalete y Orduña se
pronuncian también; el cura de Dallo y el escribano Muñagorri reúnen al
instante sus partidas y se lanzan por collados y montes a matar
liberales. En tanto daba mayor vuelo a la insurrección el General D.
Leopoldo O'Donnell, que había ganado el regimiento de Extremadura y un
escuadrón de Caballería, y con ellos proclamó la bandera de Cristina y
Fueros en la ciudadela de Pamplona. En Zaragoza, Borso di Carminati
echaba mano al segundo regimiento de la Guardia Real, y salía con él
para llevárselo a O'Donnell. Toda esta fuerza, con el batallón y los
escuadrones que Piquero había sublevado en Vitoria, eran una base
admirable de insurrección. Ya vendrían luego más pronunciamientos de
tropas donde menos se pensara, que bien se había trabajado en la
seducción de jefes. Todo era empezar: los primeros que se lanzaron daban
la mejor prueba de iniciativa heroica, de que luego tomarían ejemplo los
reacios y pudibundos. Pero las más risueñas esperanzas de los
aventureros de Vitoria estaban en Madrid, donde levantarían la propia
bandera media docena de adalides militares, los más ilustres de nuestro
ejército, la flor de los héroes de la última guerra. En cada correo
creían recibir el notición de que la Regencia elegida por las Cortes era
un cadáver, y de que sobre él se alzaba ya la soberanía incuestionable
de la Reina Gobernadora, devuelta al amor de España.

En su residencia oficial de la Diputación trabajaba D. Manuel Montes de
Oca sin dar paz a su mente ni a la pluma, despachando los asuntos varios
que en aquel embrión de Gobierno pendían de su autoridad como vicario
indiscutible de Doña María Cristina, y desempeñaba su papel con tal fe y
ardor, que era lástima no fueran aplicados a más práctico objeto. De
noche, cuando hallaba algún espacio para dar reposo a su fatigado
espíritu, solía pasearse solo o con un par de amigos fieles por la
soledad del Campillo, núcleo de la antigua ciudad, o recorría las calles
concéntricas que lo cercan; y en verdad que no podía espaciar sus
ilusiones por sitios más apropiados al carácter feudal y poético de
ellas. Los monumentales caserones habitados por el silencio, las calles
que en rueda circundaban el primitivo recinto, encorvándose unas sobre
otras, y enlazando su término con el punto de partida, reproducían al
exterior el giro poético de la imaginación del paladín que amaba el
pasado, y lo llevaba de continuo en el pensamiento, en una u otra forma,
siempre volteando sobre sí mismo. La colegiata, majestuosa en el
barroquismo de su robusta torre; los palacios del Cordón, de Álava y de
Bendaña, que hablaban con sus rostros de piedra el lenguaje medieval, le
acariciaban los pensamientos y se los hacían más luminosos. ¿Por qué no
habíamos de ser lo que fuimos, nación de santos y de héroes? ¿Por qué no
habíamos de restablecer las grandezas de la sangre y de la inspiración,
del militar coraje y de las virtudes sublimes? Al par que esto, deseaba
la ilustración, la libertad con medida, la práctica de todas las
virtudes domésticas y públicas, y el culto de las artes y las letras. La
grosería le enfadaba; la irrupción de las muchedumbres ignorantes, que
imponer querían su fuerza, su garrulería y suciedad, le sacaba de
quicio, y por encima de todo poder ponía el histórico, que en el caso de
autos recibía mayor realce del consorcio feliz de la soberanía con la
belleza y de la majestad con la gracia.

Era, en suma, D. Manuel Montes de Oca representación viva de la
\emph{poesía política}, arte que ha tenido existencia lozana en esta
tierra de caballeros, mayormente en la época primera de nuestra
renovación política y social. Desde que se introdujo la novedad de que
todos los ciudadanos metieran su cucharada en la cosa pública, empezaron
a manifestarse los varios elementos que componían la raza; y si vinieron
al gobierno los hombres de temperamento peleón y los militares de
fortuna; si entraron los abogados y tratadistas con todos los enredos de
su saber forense y su prurito de reglamentación, no podían faltar los
trovadores, que se traían un ideal de la ciencia gubernativa, derivado,
más que de la realidad, de los manantiales literarios. Más de cuatro
poetas o trovadores hemos tenido en la vida pública de este siglo de
probaturas; que ellos son fruta espléndida, abundantísima, de uno de los
seculares árboles del terruño español, y gran daño han producido
anegando las ideas en la onda sentimental que derramaron sobre algunas
generaciones. El pobrecito Montes de Oca, por ser de los primeros y
haberle tocado la desdicha de venir con su lira en una época tumultuosa
y candente, fue víctima del error gravísimo de querer dar solución a los
problemas de gobierno por la pura emoción; pagó con su vida su
desconocimiento de la realidad; merece una piedad profunda, porque era
espejo de caballeros y el más convencido y leal de los poetas políticos.
Otros que vinieron después han perecido ahogados en su propia
inspiración.

No transcurrieron muchos días de Octubre sin que las ilusiones de los
revolucionarios de Vitoria (en nombre de la Reina Cristina y por su
expresa delegación) comenzaran a marchitarse. Por el lado de Zaragoza y
Pamplona no iban las cosas muy a gusto del Presidente del gobiernillo
provisional, porque la tropa que sacó Borso di Carminati, vivamente
perseguida por el General Ayerbe, no quiso pasar de Borja, capitularon
los oficiales, algunos soldados volvieron a la disciplina y otros se
dispersaron, quedándose solo el infeliz caudillo italiano, que pronto
había de ser cogido y fusilado. En las Provincias Vascongadas no contaba
la insurrección con éxitos notorios, porque desde San Sebastián avanzó
Alcalá, aventando a toda la chusma de Muñagorri y del cura de Dallo; y
si bien Urbistondo y los miqueletes bilbaínos adelantaban algo en el
interior de Vizcaya, se veían amenazados por Iturbe y Simón de la Torre,
que permanecieron fieles a Espartero. En tanto Zurbano, con los
Provinciales de Laredo y Logroño, se posesionaba de Miranda,
preparándose a invadir la Llanada. El incansable guerrillero supo
aprovechar la torpe división que los insurrectos habían dado a sus
fuerzas, y avanzó resueltamente, ocupando el puente de Armiñón; al paso
encontró a siete miñones que llevaban despachos de la Diputación
rebelde, y les fusiló sin piedad, dispuesto a hacer lo mismo con Piquero
y con todo jefe insurrecto que encontrase, cualquiera que fuese su
categoría.

La noticia de estas atrocidades, fruto natural de la guerra, tal como
aquí comúnmente se hacía, llegó a Vitoria juntamente con la mala nueva
del fusilamiento de Borso en Zaragoza, y un desaliento tristísimo se
apoderó de los que habían abrazado la causa sentimental. Pero el
esforzado corazón de Montes de Oca no se abatió con aquellos reveses, ni
amenguaron su confianza en el triunfo definitivo. De alguna parte había
de venir el remedio, por ser divina la causa que defendían, como pleito
del derecho contra la usurpación, y, en cierto modo, de lo bello y
delicado contra los avances de la grosería y del prosaísmo.

No tardó el gobernante sedicioso en verse poseído del delirio medieval,
a que le llevaba su numen político informado en el Romancero, y se metió
en el peligroso callejón de las represalias, de que difícilmente se
sale: la muerte de los miñones le indujo al error de poner a precio la
cabeza de Zurbano. Creyó, sin duda, que no faltarían en su mesnada
hombres con la ferocidad suficiente para cortar aquella cabeza y
llevársela, con lo cual creía fácilmente decapitado el cuerpo soez de la
bestia patriotera y repugnante que arrancado había su diadema a la más
hermosa de las reinas de fábula. Suelen tener sus quiebras estos
dramáticos arranques, y entonces se vio más que nunca la inseguridad del
procedimiento, pues Zurbano no parecía dispuesto a dejarse degollar; al
contrario, marchaba por la Llanada resuelto a cercenar todas las cabezas
que pudiese, y hacer con ellas espantoso adorno de los caminos.

En esto, el General Aleson ocupaba los desfiladeros de Pancorbo y Rodil,
con numerosa hueste, partía de Burgos para perseguir a O'Donnell y
desbaratarle si salía de la ciudadela de Pamplona. Iban tomando cada día
peor cariz las cosas del naciente reino cristino, tan mal fundado en los
cerebros de unos cuantos calaveras del ejército y la política: de pronto
supieron el fracaso de la intentona de Madrid, el combate en la escalera
de Palacio y la fuga de los audaces caudillos, que en plena Corte habían
concebido el proyecto, más propio de gigantes que de hombres, de
secuestrar a la Reina y llevársela a Vitoria, sede provisional de su
autoridad. Todo ello era absurdo, propio de un partido de orates, y así
salió\ldots{} Mas no se crea que el desengaño traído por estas noticias
se comunicó al espíritu alucinado de Montes de Oca, ni que desmayó su
temeridad, no: de su cabeza, en que bullía la leyenda; de su corazón,
inflamado en sentimientos de monarquismo romántico, brotaron nuevas
energías; y cuando los hombres prácticos, sabiendo la ocupación de la
Puebla por D. Martín, mostraron el gravísimo peligro de continuar en
Vitoria, se obstinó en permanecer en ella, organizando una defensa que
por lo brava y tenaz emulara las de Zaragoza y Gerona. Tal era su
pensamiento cuando la insensata empresa de restauración estaba perdida,
y los más ardientes auxiliares de ella no pensaban más que en la fuga o
en el escondite, aguardando a que pasara el nublado para procurarse una
saludable reconciliación con el Regente. Pero Montes de Oca no cejaba.
Abrazado había la causa de \emph{la Señora}, y enarbolado su bandera con
un ardor semejante al de los cruzados que iban a combatir por el
sepulcro de Cristo; otros procedían por egoísmo y despecho; él por una
fe generosa, y por la devoción, que otro nombre no puede dársele, de la
Reina que era su ídolo. No daba entrada al miedo en su corazón, ni
cuartel a los arbitrios de la cobardía, ni a componendas o
transacciones. Era hombre macizo, homogéneo, sin las complejidades que
la vida moderna exige a todos los que en ella buscan algo de provecho.
¡Lástima de primera materia, tan sólida y pura, en un siglo que no suele
emplear para sus grandes obras lo puramente elemental, en un siglo de
combinaciones y de alquimias cada día más complicadas! Toda la
caballería del bravo Montes de Oca, toda su exaltación de gobernante
poético, tenían por ideal sostén la soñada más que real persona de una
Reina, cuya capacidad para dirigir a la Nación no había sabido
manifestarse claramente. Él, no obstante, adoraba en ella, creyéndola
adornada de atributos intelectuales y morales no menos efectivos que los
de su seductora belleza. Valía más el Quijote que la dama, y era ella
menos ideal de lo que la suponía el ofuscado caballero. Si en la
imaginación de este ahechaba perlas, a la vista de todo el mundo
ahechaba trigo candeal superior la buena de Aldonza Lorenzo.

\hypertarget{xxiv}{%
\chapter{XXIV}\label{xxiv}}

Semejante a los héroes de un cuento infantil, se obstinaba Montes de
Oca, falto de todo recurso y amenazado de una deserción total de su
gente, en defenderse dentro de Vitoria, sacrificando la vida de esta
ciudad al orgullo de una causa que no debía interesar grandemente a los
hijos de Álava. Ya que la victoria se presentaba difícil por el momento,
quería el caballero un poco de leyenda, y si Dios disponía que él y sus
fieles pereciesen ante un enemigo superior, se enorgullecía pensando
concluir a la numantina. Pero las nubes que ennegrecían el horizonte
eran cada vez más temerosas, y aunque el hombre continuaba insensible al
miedo, confiado siempre en los auxilios imaginarios que había de recibir
de nuevas sublevaciones, por fin le determinó al abandono de la plaza un
hecho que hubo de abatirle los ánimos más que el aluvión de tropas
enemigas y la merma creciente de las suyas. Fue que los generales que
iban contra Vitoria agregaron a la orden del día un papel enviado de
Madrid, dando cuenta de la comunicación de nuestro Embajador en París,
D. Salustiano de Olózaga, el cual venía con el cuento de que la
\emph{propia cosechera}, Doña María Cristina, le había dicho,
\emph{mutatis mutandis}: «¡Pero si yo no sé nada de esa insurrección, ni
tengo nada que ver con esos locos! No sólo soy extraña al movimiento,
sino que lo repruebo terminantemente.» El efecto que esto hizo en el
valeroso paladín ya puede suponerse: no creía que el cuento del
diplomático fuese verdad; teníalo por una de tantas mentiras
diplomáticas, empleadas como resorte político; no le cabía en la cabeza
que habiendo Cristina puesto en manos de los sublevados armas y bandera,
renegase de sí misma y de su causa cuando la conceptuaba perdida, y
llamase locos a los que por ella daban su sangre y su honor. Esto no
podía ser: tales villanías, cosa corriente en el carácter falaz de
Fernando VII, no cabían en la nobilísima condición de la Reina, toda
rectitud, lealtad y entereza, según Montes de Oca. Sobre esto no tenía
duda el exaltado caballero, y la ideal Soberana no desmerecía en su
pensamiento por las malicias de Olózaga. Lo que agobió su ánimo valeroso
fue que aquellas mentiras entraron fácilmente en los cerebros de todos
los que le rodeaban; que el vecindario de Vitoria les dio fácil crédito,
y las aceptó hasta con gozo, viendo en ellas el mejor pretexto para dar
término rápido a la insurrección, y librarse de los desastres y
apreturas de un sitio. Ya no podía Montes de Oca sostener la moral de la
plaza, ni menos el entusiasmo, harto ficticio y ocasional, por la que
fue Gobernadora; cayó de golpe desde la cumbre de la poesía política a
una realidad miserable. Llegaba el momento de huir, exponiéndose a una
muerte ignominiosa, la del pirata o bandido. Salió, pues, de la plaza,
acompañado de Piquero y de los militares y paisanos comprometidos, sin
más tropas que los miñones y algunas compañías de Borbón. Muy distante
¡ay!, se hallaba de la ocasión en que puso a precio la cabeza de
Zurbano; nadie pensaba en traérsela, y en cambio, Rodil pregonaba la de
Montes de Oca, ofreciendo por ella \emph{diez mil duros}\ldots{} Vamos,
no era mal precio, dado el escaso valor que ordinariamente tenían en el
mercado de nuestras guerras civiles las cabezas humanas, aun siendo de
las mejor provistas de sólidos tornillos.

La salida fue tristísima, nocturna, sigilosa. Antes de que amaneciera,
en la rápida marcha por el puerto de Arlabán hacia Vergara, desertaron
las compañías de Borbón, y se fueron a Miranda para presentarse al
General de Espartero. Celebraban consejo los fugitivos para determinar
el camino que debían seguir. No pocos oficiales comprometidos señalaron
como la mejor dirección de escape la de la costa Cantábrica; sabían de
un barco preparado en Lequeitio para recoger a los que quisieran fiar su
salvación al mar. Montes de Oca, aunque marino, prefirió seguir por
tierra la derrota de la frontera; despidiéronse allí no pocos amigos y
compañeros de locura, entre ellos el comandante Gallo y otros que
andando el tiempo fueron generales, y se encaminaron hacia la costa;
Montes de Oca, acompañado tan sólo de Piquero, de los señores alaveses
Marqués de Alameda, Ciorroga y Egaña, y de ocho miñones, siguió
adelante. En Mondragón despidieron a los miñones, pues para nada
necesitaban ya la fuerza militar, y cuanto menor fuese el número de
fugitivos más fácilmente podían deslizarse por montes y cañadas hasta
ganar el boquete de Urdax. Pero los miñones no quisieron separarse de
los desdichados restos del Gobierno cristino, cuya suerte debían correr
todos los que en tan necia desventura se habían metido. En Vergara se
alojó la caravana en las casas exteriores de la villa, no lejos del
histórico lugar donde se habían abrazado Espartero y Maroto; cada cual
se arregló como pudo en humildes aposentos o mechinales, y a media noche
el sueño dio algún descanso al asendereado cabecilla de la insurrección
y a los que aún le seguían, más comprometidos ya por la amistad que por
la política.

Media noche sería cuando turbaba el silencio de aquella parada lúgubre
el cuchicheo de los ocho miñones, alojados en una cuadra, donde moraban
también una mula y una pareja de vacas. Los pobres chicos, desvelados
por la inquietud, se condolían de su perra suerte. ¿Quién demonios les
había metido en aquel fregado, ni qué iban ellos ganando con que
\emph{la Cristina} le birlara la Regencia a Espartero? En verdad que
habían sido unos grandes idiotas, apartándose de la ley que ligaba sus
vidas y su honor militar al Gobierno establecido. ¿Quién les metía en el
ajo de quitar y poner Regentes? ¿Quién les hizo instrumento de la
ambición de unos cuantos caballeros de Madrid, y de media docena de
militares que querían empleos y cintajos?\ldots{} ¡Y que no era flojo el
riesgo que corrían los pobrecitos miñones! Desde Vergara a la frontera
¿quién les aseguraba que no toparían con un destacamento de tropas
leales? En un abrir y cerrar de ojos serían despachados para el otro
mundo, y aun podría suceder que los señores que les habían arrastrado al
delito alcanzasen misericordia; para los hijos del pueblo, no habría más
que rigor y cuatro tiros\ldots{} Aun suponiendo que pudiesen escapar,
¿qué vida les esperaba en Francia? ¿Por ventura se encargaría de
mantenerles \emph{la Reina esa} por quien se habían jugado la vida? ¡Ay,
ay!, el pobre siempre pagaba el pato en estas tremolinas; para el pobre,
en la derrota o en el triunfo, no había más que desprecios y mal
pago\ldots{} ¡Qué mundo este! Valía más ser animal que español.

Estas ideas rumiaban, esto se decían, y en verdad que no habría sido
vituperable su razonamiento si de él no saliese, como de la fermentación
el gusano maligno, un ruin propósito. A dos de ellos se les ocurrió en
el curso de la conversación; pero no se atrevieron a manifestarlo. Un
tercero, que era sin duda el más arriscado, se lanzó a exponer la
terrible idea, y la primera impresión que en los demás produjo fue de
miedo; un miedo más vivo que el de la propia muerte. Eran hijos de
familias honradas, y desde niños habían visto en sus hogares la norma de
todas las virtudes, el temor de la infamia y el aborrecimiento de la
traición. Callaron un rato, y la perversa idea hizo nido en el cerebro
de cada uno de ellos, empollando diversas ideas que corroboraban la idea
madre. El mismo iniciador de esta la explanó hábilmente, revistiéndola
de aparato lógico; achicó los inconvenientes morales, agrandó las
ventajas. En primer lugar, salvaban sus vidas, y esto de mirar por las
vidas era cosa buena, pues para que el hombre se defendiese de la
muerte, le había dado Dios la inteligencia. En segundo lugar, se ponían
en buena disposición con los que mandaban: Dios había dicho que debe
darse al César lo que es del César. A más de esto, ¿quién dudaba que
Espartero era el más valiente entre los españoles? Zurbano no le iba en
zaga en el valor; sólo que se pasaba de bruto, hablaba mal, y tenía la
mano muy dura. Pero pues era el hombre que más podía en aquellas
tierras, hijo también del pueblo, debían favorecer sus ideas y ponerse a
su lado para todo. Por último, triunfantes o vencidos, su sino era
quedarse tan miñones como antes, con la triste paga, el rancho mísero y
la condición de soldados rasos. Buenos tontos serían si no sacaban algún
provecho de la trapisonda en que se habían metido. Cierto que alguien
saldría diciendo si eran tales o cuales\ldots{} pero ellos no
\emph{habían dado el grito}; ellos no habían levantado la bandera de
Cristina, ni entendían de estas cosas. Zurbano había ofrecido diez mil
duros por la cabeza de Montes de Oca: deber de ellos, que la tenían en
la mano, era entregar aquella cabeza, la verdaderamente culpable, la que
\emph{había dado el grito}. Y no dijeran que era una lástima entregar al
pobre D. Manuel, indefenso, para que en él se cebara el furor de los
vencedores. Por fas o por nefas, la vida de D. Manuel era cosa perdida.
En su persecución iban ya varias columnas, y pronto le cazarían como a
una liebre. Podría suceder que entregándole ellos, se compadeciera
Zurbano del infeliz señor, y que el gran Espartero le perdonase, con lo
cual quedaban todos contentos, Montes de Oca con vida, y ellos, los
pobrecitos miñones, con sus diez mil duros en el bolsillo, a mil
doscientos cincuenta duros por barba.

El que pronunció el discursillo que extractado se copia, había empezado
a estudiar para cura en Vitoria, sirviendo luego de amanuense a un
escribano de la Puebla de Arganzón, y en sus diferentes tareas escolares
se le había pegado el arte del sofista. Cedieron prontamente algunos de
los compañeros; para reducir a los otros fue necesario que el orador
emplease lo mejorcito de su arsenal dialéctico, y al fin convinieron
todos en consumar sin demora la execrable acción. La obscura noche les
estimulaba\ldots{} el silencio les envalentonó para un hecho que exigía
sin duda más arrojo que el desplegado en los combates. El coloquio
vascuence en que desarrollaron su plan y los procedimientos más seguros
para ponerlo en ejecución duró apenas un cuarto de hora; y bajaban tanto
la voz que apenas se oían, temerosos de que la mula y las vacas, únicos
testigos de la terrible conferencia, la entendiesen y renegasen de tal
villanía, como honrados animales.

\hypertarget{xxv}{%
\chapter{XXV}\label{xxv}}

El modo y forma de hacer efectivo su pensamiento fue para los miñones
sencillísimo. Lo propuso uno que en su niñez desplegaba felices
disposiciones para robar fruta en las huertas y alguna que otra gallina
en los corrales. Salieron los ocho a un cercado frontero a las dos casas
en que se alojaban los paladines de la Reina, y con fuertes voces
empezaron a gritar: «¡Zurbano, Zurbano!\ldots» El efecto de este toque
de diana fue inmediato y decisivo. Los caballeros durmientes saltaron
despavoridos de sus lechos, y a medio vestir lanzáronse fuera por los
primeros huecos que abiertos encontraron: Egaña saltó por una ventana, y
a Piquero se le vio surgir por un boquete angosto que daba al campo en
la parte posterior del edificio. Poner el pie en tierra y apretar a
correr en busca de la espesura del monte más cercano fue todo uno. Los
otros dos, tomando la salida por la puerta con más tranquilidad, no
tardaron en desaparecer. Como en los incendios y naufragios, cada cual
se afanaba por salvar su propia pelleja sin cuidarse de la del vecino.
Dos miñones pusiéronse de guardia en la escalerilla estrecha que a la
estancia ocupada por el jefe conducía, con objeto de apresarle cuando
saliese, y viendo que tardaba, presumieron que se había escondido en los
desvanes. Los inquilinos de la casa, un hombre y dos mujeres, que a poco
de sonar las primeras voces de alarma abandonaron también sus
madrigueras y vieron la veloz huida de los cuatro señores, aseguraban
que el quinto de ellos no había salido. Viéronse precisados los
traidores a subir en su busca, creyendo que, o se había muerto del
susto, o que por el escrúpulo de conciencia quería expiar sus culpas
bajo el poder del temido Zurbano.

A las primeras luces del alba subieron dos miñones, el de los discursos
y otro que blasonaba de arrojado, al aposento mísero donde reposaba en
un pobre camastro el jefe de la insurrección, y le hallaron
profundamente dormido. Su tranquilo sueño era la expresión de su ciega
confianza en los ocho corazones alaveses a quienes había entregado su
vida. Por un instante creyéronle muerto: tales eran el reposo y palidez
de sus nobles facciones. Uno de ellos le llamó: «D. Manuel, Sr.~D.
Manuel\ldots» No despertaba. Imposible parecía que con la batahola y
vocerío que armaron los guardianes durmiese con sueño de ángel aquel
hombre que reunía en su espíritu la fiebre poética y el bélico ardor.
Fue preciso sacudirle de un brazo para que despertase. Abrió al fin los
ojos, y miró largo rato a los dos chicarrones, sin darse cuenta de lo
que ocurría. «¿Es hora de salir?---dijo.---Vamos al momento. ¿Se ha
levantado Piquero?»

El más desenvuelto de los dos traidores quiso expresar el verdadero
sentido de la situación, y no halló la frase propia. «Es usted
preso---dijo el otro, cortando por lo sano;---los demás señores han
huido; usted no puede, Don Manuel, y ahora se viene con nosotros a
Vitoria.»

Empezaba el infeliz hombre a comprender la situación; pero aún no la
veía en toda su trágica realidad, ni le entraba fácilmente en la cabeza
la idea de que los honrados hijos de Álava le apresaban para venderle
por los diez mil duros que ofrecía Rodil. Se incorporó vivamente; miró
en torno suyo. No tenía armas; nunca creyó que podía necesitarlas. «¡Y
vosotros---dijo,---me prendéis y me lleváis a Vitoria\ldots! Pero no lo
haréis movidos del premio que dan por mí. No valgo yo tanto, amigos.»

---Sr.~D. Manuel---dijo el valiente, ya repuesto de su turbación,---no
nos enredemos en palabras que no vienen al caso. Vístase pronto, que
tenemos prisa.

---Está bien---replicó Montes de Oca, pasando brevemente de la ira a la
resignación, por la virtud de su grande alma.---Me vestiré al instante.
Habría sido mejor que no viniéramos acá. Mi deseo, ya lo sabéis, era no
salir de Vitoria y esperar allí a los vencedores. Entregándome yo, los
diez mil duros habrían sido para mí, aunque\ldots{} ¡sabe Dios la cuenta
que me harían!\ldots{} Bueno, hijos: pues tenéis prisa, ahora mismo nos
vamos. Dejad que me lave un poco: es costumbre mía, que vosotros sin
duda no tenéis. Amanece ya; saldremos con la fresca, y marcharemos tan
rápidamente como queráis.

Partieron a escape: a los miñones se les hacían siglos las horas que
faltaban para cobrar el importe de la res que vendían. Para recorrer la
tiradita desde Vergara a Vitoria en el menor tiempo posible, echaron por
los atajos y desfiladeros más apartados de toda población, temerosos sin
duda de que algún destacamento de tropas les quitase la gloria de su
hazaña y el precio de su botín. Dieron a D. Manuel un caballejo, y tanta
era la prisa, que no cuidaron de llevar víveres, ni fácilmente podrían
adquirirlos en las soledades por donde caminaban. Tiraron hacia Legazpi,
y de allí a los altos de Aránzazu, royendo mendrugos de pan el que los
tenía. En uno de los breves descansos que hicieron, más por dar alivio a
la caballería que al desdichado jinete, manifestaron a éste que,
hallándose preso y a disposición de las autoridades, maldita falta le
hacía el dinero que aún conservaba en sus bolsillos para los gastos de
la insurrección primero, de la fuga después. Dio Montes de Oca una
prueba de buen gusto y de austera dignidad evitando toda discusión sobre
el infame despojo, y entregoles, sin el honor de una protesta ni de un
comentario, la culebrina en que llevaba unas cuantas onzas, que no
llegaban a diez, y alguna plata menuda. Y hecho esto, arrearon de nuevo.

Hablaban los miñones entre sí el idioma vascuence, del cual el infeliz
preso no entendía palabra, resultándole de esto un tormento mayor: el
sentirse más aislado, más lejos de su patria. Entre esta y el poeta se
interponían un suelo desconocido, una gavilla de bandoleros y una jerga
que nada decía a su entendimiento ni a su corazón. En el fatigoso paso
por veredas y trochas, mortificado del hambre y la sed, sin otro
sentimiento inmediato que el desprecio que le inspiraban sus guardianes,
sufrió el desdichado caballero indecibles angustias. No había para él
más consuelo que aislarse, con esfuerzo de su viva imaginación,
procurando no ver fuera de sí más que la Naturaleza, y dentro las
hermosuras de su grande espíritu, así en el orden moral como en el
estético. Las bellezas del paisaje y del cielo, las ideas propias, que
iba sacando del magín con cariño de avaro, para en ellas recrearse y
volver a esconderlas cuidadosamente, permitiéronle, si no el completo
olvido de su desgracia, alguna distracción o alivio pasajero. Mas las
exigencias físicas del hambre y la sed le volvían a la realidad de su
martirio; otra vez era el hombre vendido, la bestia llevada al matadero
por cuatro carniceros infames, y la ininteligible cancamurria vasca otra
vez le cortaba el cerebro como una sierra.

La molestísima andadura del jaco, apaleado sin cesar por los miñones,
magullaba los huesos del pobre jinete. Habría preferido caer al suelo y
que en él le fusilaran sin compasión; pero su vida valía diez mil duros,
y no podía esperar de los mercaderes una muerte gratuita. Estas ideas
lleváronle a mayor resignación y a conformidad más profundamente
cristiana con su fiero destino. El sentimiento caballeresco y la ilusión
del sacrificio pudieron tanto en su alma, que no le fue difícil llegar a
la tranquilidad ascética que permite soportar un intenso padecer, y aun
alegrarse de los martirios. Instantes hubo en que se creyó dichoso de
ser tan infeliz, y el goce amargo de los sufrimientos refrescaba su
alma, y la erguía, y la vigorizaba para mayores resistencias. Hermoso
era el dolor, bellas las angustias que preceden a la muerte. Contra
nadie tenía queja. Y no creía ciertamente que la persona por quien en
tal suplicio se veía un hombre de bien, fuera indigna de semejante
holocausto, no. Todos los males presentes y otros peores que vinieran
los sufría gustoso por la Reina, por una divinidad que no habría sido
bastante divina si no creara mártires, si ante su triunfal carro no
cayeran aplastadas cien y cien víctimas. Bien sabía la Reina lo que sus
fieles padecían por ella, y bien empleado estaba que los caballeros
penaran y murieran, para que sobre tantos dolores y sacrificios se
alzara la gloriosa \emph{redención monárquica}.

¡Y los malditos alaveses arreando sin descanso, como diablos solicitados
de la querencia del infierno! «Basta, hombres, basta, que ya
llegaremos---les dijo Montes de Oca, compadecido del caballejo más que
de sí mismo.---Por mí no importa; pero vosotros tampoco vais a ninguna
fiesta. Tened lástima del pobre animal, que no puede ya con su alma.»
Vino la noche, y con ella redoblaron los palos sobre la
cabalgadura\ldots{} No corrían: volaban. En un día anduvieron \emph{diez
y siete leguas}, imposible jornada cuando se va en seguimiento del bien,
o a realizar una noble acción. Sólo el mal hace a los hombres tan
ligeros. A las nueve de la noche llegaban a las proximidades de Vitoria,
donde pararon, mandando aviso por dos de ellos al General Aleson, con
las nuevas de la valiosa presa que traían. Tropas llegaron al instante y
se hicieron cargo del reo, llevándole con no poco aparato de fuerza a la
Casa Consistorial, que entonces estaba en San Francisco, donde también
había cuartel. A la luz de tristes faroles entró el jefe de la
insurrección en el aposento que le destinaron, y lo primero que con él
se hizo fue registrarle para ver si tenía documentos de algún valor. En
efecto: descuidado como buen poeta, conservaba en sus bolsillos dos
papeles que había escrito antes de la salida de Vitoria, y que se olvidó
destruir. El uno era una carta dirigida a O'Donnell en que amargamente
se quejaba del abandono en que se le tenía. «Ni un fusil, ni un real, ni
una comunicación he podido conseguir a pesar de mis esfuerzos\ldots{} Si
hubiera tenido armas, a esta hora contaría la Causa de la Reina con un
ejército de 20.000 hombres\ldots{} Si se pierde esta coyuntura, la Causa
de nuestra Reina se hundió para siempre\ldots» El otro era un oficio en
que se leía: «\emph{Gobierno Provisional}\ldots{} Excelentísimo Sr.:
Este infame pueblo nos ha vendido, y su Ayuntamiento ha oficiado a
Zurbano diciéndole no harán resistencia y me entregarán\ldots{} Se hace,
pues, indispensable abandonarlo, y lo verificamos esta noche\ldots» Aquí
se ve cuán galanas cuentas hacen los revolucionarios, cuya imaginación
fácilmente traduce en realidad los deseos locos. ¡Fusiles, dineros!
¿Pero de dónde los había de sacar O'Donnell? Para él los hubiera
querido. Él que no sabe allegar estos ingredientes antes de izar la
bandera, que no se meta en tales andanzas.

Después de bien registrado, entraron a verle el General Aleson y el jefe
político, que, según se cuenta, no estuvo cortés ni generoso con la
víctima. Tras estos llegó el Coronel D. Santiago Ibero, encargado de
cumplir el sanguinario bando de Rodil, lo que en realidad no exigía
larga tramitación. Bastaba con identificar la persona para proceder al
corte de cabeza, con lo cual quedaba fuera de combate la hidra
revolucionaria. Luego declaró el reo con voz entera su nombre, el pueblo
de su nacimiento (Medinasidonia), su estado (soltero), su edad (treinta
y siete años menos dos meses). Otras cosas dijo que no fueron más que
una nueva página de poesía política.

Al quedarse solo con Ibero, Montes de Oca le dijo afectuoso: «No es la
primera vez que nos vemos.»

---En el castillo de Olite\ldots{}

---Y alguna vez antes.

---Alguna vez, sí señor---replicó Ibero saciando sus miradas en el
rostro del infeliz reo.---No una sola vez, si es fiel mi memoria\ldots{}
Perdone usted que le mire y le remire\ldots{} Deseaba mucho verle; pero
no, válgame Dios, en esta tristísima situación.

\hypertarget{xxvi}{%
\chapter{XXVI}\label{xxvi}}

---Si a usted no le parece mal---dijo Montes de Oca, sin aliento casi,
estirando sus miembros doloridos,---descansaré. No tiene usted idea de
cómo me han traído esos perros, de Vergara a Vitoria. Creí que me
quedaba en el camino, y no habría sido malo para mí.

---He mandado que le pongan a usted una buena cama, y podrá descansar.
También se le traerá la cena. Yo siento mucho que usted no hubiera sido
más cauto en su fuga. Debió usted salir de aquí en la noche del 17, en
la diligencia que le prepararon sus amigos.

---Qué quiere usted\ldots{} No tengo, no he tenido nunca el instinto de
la fuga. Me siento amarrado al puesto en que me coloca mi deber. No
quería Piquero que yo partiese sin él, ni quería yo dejarle aquí. Juntos
nos lanzamos a esta calaverada, juntos debíamos salvarnos o perecer. No
me pasó nunca por la cabeza que los miñones fueran mi Judas.

---Egaña y Ciorroga ¿por qué no impidieron este oprobio que los miñones
han arrojado sobre la raza alavesa? Si aquí mandara yo, crea usted que
después de darles el dinero les mandaría hacer testamento y les
fusilaría sin escrúpulo de conciencia.

---¡Ah!, esto no puede ser---replicó el reo, que de improviso apartó su
mente de aquel asunto, más atento a la cama que entraron los asistentes
y a designar el sitio donde debían ponerla.

Dio sus órdenes con serenidad, cual si se hallara en las ocasiones
ordinarias de la vida, y volviendo la espalda al Coronel, ayudó a
colocar los cojos bancos sobre que se ponían las desunidas tablas para
sostener los colchones.

«Agradeceré mucho---dijo cuando los asistentes traían sábanas y
abrigo,---que me den lo necesario para asearme un poco: agua, cepillo,
peines. Nada me molesta como la suciedad, y este viaje ha sido funesto
bajo el punto de vista de la pulcritud\ldots{} Mire usted qué
manos\ldots{} Mi pelo es un bardal\ldots»

Dio órdenes Ibero de que se le trajesen los avíos de tocador de que se
pudiese disponer, y agua abundante.

«Es triste cosa---dijo Montes de Oca quitándose el gabán y la levita, y
preparándose a un breve lavatorio,---que siendo yo fanático por la
limpieza me vea en tal suciedad. No se asuste usted ni me riña si le
digo que mi intento ha sido lavar al país\ldots{} Y ahora resulta que no
se deja\ldots{} como los niños mal criados que no tienen más gusto que
revolcarse en el fango de los caminos\ldots{} Y yo, tan aficionado al
aseo general, ahora me veo en la porquería particular más repugnante,
sin otro consuelo que unos cuantos buches de agua para darme un refregón
en cara y manos\ldots{} Pero, en fin, pronto no me hará falta el agua
para estar bien limpio.»

Terminada la frase con un gran suspiro, empezó sus abluciones, que la
corta medida del agua había de limitar más de lo que él quisiera. Salió
D. Santiago a prevenir la cena, ordenando que fuera lo mejor posible, y
al volver junto al preso, le encontró refregándose el rostro con la
toalla.

«Pues sí---dijo Montes de Oca expresando lo que había pensado durante el
lavatorio,---la noche de marras, ¿se acuerda usted?, cuando nos
conocimos en una casa\ldots{} el nombre de la calle se me ha ido de la
memoria\ldots{} Pues yo le emplacé a usted\ldots»

---Y yo anuncié a usted y a Gallo que esto era una locura y\ldots{}

---Justamente. Cada cual dijo lo que sentía. Este desastre, que tengo
por accidental, no modifica mis ideas sobre lo fundamental. Hoy hemos
sido vencidos; somos la primera fila de combatientes, que tropieza y
cae. Pero detrás vienen otros y otros\ldots{} No lo dude usted:
triunfarán la verdad y la justicia. No puede ser de otra manera.
Confirmo, pues, mi pronóstico.

---Y yo el mío\ldots{} Pero no es ocasión de empeñarnos en discusiones
ni en alabarnos de profetas. Los grandes cambios de la vida general
vienen cuando ellos quieren, y no está en nuestra mano traerlos fuera de
tiempo. ¿No piensa usted lo mismo?

---No, señor---dijo Montes de Oca, peinando con fruición su espléndida
cabellera,---y dispénseme que le contradiga. Es deber del hombre
impulsar los acontecimientos buenos, los que realizan la justicia y el
bien, porque si nos abandonamos, si la apatía nos vence\ldots{} el mal
se hará dueño del mundo.

---Cierto; pero no confundamos los acontecimientos buenos, como usted
dice, con los que parecen tales por la forma engañosa que les da nuestro
deseo\ldots{} o si se quiere, nuestro fanatismo.

---¡Fanatismo! Sí, a eso vamos a parar. El mío tiene por objeto de su
culto las cosas eternas. Vea usted por qué no estoy tan afligido y
agobiado como corresponde a mi situación, según el criterio vulgar.

---Muy bien, señor mío. Pero yo sé que no pensaban mucho en las cosas
eternas otros que se lanzaron a esta insensatez---afirmó Ibero, que
antes de concluir la frase, cayó en la cuenta de su inoportunidad.

---Quítense ustedes el éxito, y hablaremos de lo que es insensato y de
lo que no lo es---dijo Montes de Oca, ya peinado, sentándose frente al
Coronel, rodillas con rodillas.---Por de pronto, este pobre vencido y
condenado sostiene ahora que vale más, mucho más, hacer locuras por la
justicia y la verdad que hacer cosas muy sensatas y muy correctas por la
usurpación y por la mentira. Yo he cumplido con mi deber; mi conciencia
no hace ahora distinciones entre la demencia y la cordura: no ve más que
lo justo y lo injusto. Con lo justo estuve y estoy, con todo lo que
vemos de la parte de Dios. Soy religioso: la muerte no causa terror a
los hombres de acendrada fe. ¿Qué tiene usted que decir?

---Nada, nada más sino que admiro su entereza, y que me causa vivo dolor
ver que hombres de tal temple\ldots{} En fin, señor mío, hablemos de
otra cosa, porque al paso que vamos resultará que tendrá usted que
consolarme a mí y darme ánimos, cuando lo que procede\ldots{} Ea, ya
está aquí la cena. ¿Tiene usted apetito?

---Regular---dijo el reo preparándose a caer sobre el primer
plato.---Antes de lavarme sentía gran debilidad\ldots{} Realmente
necesito alimentarme para que no se apoderen de mí las ideas
tristes\ldots{} No le invito a usted a que me acompañe, porque habrá
cenado a hora más conveniente. Los condenados a muerte tenemos unas
horas absurdas para nuestras comidas.

Empezó con mediano apetito, y según avanzaba iba recibiendo más gusto de
la cena. Mientras esta duró, oyéronse mugidos del viento: las persianas
del único balcón de la pieza se movían con lastimero chirrido, y en los
buhardillones sonaban porrazos, como de algún batiente abierto que era
juguete de las impetuosas ráfagas del aire.

---Viento del Oeste---dijo D. Manuel con absoluta serenidad, sin dejar
de comer.---Esta tarde, cuando bajábamos por las Peñas de Zaraya,
soplaba el Sur sofocante. El cariz del cielo me dijo que antes de media
noche rolaría el viento al tercer cuadrante.

---Y tras este ventarrón tendremos agua.

---Si se agarra al Sudoeste, tal vez; pero por intermitencia de las
rachas, paréceme que rola al Noroeste\ldots{} Vendrá el agua\ldots{}
pero más tarde\ldots{} No seré yo el que se moje.

---¡Quién sabe\ldots!

---Que no, digo. Le apuesto a usted todo lo que quiera a que no me
mojo\ldots{}

Le vio Ibero soltar el tenedor y quedarse inmóvil, fija la vaga mirada
en el mantel. Quiso decirle algo, y aun pronunció algunas palabras de
vulgar consuelo; pero pronto enmudeció. Le constaba que no había
esperanza: era por tanto crueldad llevar al ánimo del reo una vana
ilusión, que al desvanecerse haría más acerbo su suplicio. No se le
ocurrió más que la simplicidad de invitarle a dormir, buscando en el
sueño la reparación de fuerzas. ¿Y para qué las necesitaba?\ldots{} Más
inquieto por su descanso que por su vida, el reo formuló una pregunta:

---Dígame: ¿querrán que esta noche amplíe mi declaración?

---Mañana quizás. No piense usted ahora más que en descansar.

---¿De modo que por esta noche no vienen a molestarme? Magnífico\ldots{}
Pues si usted me lo permite, me acostaré ahora mismo.

---Y dormirá. El cansancio es un excelente narcótico.

---Yo tengo un sueño fácil. Dormía profundamente cuando los miñones
tramaban venderme.

---¿Y este furioso viento que hace ruidos tan extraños no le impedirá
dormir?

---¡Quia! ¡No me despertó la traición, y cree usted que me despierta el
aire! Ya conozco yo al viento: somos amigos. No es malo el viento, no;
por lo menos, traidor no es. Mejor estaría yo ahora en medio de la mar
que aquí. A un temporal duro del Oeste se le capea; a una mar gruesa se
la domina poniéndole la proa; ¿pero contra estas infamias de los hombres
qué podemos?

---¿Y por qué dejó usted la vida del mar por las ignominias de la
política?

---¡Ah!, no puedo contestarle tan fácilmente\ldots{} Mucho hablaríamos
usted y yo si tuviéramos tiempo; pero ya verá usted cómo no lo
tenemos\ldots{} Llevarán las cosas muy aprisa, y más vale así.

---Sí: más vale\ldots{} Pero no se detenga usted si quiere acostarse, ni
le importe que yo esté presente.

---Gracias. Pues es usted tan amable que me permite el descanso, me
acostaré.

Y diciéndolo, iba dejando sobre dos sillas próximas las prendas que se
quitaba. Ibero, que desde la llegada y entrega del prisionero se sentía
devorado por intensísima curiosidad, anhelando aclarar un punto obscuro
de sus breves conexiones con el interesante cuanto infeliz caballero,
creyó que la ocasión era propicia para permitirse apelar a su confianza.
«Señor de Montes de Oca---le dijo cuando el reo acababa de meterse en la
cama,---quisiera que me sacase usted de una duda\ldots{} Hemos recordado
esta noche la entrevista que tuvimos Gallo, usted y yo\ldots»

---La tengo tan presente como si hubiera sido ayer.

---Y yo\ldots{} Pero no es eso. Yo estoy en que nos vimos después en
otra parte.

---¿Después\ldots{} cuándo, dónde?---preguntó el condenado mirándole un
rato con gran fijeza.

---Si no sabe usted cuándo y dónde, es que no recuerda, o que en efecto
no me vio\ldots{} o que no le conviene decirlo\ldots{}

---Desde la entrevista con Gallo, no volví a ver a usted hasta que nos
encontramos en el castillo de Olite.

---Perdone usted---dijo Santiago notando disgusto en la fisonomía del
preso;---cometo quizás una inconveniencia interrogándole\ldots{} Quitar
a su descanso algunos minutos es verdadero crimen. Me retiraré para que
usted duerma.

---Gracias. Pues mire usted, aunque parezca mentira, tengo sueño.

---¿Y dormirá?

---Creo que sí. Cuando navegaba, dormía sosegadamente en las noches de
temporal duro, siempre que no estaba de guardia, se entiende. Ahora, no
sé\ldots{} En fin, pásese usted por aquí dentro de un rato y lo verá.

\hypertarget{xxvii}{%
\chapter{XXVII}\label{xxvii}}

circunstancias del reo, su figura, su palabra, su no afectada filosofía
le trastornaban profundamente. Diera él por salvarle la vida parte de la
suya; mas no estaban las cosas para esperar clemencia, ni había
posibilidad de que por caminos indirectos e ilegales se desviase de la
muerte la desgraciada vida de D. Manuel Montes de Oca. Fue a visitar al
General Aleson para darle cuenta de las medidas tomadas para la
seguridad del prisionero, de la resignación y estoicismo de este, y
acordaron el plan de servicio para el siguiente día, que habría de ser
en Vitoria día de luto. Tímidamente apuntó Ibero la idea de perdón; mas
ni aun le dejó tiempo el General de expresarla por entero, y le mostró
la orden de Rodil, disponiendo la inmediata ejecución del preso\ldots{}
¡y hasta fijaba la hora, como suele fijarse la de una fiesta! Llena el
alma de amargura volvió Santiago al Ayuntamiento y a las habitaciones
habilitadas para prisión y capilla. En esta los soldados de guardia
dormitaban en un banco, y dos ordenanzas, asistidos por empleados del
Ayuntamiento, preparaban la mesa en que se había de poner el altar: los
candeleros y el Cristo estaban aún en el suelo, junto con una Dolorosa,
arrimadita a la pared. Encargó el Coronel a su gente que despachase
pronto la faena, evitando cuidadosamente todo ruido, para no despertar
al pobre reo. Como objetaran los tales que no podían colocar el cuadro
de la Virgen sin clavar alguna escarpia, les ordenó el jefe que toda
operación ruidosa se aplazase hasta la mañana.

Entró luego de puntillas en el dormitorio, alumbrado por un velón
delante del cual se había puesto un grueso libro de canto, haciendo de
pantalla, y vio al dulce, y este era la forma visible de una conciencia
tranquila, de un cerebro despejado de cavilaciones. Pareciole mentira al
Coronel lo que veía, y admiró al mártir dormido más que le había
admirado despierto. Cautelosamente abandonó la alcoba, despidió a los
que armaban el altar, pues tiempo había de ponerlo todo muy bonito a la
mañana siguiente, y se quedó solo con la guardia. Poco después entró el
oficial que la mandaba; acordaron entre los dos que los soldados
estarían mejor en la estancia próxima, guardando la puerta por el
exterior; y pues la alcoba del preso ofrecía completa seguridad, por no
tener otra puerta que la de comunicación con la capilla, no era preciso
poner gente en esta. El patio a que daba el balcón de la alcoba estaba
perfectamente custodiado, y ni en sueños se podía temer una evasión.
Además, el preso era un santo, un verdadero santo, que con su propia
mansedumbre, con su resignación cristiana y filosófica se guardaba. Poco
después de este breve diálogo, Ibero estaba solo en la capilla,
alumbrada por dos cirios del altar, que encendió por sí mismo, pues no
gustaba de la obscuridad. Se paseó de un ángulo a otro; pero asustado
del ruido de sus pasos se sentó en un sillón de cuero, traído
expresamente para que lo ocupase el cura en el momento de la confesión.

«Yo, que no estoy en capilla---se dijo,---no podría dormir ni un minuto
en esta creo que exista en el mundo. Señor, ¿de qué materia y de qué
espíritu le has hecho?\ldots{} ¿Esa serenidad es convencimiento de que
ha luchado y muere por una causa justa? Convencimiento es, aunque
erróneo, que es como decir obcecación. Hombres así quiero para toda
causa que yo defienda. Buen ejemplo nos da, bueno. No lo olvidaré, por
si algún día me toca la china\ldots» Divagó un instante el pensamiento
del Coronel, siempre alrededor del mismo sujeto y asunto, y vino a parar
en la idea dominante: «Voy creyendo que no es \emph{el caballero de
Rafaela}\ldots{} Avivo mi memoria, y la semejanza de este con el que vi
en aquel instante breve no es, en efecto, de esas semejanzas que alejan
toda duda. Aquel era más alto, y como guapo, qué sé yo\ldots{} Este
tiene quizás más expresión, más dulzura en el rostro\ldots{} ¿En qué me
fundaba yo para creer que aquel y este fuesen uno mismo? Era presunción
mía\ldots{} un no sé qué\ldots{} el dato de ser hombre superior, de alta
posición, según Rafaela me dijo; el dato de que allí estaban tramando
esta revolución\ldots{} No es delicado, no; no es humano que le haga yo
preguntas sobre los sitios en que conspiraba.» Al pensar esto,
sintiéndose ya con amagos de somnolencia, oyó violentísimas sacudidas
del viento y los bramidos lastimeros que daba al pasar rascándose contra
las paredes del vetusto edificio. En la techumbre sonaba también un
traqueteo metálico, como si un tubo de chimenea, tronchado por el
huracán y sujeto aún a su base por una tira de latón, quisiera
desprenderse y volar. Entre estos desapacibles ruidos, creyó sentir
también algo como un suspirar vago, como articulación de tenues
sílabas\ldots{} Sin duda Montes de Oca hablaba dormido, agobiado quizás
por una pesadilla. Asomose pausadamente Ibero a la puerta de la alcoba,
y distinguió en la penumbra el rostro del durmiente en la propia
disposición en que antes lo viera, brazos y manos en la misma postura.

Instalado de nuevo el Coronel en su sillón de cuero, que, dicho sea de
paso, no carecía de comodidad, estiró las piernas sobre una silla
próxima, diciéndose: «Parece que el sueño de ese hombre bendito, de ese
caballero sin mancilla, me contagia\ldots{} No creí que podría yo pegar
mis ojos esta noche\ldots{} Pero no, no es esto sueño: es modorra, el
gotear lento de mi tristeza\ldots{} Ahora cesa el viento\ldots{} gracias
a Dios. Se le oye distante, no como si él se alejara, sino como si le
enterraran a uno\ldots{} A ese hombre hermoso, honrado y bueno, víctima
de un fanatismo como otro cualquiera; vencido en la plenitud de la
fuerza y de la vida, le enterraremos mañana, no porque él se muera, que
bien sano está, sino porque le matamos. Y mis soldados, por orden mía,
serán los que le hagan fuego\ldots{} Esto es horrible\ldots{} Mentira
parece que se duerma uno pensando estas cosas\ldots{} Pero no es dormir:
es sentir en hondo y pensar en negro\ldots{} No me duermo, no.»

Y diciendo que no se dormía, quedose en ese estado intermedio y confuso
que es un soñar en vela, o un insomnio con descanso. Razonaba su propio
soñar de esta manera: «La prueba de que no duermo es que oigo los
mugidos del viento, y veo todo lo que hay en la capilla: las velas de
cera, la Dolorosa, que todavía está en el suelo\ldots{} Yo dispuse que
se dejara para después la operación de colgarla en su sitio, y convine
con Rafaela en que ella clavaría la escarpia\ldots{} Debe de ser la hora
convenida, porque aquí entra Rafaela Milagro con el martillo\ldots{} Se
acerca a la alcoba, observa, ve que duerme D. Manuel, y no quiere
despertarle\ldots{} Aún es pronto, mujer---dijo Santiago a su amiga, que
en forma corpórea, dormido o despierto, pues esto no estaba bien claro,
ante sí veía.---Luego colgaremos tú y yo la santa imagen, que, entre
paréntesis, se parece mucho a ti.»

Desapareció Rafaela sin que Ibero pudiese advertir por dónde, y durante
un lapso de tiempo de inapreciable dura, perdió el Coronel toda
sensación de la realidad. Sonaron de nuevo las voces del viento en forma
y tonalidad muy singulares. Por las rendijas de las cerradas maderas se
colaban los filos del aire, y tanto se oprimían, que el sonido se
aguzaba y era más lastimero y terrorífico. A ratos entraban palabras
delgadas y larguísimas, que decían cosas\ldots{} conceptos de estructura
semejante a la de una espada. Rafaela volvió a presentarse, con el
cabello suelto y una calavera en la mano, y llegándose a Ibero le dio un
golpe en el pecho, diciéndole: «Eres un cobarde, un vil, si permites que
le maten\ldots»

---¿Pero qué puedo hacer yo, mujer?\ldots{}

---Es facilísimo. Yo le despertaré. Mientras se viste, tú mandas que se
retire toda la tropa que hay en el patio. Él y yo nos descolgaremos por
el balcón. Tengo dos llaves para poder salir al otro patio y a la calle.

---¿Y yo\ldots{} pero yo\ldots?

---¡Tú!\ldots{} Harás lo que me has dicho: o pegarte un tiro, o dar la
cara como encubridor de la fuga, sacrificando tu honor militar. Escoge
lo que te parezca mejor.

---Necesito un día para pensarlo. Déjame ahora.

El chillar horrísono de las palabras que se introducían por las junturas
taladraba los oídos del buen Coronel. Llevose ambas manos a las orejas
para cortar el paso de las voces fieras, insultantes, provocativas que
querían penetrar en su cerebro\ldots{} Vio a Rafaela pasar velozmente de
una parte a otra de la estancia y meterse en el dormitorio del reo. Hizo
un movimiento para detenerla\ldots{}

\hypertarget{xxviii}{%
\chapter{XXVIII}\label{xxviii}}

Vio D. Santiago al oficial de guardia, que ante él se inclinaba,
repitiendo una pregunta que acababa de formular sin obtener
contestación. Tuvo el Coronel la palabra en la boca para decirle: «Esa
mujer que ha entrado aquí, ¿dónde está?» Pero no tardó en comprender la
incongruencia de este concepto, y sólo dijo: «¿Qué hay?»

---Mi Coronel, ya es de día. Creo que el preso ha despertado. Los
señores capellanes están a sus órdenes. ¿Les mando que entren? ¿Se
acabará el arreglo de la capilla?

---Es muy temprano aún. Retírese usted, y los capellanes que aguarden
hasta que se les avise\ldots{} Yo no dormía. Es que me duele
horriblemente la cabeza. Este maldito viento\ldots{}

Nuevamente solo, sintió toser a Montes de Oca, y allá se fue casi de un
salto. El reo había despertado, conservando la misma postura del sueño,
y recibió a su amigo con una sonrisa cariñosa y un cortés saludo. «¿Se
ha descansado?» fue lo único que dijo Ibero, que recayendo en su
incertidumbre, registró con inquieto mirar toda la estancia.

---Es de día---dijo Montes de Oca.---¡Qué pronto viene!

---Aún puede usted descansar un poco; yo se lo permito.

---Lo agradezco. Aunque no dormiré más, me quedaré un ratito en la
cama\ldots{} Créame usted: están mis pobres huesos como si me los
hubieran roto. No puedo moverme. Deme usted un cigarro.

El Coronel le alargó su petaca; cogió de la misma un cigarro para sí, y
encendiéndolo en la lámpara, dio lumbre al reo. Cuidose luego de apagar
la luz y de abrir las maderas para que entrase la claridad del día.
Iluminado por ella, el rostro del reo salía de la noche y del sueño con
marcada expresión de santidad, y cuando se incorporó con la dificultad
premiosa de sus huesos doloridos, Ibero le halló más demacrado que la
noche anterior, y notó en su semblante mayor dulzura y serenidad. Pero
debía de ser ilusión, efecto quizás de la débil luz matutina, porque no
podía una sola noche determinar cambio tan brusco, habiendo cenado y
dormido el hombre como en días normales. «Esta es la mía---se dijo Ibero
sentándose junto al lecho, y viendo cómo se confundía el humo de los dos
cigarros.---No encontraré mejor ocasión para salir de dudas. Haré mi
pregunta con la mayor delicadeza: ¿Conoce a una tal Rafaela Milagro,
viuda\ldots? ¿Salió con ella de una casa, \emph{etcétera?\ldots»}. No
había encontrado aún la fórmula más discreta para empezar, cuando Montes
de Oca se le anticipó planteando la conversación a su gusto.

«Las ocasiones críticas de nuestra existencia---dijo,---son las más
propicias para avivar en nosotros el recuerdo de cosas pasadas, a veces
muy remotas, representándonos los sucesos lejanos tan vivos como si
fueran de ayer; y lo más particular es que comúnmente reproducimos, en
estos casos críticos, escenas, pasajes y actos que no tienen nada que
ver con nuestra situación presente. Le contaré a usted un prodigio de mi
memoria, si no le molesta oírme.»

---De ningún modo\ldots{} ¿Ha tenido usted sueños, reproducción fingida
de lo que fue real\ldots?

---Algo soñé; pero fue después, hallándome despierto, poco antes de que
usted entrara, cuando vi repetirse en mi mente un suceso de mi vida
pasada\ldots{} con tal viveza, amigo mío, que llegué a creer que no
vivía en este tiempo, sino en aquel, y que no pasaba lo que ahora pasa,
sino aquello\ldots{} ¡Cosa más rara!\ldots{} Óigalo usted. Ello fue el
año 29: yo tenía entonces veinticinco años, ¡dichosa edad!, y era
alférez de navío\ldots{} No crea usted, había navegado mucho: en la
fragata \emph{Temis}, en la \emph{Sabina}, en la \emph{María Isabel}, en
la corbeta \emph{Zafiro}. Ya me conocían los mares\ldots{} Pues, como
digo, hallábame en Cádiz, cuando encalló en aquellas playas un barco de
piratas, y reducidos a prisión todos sus tripulantes, resultó la más
execrable patulea de bandidos que se pudiera imaginar. Sus declaraciones
espantaban: incendios de buques, asesinatos de navegantes, robos
inauditos, violaciones de mujeres, cuantas atrocidades ideó el
infierno\ldots{} El capitán, que era un francés de buena presencia y
modos elegantes, lo refería todo con la mayor indiferencia, contando
también las horribles crueldades que hubo de emplear para imponerse a la
vil chusma que con él servía. Nombráronme a mí su defensor\ldots{} y
figúrese usted mi compromiso. Era el francés muy simpático, y en la
cárcel, cargado de grillos, cautivaba a todo el mundo por su lenguaje
fino y discreto, y la resignación con que esperaba su sentencia. A mí
también me cautivó: aires tenía de gran señor, conocimientos de historia
y literatura, palabra muy amena y un don de simpatía irresistible.
Naturalmente, movido de esa misma simpatía y de la compasión, quise
salvarle; pero vea usted aquí lo más peregrino del caso. Verdier, que
así se llamaba, no quería por ningún caso dejarse salvar. «D.
Manuel---me decía,---no se empeñe usted en lo imposible. Mis delitos
sólo alcanzarán perdón en el Cielo: ningún tribunal del mundo puede ni
debe absolverme.» Firme en su resolución, que sostenía con una tenacidad
admirable, todos los esfuerzos que yo hacía para disculpar sus crímenes
los destruía el francés declarando más horrores, y presentando ante el
tribunal nuevos cuadros de maldad sanguinaria. Aquel hombre, créalo
usted, me ponía en gran confusión. ¿Cómo negar su grandeza, no inferior
a sus crímenes? «D. Manuel---repetía,---es inútil cuanto usted haga para
salvarme. No quiero, no quiero. Emplee su talento en defender a otros,
que también están manchados de sangre, pero no tanto como yo, y además
son padres de familia, tienen hijos. Yo no tengo a nadie. No tengo más
que a mi conciencia, que me manda morir.»

---¡Qué hombre! Amaba el castigo.

---Se enamoró de la muerte; la muerte era su ilusión, como lo había sido
antes el crimen. En fin, que me convencí de la imposibilidad de salvarle
la vida, y me apliqué a conseguir para otros la conmutación de pena.
Verdier subió al patíbulo, demostrando un arrepentimiento sincero, una
dignidad caballeresca y una efusión cristiana que fue el pasmo de
todos\ldots{} Y ahora voy al fin de mi cuento. Esta madrugada, un rato
en sueños, y después tan despierto como estoy ahora, vi al pirata entrar
por esa puerta. No tengo duda de que hablamos y de que me dijo: «D.
Manuel, que se le quite de la cabeza el redimirme. Ya me redimo yo.» Y
todas las escenas, todos los incidentes de la causa, cuanto hice y vi en
aquellos días, se me ha reproducido con claridad maravillosa.

---En verdad que es inaudito\ldots{} Yo también\ldots{} yo también he
visto personas y sucesos pasados, no tan remotos como los que usted
cuenta\ldots{} He visto\ldots{}

---Y fíjese en otra particularidad: ninguna relación tiene el caso del
pirata con este caso mío. ¿Por qué mi memoria eligió caprichosamente
aquel suceso de mi vida para reproducírmelo ahora con tanta
claridad\ldots? ¡Pobre Verdier\ldots! Materia de bandido, que fermentada
en la desgracia se volvió espíritu de caballero cristiano\ldots{}

Callaron ambos, pensando cada cual en cosas íntimas, y no se determinaba
Ibero a formular la interrogación consabida. No es delicado mortificar a
los reos de muerte con preguntas que sólo interesan al interpelante, y
es caritativo dejarles la iniciativa de la conversación en la angustiosa
espera de la capilla. Cortó la pausa el oficial de guardia, dando al
Coronel aviso de que el General le llamaba. Inmutose Montes de Oca con
la repentina entrada del oficial, y se preparó a salir del lecho,
murmurando: «Será tarde\ldots{} y yo aquí con esta calma\ldots{} Fuera
pereza.»

Ibero salió, aplicando con más empeño su mente a la solución del
acertijo, y aunque ningún dato nuevo justificaba su repentina
inclinación al término afirmativo, no cesaba de decirse: «¡Es,
es\ldots{} vaya si es!\ldots» Llamábale Aleson para designar de común
acuerdo \emph{la hora y el sitio}.

\hypertarget{xxix}{%
\chapter{XXIX}\label{xxix}}

Cuando volvió a la capilla, que los ordenanzas habían arreglado en lo
que se persigna un cura loco, poniendo en su lugar cada sagrado objeto,
y la Dolorosa y el Cristo, encontró a Montes de Oca en el momento
solemnísimo de oír su sentencia de muerte. Habíase vestido y acicalado
con todo el esmero posible en la pobreza de su cárcel, y en su rostro
grave y triste no se advertía ni temor ni arrogancia. Contaba ya con la
muerte, y aceptábala sin creer que la merecía, como el coronamiento más
digno de su desastre revolucionario. Vivir vencido con vilipendio no era
muy airoso, y la noble causa que había defendido se sublimaba con la
sangre de los que intentaron ser sus héroes. A la pregunta de si ampliar
quería su declaración de la noche anterior, respondió que se confirmaba
en ella. Se había sublevado contra el Gobierno, induciendo a paisanos y
tropa a la rebelión, porque en conciencia creía que era su deber
desobedecer a Espartero. Para él toda autoridad que no fuese la de la
Reina Doña María Cristina, era ilegal y usurpadora. Declarose miembro
del Gobierno Provisional, que proclamaba la Regencia legítima, y como
tal expidió decretos y efectuó diferentes actos gubernativos. ¿Quiénes
eran sus cómplices? Todos los corazones leales. Su honor no le permitía
decir más.

Dicho esto, y elegido para su confesor el cura de San Pedro, entre los
dos que le presentaron, dejáronle solo con el sacerdote. Y el buen Ibero
se alejó diciendo para sí: «Es\ldots{} es: ya no tengo duda. ¿Por qué lo
afirmo? No lo sé\ldots{} No puedo separar en mi pensamiento la imagen de
él y la imagen de ella, y me cuesta trabajo convencerme de que no fue
real lo que anoche vi\ldots{} Y yo pregunto: ¿se acordará de ella?
Quizás no. Fue un amor pasajero, aventura que se repetía en las buenas
ocasiones. Él no la amó nunca\ldots{} ¡Qué misterios! Ella insensata; él
sensato en amores, loco en política. Se asemejan más de lo que parece.
Una reina le hace a él mártir, y él ha martirizado a una pobre mujer
humilde, la cual me transmite a mí su martirio. Y véome aquí siendo el
último mártir. Él muere, moriremos todos uno tras otro\ldots{} ¡Qué
cadena de dolores y muertes!\ldots{} No doy un paso sin creer que
encuentro a la pobre Rafaela pidiéndome la vida de este hombre. Anoche
quizás habría sido posible, dejándole escapar por la ventana, y
arrojando también por ella mi honor militar y mi nombre sin tacha. Más
vale así. Muera el que debe morir ahora, el que ha faltado a la ley
política y a la ley de amor. Después seguirán cayendo las otras
víctimas, y yo la última, la que en sí acumulará el dolor y el martirio
de todas.»

Fue a su alojamiento, con idea de mudarse de ropa. Encerrado en la
estancia, ni grande ni lujosa, más bien destartalada y obscura, sufrió
un acceso de aflicción intensísima, que se tradujo en sacudidas
convulsas y en gritos de dolor. Arrojose en el lecho, de cara contra las
almohadas, y clavándose los dedos en el cráneo, no se calmaron sus
ansias terribles hasta que no hubo echado en lágrimas parte del dolor
que el alma le obstruía\ldots{} «Yo no puedo salvarle---pensaba.---Ni
debo, ni quiero. Cumpla su destino. Será dichoso. Él no hace más que
morir; los demás padecemos.» Y al reponerse de tan fiero trastorno,
entendiendo que no era ocasión de arrebatos sentimentales, se echó en
cara su flaqueza de ánimo. Si sus compañeros y subordinados, en el
tremendo acto que ya estaba próximo, le veían tan afligido, con señales
de haber llorado, creerían que el valiente Ibero había caído en
ridículas afeminaciones. Compuso su fisonomía lo mejor que pudo. La
inspección de policía que hizo en su persona fue muy rápida, y partió al
cumplimiento de sus deberes. Era la primera vez, en su vida militar, la
primera vez que temblaba. Ya conocía el miedo, y este le perseguía
haciéndole el coco en formas pueriles. Al menor ruido se estremecía;
cualquier sombrajo le asustaba. Al ver los fusiles de sus soldados, la
idea de que dispararan le causaba terror.

Procurando sobreponerse a esta ridícula mujeril flaqueza, volvió el
Coronel a la capilla y encontró a Montes de Oca ya confesado. El General
Aleson había entrado a visitarle. Agradeciéndole su cortesía y caridad,
pidió el reo se le permitiese dar vivas a Isabel II, a la Reina Cristina
y a los Fueros. En delicada forma, excitándole a renunciar a estas
demostraciones inoportunas, negó su permiso el General. No debía pensar
más que en Dios, apartando en absoluto su espíritu de toda idea
política. Asimismo quiso el mártir que se le consintiera mandar el
fuego, y con tal afán lo pedía, que hubo de acceder Aleson, recordando
que había no pocos ejemplos de esta tolerancia en la rica historia del
fusilamiento nacional. Pero al propio tiempo que la autoridad militar
asentía, protestaba la eclesiástica: el sacerdote declaró con grave
acento que el dar la víctima las voces de mando en acto de tal
naturaleza, era contrario a los principios religiosos. La muerte en esta
forma consumada era un suicidio, y por ningún caso la autorizaba.

Ausente el General, después de reiterar al preso sus sentimientos de
piedad y cariño, se reanudó la cuestión, pues Montes de Oca insistía en
mandar el fuego, y el cura, inflexible, llevando su negativa a los
extremos de la intolerancia, declaró que se retiraría si el reo no se
conformaba con que diese las órdenes el oficial encargado de esta triste
función. El debate fue empeñadísimo: tomó Ibero partido en él por Montes
de Oca, y en apoyo del sacerdote acudieron otros dos clérigos, que
hicieron gala de su saber teológico. Por fin, el mismo Coronel, viendo
que se prolongaba demasiado la contienda, propuso a su amigo esta forma
de transacción: «En vez de dar las voces de mando, usted dirá:
\emph{Granaderos, la religión me prohíbe el mandaros hacerme fuego: el
caballero oficial cumplirá este deber}. Y para satisfacción de usted, no
mandará el oficial; mandaré yo, que es como si usted mismo mandara con
su voluntad, no con su palabra.» Pareciole al condenado muy aceptable
esta proposición, y los clérigos, aunque entre sí rezongaban, no dijeron
nada en contra.

\hypertarget{xxx}{%
\chapter{XXX}\label{xxx}}

La hora se acercaba. Trajeron un breve almuerzo que D. Manuel había
pedido, y de él comió muy poco, sin apetito, bebiendo algo de vino y
bastante café. Sentado frente a él, Ibero le contemplaba silencioso, sin
atreverse a pronunciar palabra: tal era el respeto que aquel inmenso
infortunio, soportado con tanta grandeza de alma, le infundía. En el
rostro del reo se hacía visible, desde el amanecer, una lenta
transfiguración. Parecía de purísima cera, la frente más blanca que todo
lo demás, de una blancura ideal. A ratos, mientras comía, fijaba D.
Manuel sus ojos azules en los negros de Ibero. Era el cielo mirando a la
tierra.

La expresión inefable, dulce y amorosa de aquellos ojos removía toda el
alma del Coronel, y tan pronto le devolvía su valor perdido como se lo
quitaba por entero. En una de aquellas miradas, Ibero pensó que el reo
quería decirle algo. Sí, sí: llegaba el momento de expresar la última
idea de este mundo y pronunciar la palabra última de los idiomas
terrestres. Habló nuevamente Montes de Oca con el sacerdote, apartados
junto al altar, y luego acercose a Santiago y le dijo: «Amigo mío, le
veo a usted demasiado afligido y como temeroso\ldots»

---He tenido miedo---replicó el alavés abrazándole con efusión;---podía
mi compasión más que mi entereza. Pero la presencia de usted me
restablece en mi carácter, en mi valentía natural. Para no perderla en
lo que pueda, me hago cargo de que los dos vamos a morir juntos, sin
duda porque merecemos el mismo fin. Con esta idea, la grandeza de usted
se me comunica. Ya no tiemblo. Yo, ejecutor, soy tan bravo como el reo.

---¿Es hora ya?

---Sí\ldots{} Un momento más. ¿No tiene usted algo que
encargarme?\ldots{} ¿No tiene algo que decirme? Aunque ha dejado
escritas sus disposiciones, puede haber persona o suceso que se hayan
extraviado en su memoria\ldots{} persona o suceso que no merezcan
olvido\ldots{}

Montes de Oca, sin perder un momento su serenidad ni el tono claro de su
voz, le abrazó dos veces, diciendo sucesivamente: «Este abrazo por
usted, señal de un afecto que es mi mayor consuelo, después de la idea
de Dios, en la hora de mi muerte\ldots{} Este otro\ldots{} ya ve usted
que también es apretado\ldots{} este otro para que usted lo transmita a
las personas que me han querido.»

---¿A las\ldots{} a quién?

---A toda persona de quien usted sepa que me ha querido mucho\ldots{}
Vámonos. El tambor nos llama.

Salió sin sombrero. En el patio que daba a la calle de San Francisco
esperaba una carretela. A ella subió el reo, con el capellán a un lado y
el Coronel enfrente. Muy bien cumplida por el cochero la orden de
acelerar el paso, pronto llegaron a la Florida. Poca gente había en las
calles y a la entrada del paseo. El honrado pueblo de Vitoria hizo al
mártir los honores de un respetuoso duelo, alejándose del teatro de su
martirio. Las personas que acudieron a verle pasar le compadecieron
silenciosas. Algunas le miraron llorando. Durante el trayecto fúnebre,
Montes de Oca habló algo con el capellán, menos con el Coronel; el sol
hería de frente su rostro, y con su mano bien firme, no afectada ni de
ligero temblor defendía sus ojos de la viva luz.

La parte de ciudad que recorrió dejaba en su alma impresión de soledad,
de silencio, de olvido. Creyó que muriendo él, moría también Vitoria, la
que había sido capital del efímero reino de Cristina. En Cristina
pensaba el mártir cuando bajó del coche en el lugar donde formaba el
cuadro, y al ver a los soldados del regimiento que llevaba el nombre de
la augusta Princesa, de la diosa, del ídolo, de la Dulcinea más soñada
que real, sintió por primera vez el frío de la muerte, y una congoja que
hubo de sofocar con titánico esfuerzo para que no se le conociera en el
rostro\ldots{}

Pusiéronle en el sitio donde debía morir; le abrazaron nuevamente con
efusión el capellán y el Coronel. Las cláusulas del Credo gemían en los
labios temblorosos. Santiago no pudo cumplir su promesa de mandar el
fuego: su valor, rehecho con ayuda de Dios, a tanto no llegaba. Dos
palabras dijo al oficial, mientras el bravo Montes de Oca, con acento
firme y sonora voz, dirigía la breve alocución a los granaderos y daba
los vivas a Isabel y a Cristina. El Credo seguía lento, premioso\ldots{}
la bendita oración era como un ser vivo que no quería dejarse rezar.
Sonó la descarga, y herido en el vientre, el reo permaneció en pie, las
manos en los bolsillos del gabán, presentando el pecho a los fusiles.
Dio un paso hacia la izquierda; la segunda descarga le hirió en el
pecho; se tambaleó, cayendo por fin. Pero continuaba vivo. Ibero se
acercó: los azules ojos del mártir le miraron, y sus dos manos señalaron
las sienes. Ojos y manos le decían: «Tirarme aquí, y acabemos.» Un
soldado le remató.

Sólo falta decir, por ahora, que D. Santiago Ibero no se apartó del
muerto hasta que le puso con sus propias manos en la fosa, abrigándole
con la tierra y señalándole con una cruz. Quédese para otra ocasión lo
restante del cuento de este noble militar, el luto que guardó a su
amigo, las resoluciones que tomó, instigado por la dulce y trágica
memoria del mártir, los falsos caminos por donde le llevaron sus
desdichados pensamientos, y los desmayos y caídas que en ellos sufrió
hasta encontrar por aviso de Dios la vía verdadera.

\flushright{Madrid, Marzo-Abril de 1900.}

~

\bigskip
\bigskip
\begin{center}
\textsc{fin de montes de oca}
\end{center}

\end{document}
